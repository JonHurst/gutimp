\newenvironment{letter}{

\strut

}{

\strut

}

\newcommand{\LetterDate}[1]{
\vbox{\strut\itshape\small\raggedleft #1}
}

\newcommand{\LetterSig}[1]{
\vbox{\strut\scshape\raggedleft #1}
}

\frenchspacing

\frontmatter
\title{Pride and Prejudice}
\author{Jane Austen}
\date{}

\maketitle
\tableofcontents

\mainmatter
\Part{VOL. I.}
%%002%%

\Chapter{CHAPTER I.}

It is a truth universally acknowledged, that a single man
in possession of a good fortune, must be in want of a wife.

However little known the feelings or views of such a
man may be on his first entering a neighbourhood, this
truth is so well fixed in the minds of the surrounding
families, that he is considered as the rightful property of
some one or other of their daughters.

“My dear Mr. Bennet,” said his lady to him one day,
“have you heard that Netherfield Park is let at last?”

Mr. Bennet replied that he had not.

“But it is,” returned she; “for Mrs. Long has just
been here, and she told me all about it.”

Mr. Bennet made no answer.

“Do not you want to know who has taken it?” cried
his wife impatiently.

“\textit{You} want to tell me, and I have no objection to
hearing it.”

This was invitation enough.

“Why, my dear, you must know, Mrs. Long says that
Netherfield is taken by a young man of large fortune from
the north of England; that he came down on Monday
in a chaise and four to see the place, and was so much
delighted with it that he agreed with Mr. Morris immediately;
that he is to take possession before Michaelmas,
and some of his servants are to be in the house by the end
of next week.”

“What is his name?”

“Bingley.”

“Is he married or single?”

“Oh! single, my dear, to be sure! A single man of
%%003%%
large fortune; four or five thousand a year. What a fine
thing for our girls!”

“How so? how can it affect them?”

“My dear Mr. Bennet,” replied his wife, “how can you
be so tiresome! You must know that I am thinking of his
marrying one of them.”

“Is that his design in settling here?”

“Design! nonsense, how can you talk so! But it is
very likely that he \textit{may} fall in love with one of them, and
therefore you must visit him as soon as he comes.”

“I see no occasion for that. You and the girls may go,
or you may send them by themselves, which perhaps will
be still better, for as you are as handsome as any of them,
Mr. Bingley might like you the best of the party.”

“My dear, you flatter me. I certainly \textit{have} had my
share of beauty, but I do not pretend to be any thing
extraordinary now. When a woman has five grown up daughters,
she ought to give over thinking of her own beauty.”

“In such cases, a woman has not often much beauty
to think of.”

“But, my dear, you must indeed go and see Mr. Bingley
when he comes into the neighbourhood.”

“It is more than I engage for, I assure you.”

“But consider your daughters. Only think what an
establishment it would be for one of them. Sir William
and Lady Lucas are determined to go, merely on that
account, for in general you know they visit no new comers.
Indeed you must go, for it will be impossible for \textit{us} to
visit him, if you do not.”

“You are over scrupulous surely. I dare say Mr.
Bingley will be very glad to see you; and I will send
a few lines by you to assure him of my hearty consent
to his marrying which ever he chuses of the girls; though
I must throw in a good word for my little Lizzy.”

“I desire you will do no such thing. Lizzy is not
a bit better than the others; and I am sure she is not half
so handsome as Jane, nor half so good humoured as Lydia.
But you are always giving \textit{her} the preference.”
%%004%%

“They have none of them much to recommend them,”
replied he; “they are all silly and ignorant like other
girls; but Lizzy has something more of quickness
than her sisters.”

“Mr. Bennet, how can you abuse your own children
in such a way? You take delight in vexing me. You
have no compassion on my poor nerves.”

“You mistake me, my dear. I have a high respect for
your nerves. They are my old friends. I have heard you
mention them with consideration these twenty years at
least.”

“Ah! you do not know what I suffer.”

“But I hope you will get over it, and live to see many
young men of four thousand a year come into the
neighbourhood.”

“It will be no use to us, if twenty such should come
since you will not visit them.”

“Depend upon it, my dear, that when there are twenty,
I will visit them all.”

Mr. Bennet was so odd a mixture of quick parts, sarcastic
humour, reserve, and caprice, that the experience of three
and twenty years had been insufficient to make his wife
understand his character. \textit{Her} mind was less difficult to
develope. She was a woman of mean understanding, little
information, and uncertain temper. When she was discontented
she fancied herself nervous. The business of her
life was to get her daughters married; its solace was
visiting and news.
%%005%%

\Chapter{CHAPTER II.}

Mr. Bennet was among the earliest of those who
waited on Mr. Bingley. He had always intended to visit
him, though to the last always assuring his wife that he
should not go; and till the evening after the visit was paid,
she had no knowledge of it. It was then disclosed in the
following manner. Observing his second daughter employed
in trimming a hat, he suddenly addressed her with,

“I hope Mr. Bingley will like it Lizzy.”

“We are not in a way to know \textit{what} Mr. Bingley likes,”
said her mother resentfully, “since we are not to visit.”

“But you forget, mama,” said Elizabeth, “that we
shall meet him at the assemblies, and that Mrs. Long has
promised to introduce him.”

“I do not believe Mrs. Long will do any such thing.
She has two neices of her own. She is a selfish, hypocritical
woman, and I have no opinion of her.”

“No more have I,” said Mr. Bennet; “and I am glad
to find that you do not depend on her serving you.”

Mrs. Bennet deigned not to make any reply; but
unable to contain herself, began scolding one of her
daughters.

“Don’t keep coughing so, Kitty, for heaven’s sake!
Have a little compassion on my nerves. You tear them
to pieces.”

“Kitty has no discretion in her coughs,” said her
father; “she times them ill.”

“I do not cough for my own amusement,” replied
Kitty fretfully.

“When is your next ball to be, Lizzy?”

“To-morrow fortnight.”

“Aye, so it is,” cried her mother, “and Mrs. Long
does not come back till the day before; so, it will be
impossible for her to introduce him, for she will not know
him herself.”
%%006%%

“Then, my dear, you may have the advantage of your
friend, and introduce Mr. Bingley to \textit{her}.”

“Impossible, Mr. Bennet, impossible, when I am
not acquainted with him myself; how can you be so
teazing?”

“I honour your circumspection. A fortnight’s acquaintance
is certainly very little. One cannot know what
a man really is by the end of a fortnight. But if \textit{we} do
not venture, somebody else will; and after all, Mrs. Long
and her neices must stand their chance; and therefore,
as she will think it an act of kindness, if you decline the
office, I will take it on myself.”

The girls stared at their father. Mrs. Bennet said only,
“Nonsense, nonsense!”

“What can be the meaning of that emphatic exclamation?”
cried he. “Do you consider the forms of introduction,
and the stress that is laid on them, as nonsense?
I cannot quite agree with you \textit{there}. What say you,
Mary? for you are a young lady of deep reflection I know,
and read great books, and make extracts.”

Mary wished to say something very sensible, but knew
not how.

“While Mary is adjusting her ideas,” he continued,
“let us return to Mr. Bingley.”

“I am sick of Mr. Bingley,” cried his wife.

“I am sorry to hear \textit{that}; but why did not you tell
me so before? If I had known as much this morning,
I certainly would not have called on him. It is very
unlucky; but as I have actually paid the visit, we cannot
escape the acquaintance now.”

The astonishment of the ladies was just what he wished;
that of Mrs. Bennet perhaps surpassing the rest; though
when the first tumult of joy was over, she began to declare
that it was what she had expected all the while.

“How good it was in you, my dear Mr. Bennet! But
I knew I should persuade you at last. I was sure you
loved your girls too well to neglect such an acquaintance.
Well, how pleased I am! and it is such a good joke, too,
%%007%%
that you should have gone this morning, and never said
a word about it till now.”

“Now, Kitty, you may cough as much as you chuse,”
said Mr. Bennet; and, as he spoke, he left the room,
fatigued with the raptures of his wife.

“What an excellent father you have, girls,” said she,
when the door was shut. “I do not know how you will
ever make him amends for his kindness; or me either,
for that matter. At our time of life, it is not so pleasant
I can tell you, to be making new acquaintance every day;
but for your sakes, we would do any thing. Lydia, my
love, though you \textit{are} the youngest, I dare say Mr. Bingley
will dance with you at the next ball.”

“Oh!” said Lydia stoutly, “I am not afraid; for
though I \textit{am} the youngest, I’m the tallest.”

The rest of the evening was spent in conjecturing how
soon he would return Mr. Bennet’s visit, and determining
when they should ask him to dinner.
%%008%%

\Chapter{CHAPTER III.}

Not all that Mrs. Bennet, however, with the assistance
of her five daughters, could ask on the subject was sufficient
to draw from her husband any satisfactory description
of Mr. Bingley. They attacked him in various ways;
with barefaced questions, ingenious suppositions, and
distant surmises; but he eluded the skill of them all;
and they were at last obliged to accept the second-hand
intelligence of their neighbour Lady Lucas. Her report
was highly favourable. Sir William had been delighted
with him. He was quite young, wonderfully handsome,
extremely agreeable, and to crown the whole, he meant to
be at the next assembly with a large party. Nothing
could be more delightful! To be fond of dancing was
a certain step towards falling in love; and very lively
hopes of Mr. Bingley’s heart were entertained.

“If I can but see one of my daughters happily settled
at Netherfield,” said Mrs. Bennet to her husband, “and
all the others equally well married, I shall have nothing
to wish for.”

In a few days Mr. Bingley returned Mr. Bennet’s visit,
and sat about ten minutes with him in his library. He
had entertained hopes of being admitted to a sight of the
young ladies, of whose beauty he had heard much; but
he saw only the father. The ladies were somewhat more
fortunate, for they had the advantage of ascertaining from
an upper window, that he wore a blue coat and rode a black
horse.

An invitation to dinner was soon afterwards dispatched;
and already had Mrs. Bennet planned the courses that were
to do credit to her housekeeping, when an answer arrived
which deferred it all. Mr. Bingley was obliged to be in
town the following day, and consequently unable to accept
the honour of their invitation, \&c. Mrs. Bennet was
%%009%%
quite disconcerted. She could not imagine what business
he could have in town so soon after his arrival in
Hertfordshire; and she began to fear that he might be always
flying about from one place to another, and never settled
at Netherfield as he ought to be. Lady Lucas quieted her
fears a little by starting the idea of his being gone to
London only to get a large party for the ball; and a report
soon followed that Mr. Bingley was to bring twelve ladies
and seven gentlemen with him to the assembly. The girls
grieved over such a number of ladies; but were comforted
the day before the ball by hearing, that instead of twelve,
he had brought only six with him from London, his five
sisters and a cousin. And when the party entered the
assembly room, it consisted of only five altogether;
Mr. Bingley, his two sisters, the husband of the eldest,
and another young man.

Mr. Bingley was good looking and gentlemanlike; he
had a pleasant countenance, and easy, unaffected manners.
His sisters were fine women, with an air of decided fashion.
His brother-in-law, Mr. Hurst, merely looked the gentleman;
but his friend Mr. Darcy soon drew the attention
of the room by his fine, tall person, handsome features,
noble mien; and the report which was in general circulation
within five minutes after his entrance, of his having
ten thousand a year. The gentlemen pronounced him to
be a fine figure of a man, the ladies declared he was much
handsomer than Mr. Bingley, and he was looked at with
great admiration for about half the evening, till his
manners gave a disgust which turned the tide of his
popularity; for he was discovered to be proud, to be
above his company, and above being pleased; and not
all his large estate in Derbyshire could then save him from
having a most forbidding, disagreeable countenance, and
being unworthy to be compared with his friend.

Mr. Bingley had soon made himself acquainted with
all the principal people in the room; he was lively and
unreserved, danced every dance, was angry that the ball
closed so early, and talked of giving one himself at
%%010%%
Netherfield. Such amiable qualities must speak for themselves.
What a contrast between him and his friend! Mr. Darcy
danced only once with Mrs. Hurst and once with Miss
Bingley, declined being introduced to any other lady, and
spent the rest of the evening in walking about the room,
speaking occasionally to one of his own party. His character
was decided. He was the proudest, most disagreeable
man in the world, and every body hoped that he would
never come there again. Amongst the most violent
against him was Mrs. Bennet, whose dislike of his general
behaviour, was sharpened into particular resentment, by
his having slighted one of her daughters.

Elizabeth Bennet had been obliged, by the scarcity of
gentlemen, to sit down for two dances; and during part
of that time, Mr. Darcy had been standing near enough
for her to overhear a conversation between him and
Mr. Bingley, who came from the dance for a few minutes,
to press his friend to join it.

“Come, Darcy,” said he, “I must have you dance.
I hate to see you standing about by yourself in this stupid
manner. You had much better dance.”

“I certainly shall not. You know how I detest it,
unless I am particularly acquainted with my partner. At
such an assembly as this, it would be insupportable.
Your sisters are engaged, and there is not another woman
in the room, whom it would not be a punishment to me
to stand up with.”

“I would not be so fastidious as you are,” cried Bingley,
“for a kingdom! Upon my honour, I never met with
so many pleasant girls in my life, as I have this evening;
and there are several of them you see uncommonly pretty.”

“\textit{You} are dancing with the only handsome girl in the
room,” said Mr. Darcy, looking at the eldest Miss Bennet.

“Oh! she is the most beautiful creature I ever beheld!
But there is one of her sisters sitting down just behind
you, who is very pretty, and I dare say, very agreeable.
Do let me ask my partner to introduce you.”

“Which do you mean?” and turning round, he looked
%%011%%
for a moment at Elizabeth, till catching her eye, he withdrew
his own and coldly said, “She is tolerable; but not
handsome enough to tempt \textit{me}; and I am in no humour
at present to give consequence to young ladies who are
slighted by other men. You had better return to your
partner and enjoy her smiles, for you are wasting your
time with me.”

Mr. Bingley followed his advice. Mr. Darcy walked off;
and Elizabeth remained with no very cordial feelings
towards him. She told the story however with great
spirit among her friends; for she had a lively, playful
disposition, which delighted in any thing ridiculous.

The evening altogether passed off pleasantly to the
whole family. Mrs. Bennet had seen her eldest daughter
much admired by the Netherfield party. Mr. Bingley had
danced with her twice, and she had been distinguished
by his sisters. Jane was as much gratified by this, as
her mother could be, though in a quieter way. Elizabeth
felt Jane’s pleasure. Mary had heard herself mentioned
to Miss Bingley as the most accomplished girl in the
neighbourhood; and Catherine and Lydia had been
fortunate enough to be never without partners, which
was all that they had yet learnt to care for at a ball.
They returned therefore in good spirits to Longbourn, the
village where they lived, and of which they were the
principal inhabitants. They found Mr. Bennet still up.
With a book he was regardless of time; and on the present
occasion he had a good deal of curiosity as to the event
of an evening which had raised such splendid expectations.
He had rather hoped that all his wife’s views on the
stranger would be disappointed; but he soon found that
he had a very different story to hear.

“Oh! my dear Mr. Bennet,” as she entered the room,
“we have had a most delightful evening, a most excellent
ball. I wish you had been there. Jane was so admired,
nothing could be like it. Every body said how well she
looked; and Mr. Bingley thought her quite beautiful,
and danced with her twice. Only think of \textit{that} my dear;
%%012%%
he actually danced with her twice; and she was the only
creature in the room that he asked a second time. First
of all, he asked Miss Lucas. I was so vexed to see him
stand up with her; but, however, he did not admire her
at all: indeed, nobody can, you know; and he seemed
quite struck with Jane as she was going down the dance.
So, he enquired who she was, and got introduced, and asked
her for the two next. Then, the two third he danced with
Miss King, and the two fourth with Maria Lucas, and the
two fifth with Jane again, and the two sixth with Lizzy,
and the Boulanger------”

“If he had had any compassion for \textit{me},” cried her
husband impatiently, “he would not have danced half
so much! For God’s sake, say no more of his partners.
Oh! that he had sprained his ancle in the first dance!”

“Oh! my dear,” continued Mrs. Bennet, “I am quite
delighted with him. He is so excessively handsome! and
his sisters are charming women. I never in my life saw
any thing more elegant than their dresses. I dare say the
lace upon Mrs. Hurst’s gown------”

Here she was interrupted again. Mr. Bennet protested
against any description of finery. She was therefore
obliged to seek another branch of the subject, and related,
with much bitterness of spirit and some exaggeration, the
shocking rudeness of Mr. Darcy.

“But I can assure you,” she added, “that Lizzy does
not lose much by not suiting \textit{his} fancy; for he is a most
disagreeable, horrid man, not at all worth pleasing. So
high and so conceited that there was no enduring him!
He walked here, and he walked there, fancying himself
so very great! Not handsome enough to dance with!
I wish you had been there, my dear, to have given him
one of your set downs. I quite detest the man.”
%%013%%

\Chapter{CHAPTER IV.}

When Jane and Elizabeth were alone, the former, who
had been cautious in her praise of Mr. Bingley before,
expressed to her sister how very much she admired him.

“He is just what a young man ought to be,” said she,
“sensible, good humoured, lively; and I never saw such
happy manners! -- so much ease, with such perfect good
breeding!”

“He is also handsome,” replied Elizabeth, “which
a young man ought likewise to be, if he possibly can.
His character is thereby complete.”

“I was very much flattered by his asking me to dance
a second time. I did not expect such a compliment.”

“Did not you? \textit{I} did for you. But that is one great
difference between us. Compliments always take \textit{you} by
surprise, and \textit{me} never. What could be more natural than
his asking you again? He could not help seeing that you
were about five times as pretty as every other woman in
the room. No thanks to his gallantry for that. Well,
he certainly is very agreeable, and I give you leave to
like him. You have liked many a stupider person.”

“Dear Lizzy!”

“Oh! you are a great deal too apt you know, to like
people in general. You never see a fault in any body.
All the world are good and agreeable in your eyes. I never
heard you speak ill of a human being in my life.”

“I would wish not to be hasty in censuring any one;
but I always speak what I think.”

“I know you do; and it is \textit{that} which makes the wonder.
With \textit{your} good sense, to be so honestly blind to the follies
and nonsense of others! Affectation of candour is common
enough; -- one meets it every where. But to be candid
without ostentation or design -- to take the good of every
body’s character and make it still better, and say nothing
%%014%%
of the bad -- belongs to you alone. And so, you like this
man’s sisters too, do you? Their manners are not equal
to his.”

“Certainly not; at first. But they are very pleasing
women when you converse with them. Miss Bingley
is to live with her brother and keep his house; and I am
much mistaken if we shall not find a very charming
neighbour in her.”

Elizabeth listened in silence, but was not convinced;
their behaviour at the assembly had not been calculated
to please in general; and with more quickness of observation
and less pliancy of temper than her sister, and
with a judgment too unassailed by any attention to herself,
she was very little disposed to approve them. They were
in fact very fine ladies; not deficient in good humour
when they were pleased, nor in the power of being agreeable
where they chose it; but proud and conceited. They
were rather handsome, had been educated in one of the
first private seminaries in town, had a fortune of twenty
thousand pounds, were in the habit of spending more than
they ought, and of associating with people of rank; and
were therefore in every respect entitled to think well of
themselves, and meanly of others. They were of a respectable
family in the north of England; a circumstance more
deeply impressed on their memories than that their
brother’s fortune and their own had been acquired by
trade.

Mr. Bingley inherited property to the amount of nearly
an hundred thousand pounds from his father, who had
intended to purchase an estate, but did not live to do it. --
Mr. Bingley intended it likewise, and sometimes made
choice of his county; but as he was now provided with
a good house and the liberty of a manor, it was doubtful
to many of those who best knew the easiness of his
temper, whether he might not spend the remainder of his
days at Netherfield, and leave the next generation to
purchase.

His sisters were very anxious for his having an estate
%%015%%
of his own; but though he was now established only as
a tenant, Miss Bingley was by no means unwilling to
preside at his table, nor was Mrs. Hurst, who had married
a man of more fashion than fortune, less disposed to
consider his house as her home when it suited her. Mr.
Bingley had not been of age two years, when he was
tempted by an accidental recommendation to look at
Netherfield House. He did look at it and into it for half
an hour, was pleased with the situation and the principal
rooms, satisfied with what the owner said in its praise,
and took it immediately.

Between him and Darcy there was a very steady
friendship, in spite of a great opposition of character. --
Bingley was endeared to Darcy by the easiness, openness,
ductility of his temper, though no disposition could offer
a greater contrast to his own, and though with his own he
never appeared dissatisfied. On the strength of Darcy’s
regard Bingley had the firmest reliance, and of his judgment
the highest opinion. In understanding Darcy was
the superior. Bingley was by no means deficient, but
Darcy was clever. He was at the same time haughty,
reserved, and fastidious, and his manners, though well
bred, were not inviting. In that respect his friend had
greatly the advantage. Bingley was sure of being liked
wherever he appeared, Darcy was continually giving
offence.

The manner in which they spoke of the Meryton
assembly was sufficiently characteristic. Bingley had
never met with pleasanter people or prettier girls in his
life; every body had been most kind and attentive to
him, there had been no formality, no stiffness, he had
soon felt acquainted with all the room; and as to Miss
Bennet, he could not conceive an angel more beautiful.
Darcy, on the contrary, had seen a collection of people
in whom there was little beauty and no fashion, for none
of whom he had felt the smallest interest, and from none
received either attention or pleasure. Miss Bennet he
acknowledged to be pretty, but she smiled too much.
%%016%%

Mrs. Hurst and her sister allowed it to be so -- but still
they admired her and liked her, and pronounced her to
be a sweet girl, and one whom they should not object to
know more of. Miss Bennet was therefore established as
a sweet girl, and their brother felt authorised by such
commendation to think of her as he chose.
%%017%%

\Chapter{CHAPTER V.}

Within a short walk of Longbourn lived a family with
whom the Bennets were particularly intimate. Sir William
Lucas had been formerly in trade in Meryton, where he had
made a tolerable fortune and risen to the honour of knighthood
by an address to the King, during his mayoralty. The
distinction had perhaps been felt too strongly. It had given
him a disgust to his business and to his residence in a small
market town; and quitting them both, he had removed
with his family to a house about a mile from Meryton,
denominated from that period Lucas Lodge, where he
could think with pleasure of his own importance, and
unshackled by business, occupy himself solely in being
civil to all the world. For though elated by his rank,
it did not render him supercilious; on the contrary, he
was all attention to every body. By nature inoffensive,
friendly and obliging, his presentation at St. James’s had
made him courteous.

Lady Lucas was a very good kind of woman, not too
clever to be a valuable neighbour to Mrs. Bennet. -- They
had several children. The eldest of them, a sensible,
intelligent young woman, about twenty-seven, was Elizabeth’s
intimate friend.

That the Miss Lucases and the Miss Bennets should
meet to talk over a ball was absolutely necessary; and
the morning after the assembly brought the former to
Longbourn to hear and to communicate.

“\textit{You} began the evening well, Charlotte,” said Mrs.
Bennet with civil self-command to Miss Lucas. “\textit{You}
were Mr. Bingley’s first choice.”

“Yes; -- but he seemed to like his second better.”

“Oh! -- you mean Jane, I suppose -- because he danced
with her twice. To be sure that \textit{did} seem as if he admired
her -- indeed I rather believe he \textit{did} -- I heard something
%%018%%
about it -- but I hardly know what -- something about
Mr. Robinson.”

“Perhaps you mean what I overheard between him
and Mr. Robinson; did not I mention it to you? Mr.
Robinson’s asking him how he liked our Meryton assemblies,
and whether he did not think there were a great
many pretty women in the room, and \textit{which} he thought
the prettiest? and his answering immediately to the last
question -- Oh! the eldest Miss Bennet beyond a doubt,
there cannot be two opinions on that point.”

“Upon my word! -- Well, that was very decided indeed -- that
does seem as if------but however, it may all come
to nothing you know.”

“\textit{My} overhearings were more to the purpose than \textit{yours},
Eliza,” said Charlotte. “Mr. Darcy is not so well worth
listening to as his friend, is he? -- Poor Eliza! -- to be only
just \textit{tolerable}.”

“I beg you would not put it into Lizzy’s head to
be vexed by his ill-treatment; for he is such a disagreeable
man that it would be quite a misfortune to be
liked by him. Mrs. Long told me last night that he
sat close to her for half an hour without once opening his
lips.”

“Are you quite sure, Ma’am? -- is not there a little
mistake?” said Jane. -- “I certainly saw Mr. Darcy
speaking to her.”

“Aye -- because she asked him at last how he liked
Netherfield, and he could not help answering her; -- but
she said he seemed very angry at being spoke to.”

“Miss Bingley told me,” said Jane, “that he never
speaks much unless among his intimate acquaintance.
With \textit{them} he is remarkably agreeable.”

“I do not believe a word of it, my dear. If he had been
so very agreeable he would have talked to Mrs. Long.
But I can guess how it was; every body says that he is
ate up with pride, and I dare say he had heard somehow
that Mrs. Long does not keep a carriage, and had come
to the ball in a hack chaise.”
%%019%%

“I do not mind his not talking to Mrs. Long,” said
Miss Lucas, “but I wish he had danced with Eliza.”

“Another time, Lizzy,” said her mother, “I would not
dance with \textit{him}, if I were you.”

“I believe, Ma’am, I may safely promise you \textit{never} to
dance with him.”

“His pride,” said Miss Lucas, “does not offend \textit{me} so
much as pride often does, because there is an excuse for it.
One cannot wonder that so very fine a young man, with
family, fortune, every thing in his favour, should think
highly of himself. If I may so express it, he has a \textit{right}
to be proud.”

“That is very true,” replied Elizabeth, “and I could
easily forgive \textit{his} pride, if he had not mortified \textit{mine}.”

“Pride,” observed Mary, who piqued herself upon the
solidity of her reflections, “is a very common failing I
believe. By all that I have ever read, I am convinced
that it is very common indeed, that human nature is
particularly prone to it, and that there are very few of
us who do not cherish a feeling of self-complacency on
the score of some quality or other, real or imaginary.
Vanity and pride are different things, though the words
are often used synonimously. A person may be proud
without being vain. Pride relates more to our opinion
of ourselves, vanity to what we would have others think
of us.”

“If I were as rich as Mr. Darcy,” cried a young Lucas
who came with his sisters, “I should not care how proud
I was. I would keep a pack of foxhounds, and drink
a bottle of wine every day.”

“Then you would drink a great deal more than you
ought,” said Mrs. Bennet; “and if I were to see you at
it I should take away your bottle directly.”

The boy protested that she should not; she continued
to declare that she would, and the argument ended only
with the visit.
%%020%%

\Chapter{CHAPTER VI.}

The ladies of Longbourn soon waited on those of
Netherfield. The visit was returned in due form. Miss
Bennet’s pleasing manners grew on the good will of Mrs.
Hurst and Miss Bingley; and though the mother was
found to be intolerable and the younger sisters not worth
speaking to, a wish of being better acquainted with \textit{them},
was expressed towards the two eldest. By Jane this
attention was received with the greatest pleasure; but
Elizabeth still saw superciliousness in their treatment of
every body, hardly excepting even her sister, and could
not like them; though their kindness to Jane, such as it
was, had a value as arising in all probability from the
influence of their brother’s admiration. It was generally
evident whenever they met, that he \textit{did} admire her; and
to \textit{her} it was equally evident that Jane was yielding to the
preference which she had begun to entertain for him from
the first, and was in a way to be very much in love; but
she considered with pleasure that it was not likely to be
discovered by the world in general, since Jane united with
great strength of feeling, a composure of temper and a
uniform cheerfulness of manner, which would guard her
from the suspicions of the impertinent. She mentioned
this to her friend Miss Lucas.

“It may perhaps be pleasant,” replied Charlotte, “to
be able to impose on the public in such a case; but it is
sometimes a disadvantage to be so very guarded. If
a woman conceals her affection with the same skill from
the object of it, she may lose the opportunity of fixing
him; and it will then be but poor consolation to believe
the world equally in the dark. There is so much of gratitude
or vanity in almost every attachment, that it is not
safe to leave any to itself. We can all \textit{begin} freely -- a slight
preference is natural enough; but there are very few of
%%021%%
us who have heart enough to be really in love without
encouragement. In nine cases out of ten, a woman had
better shew \textit{more} affection than she feels. Bingley likes
your sister undoubtedly; but he may never do more than
like her, if she does not help him on.”

“But she does help him on, as much as her nature will
allow. If \textit{I} can perceive her regard for him, he must be
a simpleton indeed not to discover it too.”

“Remember, Eliza, that he does not know Jane’s
disposition as you do.”

“But if a woman is partial to a man, and does not
endeavour to conceal it, he must find it out.”

“Perhaps he must, if he sees enough of her. But
though Bingley and Jane meet tolerably often, it is never
for many hours together; and as they always see each
other in large mixed parties, it is impossible that every
moment should be employed in conversing together.
Jane should therefore make the most of every half hour
in which she can command his attention. When she is
secure of him, there will be leisure for falling in love as
much as she chuses.”

“Your plan is a good one,” replied Elizabeth, “where
nothing is in question but the desire of being well married;
and if I were determined to get a rich husband, or any
husband, I dare say I should adopt it. But these are
not Jane’s feelings; she is not acting by design. As yet,
she cannot even be certain of the degree of her own
regard, nor of its reasonableness. She has known him only
a fortnight. She danced four dances with him at Meryton;
she saw him one morning at his own house, and has since
dined in company with him four times. This is not quite
enough to make her understand his character.”

“Not as you represent it. Had she merely \textit{dined} with
him, she might only have discovered whether he had
a good appetite; but you must remember that four
evenings have been also spent together -- and four evenings
may do a great deal.”

“Yes; these four evenings have enabled them to
%%022%%
ascertain that they both like Vingt-un better than Commerce;
but with respect to any other leading characteristic,
I do not imagine that much has been unfolded.”

“Well,” said Charlotte, “I wish Jane success with all
my heart; and if she were married to him to-morrow,
I should think she had as good a chance of happiness, as
if she were to be studying his character for a twelvemonth.
Happiness in marriage is entirely a matter of
chance. If the dispositions of the parties are ever so
well known to each other, or ever so similar before-hand,
it does not advance their felicity in the least. They
always continue to grow sufficiently unlike afterwards to
have their share of vexation; and it is better to know
as little as possible of the defects of the person with whom
you are to pass your life.”

“You make me laugh, Charlotte; but it is not sound.
You know it is not sound, and that you would never
act in this way yourself.”

Occupied in observing Mr. Bingley’s attentions to her
sister, Elizabeth was far from suspecting that she was
herself becoming an object of some interest in the eyes
of his friend. Mr. Darcy had at first scarcely allowed
her to be pretty; he had looked at her without admiration
at the ball; and when they next met, he looked at her
only to criticise. But no sooner had he made it clear to
himself and his friends that she had hardly a good feature
in her face, than he began to find it was rendered uncommonly
intelligent by the beautiful expression of her dark
eyes. To this discovery succeeded some others equally
mortifying. Though he had detected with a critical eye
more than one failure of perfect symmetry in her form,
he was forced to acknowledge her figure to be light and
pleasing; and in spite of his asserting that her manners
were not those of the fashionable world, he was caught
by their easy playfulness. Of this she was perfectly
unaware; -- to her he was only the man who made himself
agreeable no where, and who had not thought her handsome
enough to dance with.
%%023%%

He began to wish to know more of her, and as a step
towards conversing with her himself, attended to her
conversation with others. His doing so drew her notice.
It was at Sir William Lucas’s, where a large party were
assembled.

“What does Mr. Darcy mean,” said she to Charlotte,
“by listening to my conversation with Colonel Forster?”

“That is a question which Mr. Darcy only can answer.”

“But if he does it any more I shall certainly let him
know that I see what he is about. He has a very satirical
eye, and if I do not begin by being impertinent myself,
I shall soon grow afraid of him.”

On his approaching them soon afterwards, though
without seeming to have any intention of speaking, Miss
Lucas defied her friend to mention such a subject to him,
which immediately provoking Elizabeth to do it, she
turned to him and said,

“Did not you think, Mr. Darcy, that I expressed myself
uncommonly well just now, when I was teazing Colonel
Forster to give us a ball at Meryton?”

“With great energy; -- but it is a subject which always
makes a lady energetic.”

“You are severe on us.”

“It will be \textit{her} turn soon to be teazed,” said Miss
Lucas. “I am going to open the instrument, Eliza, and
you know what follows.”

“You are a very strange creature by way of a friend! -- always
wanting me to play and sing before any body and
every body! -- If my vanity had taken a musical turn,
you would have been invaluable, but as it is, I would
really rather not sit down before those who must be in
the habit of hearing the very best performers.” On Miss
Lucas’s persevering, however, she added, “Very well;
if it must be so, it must.” And gravely glancing at
Mr. Darcy, “There is a fine old saying, which every body
here is of course familiar with -- ‘Keep your breath to cool
your porridge,’ -- and I shall keep mine to swell my
song.”
%%024%%

Her performance was pleasing, though by no means
capital. After a song or two, and before she could reply
to the entreaties of several that she would sing again, she
was eagerly succeeded at the instrument by her sister
Mary, who having, in consequence of being the only plain
one in the family, worked hard for knowledge and accomplishments,
was always impatient for display.

Mary had neither genius nor taste; and though vanity
had given her application, it had given her likewise a
pedantic air and conceited manner, which would have
injured a higher degree of excellence than she had reached.
Elizabeth, easy and unaffected, had been listened to with
much more pleasure, though not playing half so well;
and Mary, at the end of a long concerto, was glad to purchase
praise and gratitude by Scotch and Irish airs, at
the request of her younger sisters, who with some of the
Lucases and two or three officers joined eagerly in dancing
at one end of the room.

Mr. Darcy stood near them in silent indignation at such
a mode of passing the evening, to the exclusion of all
conversation, and was too much engrossed by his own thoughts
to perceive that Sir William Lucas was his neighbour, till
Sir William thus began.

“What a charming amusement for young people this
is, Mr. Darcy! -- There is nothing like dancing after all. --
I consider it as one of the first refinements of polished
societies.”

“Certainly, Sir; -- and it has the advantage also of
being in vogue amongst the less polished societies of the
world. -- Every savage can dance.”

Sir William only smiled. “Your friend performs
delightfully;” he continued after a pause, on seeing
Bingley join the group; -- “and I doubt not that you are
an adept in the science yourself, Mr. Darcy.”

“You saw me dance at Meryton, I believe, Sir.”

“Yes, indeed, and received no inconsiderable pleasure
from the sight. Do you often dance at St. James’s?”

“Never, sir.”
%%025%%

“Do you not think it would be a proper compliment to
the place?”

“It is a compliment which I never pay to any place
if I can avoid it.”

“You have a house in town, I conclude?”

Mr. Darcy bowed.

“I had once some thoughts of fixing in town myself -- for
I am fond of superior society; but I did not feel
quite certain that the air of London would agree with
Lady Lucas.”

He paused in hopes of an answer; but his companion
was not disposed to make any; and Elizabeth at that
instant moving towards them, he was struck with the
notion of doing a very gallant thing, and called out to her,

“My dear Miss Eliza, why are not you dancing? -- Mr.
Darcy, you must allow me to present this young lady
to you as a very desirable partner. -- You cannot refuse
to dance, I am sure, when so much beauty is before you.”
And taking her hand, he would have given it to Mr. Darcy,
who, though extremely surprised, was not unwilling to
receive it, when she instantly drew back, and said with
some discomposure to Sir William,

“Indeed, Sir, I have not the least intention of dancing. --
I entreat you not to suppose that I moved this way in
order to beg for a partner.”

Mr. Darcy with grave propriety requested to be allowed
the honour of her hand; but in vain. Elizabeth was determined;
nor did Sir William at all shake her purpose by his
attempt at persuasion.

“You excel so much in the dance, Miss Eliza, that it is
cruel to deny me the happiness of seeing you; and though
this gentleman dislikes the amusement in general, he can
have no objection, I am sure, to oblige us for one half hour.”

“Mr. Darcy is all politeness,” said Elizabeth, smiling.

“He is indeed -- but considering the inducement, my
dear Miss Eliza, we cannot wonder at his complaisance;
for who would object to such a partner?”

Elizabeth looked archly, and turned away. Her
%%026%%
resistance had not injured her with the gentleman, and
he was thinking of her with some complacency, when thus
accosted by Miss Bingley,

“I can guess the subject of your reverie.”

“I should imagine not.”

“You are considering how insupportable it would be
to pass many evenings in this manner -- in such society;
and indeed I am quite of your opinion. I was never
more annoyed! The insipidity and yet the noise; the
nothingness and yet the self-importance of all these
people! -- What would I give to hear your strictures on
them!”

“Your conjecture is totally wrong, I assure you. My
mind was more agreeably engaged. I have been meditating
on the very great pleasure which a pair of fine eyes
in the face of a pretty woman can bestow.”

Miss Bingley immediately fixed her eyes on his face,
and desired he would tell her what lady had the credit
of inspiring such reflections. Mr. Darcy replied with
great intrepidity,

“Miss Elizabeth Bennet.”

“Miss Elizabeth Bennet!” repeated Miss Bingley.
“I am all astonishment. How long has she been such
a favourite? -- and pray when am I to wish you joy?”

“That is exactly the question which I expected you
to ask. A lady’s imagination is very rapid; it jumps
from admiration to love, from love to matrimony in a
moment. I knew you would be wishing me joy.”

“Nay, if you are so serious about it, I shall consider
the matter as absolutely settled. You will have a charming
mother-in-law, indeed, and of course she will be always
at Pemberley with you.”

He listened to her with perfect indifference, while she
chose to entertain herself in this manner, and as his composure
convinced her that all was safe, her wit flowed long.
%%027%%

\Chapter{CHAPTER VII.}

Mr. Bennet’s property consisted almost entirely in an
estate of two thousand a year, which, unfortunately for
his daughters, was entailed in default of heirs male, on
a distant relation; and their mother’s fortune, though
ample for her situation in life, could but ill supply the
deficiency of his. Her father had been an attorney in
Meryton, and had left her four thousand pounds.

She had a sister married to a Mr. Phillips, who had
been a clerk to their father, and succeeded him in the
business, and a brother settled in London in a respectable
line of trade.

The village of Longbourn was only one mile from
Meryton; a most convenient distance for the young
ladies, who were usually tempted thither three or four
times a week, to pay their duty to their aunt and to a
milliner’s shop just over the way. The two youngest
of the family, Catherine and Lydia, were particularly
frequent in these attentions; their minds were more
vacant than their sisters’, and when nothing better offered,
a walk to Meryton was necessary to amuse their morning
hours and furnish conversation for the evening; and
however bare of news the country in general might be,
they always contrived to learn some from their aunt.
At present, indeed, they were well supplied both with news
and happiness by the recent arrival of a militia regiment
in the neighbourhood; it was to remain the whole winter,
and Meryton was the head quarters.

Their visits to Mrs. Philips were now productive of
the most interesting intelligence. Every day added
something to their knowledge of the officers’ names and
connections. Their lodgings were not long a secret, and
at length they began to know the officers themselves.
Mr. Philips visited them all, and this opened to his nieces
%%028%%
a source of felicity unknown before. They could talk of
nothing but officers; and Mr. Bingley’s large fortune,
the mention of which gave animation to their mother,
was worthless in their eyes when opposed to the regimentals
of an ensign.

After listening one morning to their effusions on this
subject, Mr. Bennet coolly observed,

“From all that I can collect by your manner of talking,
you must be two of the silliest girls in the country. I have
suspected it some time, but I am now convinced.”

Catherine was disconcerted, and made no answer; but
Lydia, with perfect indifference, continued to express her
admiration of Captain Carter, and her hope of seeing him
in the course of the day, as he was going the next morning
to London.

“I am astonished, my dear,” said Mrs. Bennet, “that
you should be so ready to think your own children silly.
If I wished to think slightingly of any body’s children,
it should not be of my own however.”

“If my children are silly I must hope to be always
sensible of it.”

“Yes -- but as it happens, they are all of them very
clever.”

“This is the only point, I flatter myself, on which we
do not agree. I had hoped that our sentiments coincided
in every particular, but I must so far differ from you
as to think our two youngest daughters uncommonly
foolish.”

“My dear Mr. Bennet, you must not expect such girls
to have the sense of their father and mother. -- When
they get to our age I dare say they will not think about
officers any more than we do. I remember the time when
I liked a red coat myself very well -- and indeed so I do
still at my heart; and if a smart young colonel, with
five or six thousand a year, should want one of my girls,
I shall not say nay to him; and I thought Colonel Forster
looked very becoming the other night at Sir William’s in
his regimentals.”
%%029%%

“Mama,” cried Lydia, “my aunt says that Colonel
Forster and Captain Carter do not go so often to Miss
Watson’s as they did when they first came; she sees
them now very often standing in Clarke’s library.”

Mrs. Bennet was prevented replying by the entrance of
the footman with a note for Miss Bennet; it came from
Netherfield, and the servant waited for an answer. Mrs.
Bennet’s eyes sparkled with pleasure, and she was eagerly
calling out, while her daughter read,

“Well, Jane, who is it from? what is it about? what
does he say? Well, Jane, make haste and tell us; make
haste, my love.”

“It is from Miss Bingley,” said Jane, and then read
it aloud.

\begin{letter}
“My dear Friend,

“If you are not so compassionate as to dine to-day
with Louisa and me, we shall be in danger of hating each
other for the rest of our lives, for a whole day’s tête-à-tête
between two women can never end without a quarrel.
Come as soon as you can on the receipt of this. My
brother and the gentlemen are to dine with the officers.
Yours ever,

\LetterSig{“Caroline Bingley.”}
\end{letter}

“With the officers!” cried Lydia. “I wonder my
aunt did not tell us of \textit{that}.”

“Dining out,” said Mrs. Bennet, “that is very
unlucky.”

“Can I have the carriage?” said Jane.

“No, my dear, you had better go on horseback, because
it seems likely to rain; and then you must stay all night.”

“That would be a good scheme,” said Elizabeth, “if you
were sure that they would not offer to send her home.”

“Oh! but the gentlemen will have Mr. Bingley’s chaise
to go to Meryton; and the Hursts have no horses to
theirs.”

“I had much rather go in the coach.”

“But, my dear, your father cannot spare the horses,
%%030%%
I am sure. They are wanted in the farm, Mr. Bennet,
are not they?”

“They are wanted in the farm much oftener than I can
get them.”

“But if you have got them to day,” said Elizabeth,
“my mother’s purpose will be answered.”

She did at last extort from her father an acknowledgment
that the horses were engaged. Jane was therefore
obliged to go on horseback, and her mother attended her
to the door with many cheerful prognostics of a bad day.
Her hopes were answered; Jane had not been gone long
before it rained hard. Her sisters were uneasy for her, but
her mother was delighted. The rain continued the whole
evening without intermission; Jane certainly could not
come back.

“This was a lucky idea of mine, indeed!” said Mrs.
Bennet, more than once, as if the credit of making it rain
were all her own. Till the next morning, however, she
was not aware of all the felicity of her contrivance. Breakfast
was scarcely over when a servant from Netherfield
brought the following note for Elizabeth:

\begin{letter}
“My dearest Lizzy,

“I \textsc{find} myself very unwell this morning, which,
I suppose, is to be imputed to my getting wet through
yesterday. My kind friends will not hear of my returning
home till I am better. They insist also on my seeing
Mr. Jones -- therefore do not be alarmed if you should
hear of his having been to me -- and excepting a sore-%
throat and head-ache there is not much the matter with
me.

\raggedleft “Yours, \&c.”
\end{letter}

“Well, my dear,” said Mr. Bennet, when Elizabeth had
read the note aloud, “if your daughter should have a
dangerous fit of illness, if she should die, it would be
a comfort to know that it was all in pursuit of Mr. Bingley,
and under your orders.”

“Oh! I am not at all afraid of her dying. People
do not die of little trifling colds. She will be taken good
%%031%%
care of. As long as she stays there, it is all very well.
I would go and see her, if I could have the carriage.”

Elizabeth, feeling really anxious, was determined to go
to her, though the carriage was not to be had; and as
she was no horse-woman, walking was her only alternative.
She declared her resolution.

“How can you be so silly,” cried her mother, “as to
think of such a thing, in all this dirt! You will not be
fit to be seen when you get there.”

“I shall be very fit to see Jane -- which is all I want.”

“Is this a hint to me, Lizzy,” said her father, “to send
for the horses?”

“No, indeed. I do not wish to avoid the walk. The
distance is nothing, when one has a motive; only three
miles. I shall be back by dinner.”

“I admire the activity of your benevolence,” observed
Mary, “but every impulse of feeling should be guided by
reason; and, in my opinion, exertion should always be
in proportion to what is required.”

“We will go as far as Meryton with you,” said Catherine
and Lydia. -- Elizabeth accepted their company, and the
three young ladies set off together.

“If we make haste,” said Lydia, as they walked along,
“perhaps we may see something of Captain Carter before
he goes.”

In Meryton they parted; the two youngest repaired
to the lodgings of one of the officers’ wives, and Elizabeth
continued her walk alone, crossing field after field at a
quick pace, jumping over stiles and springing over puddles
with impatient activity, and finding herself at last within
view of the house, with weary ancles, dirty stockings, and
a face glowing with the warmth of exercise.

She was shewn into the breakfast-parlour, where all
but Jane were assembled, and where her appearance
created a great deal of surprise. -- That she should have
walked three miles so early in the day, in such dirty
weather, and by herself, was almost incredible to Mrs.
Hurst and Miss Bingley; and Elizabeth was convinced
%%032%%
that they held her in contempt for it. She was received,
however, very politely by them; and in their brother’s
manners there was something better than politeness;
there was good humour and kindness. -- Mr. Darcy said
very little, and Mr. Hurst nothing at all. The former
was divided between admiration of the brilliancy which
exercise had given to her complexion, and doubt as to
the occasion’s justifying her coming so far alone. The
latter was thinking only of his breakfast.

Her enquiries after her sister were not very favourably
answered. Miss Bennet had slept ill, and though up,
was very feverish and not well enough to leave her room.
Elizabeth was glad to be taken to her immediately; and
Jane, who had only been withheld by the fear of giving
alarm or inconvenience, from expressing in her note how
much she longed for such a visit, was delighted at her
entrance. She was not equal, however, to much conversation,
and when Miss Bingley left them together,
could attempt little beside expressions of gratitude for
the extraordinary kindness she was treated with. Elizabeth
silently attended her.

When breakfast was over, they were joined by the
sisters; and Elizabeth began to like them herself, when
she saw how much affection and solicitude they shewed
for Jane. The apothecary came, and having examined
his patient, said, as might be supposed, that she had
caught a violent cold, and that they must endeavour to
get the better of it; advised her to return to bed, and
promised her some draughts. The advice was followed
readily, for the feverish symptoms increased, and her
head ached acutely. Elizabeth did not quit her room
for a moment, nor were the other ladies often absent;
the gentlemen being out, they had in fact nothing to do
elsewhere.

When the clock struck three, Elizabeth felt that she
must go; and very unwillingly said so. Miss Bingley
offered her the carriage, and she only wanted a little
pressing to accept it, when Jane testified such concern
%%033%%
in parting with her, that Miss Bingley was obliged to
convert the offer of the chaise into an invitation to remain
at Netherfield for the present. Elizabeth most thankfully
consented, and a servant was dispatched to Longbourn
to acquaint the family with her stay, and bring back
a supply of clothes.
%%034%%

\Chapter{CHAPTER VIII.}

At five o’clock the two ladies retired to dress, and
at half past six Elizabeth was summoned to dinner. To
the civil enquiries which then poured in, and amongst
which she had the pleasure of distinguishing the much
superior solicitude of Mr. Bingley’s, she could not make
a very favourable answer. Jane was by no means better.
The sisters, on hearing this, repeated three or four times
how much they were grieved, how shocking it was to have
a bad cold, and how excessively they disliked being ill
themselves; and then thought no more of the matter:
and their indifference towards Jane when not immediately
before them, restored Elizabeth to the enjoyment of all
her original dislike.

Their brother, indeed, was the only one of the party
whom she could regard with any complacency. His
anxiety for Jane was evident, and his attentions to herself
most pleasing, and they prevented her feeling herself so
much an intruder as she believed she was considered by
the others. She had very little notice from any but him.
Miss Bingley was engrossed by Mr. Darcy, her sister
scarcely less so; and as for Mr. Hurst, by whom Elizabeth
sat, he was an indolent man, who lived only to eat, drink,
and play at cards, who when he found her prefer a plain
dish to a ragout, had nothing to say to her.

When dinner was over, she returned directly to Jane,
and Miss Bingley began abusing her as soon as she was
out of the room. Her manners were pronounced to be
very bad indeed, a mixture of pride and impertinence;
she had no conversation, no stile, no taste, no beauty.
Mrs. Hurst thought the same, and added,

“She has nothing, in short, to recommend her, but
being an excellent walker. I shall never forget her
appearance this morning. She really looked almost wild.”
%%035%%

“She did indeed, Louisa. I could hardly keep my
countenance. Very nonsensical to come at all! Why
must \textit{she} be scampering about the country, because her
sister had a cold? Her hair so untidy, so blowsy!”

“Yes, and her petticoat; I hope you saw her petticoat,
six inches deep in mud, I am absolutely certain; and the
gown which had been let down to hide it, not doing its
office.”

“Your picture may be very exact, Louisa,” said
Bingley; “but this was all lost upon me. I thought
Miss Elizabeth Bennet looked remarkably well, when she
came into the room this morning. Her dirty petticoat
quite escaped my notice.”

“\textit{You} observed it, Mr. Darcy, I am sure,” said Miss
Bingley; “and I am inclined to think that you would
not wish to see \textit{your sister} make such an exhibition.”

“Certainly not.”

“To walk three miles, or four miles, or five miles, or
whatever it is, above her ancles in dirt, and alone, quite
alone! what could she mean by it? It seems to me
to shew an abominable sort of conceited independence,
a most country town indifference to decorum.”

“It shews an affection for her sister that is very
pleasing,” said Bingley.

“I am afraid, Mr. Darcy,” observed Miss Bingley, in
a half whisper, “that this adventure has rather affected
your admiration of her fine eyes.”

“Not at all,” he replied; “they were brightened by
the exercise.” -- A short pause followed this speech, and
Mrs. Hurst began again.

“I have an excessive regard for Jane Bennet, she is
really a very sweet girl, and I wish with all my heart she
were well settled. But with such a father and mother, and
such low connections, I am afraid there is no chance of it.”

“I think I have heard you say, that their uncle is an
attorney in Meryton.”

“Yes; and they have another, who lives somewhere
near Cheapside.”
%%036%%

“That is capital,” added her sister, and they both
laughed heartily.

“If they had uncles enough to fill \textit{all} Cheapside,” cried
Bingley, “it would not make them one jot less agreeable.”

“But it must very materially lessen their chance of
marrying men of any consideration in the world,” replied
Darcy.

To this speech Bingley made no answer; but his sisters
gave it their hearty assent, and indulged their mirth for
some time at the expense of their dear friend’s vulgar
relations.

With a renewal of tenderness, however, they repaired
to her room on leaving the dining-parlour, and sat with
her till summoned to coffee. She was still very poorly,
and Elizabeth would not quit her at all, till late in the
evening, when she had the comfort of seeing her asleep,
and when it appeared to her rather right than pleasant
that she should go down stairs herself. On entering the
drawing-room she found the whole party at loo, and was
immediately invited to join them; but suspecting them
to be playing high she declined it, and making her sister
the excuse, said she would amuse herself for the short
time she could stay below with a book. Mr. Hurst looked
at her with astonishment.

“Do you prefer reading to cards?” said he; “that
is rather singular.”

“Miss Eliza Bennet,” said Miss Bingley, “despises cards.
She is a great reader and has no pleasure in anything else.”

“I deserve neither such praise nor such censure,” cried
Elizabeth; “I am \textit{not} a great reader, and I have pleasure
in many things.”

“In nursing your sister I am sure you have pleasure,”
said Bingley; “and I hope it will soon be increased by
seeing her quite well.”

Elizabeth thanked him from her heart, and then walked
towards a table where a few books were lying. He immediately
offered to fetch her others; all that his library
afforded.
%%037%%

“And I wish my collection were larger for your benefit
and my own credit; but I am an idle fellow, and though
I have not many, I have more than I ever look into.”

Elizabeth assured him that she could suit herself perfectly
with those in the room.

“I am astonished,” said Miss Bingley, “that my father
should have left so small a collection of books. -- What
a delightful library you have at Pemberley, Mr. Darcy!”

“It ought to be good,” he replied, “it has been the
work of many generations.”

“And then you have added so much to it yourself, you
are always buying books.”

“I cannot comprehend the neglect of a family library
in such days as these.”

“Neglect! I am sure you neglect nothing that can add
to the beauties of that noble place. Charles, when you
build \textit{your} house, I wish it may be half as delightful as
Pemberley.”

“I wish it may.”

“But I would really advise you to make your purchase in
that neighbourhood, and take Pemberley for a kind of model.
There is not a finer county in England than Derbyshire.”

“With all my heart; I will buy Pemberley itself if
Darcy will sell it.”

“I am talking of possibilities, Charles.”

“Upon my word, Caroline, I should think it more
possible to get Pemberley by purchase than by imitation.”

Elizabeth was so much caught by what passed, as to
leave her very little attention for her book; and soon
laying it wholly aside, she drew near the card-table, and
stationed herself between Mr. Bingley and his eldest sister,
to observe the game.

“Is Miss Darcy much grown since the spring?” said
Miss Bingley; “will she be as tall as I am?”

“I think she will. She is now about Miss Elizabeth
Bennet’s height, or rather taller.”

“How I long to see her again! I never met with
anybody who delighted me so much. Such a countenance,
%%038%%
such manners! and so extremely accomplished for her
age! Her performance on the piano-forte is exquisite.”

“It is amazing to me,” said Bingley, “how young
ladies can have patience to be so very accomplished, as
they all are.”

“All young ladies accomplished! My dear Ch\-arles, what
do you mean?”

“Yes, all of them, I think. They all paint tables,
cover skreens and net purses. I scarcely know any one
who cannot do all this, and I am sure I never heard
a young lady spoken of for the first time, without being
informed that she was very accomplished.”

“Your list of the common extent of accomplishments,”
said Darcy, “has too much truth. The word is applied
to many a woman who deserves it no otherwise than by
netting a purse, or covering a skreen. But I am very far
from agreeing with you in your estimation of ladies in
general. I cannot boast of knowing more than half a dozen,
in the whole range of my acquaintance, that are really
accomplished.”

“Nor I, I am sure,” said Miss Bingley.

“Then,” observed Elizabeth, “you must comprehend
a great deal in your idea of an accomplished woman.”

“Yes; I do comprehend a great deal in it.”

“Oh! certainly,” cried his faithful assistant, “no one
can be really esteemed accomplished, who does not greatly
surpass what is usually met with. A woman must have
a thorough knowledge of music, singing, drawing, dancing,
and the modern languages, to deserve the word; and besides
all this, she must possess a certain something in her
air and manner of walking, the tone of her voice, her address
and expressions, or the word will be but half deserved.”

“All this she must possess,” added Darcy, “and to
all this she must yet add something more substantial, in
the improvement of her mind by extensive reading.”

“I am no longer surprised at your knowing \textit{only} six
accomplished women. I rather wonder now at your
knowing \textit{any}.”
%%039%%

“Are you so severe upon your own sex, as to doubt the
possibility of all this?”

“\textit{I} never saw such a woman. \textit{I} never saw such capacity,
and taste, and application, and elegance, as you describe,
united.”

Mrs. Hurst and Miss Bingley both cried out against the
injustice of her implied doubt, and were both protesting that
they knew many women who answered this description,
when Mr. Hurst called them to order, with bitter complaints
of their inattention to what was going forward. As all
conversation was thereby at an end, Elizabeth soon afterwards
left the room.

“Eliza Bennet,” said Miss Bingley, when the door was
closed on her, “is one of those young ladies who seek to
recommend themselves to the other sex, by undervaluing
their own; and with many men, I dare say, it succeeds.
But, in my opinion, it is a paltry device, a very mean art.”

“Undoubtedly,” replied Darcy, to whom this remark
was chiefly addressed, “there is meanness in \textit{all} the arts
which ladies sometimes condescend to employ for captivation.
Whatever bears affinity to cunning is despicable.”

Miss Bingley was not so entirely satisfied with this reply
as to continue the subject.

Elizabeth joined them again only to say that her sister
was worse, and that she could not leave her. Bingley
urged Mr. Jones’s being sent for immediately; while his
sisters, convinced that no country advice could be of any
service, recommended an express to town for one of the
most eminent physicians. This, she would not hear of;
but she was not so unwilling to comply with their brother’s
proposal; and it was settled that Mr. Jones should be
sent for early in the morning, if Miss Bennet were not
decidedly better. Bingley was quite uncomfortable; his
sisters declared that they were miserable. They solaced
their wretchedness, however, by duets after supper, while
he could find no better relief to his feelings than by giving
his housekeeper directions that every possible attention
might be paid to the sick lady and her sister.
%%040%%

\Chapter{CHAPTER IX.}

Elizabeth passed the chief of the night in her sister’s
room, and in the morning had the pleasure of being able
to send a tolerable answer to the enquiries which she very
early received from Mr. Bingley by a housemaid, and some
time afterwards from the two elegant ladies who waited
on his sisters. In spite of this amendment, however, she
requested to have a note sent to Longbourn, desiring her
mother to visit Jane, and form her own judgment of her
situation. The note was immediately dispatched, and its
contents as quickly complied with. Mrs. Bennet, accompanied
by her two youngest girls, reached Netherfield soon
after the family breakfast.

Had she found Jane in any apparent danger, Mrs.
Bennet would have been very miserable; but being
satisfied on seeing her that her illness was not alarming,
she had no wish of her recovering immediately, as her
restoration to health would probably remove her from
Netherfield. She would not listen therefore to her
daughter’s proposal of being carried home; neither did
the apothecary, who arrived about the same time, think
it at all advisable. After sitting a little while with Jane,
on Miss Bingley’s appearance and invitation, the mother
and three daughters all attended her into the breakfast
parlour. Bingley met them with hopes that Mrs. Bennet
had not found Miss Bennet worse than she expected.

“Indeed I have, Sir,” was her answer. “She is a great
deal too ill to be moved. Mr. Jones says we must not
think of moving her. We must trespass a little longer
on your kindness.”

“Removed!” cried Bingley. “It must not be
th\-ought of. My sister, I am sure, will not hear of her
removal.”

“You may depend upon it, Madam,” said Miss Bingley,
%%041%%
with cold civility, “that Miss Bennet shall receive every
possible attention while she remains with us.”

Mrs. Bennet was profuse in her acknowledgments.

“I am sure,” she added, “if it was not for such good
friends I do not know what would become of her, for she
is very ill indeed, and suffers a vast deal, though with the
greatest patience in the world, which is always the way with
her, for she has, without exception, the sweetest temper I
ever met with. I often tell my other girls they are nothing
to \textit{her}. You have a sweet room here, Mr. Bingley, and a
charming prospect over that gravel walk. I do not know
a place in the country that is equal to Netherfield. You
will not think of quitting it in a hurry I hope, though
you have but a short lease.”

“Whatever I do is done in a hurry,” replied he; “and
therefore if I should resolve to quit Netherfield, I should
probably be off in five minutes. At present, however,
I consider myself as quite fixed here.”

“That is exactly what I should have supposed of you,”
said Elizabeth.

“You begin to comprehend me, do you?” cried he,
turning towards her.

“Oh! yes -- I understand you perfectly.”

“I wish I might take this for a compliment; but to
be so easily seen through I am afraid is pitiful.”

“That is as it happens. It does not necessarily follow
that a deep, intricate character is more or less estimable
than such a one as yours.”

“Lizzy,” cried her mother, “remember where you are,
and do not run on in the wild manner that you are suffered
to do at home.”

“I did not know before,” continued Bingley immediately,
“that you were a studier of character. It must be
an amusing study.”

“Yes; but intricate characters are the \textit{most} amusing.
They have at least that advantage.”

“The country,” said Darcy, “can in general supply but
few subjects for such a study. In a country
%%042%%
neighbourhood you move in a very confined and unvarying
society.”

“But people themselves alter so much, that there is
something new to be observed in them for ever.”

“Yes, indeed,” cried Mrs. Bennet, offended by his
manner of mentioning a country neighbourhood. “I
assure you there is quite as much of \textit{that} going on in the
country as in town.”

Every body was surprised; and Darcy, after looking
at her for a moment, turned silently away. Mrs. Bennet,
who fancied she had gained a complete victory over him,
continued her triumph.

“I cannot see that London has any great advantage
over the country for my part, except the shops and public
places. The country is a vast deal pleasanter, is not it,
Mr. Bingley?”

“When I am in the country,” he replied, “I never
wish to leave it; and when I am in town it is pretty much
the same. They have each their advantages, and I can
be equally happy in either.”

“Aye -- that is because you have the right disposition.
But that gentleman,” looking at Darcy, “seemed to think
the country was nothing at all.”

“Indeed, Mama, you are mistaken,” said Elizabeth,
blushing for her mother. “You quite mistook Mr. Darcy.
He only meant that there were not such a variety of people
to be met with in the country as in town, which you must
acknowledge to be true.”

“Certainly, my dear, nobody said there were; but as
to not meeting with many people in this neighbourhood,
I believe there are few neighbourhoods larger. I know
we dine with four and twenty families.”

Nothing but concern for Elizabeth could enable Bingley
to keep his countenance. His sister was less delicate, and
directed her eye towards Mr. Darcy with a very expressive
smile. Elizabeth, for the sake of saying something that
might turn her mother’s thoughts, now asked her if Charlotte
Lucas had been at Longbourn since \textit{her} coming away.
%%043%%

“Yes, she called yesterday with her father. What an
agreeable man Sir William is, Mr. Bingley -- is not he?
so much the man of fashion! so genteel and so easy! -- He
has always something to say to every body. -- \textit{That} is
my idea of good breeding; and those persons who fancy
themselves very important and never open their mouths,
quite mistake the matter.”

“Did Charlotte dine with you?”

“No, she would go home. I fancy she was wanted
about the mince pies. For my part, Mr. Bingley, \textit{I} always
keep servants that can do their own work; \textit{my} daughters
are brought up differently. But every body is to judge
for themselves, and the Lucases are very good sort of
girls, I assure you. It is a pity they are not handsome!
Not that \textit{I} think Charlotte so \textit{very} plain -- but then she is
our particular friend.”

“She seems a very pleasant young woman,” said
Bingley.

“Oh! dear, yes; -- but you must own she is very plain.
Lady Lucas herself has often said so, and envied me
Jane’s beauty. I do not like to boast of my own child,
but to be sure, Jane -- one does not often see any body
better looking. It is what every body says. I do not
trust my own partiality. When she was only fifteen,
there was a gentleman at my brother Gardiner’s in town,
so much in love with her, that my sister-in-law was sure
he would make her an offer before we came away. But
however he did not. Perhaps he thought her too young.
However, he wrote some verses on her, and very pretty
they were.”

“And so ended his affection,” said Elizabeth impatiently.
“There has been many a one, I fancy, overcome
in the same way. I wonder who first discovered
the efficacy of poetry in driving away love!”

“I have been used to consider poetry as the \textit{food} of
love,” said Darcy.

“Of a fine, stout, healthy love it may. Every thing
nourishes what is strong already. But if it be only a
%%044%%
slight, thin sort of inclination, I am convinced that one
good sonnet will starve it entirely away.”

Darcy only smiled; and the general pause which ensued
made Elizabeth tremble lest her mother should be exposing
herself again. She longed to speak, but could think of
nothing to say; and after a short silence Mrs. Bennet
began repeating her thanks to Mr. Bingley for his kindness
to Jane, with an apology for troubling him also with
Lizzy. Mr. Bingley was unaffectedly civil in his answer,
and forced his younger sister to be civil also, and say what
the occasion required. She performed her part indeed
without much graciousness, but Mrs. Bennet was satisfied,
and soon afterwards ordered her carriage. Upon this
signal, the youngest of her daughters put herself forward.
The two girls had been whispering to each other during
the whole visit, and the result of it was, that the youngest
should tax Mr. Bingley with having promised on his first
coming into the country to give a ball at Netherfield.

Lydia was a stout, well-grown girl of fifteen, with a fine
complexion and good-humoured countenance; a favourite
with her mother, whose affection had brought her into
public at an early age. She had high animal spirits, and
a sort of natural self-consequence, which the attentions
of the officers, to whom her uncle’s good dinners and
her own easy manners recommended her, had increased
into assurance. She was very equal therefore to address
Mr. Bingley on the subject of the ball, and abruptly
reminded him of his promise; adding, that it would be
the most shameful thing in the world if he did not keep it.
His answer to this sudden attack was delightful to their
mother’s ear.

“I am perfectly ready, I assure you, to keep my engagement;
and when your sister is recovered, you shall if you
please name the very day of the ball. But you would not
wish to be dancing while she is ill.”

Lydia declared herself satisfied. “Oh! yes -- it would
be much better to wait till Jane was well, and by that time
%%045%%
most likely Captain Carter would be at Meryton again.
And when you have given \textit{your} ball,” she added, “I shall
insist on their giving one also. I shall tell Colonel Forster
it will be quite a shame if he does not.”

Mrs. Bennet and her daughters then departed, and
Elizabeth returned instantly to Jane, leaving her own
and her relations’ behaviour to the remarks of the two
ladies and Mr. Darcy; the latter of whom, however,
could not be prevailed on to join in their censure of \textit{her},
in spite of all Miss Bingley’s witticisms on \textit{fine eyes}.
%%046%%

\Chapter{CHAPTER X.}

The day passed much as the day before had done.
Mrs. Hurst and Miss Bingley had spent some hours of the
morning with the invalid, who continued, though slowly,
to mend; and in the evening Elizabeth joined their party
in the drawing-room. The loo table, however, did not
appear. Mr. Darcy was writing, and Miss Bingley, seated
near him, was watching the progress of his letter, and
repeatedly calling off his attention by messages to his
sister. Mr. Hurst and Mr. Bingley were at piquet, and
Mrs. Hurst was observing their game.

Elizabeth took up some needlework, and was sufficiently
amused in attending to what passed between Darcy and
his companion. The perpetual commendations of the
lady either on his hand-writing, or on the evenness of
his lines, or on the length of his letter, with the perfect
unconcern with which her praises were received, formed
a curious dialogue, and was exactly in unison with her
opinion of each.

“How delighted Miss Darcy will be to receive such
a letter!”

He made no answer.

“You write uncommonly fast.”

“You are mistaken. I write rather slowly.”

“How many letters you must have occasion to write
in the course of the year! Letters of business too! How
odious I should think them!”

“It is fortunate, then, that they fall to my lot instead
of to yours.”

“Pray tell your sister that I long to see her.”

“I have already told her so once, by your desire.”

“I am afraid you do not like your pen. Let me mend
it for you. I mend pens remarkably well.”

“Thank you -- but I always mend my own.”
%%047%%

“How can you contrive to write so even?”

He was silent.

“Tell your sister I am delighted to hear of her improvement
on the harp, and pray let her know that I am quite
in raptures with her beautiful little design for a table,
and I think it infinitely superior to Miss Grantley’s.”

“Will you give me leave to defer your raptures till
I write again? -- At present I have not room to do them
justice.”

“Oh! it is of no consequence. I shall see her in January.
But do you always write such charming long letters to
her, Mr. Darcy?”

“They are generally long; but whether always
charming, it is not for me to determine.”

“It is a rule with me, that a person who can write
a long letter, with ease, cannot write ill.”

“That will not do for a compliment to Darcy, Caroline,”
cried her brother -- “because he does \textit{not} write with ease.
He studies too much for words of four syllables. -- Do not
you, Darcy?”

“My stile of writing is very different from yours.”

“Oh!” cried Miss Bingley, “Charles writes in the most
careless way imaginable. He leaves out half his words,
and blots the rest.”

“My ideas flow so rapidly that I have not time to
express them -- by which means my letters sometimes
convey no ideas at all to my correspondents.”

“Your humility, Mr. Bingley,” said Elizabeth, “must
disarm reproof.”

“Nothing is more deceitful,” said Darcy, “than the
appearance of humility. It is often only carelessness of
opinion, and sometimes an indirect boast.”

“And which of the two do you call \textit{my} little recent
piece of modesty?”

“The indirect boast; -- for you are really proud of your
defects in writing, because you consider them as proceeding
from a rapidity of thought and carelessness of
execution, which if not estimable, you think at least
%%048%%
highly interesting. The power of doing any thing with
quickness is always much prized by the possessor, and
often without any attention to the imperfection of the
performance. When you told Mrs. Bennet this morning
that if you ever resolved on quitting Netherfield you
should be gone in five minutes, you meant it to be a sort
of panegyric, of compliment to yourself -- and yet what
is there so very laudable in a precipitance which must
leave very necessary business undone, and can be of no real
advantage to yourself or any one else?”

“Nay,” cried Bingley, “this is too much, to remember
at night all the foolish things that were said in the morning.
And yet, upon my honour, I believed what I said of myself
to be true, and I believe it at this moment. At least,
therefore, I did not assume the character of needless
precipitance merely to shew off before the ladies.”

“I dare say you believed it; but I am by no means
convinced that you would be gone with such celerity.
Your conduct would be quite as dependant on chance as
that of any man I know; and if, as you were mounting
your horse, a friend were to say, ‘Bingley, you had better
stay till next week,’ you would probably do it, you would
probably not go -- and, at another word, might stay a
month.”

“You have only proved by this,” cried Elizabeth,
“that Mr. Bingley did not do justice to his own disposition.
You have shewn him off now much more than
he did himself.”

“I am exceedingly gratified,” said Bingley, “by your
converting what my friend says into a compliment on the
sweetness of my temper. But I am afraid you are giving
it a turn which that gentleman did by no means intend;
for he would certainly think the better of me, if under
such a circumstance I were to give a flat denial, and ride
off as fast as I could.”

“Would Mr. Darcy then consider the rashness of your
original intention as atoned for by your obstinacy in
adhering to it?”
%%049%%

“Upon my word I cannot exactly explain the matter,
Darcy must speak for himself.”

“You expect me to account for opinions which you
chuse to call mine, but which I have never acknowledged.
Allowing the case, however, to stand according to your
representation, you must remember, Miss Bennet, that the
friend who is supposed to desire his return to the house,
and the delay of his plan, has merely desired it, asked it
without offering one argument in favour of its propriety.”

“To yield readily -- easily -- to the \textit{persuasion} of a friend
is no merit with you.”

“To yield without conviction is no compliment to the
understanding of either.”

“You appear to me, Mr. Darcy, to allow nothing for
the influence of friendship and affection. A regard for
the requester would often make one readily yield to a
request, without waiting for arguments to reason one into
it. I am not particularly speaking of such a case as you
have supposed about Mr. Bingley. We may as well wait,
perhaps, till the circumstance occurs, before we discuss
the discretion of his behaviour thereupon. But in general
and ordinary cases between friend and friend, where one
of them is desired by the other to change a resolution
of no very great moment, should you think ill of that person
for complying with the desire, without waiting to be
argued into it?”

“Will it not be advisable, before we proceed on this
subject, to arrange with rather more precision the degree
of importance which is to appertain to this request, as well
as the degree of intimacy subsisting between the parties?”

“By all means,” cried Bingley; “let us hear all the
particulars, not forgetting their comparative height and
size; for that will have more weight in the argument,
Miss Bennet, than you may be aware of. I assure you
that if Darcy were not such a great tall fellow, in comparison
with myself, I should not pay him half so much
deference. I declare I do not know a more aweful object
than Darcy, on particular occasions, and in particular
%%050%%
places; at his own house especially, and of a Sunday
evening when he has nothing to do.”

Mr. Darcy smiled; but Elizabeth thought she could
perceive that he was rather offended; and therefore
checked her laugh. Miss Bingley warmly resented the
indignity he had received, in an expostulation with her
brother for talking such nonsense.

“I see your design, Bingley,” said his friend. -- “You
dislike an argument, and want to silence this.”

“Perhaps I do. Arguments are too much like disputes.
If you and Miss Bennet will defer yours till I am out of
the room, I shall be very thankful; and then you may
say whatever you like of me.”

“What you ask,” said Elizabeth, “is no sacrifice on
my side; and Mr. Darcy had much better finish his letter.”

Mr. Darcy took her advice, and did finish his letter.

When that business was over, he applied to Miss Bingley
and Elizabeth for the indulgence of some music. Miss
Bingley moved with alacrity to the piano-forte, and after
a polite request that Elizabeth would lead the way, which
the other as politely and more earnestly negatived, she
seated herself.

Mrs. Hurst sang with her sister, and while they were
thus employed Elizabeth could not help observing as she
turned over some music books that lay on the instrument,
how frequently Mr. Darcy’s eyes were fixed on her. She
hardly knew how to suppose that she could be an object
of admiration to so great a man; and yet that he should
look at her because he disliked her, was still more strange.
She could only imagine however at last, that she drew
his notice because there was a something about her more
wrong and reprehensible, according to his ideas of right,
than in any other person present. The supposition did
not pain her. She liked him too little to care for his
approbation.

After playing some Italian songs, Miss Bingley varied
the charm by a lively Scotch air; and soon afterwards
Mr. Darcy, drawing near Elizabeth, said to her --
%%051%%

“Do not you feel a great inclination, Miss Bennet, to
seize such an opportunity of dancing a reel?”

She smiled, but made no answer. He repeated the
question, with some surprise at her silence.

“Oh!” said she, “I heard you before; but I could not
immediately determine what to say in reply. You wanted
me, I know, to say ‘Yes,’ that you might have the pleasure
of despising my taste; but I always delight in overthrowing
those kind of schemes, and cheating a person of their
premeditated contempt. I have therefore made up my
mind to tell you, that I do not want to dance a reel at all -- and
now despise me if you dare.”

“Indeed I do not dare.”

Elizabeth, having rather expected to affront him, was
amazed at his gallantry; but there was a mixture of
sweetness and archness in her manner which made it
difficult for her to affront anybody; and Darcy had never
been so bewitched by any woman as he was by her. He
really believed, that were it not for the inferiority of her
connections, he should be in some danger.

Miss Bingley saw, or suspected enough to be jealous;
and her great anxiety for the recovery of her dear friend
Jane, received some assistance from her desire of getting
rid of Elizabeth.

She often tried to provoke Darcy into disliking her
guest, by talking of their supposed marriage, and planning
his happiness in such an alliance.

“I hope,” said she, as they were walking together in
the shrubbery the next day, “you will give your mother-in-law
a few hints, when this desirable event takes place,
as to the advantage of holding her tongue; and if you
can compass it, do cure the younger girls of running after
the officers. -- And, if I may mention so delicate a subject,
endeavour to check that little something, bordering on
conceit and impertinence, which your lady possesses.”

“Have you any thing else to propose for my domestic
felicity?”

“Oh! yes. -- Do let the portraits of your uncle and aunt
%%052%%
Philips be placed in the gallery at Pemberley. Put them
next to your great uncle the judge. They are in the
same profession, you know; only in different lines. As for
your Elizabeth’s picture, you must not attempt to have it
taken, for what painter could do justice to those beautiful
eyes?”

“It would not be easy, indeed, to catch their expression,
but their colour and shape, and the eye-lashes, so remarkably
fine, might be copied.”

At that moment they were met from another walk, by
Mrs. Hurst and Elizabeth herself.

“I did not know that you intended to walk,” said
Miss Bingley, in some confusion, lest they had been
overheard.

“You used us abominably ill,” answered Mrs. Hurst,
“in running away without telling us that you were coming
out.”

Then taking the disengaged arm of Mr. Darcy, she left
Elizabeth to walk by herself. The path just admitted
three. Mr. Darcy felt their rudeness and immediately
said, --

“This walk is not wide enough for our party. We had
better go into the avenue.”

But Elizabeth, who had not the least inclination to
remain with them, laughingly answered,

“No, no; stay where you are. -- You are charmingly
group’d, and appear to uncommon advantage. The
picturesque would be spoilt by admitting a fourth.
Good bye.”

She then ran gaily off, rejoicing as she rambled about,
in the hope of being at home again in a day or two. Jane
was already so much recovered as to intend leaving her
room for a couple of hours that evening.
%%053%%

\Chapter{CHAPTER XI.}

When the ladies removed after dinner, Elizabeth ran
up to her sister, and seeing her well guarded from cold,
attended her into the drawing-room; where she was
welcomed by her two friends with many professions of
pleasure; and Elizabeth had never seen them so agreeable
as they were during the hour which passed before
the gentlemen appeared. Their powers of conversation
were considerable. They could describe an entertainment
with accuracy, relate an anecdote with humour, and laugh
at their acquaintance with spirit.

But when the gentlemen entered, Jane was no longer
the first object. Miss Bingley’s eyes were instantly turned
towards Darcy, and she had something to say to him
before he had advanced many steps. He addressed himself
directly to Miss Bennet, with a polite congratulation;
Mr. Hurst also made her a slight bow, and said he was
“very glad;” but diffuseness and warmth remained for
Bingley’s salutation. He was full of joy and attention.
The first half hour was spent in piling up the fire, lest she
should suffer from the change of room; and she removed
at his desire to the other side of the fire-place, that she
might be farther from the door. He then sat down by
her, and talked scarcely to any one else. Elizabeth, at
work in the opposite corner, saw it all with great delight.

When tea was over, Mr. Hurst reminded his sister-in-law
of the card-table -- but in vain. She had obtained
private intelligence that Mr. Darcy did not wish for cards;
and Mr. Hurst soon found even his open petition rejected.
She assured him that no one intended to play, and the
silence of the whole party on the subject, seemed to justify
her. Mr. Hurst had therefore nothing to do, but to stretch
himself on one of the sophas and go to sleep. Darcy took
up a book; Miss Bingley did the same; and Mrs. Hurst,
%%054%%
principally occupied in playing with her bracelets and rings,
joined now and then in her brother’s conversation with
Miss Bennet.

Miss Bingley’s attention was quite as much engaged in
watching Mr. Darcy’s progress through \textit{his} book, as in
reading her own; and she was perpetually either making
some inquiry, or looking at his page. She could not win
him, however, to any conversation; he merely answered
her question, and read on. At length, quite exhausted
by the attempt to be amused with her own book, which
she had only chosen because it was the second volume
of his, she gave a great yawn and said, “How pleasant
it is to spend an evening in this way! I declare after all
there is no enjoyment like reading! How much sooner
one tires of any thing than of a book! -- When I have
a house of my own, I shall be miserable if I have not an
excellent library.”

No one made any reply. She then yawned again, threw
aside her book, and cast her eyes round the room in
quest of some amusement; when hearing her brother
mentioning a ball to Miss Bennet, she turned suddenly
towards him and said,

“By the bye, Charles, are you really serious in meditating
a dance at Netherfield? -- I would advise you, before
you determine on it, to consult the wishes of the present
party; I am much mistaken if there are not some among
us to whom a ball would be rather a punishment than
a pleasure.”

“If you mean Darcy,” cried her brother, “he may go
to bed, if he chuses, before it begins -- but as for the ball,
it is quite a settled thing; and as soon as Nicholls has
made white soup enough I shall send round my cards.”

“I should like balls infinitely better,” she replied, “if
they were carried on in a different manner; but there is
something insufferably tedious in the usual process of
such a meeting. It would surely be much more rational
if conversation instead of dancing made the order of
the day.”
%%055%%

“Much more rational, my dear Caroline, I dare say
but it would not be near so much like a ball.”

Miss Bingley made no answer; and soon afterwards
got up and walked about the room. Her figure was
elegant, and she walked well; -- but Darcy, at whom it
was all aimed, was still inflexibly studious. In the desperation
of her feelings she resolved on one effort more; and,
turning to Elizabeth, said,

“Miss Eliza Bennet, let me persuade you to follow my
example, and take a turn about the room. -- I assure you
it is very refreshing after sitting so long in one attitude.”

Elizabeth was surprised, but agreed to it immediately.
Miss Bingley succeeded no less in the real object of her
civility; Mr. Darcy looked up. He was as much awake
to the novelty of attention in that quarter as Elizabeth
herself could be, and unconsciously closed his book. He
was directly invited to join their party, but he declined
it, observing, that he could imagine but two motives for
their chusing to walk up and down the room together,
with either of which motives his joining them would
interfere. “What could he mean? she was dying to
know what could be his meaning” -- and asked Elizabeth
whether she could at all understand him?

“Not at all,” was her answer; “but depend upon it,
he means to be severe on us, and our surest way of disappointing
him, will be to ask nothing about it.”

Miss Bingley, however, was incapable of disappointing
Mr. Darcy in any thing, and persevered therefore in
requiring an explanation of his two motives.

“I have not the smallest objection to explaining them,”
said he, as soon as she allowed him to speak. “You
either chuse this method of passing the evening because
you are in each other’s confidence and have secret affairs
to discuss, or because you are conscious that your figures
appear to the greatest advantage in walking; -- if the first,
I should be completely in your way; -- and if the second,
I can admire you much better as I sit by the fire.”

“Oh! shocking!” cried Miss Bingley. “I never heard
%%056%%
any thing so abominable. How shall we punish him for
such a speech?”

“Nothing so easy, if you have but the inclination,”
said Elizabeth. “We can all plague and punish one
another. Teaze him -- laugh at him. -- Intimate as you
are, you must know how it is to be done.”

“But upon my honour I do \textit{not}. I do assure you that
my intimacy has not yet taught me \textit{that}. Teaze calmness of
temper and presence of mind! No, no -- I feel he may defy
us there. And as to laughter, we will not expose ourselves,
if you please, by attempting to laugh without a subject.
Mr. Darcy may hug himself.”

“Mr. Darcy is not to be laughed at!” cried Elizabeth.
“That is an uncommon advantage, and uncommon I hope
it will continue, for it would be a great loss to \textit{me} to have
many such acquaintance. I dearly love a laugh.”

“Miss Bingley,” said he, “has given me credit for more
than can be. The wisest and the best of men, nay, the
wisest and best of their actions, may be rendered ridiculous
by a person whose first object in life is a joke.”

“Certainly,” replied Elizabeth -- “there are such people,
but I hope I am not one of \textit{them}. I hope I never ridicule
what is wise or good. Follies and nonsense, whims and
inconsistencies \textit{do} divert me, I own, and I laugh at them
whenever I can. -- But these, I suppose, are precisely what
you are without.”

“Perhaps that is not possible for any one. But it has
been the study of my life to avoid those weaknesses which
often expose a strong understanding to ridicule.”

“Such as vanity and pride.”

“Yes, vanity is a weakness indeed. But pride -- where
there is a real superiority of mind, pride will be always
under good regulation.”

Elizabeth turned away to hide a smile.

“Your examination of Mr. Darcy is over, I presume,”
said Miss Bingley; -- “and pray what is the result?”

“I am perfectly convinced by it that Mr. Darcy has
no defect. He owns it himself without disguise.”
%%057%%

“No” -- said Darcy, “I have made no such pretension.
I have faults enough, but they are not, I hope, of understanding.
My temper I dare not vouch for. -- It is I believe
too little yielding -- certainly too little for the convenience
of the world. I cannot forget the follies and vices of others
so soon as I ought, nor their offences against myself.
My feelings are not puffed about with every attempt to
move them. My temper would perhaps be called
resentful. -- My good opinion once lost is lost for ever.”

“\textit{That} is a failing indeed!” -- cried Elizabeth. “Implacable
resentment \textit{is} a shade in a character. But you
have chosen your fault well. -- I really cannot \textit{laugh} at it.
You are safe from me.”

“There is, I believe, in every disposition a tendency
to some particular evil, a natural defect, which not even
the best education can overcome.”

“And \textit{your} defect is a propensity to hate every body.”

“And yours,” he replied with a smile, “is wilfully to
misunderstand them.”

“Do let us have a little music,” -- cried Miss Bingley,
tired of a conversation in which she had no share. --
“Louisa, you will not mind my waking Mr. Hurst.”

Her sister made not the smallest objection, and the
piano forte was opened, and Darcy, after a few moments
recollection, was not sorry for it. He began to feel the
danger of paying Elizabeth too much attention.
%%058%%

\Chapter{CHAPTER XII.}

In consequence of an agreement between the sisters,
Elizabeth wrote the next morning to her mother, to beg
that the carriage might be sent for them in the course
of the day. But Mrs. Bennet, who had calculated on her
daughters remaining at Netherfield till the following
Tuesday, which would exactly finish Jane’s week, could
not bring herself to receive them with pleasure before.
Her answer, therefore, was not propitious, at least not to
Elizabeth’s wishes, for she was impatient to get home.
Mrs. Bennet sent them word that they could not possibly
have the carriage before Tuesday; and in her postscript
it was added, that if Mr. Bingley and his sister pressed them
to stay longer, she could spare them very well. -- Against
staying longer, however, Elizabeth was positively resolved -- nor
did she much expect it would be asked; and fearful,
on the contrary, as being considered as intruding themselves
needlessly long, she urged Jane to borrow Mr.
Bingley’s carriage immediately, and at length it was
settled that their original design of leaving Netherfield
that morning should be mentioned, and the request made.

The communication excited many professions of concern;
and enough was said of wishing them to stay at least till
the following day to work on Jane; and till the morrow,
their going was deferred. Miss Bingley was then sorry that
she had proposed the delay, for her jealousy and dislike
of one sister much exceeded her affection for the other.

The master of the house heard with real sorrow that
they were to go so soon, and repeatedly tried to persuade
Miss Bennet that it would not be safe for her -- that she
was not enough recovered; but Jane was firm where she
felt herself to be right.

To Mr. Darcy it was welcome intelligence -- Elizabeth
had been at Netherfield long enough. She attracted him
more than he liked -- and Miss Bingley was uncivil to \textit{her},
%%059%%
and more teazing than usual to himself. He wisely
resolved to be particularly careful that no sign of admiration
should \textit{now} escape him, nothing that could elevate
her with the hope of influencing his felicity; sensible that
if such an idea had been suggested, his behaviour during
the last day must have material weight in confirming or
crushing it. Steady to his purpose, he scarcely spoke ten
words to her through the whole of Saturday, and though
they were at one time left by themselves for half an hour,
he adhered most conscientiously to his book, and would
not even look at her.

On Sunday, after morning service, the separation, so
agreeable to almost all, took place. Miss Bingley’s civility
to Elizabeth increased at last very rapidly, as well as her
affection for Jane; and when they parted, after assuring
the latter of the pleasure it would always give her to see her
either at Longbourn or Netherfield, and embracing her most
tenderly, she even shook hands with the former. -- Elizabeth
took leave of the whole party in the liveliest spirits.

They were not welcomed home very cordially by their
mother. Mrs. Bennet wondered at their coming, and
thought them very wrong to give so much trouble, and
was sure Jane would have caught cold again. -- But their
father, though very laconic in his expressions of pleasure,
was really glad to see them; he had felt their importance
in the family circle. The evening conversation, when they
were all assembled, had lost much of its animation, and
almost all its sense, by the absence of Jane and Elizabeth.

They found Mary, as usual, deep in the study of thorough
bass and human nature; and had some new extracts to
admire, and some new observations of thread-bare morality
to listen to. Catherine and Lydia had information for
them of a different sort. Much had been done, and much
had been said in the regiment since the preceding Wednesday;
several of the officers had dined lately with their
uncle, a private had been flogged, and it had actually
been hinted that Colonel Forster was going to be married.
%%060%%

\Chapter{CHAPTER XIII.}

“I hope, my dear,” said Mr. Bennet to his wife, as
they were at breakfast the next morning, “that you have
ordered a good dinner to-day, because I have reason to
expect an addition to our family party.”

“Who do you mean, my dear? I know of nobody that
is coming I am sure, unless Charlotte Lucas should happen
to call in, and I hope \textit{my} dinners are good enough for her.
I do not believe she often sees such at home.”

“The person of whom I speak, is a gentleman and a
stranger.” Mrs. Bennet’s eyes sparkled. -- “A gentleman
and a stranger! It is Mr. Bingley I am sure. Why Jane
-- you never dropt a word of this; you sly thing! Well,
I am sure I shall be extremely glad to see Mr. Bingley. --
But -- good lord! how unlucky! there is not a bit of fish
to be got to-day. Lydia, my love, ring the bell. I must
speak to Hill, this moment.”

“It is \textit{not} Mr. Bingley,” said her husband; “it is
a person whom I never saw in the whole course of my
life.”

This roused a general astonishment; and he had the
pleasure of being eagerly questioned by his wife and five
daughters at once.

After amusing himself some time with their curiosity,
he thus explained. “About a month ago I received this
letter, and about a fortnight ago I answered it, for I
thought it a case of some delicacy, and requiring early
attention. It is from my cousin, Mr. Collins, who, when
I am dead, may turn you all out of this house as soon as
he pleases.”

“Oh! my dear,” cried his wife, “I cannot bear to
hear that mentioned. Pray do not talk of that odious
man. I do think it is the hardest thing in the world, that
your estate should be entailed away from your own
%%061%%
children; and I am sure if I had been you, I should have
tried long ago to do something or other about it.”

Jane and Elizabeth attempted to explain to her the
nature of an entail. They had often attempted it before,
but it was a subject on which Mrs. Bennet was beyond
the reach of reason; and she continued to rail bitterly
against the cruelty of settling an estate away from a family
of five daughters, in favour of a man whom nobody cared
anything about.

“It certainly is a most iniquitous affair,” said Mr.
Bennet, “and nothing can clear Mr. Collins from the guilt
of inheriting Longbourn. But if you will listen to his
letter, you may perhaps be a little softened by his manner
of expressing himself.”

“No, that I am sure I shall not; and I think it was
very impertinent of him to write to you at all, and very
hypocritical. I hate such false friends. Why could not
he keep on quarrelling with you, as his father did before
him?”

“Why, indeed, he does seem to have had some filial
scruples on that head, as you will hear.”

\begin{letter}
\LetterDate{Hunsford, near Westerham, Kent,\\
15th October.}

\textsc{Dear Sir},

The disagreement subsisting between yourself and
my late honoured father, always gave me much uneasiness,
and since I have had the misfortune to lose him, I have
frequently wished to heal the breach; but for some time
I was kept back by my own doubts, fearing lest it might
seem disrespectful to his memory for me to be on good
terms with any one, with whom it had always pleased
him to be at variance. -- “There, Mrs. Bennet.” -- My mind
however is now made up on the subject, for having received
ordination at Easter, I have been so fortunate as to be
distinguished by the patronage of the Right Honourable
Lady Catherine de Bourgh, widow of Sir Lewis de Bourgh,
whose bounty and beneficence has preferred me to the
%%062%%
valuable rectory of this parish, where it shall be my
earnest endeavour to demean myself with grateful respect
towards her Ladyship, and be ever ready to perform those
rites and ceremonies which are instituted by the Church
of England. As a clergyman, moreover, I feel it my duty
to promote and establish the blessing of peace in all families
within the reach of my influence; and on these grounds
I flatter myself that my present overtures of good-will
are highly commendable, and that the circumstance of
my being next in the entail of Longbourn estate, will be
kindly overlooked on your side, and not lead you to reject
the offered olive branch. I cannot be otherwise than
concerned at being the means of injuring your amiable
daughters, and beg leave to apologise for it, as well as to
assure you of my readiness to make them every possible
amends, -- but of this hereafter. If you should have no
objection to receive me into your house, I propose myself
the satisfaction of waiting on you and your family,
Monday, November 18th, by four o’clock, and shall
probably trespass on your hospitality till the Saturday
se’night following, which I can do without any inconvenience,
as Lady Catherine is far from objecting to
my occasional absence on a Sunday, provided that some
other clergyman is engaged to do the duty of the day.
I remain, dear sir, with respectful compliments to your
lady and daughters, your well-wisher and friend,

\LetterSig{William Collins.”}
\end{letter}

“At four o’clock, therefore, we may expect this peace-making
gentleman,” said Mr. Bennet, as he folded up the
letter. “He seems to be a most conscientious and polite
young man, upon my word; and I doubt not will prove
a valuable acquaintance, especially if Lady Catherine
should be so indulgent as to let him come to us again.”

“There is some sense in what he says about the girls
however; and if he is disposed to make them any amends,
I shall not be the person to discourage him.”

“Though it is difficult,” said Jane, “to guess in what
%%063%%
way he can mean to make us the atonement he thinks
our due, the wish is certainly to his credit.”

Elizabeth was chiefly struck with his extraordinary
deference for Lady Catherine, and his kind intention of
christening, marrying, and burying his parishioners whenever
it were required.

“He must be an oddity, I think,” said she. “I cannot
make him out. -- There is something very pompous in his
stile. -- And what can he mean by apologizing for being
next in the entail? -- We cannot suppose he would help
it, if he could. -- Can he be a sensible man, sir?”

“No, my dear; I think not. I have great hopes of
finding him quite the reverse. There is a mixture of
servility and self-importance in his letter, which promises
well. I am impatient to see him.”

“In point of composition,” said Mary, “his letter does
not seem defective. The idea of the olive branch perhaps
is not wholly new, yet I think it is well expressed.”

To Catherine and Lydia, neither the letter nor its
writer were in any degree interesting. It was next to
impossible that their cousin should come in a scarlet coat,
and it was now some weeks since they had received
pleasure from the society of a man in any other colour.
As for their mother, Mr. Collins’s letter had done away
much of her ill-will, and she was preparing to see him
with a degree of composure, which astonished her husband
and daughters.

Mr. Collins was punctual to his time, and was received
with great politeness by the whole family. Mr. Bennet
indeed said little; but the ladies were ready enough to
talk, and Mr. Collins seemed neither in need of encouragement,
nor inclined to be silent himself. He was a tall, heavy
looking young man of five and twenty. His air was grave
and stately, and his manners were very formal. He had
not been long seated before he complimented Mrs. Bennet
on having so fine a family of daughters, said he had heard
much of their beauty, but that, in this instance, fame had
fallen short of the truth; and added, that he did not
%%064%%
doubt her seeing them all in due time well disposed of in
marriage. This gallantry was not much to the taste of
some of his hearers, but Mrs. Bennet, who quarrelled with
no compliments, answered most readily,

“You are very kind, sir, I am sure; and I wish with all
my heart it may prove so; for else they will be destitute
enough. Things are settled so oddly.”

“You allude perhaps to the entail of this estate.”

“Ah! sir, I do indeed. It is a grievous affair to my
poor girls, you must confess. Not that I mean to find
fault with \textit{you}, for such things I know are all chance in
this world. There is no knowing how estates will go when
once they come to be entailed.”

“I am very sensible, madam, of the hardship to my
fair cousins, -- and could say much on the subject, but that
I am cautious of appearing forward and precipitate.
But I can assure the young ladies that I come prepared
to admire them. At present I will not say more, but
perhaps when we are better acquainted------”

He was interrupted by a summons to dinner; and the
girls smiled on each other. They were not the only objects
of Mr. Collins’s admiration. The hall, the dining-room,
and all its furniture were examined and praised; and his
commendation of every thing would have touched Mrs.
Bennet’s heart, but for the mortifying supposition of his
viewing it all as his own future property. The dinner too
in its turn was highly admired; and he begged to know
to which of his fair cousins, the excellence of its cookery
was owing. But here he was set right by Mrs. Bennet,
who assured him with some asperity that they were very
well able to keep a good cook, and that her daughters had
nothing to do in the kitchen. He begged pardon for having
displeased her. In a softened tone she declared herself
not at all offended; but he continued to apologise for
about a quarter of an hour.
%%065%%

\Chapter{CHAPTER XIV.}

During dinner, Mr. Bennet scarcely spoke at all; but
when the servants were withdrawn, he thought it time
to have some conversation with his guest, and therefore
started a subject in which he expected him to shine, by
observing that he seemed very fortunate in his patroness.
Lady Catherine de Bourgh’s attention to his wishes, and
consideration for his comfort, appeared very remarkable.
Mr. Bennet could not have chosen better. Mr. Collins
was eloquent in her praise. The subject elevated him to
more than usual solemnity of manner, and with a most
important aspect he protested that he had never in his
life witnessed such behaviour in a person of rank -- such
affability and condescension, as he had himself experienced
from Lady Catherine. She had been graciously pleased
to approve of both the discourses, which he had already
had the honour of preaching before her. She had also
asked him twice to dine at Rosings, and had sent for him
only the Saturday before, to make up her pool of quadrille
in the evening. Lady Catherine was reckoned proud by
many people he knew, but \textit{he} had never seen any thing
but affability in her. She had always spoken to him as
she would to any other gentleman; she made not the
smallest objection to his joining in the society of the
neighbourhood, nor to his leaving his parish occasionally
for a week or two, to visit his relations. She had even
condescended to advise him to marry as soon as he could,
provided he chose with discretion; and had once paid
him a visit in his humble parsonage; where she had
perfectly approved all the alterations he had been making,
and had even vouchsafed to suggest some herself, -- some
shelves in the closets up stairs.

“That is all very proper and civil, I am sure,” said
Mrs. Bennet, “and I dare say she is a very agreeable
%%066%%
woman. It is a pity that great ladies in general are not
more like her. Does she live near you, sir?”

“The garden in which stands my humble abode, is
separated only by a lane from Rosings Park, her ladyship’s
residence.”

“I think you said she was a widow, sir? has she any
family?”

“She has one only daughter, the heiress of Rosings,
and of very extensive property.”

“Ah!” cried Mrs. Bennet, shaking her head, “then
she is better off than many girls. And what sort of young
lady is she? is she handsome?”

“She is a most charming young lady indeed. Lady
Catherine herself says that in point of true beauty, Miss
De Bourgh is far superior to the handsomest of her sex;
because there is that in her features which marks the
young woman of distinguished birth. She is unfortunately
of a sickly constitution, which has prevented her
making that progress in many accomplishments, which
she could not otherwise have failed of; as I am informed
by the lady who superintended her education, and who
still resides with them. But she is perfectly amiable,
and often condescends to drive by my humble abode in
her little phaeton and ponies.”

“Has she been presented? I do not remember her
name among the ladies at court.”

“Her indifferent state of health unhappily prevents
her being in town; and by that means, as I told Lady
Catherine myself one day, has deprived the British court
of its brightest ornament. Her ladyship seemed pleased
with the idea, and you may imagine that I am happy
on every occasion to offer those little delicate compliments
which are always acceptable to ladies. I have
more than once observed to Lady Catherine, that her
charming daughter seemed born to be a duchess, and that
the most elevated rank, instead of giving her consequence,
would be adorned by her. -- These are the kind of little
things which please her ladyship, and it is a sort of
%%067%%
attention which I conceive myself peculiarly bound to
pay.”

“You judge very properly,” said Mr. Bennet, “and it
is happy for you that you possess the talent of flattering
with delicacy. May I ask whether these pleasing attentions
proceed from the impulse of the moment, or are the
result of previous study?”

“They arise chiefly from what is passing at the time,
and though I sometimes amuse myself with suggesting
and arranging such little elegant compliments as may be
adapted to ordinary occasions, I always wish to give them
as unstudied an air as possible.”

Mr. Bennet’s expectations were fully answered. His
cousin was as absurd as he had hoped, and he listened
to him with the keenest enjoyment, maintaining at the
same time the most resolute composure of countenance,
and except in an occasional glance at Elizabeth, requiring
no partner in his pleasure.

By tea-time however the dose had been enough, and
Mr. Bennet was glad to take his guest into the drawing-room
again, and when tea was over, glad to invite him
to read aloud to the ladies. Mr. Collins readily assented,
and a book was produced; but on beholding it, (for
every thing announced it to be from a circulating library,)
he started back, and begging pardon, protested that he
never read novels. -- Kitty stared at him, and Lydia
exclaimed. -- Other books were produced, and after some
deliberation he chose Fordyce’s Sermons. Lydia gaped
as he opened the volume, and before he had, with very
monotonous solemnity, read three pages, she interrupted
him with,

“Do you know, mama, that my uncle Philips talks
of turning away Richard, and if he does, Colonel Forster
will hire him. My aunt told me so herself on Saturday.
I shall walk to Meryton to-morrow to hear more about
it, and to ask when Mr. Denny comes back from
town.”

Lydia was bid by her two eldest sisters to hold her
%%068%%
tongue; but Mr. Collins, much offended, laid aside his
book, and said,

“I have often observed how little young ladies are
interested by books of a serious stamp, though written
solely for their benefit. It amazes me, I confess; -- for
certainly, there can be nothing so advantageous to them
as instruction. But I will no longer importune my young
cousin.”

Then turning to Mr. Bennet, he offered himself as his
antagonist at backgammon. Mr. Bennet accepted the
challenge, observing that he acted very wisely in leaving
the girls to their own trifling amusements. Mrs. Bennet
and her daughters apologised most civilly for Lydia’s
interruption, and promised that it should not occur again,
if he would resume his book; but Mr. Collins, after
assuring them that he bore his young cousin no ill will,
and should never resent her behaviour as any affront,
seated himself at another table with Mr. Bennet, and
prepared for backgammon.
%%069%%

\Chapter{CHAPTER XV.}

Mr. Collins was not a sensible man, and the deficiency
of nature had been but little assisted by education or
society; the greatest part of his life having been spent
under the guidance of an illiterate and miserly father;
and though he belonged to one of the universities, he had
merely kept the necessary terms, without forming at it
any useful acquaintance. The subjection in which his
father had brought him up, had given him originally great
humility of manner, but it was now a good deal counteracted
by the self-conceit of a weak head, living in retirement,
and the consequential feelings of early and unexpected
prosperity. A fortunate chance had recommended
him to Lady Catherine de Bourgh when the living of
Hunsford was vacant; and the respect which he felt for
her high rank, and his veneration for her as his patroness,
mingling with a very good opinion of himself, of his
authority as a clergyman, and his rights as a rector, made
him altogether a mixture of pride and obsequiousness,
self-importance and humility.

Having now a good house and very sufficient income,
he intended to marry; and in seeking a reconciliation
with the Longbourn family he had a wife in view, as he
meant to chuse one of the daughters, if he found them
as handsome and amiable as they were represented by
common report. This was his plan of amends -- of atonement -- for
inheriting their father’s estate; and he thought
it an excellent one, full of eligibility and suitableness,
and excessively generous and disinterested on his own
part.

His plan did not vary on seeing them. -- Miss Bennet’s
lovely face confirmed his views, and established all his
strict\-est notions of what was due to seniority; and for
the first evening \textit{she} was his settled choice. The next
%%070%%
morning, however, made an alteration; for in a quarter
of an hour’s tête-à-tête with Mrs. Bennet before breakfast,
a conversation beginning with his parsonage-house, and
leading naturally to the avowal of his hopes, that a mistress
for it might be found at Longbourn, produced from her,
amid very complaisant smiles and general encouragement,
a caution against the very Jane he had fixed on. -- “As
to her \textit{younger} daughters she could not take upon her to
say -- she could not positively answer -- but she did not
\textit{know} of any prepossession; -- her \textit{eldest} daughter, she must
just mention -- she felt it incumbent on her to hint, was
likely to be very soon engaged.”

Mr. Collins had only to change from Jane to Elizabeth -- and
it was soon done -- done while Mrs. Bennet was
stirring the fire. Elizabeth, equally next to Jane in birth
and beauty, succeeded her of course.

Mrs. Bennet treasured up the hint, and trusted that she
might soon have two daughters married; and the man
whom she could not bear to speak of the day before, was
now high in her good graces.

Lydia’s intention of walking to Meryton was not
forgotten; every sister except Mary agreed to go with
her; and Mr. Collins was to attend them, at the request
of Mr. Bennet, who was most anxious to get rid of him,
and have his library to himself; for thither Mr. Collins
had followed him after breakfast, and there he would
continue, nominally engaged with one of the largest folios
in the collection, but really talking to Mr. Bennet, with
little cessation, of his house and garden at Hunsford.
Such doings discomposed Mr. Bennet exceedingly. In his
library he had been always sure of leisure and tranquillity;
and though prepared, as he told Elizabeth, to meet with
folly and conceit in every other room in the house, he was
used to be free from them there; his civility, therefore,
was most prompt in inviting Mr. Collins to join his
daughters in their walk; and Mr. Collins, being in fact
much better fitted for a walker than a reader, was extremely
well pleased to close his large book, and go.
%%071%%

In pompous nothings on his side, and civil assents on
that of his cousins, their time passed till they entered
Meryton. The attention of the younger ones was then
no longer to be gained by \textit{him}. Their eyes were immediately
wandering up in the street in quest of the officers,
and nothing less than a very smart bonnet indeed, or
a really new muslin in a shop window, could recal them.

But the attention of every lady was soon caught by
a young man, whom they had never seen before, of most
gentlemanlike appearance, walking with an officer on the
other side of the way. The officer was the very Mr. Denny,
concerning whose return from London Lydia came to
inquire, and he bowed as they passed. All were struck
with the stranger’s air, all wondered who he could be, and
Kitty and Lydia, determined if possible to find out, led
the way across the street, under pretence of wanting
something in an opposite shop, and fortunately had just
gained the pavement when the two gentlemen turning
back had reached the same spot. Mr. Denny addressed
them directly, and entreated permission to introduce his
friend, Mr. Wickham, who had returned with him the day
before from town, and he was happy to say had accepted
a commission in their corps. This was exactly as it should
be; for the young man wanted only regimentals to make
him completely charming. His appearance was greatly
in his favour; he had all the best part of beauty, a fine
countenance, a good figure, and very pleasing address.
The introduction was followed up on his side by a happy
readiness of conversation -- a readiness at the same time
perfectly correct and unassuming; and the whole party
were still standing and talking together very agreeably,
when the sound of horses drew their notice, and Darcy
and Bingley were seen riding down the street. On distinguishing
the ladies of the group, the two gentlemen
came directly towards them, and began the usual civilities.
Bingley was the principal spokesman, and Miss Bennet the
principal object. He was then, he said, on his way to
Longbourn on purpose to inquire after her. Mr. Darcy
%%072%%
corroborated it with a bow, and was beginning to determine
not to fix his eyes on Elizabeth, when they were
suddenly arrested by the sight of the stranger, and
Elizabeth happening to see the countenance of both as
they looked at each other, was all astonishment at the
effect of the meeting. Both changed colour, one looked
white, the other red. Mr. Wickham, after a few moments,
touched his hat -- a salutation which Mr. Darcy just
deigned to return. What could be the meaning of it? -- It
was impossible to imagine; it was impossible not to
long to know.

In another minute Mr. Bingley, but without seeming
to have noticed what passed, took leave and rode on with
his friend.

Mr. Denny and Mr. Wickham walked with the young
ladies to the door of Mr. Philips’s house, and then made
their bows, in spite of Miss Lydia’s pressing entreaties
that they would come in, and even in spite of Mrs. Philips’
throwing up the parlour window, and loudly seconding the
invitation.

Mrs. Philips was always glad to see her nieces, and the
two eldest, from their recent absence, were particularly
welcome, and she was eagerly expressing her surprise at
their sudden return home, which, as their own carriage
had not fetched them, she should have known nothing
about, if she had not happened to see Mr. Jones’s shop boy
in the street, who had told her that they were not to send
any more draughts to Netherfield because the Miss
Bennets were come away, when her civility was claimed
towards Mr. Collins by Jane’s introduction of him. She
received him with her very best politeness, which he
returned with as much more, apologising for his intrusion,
without any previous acquaintance with her, which he
could not help flattering himself however might be justified
by his relationship to the young ladies who introduced
him to her notice. Mrs. Philips was quite awed by such
an excess of good breeding; but her contemplation of
one stranger was soon put an end to by exclamations and
%%073%%
inquiries about the other, of whom, however, she could
only tell her nieces what they already knew, that Mr.
Denny had brought him from London, and that he was
to have a lieutenant’s commission in the ------shire. She
had been watching him the last hour, she said, as he walked
up and down the street, and had Mr. Wickham appeared
Kitty and Lydia would certainly have continued the
occupation, but unluckily no one passed the windows
now except a few of the officers, who in comparison with
the stranger, were become “stupid, disagreeable fellows.”
Some of them were to dine with the Philipses the next
day, and their aunt promised to make her husband call
on Mr. Wickham, and give him an invitation also, if the
family from Longbourn would come in the evening. This
was agreed to, and Mrs. Philips protested that they would
have a nice comfortable noisy game of lottery tickets, and
a little bit of hot supper afterwards. The prospect of
such delights was very cheering, and they parted in mutual
good spirits. Mr. Collins repeated his apologies in quitting
the room, and was assured with unwearying civility that
they were perfectly needless.

As they walked home, Elizabeth related to Jane what
she had seen pass between the two gentlemen; but though
Jane would have defended either or both, had they
appeared to be wrong, she could no more explain such
behaviour than her sister.

Mr. Collins on his return highly gratified Mrs. Bennet
by admiring Mrs. Philips’s manners and politeness. He
protested that except Lady Catherine and her daughter,
he had never seen a more elegant woman; for she had
not only received him with the utmost civility, but had
even pointedly included him in her invitation for the next
evening, although utterly unknown to her before. Something
he supposed might be attributed to his connection
with them, but yet he had never met with so much
attention in the whole course of his life.
%%074%%

\Chapter{CHAPTER XVI.}

As no objection was made to the young people’s engagement
with their aunt, and all Mr. Collins’s scruples of
leaving Mr. and Mrs. Bennet for a single evening during
his visit were most steadily resisted, the coach conveyed
him and his five cousins at a suitable hour to Meryton;
and the girls had the pleasure of hearing, as they entered
the drawing-room, that Mr. Wickham had accepted their
uncle’s invitation, and was then in the house.

When this information was given, and they had all
taken their seats, Mr. Collins was at leisure to look around
him and admire, and he was so much struck with the size
and furniture of the apartment, that he declared he might
almost have supposed himself in the small summer
breakfast parlour at Rosings; a comparison that did not
at first convey much gratification; but when Mrs. Philips
understood from him what Rosings was, and who was its
proprietor, when she had listened to the description of
only one of Lady Catherine’s drawing-rooms, and found
that the chimney-piece alone had cost eight hundred
pounds, she felt all the force of the compliment, and would
hardly have resented a comparison with the housekeeper’s
room.

In describing to her all the grandeur of Lady Catherine
and her mansion, with occasional digressions in praise of
his own humble abode, and the improvements it was
receiving, he was happily employed until the gentlemen
joined them; and he found in Mrs. Philips a very attentive
listener, whose opinion of his consequence increased with
what she heard, and who was resolving to retail it all
among her neighbours as soon as she could. To the girls,
who could not listen to their cousin, and who had nothing
to do but to wish for an instrument, and examine their
own indifferent imitations of china on the mantlepiece, the
%%075%%
interval of waiting appeared very long. It was over at
last however. The gentlemen did approach; and when
Mr. Wickham walked into the room, Elizabeth felt that
she had neither been seeing him before, nor thinking of him
since, with the smallest degree of unreasonable admiration.
The officers of the ------shire were in general a very creditable,
gentlemanlike set, and the best of them were of the
present party; but Mr. Wickham was as far beyond them
all in person, countenance, air, and walk, as \textit{they} were
superior to the broad-faced stuffy uncle Philips, breathing
port wine, who followed them into the room.

Mr. Wickham was the happy man towards whom almost
every female eye was turned, and Elizabeth was the
happy woman by whom he finally seated himself; and
the agreeable manner in which he immediately fell into
conversation, though it was only on its being a wet night,
and on the probability of a rainy season, made her feel
that the commonest, dullest, most threadbare topic might
be rendered interesting by the skill of the speaker.

With such rivals for the notice of the fair, as Mr. Wickham
and the officers, Mr. Collins seemed likely to sink
into insignificance; to the young ladies he certainly was
nothing; but he had still at intervals a kind listener in
Mrs. Philips, and was, by her watchfulness, most abundantly
supplied with coffee and muffin.

When the card tables were placed, he had an opportunity
of obliging her in return, by sitting down to whist.

“I know little of the game, at present,” said he, “but
I shall be glad to improve myself, for in my situation of
life------” Mrs. Philips was very thankful for his compliance,
but could not wait for his reason.

Mr. Wickham did not play at whist, and with ready
delight was he received at the other table between Elizabeth
and Lydia. At first there seemed danger of Lydia’s
engrossing him entirely, for she was a most determined
talker; but being likewise extremely fond of lottery
tickets, she soon grew too much interested in the game,
too eager in making bets and exclaiming after prizes, to
%%076%%
have attention for any one in particular. Allowing for
the common demands of the game, Mr. Wickham was
therefore at leisure to talk to Elizabeth, and she was very
willing to hear him, though what she chiefly wished to
hear she could not hope to be told, the history of his
acquaintance with Mr. Darcy. She dared not even
mention that gentleman. Her curiosity however was
unexpectedly relieved. Mr. Wickham began the subject
himself. He inquired how far Netherfield was from
Meryton; and, after receiving her answer, asked in an
hesitating manner how long Mr. Darcy had been staying
there.

“About a month,” said Elizabeth; and then, unwilling
to let the subject drop, added, “He is a man of very large
property in Derbyshire, I understand.”

“Yes,” replied Wickham; -- “his estate there is a noble
one. A clear ten thousand per annum. You could not
have met with a person more capable of giving you certain
information on that head than myself -- for I have been
connected with his family in a particular manner from my
infancy.”

Elizabeth could not but look surprised.

“You may well be surprised, Miss Bennet, at such an
assertion, after seeing, as you probably might, the very
cold manner of our meeting yesterday. -- Are you much
acquainted with Mr. Darcy?”

“As much as I ever wish to be,” cried Elizabeth
warmly, -- “I have spent four days in the same house
with him, and I think him very disagreeable.”

“I have no right to give \textit{my} opinion,” said Wickham,
“as to his being agreeable or otherwise. I am not qualified
to form one. I have known him too long and too well to
be a fair judge. It is impossible for \textit{me} to be impartial.
But I believe your opinion of him would in general
astonish -- and perhaps you would not express it quite so
strongly anywhere else. -- Here you are in your own family.”

“Upon my word I say no more \textit{here} than I might say
in any house in the neighbourhood, except Netherfield.
%%077%%
He is not at all liked in Hertfordshire. Every body is
disgusted with his pride. You will not find him more
favourably spoken of by any one.”

“I cannot pretend to be sorry,” said Wickham, after
a short interruption, “that he or that any man should
not be estimated beyond their deserts; but with \textit{him}
I believe it does not often happen. The world is blinded
by his fortune and consequence, or frightened by his high
and imposing manners, and sees him only as he chuses to
be seen.”

“I should take him, even on \textit{my} slight acquaintance,
to be an ill-tempered man.” Wickham only shook his
head.

“I wonder,” said he, at the next opportunity of
speaking, “whether he is likely to be in this country much
longer.”

“I do not at all know; but I \textit{heard} nothing of his going
away when I was at Netherfield. I hope your plans in
favour of the ------shire will not be affected by his being
in the neighbourhood.”

“Oh! no -- it is not for \textit{me} to be driven away by
Mr. Darcy. If \textit{he} wishes to avoid seeing \textit{me}, he must go.
We are not on friendly terms, and it always gives me pain
to meet him, but I have no reason for avoiding \textit{him}
but what I might proclaim to all the world; a sense of
very great ill usage, and most painful regrets at his being
what he is. His father, Miss Bennet, the late Mr. Darcy,
was one of the best men that ever breathed, and the truest
friend I ever had; and I can never be in company with
this Mr. Darcy without being grieved to the soul by a
thousand tender recollections. His behaviour to myself
has been scandalous; but I verily believe I could forgive
him any thing and every thing, rather than his disappointing
the hopes and disgracing the memory of his father.”

Elizabeth found the interest of the subject increase, and
listened with all her heart; but the delicacy of it prevented
farther inquiry.

Mr. Wickham began to speak on more general topics,
%%078%%
Meryton, the neighbourhood, the society, appearing highly
pleased with all that he had yet seen, and speaking of the
latter especially, with gentle but very intelligible gallantry.

“It was the prospect of constant society, and good
society,” he added, “which was my chief inducement to
enter the ------shire. I knew it to be a most respectable,
agreeable corps, and my friend Denny tempted me farther
by his account of their present quarters, and the very great
attentions and excellent acquaintance Meryton had procured
them. Society, I own, is necessary to me. I have
been a disappointed man, and my spirits will not bear
solitude. I \textit{must} have employment and society. A
military life is not what I was intended for, but circumstances
have now made it eligible. The church \textit{ought} to
have been my profession -- I was brought up for the church,
and I should at this time have been in possession of a most
valuable living, had it pleased the gentleman we were
speaking of just now.”

“Indeed!”

“Yes -- the late Mr. Darcy bequeathed me the next
presentation of the best living in his gift. He was my
godfather, and excessively attached to me. I cannot do
justice to his kindness. He meant to provide for me
amply, and thought he had done it; but when the living
fell, it was given elsewhere.”

“Good heavens!” cried Elizabeth; “but how could
\textit{that} be? -- How could his will be disregarded? -- Why did
not you seek legal redress?”

“There was just such an informality in the terms of
the bequest as to give me no hope from law. A man of
honour could not have doubted the intention, but Mr.
Darcy chose to doubt it -- or to treat it as a merely conditional
recommendation, and to assert that I had forfeited
all claim to it by extravagance, imprudence, in short
any thing or nothing. Certain it is, that the living became
vacant two years ago, exactly as I was of an age to hold
it, and that it was given to another man; and no less
certain is it, that I cannot accuse myself of having really
%%079%%
done any thing to deserve to lose it. I have a warm,
unguarded temper, and I may perhaps have sometimes
spoken my opinion \textit{of} him, and \textit{to} him, too freely. I can
recal nothing worse. But the fact is, that we are very
different sort of men, and that he hates me.”

“This is quite shocking! -- He deserves to be publicly
disgraced.”

“Some time or other he \textit{will} be -- but it shall not be
by \textit{me}. Till I can forget his father, I can never defy or
expose \textit{him}.”

Elizabeth honoured him for such feelings, and thought
him handsomer than ever as he expressed them.

“But what,” said she, after a pause, “can have been
his motive? -- what can have induced him to behave so
cruelly?”

“A thorough, determined dislike of me -- a dislike which
I cannot but attribute in some measure to jealousy. Had
the late Mr. Darcy liked me less, his son might have borne
with me better; but his father’s uncommon attachment
to me, irritated him I believe very early in life. He had
not a temper to bear the sort of competition in which we
stood -- the sort of preference which was often given me.”

“I had not thought Mr. Darcy so bad as this -- though
I have never liked him, I had not thought so very ill of
him -- I had supposed him to be despising his fellow-creatures
in general, but did not suspect him of descending
to such malicious revenge, such injustice, such inhumanity
as this!”

After a few minutes reflection, however, she continued,
“I \textit{do} remember his boasting one day, at Netherfield, of
the implacability of his resentments, of his having an
unforgiving temper. His disposition must be dreadful.”

“I will not trust myself on the subject,” replied Wickham,
“\textit{I} can hardly be just to him.”

Elizabeth was again deep in thought, and after a time
exclaimed, “To treat in such a manner, the godson, the
friend, the favourite of his father!” -- She could have
added, “A young man too, like \textit{you}, whose very
%%080%%
countenance may vouch for your being amiable” -- but she
contented herself with “And one, too, who had probably
been his own companion from childhood, connected
together, as I think you said, in the closest manner!”

“We were born in the same parish, within the same
park, the greatest part of our youth was passed together;
inmates of the same house, sharing the same amusements,
objects of the same parental care. \textit{My} father began life
in the profession which your uncle, Mr. Philips, appears
to do so much credit to -- but he gave up every thing to
be of use to the late Mr. Darcy, and devoted all his time
to the care of the Pemberley property. He was most
highly esteemed by Mr. Darcy, a most intimate, confidential
friend. Mr. Darcy often acknowledged himself to
be under the greatest obligations to my father’s active
superintendance, and when immediately before my father’s
death, Mr. Darcy gave him a voluntary promise of providing
for me, I am convinced that he felt it to be as much
a debt of gratitude to \textit{him}, as of affection to myself.”

“How strange!” cried Elizabeth. “How abominable! -- I
wonder that the very pride of this Mr. Darcy
has not made him just to you! -- If from no better motive,
that he should not have been too proud to be dishonest, -- for
dishonesty I must call it.”

“It \textit{is} wonderful,” -- replied Wickham, -- “for almost
all his actions may be traced to pride; -- and pride has
often been his best friend. It has connected him nearer
with virtue than any other feeling. But we are none of
us consistent; and in his behaviour to me, there were
stronger impulses even than pride.”

“Can such abominable pride as his, have ever done
him good?”

“Yes. It has often led him to be liberal and generous, -- to
give his money freely, to display hospitality, to assist
his tenants, and relieve the poor. Family pride, and \textit{filial}
pride, for he is very proud of what his father was, have
done this. Not to appear to disgrace his family, to degenerate
from the popular qualities, or lose the influence of the
%%081%%
Pemberley House, is a powerful motive. He has also
\textit{brotherly} pride, which with \textit{some} brotherly affection, makes
him a very kind and careful guardian of his sister; and you
will hear him generally cried up as the most attentive and
best of brothers.”

“What sort of a girl is Miss Darcy?”

He shook his head. -- “I wish I could call her amiable.
It gives me pain to speak ill of a Darcy. But she is too
much like her brother, -- very, very proud. -- As a child,
she was affectionate and pleasing, and extremely fond of
me; and I have devoted hours and hours to her amusement.
But she is nothing to me now. She is a handsome
girl, about fifteen or sixteen, and I understand highly
accomplished. Since her father’s death, her home has
been London, where a lady lives with her, and superintends
her education.”

After many pauses and many trials of other subjects,
Elizabeth could not help reverting once more to the first,
and saying,

“I am astonished at his intimacy with Mr. Bingley!
How can Mr. Bingley, who seems good humour itself, and
is, I really believe, truly amiable, be in friendship with
such a man? How can they suit each other? -- Do you
know Mr. Bingley?”

“Not at all.”

“He is a sweet tempered, amiable, charming man.
He cannot know what Mr. Darcy is.”

“Probably not; -- but Mr. Darcy can please where he
chuses. He does not want abilities. He can be a conversible
companion if he thinks it worth his while. Among
those who are at all his equals in consequence, he is a very
different man from what he is to the less prosperous.
His pride never deserts him; but with the rich, he is
liberal-minded, just, sincere, rational, honourable, and
perhaps agreeable, -- allowing something for fortune and
figure.”

The whist party soon afterwards breaking up, the players
gathered round the other table, and Mr. Collins took his
%%082%%
station between his cousin Elizabeth and Mrs. Philips. -- The
usual inquiries as to his success were made by the latter.
It had not been very great; he had lost every point;
but when Mrs. Philips began to express her concern thereupon,
he assured her with much earnest gravity that it
was not of the least importance, that he considered the
money as a mere trifle, and begged she would not make
herself uneasy.

“I know very well, madam,” said he, “that when
persons sit down to a card table, they must take their
chance of these things, -- and happily I am not in such
circumstances as to make five shillings any object. There
are undoubtedly many who could not say the same, but
thanks to Lady Catherine de Bourgh, I am removed far
beyond the necessity of regarding little matters.”

Mr. Wickham’s attention was caught; and after observing
Mr. Collins for a few moments, he asked Elizabeth in
a low voice whether her relation were very intimately
acquainted with the family of de Bourgh.

“Lady Catherine de Bourgh,” she replied, “has very
lately given him a living. I hardly know how Mr. Collins
was first introduced to her notice, but he certainly has
not known her long.”

“You know of course that Lady Catherine de Bourgh
and Lady Anne Darcy were sisters; consequently that
she is aunt to the present Mr. Darcy.”

“No, indeed, I did not. -- I knew nothing at all of Lady
Catherine’s connections. I never heard of her existence
till the day before yesterday.”

“Her daughter, Miss de Bourgh, will have a very large
fortune, and it is believed that she and her cousin will
unite the two estates.”

This information made Elizabeth smile, as she thought
of poor Miss Bingley. Vain indeed must be all her attentions,
vain and useless her affection for his sister and her
praise of himself, if he were already self-destined to
another.

“Mr. Collins,” said she, “speaks highly both of Lady
%%083%%
Catherine and her daughter; but from some particulars
that he has related of her ladyship, I suspect his gratitude
misleads him, and that in spite of her being his patroness,
she is an arrogant, conceited woman.”

“I believe her to be both in a great degree,” replied
Wickham; “I have not seen her for many years, but
I very well remember that I never liked her, and that
her manners were dictatorial and insolent. She has the
reputation of being remarkably sensible and clever; but
I rather believe she derives part of her abilities from her
rank and fortune, part from her authoritative manner,
and the rest from the pride of her nephew, who chuses
that every one connected with him should have an understanding
of the first class.”

Elizabeth allowed that he had given a very rational
account of it, and they continued talking together with
mutual satisfaction till supper put an end to cards; and
gave the rest of the ladies their share of Mr. Wickham’s
attentions. There could be no conversation in the noise
of Mrs. Philips’s supper party, but his manners recommended
him to every body. Whatever he said, was said
well; and whatever he did, done gracefully. Elizabeth
went away with her head full of him. She could think of
nothing but of Mr. Wickham, and of what he had told her,
all the way home; but there was not time for her even
to mention his name as they went, for neither Lydia nor
Mr. Collins were once silent. Lydia talked incessantly of
lottery tickets, of the fish she had lost and the fish she had
won, and Mr. Collins, in describing the civility of Mr. and
Mrs. Philips, protesting that he did not in the least regard
his losses at whist, enumerating all the dishes at supper,
and repeatedly fearing that he crouded his cousins, had
more to say than he could well manage before the carriage
stopped at Longbourn House.
%%084%%

\Chapter{CHAPTER XVII.}

Elizabeth related to Jane the next day, what had
passed between Mr. Wickham and herself. Jane listened
with astonishment and concern; -- she knew not how to
believe that Mr. Darcy could be so unworthy of Mr.
Bingley’s regard; and yet, it was not in her nature to
question the veracity of a young man of such amiable
appearance as Wickham. -- The possibility of his having
really endured such unkindness, was enough to interest
all her tender feelings; and nothing therefore remained
to be done, but to think well of them both, to defend the
conduct of each, and throw into the account of accident
or mistake, whatever could not be otherwise explained.

“They have both,” said she, “been deceived, I dare
say, in some way or other, of which we can form no idea.
Interested people have perhaps misrepresented each to
the other. It is, in short, impossible for us to conjecture
the causes or circumstances which may have alienated
them, without actual blame on either side.”

“Very true, indeed; -- and now, my dear Jane, what
have you got to say in behalf of the interested people
who have probably been concerned in the business? -- Do
clear \textit{them} too, or we shall be obliged to think ill of
somebody.”

“Laugh as much as you chuse, but you will not laugh
me out of my opinion. My dearest Lizzy, do but consider
in what a disgraceful light it places Mr. Darcy, to be
treating his father’s favourite in such a manner, -- one,
whom his father had promised to provide for. -- It is impossible.
No man of common humanity, no man who had
any value for his character, could be capable of it. Can
his most intimate friends be so excessively deceived in
him? oh! no.”

“I can much more easily believe Mr. Bingley’s being
%%085%%
imposed on, than that Mr. Wickham should invent such
a history of himself as he gave me last night; names,
facts, every thing mentioned without ceremony. -- If it be
not so, let Mr. Darcy contradict it. Besides, there was
truth in his looks.”

“It is difficult indeed -- it is distressing. -- One does not
know what to think.”

“I beg your pardon; -- one knows exactly what to
think.”

But Jane could think with certainty on only one
point, -- that Mr. Bingley, if he \textit{had been} imposed on,
would have much to suffer when the affair became public.

The two young ladies were summoned from the shrubbery
where this conversation passed, by the arrival of
some of the very persons of whom they had been speaking;
Mr. Bingley and his sisters came to give their personal
invitation for the long expected ball at Netherfield, which
was fixed for the following Tuesday. The two ladies were
delighted to see their dear friend again, called it an age
since they had met, and repeatedly asked what she had
been doing with herself since their separation. To the rest
of the family they paid little attention; avoiding Mrs.
Bennet as much as possible, saying not much to Elizabeth,
and nothing at all to the others. They were soon gone
again, rising from their seats with an activity which took
their brother by surprise, and hurrying off as if eager
to escape from Mrs. Bennet’s civilities.

The prospect of the Netherfield ball was extremely
agreeable to every female of the family. Mrs. Bennet
chose to consider it as given in compliment to her eldest
daughter, and was particularly flattered by receiving the
invitation from Mr. Bingley himself, instead of a ceremonious
card. Jane pictured to herself a happy evening
in the society of her two friends, and the attentions of their
brother; and Elizabeth thought with pleasure of dancing
a great deal with Mr. Wickham, and of seeing a confirmation
of every thing in Mr. Darcy’s looks and behaviour.
The happiness anticipated by Catherine and Lydia,
%%086%%
depended less on any single event, or any particular
person, for though they each, like Elizabeth, meant to
dance half the evening with Mr. Wickham, he was by no
means the only partner who could satisfy them, and a ball
was at any rate, a ball. And even Mary could assure her
family that she had no disinclination for it.

“While I can have my mornings to myself,” said she,
“it is enough. -- I think it no sacrifice to join occasionally
in evening engagements. Society has claims on us all;
and I profess myself one of those who consider intervals
of recreation and amusement as desirable for every body.”

Elizabeth’s spirits were so high on the occasion, that
though she did not often speak unnecessarily to Mr. Collins,
she could not help asking him whether he intended to
accept Mr. Bingley’s invitation, and if he did, whether
he would think it proper to join in the evening’s amusement;
and she was rather surprised to find that he entertained
no scruple whatever on that head, and was very
far from dreading a rebuke either from the Archbishop,
or Lady Catherine de Bourgh, by venturing to dance.

“I am by no means of opinion, I assure you,” said he,
“that a ball of this kind, given by a young man of character,
to respectable people, can have any evil tendency;
and I am so far from objecting to dancing myself that
I shall hope to be honoured with the hands of all my fair
cousins in the course of the evening, and I take this opportunity
of soliciting yours, Miss Elizabeth, for the two first
dances especially, -- a preference which I trust my cousin
Jane will attribute to the right cause, and not to any
disrespect for her.”

Elizabeth felt herself completely taken in. She had
fully proposed being engaged by Wickham for those
very dances:-- and to have Mr. Collins instead! her liveliness
had been never worse timed. There was no help for
it however. Mr. Wickham’s happiness and her own was
per force delayed a little longer, and Mr. Collins’s proposal
accepted with as good a grace as she could. She was not
the better pleased with his gallantry, from the idea it
%%087%%
suggested of something more. -- It now first struck her,
that \textit{she} was selected from among her sisters as worthy
of being the mistress of Hunsford Parsonage, and of
assisting to form a quadrille table at Rosings, in the
absence of more eligible visitors. The idea soon reached
to conviction, as she observed his increasing civilities
toward herself, and heard his frequent attempt at a compliment
on her wit and vivacity; and though more astonished
than gratified herself, by this effect of her charms,
it was not long before her mother gave her to understand
that the probability of their marriage was exceedingly
agreeable to \textit{her}. Elizabeth however did not chuse to take
the hint, being well aware that a serious dispute must be
the consequence of any reply. Mr. Collins might never
make the offer, and till he did, it was useless to quarrel
about him.

If there had not been a Netherfield ball to prepare for
and talk of, the younger Miss Bennets would have been
in a pitiable state at this time, for from the day of the
invitation, to the day of the ball, there was such a succession
of rain as prevented their walking to Meryton once.
No aunt, no officers, no news could be sought after; -- the
very shoe-roses for Netherfield were got by proxy. Even
Elizabeth might have found some trial of her patience
in weather, which totally suspended the improvement of
her acquaintance with Mr. Wickham; and nothing less
than a dance on Tuesday, could have made such a Friday,
Saturday, Sunday and Monday, endurable to Kitty and
Lydia.
%%088%%

\Chapter{CHAPTER XVIII.}

Till Elizabeth entered the drawing-room at Netherfield
and looked in vain for Mr. Wickham among the cluster
of red coats there assembled, a doubt of his being present
had never occurred to her. The certainty of meeting him
had not been checked by any of those recollections that
might not unreasonably have alarmed her. She had
dressed with more than usual care, and prepared in the
highest spirits for the conquest of all that remained
unsubdued of his heart, trusting that it was not more
than might be won in the course of the evening. But
in an instant arose the dreadful suspicion of his being
purposely omitted for Mr. Darcy’s pleasure in the Bingleys’
invitation to the officers; and though this was not exactly
the case, the absolute fact of his absence was pronounced
by his friend Mr. Denny, to whom Lydia eagerly applied,
and who told them that Wickham had been obliged to go
to town on business the day before, and was not yet
returned; adding, with a significant smile,

“I do not imagine his business would have called him
away just now, if he had not wished to avoid a certain
gentleman here.”

This part of his intelligence, though unheard by Lydia,
was caught by Elizabeth, and as it assured her that Darcy
was not less answerable for Wickham’s absence than if her
first surmise had been just, every feeling of displeasure
against the former was so sharpened by immediate disappointment,
that she could hardly reply with tolerable
civility to the polite inquiries which he directly afterwards
approached to make. -- Attention, forbearance, patience
with Darcy, was injury to Wickham. She was resolved
against any sort of conversation with him, and turned
away with a degree of ill humour, which she could not
%%089%%
wholly surmount even in speaking to Mr. Bingley, whose
blind partiality provoked her.

But Elizabeth was not formed for ill-humour; and
though every prospect of her own was destroyed for the
evening, it could not dwell long on her spirits; and
having told all her griefs to Charlotte Lucas, whom she
had not seen for a week, she was soon able to make a
voluntary transition to the oddities of her cousin, and
to point him out to her particular notice. The two first
dances, however, brought a return of distress; they were
dances of mortification. Mr. Collins, awkward and solemn,
apologising instead of attending, and often moving wrong
without being aware of it, gave her all the shame and
misery which a disagreeable partner for a couple of dances
can give. The moment of her release from him was exstacy.

She danced next with an officer, and had the refreshment
of talking of Wickham, and of hearing that he was
universally liked. When those dances were over she
returned to Charlotte Lucas, and was in conversation
with her, when she found herself suddenly addressed by
Mr. Darcy, who took her so much by surprise in his
application for her hand, that, without knowing what she
did, she accepted him. He walked away again immediately,
and she was left to fret over her own want of presence
of mind; Charlotte tried to console her.

“I dare say you will find him very agreeable.”

“Heaven forbid! -- \textit{That} would be the greatest misfortune
of all! -- To find a man agreeable whom one is
determined to hate! -- Do not wish me such an evil.”

When the dancing recommenced, however, and Darcy
approached to claim her hand, Charlotte could not help
cautioning her in a whisper not to be a simpleton and
allow her fancy for Wickham to make her appear unpleasant
in the eyes of a man of ten times his consequence.
Elizabeth made no answer, and took her place in the set,
amazed at the dignity to which she was arrived in being
allowed to stand opposite to Mr. Darcy, and reading in her
neighbours’ looks their equal amazement in beholding it.
%%090%%
They stood for some time without speaking a word; and
she began to imagine that their silence was to last through
the two dances, and at first was resolved not to break it;
till suddenly fancying that it would be the greater punishment
to her partner to oblige him to talk, she made some
slight observation on the dance. He replied, and was
again silent. After a pause of some minutes she addressed
him a second time with

“It is \textit{your} turn to say something now, Mr. Darcy. -- \textit{I}
talked about the dance, and \textit{you} ought to make some
kind of remark on the size of the room, or the number
of couples.”

He smiled, and assured her that whatever she wished
him to say should be said.

“Very well. -- That reply will do for the present. -- Perhaps
by and bye I may observe that private balls are
much pleasanter than public ones. -- But \textit{now} we may be
silent.”

“Do you talk by rule then, while you are dancing?”

“Sometimes. One must speak a little, you know. It
would look odd to be entirely silent for half an hour
together, and yet for the advantage of \textit{some}, conversation
ought to be so arranged as that they may have the trouble
of saying as little as possible.”

“Are you consulting your own feelings in the present
case, or do you imagine that you are gratifying mine?”

“Both,” replied Elizabeth archly; “for I have always
seen a great similarity in the turn of our minds. -- We are
each of an unsocial, taciturn disposition, unwilling to
speak, unless we expect to say something that will amaze
the whole room, and be handed down to posterity with
all the eclat of a proverb.”

“This is no very striking resemblance of your own
character, I am sure,” said he. “How near it may be
to \textit{mine}, I cannot pretend to say. -- \textit{You} think it a faithful
portrait undoubtedly.”

“I must not decide on my own performance.”

He made no answer, and they were again silent till they
%%091%%
had gone down the dance, when he asked her if she and
her sisters did not very often walk to Meryton. She
answered in the affirmative, and, unable to resist the
temptation, added, “When you met us there the other
day, we had just been forming a new acquaintance.”

The effect was immediate. A deeper shade of hauteur
overspread his features, but he said not a word, and
Elizabeth, though blaming herself for her own weakness,
could not go on. At length Darcy spoke, and in a constrained
manner said,

“Mr. Wickham is blessed with such happy manners as
may ensure his \textit{making} friends -- whether he may be equally
capable of \textit{retaining} them, is less certain.”

“He has been so unlucky as to lose \textit{your} friendship,”
replied Elizabeth with emphasis, “and in a manner which
he is likely to suffer from all his life.”

Darcy made no answer, and seemed desirous of changing
the subject. At that moment Sir William Lucas appeared
close to them, meaning to pass through the set to the other
side of the room; but on perceiving Mr. Darcy he stopt
with a bow of superior courtesy to compliment him on
his dancing and his partner.

“I have been most highly gratified indeed, my dear Sir.
Such very superior dancing is not often seen. It is evident
that you belong to the first circles. Allow me to say,
however, that your fair partner does not disgrace you,
and that I must hope to have this pleasure often repeated,
especially when a certain desirable event, my dear Miss
Eliza, (glancing at her sister and Bingley,) shall take place.
What congratulations will then flow in! I appeal to
Mr. Darcy:-- but let me not interrupt you, Sir. -- You will
not thank me for detaining you from the bewitching
converse of that young lady, whose bright eyes are also
upbraiding me.”

The latter part of this address was scarcely heard by
Darcy; but Sir William’s allusion to his friend seemed to
strike him forcibly, and his eyes were directed with a very
serious expression towards Bingley and Jane, who were
%%092%%
dancing together. Recovering himself, however, shortly,
he turned to his partner, and said,

“Sir William’s interruption has made me forget what
we were talking of.”

“I do not think we were speaking at all. Sir William
could not have interrupted any two people in the room
who had less to say for themselves. -- We have tried two
or three subjects already without success, and what we
are to talk of next I cannot imagine.”

“What think you of books?” said he, smiling.

“Books -- Oh! no. -- I am sure we never read the same,
or not with the same feelings.”

“I am sorry you think so; but if that be the case,
there can at least be no want of subject. -- We may
compare our different opinions.”

“No -- I cannot talk of books in a ball-room; my head
is always full of something else.”

“The \textit{present} always occupies you in such scenes -- does
it?” said he, with a look of doubt.

“Yes, always,” she replied, without knowing what she
said, for her thoughts had wandered far from the subject,
as soon afterwards appeared by her suddenly exclaiming,
“I remember hearing you once say, Mr. Darcy, that you
hardly ever forgave, that your resentment once created
was unappeasable. You are very cautious, I suppose,
as to its \textit{being created}.”

“I am,” said he, with a firm voice.

“And never allow yourself to be blinded by prejudice?”

“I hope not.”

“It is particularly incumbent on those who never change
their opinion, to be secure of judging properly at first.”

“May I ask to what these questions tend?”

“Merely to the illustration of \textit{your} character,” said she,
endeavouring to shake off her gravity. “I am trying to
make it out.”

“And what is your success?”

She shook her head. “I do not get on at all. I hear
such different accounts of you as puzzle me exceedingly.”
%%093%%

“I can readily believe,” answered he gravely, “that
report may vary greatly with respect to me; and I could
wish, Miss Bennet, that you were not to sketch my
character at the present moment, as there is reason to
fear that the performance would reflect no credit on
either.”

“But if I do not take your likeness now, I may never
have another opportunity.”

“I would by no means suspend any pleasure of yours,”
he coldly replied. She said no more, and they went down
the other dance and parted in silence; on each side
dissatisfied, though not to an equal degree, for in Darcy’s
breast there was a tolerable powerful feeling towards her,
which soon procured her pardon, and directed all his anger
against another.

They had not long separated when Miss Bingley came
towards her, and with an expression of civil disdain thus
accosted her,

“So, Miss Eliza, I hear you are quite delighted with
George Wickham! -- Your sister has been talking to me
about him, and asking me a thousand questions; and
I find that the young man forgot to tell you, among his
other communications, that he was the son of old Wickham,
the late Mr. Darcy’s steward. Let me recommend you,
however, as a friend, not to give implicit confidence to all
his assertions; for as to Mr. Darcy’s using him ill, it is
perfectly false; for, on the contrary, he has been always
remarkably kind to him, though George Wickham has
treated Mr. Darcy in a most infamous manner. I do not
know the particulars, but I know very well that Mr. Darcy
is not in the least to blame, that he cannot bear to hear
George Wickham mentioned, and that though my brother
thought he could not well avoid including him in his
invitation to the officers, he was excessively glad to find
that he had taken himself out of the way. His coming
into the country at all, is a most insolent thing indeed,
and I wonder how he could presume to do it. I pity you,
Miss Eliza, for this discovery of your favourite’s guilt;
%%094%%
but really considering his descent, one could not expect
much better.”

“His guilt and his descent appear by your account to
be the same,” said Elizabeth angrily; “for I have heard
you accuse him of nothing worse than of being the son
of Mr. Darcy’s steward, and of \textit{that}, I can assure you, he
informed me himself.”

“I beg your pardon,” replied Miss Bingley, turning
away with a sneer. “Excuse my interference. -- It was
kindly meant.”

“Insolent girl!” said Elizabeth to herself. -- “You are
much mistaken if you expect to influence me by such
a paltry attack as this. I see nothing in it but your own
wilful ignorance and the malice of Mr. Darcy.” She then
sought her eldest sister, who had undertaken to make
inquiries on the same subject of Bingley. Jane met her
with a smile of such sweet complacency, a glow of such
happy expression, as sufficiently marked how well she was
satisfied with the occurrences of the evening. -- Elizabeth
instantly read her feelings, and at that moment solicitude
for Wickham, resentment against his enemies, and every
thing else gave way before the hope of Jane’s being in the
fairest way for happiness.

“I want to know,” said she, with a countenance no
less smiling than her sister’s, “what you have learnt
about Mr. Wickham. But perhaps you have been too
pleasantly engaged to think of any third person; in which
case you may be sure of my pardon.”

“No,” replied Jane, “I have not forgotten him; but
I have nothing satisfactory to tell you. Mr. Bingley does
not know the whole of his history, and is quite ignorant
of the circumstances which have principally offended
Mr. Darcy; but he will vouch for the good conduct, the
probity and honour of his friend, and is perfectly convinced
that Mr. Wickham has deserved much less attention from
Mr. Darcy than he has received; and I am sorry to say
that by his account as well as his sister’s, Mr. Wickham
is by no means a respectable young man. I am afraid
%%095%%
he has been very imprudent, and has deserved to lose
Mr. Darcy’s regard.”

“Mr. Bingley does not know Mr. Wickham himself?”

“No; he never saw him till the other morning at
Meryton.”

“This account then is what he has received from
Mr. Darcy. I am perfectly satisfied. But what does he
say of the living?”

“He does not exactly recollect the circumstances,
though he has heard them from Mr. Darcy more than
once, but he believes that it was left to him \textit{conditionally}
only.”

“I have not a doubt of Mr. Bingley’s sincerity,” said
Elizabeth warmly; “but you must excuse my not being
convinced by assurances only. Mr. Bingley’s defence of
his friend was a very able one I dare say, but since he
is unacquainted with several parts of the story, and has
learnt the rest from that friend himself, I shall venture
still to think of both gentlemen as I did before.”

She then changed the discourse to one more gratifying
to each, and on which there could be no difference of
sentiment. Elizabeth listened with delight to the happy,
though modest hopes which Jane entertained of Bingley’s
regard, and said all in her power to heighten her confidence
in it. On their being joined by Mr. Bingley
himself, Elizabeth withdrew to Miss Lucas; to whose
inquiry after the pleasantness of her last partner she had
scarcely replied, before Mr. Collins came up to them and
told her with great exultation that he had just been so
fortunate as to make a most important discovery.

“I have found out,” said he, “by a singular accident,
that there is now in the room a near relation of my
patroness. I happened to overhear the gentleman himself
mentioning to the young lady who does the honours of
this house the names of his cousin Miss de Bourgh, and of
her mother Lady Catherine. How wonderfully these sort
of things occur! Who would have thought of my meeting
with -- perhaps -- a nephew of Lady Catherine de Bourgh
%%096%%
in this assembly! -- I am most thankful that the discovery
is made in time for me to pay my respects to him, which
I am now going to do, and trust he will excuse my not
having done it before. My total ignorance of the connection
must plead my apology.”

“You are not going to introduce yourself to Mr.
Darcy?”

“Indeed I am. I shall intreat his pardon for not
having done it earlier. I believe him to be Lady Catherine’s
\textit{nephew}. It will be in my power to assure him that her
ladyship was quite well yesterday se’nnight.”

Elizabeth tried hard to dissuade him from such a
scheme; assuring him that Mr. Darcy would consider his
addressing him without introduction as an impertinent
freedom, rather than a compliment to his aunt; that it
was not in the least necessary there should be any notice
on either side, and that if it were, it must belong to
Mr. Darcy, the superior in consequence, to begin the
acquaintance. -- Mr. Collins listened to her with the determined
air of following his own inclination, and when she
ceased speaking, replied thus,

“My dear Miss Elizabeth, I have the highest opinion
in the world of your excellent judgment in all matters
within the scope of your understanding, but permit me
to say that there must be a wide difference between the
established forms of ceremony amongst the laity, and those
which regulate the clergy; for give me leave to observe
that I consider the clerical office as equal in point of
dignity with the highest rank in the kingdom -- provided
that a proper humility of behaviour is at the same time
maintained. You must therefore allow me to follow the
dictates of my conscience on this occasion, which leads
me to perform what I look on as a point of duty. Pardon
me for neglecting to profit by your advice, which on every
other subject shall be my constant guide, though in the
case before us I consider myself more fitted by education
and habitual study to decide on what is right than a young
lady like yourself.” And with a low bow he left her to
%%097%%
attack Mr. Darcy, whose reception of his advances she
eagerly watched, and whose astonishment at being so
addressed was very evident. Her cousin prefaced his
speech with a solemn bow, and though she could not hear
a word of it, she felt as if hearing it all, and saw in the
motion of his lips the words “apology,” “Hunsford,” and
“Lady Catherine de Bourgh.” -- It vexed her to see him
expose himself to such a man. Mr. Darcy was eyeing him
with unrestrained wonder, and when at last Mr. Collins
allowed him time to speak, replied with an air of distant
civility. Mr. Collins, however, was not discouraged from
speaking again, and Mr. Darcy’s contempt seemed abundantly
increasing with the length of his second speech,
and at the end of it he only made him a slight bow,
and moved another way. Mr. Collins then returned to
Elizabeth.

“I have no reason, I assure you,” said he, “to be
dissatisfied with my reception. Mr. Darcy seemed much
pleased with the attention. He answered me with the
utmost civility, and even paid me the compliment of
saying, that he was so well convinced of Lady Catherine’s
discernment as to be certain she could never bestow
a favour unworthily. It was really a very handsome
thought. Upon the whole, I am much pleased with him.”

As Elizabeth had no longer any interest of her own to
pursue, she turned her attention almost entirely on her
sister and Mr. Bingley, and the train of agreeable reflections
which her observations gave birth to, made her
perhaps almost as happy as Jane. She saw her in idea
settled in that very house in all the felicity which a marriage
of true affection could bestow; and she felt capable
under such circumstances, of endeavouring even to like
Bingley’s two sisters. Her mother’s thoughts she plainly
saw were bent the same way, and she determined not to
venture near her, lest she might hear too much. When they
sat down to supper, therefore, she considered it a most
unlucky perverseness which placed them within one of
each other; and deeply was she vexed to find that her
%%098%%
mother was talking to that one person (Lady Lucas)
freely, openly, and of nothing else but of her expectation
that Jane would be soon married to Mr. Bingley. -- It was
an animating subject, and Mrs. Bennet seemed incapable
of fatigue while enumerating the advantages of the match.
His being such a charming young man, and so rich, and
living but three miles from them, were the first points
of self-gratulation; and then it was such a comfort to
think how fond the two sisters were of Jane, and to be
certain that they must desire the connection as much as
she could do. It was, moreover, such a promising thing for
her younger daughters, as Jane’s marrying so greatly
must throw them in the way of other rich men; and
lastly, it was so pleasant at her time of life to be able to
consign her single daughters to the care of their sister,
that she might not be obliged to go into company more
than she liked. It was necessary to make this circumstance
a matter of pleasure, because on such occasions
it is the etiquette; but no one was less likely than
Mrs. Bennet to find comfort in staying at home at any period
of her life. She concluded with many good wishes that
Lady Lucas might soon be equally fortunate, though
evidently and triumphantly believing there was no chance
of it.

In vain did Elizabeth endeavour to check the rapidity
of her mother’s words, or persuade her to describe her
felicity in a less audible whisper; for to her inexpressible
vexation, she could perceive that the chief of it
was overheard by Mr. Darcy, who sat opposite to them.
Her mother only scolded her for being nonsensical.

“What is Mr. Darcy to me, pray, that I should be
afraid of him? I am sure we owe him no such particular
civility as to be obliged to say nothing \textit{he} may not like
to hear.”

“For heaven’s sake, madam, speak lower. -- What
advantage can it be to you to offend Mr. Darcy? -- You
will never recommend yourself to his friend by so doing.”

Nothing that she could say, however, had any influence.
%%099%%
Her mother would talk of her views in the same intelligible
tone. Elizabeth blushed and blushed again with shame
and vexation. She could not help frequently glancing her
eye at Mr. Darcy, though every glance convinced her of
what she dreaded; for though he was not always looking
at her mother, she was convinced that his attention was
invariably fixed by her. The expression of his face changed
gradually from indignant contempt to a composed and
steady gravity.

At length however Mrs. Bennet had no more to say;
and Lady Lucas, who had been long yawning at the
repetition of delights which she saw no likelihood of
sharing, was left to the comforts of cold ham and chicken.
Elizabeth now began to revive. But not long was the
interval of tranquillity; for when supper was over, singing
was talked of, and she had the mortification of seeing
Mary, after very little entreaty, preparing to oblige the
company. By many significant looks and silent entreaties,
did she endeavour to prevent such a proof of complaisance,
-- but in vain; Mary would not understand them; such
an opportunity of exhibiting was delightful to her, and
she began her song. Elizabeth’s eyes were fixed on her
with most painful sensations; and she watched her progress
through the several stanzas with an impatience which was
very ill rewarded at their close; for Mary, on receiving
amongst the thanks of the table, the hint of a hope that
she might be prevailed on to favour them again, after the
pause of half a minute began another. Mary’s powers
were by no means fitted for such a display; her voice
was weak, and her manner affected. -- Elizabeth was in
agonies. She looked at Jane, to see how she bore it; but
Jane was very composedly talking to Bingley. She looked
at his two sisters, and saw them making signs of derision
at each other, and at Darcy, who continued however
impenetrably grave. She looked at her father to entreat
his interference, lest Mary should be singing all night.
He took the hint, and when Mary had finished her second
song, said aloud,
%%100%%

“That will do extremely well, child. You have delighted
us long enough. Let the other young ladies have time
to exhibit.”

Mary, though pretending not to hear, was somewhat
disconcerted; and Elizabeth sorry for her, and sorry for
her father’s speech, was afraid her anxiety had done no
good. -- Others of the party were now applied to.

“If I,” said Mr. Collins, “were so fortunate as to be
able to sing, I should have great pleasure, I am sure, in
obliging the company with an air; for I consider music
as a very innocent diversion, and perfectly compatible
with the profession of a clergyman. -- I do not mean however
to assert that we can be justified in devoting too much
of our time to music, for there are certainly other things
to be attended to. The rector of a parish has much to do. --
In the first place, he must make such an agreement for
tythes as may be beneficial to himself and not offensive
to his patron. He must write his own sermons; and the
time that remains will not be too much for his parish
duties, and the care and improvement of his dwelling,
which he cannot be excused from making as comfortable
as possible. And I do not think it of light importance that
he should have attentive and conciliatory manners towards
every body, especially towards those to whom he owes
his preferment. I cannot acquit him of that duty; nor
could I think well of the man who should omit an occasion
of testifying his respect towards any body connected with
the family.” And with a bow to Mr. Darcy, he concluded
his speech, which had been spoken so loud as to be heard
by half the room. -- Many stared. -- Many smiled; but no
one looked more amused than Mr. Bennet himself, while
his wife seriously commended Mr. Collins for having spoken
so sensibly, and observed in a half-whisper to Lady Lucas,
that he was a remarkably clever, good kind of young man.

To Elizabeth it appeared, that had her family made
an agreement to expose themselves as much as they
could during the evening, it would have been impossible
for them to play their parts with more spirit, or finer
%%101%%
success; and happy did she think it for Bingley and her
sister that some of the exhibition had escaped his notice,
and that his feelings were not of a sort to be much distressed
by the folly which he must have witnessed. That
his two sisters and Mr. Darcy, however, should have such
an opportunity of ridiculing her relations was bad enough,
and she could not determine whether the silent contempt
of the gentleman, or the insolent smiles of the ladies, were
more intolerable.

The rest of the evening brought her little amusement.
She was teazed by Mr. Collins, who continued most perseveringly
by her side, and though he could not prevail
with her to dance with him again, put it out of her power
to dance with others. In vain did she entreat him to stand
up with somebody else, and offer to introduce him to any
young lady in the room. He assured her that as to dancing,
he was perfectly indifferent to it; that his chief object
was by delicate attentions to recommend himself to her,
and that he should therefore make a point of remaining
close to her the whole evening. There was no arguing
upon such a project. She owed her greatest relief to her
friend Miss Lucas, who often joined them, and good-naturedly
engaged Mr. Collins’s conversation to herself.

She was at least free from the offence of Mr. Darcy’s
farther notice; though often standing within a very short
distance of her, quite disengaged, he never came near
enough to speak. She felt it to be the probable consequence
of her allusions to Mr. Wickham, and rejoiced in it.

The Longbourn party were the last of all the company
to depart; and by a manœuvre of Mrs. Bennet had to
wait for their carriages a quarter of an hour after every
body else was gone, which gave them time to see how
heartily they were wished away by some of the family.
Mrs. Hurst and her sister scarcely opened their mouths
except to complain of fatigue, and were evidently impatient
to have the house to themselves. They repulsed every
attempt of Mrs. Bennet at conversation, and by so doing,
threw a languor over the whole party, which was very
%%102%%
little relieved by the long speeches of Mr. Collins, who was
complimenting Mr. Bingley and his sisters on the elegance
of their entertainment, and the hospitality and politeness
which had marked their behaviour to their guests. Darcy
said nothing at all. Mr. Bennet, in equal silence, was
enjoying the scene. Mr. Bingley and Jane were standing
together, a little detached from the rest, and talked only
to each other. Elizabeth preserved as steady a silence as
either Mrs. Hurst or Miss Bingley; and even Lydia was
too much fatigued to utter more than the occasional
exclamation of “Lord, how tired I am!” accompanied
by a violent yawn.

When at length they arose to take leave, Mrs. Bennet
was most pressingly civil in her hope of seeing the whole
family soon at Longbourn; and addressed herself particularly
to Mr. Bingley, to assure him how happy he would
make them, by eating a family dinner with them at any
time, without the ceremony of a formal invitation. Bingley
was all grateful pleasure, and he readily engaged for taking
the earliest opportunity of waiting on her, after his return
from London, whither he was obliged to go the next day
for a short time.

Mrs. Bennet was perfectly satisfied; and quitted the
house under the delightful persuasion that, allowing for
the necessary preparations of settlements, new carriages
and wedding clothes, she should undoubtedly see her
daughter settled at Netherfield, in the course of three or
four months. Of having another daughter married to
Mr. Collins, she thought with equal certainty, and with
considerable, though not equal, pleasure. Elizabeth was
the least dear to her of all her children; and though the
man and the match were quite good enough for \textit{her},
the worth of each was eclipsed by Mr. Bingley and
Netherfield.
%%103%%

\Chapter{CHAPTER XIX.}

The next day opened a new scene at Longbourn.
Mr. Collins made his declaration in form. Having resolved
to do it without loss of time, as his leave of absence
extended only to the following Saturday, and having no
feelings of diffidence to make it distressing to himself
even at the moment, he set about it in a very orderly
manner, with all the observances which he supposed a
regular part of the business. On finding Mrs. Bennet,
Elizabeth, and one of the younger girls together, soon
after breakfast, he addressed the mother in these words,

“May I hope, Madam, for your interest with your fair
daughter Elizabeth, when I solicit for the honour of a
private audience with her in the course of this morning?”

Before Elizabeth had time for any thing but a blush
of surprise, Mrs. Bennet instantly answered,

“Oh dear! -- Yes -- certainly. -- I am sure Lizzy will be
very happy -- I am sure she can have no objection. -- Come,
Kitty, I want you up stairs.” And gathering her work
together, she was hastening away, when Elizabeth called
out,

“Dear Ma’am, do not go. -- I beg you will not go. -- Mr.
Collins must excuse me. -- He can have nothing to
say to me that any body need not hear. I am going away
myself.”

“No, no, nonsense, Lizzy. -- I desire you will stay
where you are.” -- And upon Elizabeth’s seeming really,
with vexed and embarrassed looks, about to escape, she
added, “Lizzy, I \textit{insist} upon your staying and hearing
Mr. Collins.”

Elizabeth would not oppose such an injunction -- and
a moment’s consideration making her also sensible that
it would be wisest to get it over as soon and as quietly
as possible, she sat down again, and tried to conceal by
%%104%%
incessant employment the feelings which were divided
between distress and diversion. Mrs. Bennet and Kitty
walked off, and as soon as they were gone Mr. Collins
began.

“Believe me, my dear Miss Elizabeth, that your
modesty, so far from doing you any disservice, rather
adds to your other perfections. You would have been
less amiable in my eyes had there \textit{not} been this little
unwillingness; but allow me to assure you that I have
your respected mother’s permission for this address.
You can hardly doubt the purport of my discourse,
however your natural delicacy may lead you to dissemble;
my attentions have been too marked to be mistaken.
Almost as soon as I entered the house I singled you out
as the companion of my future life. But before I am run
away with by my feelings on this subject, perhaps it will
be advisable for me to state my reasons for marrying -- and
moreover for coming into Hertfordshire with the
design of selecting a wife, as I certainly did.”

The idea of Mr. Collins, with all his solemn composure,
being run away with by his feelings, made Elizabeth so
near laughing that she could not use the short pause he
allowed in any attempt to stop him farther, and he
continued:

“My reasons for marrying are, first, that I think it
a right thing for every clergyman in easy circumstances
(like myself) to set the example of matrimony in his
parish. Secondly, that I am convinced it will add very
greatly to my happiness; and thirdly -- which perhaps
I ought to have mentioned earlier, that it is the particular
advice and recommendation of the very noble lady whom
I have the honour of calling patroness. Twice has she
condescended to give me her opinion (unasked too!) on
this subject; and it was but the very Saturday night
before I left Hunsford -- between our pools at quadrille,
while Mrs. Jenkinson was arranging Miss de Bourgh’s
foot-stool, that she said, ‘Mr. Collins, you must marry.
A clergyman like you must marry. -- Chuse properly, chuse
%%105%%
a gentlewoman for \textit{my} sake; and for your \textit{own}, let her
be an active, useful sort of person, not brought up high,
but able to make a small income go a good way. This is
my advice. Find such a woman as soon as you can, bring
her to Hunsford, and I will visit her.’ Allow me, by the
way, to observe, my fair cousin, that I do not reckon
the notice and kindness of Lady Catherine de Bourgh
as among the least of the advantages in my power to
offer. You will find her manners beyond any thing I can
describe; and your wit and vivacity I think must be
acceptable to her, especially when tempered with the
silence and respect which her rank will inevitably excite.
Thus much for my general intention in favour of matrimony;
it remains to be told why my views were directed
to Longbourn instead of my own neighbourhood, where
I assure you there are many amiable young women.
But the fact is, that being, as I am, to inherit this estate
after the death of your honoured father, (who, however,
may live many years longer,) I could not satisfy myself
without resolving to chuse a wife from among his daughters,
that the loss to them might be as little as possible, when
the melancholy event takes place -- which, however, as
I have already said, may not be for several years. This
has been my motive, my fair cousin, and I flatter myself
it will not sink me in your esteem. And now nothing
remains for me but to assure you in the most animated
language of the violence of my affection. To fortune I am
perfectly indifferent, and shall make no demand of that
nature on your father, since I am well aware that it could
not be complied with; and that one thousand pounds
in the 4 per cents. which will not be yours till after your
mother’s decease, is all that you may ever be entitled to.
On that head, therefore, I shall be uniformly silent; and
you may assure yourself that no ungenerous reproach
shall ever pass my lips when we are married.”

It was absolutely necessary to interrupt him now.

“You are too hasty, Sir,” she cried. “You forget that
I have made no answer. Let me do it without farther
%%106%%
loss of time. Accept my thanks for the compliment you
are paying me. I am very sensible of the honour of your
proposals, but it is impossible for me to do otherwise
than decline them.”

“I am not now to learn,” replied Mr. Collins, with
a formal wave of the hand, “that it is usual with young
ladies to reject the addresses of the man whom they
secretly mean to accept, when he first applies for their
favour; and that sometimes the refusal is repeated a
second or even a third time. I am therefore by no means
discouraged by what you have just said, and shall hope
to lead you to the altar ere long.”

“Upon my word, Sir,” cried Elizabeth, “your hope is
rather an extraordinary one after my declaration. I do
assure you that I am not one of those young ladies (if such
young ladies there are) who are so daring as to risk their
happiness on the chance of being asked a second time.
I am perfectly serious in my refusal. -- You could not
make \textit{me} happy, and I am convinced that I am the last
woman in the world who would make \textit{you} so. -- Nay, were
your friend Lady Catherine to know me, I am persuaded
she would find me in every respect ill qualified for the
situation.”

“Were it certain that Lady Catherine would think so,”
said Mr. Collins very gravely -- “but I cannot imagine
that her ladyship would at all disapprove of you. And
you may be certain that when I have the honour of seeing
her again I shall speak in the highest terms of your
modesty, economy, and other amiable qualifications.”

“Indeed, Mr. Collins, all praise of me will be unnecessary.
You must give me leave to judge for myself, and
pay me the compliment of believing what I say. I wish
you very happy and very rich, and by refusing your hand,
do all in my power to prevent your being otherwise. In
making me the offer, you must have satisfied the delicacy
of your feelings with regard to my family, and may take
possession of Longbourn estate whenever it falls, without
any self-reproach. This matter may be considered,
%%107%%
therefore, as finally settled.” And rising as she thus spoke, she
would have quitted the room, had not Mr. Collins thus
addressed her,

“When I do myself the honour of speaking to you
next on this subject I shall hope to receive a more favourable
answer than you have now given me; though I am
far from accusing you of cruelty at present, because
I know it to be the established custom of your sex to reject
a man on the first application, and perhaps you have even
now said as much to encourage my suit as would be consistent
with the true delicacy of the female character.”

“Really, Mr. Collins,” cried Elizabeth with some
warmth, “you puzzle me exceedingly. If what I have
hitherto said can appear to you in the form of encouragement,
I know not how to express my refusal in such a way
as may convince you of its being one.”

“You must give me leave to flatter myself, my dear
cousin, that your refusal of my addresses is merely
words of course. My reasons for believing it are briefly
these:-- It does not appear to me that my hand is unworthy
your acceptance, or that the establishment I can
offer would be any other than highly desirable. My
situation in life, my connections with the family of
De Bourgh, and my relationship to your own, are circumstances
highly in my favour; and you should take
it into farther consideration that in spite of your manifold
attractions, it is by no means certain that another offer
of marriage may ever be made you. Your portion is
unhappily so small that it will in all likelihood undo the
effects of your loveliness and amiable qualifications. As
I must therefore conclude that you are not serious in your
rejection of me, I shall chuse to attribute it to your wish
of increasing my love by suspense, according to the usual
practice of elegant females.”

“I do assure you, Sir, that I have no pretension
whatever to that kind of elegance which consists in
tormenting a respectable man. I would rather be paid
%%108%%
the compliment of being believed sincere. I thank you
again and again for the honour you have done me in
your proposals, but to accept them is absolutely impossible.
My feelings in every respect forbid it. Can
I speak plainer? Do not consider me now as an elegant
female intending to plague you, but as a rational creature
speaking the truth from her heart.”

“You are uniformly charming!” cried he, with an air
of awkward gallantry; “and I am persuaded that when
sanctioned by the express authority of both your excellent
parents, my proposals will not fail of being acceptable.”

To such perseverance in wilful self-deception Elizabeth
would make no reply, and immediately and in silence
withdrew; determined, if he persisted in considering her
repeated refusals as flattering encouragement, to apply
to her father, whose negative might be uttered in such
a manner as must be decisive, and whose behaviour
at least could not be mistaken for the affectation and
coquetry of an elegant female.
%%109%%

\Chapter{CHAPTER XX.}

Mr. Collins was not left long to the silent contemplation
of his successful love; for Mrs. Bennet, having dawdled
about in the vestibule to watch for the end of the conference,
no sooner saw Elizabeth open the door and with
quick step pass her towards the staircase, than she entered
the breakfast-room, and congratulated both him and herself
in warm terms on the happy prospect of their nearer
connection. Mr. Collins received and returned these
felicitations with equal pleasure, and then proceeded to
relate the particulars of their interview, with the result
of which he trusted he had every reason to be satisfied,
since the refusal which his cousin had stedfastly given
him would naturally flow from her bashful modesty and
the genuine delicacy of her character.

This information, however, startled Mrs. Bennet; -- she
would have been glad to be equally satisfied that her
daughter had meant to encourage him by protesting
against his proposals, but she dared not to believe it, and
could not help saying so.

“But depend upon it, Mr. Collins,” she added, “that
Lizzy shall be brought to reason. I will speak to her
about it myself directly. She is a very headstrong foolish
girl, and does not know her own interest; but I will
\textit{make} her know it.”

“Pardon me for interrupting you, Madam,” cried
Mr. Coll\-ins; “but if she is really headstrong and foolish,
I know not whether she would altogether be a very
desirable wife to a man in my situation, who naturally
looks for happiness in the marriage state. If therefore
she actually persists in rejecting my suit, perhaps it were
better not to force her into accepting me, because if liable
to such defects of temper, she could not contribute much
to my felicity.”
%%110%%

“Sir, you quite misunderstand me,” said Mrs. Bennet,
alarmed. “Lizzy is only headstrong in such matters as
these. In every thing else she is as good natured a girl
as ever lived. I will go directly to Mr. Bennet, and we
shall very soon settle it with her, I am sure.”

She would not give him time to reply, but hurrying
instantly to her husband, called out as she entered the
library,

“Oh! Mr. Bennet, you are wanted immediately; we
are all in an uproar. You must come and make Lizzy
marry Mr. Collins, for she vows she will not have him,
and if you do not make haste he will change his mind and
not have \textit{her}.”

Mr. Bennet raised his eyes from his book as she entered,
and fixed them on her face with a calm unconcern which
was not in the least altered by her communication.

“I have not the pleasure of understanding you,” said
he, when she had finished her speech. “Of what are you
talking?”

“Of Mr. Collins and Lizzy. Lizzy declares she will not
have Mr. Collins, and Mr. Collins begins to say that he
will not have Lizzy.”

“And what am I to do on the occasion? -- It seems an
hopeless business.”

“Speak to Lizzy about it yourself. Tell her that you
insist upon her marrying him.”

“Let her be called down. She shall hear my opinion.”

Mrs. Bennet rang the bell, and Miss Elizabeth was
summoned to the library.

“Come here, child,” cried her father as she appeared.
“I have sent for you on an affair of importance. I understand
that Mr. Collins has made you an offer of marriage.
Is it true?” Elizabeth replied that it was. “Very well -- and
this offer of marriage you have refused?”

“I have, Sir.”

“Very well. We now come to the point. Your mother
insists upon your accepting it. Is not it so, Mrs. Bennet?”

“Yes, or I will never see her again.”
%%111%%

“An unhappy alternative is before you, Elizabeth.
From this day you must be a stranger to one of your
parents. -- Your mother will never see you again if you
do \textit{not} marry Mr. Collins, and I will never see you again
if you \textit{do}.”

Elizabeth could not but smile at such a conclusion of
such a beginning; but Mrs. Bennet, who had persuaded
herself that her husband regarded the affair as she wished,
was excessively disappointed.

“What do you mean, Mr. Bennet, by talking in this
way? You promised me to \textit{insist} upon her marrying him.”

“My dear,” replied her husband, “I have two small
favours to request. First, that you will allow me the free
use of my understanding on the present occasion; and
secondly, of my room. I shall be glad to have the library
to myself as soon as may be.”

Not yet, however, in spite of her disappointment in her
husband, did Mrs. Bennet give up the point. She talked
to Elizabeth again and again; coaxed and threatened her
by turns. She endeavoured to secure Jane in her interest,
but Jane with all possible mildness declined interfering; -- and
Elizabeth sometimes with real earnestness and sometimes
with playful gaiety replied to her attacks. Though
her manner varied however, her determination never did.

Mr. Collins, meanwhile, was meditating in solitude on
what had passed. He thought too well of himself to comprehend
on what motive his cousin could refuse him; and
though his pride was hurt, he suffered in no other way.
His regard for her was quite imaginary; and the possibility
of her deserving her mother’s reproach prevented his
feeling any regret.

While the family were in this confusion, Charlotte
Lucas came to spend the day with them. She was met
in the vestibule by Lydia, who, flying to her, cried in a
half whisper, “I am glad you are come, for there is such
fun here! -- What do you think has happened this
morning? -- Mr. Collins has made an offer to Lizzy, and she will
not have him.”
%%112%%

Charlotte had hardly time to answer, before they were
joined by Kitty, who came to tell the same news, and no
sooner had they entered the breakfast-room, where Mrs.
Bennet was alone, than she likewise began on the subject,
calling on Miss Lucas for her compassion, and entreating
her to persuade her friend Lizzy to comply with the wishes
of all her family. “Pray do, my dear Miss Lucas,” she
added in a melancholy tone, “for nobody is on my side,
nobody takes part with me, I am cruelly used, nobody
feels for my poor nerves.”

Charlotte’s reply was spared by the entrance of Jane
and Elizabeth.

“Aye, there she comes,” continued Mrs. Bennet,
“looking as unconcerned as may be, and caring no more
for us than if we were at York, provided she can have
her own way. -- But I tell you what, Miss Lizzy, if you
take it into your head to go on refusing every offer of
marriage in this way, you will never get a husband at
all -- and I am sure I do not know who is to maintain you
when your father is dead. -- \textit{I} shall not be able to keep
you -- and so I warn you. -- I have done with you from
this very day. -- I told you in the library, you know, that
I should never speak to you again, and you will find me
as good as my word. I have no pleasure in talking to
undutiful children. -- Not that I have much pleasure indeed
in talking to any body. People who suffer as I do from
nervous complaints can have no great inclination for
talking. Nobody can tell what I suffer! -- But it is always
so. Those who do not complain are never pitied.”

Her daughters listened in silence to this effusion, sensible
that any attempt to reason with or sooth her would only
increase the irritation. She talked on, therefore, without
interruption from any of them till they were joined by
Mr. Collins, who entered with an air more stately than
usual, and on perceiving whom, she said to the girls,

“Now, I do insist upon it, that you, all of you, hold
your tongues, and let Mr. Collins and me have a little
conversation together.”
%%113%%

Elizabeth passed quietly out of the room, Jane and
Kitty followed, but Lydia stood her ground, determined
to hear all she could; and Charlotte, detained first by
the civility of Mr. Collins, whose inquiries after herself
and all her family were very minute, and then by a little
curiosity, satisfied herself with walking to the window and
pretending not to hear. In a doleful voice Mrs. Bennet
thus began the projected conversation. -- “Oh! Mr.
Collins!” --

“My dear Madam,” replied he, “let us be for ever
silent on this point. Far be it from me,” he presently
continued in a voice that marked his displeasure, “to
resent the behaviour of your daughter. Resignation to
inevitable evils is the duty of us all; the peculiar duty
of a young man who has been so fortunate as I have been
in early preferment; and I trust I am resigned. Perhaps
not the less so from feeling a doubt of my positive happiness
had my fair cousin honoured me with her hand;
for I have often observed that resignation is never so
perfect as when the blessing denied begins to lose somewhat
of its value in our estimation. You will not, I hope,
consider me as shewing any disrespect to your family,
my dear Madam, by thus withdrawing my pretensions to
your daughter’s favour, without having paid yourself and
Mr. Bennet the compliment of requesting you to interpose
your authority in my behalf. My conduct may
I fear be objectionable in having accepted my dismission
from your daughter’s lips instead of your own. But we
are all liable to error. I have certainly meant well through
the whole affair. My object has been to secure an amiable
companion for myself, with due consideration for the
advantage of all your family, and if my \textit{manner} has been
at all reprehensible, I here beg leave to apologise.”
%%114%%

\Chapter{CHAPTER XXI.}

The discussion of Mr. Collins’s offer was now nearly at
an end, and Elizabeth had only to suffer from the uncomfortable
feelings necessarily attending it, and occasionally
from some peevish allusion of her mother. As for the
gentleman himself, \textit{his} feelings were chiefly expressed, not
by embarrassment or dejection, or by trying to avoid her,
but by stiffness of manner and resentful silence. He
scarcely ever spoke to her, and the assiduous attentions
which he had been so sensible of himself, were transferred
for the rest of the day to Miss Lucas, whose civility in
listening to him, was a seasonable relief to them all, and
especially to her friend.

The morrow produced no abatement of Mrs. Bennet’s
ill humour or ill health. Mr. Collins was also in the
same state of angry pride. Elizabeth had hoped that his
resentment might shorten his visit, but his plan did not
appear in the least affected by it. He was always to have
gone on Saturday, and to Saturday he still meant to stay.

After breakfast, the girls walked to Meryton to inquire
if Mr. Wickham were returned, and to lament over his
absence from the Netherfield ball. He joined them on
their entering the town and attended them to their aunt’s,
where his regret and vexation, and the concern of every
body was well talked over. -- To Elizabeth, however, he
voluntarily acknowledged that the necessity of his absence
\textit{had} been self imposed.

“I found,” said he, “as the time drew near, that I had
better not meet Mr. Darcy; -- that to be in the same
room, the same party with him for so many hours together,
might be more than I could bear, and that scenes might
arise unpleasant to more than myself.”

She highly approved his forbearance, and they had
leis\-ure for a full discussion of it, and for all the
%%115%%
commendation which they civilly bestowed on each other,
as Wickham and another officer walked back with them
to Longbourn, and during the walk, he particularly
attended to her. His accompanying them was a double
advantage; she felt all the compliment it offered to
herself, and it was most acceptable as an occasion of
introducing him to her father and mother.

Soon after their return, a letter was delivered to Miss
Bennet; it came from Netherfield, and was opened
immediately. The envelope contained a sheet of elegant,
little, hot pressed paper, well covered with a lady’s fair,
flowing hand; and Elizabeth saw her sister’s countenance
change as she read it, and saw her dwelling intently on
some particular passages. Jane recollected herself soon,
and putting the letter away, tried to join with her usual
cheerfulness in the general conversation; but Elizabeth
felt an anxiety on the subject which drew off her attention
even from Wickham; and no sooner had he and his
companion taken leave, than a glance from Jane invited
her to follow her up stairs. When they had gained their
own room, Jane taking out the letter, said,

“This is from Caroline Bingley; what it contains, has
surprised me a good deal. The whole party have left
Netherfield by this time, and are on their way to town;
and without any intention of coming back again. You
shall hear what she says.”

She then read the first sentence aloud, which comprised
the information of their having just resolved to follow
their brother to town directly, and of their meaning to
dine that day in Grosvenor street, where Mr. Hurst had
a house. The next was in these words. “I do not pretend
to regret any thing I shall leave in Hertfordshire, except
your society, my dearest friend; but we will hope at
some future period, to enjoy many returns of the delightful
intercourse we have known, and in the mean while may
lessen the pain of separation by a very frequent and most
unreserved correspondence. I depend on you for that.”
To these high flown expressions, Elizabeth listened with all
%%116%%
the insensibility of distrust; and though the suddenness
of their removal surprised her, she saw nothing in it really
to lament; it was not to be supposed that their absence
from Netherfield would prevent Mr. Bingley’s being there;
and as to the loss of their society, she was persuaded that
Jane must soon cease to regard it, in the enjoyment of his.

“It is unlucky,” said she, after a short pause, “that
you should not be able to see your friends before they
leave the country. But may we not hope that the period
of future happiness to which Miss Bingley looks forward,
may arrive earlier than she is aware, and that the delightful
intercourse you have known as friends, will be renewed
with yet greater satisfaction as sisters? -- Mr. Bingley will
not be detained in London by them.”

“Caroline decidedly says that none of the party will
return into Hertfordshire this winter. I will read it to
you --

“When my brother left us yesterday, he imagined
that the business which took him to London, might be
concluded in three or four days, but as we are certain
it cannot be so, and at the same time convinced that
when Charles gets to town, he will be in no hurry to leave
it again, we have determined on following him thither,
that he may not be obliged to spend his vacant hours in
a comfortless hotel. Many of my acquaintance are already
there for the winter; I wish I could hear that you, my
dearest friend, had any intention of making one in the
croud, but of that I despair. I sincerely hope your Christmas
in Hertfordshire may abound in the gaieties which
that season generally brings, and that your beaux will be
so numerous as to prevent your feeling the loss of the three,
of whom we shall deprive you.”

“It is evident by this,” added Jane, “that he comes
back no more this winter.”

“It is only evident that Miss Bingley does not mean
he \textit{should}.”

“Why will you think so? It must be his own doing. -- He
is his own master. But you do not know \textit{all}. I \textit{will}
%%117%%
read you the passage which particularly hurts me. I will
have no reserves from \textit{you}.” “Mr. Darcy is impatient
to see his sister, and to confess the truth, \textit{we} are
scarcely less eager to meet her again. I really do
not think Georgiana Darcy has her equal for beauty,
elegance, and accomplishments; and the affection she
inspires in Louisa and myself, is heightened into something
still more interesting, from the hope we dare to
entertain of her being hereafter our sister. I do not
know whether I ever before mentioned to you my
feelings on this subject, but I will not leave the country
without confiding them, and I trust you will not esteem
them unreasonable. My brother admires her greatly
already, he will have frequent opportunity now of seeing
her on the most intimate footing, her relations all wish
the connection as much as his own, and a sister’s partiality
is not misleading me, I think, when I call Charles most
capable of engaging any woman’s heart. With all these
circumstances to favour an attachment and nothing to prevent
it, am I wrong, my dearest Jane, in indulging the hope
of an event which will secure the happiness of so many?”

“What think you of \textit{this} sentence, my dear Lizzy?” -- said
Jane as she finished it. “Is it not clear enough? -- Does
it not expressly declare that Caroline neither expects
nor wishes me to be her sister; that she is perfectly convinced
of her brother’s indifference, and that if she suspects
the nature of my feelings for him, she means (most kindly!)
to put me on my guard? Can there be any other opinion
on the subject?”

“Yes, there can; for mine is totally different. -- Will
you hear it?”

“Most willingly.”

“You shall have it in few words. Miss Bingley sees
that her brother is in love with you, and wants him to
marry Miss Darcy. She follows him to town in the hope
of keeping him there, and tries to persuade you that he
does not care about you.”

Jane shook her head.
%%118%%

“Indeed, Jane, you ought to believe me. -- No one who
has ever seen you together, can doubt his affection. Miss
Bingley I am sure cannot. She is not such a simpleton.
Could she have seen half as much love in Mr. Darcy for
herself, she would have ordered her wedding clothes. But
the case is this. We are not rich enough, or grand enough
for them; and she is the more anxious to get Miss Darcy
for her brother, from the notion that when there has been
\textit{one} intermarriage, she may have less trouble in achieving
a second; in which there is certainly some ingenuity, and
I dare say it would succeed, if Miss de Bourgh were out
of the way. But, my dearest Jane, you cannot seriously
imagine that because Miss Bingley tells you her brother
greatly admires Miss Darcy, he is in the smallest degree
less sensible of \textit{your} merit than when he took leave of you
on Tuesday, or that it will be in her power to persuade
him that instead of being in love with you, he is very
much in love with her friend.”

“If we thought alike of Miss Bingley,” replied Jane,
“your representation of all this, might make me quite
easy. But I know the foundation is unjust. Caroline is
incapable of wilfully deceiving any one; and all that
I can hope in this case is, that she is deceived herself.”

“That is right. -- You could not have started a more
happy idea, since you will not take comfort in mine.
Believe her to be deceived by all means. You have now
done your duty by her, and must fret no longer.”

“But, my dear sister, can I be happy, even supposing
the best, in accepting a man whose sisters and friends are
all wishing him to marry elsewhere?”

“You must decide for yourself,” said Elizabeth, “and
if upon mature deliberation, you find that the misery of
disobliging his two sisters is more than equivalent to the
happiness of being his wife, I advise you by all means
to refuse him.”

“How can you talk so?” -- said Jane faintly smiling, -- “You
must know that though I should be exceedingly
grieved at their disapprobation, I could not hesitate.”
%%119%%

“I did not think you would; -- and that being the case,
I cannot consider your situation with much compassion.”

“But if he returns no more this winter, my choice will
never be required. A thousand things may arise in six
months!”

The idea of his returning no more Elizabeth treated
with the utmost contempt. It appeared to her merely the
suggestion of Caroline’s interested wishes, and she could
not for a moment suppose that those wishes, however
openly or artfully spoken, could influence a young man so
totally independent of every one.

She represented to her sister as forcibly as possible
what she felt on the subject, and had soon the pleasure
of seeing its happy effect. Jane’s temper was not desponding,
and she was gradually led to hope, though the
diffidence of affection sometimes overcame the hope, that
Bingley would return to Netherfield and answer every wish
of her heart.

They agreed that Mrs. Bennet should only hear of the
departure of the family, without being alarmed on
the score of the gentleman’s conduct; but even this
partial communication gave her a great deal of concern,
and she bewailed it as exceedingly unlucky that the ladies
should happen to go away, just as they were all getting
so intimate together. After lamenting it however at some
length, she had the consolation of thinking that Mr.
Bingley would be soon down again and soon dining at
Longbourn, and the conclusion of all was the comfortable
declaration that, though he had been invited only to a
family dinner, she would take care to have two full
courses.
%%120%%

\Chapter{CHAPTER XXII.}

The Bennets were engaged to dine with the Lucases,
and again during the chief of the day, was Miss Lucas so
kind as to listen to Mr. Collins. Elizabeth took an opportunity
of thanking her. “It keeps him in good humour,”
said she, “and I am more obliged to you than I can
express.” Charlotte assured her friend of her satisfaction
in being useful, and that it amply repaid her for the little
sacrifice of her time. This was very amiable, but Charlotte’s
kindness extended farther than Elizabeth had any
conception of; -- its object was nothing less, than to secure
her from any return of Mr. Collins’s addresses, by engaging
them towards herself. Such was Miss Lucas’s scheme;
and appearances were so favourable that when they
parted at night, she would have felt almost sure of success
if he had not been to leave Hertfordshire so very soon.
But here, she did injustice to the fire and independence
of his character, for it led him to escape out of Longbourn
House the next morning with admirable slyness, and
hasten to Lucas Lodge to throw himself at her feet. He
was anxious to avoid the notice of his cousins, from
a conviction that if they saw him depart, they could not
fail to conjecture his design, and he was not willing to
have the attempt known till its success could be known
likewise; for though feeling almost secure, and with
reason, for Charlotte had been tolerably encouraging, he
was comparatively diffident since the adventure of Wednesday.
His reception however was of the most flattering
kind. Miss Lucas perceived him from an upper window
as he walked towards the house, and instantly set out
to meet him accidentally in the lane.But little had she
dared to hope that so much love and eloquence awaited
her there.

In as short a time as Mr. Collins’s long speeches would
%%121%%
allow, every thing was settled between them to the satisfaction
of both; and as they entered the house, he earnestly
entreated her to name the day that was to make him the
happiest of men; and though such a solicitation must be
waved for the present, the lady felt no inclination to
trifle with his happiness. The stupidity with which he
was favoured by nature, must guard his courtship from
any charm that could make a woman wish for its continuance;
and Miss Lucas, who accepted him solely from the
pure and disinterested desire of an establishment, cared
not how soon that establishment were gained.

Sir William and Lady Lucas were speedily applied to
for their consent; and it was bestowed with a most
joyful alacrity. Mr. Collins’s present circumstances made
it a most eligible match for their daughter, to whom they
could give little fortune; and his prospects of future
wealth were exceedingly fair. Lady Lucas began directly
to calculate with more interest than the matter had ever
excited before, how many years longer Mr. Bennet was
likely to live; and Sir William gave it as his decided
opinion, that whenever Mr. Collins should be in possession
of the Longbourn estate, it would be highly expedient
that both he and his wife should make their appearance
at St. James’s. The whole family in short were properly
overjoyed on the occasion. The younger girls formed
hopes of \textit{coming out} a year or two sooner than they might
otherwise have done; and the boys were relieved from
their apprehension of Charlotte’s dying an old maid.
Charlotte herself was tolerably composed. She had gained
her point, and had time to consider of it. Her reflections
were in general satisfactory. Mr. Collins to be sure was
neither sensible nor agreeable; his society was irksome,
and his attachment to her must be imaginary. But still
he would be her husband. -- Without thinking highly either
of men or of matrimony, marriage had always been her
object; it was the only honourable provision for well-educated
young women of small fortune, and however
uncertain of giving happiness, must be their pleasantest
%%122%%
preservative from want. This preservative she had now
obtained; and at the age of twenty-seven, without having
ever been handsome, she felt all the good luck of it. The
least agreeable circumstance in the business, was the surprise
it must occasion to Elizabeth Bennet, whose friendship
she valued beyond that of any other person. Elizabeth
would wonder, and probably would blame her; and though
her resolution was not to be shaken, her feelings must be
hurt by such disapprobation. She resolved to give her
the information herself, and therefore charged Mr. Collins
when he returned to Longbourn to dinner, to drop no
hint of what had passed before any of the family. A promise
of secrecy was of course very dutifully given, but it
could not be kept without difficulty; for the curiosity
excited by his long absence, burst forth in such very
direct questions on his return, as required some ingenuity
to evade, and he was at the same time exercising great
self-denial, for he was longing to publish his prosperous love.

As he was to begin his journey too early on the morrow
to see any of the family, the ceremony of leave-taking
was performed when the ladies moved for the night; and
Mrs. Bennet with great politeness and cordiality said how
happy they should be to see him at Longbourn again, whenever
his other engagements might allow him to visit them.

“My dear Madam,” he replied, “this invitation is
particularly gratifying, because it is what I have been
hoping to receive; and you may be very certain that I
shall avail myself of it as soon as possible.”

They were all astonished; and Mr. Bennet, who could
by no means wish for so speedy a return, immediately said,

“But is there not danger of Lady Catherine’s disapprobation
here, my good sir? -- You had better neglect your
relations, than run the risk of offending your patroness.”

“My dear sir,” replied Mr. Collins, “I am particularly
obliged to you for this friendly caution, and you may
depend upon my not taking so material a step without her
ladyship’s concurrence.”

“You cannot be too much on your guard. Risk
%%123%%
any thing rather than her displeasure; and if you find it
likely to be raised by your coming to us again, which I
should think exceedingly probable, stay quietly at home,
and be satisfied that \textit{we} shall take no offence.”

“Believe me, my dear sir, my gratitude is warmly
excited by such affectionate attention; and depend upon
it, you will speedily receive from me a letter of thanks for
this, as well as for every other mark of your regard during
my stay in Hertfordshire. As for my fair cousins, though
my absence may not be long enough to render it necessary,
I shall now take the liberty of wishing them health and
happiness, not excepting my cousin Elizabeth.”

With proper civilities the ladies then withdrew; all of
them equally surprised to find that he meditated a quick
return. Mrs. Bennet wished to understand by it that he
thought of paying his addresses to one of her younger
girls, and Mary might have been prevailed on to accept
him. She rated his abilities much higher than any of the
others; there was a solidity in his reflections which often
struck her, and though by no means so clever as herself,
she thought that if encouraged to read and improve himself
by such an example as her’s, he might become a very
agreeable companion. But on the following morning,
every hope of this kind was done away. Miss Lucas called
soon after breakfast, and in a private conference with
Elizabeth related the event of the day before.

The possibility of Mr. Collins’s fancying himself in love
with her friend had once occurred to Elizabeth within the
last day or two; but that Charlotte could encourage him,
seemed almost as far from possibility as that she could
encourage him herself, and her astonishment was consequently
so great as to overcome at first the bounds of
decorum, and she could not help crying out,

“Engaged to Mr. Collins! my dear Charlotte, --
impossible!”

The steady countenance which Miss Lucas had commanded
in telling her story, gave way to a momentary
confusion here on receiving so direct a reproach; though,
%%124%%
as it was no more than she expected, she soon regained her
composure, and calmly replied,

“Why should you be surprised, my dear Eliza? -- Do
you think it incredible that Mr. Collins should be able to
procure any woman’s good opinion, because he was not so
happy as to succeed with you?”

But Elizabeth had now recollected herself, and making a
strong effort for it, was able to assure her with tolerable
firmness that the prospect of their relationship was highly grateful
to her, and that she wished her all imaginable happiness.

“I see what you are feeling,” replied Charlotte, -- “you
must be surprised, very much surprised, -- so lately as
Mr. Collins was wishing to marry you. But when you
have had time to think it all over, I hope you will be
satisfied with what I have done. I am not romantic you
know. I never was. I ask only a comfortable home;
and considering Mr. Collins’s character, connections, and
situation in life, I am convinced that my chance of happiness
with him is as fair, as most people can boast on
entering the marriage state.”

Elizabeth quietly answered “Undoubtedly;” -- and
after an awkward pause, they returned to the rest of the
family. Charlotte did not stay much longer, and Elizabeth
was then left to reflect on what she had heard. It was
a long time before she became at all reconciled to the idea
of so unsuitable a match. The strangeness of Mr. Collins’s
making two offers of marriage within three days, was
nothing in comparison of his being now accepted. She
had always felt that Charlotte’s opinion of matrimony
was not exactly like her own, but she could not have
supposed it possible that when called into action, she
would have sacrificed every better feeling to worldly
advantage. Charlotte the wife of Mr. Collins, was a most
humiliating picture! -- And to the pang of a friend disgracing
herself and sunk in her esteem, was added the
distressing conviction that it was impossible for that
friend to be tolerably happy in the lot she had chosen.
%%125%%

\Chapter{CHAPTER XXIII.}

Elizabeth was sitting with her mother and sisters,
reflecting on what she had heard, and doubting whether
she were authorised to mention it, when Sir William Lucas
himself appeared, sent by his daughter to announce her
engagement to the family. With many compliments to
them, and much self-gratulation on the prospect of a
connection between the houses, he unfolded the matter, -- to
an audience not merely wondering, but incredulous;
for Mrs. Bennet, with more perseverance than politeness,
protested he must be entirely mistaken, and Lydia, always
unguarded and often uncivil, boisterously exclaimed,

“Good Lord! Sir William, how can you tell such
a story? -- Do not you know that Mr. Collins wants to
marry Lizzy?”

Nothing less than the complaisance of a courtier could
have borne without anger such treatment; but Sir
William’s good breeding carried him through it all; and
though he begged leave to be positive as to the truth
of his information, he listened to all their impertinence
with the most forbearing courtesy.

Elizabeth, feeling it incumbent on her to relieve him
from so unpleasant a situation, now put herself forward
to confirm his account, by mentioning her prior knowledge
of it from Charlotte herself; and endeavoured to put
a stop to the exclamations of her mother and sisters, by
the earnestness of her congratulations to Sir William,
in which she was readily joined by Jane, and by making
a variety of remarks on the happiness that might be
expected from the match, the excellent character of
Mr. Collins, and the convenient distance of Hunsford from
London.

Mrs. Bennet was in fact too much overpowered to
say a great deal while Sir William remained; but no
%%126%%
sooner had he left them than her feelings found a rapid
vent. In the first place, she persisted in disbelieving the
whole of the matter; secondly, she was very sure that
Mr. Collins had been taken in; thirdly, she trusted that
they would never be happy together; and fourthly, that
the match might be broken off. Two inferences, however,
were plainly deduced from the whole; one, that Elizabeth
was the real cause of all the mischief; and the other, that
she herself had been barbarously used by them all; and
on these two points she principally dwelt during the rest
of the day. Nothing could console and nothing appease
her. -- Nor did that day wear out her resentment. A week
elapsed before she could see Elizabeth without scolding
her, a month passed away before she could speak to
Sir William or Lady Lucas without being rude, and many
months were gone before she could at all forgive their
daughter.

Mr. Bennet’s emotions were much more tranquil on the
occasion, and such as he did experience he pronounced
to be of a most agreeable sort; for it gratified him, he
said, to discover that Charlotte Lucas, whom he had been
used to think tolerably sensible, was as foolish as his wife,
and more foolish than his daughter!

Jane confessed herself a little surprised at the match;
but she said less of her astonishment than of her earnest
desire for their happiness; nor could Elizabeth persuade
her to consider it as improbable. Kitty and Lydia were
far from envying Miss Lucas, for Mr. Collins was only
a clergyman; and it affected them in no other way than
as a piece of news to spread at Meryton.

Lady Lucas could not be insensible of triumph on being
able to retort on Mrs. Bennet the comfort of having a
daughter well married; and she called at Longbourn
rather oftener than usual to say how happy she was,
though Mrs. Bennet’s sour looks and ill-natured remarks
might have been enough to drive happiness away.

Between Elizabeth and Charlotte there was a restraint
which kept them mutually silent on the subject; and
%%127%%
Elizabeth felt persuaded that no real confidence could
ever subsist between them again. Her disappointment
in Charlotte made her turn with fonder regard to her
sister, of whose rectitude and delicacy she was sure her
opinion could never be shaken, and for whose happiness
she grew daily more anxious, as Bingley had now been
gone a week, and nothing was heard of his return.

Jane had sent Caroline an early answer to her letter,
and was counting the days till she might reasonably hope
to hear again. The promised letter of thanks from
Mr. Collins arrived on Tuesday, addressed to their father,
and written with all the solemnity of gratitude which a
twelvemonth’s abode in the family might have prompted.
After discharging his conscience on that head, he proceeded
to inform them, with many rapturous expressions,
of his happiness in having obtained the affection of their
amiable neighbour, Miss Lucas, and then explained that
it was merely with the view of enjoying her society that
he had been so ready to close with their kind wish of
seeing him again at Longbourn, whither he hoped to be
able to return on Monday fortnight; for Lady Catherine,
he added, so heartily approved his marriage, that she
wished it to take place as soon as possible, which he trusted
would be an unanswerable argument with his amiable
Charlotte to name an early day for making him the happiest
of men.

Mr. Collins’s return into Hertfordshire was no longer
a matter of pleasure to Mrs. Bennet. On the contrary
she was as much disposed to complain of it as her
husband. -- It was very strange that he should come to
Longbourn instead of to Lucas Lodge; it was also very
inconvenient and exceedingly troublesome. -- She hated
having visitors in the house while her health was so
indifferent, and lovers were of all people the most disagreeable.
Such were the gentle murmurs of Mrs. Bennet,
and they gave way only to the greater distress of Mr.
Bingley’s continued absence.
%%128%%

Neither Jane nor Elizabeth were comfortable on this
subject. Day after day passed away without bringing
any other tidings of him than the report which shortly
prevailed in Meryton of his coming no more to Netherfield
the whole winter; a report which highly incensed Mrs.
Bennet, and which she never failed to contradict as a most
scandalous falsehood.

Even Elizabeth began to fear -- not that Bingley was
indifferent -- but that his sisters would be successful in
keeping him away. Unwilling as she was to admit an
idea so destructive of Jane’s happiness, and so dishonourable
to the stability of her lover, she could not prevent
its frequently recurring. The united efforts of his two
unfeeling sisters and of his overpowering friend, assisted
by the attractions of Miss Darcy and the amusements
of London, might be too much, she feared, for the strength
of his attachment.

As for Jane, \textit{her} anxiety under this suspence was, of
course, more painful than Elizabeth’s; but whatever she
felt she was desirous of concealing, and between herself
and Elizabeth, therefore, the subject was never alluded to.
But as no such delicacy restrained her mother, an hour
seldom passed in which she did not talk of Bingley, express
her impatience for his arrival, or even require Jane to
confess that if he did not come back, she should think
herself very ill used. It needed all Jane’s steady mildness
to bear these attacks with tolerable tranquillity.

Mr. Collins returned most punctually on the Monday
fortnight, but his reception at Longbourn was not quite
so gracious as it had been on his first introduction. He
was too happy, however, to need much attention; and
luckily for the others, the business of love-making relieved
them from a great deal of his company. The chief of
every day was spent by him at Lucas Lodge, and he
sometimes returned to Longbourn only in time to make
an apology for his absence before the family went to bed.

Mrs. Bennet was really in a most pitiable state. The
very mention of any thing concerning the match threw
%%129%%
her into an agony of ill humour, and wherever she went
she was sure of hearing it talked of. The sight of Miss
Lucas was odious to her. As her successor in that house,
she regarded her with jealous abhorrence. Whenever
Charlotte came to see them she concluded her to be
anticipating the hour of possession; and whenever she
spoke in a low voice to Mr. Collins, was convinced that
they were talking of the Longbourn estate, and resolving
to turn herself and her daughters out of the house, as soon
as Mr. Bennet were dead. She complained bitterly of all
this to her husband.

“Indeed, Mr. Bennet,” said she, “it is very hard to
think that Charlotte Lucas should ever be mistress of
this house, that I should be forced to make way for \textit{her},
and live to see her take my place in it!”

“My dear, do not give way to such gloomy thoughts.
Let us hope for better things. Let us flatter ourselves
that \textit{I} may be the survivor.”

This was not very consoling to Mrs. Bennet, and, therefore,
instead of making any answer, she went on as before,

“I cannot bear to think that they should have all
this estate. If it was not for the entail I should not
mind it.”

“What should not you mind?”

“I should not mind any thing at all.”

“Let us be thankful that you are preserved from a state
of such insensibility.”

“I never can be thankful, Mr. Bennet, for any thing
about the entail. How any one could have the conscience
to entail away an estate from one’s own daughters
I cannot understand; and all for the sake of Mr. Collins
too! -- Why should \textit{he} have it more than anybody else?”

“I leave it to yourself to determine,” said Mr. Bennet.

%%130%%

\Part{VOL. II.}
%%132%%

\Chapter{CHAPTER I.}

Miss Bingley’s letter arrived, and put an end to doubt.
The very first sentence conveyed the assurance of their
being all settled in London for the winter, and concluded
with her brother’s regret at not having had time to pay
his respects to his friends in Hertfordshire before he left
the country.

Hope was over, entirely over; and when Jane could
attend to the rest of the letter, she found little, except
the professed affection of the writer, that could give her
any comfort. Miss Darcy’s praise occupied the chief of it.
Her many attractions were again dwelt on, and Caroline
boasted joyfully of their increasing intimacy, and ventured
to predict the accomplishment of the wishes which had
been unfolded in her former letter. She wrote also with
great pleasure of her brother’s being an inmate of Mr.
Darcy’s house, and mentioned with raptures, some plans
of the latter with regard to new furniture.

Elizabeth, to whom Jane very soon communicated the
chief of all this, heard it in silent indignation. Her heart
was divided between concern for her sister, and resentment
against all the others. To Caroline’s assertion of her
brother’s being partial to Miss Darcy she paid no credit.
That he was really fond of Jane, she doubted no more than
she had ever done; and much as she had always been
disposed to like him, she could not think without anger,
hardly without contempt, on that easiness of temper, that
want of proper resolution which now made him the slave
of his designing friends, and led him to sacrifice his own
happiness to the caprice of their inclinations. Had his
own happiness, however, been the only sacrifice, he might
%%133%%
have been allowed to sport with it in what ever manner
he thought best; but her sister’s was involved in it, as
she thought he must be sensible himself. It was a subject,
in short, on which reflection would be long indulged, and
must be unavailing. She could think of nothing else, and
yet whether Bingley’s regard had really died away, or
were suppressed by his friends’ interference; whether
he had been aware of Jane’s attachment, or whether it
had escaped his observation; whichever were the case,
though her opinion of him must be materially affected
by the difference, her sister’s situation remained the same,
her peace equally wounded.

A day or two passed before Jane had courage to speak
of her feelings to Elizabeth; but at last on Mrs. Bennet’s
leaving them together, after a longer irritation than usual
about Netherfield and its master, she could not help
saying,

“Oh! that my dear mother had more command over
herself; she can have no idea of the pain she gives me
by her continual reflections on him. But I will not
repine. It cannot last long. He will be forgot, and we
shall all be as we were before.”

Elizabeth looked at her sister with incredulous solicitude,
but said nothing.

“You doubt me,” cried Jane, slightly colouring;
“indeed you have no reason. He may live in my memory
as the most amiable man of my acquaintance, but that
is all. I have nothing either to hope or fear, and nothing
to reproach him with. Thank God! I have not \textit{that}
pain. A little time therefore. -- I shall certainly try to
get the better.”

With a stronger voice she soon added, “I have this
comfort immediately, that it has not been more than an
error of fancy on my side, and that it has done no harm
to any one but myself.”

“My dear Jane!” exclaimed Elizabeth, “you are too
good. Your sweetness and disinterestedness are really
angelic; I do not know what to say to you. I feel as if
%%134%%
I had never done you justice, or loved you as you
deserve.”

Miss Bennet eagerly disclaimed all extraordinary merit,
and threw back the praise on her sister’s warm affection.

“Nay,” said Elizabeth, “this is not fair. \textit{You} wish to
think all the world respectable, and are hurt if I speak ill
of any body. \textit{I} only want to think \textit{you} perfect, and you
set yourself against it. Do not be afraid of my running
into any excess, of my encroaching on your privilege of
universal good will. You need not. There are few people
whom I really love, and still fewer of whom I think well.
The more I see of the world, the more am I dissatisfied
with it; and every day confirms my belief of the inconsistency
of all human characters, and of the little dependence
that can be placed on the appearance of either merit
or sense. I have met with two instances lately; one
I will not mention; the other is Charlotte’s marriage.
It is unaccountable! in every view it is unaccountable!”

“My dear Lizzy, do not give way to such feelings as
these. They will ruin your happiness. You do not make
allowance enough for difference of situation and temper.
Consider Mr. Collins’s respectability, and Charlotte’s
prudent, steady character. Remember that she is one of
a large family; that as to fortune, it is a most eligible
match; and be ready to believe, for every body’s sake,
that she may feel something like regard and esteem for
our cousin.”

“To oblige you, I would try to believe almost any
thing, but no one else could be benefited by such a belief
as this; for were I persuaded that Charlotte had any
regard for him, I should only think worse of her understanding,
than I now do of her heart. My dear Jane,
Mr. Collins is a conceited, pompous, narrow-minded, silly
man; you know he is, as well as I do; and you must
feel, as well as I do, that the woman who marries him,
cannot have a proper way of thinking. You shall not
defend her, though it is Charlotte Lucas. You shall
not, for the sake of one individual, change the meaning
%%135%%
of principle and integrity, nor endeavour to persuade
yourself or me, that selfishness is prudence, and insensibility
of danger, security for happiness.”

“I must think your language too strong in speaking
of both,” replied Jane, “and I hope you will be convinced
of it, by seeing them happy together. But enough of this.
You alluded to something else. You mentioned \textit{two}
instances. I cannot misunderstand you, but I intreat
you, dear Lizzy, not to pain me by thinking \textit{that person}
to blame, and saying your opinion of him is sunk. We
must not be so ready to fancy ourselves intentionally
injured. We must not expect a lively young man to be
always so guarded and circumspect. It is very often
nothing but our own vanity that deceives us. Women
fancy admiration means more than it does.”

“And men take care that they should.”

“If it is designedly done, they cannot be justified;
but I have no idea of there being so much design in the
world as some persons imagine.”

“I am far from attributing any part of Mr. Bingley’s
conduct to design,” said Elizabeth; “but without
scheming to do wrong, or to make others unhappy, there
may be error, and there may be misery. Thoughtlessness,
want of attention to other people’s feelings, and want of
resolution, will do the business.”

“And do you impute it to either of those?”

“Yes; to the last. But if I go on, I shall displease
you by saying what I think of persons you esteem. Stop
me whilst you can.”

“You persist, then, in supposing his sisters influence
him.”

“Yes, in conjunction with his friend.”

“I cannot believe it. Why should they try to influence
him? They can only wish his happiness, and if he is
attached to me, no other woman can secure it.”

“Your first position is false. They may wish many
things besides his happiness; they may wish his increase
of wealth and consequence; they may wish him to marry
%%136%%
a girl who has all the importance of money, great connections,
and pride.”

“Beyond a doubt, they \textit{do} wish him to chuse Miss
Darcy,” replied Jane; “but this may be from better
feelings than you are supposing. They have known her
much longer than they have known me; no wonder if they
love her better. But, whatever may be their own wishes,
it is very unlikely they should have opposed their brother’s.
What sister would think herself at liberty to do it, unless
there were something very objectionable? If they believed
him attached to me, they would not try to part us; if he
were so, they could not succeed. By supposing such an
affection, you make every body acting unnaturally and
wrong, and me most unhappy. Do not distress me by the
idea. I am not ashamed of having been mistaken -- or,
at least, it is slight, it is nothing in comparison of what
I should feel in thinking ill of him or his sisters. Let me
take it in the best light, in the light in which it may be
understood.”

Elizabeth could not oppose such a wish; and from this
time Mr. Bingley’s name was scarcely ever mentioned
between them.

Mrs. Bennet still continued to wonder and repine at
his returning no more, and though a day seldom passed
in which Elizabeth did not account for it clearly, there
seemed little chance of her ever considering it with less
perplexity. Her daughter endeavoured to convince her
of what she did not believe herself, that his attentions
to Jane had been merely the effect of a common and
transient liking, which ceased when he saw her no more;
but though the probability of the statement was admitted
at the time, she had the same story to repeat every day.
Mrs. Bennet’s best comfort was, that Mr. Bingley must
be down again in the summer.

Mr. Bennet treated the matter differently. “So,
Lizzy,” said he one day, “your sister is crossed in love
I find. I congratulate her. Next to being married, a girl
likes to be crossed in love a little now and then. It is
%%137%%
something to think of, and gives her a sort of distinction
among her companions. When is your turn to come?
You will hardly bear to be long outdone by Jane. Now
is your time. Here are officers enough at Meryton to
disappoint all the young ladies in the country. Let
Wickham be \textit{your} man. He is a pleasant fellow, and would
jilt you creditably.”

“Thank you, Sir, but a less agreeable man would satisfy
me. We must not all expect Jane’s good fortune.”

“True,” said Mr. Bennet, “but it is a comfort to think
that, whatever of that kind may befal you, you have an
affectionate mother who will always make the most of it.”

Mr. Wickham’s society was of material service in dispelling
the gloom, which the late perverse occurrences had
thrown on many of the Longbourn family. They saw
him often, and to his other recommendations was now
added that of general unreserve. The whole of what
Elizabeth had already heard, his claims on Mr. Darcy,
and all that he had suffered from him, was now openly
acknowledged and publicly canvassed; and every body
was pleased to think how much they had always disliked
Mr. Darcy before they had known any thing of the matter.

Miss Bennet was the only creature who could suppose
there might be any extenuating circumstances in the case,
unknown to the society of Hertfordshire; her mild and
steady candour always pleaded for allowances, and urged
the possibility of mistakes -- but by everybody else
Mr. Darcy was condemned as the worst of men.
%%138%%

\Chapter{CHAPTER II.}

After a week spent in professions of love and schemes
of felicity, Mr. Collins was called from his amiable Charlotte
by the arrival of Saturday. The pain of separation,
however, might be alleviated on his side, by preparations
for the reception of his bride, as he had reason to hope,
that shortly after his next return into Hertfordshire, the
day would be fixed that was to make him the happiest
of men. He took leave of his relations at Longbourn
with as much solemnity as before; wished his fair cousins
health and happiness again, and promised their father
another letter of thanks.

On the following Monday, Mrs. Bennet had the pleasure
of receiving her brother and his wife, who came as usual
to spend the Christmas at Longbourn. Mr. Gardiner was
a sensible, gentlemanlike man, greatly superior to his
sister, as well by nature as education. The Netherfield
ladies would have had difficulty in believing that a man
who lived by trade, and within view of his own warehouses,
could have been so well bred and agreeable. Mrs. Gardiner,
who was several years younger than Mrs. Bennet and
Mrs. Philips, was an amiable, intelligent, elegant woman,
and a great favourite with all her Longbourn nieces.
Between the two eldest and herself especially, there subsisted
a very particular regard. They had frequently
been staying with her in town.

The first part of Mrs. Gardiner’s business on her arrival,
was to distribute her presents and describe the newest
fashions. When this was done, she had a less active part
to play. It became her turn to listen. Mrs. Bennet had
many grievances to relate, and much to complain of.
They had all been very ill-used since she last saw her
sister. Two of her girls had been on the point of marriage,
and after all there was nothing in it.
%%139%%

“I do not blame Jane,” she continued, “for Jane
would have got Mr. Bingley, if she could. But, Lizzy!
Oh, sister! it is very hard to think that she might have
been Mr. Collins’s wife by this time, had not it been for
her own perverseness. He made her an offer in this very
room, and she refused him. The consequence of it is,
that Lady Lucas will have a daughter married before I
have, and that Longbourn estate is just as much entailed
as ever. The Lucases are very artful people indeed, sister.
They are all for what they can get. I am sorry to say
it of them, but so it is. It makes me very nervous and
poorly, to be thwarted so in my own family, and to have
neighbours who think of themselves before anybody else.
However, your coming just at this time is the greatest
of comforts, and I am very glad to hear what you tell
us, of long sleeves.”

Mrs. Gardiner, to whom the chief of this news had been
given before, in the course of Jane and Elizabeth’s
correspondence with her, made her sister a slight answer, and
in compassion to her nieces turned the conversation.

When alone with Elizabeth afterwards, she spoke more
on the subject. “It seems likely to have been a desirable
match for Jane,” said she. “I am sorry it went off.
But these things happen so often! A young man, such as
you describe Mr. Bingley, so easily falls in love with
a pretty girl for a few weeks, and when accident separates
them, so easily forgets her, that these sort of inconstancies
are very frequent.”

“An excellent consolation in its way,” said Elizabeth,
“but it will not do for \textit{us}. We do not suffer by \textit{accident}.
It does not often happen that the interference of friends
will persuade a young man of independent fortune to
think no more of a girl, whom he was violently in love
with only a few days before.”

“But that expression of ‘violently in love’ is so
hackneyed, so doubtful, so indefinite, that it gives me
very little idea. It is as often applied to feelings which
arise from an half-hour’s acquaintance, as to a real,
%%140%%
strong attachment. Pray, how \textit{violent was} Mr. Bingley’s
love?”

“I never saw a more promising inclination. He was
growing quite inattentive to other people, and wholly
engrossed by her. Every time they met, it was more
decided and remarkable. At his own ball he offended two
or three young ladies, by not asking them to dance, and
I spoke to him twice myself, without receiving an answer.
Could there be finer symptoms? Is not general incivility
the very essence of love?”

“Oh, yes! -- of that kind of love which I suppose him
to have felt. Poor Jane! I am sorry for her, because,
with her disposition, she may not get over it immediately.
It had better have happened to \textit{you}, Lizzy; you would
have laughed yourself out of it sooner. But do you think
she would be prevailed on to go back with us? Change
of scene might be of service -- and perhaps a little relief
from home, may be as useful as anything.”

Elizabeth was exceedingly pleased with this proposal,
and felt persuaded of her sister’s ready acquiescence.

“I hope,” added Mrs. Gardiner, “that no consideration
with regard to this young man will influence her. We live
in so different a part of town, all our connections are so
different, and, as you well know, we go out so little, that
it is very improbable they should meet at all, unless he
really comes to see her.”

“And \textit{that} is quite impossible; for he is now in the
custody of his friend, and Mr. Darcy would no more suffer
him to call on Jane in such a part of London! My dear
aunt, how could you think of it? Mr. Darcy may perhaps
have \textit{heard} of such a place as Gracechurch Street, but he
would hardly think a month’s ablution enough to cleanse
him from its impurities, were he once to enter it; and
depend upon it, Mr. Bingley never stirs without him.”

“So much the better. I hope they will not meet at
all. But does not Jane correspond with the sister? \textit{She}
will not be able to help calling.”

“She will drop the acquaintance entirely.”
%%141%%

But in spite of the certainty in which Elizabeth affected
to place this point, as well as the still more interesting
one of Bingley’s being withheld from seeing Jane, she
felt a solicitude on the subject which convinced her, on
examination, that she did not consider it entirely hopeless.
It was possible, and sometimes she thought it probable,
that his affection might be re-animated, and the influence
of his friends successfully combated by the more natural
influence of Jane’s attractions.

Miss Bennet accepted her aunt’s invitation with
pleasure; and the Bingleys were no otherwise in her
thoughts at the time, than as she hoped that, by Caroline’s
not living in the same house with her brother, she might
occasionally spend a morning with her, without any danger
of seeing him.

The Gardiners staid a week at Longbourn; and what
with the Philipses, the Lucases, and the officers, there
was not a day without its engagement. Mrs. Bennet
had so carefully provided for the entertainment of her
brother and sister, that they did not once sit down to
a family dinner. When the engagement was for home,
some of the officers always made part of it, of which officers
Mr. Wickham was sure to be one; and on these occasions,
Mrs. Gardiner, rendered suspicious by Elizabeth’s warm
commendation of him, narrowly observed them both.
Without supposing them, from what she saw, to be very
seriously in love, their preference of each other was plain
enough to make her a little uneasy; and she resolved to
speak to Elizabeth on the subject before she left Hertfordshire,
and represent to her the imprudence of encouraging
such an attachment.

To Mrs. Gardiner, Wickham had one means of affording
pleasure, unconnected with his general powers. About
ten or a dozen years ago, before her marriage, she had
spent a considerable time in that very part of Derbyshire,
to which he belonged. They had, therefore, many acquaintance
in common; and, though Wickham had been little
there since the death of Darcy’s father, five years before,
%%142%%
it was yet in his power to give her fresher intelligence of
her former friends, than she had been in the way of
procuring.

Mrs. Gardiner had seen Pemberley, and known the late
Mr. Darcy by character perfectly well. Here consequently
was an inexhaustible subject of discourse. In comparing
her recollection of Pemberley, with the minute description
which Wickham could give, and in bestowing her tribute
of praise on the character of its late possessor, she was
delighting both him and herself. On being made acquainted
with the present Mr. Darcy’s treatment of him, she tried
to remember something of that gentleman’s reputed disposition
when quite a lad, which might agree with it, and
was confident at last, that she recollected having heard
Mr. Fitzwilliam Darcy formerly spoken of as a very proud,
ill-natured boy.
%%143%%

\Chapter{CHAPTER III.}

Mrs. Gardiner’s caution to Elizabeth was punctually
and kindly given on the first favourable opportunity of
speaking to her alone; after honestly telling her what
she thought, she thus went on:

“You are too sensible a girl, Lizzy, to fall in love
merely because you are warned against it; and, therefore,
I am not afraid of speaking openly. Seriously, I would
have you be on your guard. Do not involve yourself,
or endeavour to involve him in an affection which the
want of fortune would make so very imprudent. I have
nothing to say against \textit{him}; he is a most interesting
young man; and if he had the fortune he ought to have,
I should think you could not do better. But as it is -- you
must not let your fancy run away with you. You have
sense, and we all expect you to use it. Your father would
depend on \textit{your} resolution and good conduct, I am sure.
You must not disappoint your father.”

“My dear aunt, this is being serious indeed.”

“Yes, and I hope to engage you to be serious
likewise.”

“Well, then, you need not be under any alarm. I will
take care of myself, and of Mr. Wickham too. He shall
not be in love with me, if I can prevent it.”

“Elizabeth, you are not serious now.”

“I beg your pardon. I will try again. At present I am
not in love with Mr. Wickham; no, I certainly am not.
But he is, beyond all comparison, the most agreeable
man I ever saw -- and if he becomes really attached to me -- I
believe it will be better that he should not. I see the
imprudence of it. -- Oh! \textit{that} abominable Mr. Darcy! -- My
father’s opinion of me does me the greatest honor;
and I should be miserable to forfeit it. My father, however,
is partial to Mr. Wickham. In short, my dear aunt,
%%144%%
I should be very sorry to be the means of making any
of you unhappy; but since we see every day that where
there is affection, young people are seldom withheld by
immediate want of fortune, from entering into engagements
with each other, how can I promise to be wiser
than so many of my fellow creatures if I am tempted,
or how am I even to know that it would be wisdom to
resist? All that I can promise you, therefore, is not to
be in a hurry. I will not be in a hurry to believe myself
his first object. When I am in company with him, I will
not be wishing. In short, I will do my best.”

“Perhaps it will be as well, if you discourage his coming
here so very often. At least, you should not \textit{remind} your
Mother of inviting him.”

“As I did the other day,” said Elizabeth, with a conscious
smile; “very true, it will be wise in me to
refrain from \textit{that}. But do not imagine that he is always
here so often. It is on your account that he has been so
frequently invited this week. You know my mother’s
ideas as to the necessity of constant company for her
friends. But really, and upon my honour, I will try to
do what I think to be wisest; and now, I hope you are
satisfied.”

Her aunt assured her that she was; and Elizabeth
having thanked her for the kindness of her hints, they
parted; a wonderful instance of advice being given on
such a point, without being resented.

Mr. Collins returned into Hertfordshire soon after it had
been quitted by the Gardiners and Jane; but as he took
up his abode with the Lucases, his arrival was no great
inconvenience to Mrs. Bennet. His marriage was now
fast approaching, and she was at length so far resigned
as to think it inevitable, and even repeatedly to say in
an ill-natured tone that she “\textit{wished} they might be happy.”
Thursday was to be the wedding day, and on Wednesday
Miss Lucas paid her farewell visit; and when she rose
to take leave, Elizabeth, ashamed of her mother’s
%%145%%
ungracious and reluctant good wishes, and sincerely affected
herself, accompanied her out of the room. As they went
down stairs together, Charlotte said,

“I shall depend on hearing from you very often,
Eliza.”

“\textit{That} you certainly shall.”

“And I have another favour to ask. Will you come
and see me?”

“We shall often meet, I hope, in Hertfordshire.”

“I am not likely to leave Kent for some time. Promise
me, therefore, to come to Hunsford.”

Elizabeth could not refuse, though she foresaw little
pleasure in the visit.

“My father and Maria are to come to me in March,”
added Charlotte, “and I hope you will consent to be of
the party. Indeed, Eliza, you will be as welcome to me
as either of them.”

The wedding took place; the bride and bridegroom set
off for Kent from the church door, and every body had
as much to say or to hear on the subject as usual. Elizabeth
soon heard from her friend; and their correspondence
was as regular and frequent as it had ever been;
that it should be equally unreserved was impossible.
Elizabeth could never address her without feeling that
all the comfort of intimacy was over, and, though
determined not to slacken as a correspondent, it was for
the sake of what had been, rather than what was. Charlotte’s
first letters were received with a good deal of
eagerness; there could not but be curiosity to know how
she would speak of her new home, how she would like
Lady Catherine, and how happy she would dare pronounce
herself to be; though, when the letters were read, Elizabeth
felt that Charlotte expressed herself on every point
exactly as she might have foreseen. She wrote cheerfully,
seemed surrounded with comforts, and mentioned nothing
which she could not praise. The house, furniture, neighbourhood,
and roads, were all to her taste, and Lady
Catherine’s behaviour was most friendly and obliging.
%%146%%
It was Mr. Collins’s picture of Hunsford and Rosings
rationally softened; and Elizabeth perceived that she
must wait for her own visit there, to know the rest.

Jane had already written a few lines to her sister to
announce their safe arrival in London; and when she
wrote again, Elizabeth hoped it would be in her power
to say something of the Bingleys.

Her impatience for this second letter was as well
rewarded as impatience generally is. Jane had been a week
in town, without either seeing or hearing from Caroline.
She accounted for it, however, by supposing that her last
letter to her friend from Longbourn, had by some accident
been lost.

“My aunt,” she continued, “is going to-morrow into
that part of the town, and I shall take the opportunity
of calling in Grosvenor-street.”

She wrote again when the visit was paid, and she had
seen Miss Bingley. “I did not think Caroline in spirits,”
were her words, “but she was very glad to see me, and
reproached me for giving her no notice of my coming to
London. I was right, therefore; my last letter had never
reached her. I enquired after their brother, of course.
He was well, but so much engaged with Mr. Darcy, that
they scarcely ever saw him. I found that Miss Darcy
was expected to dinner. I wish I could see her. My
visit was not long, as Caroline and Mrs. Hurst were going
out. I dare say I shall soon see them here.”

Elizabeth shook her head over this letter. It convinced
her, that accident only could discover to Mr.
Bingley her sister’s being in town.

Four weeks passed away, and Jane saw nothing of him.
She endeavoured to persuade herself that she did not
regret it; but she could no longer be blind to Miss Bingley’s
inattention. After waiting at home every morning for
a fortnight, and inventing every evening a fresh excuse
for her, the visitor did at last appear; but the shortness
of her stay, and yet more, the alteration of her manner,
would allow Jane to deceive herself no longer. The letter
%%147%%
which she wrote on this occasion to her sister, will prove
what she felt.

\begin{letter}
“My dearest Lizzy will, I am sure, be incapable of
triumphing in her better judgment, at my expence, when
I confess myself to have been entirely deceived in Miss
Bingley’s regard for me. But, my dear sister, though
the event has proved you right, do not think me obstinate
if I still assert, that, considering what her behaviour was,
my confidence was as natural as your suspicion. I do
not at all comprehend her reason for wishing to be intimate
with me, but if the same circumstances were to happen
again, I am sure I should be deceived again. Caroline
did not return my visit till yesterday; and not a note,
not a line, did I receive in the mean time. When she did
come, it was very evident that she had no pleasure in it;
she made a slight, formal, apology, for not calling before,
said not a word of wishing to see me again, and was in
every respect so altered a creature, that when she went
away, I was perfectly resolved to continue the acquaintance
no longer. I pity, though I cannot help blaming her.
She was very wrong in singling me out as she did; I can
safely say, that every advance to intimacy began on her
side. But I pity her, because she must feel that she has
been acting wrong, and because I am very sure that
anxiety for her brother is the cause of it. I need not
explain myself farther; and though \textit{we} know this anxiety
to be quite needless, yet if she feels it, it will easily account
for her behaviour to me; and so deservedly dear as he
is to his sister, whatever anxiety she may feel on his
behalf, is natural and amiable. I cannot but wonder,
however, at her having any such fears now, because, if
he had at all cared about me, we must have met long,
long ago. He knows of my being in town, I am certain,
from something she said herself; and yet it should seem
by her manner of talking, as if she wanted to persuade
herself that he is really partial to Miss Darcy. I cannot
understand it. If I were not afraid of judging harshly,
%%148%%
I should be almost tempted to say, that there is a strong
appearance of duplicity in all this. But I will endeavour
to banish every painful thought, and think only of what
will make me happy, your affection, and the invariable
kindness of my dear uncle and aunt. Let me hear from
you very soon. Miss Bingley said something of his never
returning to Netherfield again, of giving up the house,
but not with any certainty. We had better not mention it.
I am extremely glad that you have such pleasant accounts
from our friends at Hunsford. Pray go to see them, with
Sir William and Maria. I am sure you will be very
comfortable there.

\raggedleft “Your’s, \&c.”
\end{letter}

This letter gave Elizabeth some pain; but her spirits
returned as she considered that Jane would no longer be
duped, by the sister at least. All expectation from the
brother was now absolutely over. She would not even
wish for any renewal of his attentions. His character
sunk on every review of it; and as a punishment for
him, as well as a possible advantage to Jane, she seriously
hoped he might really soon marry Mr. Darcy’s sister, as,
by Wickham’s account, she would make him abundantly
regret what he had thrown away.

Mrs. Gardiner about this time reminded Elizabeth of
her promise concerning that gentleman, and required
information; and Elizabeth had such to send as might
rather give contentment to her aunt than to herself.
His apparent partiality had subsided, his attentions were
over, he was the admirer of some one else. Elizabeth
was watchful enough to see it all, but she could see it
and write of it without material pain. Her heart had
been but slightly touched, and her vanity was satisfied
with believing that \textit{she} would have been his only choice,
had fortune permitted it. The sudden acquisition of ten
thousand pounds was the most remarkable charm of the
young lady, to whom he was now rendering himself agreeable;
but Elizabeth, less clear-sighted perhaps in his
%%149%%
case than in Charlotte’s, did not quarrel with him for his
wish of independence. Nothing, on the contrary, could
be more natural; and while able to suppose that it cost
him a few struggles to relinquish her, she was ready to
allow it a wise and desirable measure for both, and could
very sincerely wish him happy.

All this was acknowledged to Mrs. Gardiner; and after
relating the circumstances, she thus went on:-- “I am
now convinced, my dear aunt, that I have never been much
in love; for had I really experienced that pure and
elevating passion, I should at present detest his very
name, and wish him all manner of evil. But my feelings
are not only cordial towards \textit{him}; they are even impartial
towards Miss King. I cannot find out that I hate her
at all, or that I am in the least unwilling to think her
a very good sort of girl. There can be no love in all this.
My watchfulness has been effectual; and though I should
certainly be a more interesting object to all my acquaintance,
were I distractedly in love with him, I cannot say
that I regret my comparative insignificance. Importance
may sometimes be purchased too dearly. Kitty and Lydia
take his defection much more to heart than I do. They are
young in the ways of the world, and not yet open to the
mortifying conviction that handsome young men must
have something to live on, as well as the plain.”
%%150%%

\Chapter{CHAPTER IV.}

With no greater events than these in the Longbourn
family, and otherwise diversified by little beyond the walks
to Meryton, sometimes dirty and sometimes cold, did
January and February pass away. March was to take
Elizabeth to Hunsford. She had not at first thought very
seriously of going thither; but Charlotte, she soon found,
was depending on the plan, and she gradually learned to
consider it herself with greater pleasure as well as greater
certainty. Absence had increased her desire of seeing
Charlotte again, and weakened her disgust of Mr. Collins.
There was novelty in the scheme, and as, with such a
mother and such uncompanionable sisters, home could
not be faultless, a little change was not unwelcome for
its own sake. The journey would moreover give her a peep
at Jane; and, in short, as the time drew near, she would
have been very sorry for any delay. Every thing, however,
went on smoothly, and was finally settled according to
Charlotte’s first sketch. She was to accompany Sir William
and his second daughter. The improvement of spending
a night in London was added in time, and the plan became
perfect as plan could be.

The only pain was in leaving her father, who would
certainly miss her, and who, when it came to the point,
so little liked her going, that he told her to write to him,
and almost promised to answer her letter.

The farewell between herself and Mr. Wickham was
perfectly friendly; on his side even more. His present
pursuit could not make him forget that Elizabeth had
been the first to excite and to deserve his attention, the
first to listen and to pity, the first to be admired; and
in his manner of bidding her adieu, wishing her every
enjoyment, reminding her of what she was to expect in
Lady Catherine de Bourgh, and trusting their opinion of
%%151%%
her -- their opinion of every body -- would always coincide,
there was a solicitude, an interest which she felt must
ever attach her to him with a most sincere regard; and
she parted from him convinced, that whether married or
single, he must always be her model of the amiable and
pleasing.

Her fellow-travellers the next day, were not of a kind
to make her think him less agreeable. Sir William Lucas,
and his daughter Maria, a good humoured girl, but as
empty-headed as himself, had nothing to say that could
be worth hearing, and were listened to with about as much
delight as the rattle of the chaise. Elizabeth loved
absurdities, but she had known Sir William’s too long.
He could tell her nothing new of the wonders of his
presentation and knighthood; and his civilities were worn
out like his information.

It was a journey of only twenty-four miles, and they
began it so early as to be in Gracechurch-street by noon.
As they drove to Mr. Gardiner’s door, Jane was at a
drawing-room window watching their arrival; when they
entered the passage she was there to welcome them, and
Elizabeth, looking earnestly in her face, was pleased to see
it healthful and lovely as ever. On the stairs were a troop of
little boys and girls, whose eagerness for their cousin’s
appearance would not allow them to wait in the drawing-room,
and whose shyness, as they had not seen her for a
twelvemonth, prevented their coming lower. All was joy
and kindness. The day passed most pleasantly away;
the morning in bustle and shopping, and the evening at
one of the theatres.

Elizabeth then contrived to sit by her aunt. Their
first subject was her sister; and she was more grieved
than astonished to hear, in reply to her minute enquiries,
that though Jane always struggled to support her spirits,
there were periods of dejection. It was reasonable, however,
to hope, that they would not continue long. Mrs.
Gardiner gave her the particulars also of Miss Bingley’s
visit in Gracechurch-street, and repeated conversations
%%152%%
occurring at different times between Jane and herself,
which proved that the former had, from her heart, given
up the acquaintance.

Mrs. Gardiner then rallied her niece on Wickham’s
desertion, and complimented her on bearing it so well.

“But, my dear Elizabeth,” she added, “what sort of
girl is Miss King? I should be sorry to think our friend
mercenary.”

“Pray, my dear aunt, what is the difference in matrimonial
affairs, between the mercenary and the prudent
motive? Where does discretion end, and avarice begin?
Last Christmas you were afraid of his marrying me,
because it would be imprudent; and now, because he is
trying to get a girl with only ten thousand pounds, you
want to find out that he is mercenary.”

“If you will only tell me what sort of girl Miss King is,
I shall know what to think.”

“She is a very good kind of girl, I believe. I know no
harm of her.”

“But he paid her not the smallest attention, till her
grandfather’s death made her mistress of this fortune.”

“No -- why should he? If it was not allowable for
him to gain \textit{my} affections, because I had no money, what
occasion could there be for making love to a girl whom
he did not care about, and who was equally poor?”

“But there seems indelicacy in directing his attentions
towards her, so soon after this event.”

“A man in distressed circumstances has not time for
all those elegant decorums which other people may
observe. If \textit{she} does not object to it, why should \textit{we?}”

“\textit{Her} not objecting, does not justify \textit{him}. It only
shews her being deficient in something herself -- sense or
feeling.”

“Well,” cried Elizabeth, “have it as you choose. \textit{He}
shall be mercenary, and \textit{she} shall be foolish.”

“No, Lizzy, that is what I do \textit{not} choose. I should be
sorry, you know, to think ill of a young man who has lived
so long in Derbyshire.”
%%153%%

“Oh! if that is all, I have a very poor opinion of young
men who live in Derbyshire; and their intimate friends
who live in Hertfordshire are not much better. I am sick
of them all. Thank Heaven! I am going to-morrow where
I shall find a man who has not one agreeable quality, who
has neither manner nor sense to recommend him. Stupid
men are the only ones worth knowing, after all.”

“Take care, Lizzy; that speech savours strongly of
disappointment.”

Before they were separated by the conclusion of the
play, she had the unexpected happiness of an invitation
to accompany her uncle and aunt in a tour of pleasure
which they proposed taking in the summer.

“We have not quite determined how far it shall carry
us,” said Mrs. Gardiner, “but perhaps to the Lakes.”

No scheme could have been more agreeable to Elizabeth,
and her acceptance of the invitation was most ready and
grateful. “My dear, dear aunt,” she rapturously cried,
“what delight! what felicity! You give me fresh life
and vigour. Adieu to disappointment and spleen. What
are men to rocks and mountains? Oh! what hours of
transport we shall spend! And when we \textit{do} return, it shall
not be like other travellers, without being able to give
one accurate idea of any thing. We \textit{will} know where we
have gone -- we \textit{will} recollect what we have seen. Lakes,
mountains, and rivers, shall not be jumbled together in our
imaginations; nor, when we attempt to describe any
particular scene, will we begin quarrelling about its relative
situation. Let \textit{our} first effusions be less insupportable
than those of the generality of travellers.”
%%154%%

\Chapter{CHAPTER V.}

Every object in the next day’s journey was new and
interesting to Elizabeth; and her spirits were in a state
for enjoyment; for she had seen her sister looking so well
as to banish all fear for her health, and the prospect of her
northern tour was a constant source of delight.

When they left the high road for the lane to Hunsford,
every eye was in search of the Parsonage, and every
turning expected to bring it in view. The paling of Rosings
Park was their boundary on one side. Elizabeth smiled
at the recollection of all that she had heard of its
inhabitants.

At length the Parsonage was discernible. The garden
sloping to the road, the house standing in it, the green
pales and the laurel hedge, every thing declared they
were arriving. Mr. Collins and Charlotte appeared at the
door, and the carriage stopped at the small gate, which
led by a short gravel walk to the house, amidst the nods
and smiles of the whole party. In a moment they were
all out of the chaise, rejoicing at the sight of each other.
Mrs. Collins welcomed her friend with the liveliest pleasure,
and Elizabeth was more and more satisfied with coming,
when she found herself so affectionately received. She
saw instantly that her cousin’s manners were not altered
by his marriage; his formal civility was just what it had
been, and he detained her some minutes at the gate to
hear and satisfy his enquiries after all her family. They
were then, with no other delay than his pointing out the
neatness of the entrance, taken into the house; and as
soon as they were in the parlour, he welcomed them a
second time with ostentatious formality to his humble
abode, and punctually repeated all his wife’s offers of
refreshment.

Elizabeth was prepared to see him in his glory; and
%%155%%
she could not help fancying that in displaying the good
proportion of the room, its aspect and its furniture, he
addressed himself particularly to her, as if wishing to
make her feel what she had lost in refusing him. But
though every thing seemed neat and comfortable, she
was not able to gratify him by any sigh of repentance;
and rather looked with wonder at her friend that she
could have so cheerful an air, with such a companion.
When Mr. Collins said any thing of which his wife might
reasonably be ashamed, which certainly was not unseldom,
she involuntarily turned her eye on Charlotte. Once or
twice she could discern a faint blush; but in general
Charlotte wisely did not hear. After sitting long enough
to admire every article of furniture in the room, from the
sideboard to the fender, to give an account of their journey
and of all that had happened in London, Mr. Collins
invited them to take a stroll in the garden, which was large
and well laid out, and to the cultivation of which he
attended himself. To work in his garden was one of his
most respectable pleasures; and Elizabeth admired the
command of countenance with which Charlotte talked of
the healthfulness of the exercise, and owned she encouraged
it as much as possible. Here, leading the way through
every walk and cross walk, and scarcely allowing them an
interval to utter the praises he asked for, every view was
pointed out with a minuteness which left beauty entirely
behind. He could number the fields in every direction,
and could tell how many trees there were in the most
distant clump. But of all the views which his garden,
or which the country, or the kingdom could boast, none
were to be compared with the prospect of Rosings, afforded
by an opening in the trees that bordered the park nearly
opposite the front of his house. It was a handsome
modern building, well situated on rising ground.

From his garden, Mr. Collins would have led them round
his two meadows, but the ladies not having shoes to
encounter the remains of a white frost, turned back; and
while Sir William accompanied him, Charlotte took her
%%156%%
sister and friend over the house, extremely well pleased,
probably, to have the opportunity of shewing it without
her husband’s help. It was rather small, but well built
and convenient; and every thing was fitted up and
arranged with a neatness and consistency of which Elizabeth
gave Charlotte all the credit. When Mr. Collins
could be forgotten, there was really a great air of comfort
throughout, and by Charlotte’s evident enjoyment of it,
Elizabeth supposed he must be often forgotten.

She had already learnt that Lady Catherine was still
in the country. It was spoken of again while they were
at dinner, when Mr. Collins joining in, observed,

“Yes, Miss Elizabeth, you will have the honour of
seeing Lady Catherine de Bourgh on the ensuing Sunday
at church, and I need not say you will be delighted with
her. She is all affability and condescension, and I doubt not
but you will be honoured with some portion of her notice
when service is over. I have scarcely any hesitation in saying
that she will include you and my sister Maria in every
invitation with which she honours us during your stay
here. Her behaviour to my dear Charlotte is charming.
We dine at Rosings twice every week, and are never
allowed to walk home. Her ladyship’s carriage is regularly
ordered for us. I \textit{should} say, one of her ladyship’s carriages,
for she has several.”

“Lady Catherine is a very respectable, sensible woman
indeed,” added Charlotte, “and a most attentive
neighbour.”

“Very true, my dear, that is exactly what I say. She
is the sort of woman whom one cannot regard with too
much deference.”

The evening was spent chiefly in talking over Hertfordshire
news, and telling again what had been already
written; and when it closed, Elizabeth in the solitude
of her chamber had to meditate upon Charlotte’s degree
of contentment, to understand her address in guiding,
and composure in bearing with her husband, and to
acknowledge that it was all done very well. She had also
%%157%%
to anticipate how her visit would pass, the quiet tenor
of their usual employments, the vexatious interruptions
of Mr. Collins, and the gaieties of their intercourse with
Rosings. A lively imagination soon settled it all.

About the middle of the next day, as she was in her
room getting ready for a walk, a sudden noise below
seemed to speak the whole house in confusion; and after
listening a moment, she heard somebody running up stairs
in a violent hurry, and calling loudly after her. She
opened the door, and met Maria in the landing place, who,
breathless with agitation, cried out,

“Oh, my dear Eliza! pray make haste and come
into the dining-room, for there is such a sight to be seen!
I will not tell you what it is. Make haste, and come down
this moment.”

Elizabeth asked questions in vain; Maria would tell
her nothing more, and down they ran into the dining-room,
which fronted the lane, in quest of this wonder;
it was two ladies stopping in a low phaeton at the garden
gate.

“And is this all?” cried Elizabeth. “I expected at
least that the pigs were got into the garden, and here is
nothing but Lady Catherine and her daughter!”

“La! my dear,” said Maria quite shocked at the
mistake, “it is not Lady Catherine. The old lady is
Mrs. Jenkinson, who lives with them. The other is Miss
De Bourgh. Only look at her. She is quite a little
creature. Who would have thought she could be so thin
and small!”

“She is abominably rude to keep Charlotte out of doors
in all this wind. Why does she not come in?”

“Oh! Charlotte says, she hardly ever does. It is the
greatest of favours when Miss De Bourgh comes in.”

“I like her appearance,” said Elizabeth, struck with
other ideas. “She looks sickly and cross. -- Yes, she will
do for him very well. She will make him a very proper
wife.”

Mr. Collins and Charlotte were both standing at the
%%158%%
gate in conversation with the ladies; and Sir William,
to Elizabeth’s high diversion, was stationed in the doorway,
in earnest contemplation of the greatness before
him, and constantly bowing whenever Miss De Bourgh
looked that way.

At length there was nothing more to be said; the
ladies drove on, and the others returned into the house.
Mr. Collins no sooner saw the two girls than he began
to congratulate them on their good fortune, which Charlotte
explained by letting them know that the whole
party was asked to dine at Rosings the next day.
%%159%%

\Chapter{CHAPTER VI.}

Mr. Collins’s triumph in consequence of this invitation
was complete. The power of displaying the grandeur of
his patroness to his wondering visitors, and of letting them
see her civility towards himself and his wife, was exactly
what he had wished for; and that an opportunity of
doing it should be given so soon, was such an instance of
Lady Catherine’s condescension as he knew not how to
admire enough.

“I confess,” said he, “that I should not have been at
all surprised by her Ladyship’s asking us on Sunday to
drink tea and spend the evening at Rosings. I rather
expected, from my knowledge of her affability, that it
would happen. But who could have foreseen such an
attention as this? Who could have imagined that we
should receive an invitation to dine there (an invitation
moreover including the whole party) so immediately after
your arrival!”

“I am the less surprised at what has happened,”
replied Sir William, “from that knowledge of what the
manners of the great really are, which my situation in
life has allowed me to acquire. About the Court, such
instances of elegant breeding are not uncommon.”

Scarcely any thing was talked of the whole day or next
morning, but their visit to Rosings. Mr. Collins was
carefully instructing them in what they were to expect,
that the sight of such rooms, so many servants, and so
splendid a dinner might not wholly overpower them.

When the ladies were separating for the toilette, he said
to Elizabeth,

“Do not make yourself uneasy, my dear cousin, about
your apparel. Lady Catherine is far from requiring that
elegance of dress in us, which becomes herself and daughter.
I would advise you merely to put on whatever of your
%%160%%
clothes is superior to the rest, there is no occasion for
any thing more. Lady Catherine will not think the worse
of you for being simply dressed. She likes to have the
distinction of rank preserved.”

While they were dressing, he came two or three times
to their different doors, to recommend their being quick,
as Lady Catherine very much objected to be kept waiting
for her dinner. -- Such formidable accounts of her Ladyship,
and her manner of living, quite frightened Maria Lucas,
who had been little used to company, and she looked
forward to her introduction at Rosings, with as much
apprehension, as her father had done to his presentation
at St. James’s.

As the weather was fine, they had a pleasant walk of
about half a mile across the park. -- Every park has its
beauty and its prospects; and Elizabeth saw much to
be pleased with, though she could not be in such raptures
as Mr. Collins expected the scene to inspire, and was but
slightly affected by his enumeration of the windows in
front of the house, and his relation of what the glazing
altogether had originally cost Sir Lewis De Bourgh.

When they ascended the steps to the hall, Maria’s
alarm was every moment increasing, and even Sir William
did not look perfectly calm. -- Elizabeth’s courage did not
fail her. She had heard nothing of Lady Catherine that
spoke her awful from any extraordinary talents or miraculous
virtue, and the mere stateliness of money and rank,
she thought she could witness without trepidation.

From the entrance hall, of which Mr. Collins pointed
out, with a rapturous air, the fine proportion and finished
ornaments, they followed the servants through an anti-chamber,
to the room where Lady Catherine, her daughter,
and Mrs. Jenkinson were sitting. -- Her Ladyship, with
great condescension, arose to receive them; and as Mrs.
Collins had settled it with her husband that the office of
introduction should be her’s, it was performed in a proper
manner, without any of those apologies and thanks which
he would have thought necessary.
%%161%%

In spite of having been at St. James’s, Sir William was
so completely awed, by the grandeur surrounding him, that
he had but just courage enough to make a very low bow,
and take his seat without saying a word; and his daughter,
frightened almost out of her senses, sat on the edge of her
chair, not knowing which way to look. Elizabeth found
herself quite equal to the scene, and could observe the three
ladies before her composedly. -- Lady Catherine was a tall,
large woman, with strongly-marked features, which might
once have been handsome. Her air was not conciliating, nor
was her manner of receiving them, such as to make her
visitors forget their inferior rank. She was not rendered
formidable by silence; but whatever she said, was spoken
in so authoritative a tone, as marked her self-importance,
and brought Mr. Wickham immediately to Elizabeth’s
mind; and from the observation of the day altogether,
she believed Lady Catherine to be exactly what he had
represented.

When, after examining the mother, in whose countenance
and deportment she soon found some resemblance
of Mr. Darcy, she turned her eyes on the daughter, she
could almost have joined in Maria’s astonishment, at her
being so thin, and so small. There was neither in figure
nor face, any likeness between the ladies. Miss De Bourgh
was pale and sickly; her features, though not plain, were
insignificant; and she spoke very little, except in a low
voice, to Mrs. Jenkinson, in whose appearance there was
nothing remarkable, and who was entirely engaged in
listening to what she said, and placing a screen in the
proper direction before her eyes.

After sitting a few minutes, they were all sent to one
of the windows, to admire the view, Mr. Collins attending
them to point out its beauties, and Lady Catherine kindly
informing them that it was much better worth looking at
in the summer.

The dinner was exceedingly handsome, and there were
all the servants, and all the articles of plate which Mr.
Collins had promised; and, as he had likewise foretold,
%%162%%
he took his seat at the bottom of the table, by her ladyship’s
desire, and looked as if he felt that life could furnish
nothing greater. -- He carved, and ate, and praised with
delighted alacrity; and every dish was commended, first
by him, and then by Sir William, who was now enough
recovered to echo whatever his son in law said, in a manner
which Elizabeth wondered Lady Catherine could bear.
But Lady Catherine seemed gratified by their excessive
admiration, and gave most gracious smiles, especially when
any dish on the table proved a novelty to them. The
party did not supply much conversation. Elizabeth was
ready to speak whenever there was an opening, but she
was seated between Charlotte and Miss De Bourgh -- the
former of whom was engaged in listening to Lady Catherine,
and the latter said not a word to her all dinner time.
Mrs. Jenkinson was chiefly employed in watching how
little Miss De Bourgh ate, pressing her to try some other
dish, and fearing she were indisposed. Maria thought
speaking out of the question, and the gentlemen did
nothing but eat and admire.

When the ladies returned to the drawing room, there
was little to be done but to hear Lady Catherine talk,
which she did without any intermission till coffee came
in, delivering her opinion on every subject in so decisive
a manner as proved that she was not used to have her
judgment controverted. She enquired into Charlotte’s
domestic concerns familiarly and minutely, and gave her
a great deal of advice, as to the management of them all;
told her how every thing ought to be regulated in so
small a family as her’s, and instructed her as to the care
of her cows and her poultry. Elizabeth found that
nothing was beneath this great Lady’s attention, which
could furnish her with an occasion of dictating to others.
In the intervals of her discourse with Mrs. Collins, she
addressed a variety of questions to Maria and Elizabeth,
but especially to the latter, of whose connections she knew
the least, and who she observed to Mrs. Collins, was a very
genteel, pretty kind of girl. She asked her at different
%%163%%
times, how many sisters she had, whether they were older
or younger than herself, whether any of them were likely
to be married, whether they were handsome, where they
had been educated, what carriage her father kept, and what
had been her mother’s maiden name? -- Elizabeth felt all
the impertinence of her questions, but answered them very
composedly. -- Lady Catherine then observed,

“Your father’s estate is entailed on Mr. Collins, I think.
For your sake,” turning to Charlotte, “I am glad of it;
but otherwise I see no occasion for entailing estates from
the female line. -- It was not thought necessary in Sir
Lewis de Bourgh’s family. -- Do you play and sing, Miss
Bennet?”

“A little.”

“Oh! then -- some time or other we shall be happy to
hear you. Our instrument is a capital one, probably
superior to------ You shall try it some day. -- Do your sisters
play and sing?”

“One of them does.”

“Why did not you all learn? -- You ought all to have
learned. The Miss Webbs all play, and their father has
not so good an income as your’s. -- Do you draw?”

“No, not at all.”

“What, none of you?”

“Not one.”

“That is very strange. But I suppose you had no
opportunity. Your mother should have taken you to
town every spring for the benefit of masters.”

“My mother would have had no objection, but my
father hates London.”

“Has your governess left you?”

“We never had any governess.”

“No governess! How was that possible? Five
daughters brought up at home without a governess! -- I
never heard of such a thing. Your mother must have
been quite a slave to your education.”

Elizabeth could hardly help smiling, as she assured her
that had not been the case.
%%164%%

“Then, who taught you? who attended to you?
Without a governess you must have been neglected.”

“Compared with some families, I believe we were;
but such of us as wished to learn, never wanted the
means. We were always encouraged to read, and had
all the masters that were necessary. Those who chose
to be idle, certainly might.”

“Aye, no doubt; but that is what a governess will
prevent, and if I had known your mother, I should have
advised her most strenuously to engage one. I always
say that nothing is to be done in education without steady
and regular instruction, and nobody but a governess can
give it. It is wonderful how many families I have been
the means of supplying in that way. I am always glad
to get a young person well placed out. Four nieces of
Mrs. Jenkinson are most delightfully situated through my
means; and it was but the other day, that I recommended
another young person, who was merely accidentally mentioned
to me, and the family are quite delighted with her.
Mrs. Collins, did I tell you of Lady Metcalfe’s calling
yesterday to thank me? She finds Miss Pope a treasure.
‘Lady Catherine,’ said she, ‘you have given me a treasure.’
Are any of your younger sisters out, Miss Bennet?”

“Yes, Ma’am, all.”

“All! -- What, all five out at once? Very odd! -- And
you only the second. -- The younger ones out before the
elder are married! -- Your younger sisters must be very
young?”

“Yes, my youngest is not sixteen. Perhaps \textit{she} is full
young to be much in company. But really, Ma’am,
I think it would be very hard upon younger sisters, that
they should not have their share of society and amusement
because the elder may not have the means or inclination
to marry early. -- The last born has as good a right to the
pleasures of youth, as the first. And to be kept back on
\textit{such} a motive! -- I think it would not be very likely to
promote sisterly affection or delicacy of mind.”

“Upon my word,” said her Ladyship, “you give your
%%165%%
opinion very decidedly for so young a person. -- Pray, what
is your age?”

“With three younger sisters grown up,” replied Elizabeth
smiling, “your Ladyship can hardly expect me to own
it.”

Lady Catherine seemed quite astonished at not receiving
a direct answer; and Elizabeth suspected herself to be
the first creature who had ever dared to trifle with so
much dignified impertinence.

“You cannot be more than twenty, I am sure, -- therefore
you need not conceal your age.”

“I am not one and twenty.”

When the gentlemen had joined them, and tea was over,
the card tables were placed. Lady Catherine, Sir William,
and Mr. and Mrs. Collins sat down to quadrille; and as
Miss De Bourgh chose to play at cassino, the two girls
had the honour of assisting Mrs. Jenkinson to make up
her party. Their table was superlatively stupid. Scarcely
a syllable was uttered that did not relate to the game,
except when Mrs. Jenkinson expressed her fears of Miss
De Bourgh’s being too hot or too cold, or having too much
or too little light. A great deal more passed at the other
table. Lady Catherine was generally speaking -- stating
the mistakes of the three others, or relating some anecdote
of herself. Mr. Collins was employed in agreeing to
every thing her Ladyship said, thanking her for every
fish he won, and apologising if he thought he won too
many. Sir William did not say much. He was storing
his memory with anecdotes and noble names.

When Lady Catherine and her daughter had played as
long as they chose, the tables were broke up, the carriage
was offered to Mrs. Collins, gratefully accepted, and
immediately ordered. The party then gathered round the
fire to hear Lady Catherine determine what weather they
were to have on the morrow. From these instructions
they were summoned by the arrival of the coach, and
with many speeches of thankfulness on Mr. Collins’s side,
and as many bows on Sir William’s, they departed. As
%%166%%
soon as they had driven from the door, Elizabeth was
called on by her cousin, to give her opinion of all that she
had seen at Rosings, which, for Charlotte’s sake, she made
more favourable than it really was. But her commendation,
though costing her some trouble, could by no means
satisfy Mr. Collins, and he was very soon obliged to take
her Ladyship’s praise into his own hands.
%%167%%

\Chapter{CHAPTER VII.}

Sir William staid only a week at Hunsford; but his
visit was long enough to convince him of his daughter’s
being most comfortably settled, and of her possessing such
a husband and such a neighbour as were not often met
with. While Sir William was with them, Mr. Collins
devoted his mornings to driving him out in his gig, and
shewing him the country; but when he went away, the
whole family returned to their usual employments, and
Elizabeth was thankful to find that they did not see more
of her cousin by the alteration, for the chief of the time
between breakfast and dinner was now passed by him
either at work in the garden, or in reading and writing, and
looking out of window in his own book room, which fronted
the road. The room in which the ladies sat was backwards.
Elizabeth at first had rather wondered that Charlotte should
not prefer the dining parlour for common use; it was
a better sized room, and had a pleasanter aspect; but
she soon saw that her friend had an excellent reason for
what she did, for Mr. Collins would undoubtedly have been
much less in his own apartment, had they sat in one
equally lively; and she gave Charlotte credit for the
arrangement.

From the drawing room they could distinguish nothing
in the lane, and were indebted to Mr. Collins for the
knowledge of what carriages went along, and how often
especially Miss De Bourgh drove by in her phaeton, which
he never failed coming to inform them of, though it happened
almost every day. She not unfrequently stopped
at the Parsonage, and had a few minutes’ conversation
with Charlotte, but was scarcely ever prevailed on to
get out.

Very few days passed in which Mr. Collins did not walk
to Rosings, and not many in which his wife did not think
%%168%%
it necessary to go likewise; and till Elizabeth recollected
that there might be other family livings to be disposed of,
she could not understand the sacrifice of so many hours.
Now and then, they were honoured with a call from her
Ladyship, and nothing escaped her observation that was
passing in the room during these visits. She examined
into their employments, looked at their work, and advised
them to do it differently; found fault with the arrangement
of the furniture, or detected the housemaid in
negligence; and if she accepted any refreshment, seemed
to do it only for the sake of finding out that Mrs. Collins’s
joints of meat were too large for her family.

Elizabeth soon perceived that though this great lady
was not in the commission of the peace for the county,
she was a most active magistrate in her own parish, the
minutest concerns of which were carried to her by Mr.
Collins; and whenever any of the cottagers were disposed
to be quarrelsome, discontented or too poor, she sallied
forth into the village to settle their differences, silence
their complaints, and scold them into harmony and plenty.

The entertainment of dining at Rosings was repeated
about twice a week; and, allowing for the loss of Sir
William, and there being only one card table in the
evening, every such entertainment was the counterpart
of the first. Their other engagements were few; as the
style of living of the neighbourhood in general, was
beyond the Collinses’ reach. This however was no evil
to Elizabeth, and upon the whole she spent her time
comfortably enough; there were half hours of pleasant
conversation with Charlotte, and the weather was so fine
for the time of year, that she had often great enjoyment
out of doors. Her favourite walk, and where she frequently
went while the others were calling on Lady
Catherine, was along the open grove which edged that
side of the park, where there was a nice sheltered path,
which no one seemed to value but herself, and where she
felt beyond the reach of Lady Catherine’s curiosity.

In this quiet way, the first fortnight of her visit soon
%%169%%
passed away. Easter was approaching, and the week
preceding it, was to bring an addition to the family at
Rosings, which in so small a circle must be important.
Elizabeth had heard soon after her arrival, that Mr. Darcy
was expected there in the course of a few weeks, and
though there were not many of her acquaintance whom
she did not prefer, his coming would furnish one comparatively
new to look at in their Rosings parties, and she
might be amused in seeing how hopeless Miss Bingley’s
designs on him were, by his behaviour to his cousin, for
whom he was evidently destined by Lady Catherine;
who talked of his coming with the greatest satisfaction,
spoke of him in terms of the highest admiration, and
seemed almost angry to find that he had already been
frequently seen by Miss Lucas and herself.

His arrival was soon known at the Parsonage, for
Mr. Collins was walking the whole morning within view
of the lodges opening into Hunsford Lane, in order to
have the earliest assurance of it; and after making his
bow as the carriage turned into the Park, hurried home
with the great intelligence. On the following morning
he hastened to Rosings to pay his respects. There were
two nephews of Lady Catherine to require them, for
Mr. Darcy had brought with him a Colonel Fitzwilliam,
the younger son of his uncle, Lord ------ and to the great
surprise of all the party, when Mr. Collins returned the
gentlemen accompanied him. Charlotte had seen them
from her husband’s room, crossing the road, and immediately
running into the other, told the girls what an
honour they might expect, adding,

“I may thank you, Eliza, for this piece of civility.
Mr. Darcy would never have come so soon to wait upon
me.”

Elizabeth had scarcely time to disclaim all right to
the compliment, before their approach was announced by
the door-bell, and shortly afterwards the three gentlemen
entered the room. Colonel Fitzwilliam, who led the way,
was about thirty, not handsome, but in person and address
%%170%%
most truly the gentleman. Mr. Darcy looked just as he
had been used to look in Hertfordshire, paid his compliments,
with his usual reserve, to Mrs. Collins; and
whatever might be his feelings towards her friend, met
her with every appearance of composure. Elizabeth
merely curtseyed to him, without saying a word.

Colonel Fitzwilliam entered into conversation directly
with the readiness and ease of a well-bred man, and talked
very pleasantly; but his cousin, after having addressed
a slight observation on the house and garden to Mrs.
Collins, sat for some time without speaking to any body.
At length, however, his civility was so far awakened as
to enquire of Elizabeth after the health of her family.
She answered him in the usual way, and after a moment’s
pause, added,

“My eldest sister has been in town these three months.
Have you never happened to see her there?”

She was perfectly sensible that he never had; but she
wished to see whether he would betray any consciousness
of what had passed between the Bingleys and Jane; and
she thought he looked a little confused as he answered
that he had never been so fortunate as to meet Miss
Bennet. The subject was pursued no farther, and the
gentlemen soon afterwards went away.
%%171%%

\Chapter{CHAPTER VIII.}

Colonel Fitzwilliam’s manners were very much admired
at the parsonage, and the ladies all felt that he
must add considerably to the pleasure of their engagements
at Rosings. It was some days, however, before
they received any invitation thither, for while there were
visitors in the house, they could not be necessary; and
it was not till Easter-day, almost a week after the gentlemen’s
arrival, that they were honoured by such an attention,
and then they were merely asked on leaving church
to come there in the evening. For the last week they had
seen very little of either Lady Catherine or her daughter.
Colonel Fitzwilliam had called at the parsonage more than
once during the time, but Mr. Darcy they had only seen
at church.

The invitation was accepted of course, and at a proper
hour they joined the party in Lady Catherine’s drawing
room. Her ladyship received them civilly, but it was
plain that their company was by no means so acceptable
as when she could get nobody else; and she was, in fact,
almost engrossed by her nephews, speaking to them,
especially to Darcy, much more than to any other person
in the room.

Colonel Fitzwilliam seemed really glad to see them;
any thing was a welcome relief to him at Rosings; and
Mrs. Collins’s pretty friend had moreover caught his fancy
very much. He now seated himself by her, and talked
so agreeably of Kent and Hertfordshire, of travelling and
staying at home, of new books and music, that Elizabeth
had never been half so well entertained in that room before;
and they conversed with so much spirit and flow, as to
draw the attention of Lady Catherine herself, as well as
of Mr. Darcy. \textit{His} eyes had been soon and repeatedly
turned towards them with a look of curiosity; and that
%%172%%
her ladyship after a while shared the feeling, was more
openly acknowledged, for she did not scruple to call out,

“What is that you are saying, Fitzwilliam? What is
it you are talking of? What are you telling Miss Bennet?
Let me hear what it is.”

“We are speaking of music, Madam,” said he, when
no longer able to avoid a reply.

“Of music! Then pray speak aloud. It is of all
subjects my delight. I must have my share in the conversation,
if you are speaking of music. There are few people
in England, I suppose, who have more true enjoyment
of music than myself, or a better natural taste.
If I had ever learnt, I should have been a great proficient.
And so would Anne, if her health had allowed her to apply.
I am confident that she would have performed delightfully.
How does Georgiana get on, Darcy?”

Mr. Darcy spoke with affectionate praise of his sister’s
proficiency.

“I am very glad to hear such a good account of her,”
said Lady Catherine; “and pray tell her from me, that
she cannot expect to excel, if she does not practise a great
deal.”

“I assure you, Madam,” he replied, “that she does
not need such advice. She practises very constantly.”

“So much the better. It cannot be done too much;
and when I next write to her, I shall charge her not to
neglect it on any account. I often tell young ladies,
that no excellence in music is to be acquired, without
constant practice. I have told Miss Bennet several times,
that she will never play really well, unless she practises
more; and though Mrs. Collins has no instrument, she
is very welcome, as I have often told her, to come to
Rosings every day, and play on the piano forte in Mrs.
Jenkinson’s room. She would be in nobody’s way, you
know, in that part of the house.”

Mr. Darcy looked a little ashamed of his aunt’s ill
breeding, and made no answer.

When coffee was over, Colonel Fitzwilliam reminded
%%173%%
Elizabeth of having promised to play to him; and she
sat down directly to the instrument. He drew a chair
near her. Lady Catherine listened to half a song, and
then talked, as before, to her other nephew; till the
latter walked away from her, and moving with his usual
deliberation towards the piano forte, stationed himself so
as to command a full view of the fair performer’s countenance.
Elizabeth saw what he was doing, and at the
first convenient pause, turned to him with an arch smile,
and said,

“You mean to frighten me, Mr. Darcy, by coming in
all this state to hear me? But I will not be alarmed
though your sister \textit{does} play so well. There is a stubbornness
about me that never can bear to be frightened at
the will of others. My courage always rises with every
attempt to intimidate me.”

“I shall not say that you are mistaken,” he replied,
“because you could not really believe me to entertain
any design of alarming you; and I have had the pleasure
of your acquaintance long enough to know, that you find
great enjoyment in occasionally professing opinions which
in fact are not your own.”

Elizabeth laughed heartily at this picture of herself,
and said to Colonel Fitzwilliam, “Your cousin will give
you a very pretty notion of me, and teach you not to
believe a word I say. I am particularly unlucky in meeting
with a person so well able to expose my real character,
in a part of the world, where I had hoped to pass myself
off with some degree of credit. Indeed, Mr. Darcy, it is
very ungenerous in you to mention all that you knew to
my disadvantage in Hertfordshire -- and, give me leave
to say, very impolitic too -- for it is provoking me to
retaliate, and such things may come out, as will shock
your relations to hear.”

“I am not afraid of you,” said he, smilingly.

“Pray let me hear what you have to accuse him of,”
cried Colonel Fitzwilliam. “I should like to know how
he behaves among strangers.”
%%174%%

“You shall hear then -- but prepare yourself for something
very dreadful. The first time of my ever seeing
him in Hertfordshire, you must know, was at a ball -- and
at this ball, what do you think he did? He danced
only four dances! I am sorry to pain you -- but so it was.
He danced only four dances, though gentlemen were
scarce; and, to my certain knowledge, more than one
young lady was sitting down in want of a partner. Mr.
Darcy, you cannot deny the fact.”

“I had not at that time the honour of knowing any
lady in the assembly beyond my own party.”

“True; and nobody can ever be introduced in a ball
room. Well, Colonel Fitzwilliam, what do I play next?
My fingers wait your orders.”

“Perhaps,” said Darcy, “I should have judged better,
had I sought an introduction, but I am ill qualified to
recommend myself to strangers.”

“Shall we ask your cousin the reason of this?” said
Elizabeth, still addressing Colonel Fitzwilliam. “Shall
we ask him why a man of sense and education, and who
has lived in the world, is ill qualified to recommend himself
to strangers?”

“I can answer your question,” said Fitzwilliam, “without
applying to him. It is because he will not give himself
the trouble.”

“I certainly have not the talent which some people
possess,” said Darcy, “of conversing easily with those
I have never seen before. I cannot catch their tone of
conversation, or appear interested in their concerns, as
I often see done.”

“My fingers,” said Elizabeth, “do not move over this
instrument in the masterly manner which I see so many
women’s do. They have not the same force or rapidity,
and do not produce the same expression. But then I have
always supposed it to be my own fault -- because I would
not take the trouble of practising. It is not that I do not
believe \textit{my} fingers as capable as any other woman’s of
superior execution.”
%%175%%

Darcy smiled and said, “You are perfectly right. You
have employed your time much better. No one admitted
to the privilege of hearing you, can think any thing wanting.
We neither of us perform to strangers.”

Here they were interrupted by Lady Catherine, who
called out to know what they were talking of. Elizabeth
immediately began playing again. Lady Catherine
approached, and, after listening for a few minutes, said
to Darcy,

“Miss Bennet would not play at all amiss, if she practised
more, and could have the advantage of a London
master. She has a very good notion of fingering, though
her taste is not equal to Anne’s. Anne would have been
a delightful performer, had her health allowed her to
learn.”

Elizabeth looked at Darcy to see how cordially he
assented to his cousin’s praise; but neither at that
moment nor at any other could she discern any symptom
of love; and from the whole of his behaviour to Miss
De Bourgh she derived this comfort for Miss Bingley,
that he might have been just as likely to marry \textit{her}, had
she been his relation.

Lady Catherine continued her remarks on Elizabeth’s
performance, mixing with them many instructions on
execution and taste. Elizabeth received them with all
the forbearance of civility; and at the request of the
gentlemen remained at the instrument till her Ladyship’s
carriage was ready to take them all home.
%%176%%

\Chapter{CHAPTER IX.}

Elizabeth was sitting by herself the next morning, and
writing to Jane, while Mrs. Collins and Maria were gone
on business into the village, when she was startled by a ring
at the door, the certain signal of a visitor. As she had
heard no carriage, she thought it not unlikely to be
Lady Catherine, and under that apprehension was putting
away her half-finished letter that she might escape all
impertinent questions, when the door opened, and to her
very great surprise, Mr. Darcy, and Mr. Darcy only,
entered the room.

He seemed astonished too on finding her alone, and
apologised for his intrusion, by letting her know that he
had understood all the ladies to be within.

They then sat down, and when her enquiries after
Rosings were made, seemed in danger of sinking into
total silence. It was absolutely necessary, therefore, to
think of something, and in this emergence recollecting
\textit{when} she had seen him last in Hertfordshire, and feeling
curious to know what he would say on the subject of their
hasty departure, she observed,

“How very suddenly you all quitted Netherfield last
Nov\-ember, Mr. Darcy! It must have been a most agreeable
surprise to Mr. Bingley to see you all after him so
soon; for, if I recollect right, he went but the day
before. He and his sisters were well, I hope, when you
left London.”

“Perfectly so -- I thank you.”

She found that she was to receive no other answer -- and,
after a short pause, added,

“I think I have understood that Mr. Bingley has not
much idea of ever returning to Netherfield again?”

“I have never heard him say so; but it is probable
that he may spend very little of his time there in future.
%%177%%
He has many friends, and he is at a time of life when
friends and engagements are continually increasing.”

“If he means to be but little at Netherfield, it would
be better for the neighbourhood that he should give up
the place entirely, for then we might possibly get a
settled family there. But perhaps Mr. Bingley did not
take the house so much for the convenience of the neighbourhood
as for his own, and we must expect him to keep
or quit it on the same principle.”

“I should not be surprised,” said Darcy, “if he were
to give it up, as soon as any eligible purchase offers.”

Elizabeth made no answer. She was afraid of talking
longer of his friend; and, having nothing else to say,
was now determined to leave the trouble of finding a
subject to him.

He took the hint, and soon began with, “This seems
a very comfortable house. Lady Catherine, I believe,
did a great deal to it when Mr. Collins first came to
Hunsford.”

“I believe she did -- and I am sure she could not have
bestowed her kindness on a more grateful object.”

“Mr. Collins appears very fortunate in his choice of
a wife.”

“Yes, indeed; his friends may well rejoice in his
having met with one of the very few sensible women
who would have accepted him, or have made him happy
if they had. My friend has an excellent understanding -- though
I am not certain that I consider her marrying
Mr. Collins as the wisest thing she ever did. She seems
perfectly happy, however, and in a prudential light, it is
certainly a very good match for her.”

“It must be very agreeable to her to be settled within
so easy a distance of her own family and friends.”

“An easy distance do you call it? It is nearly fifty
miles.”

“And what is fifty miles of good road? Little more
than half a day’s journey. Yes, I call it a \textit{very} easy
distance.”
%%178%%

“I should never have considered the distance as one
of the \textit{advantages} of the match,” cried Elizabeth. “I
should never have said Mrs. Collins was settled \textit{near} her
family.”

“It is a proof of your own attachment to Hertfordshire.
Any thing beyond the very neighbourhood of
Longbourn, I suppose, would appear far.”

As he spoke there was a sort of smile, which Elizabeth
fancied she understood; he must be supposing her to be
thinking of Jane and Netherfield, and she blushed as she
answered,

“I do not mean to say that a woman may not be
settled too near her family. The far and the near must
be relative, and depend on many varying circumstances.
Where there is fortune to make the expence of travelling
unimportant, distance becomes no evil. But that is not
the case \textit{here}. Mr. and Mrs. Collins have a comfortable
income, but not such a one as will allow of frequent
journeys -- and I am persuaded my friend would not call
herself \textit{near} her family under less than \textit{half} the present
distance.”

Mr. Darcy drew his chair a little towards her, and
said, “\textit{You} cannot have a right to such very strong
local attachment. \textit{You} cannot have been always at
Longbourn.”

Elizabeth looked surprised. The gentleman experienced
some change of feeling; he drew back his chair, took
a newspaper from the table, and, glancing over it, said,
in a colder voice,

“Are you pleased with Kent?”

A short dialogue on the subject of the country ensued,
on either side calm and concise -- and soon put an end to
by the entrance of Charlotte and her sister, just returned
from their walk. The tête a tête surprised them. Mr.
Darcy related the mistake which had occasioned his intruding
on Miss Bennet, and after sitting a few minutes
longer without saying much to any body, went away.

“What can be the meaning of this!” said Charlotte,
%%179%%
as soon as he was gone. “My dear Eliza he must be
in love with you, or he would never have called on us in
this familiar way.”

But when Elizabeth told of his silence, it did not seem
very likely, even to Charlotte’s wishes, to be the case;
and after various conjectures, they could at last only
suppose his visit to proceed from the difficulty of finding
any thing to do, which was the more probable from the
time of year. All field sports were over. Within doors
there was Lady Catherine, books, and a billiard table,
but gentlemen cannot be always within doors; and in
the nearness of the Parsonage, or the pleasantness of the
walk to it, or of the people who lived in it, the two cousins
found a temptation from this period of walking thither
almost every day. They called at various times of the
morning, sometimes separately, sometimes together, and
now and then accompanied by their aunt. It was plain
to them all that Colonel Fitzwilliam came because he had
pleasure in their society, a persuasion which of course
recommended him still more; and Elizabeth was reminded
by her own satisfaction in being with him, as well as by
his evident admiration of her, of her former favourite
George Wickham; and though, in comparing them, she
saw there was less captivating softness in Colonel Fitzwilliam’s
manners, she believed he might have the best
informed mind.

But why Mr. Darcy came so often to the Parsonage,
it was more difficult to understand. It could not be for
society, as he frequently sat there ten minutes together
without opening his lips; and when he did speak, it
seemed the effect of necessity rather than of choice -- a
sacrifice to propriety, not a pleasure to himself. He
seldom appeared really animated. Mrs. Collins knew not
what to make of him. Colonel Fitzwilliam’s occasionally
laughing at his stupidity, proved that he was generally
different, which her own knowledge of him could not
have told her; and as she would have liked to believe
this change the effect of love, and the object of that love,
%%180%%
her friend Eliza, she sat herself seriously to work to find
it out. -- She watched him whenever they were at Rosings,
and whenever he came to Hunsford; but without much
success. He certainly looked at her friend a great deal,
but the expression of that look was disputable. It was
an earnest, steadfast gaze, but she often doubted whether
there were much admiration in it, and sometimes it seemed
nothing but absence of mind.

She had once or twice suggested to Elizabeth the
possibility of his being partial to her, but Elizabeth
always laughed at the idea; and Mrs. Collins did not
think it right to press the subject, from the danger of
raising expectations which might only end in disappointment;
for in her opinion it admitted not of a doubt,
that all her friend’s dislike would vanish, if she could
suppose him to be in her power.

In her kind schemes for Elizabeth, she sometimes
plann\-ed her marrying Colonel Fitzwilliam. He was beyond
comparison the pleasantest man; he certainly admired
her, and his situation in life was most eligible; but, to
counterbalance these advantages, Mr. Darcy had considerable
patronage in the church, and his cousin could
have none at all.
%%181%%

\Chapter{CHAPTER X.}

More than once did Elizabeth in her ramble within
the Park, unexpectedly meet Mr. Darcy. -- She felt all the
perverseness of the mischance that should bring him where
no one else was brought; and to prevent its ever happening
again, took care to inform him at first, that it was a
favourite haunt of hers. -- How it could occur a second
time therefore was very odd! -- Yet it did, and even a third.
It seemed like wilful ill-nature, or a voluntary penance,
for on these occasions it was not merely a few formal
enquiries and an awkward pause and then away, but he
actually thought it necessary to turn back and walk with
her. He never said a great deal, nor did she give herself
the trouble of talking or of listening much; but it struck
her in the course of their third rencontre that he was
asking some odd unconnected questions -- about her
pleasure in being at Hunsford, her love of solitary walks,
and her opinion of Mr. and Mrs. Collins’s happiness;
and that in speaking of Rosings and her not perfectly
understanding the house, he seemed to expect that whenever
she came into Kent again she would be staying \textit{there}
too. His words seemed to imply it. Could he have
Colonel Fitzwilliam in his thoughts? She supposed, if he
meant any thing, he must mean an allusion to what might
arise in that quarter. It distressed her a little, and she
was quite glad to find herself at the gate in the pales
opposite the Parsonage.

She was engaged one day as she walked, in re-perusing
Jane’s last letter, and dwelling on some passages which
proved that Jane had not written in spirits, when, instead
of being again surprised by Mr. Darcy, she saw on looking
up that Colonel Fitzwilliam was meeting her. Putting
away the letter immediately and forcing a smile, she said,

“I did not know before that you ever walked this way.”
%%182%%

“I have been making the tour of the Park,” he replied,
“as I generally do every year, and intend to close it with
a call at the Parsonage. Are you going much farther?”

“No, I should have turned in a moment.”

And accordingly she did turn, and they walked towards
the Parsonage together.

“Do you certainly leave Kent on Saturday?” said she.

“Yes -- if Darcy does not put it off again. But I am
at his disposal. He arranges the business just as he
pleases.”

“And if not able to please himself in the arrangement,
he has at least great pleasure in the power of choice. I do
not know any body who seems more to enjoy the power
of doing what he likes than Mr. Darcy.”

“He likes to have his own way very well,” replied
Colonel Fitzwilliam. “But so we all do. It is only that
he has better means of having it than many others,
because he is rich, and many others are poor. I speak
feelingly. A younger son, you know, must be inured to
self-denial and dependence.”

“In my opinion, the younger son of an Earl can know
very little of either. Now, seriously, what have you ever
known of self-denial and dependence? When have you
been prevented by want of money from going wherever
you chose, or procuring any thing you had a fancy for?”

“These are home questions -- and perhaps I cannot say
that I have experienced many hardships of that nature.
But in matters of greater weight, I may suffer from the
want of money. Younger sons cannot marry where they
like.”

“Unless where they like women of fortune, which I think
they very often do.”

“Our habits of expence make us too dependant, and
there are not many in my rank of life who can afford to
marry without some attention to money.”

“Is this,” thought Elizabeth, “meant for me?” and
she coloured at the idea; but, recovering herself, said
in a lively tone, “And pray, what is the usual price of an
%%183%%
Earl’s younger son? Unless the elder brother is very
sickly, I suppose you would not ask above fifty thousand
pounds.”

He answered her in the same style, and the subject
dropped. To interrupt a silence which might make him
fancy her affected with what had passed, she soon afterwards
said,

“I imagine your cousin brought you down with him
chiefly for the sake of having somebody at his disposal.
I wonder he does not marry, to secure a lasting convenience
of that kind. But, perhaps his sister does as
well for the present, and, as she is under his sole care,
he may do what he likes with her.”

“No,” said Colonel Fitzwilliam, “that is an advantage
which he must divide with me. I am joined with him in
the guardianship of Miss Darcy.”

“Are you, indeed? And pray what sort of gua\-rdians
do you make? Does your charge give you much trouble?
Young ladies of her age, are sometimes a little difficult
to manage, and if she has the true Darcy spirit, she may
like to have her own way.”

As she spoke, she observed him looking at her earnestly,
and the manner in which he immediately asked her why
she supposed Miss Darcy likely to give them any uneasiness,
convinced her that she had somehow or other got
pretty near the truth. She directly replied,

“You need not be frightened. I never heard any harm
of her; and I dare say she is one of the most tractable
creatures in the world. She is a very great favourite with
some ladies of my acquaintance, Mrs. Hurst and Miss
Bingley. I think I have heard you say that you know
them.”

“I know them a little. Their brother is a pleasant
gentle\-man-like man -- he is a great friend of Darcy’s.”

“Oh! yes,” said Elizabeth drily -- “Mr. Darcy is
uncommonly kind to Mr. Bingley, and takes a prodigious
deal of care of him.”

“Care of him! -- Yes, I really believe Darcy \textit{does} take
%%184%%
care of him in those points where he most wants care.
From something that he told me in our journey hither,
I have reason to think Bingley very much indebted to
him. But I ought to beg his pardon, for I have no right
to suppose that Bingley was the person meant. It was
all conjecture.”

“What is it you mean?”

“It is a circumstance which Darcy of course would
not wish to be generally known, because if it were to
get round to the lady’s family, it would be an unpleasant
thing.”

“You may depend upon my not mentioning it.”

“And remember that I have not much reason for
supposing it to be Bingley. What he told me was merely
this; that he congratulated himself on having lately saved
a friend from the inconveniences of a most imprudent
marriage, but without mentioning names or any other
particulars, and I only suspected it to be Bingley from
believing him the kind of young man to get into a scrape
of that sort, and from knowing them to have been together
the whole of last summer.”

“Did Mr. Darcy give you his reasons for this
interference?”

“I understood that there were some very strong objections
against the lady.”

“And what arts did he use to separate them?”

“He did not talk to me of his own arts,” said Fitzwilliam
smiling. “He only told me, what I have now
told you.”

Elizabeth made no answer, and walked on, her heart
swelling with indignation. After watching her a little,
Fitz\-william asked her why she was so thoughtful.

“I am thinking of what you have been telling me,”
said she. “Your cousin’s conduct does not suit my
feelings. Why was he to be the judge?”

“You are rather disposed to call his interference
officious?”

“I do not see what right Mr. Darcy had to decide on
%%185%%
the propriety of his friend’s inclination, or why, upon his
own judgment alone, he was to determine and direct in
what manner that friend was to be happy.” “But,” she
continued, recollecting herself, “as we know none of the
particulars, it is not fair to condemn him. It is not to
be supposed that there was much affection in the case.”

“That is not an unnatural surmise,” said Fitz\-william,
“but it is lessening the honour of my cousin’s triumph
very sadly.”

This was spoken jestingly, but it appeared to her so
just a picture of Mr. Darcy, that she would not trust
herself with an answer; and, therefore, abruptly changing
the conversation, talked on indifferent matters till they
reached the parsonage. There, shut into her own room,
as soon as their visitor left them, she could think without
interruption of all that she had heard. It was not to be
supposed that any other people could be meant than those
with whom she was connected. There could not exist in
the world \textit{two} men, over whom Mr. Darcy could have such
boundless influence. That he had been concerned in the
measures taken to separate Mr. Bingley and Jane, she had
never doubted; but she had always attributed to Miss
Bingley the principal design and arrangement of them.
If his own vanity, however, did not mislead him, \textit{he} was
the cause, his pride and caprice were the cause of all
that Jane had suffered, and still continued to suffer. He
had ruined for a while every hope of happiness for the most
affectionate, generous heart in the world; and no one
could say how lasting an evil he might have inflicted.

“There were some very strong objections ag\-ainst the
lady,” were Colonel Fitzwilliam’s words, and these strong
objections probably were, her having one uncle who was
a country attorney, and another who was in business in
London.

“To Jane herself,” she exclaimed, “there could be no
possibility of objection. All loveliness and goodness as
she is! Her understanding excellent, her mind improved,
and her manners captivating. Neither could any thing
%%186%%
be urged against my father, who, though with some
peculiarities, has abilities which Mr. Darcy himself need
not disdain, and respectability which he will probably
never reach.” When she thought of her mother indeed,
her confidence gave way a little, but she would not allow
that any objections \textit{there} had material weight with
Mr. Darcy, whose pride, she was convinced, would receive
a deeper wound from the want of importance in his friend’s
connections, than from their want of sense; and she was
quite decided at last, that he had been partly governed
by this worst kind of pride, and partly by the wish of
retaining Mr. Bingley for his sister.

The agitation and tears which the subject occasioned,
brought on a headach; and it grew so much worse
towards the evening that, added to her unwillingness to
see Mr. Darcy, it determined her not to attend her cousins
to Rosings, where they were engaged to drink tea. Mrs.
Collins, seeing that she was really unwell, did not press
her to go, and as much as possible prevented her husband
from pressing her, but Mr. Collins could not conceal his
apprehension of Lady Catherine’s being rather displeased
by her staying at home.
%%187%%

\Chapter{CHAPTER XI.}

When they were gone, Elizabeth, as if intending to
exasperate herself as much as possible against Mr. Darcy,
chose for her employment the examination of all the letters
which Jane had written to her since her being in Kent.
They contained no actual complaint, nor was there any
revival of past occurrences, or any communication of
present suffering. But in all, and in almost every line of
each, there was a want of that cheerfulness which had
been used to characterize her style, and which, proceeding
from the serenity of a mind at ease with itself, and kindly
disposed towards every one, had been scarcely ever clouded.
Elizabeth noticed every sentence conveying the idea of
uneasiness, with an attention which it had hardly received
on the first perusal. Mr. Darcy’s shameful boast of what
misery he had been able to inflict, gave her a keener sense
of her sister’s sufferings. It was some consolation to
think that his visit to Rosings was to end on the day after
the next, and a still greater, that in less than a fortnight
she should herself be with Jane again, and enabled to
contribute to the recovery of her spirits, by all that
affection could do.

She could not think of Darcy’s leaving Kent, without
remembering that his cousin was to go with him; but
Colonel Fitzwilliam had made it clear that he had no
intentions at all, and agreeable as he was, she did not
mean to be unhappy about him.

While settling this point, she was suddenly roused by
the sound of the door bell, and her spirits were a little
fluttered by the idea of its being Colonel Fitzwilliam
himself, who had once before called late in the evening,
and might now come to enquire particularly after her.
But this idea was soon banished, and her spirits were
very differently affected, when, to her utter amazement,
%%188%%
she saw Mr. Darcy walk into the room. In an hurried
manner he immediately began an enquiry after her health,
imputing his visit to a wish of hearing that she were
better. She answered him with cold civility. He sat
down for a few moments, and then getting up walked
about the room. Elizabeth was surprised, but said not
a word. After a silence of several minutes he came
towards her in an agitated manner, and thus began,

“In vain have I struggled. It will not do. My feelings
will not be repressed. You must allow me to tell you
how ardently I admire and love you.”

Elizabeth’s astonishment was beyond expression. She
stared, coloured, doubted, and was silent. This he considered
sufficient encouragement, and the avowal of all
that he felt and had long felt for her, immediately followed.
He spoke well, but there were feelings besides those of the
heart to be detailed, and he was not more eloquent on the
subject of tenderness than of pride. His sense of her
inferiority -- of its being a degradation -- of the family
obstacles which judgment had always opposed to inclination,
were dwelt on with a warmth which seemed due
to the consequence he was wounding, but was very unlikely
to recommend his suit.

In spite of her deeply-rooted dislike, she could not be
insensible to the compliment of such a man’s affection,
and though her intentions did not vary for an instant,
she was at first sorry for the pain he was to receive; till,
roused to resentment by his subsequent language, she lost
all compassion in anger. She tried, however, to compose
herself to answer him with patience, when he should have
done. He concluded with representing to her the strength
of that attachment which, in spite of all his endeavours,
he had found impossible to conquer; and with expressing
his hope that it would now be rewarded by her acceptance
of his hand. As he said this, she could easily see that he
had no doubt of a favourable answer. He \textit{spoke} of apprehension
and anxiety, but his countenance expressed real
security. Such a circumstance could only exasperate
%%189%%
farther, and when he ceased, the colour rose into her cheeks,
and she said,

“In such cases as this, it is, I believe, the established
mode to express a sense of obligation for the sentiments
avowed, however unequally they may be returned. It is
natural that obligation should be felt, and if I could \textit{feel}
gratitude, I would now thank you. But I cannot -- I have
never desired your good opinion, and you have certainly
bestowed it most unwillingly. I am sorry to have occasioned
pain to any one. It has been most unconsciously
done, however, and I hope will be of short duration.
The feelings which, you tell me, have long prevented the
acknowledgment of your regard, can have little difficulty
in overcoming it after this explanation.”

Mr. Darcy, who was leaning against the mantle-piece
with his eyes fixed on her face, seemed to catch her words
with no less resentment than surprise. His complexion
became pale with anger, and the disturbance of his mind
was visible in every feature. He was struggling for the
appearance of composure, and would not open his lips,
till he believed himself to have attained it. The pause
was to Elizabeth’s feelings dreadful. At length, in a voice
of forced calmness, he said,

“And this is all the reply which I am to have the
honour of expecting! I might, perhaps, wish to be informed
why, with so little \textit{endeavour} at civility, I am thus
rejected. But it is of small importance.”

“I might as well enquire,” replied she, “why with so
evident a design of offending and insulting me, you chose
to tell me that you liked me against your will, against
your reason, and even against your character? Was not
this some excuse for incivility, if I \textit{was} uncivil? But I have
other provocations. You know I have. Had not my own
feelings decided against you, had they been indifferent,
or had they even been favourable, do you think that any
consideration would tempt me to accept the man, who has
been the means of ruining, perhaps for ever, the happiness
of a most beloved sister?”
%%190%%

As she pronounced these words, Mr. Darcy changed
colour; but the emotion was short, and he listened
without attempting to interrupt her while she continued.

“I have every reason in the world to think ill of you.
No motive can excuse the unjust and ungenerous part
you acted \textit{there}. You dare not, you cannot deny that you
have been the principal, if not the only means of dividing
them from each other, of exposing one to the censure of
the world for caprice and instability, the other to its
derision for disappointed hopes, and involving them both
in misery of the acutest kind.”

She paused, and saw with no slight indignation that
he was listening with an air which proved him wholly
unmoved by any feeling of remorse. He even looked at
her with a smile of affected incredulity.

“Can you deny that you have done it?” she repeated.

With assumed tranquillity he then replied, “I have no wish
of denying that I did every thing in my power to separate
my friend from your sister, or that I rejoice in my success.
Towards \textit{him} I have been kinder than towards myself.”

Elizabeth disdained the appearance of noticing this civil
reflection, but its meaning did not escape, nor was it
likely to conciliate her.

“But it is not merely this affair,” she continued, “on
which my dislike is founded. Long before it had taken
place, my opinion of you was decided. Your character
was unfolded in the recital which I received many months
ago from Mr. Wickham. On this subject, what can you
have to say? In what imaginary act of friendship can you
here defend yourself? or under what misrepresentation,
can you here impose upon others?”

“You take an eager interest in that gentleman’s
concerns,” said Darcy in a less tranquil tone, and with
a heightened colour.

“Who that knows what his misfortunes have been, can
help feeling an interest in him?”

“His misfortunes!” repeated Darcy contemptuously;
“yes, his misfortunes have been great indeed.”
%%191%%

“And of your infliction,” cried Elizabeth with energy.
“You have reduced him to his present state of poverty,
comparative poverty. You have withheld the advantages,
which you must know to have been designed for him.
You have deprived the best years of his life, of that
independence which was no less his due than his desert. You
have done all this! and yet you can treat the mention
of his misfortunes with contempt and ridicule.”

“And this,” cried Darcy, as he walked with quick
steps across the room, “is your opinion of me! This is
the estimation in which you hold me! I thank you
for explaining it so fully. My faults, according to this
calculation, are heavy indeed! But perhaps,” added he,
stopping in his walk, and turning towards her, “these
offences might have been overlooked, had not your pride
been hurt by my honest confession of the scruples that
had long prevented my forming any serious design.
These bitter accusations might have been suppressed, had
I with greater policy concealed my struggles, and flattered
you into the belief of my being impelled by unqualified,
unalloyed inclination; by reason, by reflection, by
every thing. But disguise of every sort is my abhorrence.
Nor am I ashamed of the feelings I related. They
were natural and just. Could you expect me to rejoice
in the inferiority of your connections? To congratulate
myself on the hope of relations, whose condition in life
is so decidedly beneath my own?”

Elizabeth felt herself growing more angry every moment;
yet she tried to the utmost to speak with composure when
she said,

“You are mistaken, Mr. Darcy, if you suppose that
the mode of your declaration affected me in any other
way, than as it spared me the concern which I might
have felt in refusing you, had you behaved in a more
gentleman-like manner.”

She saw him start at this, but he said nothing, and she
continued,

“You could not have made me the offer of your hand
%%192%%
in any possible way that would have tempted me to
accept it.”

Again his astonishment was obvious; and he looked
at her with an expression of mingled incredulity and
mortification. She went on.

“From the very beginning, from the first moment
I may almost say, of my acquaintance with you, your
manners impressing me with the fullest belief of your
arrogance, your conceit, and your selfish disdain of the
feelings of others, were such as to form that ground-work
of disapprobation, on which succeeding events have built
so immoveable a dislike; and I had not known you a
month before I felt that you were the last man in the
world whom I could ever be prevailed on to marry.”

“You have said quite enough, madam. I perfectly
comprehend your feelings, and have now only to be
ashamed of what my own have been. Forgive me for
having taken up so much of your time, and accept my
best wishes for your health and happiness.”

And with these words he hastily left the room, and
Elizabeth heard him the next moment open the front door
and quit the house.

The tumult of her mind was now painfully great. She
knew not how to support herself, and from actual weakness
sat down and cried for half an hour. Her astonishment,
as she reflected on what had passed, was increased by
every review of it. That she should receive an offer of
marriage from Mr. Darcy! that he should have been in
love with her for so many months! so much in love
as to wish to marry her in spite of all the objections which
had made him prevent his friend’s marrying her sister, and
which must appear at least with equal force in his own case,
was almost incredible! it was gratifying to have inspired
unconsciously so strong an affection. But his pride, his
abominable pride, his shameless avowal of what he had
done with respect to Jane, his unpardonable assurance
in acknowledging, though he could not justify it, and the
unfeeling manner in which he had mentioned Mr.
%%193%%
Wickham, his cruelty towards whom he had not attempted to
deny, soon overcame the pity which the consideration of
his attachment had for a moment excited.

She continued in very agitating reflections till the sound
of Lady Catherine’s carriage made her feel how unequal
she was to encounter Charlotte’s observation, and hurried
her away to her room.
%%194%%

\Chapter{CHAPTER XII.}

Elizabeth awoke the next morning to the same
thoughts and meditations which had at length closed her
eyes. She could not yet recover from the surprise of what
had happened; it was impossible to think of any thing
else, and totally indisposed for employment, she resolved
soon after breakfast to indulge herself in air and exercise.
She was proceeding directly to her favourite walk, when
the recollection of Mr. Darcy’s sometimes coming there
stopped her, and instead of entering the park, she turned
up the lane, which led her farther from the turnpike road.
The park paling was still the boundary on one side, and
she soon passed one of the gates into the ground.

After walking two or three times along that part of the
lane, she was tempted, by the pleasantness of the morning,
to stop at the gates and look into the park. The five
weeks which she had now passed in Kent, had made
a great difference in the country, and every day was adding
to the verdure of the early trees. She was on the point
of continuing her walk, when she caught a glimpse of
a gentleman within the sort of grove which edged the
park; he was moving that way; and fearful of its being
Mr. Darcy, she was directly retreating. But the person
who advanced, was now near enough to see her, and
stepping forward with eagerness, pronounced her name.
She had turned away, but on hearing herself called,
though in a voice which proved it to be Mr. Darcy, she
moved again towards the gate. He had by that time
reached it also, and holding out a letter, which she
instinctively took, said with a look of haughty composure,
“I have been walking in the grove some time in the hope
of meeting you. Will you do me the honour of reading
that letter?” -- And then, with a slight bow, turned again
into the plantation, and was soon out of sight.
%%195%%

With no expectation of pleasure, but with the strongest
curiosity, Elizabeth opened the letter, and to her still
increasing wonder, perceived an envelope containing two
sheets of letter paper, written quite through, in a very
close hand. -- The envelope itself was likewise full. --
Pursuing her way along the lane, she then began it. It
was dated from Rosings, at eight o’clock in the morning,
and was as follows:--

\begin{letter}
“Be not alarmed, Madam, on receiving this letter, by
the apprehension of its containing any repetition of those
sentiments, or renewal of those offers, which were last
night so disgusting to you. I write without any intention
of paining you, or humbling myself, by dwelling on
wishes, which, for the happiness of both, cannot be too
soon forgotten; and the effort which the formation, and
the perusal of this letter must occasion, should have been
spared, had not my character required it to be written
and read. You must, therefore, pardon the freedom with
which I demand your attention; your feelings, I know,
will bestow it unwillingly, but I demand it of your justice.

“Two offences of a very different nature, and by no
means of equal magnitude, you last night laid to my
charge. The first mentioned was, that, regardless of the
sentiments of either, I had detached Mr. Bingley from
your sister, -- and the other, that I had, in defiance of
various claims, in defiance of honour and humanity, ruined
the immediate prosperity, and blasted the prospects of
Mr. Wickham. -- Wilfully and wantonly to have thrown off
the companion of my youth, the acknowledged favourite
of my father, a young man who had scarcely any other
dependence than on our patronage, and who had been
brought up to expect its exertion, would be a depravity,
to which the separation of two young persons, whose
affection could be the growth of only a few weeks, could
bear no comparison. -- But from the severity of that blame
which was last night so liberally bestowed, respecting each
circumstance, I shall hope to be in future secured, when
the following account of my actions and their motives
%%196%%
has been read. -- If, in the explanation of them which is
due to myself, I am under the necessity of relating feelings
which may be offensive to your’s, I can only say that I am
sorry. -- The necessity must be obeyed -- and farther apology
would be absurd. -- I had not been long in Hertfordshire,
before I saw, in common with others, that Bingley preferred
your eldest sister, to any other young woman in the
country. -- But it was not till the evening of the dance at
Netherfield that I had any apprehension of his feeling
a serious attachment. -- I had often seen him in love before.
-- At that ball, while I had the honour of dancing with
you, I was first made acquainted, by Sir William Lucas’s
accidental information, that Bingley’s attentions to your
sister had given rise to a general expectation of their
marriage. He spoke of it as a certain event, of which
the time alone could be undecided. From that moment
I observed my friend’s behaviour attentively; and I could
then perceive that his partiality for Miss Bennet was
beyond what I had ever witnessed in him. Your sister
I also watched. -- Her look and manners were open,
cheerful and engaging as ever, but without any symptom
of peculiar regard, and I remained convinced from the
evening’s scrutiny, that though she received his attentions
with pleasure, she did not invite them by any participation
of sentiment. -- If \textit{you} have not been mistaken here, \textit{I} must
have been in an error. Your superior knowledge of your
sister must make the latter probable. -- If it be so, if I have
been misled by such error, to inflict pain on her, your
resentment has not been unreasonable. But I shall not
scruple to assert, that the serenity of your sister’s countenance
and air was such, as might have given the most
acute observer, a conviction that, however amiable her
temper, her heart was not likely to be easily touched. -- That
I was desirous of believing her indifferent is certain, -- but
I will venture to say that my investigations and
decisions are not usually influenced by my hopes or fears.
-- I did not believe her to be indifferent because I wished it; --
I believed it on impartial conviction, as truly as I wished
%%197%%
it in reason. -- My objections to the marriage were not
merely those, which I last night acknowledged to have
required the utmost force of passion to put aside, in my
own case; the want of connection could not be so great
an evil to my friend as to me. -- But there were other
causes of repugnance; -- causes which, though still existing,
and existing to an equal degree in both instances, I had
myself endeavoured to forget, because they were not
immediately before me. -- These causes must be stated,
though briefly. -- The situation of your mother’s family,
though objectionable, was nothing in comparison of that
total want of propriety so frequently, so almost uniformly
betrayed by herself, by your three younger sisters, and
occasionally even by your father. -- Pardon me. -- It pains
me to offend you. But amidst your concern for the
defects of your nearest relations, and your displeasure at
this representation of them, let it give you consolation
to consider that, to have conducted yourselves so as to
avoid any share of the like censure, is praise no less
generally bestowed on you and your eldest sister, than it
is honourable to the sense and disposition of both. -- I will
only say farther, that from what passed that evening,
my opinion of all parties was confirmed, and every inducement
heightened, which could have led me before, to
preserve my friend from what I esteemed a most unhappy
connection. -- He left Netherfield for London, on the day
following, as you, I am certain, remember, with the design
of soon returning. -- The part which I acted, is now to be
explained. -- His sisters’ uneasiness had been equally
excited with my own; our coincidence of feeling was soon
discovered; and, alike sensible that no time was to be
lost in detaching their brother, we shortly resolved on
joining him directly in London. -- We accordingly went -- and
there I readily engaged in the office of pointing out
to my friend, the certain evils of such a choice. -- I described,
and enforced them earnestly. -- But, however this
remonstrance might have staggered or delayed his determination,
I do not suppose that it would ultimately have
%%198%%
prevented the marriage, had it not been seconded by the
assurance which I hesitated not in giving, of your sister’s
indifference. He had before believed her to return his
affection with sincere, if not with equal regard. -- But
Bingley has great natural modesty, with a stronger dependence
on my judgment than on his own. -- To convince
him, therefore, that he had deceived himself, was no very
difficult point. To persuade him against returning into
Hertfordshire, when that conviction had been given, was
scarcely the work of a moment. -- I cannot blame myself
for having done thus much. There is but one part of my
conduct in the whole affair, on which I do not reflect
with satisfaction; it is that I condescended to adopt the
measures of art so far as to conceal from him your sister’s
being in town. I knew it myself, as it was known to
Miss Bingley, but her brother is even yet ignorant of it. --
That they might have met without ill consequence, is
perhaps probable; -- but his regard did not appear to me
enough extinguished for him to see her without some
danger. -- Perhaps this concealment, this disguise, was
beneath me. -- It is done, however, and it was done for
the best. -- On this subject I have nothing more to say,
no other apology to offer. If I have wounded your sister’s
feelings, it was unknowingly done; and though the
motives which governed me may to you very naturally
appear insufficient, I have not yet learnt to condemn
them. -- With respect to that other, more weighty accusation,
of having injured Mr. Wickham, I can only refute
it by laying before you the whole of his connection with
my family. Of what he has \textit{particularly} accused me I am
ignorant; but of the truth of what I shall relate, I can
summon more than one witness of undoubted veracity.
Mr. Wickham is the son of a very respectable man, who
had for many years the management of all the Pemberley
estates; and whose good conduct in the discharge of his
trust, naturally inclined my father to be of service to
him, and on George Wickham, who was his god-son, his
kindness was therefore liberally bestowed. My father
%%199%%
supported him at school, and afterwards at Cambridge; -- most
important assistance, as his own father, always poor
from the extravagance of his wife, would have been
unable to give him a gentleman’s education. My father
was not only fond of this young man’s society, whose
manners were always engaging; he had also the highest
opinion of him, and hoping the church would be his
profession, intended to provide for him in it. As for
myself, it is many, many years since I first began to think
of him in a very different manner. The vicious
propensities -- the want of principle which he was careful to
guard from the knowledge of his best friend, could not
escape the observation of a young man of nearly the same
age with himself, and who had opportunities of seeing
him in unguarded moments, which Mr. Darcy could not
have. Here again I shall give you pain -- to what degree
you only can tell. But whatever may be the sentiments
which Mr. Wickham has created, a suspicion of their
nature shall not prevent me from unfolding his real
character. It adds even another motive. My excellent
father died about five years ago; and his attachment to
Mr. Wickham was to the last so steady, that in his will
he particularly recommended it to me, to promote his
advancement in the best manner that his profession might
allow, and if he took orders, desired that a valuable family
living might be his as soon as soon as it became vacant. There
was also a legacy of one thousand pounds. His own father
did not long survive mine, and within half a year from
these events, Mr. Wickham wrote to inform me that,
having finally resolved against taking orders, he hoped
I should not think it unreasonable for him to expect
some more immediate pecuniary advantage, in lieu of the
preferment, by which he could not be benefited. He had
some intention, he added, of studying the law, and I must
be aware that the interest of one thousand pounds would
be a very insufficient support therein. I rather wished,
than believed him to be sincere; but at any rate, was
perfectly ready to accede to his proposal. I knew that
%%200%%
Mr. Wickham ought not to be a clergyman. The business
was therefore soon settled. He resigned all claim to
assistance in the church, were it possible that he could
ever be in a situation to receive it, and accepted in return
three thousand pounds. All connection between us
seemed now dissolved. I thought too ill of him, to invite
him to Pemberley, or admit his society in town. In town
I believe he chiefly lived, but his studying the law was
a mere pretence, and being now free from all restraint,
his life was a life of idleness and dissipation. For about
three years I heard little of him; but on the decease
of the incumbent of the living which had been designed
for him, he applied to me again by letter for the presentation.
His circumstances, he assured me, and I had no
difficulty in believing it, were exceedingly bad. He had
found the law a most unprofitable study, and was now
absolutely resolved on being ordained, if I would present
him to the living in question -- of which he trusted there
could be little doubt, as he was well assured that I had
no other person to provide for, and I could not have
forgotten my revered father’s intentions. You will hardly
blame me for refusing to comply with this entreaty, or
for resisting every repetition of it. His resentment was
in proportion to the distress of his circumstances -- and
he was doubtless as violent in his abuse of me to others,
as in his reproaches to myself. After this period, every
appearance of acquaintance was dropt. How he lived
I know not. But last summer he was again most painfully
obtruded on my notice. I must now mention a circumstance
which I would wish to forget myself, and which
no obligation less than the present should induce me to
unfold to any human being. Having said thus much,
I feel no doubt of your secrecy. My sister, who is more
than ten years my junior, was left to the guardianship of
my mother’s nephew, Colonel Fitzwilliam, and myself.
About a year ago, she was taken from school, and an
establishment formed for her in London; and last summer
she went with the lady who presided over it, to Ramsgate;
%%201%%
and thither also went Mr. Wickham, undoubtedly by
design; for there proved to have been a prior acquaintance
between him and Mrs. Younge, in whose character we
were most unhappily deceived; and by her connivance
and aid, he so far recommended himself to Georgiana,
whose affectionate heart retained a strong impression of
his kindness to her as a child, that she was persuaded to
believe herself in love, and to consent to an elopement.
She was then but fifteen, which must be her excuse; and
after stating her imprudence, I am happy to add, that
I owed the knowledge of it to herself. I joined them
unexpectedly a day or two before the intended elopement,
and then Georgiana, unable to support the idea of grieving
and offending a brother whom she almost looked up to
as a father, acknowledged the whole to me. You may
imagine what I felt and how I acted. Regard for my
sister’s credit and feelings prevented any public exposure,
but I wrote to Mr. Wickham, who left the place immediately,
and Mrs. Younge was of course removed from her
charge. Mr. Wickham’s chief object was unquestionably
my sister’s fortune, which is thirty thousand pounds;
but I cannot help supposing that the hope of revenging
himself on me, was a strong inducement. His revenge
would have been complete indeed. This, madam, is a
faithful narrative of every event in which we have been
concerned together; and if you do not absolutely reject
it as false, you will, I hope, acquit me henceforth of cruelty
towards Mr. Wickham. I know not in what manner, under
what form of falsehood he has imposed on you; but his
success is not perhaps to be wondered at. Ignorant as you
previously were of every thing concerning either, detection
could not be in your power, and suspicion certainly
not in your inclination. You may possibly wonder why
all this was not told you last night. But I was not then
master enough of myself to know what could or ought
to be revealed. For the truth of every thing here related,
I can appeal more particularly to the testimony of Colonel
Fitzwilliam, who from our near relationship and constant
%%202%%
intimacy, and still more as one of the executors of my
father’s will, has been unavoidably acquainted with every
particular of these transactions. If your abhorrence of
\textit{me} should make \textit{my} assertions valueless, you cannot be
prevented by the same cause from confiding in my cousin;
and that there may be the possibility of consulting him,
I shall endeavour to find some opportunity of putting this
letter in your hands in the course of the morning. I will
only add, God bless you.

\LetterSig{“Fitzwilliam Darcy.”}
\end{letter}
%%203%%

\Chapter{CHAPTER XIII.}

If Elizabeth, when Mr. Darcy gave her the letter, did
not expect it to contain a renewal of his offers, she had
formed no expectation at all of its contents. But such
as they were, it may be well supposed how eagerly she
went through them, and what a contrariety of emotion
they excited. Her feelings as she read were scarcely to
be defined. With amazement did she first understand
that he believed any apology to be in his power; and
stedfastly was she persuaded that he could have no
explanation to give, which a just sense of shame would
not conceal. With a strong prejudice against every thing
he might say, she began his account of what had happened
at Netherfield. She read, with an eagerness which hardly
left her power of comprehension, and from impatience of
knowing what the next sentence might bring, was incapable
of attending to the sense of the one before her eyes. His
belief of her sister’s insensibility, she instantly resolved
to be false, and his account of the real, the worst objections
to the match, made her too angry to have any wish of
doing him justice. He expressed no regret for what he
had done which satisfied her; his style was not penitent,
but haughty. It was all pride and insolence.

But when this subject was succeeded by his account
of Mr. Wickham, when she read with somewhat clearer
attention, a relation of events, which, if true, must overthrow
every cherished opinion of his worth, and which
bore so alarming an affinity to his own history of himself,
her feelings were yet more acutely painful and more
difficult of definition. Astonishment, apprehension, and
even horror, oppressed her. She wished to discredit it
entirely, repeatedly exclaiming, “This must be false!
This cannot be! This must be the grossest falsehood!” -- and
when she had gone through the whole letter, though
%%204%%
scarcely knowing any thing of the last page or two, put
it hastily away, protesting that she would not regard it,
that she would never look in it again.

In this perturbed state of mind, with thoughts that
could rest on nothing, she walked on; but it would not
do; in half a minute the letter was unfolded again, and
collecting herself as well as she could, she again began
the mortifying perusal of all that related to Wickham,
and commanded herself so far as to examine the meaning
of every sentence. The account of his connection with
the Pemberley family, was exactly what he had related
himself; and the kindness of the late Mr. Darcy, though
she had not before known its extent, agreed equally well
with his own words. So far each recital confirmed the
other: but when she came to the will, the difference was
great. What Wickham had said of the living was fresh
in her memory, and as she recalled his very words, it was
impossible not to feel that there was gross duplicity on
one side or the other; and, for a few moments, she flattered
herself that her wishes did not err. But when she read,
and re-read with the closest attention, the particulars
immediately following of Wickham’s resigning all pretensions
to the living, of his receiving in lieu, so considerable
a sum as three thousand pounds, again was she forced
to hesitate. She put down the letter, weighed every
circumstance with what she meant to be impartiality -- deliberated
on the probability of each statement -- but
with little success. On both sides it was only assertion.
Again she read on. But every line proved more clearly
that the affair, which she had believed it impossible that
any contrivance could so represent, as to render Mr. Darcy’s
conduct in it less than infamous, was capable of a turn
which must make him entirely blameless throughout the
whole.

The extravagance and general profligacy which he
scrupled not to lay to Mr. Wickham’s charge, exceedingly
shocked her; the more so, as she could bring no proof
of its injustice. She had never heard of him before his
%%205%%
entrance into the ------shire Militia, in which he had
engaged at the persuasion of the young man, who, on
meeting him accidentally in town, had there renewed
a slight acquaintance. Of his former way of life, nothing
had been known in Hertfordshire but what he told himself.
As to his real character, had information been in her
power, she had never felt a wish of enquiring. His countenance,
voice, and manner, had established him at once
in the possession of every virtue. She tried to recollect
some instance of goodness, some distinguished trait of
integrity or benevolence, that might rescue him from
the attacks of Mr. Darcy; or at least, by the predominance
of virtue, atone for those casual errors, under which she
would endeavour to class, what Mr. Darcy had described
as the idleness and vice of many years continuance.
But no such recollection befriended her. She could see
him instantly before her, in every charm of air and address;
but she could remember no more substantial good than
the general approbation of the neighbourhood, and the
regard which his social powers had gained him in the mess.
After pausing on this point a considerable while, she once
more continued to read. But, alas! the story which
followed of his designs on Miss Darcy, received some
confirmation from what had passed between Colonel
Fitzwilliam and herself only the morning before; and at
last she was referred for the truth of every particular to
Colonel Fitzwilliam himself -- from whom she had previously
received the information of his near concern in all his
cousin’s affairs, and whose character she had no reason
to question. At one time she had almost resolved on
applying to him, but the idea was checked by the awkwardness
of the application, and at length wholly banished
by the conviction that Mr. Darcy would never have
hazarded such a proposal, if he had not been well assured
of his cousin’s corroboration.

She perfectly remembered every thing that had passed
in conversation between Wickham and herself, in their
first evening at Mr. Philips’s. Many of his expressions
%%206%%
were still fresh in her memory. She was \textit{now} struck with
the impropriety of such communications to a stranger,
and wondered it had escaped her before. She saw the
indelicacy of putting himself forward as he had done,
and the inconsistency of his professions with his conduct.
She remembered that he had boasted of having no fear
of seeing Mr. Darcy -- that Mr. Darcy might leave the
country, but that \textit{he} should stand his ground; yet he
had avoided the Netherfield ball the very next week.
She remembered also, that till the Netherfield family had
quitted the country, he had told his story to no one but
herself; but that after their removal, it had been every
where discussed; that he had then no reserves, no scruples
in sinking Mr. Darcy’s character, though he had assured
her that respect for the father, would always prevent his
exposing the son.

How differently did every thing now appear in which
he was concerned! His attentions to Miss King were
now the consequence of views solely and hatefully mercenary;
and the mediocrity of her fortune proved no longer
the moderation of his wishes, but his eagerness to grasp
at any thing. His behaviour to herself could now have
had no tolerable motive; he had either been deceived
with regard to her fortune, or had been gratifying his
vanity by encouraging the preference which she believed
she had most incautiously shewn. Every lingering struggle
in his favour grew fainter and fainter; and in farther
justification of Mr. Darcy, she could not but allow that
Mr. Bingley, when questioned by Jane, had long ago
asserted his blamelessness in the affair; that proud and
repulsive as were his manners, she had never, in the whole
course of their acquaintance, an acquaintance which had
latterly brought them much together, and given her a sort
of intimacy with his ways, seen any thing that betrayed
him to be unprincipled or unjust -- any thing that spoke
him of irreligious or immoral habits. That among his
own connections he was esteemed and valued -- that even
Wickham had allowed him merit as a brother, and that
%%207%%
she had often heard him speak so affectionately of his
sister as to prove him capable of \textit{some} amiable feeling.
That had his actions been what Wickham represented
them, so gross a violation of every thing right could hardly
have been concealed from the world; and that friendship
between a person capable of it, and such an amiable man
as Mr. Bingley, was incomprehensible.

She grew absolutely ashamed of herself. -- Of neither
Darcy nor Wickham could she think, without feeling that
she had been blind, partial, prejudiced, absurd.

“How despicably have I acted!” she cried. -- “I, who
have prided myself on my discernment! -- I, who have
valued myself on my abilities! who have often disdained
the generous candour of my sister, and gratified my vanity,
in useless or blameable distrust. -- How humiliating is this
discovery! -- Yet, how just a humiliation! -- Had I been
in love, I could not have been more wretchedly blind.
But vanity, not love, has been my folly. -- Pleased with
the preference of one, and offended by the neglect of the
other, on the very beginning of our acquaintance, I have
courted prepossession and ignorance, and driven reason
away, where either were concerned. Till this moment,
I never knew myself.”

From herself to Jane -- from Jane to Bingley, her
thoughts were in a line which soon brought to her recollection
that Mr. Darcy’s explanation \textit{there}, had appeared
very insufficient; and she read it again. Widely different
was the effect of a second perusal. -- How could she deny
that credit to his assertions, in one instance, which she
had been obliged to give in the other? -- He declared
himself to have been totally unsuspicious of her sister’s
attachment; -- and she could not help remembering what
Charlotte’s opinion had always been. -- Neither could she
deny the justice of his description of Jane. -- She felt that
Jane’s feelings, though fervent, were little displayed, and
that there was a constant complacency in her air and
manner, not often united with great sensibility.

When she came to that part of the letter in which
%%208%%
her family were mentioned, in terms of such mortifying,
yet merited reproach, her sense of shame was severe.
The justice of the charge struck her too forcibly for denial,
and the circumstances to which he particularly alluded,
as having passed at the Netherfield ball, and as confirming
all his first disapprobation, could not have made a stronger
impression on his mind than on hers.

The compliment to herself and her sister, was not
unfelt. It soothed, but it could not console her for the
contempt which had been thus self-attracted by the rest
of her family; -- and as she considered that Jane’s disappointment
had in fact been the work of her nearest
relations, and reflected how materially the credit of both
must be hurt by such impropriety of conduct, she felt
depressed beyond any thing she had ever known before.

After wandering along the lane for two hours, giving
way to every variety of thought; re-consider\-ing events,
determining probabilities, and reconciling herself as well
as she could, to a change so sudden and so important,
fatigue, and a recollection of her long absence, made her
at length return home; and she entered the house with
the wish of appearing cheerful as usual, and the resolution
of repressing such reflections as must make her unfit for
conversation.

She was immediately told, that the two gentlemen from
Rosings had each called during her absence; Mr. Darcy,
only for a few minutes to take leave, but that Colonel
Fitzwilliam had been sitting with them at least an hour,
hoping for her return, and almost resolving to walk after
her till she could be found. -- Elizabeth could but just
\textit{affect} concern in missing him; she really rejoiced at it.
Colonel Fitzwilliam was no longer an object. She could
think only of her letter.
%%209%%

\Chapter{CHAPTER XIV.}

The two gentlemen left Rosings the next morning;
and Mr. Collins having been in waiting near the lodges,
to make them his parting obeisance, was able to bring
home the pleasing intelligence, of their appearing in very
good health, and in as tolerable spirits as could be
expected, after the melancholy scene so lately gone
through at Rosings. To Rosings he then hastened to
console Lady Catherine, and her daughter; and on his
return, brought back, with great satisfaction, a message
from her Ladyship, importing that she felt herself so
dull as to make her very desirous of having them all to
dine with her.

Elizabeth could not see Lady Catherine without recollecting,
that had she chosen it, she might by this time
have been presented to her, as her future niece; nor
could she think, without a smile, of what her ladyship’s
indignation would have been. “What would she have
said? -- how would she have behaved?” were questions
with which she amused herself.

Their first subject was the diminution of the Rosings
party. -- “I assure you, I feel it exceedingly,” said Lady
Catherine; “I believe nobody feels the loss of friends so
much as I do. But I am particularly attached to these
young men; and know them to be so much attached to
me! -- They were excessively sorry to go! But so they
always are. The dear colonel rallied his spirits tolerably
till just at last; but Darcy seemed to feel it most acutely,
more I think than last year. His attachment to Rosings,
certainly increases.”

Mr. Collins had a compliment, and an allusion to throw
in here, which were kindly smiled on by the mother and
daughter.
%%210%%

Lady Catherine observed, after dinner, that Miss
Bennet seemed out of spirits, and immediately accounting
for it herself, by supposing that she did not like to
go home again so soon, she added,

“But if that is the case, you must write to your mother
to beg that you may stay a little longer. Mrs. Collins
will be very glad of your company, I am sure.”

“I am much obliged to your ladyship for your kind
invitation,” replied Elizabeth, “but it is not in my power
to accept it. -- I must be in town next Saturday.”

“Why, at that rate, you will have been here only six
weeks. I expected you to stay two months. I told Mrs.
Collins so before you came. There can be no occasion
for your going so soon. Mrs. Bennet could certainly spare
you for another fortnight.”

“But my father cannot. -- He wrote last week to hurry
my return.”

“Oh! your father of course may spare you, if your
mother can. -- Daughters are never of so much consequence
to a father. And if you will stay another \textit{month} complete,
it will be in my power to take one of you as far as London,
for I am going there early in June, for a week; and as
Dawson does not object to the Barouche box, there
will be very good room for one of you -- and indeed,
if the weather should happen to be cool, I should not
object to taking you both, as you are neither of you
large.”

“You are all kindness, Madam; but I believe we must
abide by our original plan.”

Lady Catherine seemed resigned.

“Mrs. Collins, you must send a servant with them.
You know I always speak my mind, and I cannot bear the
idea of two young women travelling post by themselves.
It is highly improper. You must contrive to send somebody.
I have the greatest dislike in the world to that
sort of thing. -- Young women should always be properly
guarded and attended, according to their situation in
life. When my niece Georgiana went to Ramsgate last
%%211%%
summer, I made a point of her having two men servants
go with her. -- Miss Darcy, the daughter of Mr. Darcy,
of Pemberley, and Lady Anne, could not have appeared
with propriety in a different manner. -- I am excessively
attentive to all those things. You must send John with
the young ladies, Mrs. Collins. I am glad it occurred to
me to mention it; for it would really be discreditable to
\textit{you} to let them go alone.”

“My uncle is to send a servant for us.”

“Oh! -- Your uncle! -- He keeps a man-servant, does
he? -- I am very glad you have somebody who thinks of
those things. Where shall you change horses? -- Oh!
Bromley, of course. -- If you mention my name at the
Bell, you will be attended to.”

Lady Catherine had many other questions to ask
respecting their journey, and as she did not answer them
all herself, attention was necessary, which Elizabeth
believed to be lucky for her; or, with a mind so occupied,
she might have forgotten where she was. Reflection
must be reserved for solitary hours; whenever she was
alone, she gave way to it as the greatest relief; and
not a day went by without a solitary walk, in which
she might indulge in all the delight of unpleasant
recollections.

Mr. Darcy’s letter, she was in a fair way of soon knowing
by heart. She studied every sentence: and her feelings
towards its writer were at times widely different. When
she remembered the style of his address, she was still
full of indignation; but when she considered how unjustly
she had condemned and upbraided him, her anger was
turned against herself; and his disappointed feelings
became the object of compassion. His attachment excited
gratitude, his general character respect; but she could
not approve him; nor could she for a moment repent
her refusal, or feel the slightest inclination ever to see
him again. In her own past behaviour, there was a constant
source of vexation and regret; and in the unhappy
defects of her family a subject of yet heavier chagrin.
%%212%%
They were hopeless of remedy. Her father, contented
with laughing at them, would never exert himself to
restrain the wild giddiness of his youngest daughters; and
her mother, with manners so far from right herself, was
entirely insensible of the evil. Elizabeth had frequently
united with Jane in an endeavour to check the imprudence
of Catherine and Lydia; but while they were supported
by their mother’s indulgence, what chance could there be
of improvement? Catherine, weak-spirited, irritable, and
completely under Lydia’s guidance, had been always
affronted by their advice; and Lydia, self-willed and
careless, would scarcely give them a hearing. They were
ignorant, idle, and vain. While there was an officer in
Meryton, they would flirt with him; and while Meryton
was within a walk of Longbourn, they would be going
there for ever.

Anxiety on Jane’s behalf, was another prevailing concern,
and Mr. Darcy’s explanation, by restoring Bingley
to all her former good opinion, heightened the sense of
what Jane had lost. His affection was proved to have
been sincere, and his conduct cleared of all blame, unless
any could attach to the implicitness of his confidence in
his friend. How grievous then was the thought that,
of a situation so desirable in every respect, so replete
with advantage, so promising for happiness, Jane had
been deprived, by the folly and indecorum of her own
family!

When to these recollections was added the developement
of Wickham’s character, it may be easily believed that
the happy spirits which had seldom been depressed before,
were now so much affected as to make it almost impossible
for her to appear tolerably cheerful.

Their engagements at Rosings were as frequent during
the last week of her stay, as they had been at first. The
very last evening was spent there; and her Ladyship
again enquired minutely into the particulars of their
journey, gave them directions as to the best method of
packing, and was so urgent on the necessity of placing
%%213%%
gowns in the only right way, that Maria thought herself
obliged, on her return, to undo all the work of the morning,
and pack her trunk afresh.

When they parted, Lady Catherine, with great condescension,
wished them a good journey, and invited them
to come to Hunsford again next year; and Miss De Bourgh
exerted herself so far as to curtsey and hold out her hand
to both.
%%214%%

\Chapter{CHAPTER XV.}

On Saturday morning Elizabeth and Mr. Collins met
for breakfast a few minutes before the others appeared;
and he took the opportunity of paying the parting civilities
which he deemed indispensably necessary.

“I know not, Miss Elizabeth,” said he, “whether
Mrs. Coll\-ins has yet expressed her sense of your kindness
in coming to us, but I am very certain you will not leave
the house without receiving her thanks for it. The favour
of your company has been much felt, I assure you. We
know how little there is to tempt any one to our humble
abode. Our plain manner of living, our small rooms, and
few domestics, and the little we see of the world, must
make Hunsford extremely dull to a young lady like yourself;
but I hope you will believe us grateful for the
condescension, and that we have done every thing in our
power to prevent your spending your time unpleasantly.”

Elizabeth was eager with her thanks and assurances of
happiness. She had spent six weeks with great enjoyment;
and the pleasure of being with Charlotte, and the kind
attentions she had received, must make \textit{her} feel the obliged.
Mr. Collins was gratified; and with a more smiling
solemnity replied,

“It gives me the greatest pleasure to hear that you
have passed your time not disagreeably. We have
certainly done our best; and most fortunately having it
in our power to introduce you to very superior society,
and from our connection with Rosings, the frequent means
of varying the humble home scene, I think we may flatter
ourselves that your Hunsford visit cannot have been
entirely irksome. Our situation with regard to Lady
Catherine’s family is indeed the sort of extraordinary
%%215%%
advantage and blessing which few can boast. You see
on what a footing we are. You see how continually we are
engaged there. In truth I must acknowledge that, with
all the disadvantages of this humble parsonage, I should
not think any one abiding in it an object of compassion,
while they are sharers of our intimacy at Rosings.”

Words were insufficient for the elevation of his feelings;
and he was obliged to walk about the room, while Elizabeth
tried to unite civility and truth in a few short
sentences.

“You may, in fact, carry a very favourable report of
us into Hertfordshire, my dear cousin. I flatter myself
at least that you will be able to do so. Lady Catherine’s
great attentions to Mrs. Collins you have been a daily
witness of; and altogether I trust it does not appear
that your friend has drawn an unfortunate -- but on this
point it will be as well to be silent. Only let me assure
you, my dear Miss Elizabeth, that I can from my heart
most cordially wish you equal felicity in marriage. My
dear Charlotte and I have but one mind and one way of
thinking. There is in every thing a most remarkable
resemblance of character and ideas between us. We seem
to have been designed for each other.”

Elizabeth could safely say that it was a great happiness
where that was the case, and with equal sincerity could
add that she firmly believed and rejoiced in his domestic
comforts. She was not sorry, however, to have the recital
of them interrupted by the entrance of the lady from whom
they sprung. Poor Charlotte! -- it was melancholy to
leave her to such society! -- But she had chosen it with
her eyes open; and though evidently regretting that her
visitors were to go, she did not seem to ask for compassion.
Her home and her housekeeping, her parish and her
poultry, and all their dependent concerns, had not yet
lost their charms.

At length the chaise arrived, the trunks were fastened
on, the parcels placed within, and it was pronounced to
be ready. After an affectionate parting between the
%%216%%
friends, Elizabeth was attended to the carriage by Mr.
Collins, and as they walked down the garden, he was
commissioning her with his best respects to all her family,
not forgetting his thanks for the kindness he had received
at Longbourn in the winter, and his compliments to Mr.
and Mrs. Gardiner, though unknown. He then handed
her in, Maria followed, and the door was on the point
of being closed, when he suddenly reminded them, with
some consternation, that they had hitherto forgotten to
leave any message for the ladies of Rosings.

“But,” he added, “you will of course wish to have
your humble respects delivered to them, with your grateful
thanks for their kindness to you while you have been here.”

Elizabeth made no objection; -- the door was then
allowed to be shut, and the carriage drove off.

“Good gracious!” cried Maria, after a few minutes
silence, “it seems but a day or two since we first came! -- and
yet how many things have happened!”

“A great many indeed,” said her companion with
a sigh.

“We have dined nine times at Rosings, besides drinking
tea there twice! -- How much I shall have to tell!”

Elizabeth privately added, “And how much I shall
have to conceal.”

Their journey was performed without much conversation,
or any alarm; and within four hours of their
leaving Hunsford, they reached Mr. Gardiner’s house,
where they were to remain a few days.

Jane looked well, and Elizabeth had little opportunity
of studying her spirits, amidst the various engagements
which the kindness of her aunt had reserved for them.
But Jane was to go home with her, and at Longbourn
there would be leisure enough for observation.

It was not without an effort meanwhile that she could
wait even for Longbourn, before she told her sister of
Mr. Darcy’s proposals. To know that she had the power
of revealing what would so exceedingly astonish Jane,
%%217%%
and must, at the same time, so highly gratify whatever
of her own vanity she had not yet been able to reason
away, was such a temptation to openness as nothing could
have conquered, but the state of indecision in which she
remained, as to the extent of what she should communicate;
and her fear, if she once entered on the subject, of
being hurried into repeating something of Bingley, which
might only grieve her sister farther.
%%218%%

\Chapter{CHAPTER XVI.}

It was the second week in May, in which the three
young ladies set out together from Gracechurch-street, for
the town of ------ in Hertfordshire; and, as they drew
near the appointed inn where Mr. Bennet’s carriage was
to meet them, they quickly perceived, in token of the
coachman’s punctuality, both Kitty and Lydia looking
out of a dining room up stairs. These two girls had been
above an hour in the place, happily employed in visiting
an opposite milliner, watching the sentinel on guard, and
dressing a sallad and cucumber.

After welcoming their sisters, they triumphantly displayed
a table set out with such cold meat as an inn larder
usually affords, exclaiming, “Is not this nice? is not
this an agreeable surprise?”

“And we mean to treat you all,” added Lydia; “but
you must lend us the money, for we have just spent ours
at the shop out there.” Then shewing her purchases:
“Look here, I have bought this bonnet. I do not think
it is very pretty; but I thought I might as well buy it
as not. I shall pull it to pieces as soon as I get home,
and see if I can make it up any better.”

And when her sisters abused it as ugly, she added,
with perfect unconcern, “Oh! but there were two or
three much uglier in the shop; and when I have bought
some prettier-coloured satin to trim it with fresh, I think
it will be very tolerable. Besides, it will not much signify
what one wears this summer, after the ------shire have
left Meryton, and they are going in a fortnight.”

“Are they indeed?” cried Elizabeth, with the greatest
satisfaction.

“They are going to be encamped near Brighton; and
I do so want papa to take us all there for the summer!
It would be such a delicious scheme, and I dare say
%%219%%
would hardly cost any thing at all. Mamma would like
to go too of all things! Only think what a miserable
summer else we shall have!”

“Yes,” thought Elizabeth, “\textit{that} would be a delightful
sch\-eme, indeed, and completely do for us at once.
Good Heaven! Brighton, and a whole campful of soldiers,
to us, who have been overset already by one poor regiment
of militia, and the monthly balls of Meryton.”

“Now I have got some news for you,” said Lydia, as
they sat down to table. “What do you think? It is
excellent news, capital news, and about a certain person
that we all like.”

Jane and Elizabeth looked at each other, and the
waiter was told that he need not stay. Lydia laughed,
and said,

“Aye, that is just like your formality and discretion.
You thought the waiter must not hear, as if he cared!
I dare say he often hears worse things said than I am
going to say. But he is an ugly fellow! I am glad he is
gone. I never saw such a long chin in my life. Well, but
now for my news: it is about dear Wickham; too good
for the waiter, is not it? There is no danger of Wickham’s
marrying Mary King. There’s for you! She is gone down
to her uncle at Liverpool; gone to stay. Wickham is
safe.”

“And Mary King is safe!” added Elizabeth; “safe
from a connection imprudent as to fortune.”

“She is a great fool for going away, if she liked him.”

“But I hope there is no strong attachment on either
side,” said Jane.

“I am sure there is not on \textit{his}. I will answer for it
he never cared three straws about her. Who \textit{could} about
such a nasty little freckled thing?”

Elizabeth was shocked to think that, however incapable
of such coarseness of \textit{expression} herself, the coarseness
of the \textit{sentiment} was little other than her own breast had
formerly harboured and fancied liberal!

As soon as all had ate, and the elder ones paid, the
%%220%%
carriage was ordered; and after some contrivance, the
whole party, with all their boxes, workbags, and parcels,
and the unwelcome addition of Kitty’s and Lydia’s
purchases, were seated in it.

“How nicely we are crammed in!” cried Lydia. “I am
glad I bought my bonnet, if it is only for the fun of having
another bandbox! Well, now let us be quite comfortable
and snug, and talk and laugh all the way home. And
in the first place, let us hear what has happened to you
all, since you went away. Have you seen any pleasant
men? Have you had any flirting? I was in great hopes
that one of you would have got a husband before you came
back. Jane will be quite an old maid soon, I declare.
She is almost three and twenty! Lord, how ashamed
I should be of not being married before three and twenty!
My aunt Philips wants you so to get husbands, you can’t
think. She says Lizzy had better have taken Mr. Collins;
but \textit{I} do not think there would have been any fun in it.
Lord! how I should like to be married before any of you;
and then I would chaperon you about to all the balls.
Dear me! we had such a good piece of fun the other day
at Colonel Forster’s. Kitty and me were to spend the
day there, and Mrs. Forster promised to have a little
dance in the evening; (by the bye, Mrs. Forster and
me are \textit{such} friends!) and so she asked the two Harringtons
to come, but Harriet was ill, and so Pen was forced to
come by herself; and then, what do you think we did?
We dressed up Chamberlayne in woman’s clothes, on
purpose to pass for a lady, -- only think what fun! Not
a soul knew of it, but Col. and Mrs. Forster, and
Kitty and me, except my aunt, for we were forced to
borrow one of her gowns; and you cannot imagine how
well he looked! When Denny, and Wickham, and Pratt,
and two or three more of the men came in, they did not
know him in the least. Lord! how I laughed! and so
did Mrs. Forster. I thought I should have died. And
\textit{that} made the men suspect something, and then they soon
found out what was the matter.”
%%221%%

With such kind of histories of their parties and good
jokes, did Lydia, assisted by Kitty’s hints and additions,
endeavour to amuse her companions all the way to Longbourn.
Elizabeth listened as little as she could, but there
was no escaping the frequent mention of Wickham’s
name.

Their reception at home was most kind. Mrs. Bennet
rejoiced to see Jane in undiminished bea\-uty; and more
than once during dinner did Mr. Bennet say voluntarily
to Elizabeth,

“I am glad you are come back, Lizzy.”

Their party in the dining-room was large, for almost
all the Lucases came to meet Maria and hear the news:
and various were the subjects which occupied them;
lady Lucas was enquiring of Maria across the table, after
the welfare and poultry of her eldest daughter; Mrs. Bennet
was doubly engaged, on one hand collecting an account
of the present fashions from Jane, who sat some way
below her, and on the other, retailing them all to the
younger Miss Lucases; and Lydia, in a voice rather
louder than any other person’s, was enumerating the
various pleasures of the morning to any body who would
hear her.

“Oh! Mary,” said she, “I wish you had gone with
us, for we had such fun! as we went along, Kitty and
me drew up all the blinds, and pretended there was
nobody in the coach; and I should have gone so all the
way, if Kitty had not been sick; and when we got to
the George, I do think we behaved very handsomely, for
we treated the other three with the nicest cold luncheon
in the world, and if you would have gone, we would have
treated you too. And then when we came away it was
such fun! I thought we never should have got into the
coach. I was ready to die of laughter. And then we were
so merry all the way home! we talked and laughed so
loud, that any body might have heard us ten miles off!”

To this, Mary very gravely replied, “Far be it from
me, my dear sister, to depreciate such pleasures. They
%%222%%
would doubtless be congenial with the generality of female
minds. But I confess they would have no charms for \textit{me}.
I should infinitely prefer a book.”

But of this answer Lydia heard not a word. She seldom
listened to any body for more than half a minute, and
never attended to Mary at all.

In the afternoon Lydia was urgent with the rest of the
girls to walk to Meryton and see how every body went on;
but Elizabeth steadily opposed the scheme. It should
not be said, that the Miss Bennets could not be at home
half a day before they were in pursuit of the officers.
There was another reason too for her opposition. She
dreaded seeing Wickham again, and was resolved to avoid
it as long as possible. The comfort to \textit{her}, of the regiment’s
approaching removal, was indeed beyond expression. In
a fortnight they were to go, and once gone, she hoped
there could be nothing more to plague her on his account.

She had not been many hours at home, before she found
that the Brighton scheme, of which Lydia had given them
a hint at the inn, was under frequent discussion between
her parents. Elizabeth saw directly that her father had
not the smallest intention of yielding; but his answers
were at the same time so vague and equivocal, that her
mother, though often disheartened, had never yet despaired
of succeeding at last.
%%223%%

\Chapter{CHAPTER XVII.}

Elizabeth’s impatience to acquaint Jane with what
had happened could no longer be overcome; and at length
resolving to suppress every particular in which her sister
was concerned, and preparing her to be surprised, she
related to her the next morning the chief of the scene
between Mr. Darcy and herself.

Miss Bennet’s astonishment was soon lessened by the
strong sisterly partiality which made any admiration of
Elizabeth appear perfectly natural; and all surprise was
shortly lost in other feelings. She was sorry that Mr.
Darcy should have delivered his sentiments in a manner
so little suited to recommend them; but still more was
she grieved for the unhappiness which her sister’s refusal
must have given him.

“His being so sure of succeeding, was wrong,” said she;
“and certainly ought not to have appeared; but consider
how much it must increase his disappointment.”

“Indeed,” replied Elizabeth, “I am heartily sorry for
him; but he has other feelings which will probably soon
drive away his regard for me. You do not blame me,
however, for refusing him?”

“Blame you! Oh, no.”

“But you blame me for having spoken so warmly of
Wickham.”

“No -- I do not know that you were wrong in saying
what you did.”

“But you \textit{will} know it, when I have told you what
happened the very next day.”

She then spoke of the letter, repeating the whole of its
contents as far as they concerned George Wickham.
What a stroke was this for poor Jane! who would willingly
have gone through the world without believing that so
much wickedness existed in the whole race of mankind,
%%224%%
as was here collected in one individual. Nor was Darcy’s
vindication, though grateful to her feelings, capable of
consoling her for such discovery. Most earnestly did she
labour to prove the probability of error, and seek to clear
one, without involving the other.

“This will not do,” said Elizabeth. “You never will
be able to make both of them good for any thing. Take
your choice, but you must be satisfied with only one.
There is but such a quantity of merit between them;
just enough to make one good sort of man; and of late
it has been shifting about pretty much. For my part,
I am inclined to believe it all Mr. Darcy’s, but you shall
do as you chuse.”

It was some time, however, before a smile could be
extorted from Jane.

“I do not know when I have been more shocked,”
said she. “Wickham so very bad! It is almost past
belief. And poor Mr. Darcy! dear Lizzy, only consider
what he must have suffered. Such a disappointment!
and with the knowledge of your ill opinion too! and having
to relate such a thing of his sister! It is really too
distressing. I am sure you must feel it so.”

“Oh! no, my regret and compassion are all done away
by seeing you so full of both. I know you will do him
such ample justice, that I am growing every moment
more unconcerned and indifferent. Your profusion makes
me saving; and if you lament over him much longer,
my heart will be as light as a feather.”

“Poor Wickham; there is such an expression of goodness
in his countenance! such an openness and gentleness
in his manner.”

“There certainly was some great mismanagement in
the education of those two young men. One has got all
the goodness, and the other all the appearance of it.”

“I never thought Mr. Darcy so deficient in the \textit{appearance}
of it as you used to do.”

“And yet I meant to be uncommonly clever in taking
so decided a dislike to him, without any reason. It is
%%225%%
such a spur to one’s genius, such an opening for wit to
have a dislike of that kind. One may be continually
abusive without saying any thing just; but one cannot
be always laughing at a man without now and then
stumbling on something witty.”

“Lizzy, when you first read that letter, I am sure you
could not treat the matter as you do now.”

“Indeed I could not. I was uncomfortable en\-ough.
I was very uncomfortable, I may say unhappy. And with
no one to speak to, of what I felt, no Jane to comfort
me and say that I had not been so very weak and vain
and nonsensical as I knew I had! Oh! how I wanted you!”

“How unfortunate that you should have used such very
strong expressions in speaking of Wickham to Mr. Darcy,
for now they \textit{do} appear wholly undeserved.”

“Certainly. But the misfortune of speaking with
bitterness, is a most natural consequence of the prejudices
I had been encouraging. There is one point, on which
I want your advice. I want to be told whether I ought,
or ought not to make our acquaintance in general understand
Wickham’s character.”

Miss Bennet paused a little and then replied, “Surely
there can be no occasion for exposing him so dreadfully.
What is your own opinion?”

“That it ought not to be attempted. Mr. Darcy has
not authorised me to make his communication public.
On the contrary every particular relative to his sister,
was meant to be kept as much as possible to myself;
and if I endeavour to undeceive people as to the rest
of his conduct, who will believe me? The general prejudice
against Mr. Darcy is so violent, that it would be
the death of half the good people in Meryton, to attempt
to place him in an amiable light. I am not equal to it.
Wickham will soon be gone; and therefore it will not
signify to anybody here, what he really is. Sometime
hence it will be all found out, and then we may laugh
at their stupidity in not knowing it before. At present
I will say nothing about it.”
%%226%%

“You are quite right. To have his errors made public
might ruin him for ever. He is now perhaps sorry for
what he has done, and anxious to re-establish a character.
We must not make him desperate.”

The tumult of Elizabeth’s mind was allayed by this
conversation. She had got rid of two of the secrets which
had weighed on her for a fortnight, and was certain of
a willing listener in Jane, whenever she might wish to
talk again of either. But there was still something lurking
behind, of which prudence forbad the disclosure. She dared
not relate the other half of Mr. Darcy’s letter, nor explain to
her sister how sincerely she had been valued by his friend.
Here was knowledge in which no one could partake;
and she was sensible that nothing less than a perfect
understanding between the parties could justify her in
throwing off this last incumbrance of mystery. “And
then,” said she, “if that very improbable event should
ever take place, I shall merely be able to tell what Bingley
may tell in a much more agreeable manner himself. The
liberty of communication cannot be mine till it has lost
all its value!”

She was now, on being settled at home, at leisure to
observe the real state of her sister’s spirits. Jane was
not happy. She still cherished a very tender affection
for Bingley. Having never even fancied herself in love
before, her regard had all the warmth of first attachment,
and from her age and disposition, greater steadiness than
first attachments often boast; and so fervently did she
value his remembrance, and prefer him to every other
man, that all her good sense, and all her attention to the
feelings of her friends, were requisite to check the indulgence
of those regrets, which must have been injurious
to her own health and their tranquillity.

“Well, Lizzy,” said Mrs. Bennet one day, “what is
your opinion \textit{now} of this sad business of Jane’s? For my
part, I am determined never to speak of it again to anybody.
I told my sister Philips so the other day. But
I cannot find out that Jane saw any thing of him in London.
%%227%%
Well, he is a very undeserving young man -- and I do not
suppose there is the least chance in the world of her ever
getting him now. There is no talk of his coming to Netherfield
again in the summer; and I have enquired of every
body too, who is likely to know.”

“I do not believe that he will ever live at Netherfield
any more.”

“Oh, well! it is just as he chooses. Nobody wants
him to come. Though I shall always say that he used
my daughter extremely ill; and if I was her, I would
not have put up with it. Well, my comfort is, I am sure
Jane will die of a broken heart, and then he will be sorry
for what he has done.”

But as Elizabeth could not receive comfort from any
such expectation, she made no answer.

“Well, Lizzy,” continued her mother soon afterwards,
“and so the Collinses live very comfortable, do they?
Well, well, I only hope it will last. And what sort of table
do they keep? Charlotte is an excellent manager, I dare
say. If she is half as sharp as her mother, she is saving
enough. There is nothing extravagant in \textit{their} housekeeping,
I dare say.”

“No, nothing at all.”

“A great deal of good management, depend upon it.
Yes, yes. \textit{They} will take care not to outrun their income.
\textit{They} will never be distressed for money. Well, much good
may it do them! And so, I suppose, they often talk of
having Longbourn when your father is dead. They look
upon it quite as their own, I dare say, whenever that
happens.”

“It was a subject which they could not mention before
me.”

“No. It would have been strange if they had. But
I make no doubt, they often talk of it between themselves.
Well, if they can be easy with an estate that is not lawfully
their own, so much the better. \textit{I} should be ashamed of
having one that was only entailed on me.”
%%228%%

\Chapter{CHAPTER XVIII.}

The first week of their return was soon gone. The
second began. It was the last of the regiment’s stay in
Meryton, and all the young ladies in the neighbourhood
were drooping apace. The dejection was almost universal.
The elder Miss Bennets alone were still able to eat, drink,
and sleep, and pursue the usual course of their employments.
Very frequently were they reproached for this
insensibility by Kitty and Lydia, whose own misery was
extreme, and who could not comprehend such hard-heartedness
in any of the family.

“Good Heaven! What is to become of us! What are
we to do!” would they often exclaim in the bitterness
of woe. “How can you be smiling so, Lizzy?”

Their affectionate mother shared all their grief; she
remembered what she had herself endured on a similar
occasion, five and twenty years ago.

“I am sure,” said she, “I cried for two days together
when Colonel Millar’s regiment went away. I thought
I should have broke my heart.”

“I am sure I shall break \textit{mine},” said Lydia.

“If one could but go to Brighton!” observed Mrs.
Bennet.

“Oh, yes! -- if one could but go to Brighton! But papa
is so disagreeable.”

“A little sea-bathing would set me up for ever.”

“And my aunt Philips is sure it would do \textit{me} a great
deal of good,” added Kitty.

Such were the kind of lamentations resounding perpetually
through Longbourn-house. Elizabeth tried to
be diverted by them; but all sense of pleasure was lost
in shame. She felt anew the justice of Mr. Darcy’s objections;
and never had she before been so much disposed
to pardon his interference in the views of his friend.
%%229%%

But the gloom of Lydia’s prospect was shortly cleared
away; for she received an invitation from Mrs. Forster,
the wife of the Colonel of the regiment, to accompany her
to Brighton. This invaluable friend was a very young
woman, and very lately married. A resemblance in good
humour and good spirits had recommended her and Lydia
to each other, and out of their \textit{three} months’ acquaintance
they had been intimate \textit{two}.

The rapture of Lydia on this occasion, her adoration of
Mrs. Forster, the delight of Mrs. Bennet, and the mortification
of Kitty, are scarcely to be described. Wholly inattentive
to her sister’s feelings, Lydia flew about the house in
restless ecstacy, calling for every one’s congratulations, and
laughing and talking with more violence than ever; whilst
the luckless Kitty continued in the parlour repining at her
fate in terms as unreasonable as her accent was peevish.

“I cannot see why Mrs. Forster should not ask \textit{me} as
well as Lydia,” said she, “though I am \textit{not} her particular
friend. I have just as much right to be asked as she has,
and more too, for I am two years older.”

In vain did Elizabeth attempt to make her reasonable,
and Jane to make her resigned. As for Elizabeth herself,
this invitation was so far from exciting in her the same
feelings as in her mother and Lydia, that she considered
it as the death-warrant of all possibility of common sense
for the latter; and detestable as such a step must make
her were it known, she could not help secretly advising
her father not to let her go. She represented to him all
the improprieties of Lydia’s general behaviour, the little
advantage she could derive from the friendship of such
a woman as Mrs. Forster, and the probability of her being
yet more imprudent with such a companion at Brighton,
where the temptations must be greater than at home.
He heard her attentively, and then said,

“Lydia will never be easy till she has exposed herself
in some public place or other, and we can never expect her
to do it with so little expense or inconvenience to her
family as under the present circumstances.”
%%230%%

“If you were aware,” said Elizabeth, “of the very great
disadvantage to us all, which must arise from the public
notice of Lydia’s unguarded and imprudent manner; nay,
which has already arisen from it, I am sure you would
judge differently in the affair.”

“Already arisen!” repeated Mr. Bennet. “What, has
she frightened away some of your lovers? Poor little
Lizzy! But do not be cast down. Such squeamish youths
as cannot bear to be connected with a little absurdity,
are not worth a regret. Come, let me see the list of the
pitiful fellows who have been kept aloof by Lydia’s folly.”

“Indeed you are mistaken. I have no such injuries
to resent. It is not of peculiar, but of general evils,
which I am now complaining. Our importance, our
respectability in the world, must be affected by the wild
volatility, the assurance and disdain of all restraint which
mark Lydia’s character. Excuse me -- for I must speak
plainly. If you, my dear father, will not take the trouble
of checking her exuberant spirits, and of teaching her that
her present pursuits are not to be the business of her life,
she will soon be beyond the reach of amendment. Her
character will be fixed, and she will, at sixteen, be the most
determined flirt that ever made herself and her family
ridiculous. A flirt too, in the worst and meanest degree
of flirtation; without any attraction beyond youth and
a tolerable person; and from the ignorance and emptiness
of her mind, wholly unable to ward off any portion of
that universal contempt which her rage for admiration
will excite. In this danger Kitty is also comprehended.
She will follow wherever Lydia leads. Vain, ignorant,
idle, and absolutely uncontrouled! Oh! my dear father,
can you suppose it possible that they will not be censured
and despised wherever they are known, and that their
sisters will not be often involved in the disgrace?”

Mr. Bennet saw that her whole heart was in the subject;
and affectionately taking her hand, said in reply,

“Do not make yourself uneasy, my love. Wherever
you and Jane are known, you must be respected and
%%231%%
valued; and you will not appear to less advantage for
having a couple of -- or I may say, three very silly sisters.
We shall have no peace at Longbourn if Lydia does not
go to Brighton. Let her go then. Colonel Forster is
a sensible man, and will keep her out of any real mischief;
and she is luckily too poor to be an object of prey to
any body. At Brighton she will be of less importance even
as a common flirt than she has been here. The officers
will find women better worth their notice. Let us hope,
therefore, that her being there may teach her her own
insignificance. At any rate, she cannot grow many degrees
worse, without authorizing us to lock her up for the rest
of her life.”

With this answer Elizabeth was forced to be content;
but her own opinion continued the same, and she left
him disappointed and sorry. It was not in her nature,
however, to increase her vexations, by dwelling on them.
She was confident of having performed her duty, and to
fret over unavoidable evils, or augment them by anxiety,
was no part of her disposition.

Had Lydia and her mother known the substance of her
conference with her father, their indignation would hardly
have found expression in their united volubility. In
Lydia’s imagination, a visit to Brighton comprised every
possibility of earthly happiness. She saw with the creative
eye of fancy, the streets of that gay bathing place covered
with officers. She saw herself the object of attention, to
tens and to scores of them at present unknown. She saw
all the glories of the camp; its tents stretched forth in
beauteous uniformity of lines, crowded with the young
and the gay, and dazzling with scarlet; and to complete
the view, she saw herself seated beneath a tent, tenderly
flirting with at least six officers at once.

Had she known that her sister sought to tear her from
such prospects and such realities as these, what would
have been her sensations? They could have been understood
only by her mother, who might have felt nearly
the same. Lydia’s going to Brighton was all that
%%232%%
consoled her for the melancholy conviction of her husband’s
never intending to go there himself.

But they were entirely ignorant of what had passed;
and their raptures continued with little intermission to the
very day of Lydia’s leaving home.

Elizabeth was now to see Mr. Wickham for the last
time. Having been frequently in company with him since
her return, agitation was pretty well over; the agitations
of former partiality entirely so. She had even learnt to
detect, in the very gentleness which had first delighted
her, an affectation and a sameness to disgust and weary.
In his present behaviour to herself, moreover, she had
a fresh source of displeasure, for the inclination he soon
testified of renewing those attentions which had marked
the early part of their acquaintance, could only serve,
after what had since passed, to provoke her. She lost
all concern for him in finding herself thus selected as the
object of such idle and frivolous gallantry; and while she
steadily repressed it, could not but feel the reproof contained
in his believing, that however long, and for whatever
cause, his attentions had been withdrawn, her vanity
would be gratified and her preference secured at any time
by their renewal.

On the very last day of the regiment’s remaining in
Meryton, he dined with others of the officers at Longbourn;
and so little was Elizabeth disposed to part from him
in good humour, that on his making some enquiry as to
the manner in which her time had passed at Hunsford,
she mentioned Colonel Fitzwilliam’s and Mr. Darcy’s
having both spent three weeks at Rosings, and asked
him if he were acquainted with the former.

He looked surprised, displeased, alarmed; but with
a moment’s recollection and a returning smile, replied,
that he had formerly seen him often; and after observing
that he was a very gentlemanlike man, asked her how she
had liked him. Her answer was warmly in his favour.
With an air of indifference he soon afterwards added,
“How long did you say that he was at Rosings?”
%%233%%

“Nearly three weeks.”

“And you saw him frequently?”

“Yes, almost every day.”

“His manners are very different from his cou\-sin’s.”

“Yes, very different. But I think Mr. Darcy improves
on acquaintance.”

“Indeed!” cried Wickham with a look which did not
escape her. “And pray may I ask?” but checking himself,
he added in a gayer tone, “Is it in address that he
improves? Has he deigned to add ought of civility to
his ordinary style? for I dare not hope,” he continued
in a lower and more serious tone, “that he is improved
in essentials.”

“Oh, no!” said Elizabeth. “In essentials, I believe,
he is very much what he ever was.”

While she spoke, Wickham looked as if scarcely knowing
whether to rejoice over her words, or to distrust their
meaning. There was a something in her countenance
which made him listen with an apprehensive and anxious
attention, while she added,

“When I said that he improved on acquaintance, I did
not mean that either his mind or manners were in a state
of improvement, but that from knowing him better, his
disposition was better understood.”

Wickham’s alarm now appeared in a heightened complexion
and agitated look; for a few minutes he was
silent; till, shaking off his embarrassment, he turned to
her again, and said in the gentlest of accents,

“You, who so well know my feelings towards Mr. Darcy,
will readily comprehend how sincerely I must rejoice that
he is wise enough to assume even the \textit{appearance} of what
is right. His pride, in that direction, may be of service,
if not to himself, to many others, for it must deter him
from such foul misconduct as I have suffered by. I only
fear that the sort of cautiousness, to which you, I imagine,
have been alluding, is merely adopted on his visits to his
aunt, of whose good opinion and judgment he stands
much in awe. His fear of her, has always operated,
%%234%%
I know, when they were together; and a good deal is to
be imputed to his wish of forwarding the match with
Miss De Bourgh, which I am certain he has very much
at heart.”

Elizabeth could not repress a smile at this, but she
answered only by a slight inclination of the head. She
saw that he wanted to engage her on the old subject of
his grievances, and she was in no humour to indulge
him. The rest of the evening passed with the \textit{appearance},
on his side, of usual cheerfulness, but with no farther
attempt to distinguish Elizabeth; and they parted at last
with mutual civility, and possibly a mutual desire of never
meeting again.

When the party broke up, Lydia returned with Mrs.
Forster to Meryton, from whence they were to set out
early the next morning. The separation between her and
her family was rather noisy than pathetic. Kitty was the
only one who shed tears; but she did weep from vexation
and envy. Mrs. Bennet was diffuse in her good wishes
for the felicity of her daughter, and impressive in her
injunctions that she would not miss the opportunity of
enjoying herself as much as possible; advice, which there
was every reason to believe would be attended to; and
in the clamorous happiness of Lydia herself in bidding
farewell, the more gentle adieus of her sisters were uttered
without being heard.
%%235%%

\Chapter{CHAPTER XIX.}

Had Elizabeth’s opinion been all drawn from her own
family, she could not have formed a very pleasing picture
of conjugal felicity or domestic comfort. Her father
captivated by youth and beauty, and that appearance
of good humour, which youth and beauty generally give,
had married a woman whose weak understanding and
illiberal mind, had very early in their marriage put an
end to all real affection for her. Respect, esteem, and
confidence, had vanished for ever; and all his views of
domestic happiness were overthrown. But Mr. Bennet
was not of a disposition to seek comfort for the disappointment
which his own imprudence had brought on, in any
of those pleasures which too often console the unfortunate
for their folly or their vice. He was fond of the country
and of books; and from these tastes had arisen his
principal enjoyments. To his wife he was very little
otherwise indebted, than as her ignorance and folly had
contributed to his amusement. This is not the sort of
happiness which a man would in general wish to owe
to his wife; but where other powers of entertainment
are wanting, the true philosopher will derive benefit from
such as are given.

Elizabeth, however, had never been blind to the impropriety
of her father’s behaviour as a husband. She
had always seen it with pain; but respecting his abilities,
and grateful for his affectionate treatment of herself, she
endeavoured to forget what she could not overlook, and
to banish from her thoughts that continual breach of
conjugal obligation and decorum which, in exposing his
wife to the contempt of her own children, was so highly
reprehensible. But she had never felt so strongly as now,
the disadvantages which must attend the children of so
unsuitable a marriage, nor ever been so fully aware of the
%%236%%
evils arising from so ill-judged a direction of talents;
talents which rightly used, might at least have preserved
the respectability of his daughters, even if incapable of
enlarging the mind of his wife.

When Elizabeth had rejoiced over Wickham’s departure,
she found little other cause for satisfaction in the loss of
the regiment. Their parties abroad were less varied than
before; and at home she had a mother and sister whose
constant repinings at the dulness of every thing around
them, threw a real gloom over their domestic circle;
and, though Kitty might in time regain her natural degree
of sense, since the disturbers of her brain were removed,
her other sister, from whose disposition greater evil might
be apprehended, was likely to be hardened in all her folly
and assurance, by a situation of such double danger as
a watering place and a camp. Upon the whole, therefore,
she found, what has been sometimes found before, that
an event to which she had looked forward with impatient
desire, did not in taking place, bring all the satisfaction
she had promised herself. It was consequently necessary
to name some other period for the commencement of
actual felicity; to have some other point on which her
wishes and hopes might be fixed, and by again enjoying
the pleasure of anticipation, console herself for the present,
and prepare for another disappointment. Her tour to
the Lakes was now the object of her happiest thoughts;
it was her best consolation for all the uncomfortable
hours, which the discontentedness of her mother and
Kitty made inevitable; and could she have included
Jane in the scheme, every part of it would have been
perfect.

“But it is fortunate,” thought she, “that I have something
to wish for. Were the whole arrangement complete,
my disappointment would be certain. But here, by
carrying with me one ceaseless source of regret in my
sister’s absence, I may reasonably hope to have all my
expectations of pleasure realized. A scheme of which
every part promises delight, can never be successful; and
%%237%%
general disappointment is only warded off by the defence
of some little peculiar vexation.”

When Lydia went away, she promised to write very
often and very minutely to her mother and Kitty; but
her letters were always long expected, and always very
short. Those to her mother, contained little else, than
that they were just returned from the library, where
such and such officers had attended them, and where she
had seen such beautiful ornaments as made her quite
wild; that she had a new gown, or a new parasol, which
she would have described more fully, but was obliged
to leave off in a violent hurry, as Mrs. Forster called her,
and they were going to the camp; -- and from her correspondence
with her sister, there was still less to be learnt -- for
her letters to Kitty, though rather longer, were much
too full of lines under the words to be made public.

After the first fortnight or three weeks of her absence,
health, good humour and cheerfulness began to re-appear
at Longbourn. Everything wore a happier aspect. The
families who had been in town for the winter came back
again, and summer finery and summer engagements arose.
Mrs. Bennet was restored to her usual querulous serenity,
and by the middle of June Kitty was so much recovered
as to be able to enter Meryton without tears; an event
of such happy promise as to make Elizabeth hope, that
by the following Christmas, she might be so tolerably
reasonable as not to mention an officer above once a day,
unless by some cruel and malicious arrangement at the
war-office, another regiment should be quartered in
Meryton.

The time fixed for the beginning of their Northern tour
was now fast approaching; and a fortnight only was
wanting of it, when a letter arrived from Mrs. Gardiner,
which at once delayed its commencement and curtailed
its extent. Mr. Gardiner would be prevented by business
from setting out till a fortnight later in July, and must
be in London again within a month; and as that left
too short a period for them to go so far, and see so much
%%238%%
as they had proposed, or at least to see it with the leisure
and comfort they had built on, they were obliged to give
up the Lakes, and substitute a more contracted tour; and,
according to the present plan, were to go no farther
northward than Derbyshire. In that county, there was
enough to be seen, to occupy the chief of their three weeks;
and to Mrs. Gardiner it had a peculiarly strong attraction.
The town where she had formerly passed some years of
her life, and where they were now to spend a few days,
was probably as great an object of her curiosity, as all
the celebrated beauties of Matlock, Chatsworth, Dovedale,
or the Peak.

Elizabeth was excessively disappointed; she had set
her heart on seeing the Lakes; and still thought there
might have been time enough. But it was her business
to be satisfied -- and certainly her temper to be happy;
and all was soon right again.

With the mention of Derbyshire, there were many ideas
connected. It was impossible for her to see the word
without thinking of Pemberley and its owner. “But
surely,” said she, “I may enter his county with impunity,
and rob it of a few petrified spars without his perceiving
me.”

The period of expectation was now doubled. Four
weeks were to pass away before her uncle and aunt’s
arrival. But they did pass away, and Mr. and Mrs.
Gardiner, with their four children, did at length appear
at Longbourn. The children, two girls of six and eight
years old, and two younger boys, were to be left under
the particular care of their cousin Jane, who was the
general favourite, and whose steady sense and sweetness
of temper exactly adapted her for attending to them in
every way -- teaching them, playing with them, and loving
them.

The Gardiners staid only one night at Longbourn, and set
off the next morning with Elizabeth in pursuit of novelty
and amusement. One enjoyment was certain -- that of
suitableness as companions; a suitableness which
%%239%%
comprehended health and temper to bear inconveniences --
cheerfulness to enhance every pleasure -- and affection and
intelligence, which might supply it among themselves if
there were disappointments abroad.

It is not the object of this work to give a description
of Derbyshire, nor of any of the remarkable places through
which their route thither lay; Oxford, Blenheim, Warwick,
Kenelworth, Birmingham, \&c. are sufficiently known.
A small part of Derbyshire is all the present concern.
To the little town of Lambton, the scene of Mrs. Gardiner’s
former residence, and where she had lately learned that
some acquaintance still remained, they bent their steps,
after having seen all the principal wonders of the country;
and within five miles of Lambton, Elizabeth found from
her aunt, that Pemberley was situated. It was not in
their direct road, nor more than a mile or two out of it.
In talking over their route the evening before, Mrs.
Gardiner expressed an inclination to see the place again.
Mr. Gardiner declared his willingness, and Elizabeth was
applied to for her approbation.

“My love, should not you like to see a place of which
you have heard so much?” said her aunt. “A place too,
with which so many of your acquaintance are connected.
Wickham passed all his youth there, you know.”

Elizabeth was distressed. She felt that she had no
business at Pemberley, and was obliged to assume a disinclination
for seeing it. She must own that she was tired
of great houses; after going over so many, she really had
no pleasure in fine carpets or satin curtains.

Mrs. Gardiner abused her stupidity. “If it were merely
a fine house richly furnished,” said she, “I should not
care about it myself; but the grounds are delightful.
They have some of the finest woods in the country.”

Elizabeth said no more -- but her mind could not
acquiesce. The possibility of meeting Mr. Darcy, while
viewing the place, instantly occurred. It would be
dreadful! She blushed at the very idea; and thought
it would be better to speak openly to her aunt, than to
%%240%%
run such a risk. But against this, there were objections;
and she finally resolved that it could be the last resource,
if her private enquiries as to the absence of the family,
were unfavourably answered.

Accordingly, when she retired at night, she asked the
chambermaid whether Pemberley were not a very fine
place, what was the name of its proprietor, and with no
little alarm, whether the family were down for the summer.
A most welcome negative followed the last question -- and
her alarms being now removed, she was at leisure to
feel a great deal of curiosity to see the house herself;
and when the subject was revived the next morning, and
she was again applied to, could readily answer, and with
a proper air of indifference, that she had not really any
dislike to the scheme.

To Pemberley, therefore, they were to go.

%%241%%

\Part{VOL III.}
%%243%%

\Chapter{CHAPTER I.}

Elizabeth, as they drove along, watched for the first
appearance of Pemberley Woods with some perturbation;
and when at length they turned in at the lodge, her
spirits were in a high flutter.

The park was very large, and contained great variety
of ground. They entered it in one of its lowest points,
and drove for some time through a beautiful wood,
stretching over a wide extent.

Elizabeth’s mind was too full for conversation, but she
saw and admired every remarkable spot and point of
view. They gradually ascended for half a mile, and then
found themselves at the top of a considerable eminence,
where the wood ceased, and the eye was instantly caught
by Pemberley House, situated on the opposite side of
a valley, into which the road with some abruptness wound.
It was a large, handsome, stone building, standing well on
rising ground, and backed by a ridge of high woody hills; -- and
in front, a stream of some natural importance was
swelled into greater, but without any artificial appearance.
Its banks were neither formal, nor falsely adorned. Elizabeth
was delighted. She had never seen a place for which
nature had done more, or where natural beauty had been
so little counteracted by an awkward taste. They were
all of them warm in their admiration; and at that moment
she felt, that to be mistress of Pemberley might be
something!

They descended the hill, crossed the bridge, and drove
to the door; and, while examining the nearer aspect of
the house, all her apprehensions of meeting its owner
returned. She dreaded lest the chambermaid had been
%%245%%
mistaken. On applying to see the place, they were
admitted into the hall; and Elizabeth, as they waited
for the housekeeper, had leisure to wonder at her being
where she was.

The housekeeper came; a respectable-looking, elderly
woman, much less fine, and more civil, than she had any
notion of finding her. They followed her into the dining-parlour.
It was a large, well-proportioned room, handsomely
fitted up. Elizabeth, after slightly surveying it,
went to a window to enjoy its prospect. The hill, crowned
with wood, from which they had descended, receiving
increased abruptness from the distance, was a beautiful
object. Every disposition of the ground was good; and
she looked on the whole scene, the river, the trees scattered
on its banks, and the winding of the valley, as far as she
could trace it, with delight. As they passed into other
rooms, these objects were taking different positions; but
from every window there were beauties to be seen. The
rooms were lofty and handsome, and their furniture
suitable to the fortune of their proprietor; but Elizabeth
saw, with admiration of his taste, that it was neither
gaudy nor uselessly fine; with less of splendor, and more
real elegance, than the furniture of Rosings.

“And of this place,” thought she, “I might have been
mistress! With these rooms I might now have been
familiarly acquainted! Instead of viewing them as a
stranger, I might have rejoiced in them as my own, and
welcomed to them as visitors my uncle and aunt. -- But
no,” -- recollecting herself, -- “that could never be: my
uncle and aunt would have been lost to me: I should
not have been allowed to invite them.”

This was a lucky recollection -- it saved her from something
like regret.

She longed to enquire of the housekeeper, whether her
master were really absent, but had not courage for it.
At length, however, the question was asked by her uncle;
and she turned away with alarm, while Mrs. Reynolds
replied, that he was, adding, “but we expect him
%%246%%
to-morrow, with a large party of friends.” How rejoiced
was Elizabeth that their own journey had not by any
circumstance been delayed a day!

Her aunt now called her to look at a picture. She
approached, and saw the likeness of Mr. Wickham suspended,
amongst several other miniatures, over the mantle-piece.
Her aunt asked her, smilingly, how she liked it.
The housekeeper came forward, and told them it was the
picture of a young gentleman, the son of her late master’s
steward, who had been brought up by him at his own
expence. -- “He is now gone into the army,” she added,
“but I am afraid he has turned out very wild.”

Mrs. Gardiner looked at her niece with a smile, but
Elizabeth could not return it.

“And that,” said Mrs. Reynolds, pointing to another
of the miniatures, “is my master -- and very like him.
It was drawn at the same time as the other -- about eight
years ago.”

“I have heard much of your master’s fine person,”
said Mrs. Gardiner, looking at the picture; “it is a handsome
face. But, Lizzy, you can tell us whether it is like
or not.”

Mrs. Reynolds’s respect for Elizabeth seemed to increase
on this intimation of her knowing her master.

“Does that young lady know Mr. Darcy?”

Elizabeth coloured, and said -- “A little.”

“And do not you think him a very handsome gentleman,
Ma’am?”

“Yes, very handsome.”

“I am sure \textit{I} know none so handsome; but in the
gallery up stairs you will see a finer, larger picture of him
than this. This room was my late master’s favourite
room, and these miniatures are just as they used to be
then. He was very fond of them.”

This accounted to Elizabeth for Mr. Wickham’s being
among them.

Mrs. Reynolds then directed their attention to one of
Miss Darcy, drawn when she was only eight years old.
%%247%%

“And is Miss Darcy as handsome as her bro\-ther?”
said Mr. Gardiner.

“Oh! yes -- the handsomest young lady that ever was
seen; and so accomplished! -- She plays and sings all day
long. In the next room is a new instrument just come
down for her -- a present from my master; she comes here
to-morrow with him.”

Mr. Gardiner, whose manners were easy and pleasant,
encouraged her communicativeness by his questions and
remarks; Mrs. Reynolds, either from pride or attachment,
had evidently great pleasure in talking of her
master and his sister.

“Is your master much at Pemberley in the course of
the year?”

“Not so much as I could wish, Sir; but I dare say he
may spend half his time here; and Miss Darcy is always
down for the summer months.”

“Except,” thought Elizabeth, “when she goes to
Ramsgate.”

“If your master would marry, you might see more of
him.”

“Yes, Sir; but I do not know when \textit{that} will be. I do
not know who is good enough for him.”

Mr. and Mrs. Gardiner smiled. Elizabeth could not
help saying, “It is very much to his credit, I am sure, that
you should think so.”

“I say no more than the truth, and what every body
will say that knows him,” replied the other. Elizabeth
thought this was going pretty far; and she listened with
increasing astonishment as the housekeeper added,
“I have never had a cross word from him in my
life, and I have known him ever since he was four years
old.”

This was praise, of all others most extraordinary, most
opposite to her ideas. That he was not a good-tempered
man, had been her firmest opinion. Her keenest attention
was awakened; she longed to hear more, and was grateful
to her uncle for saying,
%%248%%

“There are very few people of whom so much can be
said. You are lucky in having such a master.”

“Yes, Sir, I know I am. If I was to go through the
world, I could not meet with a better. But I have always
observed, that they who are good-natured when children,
are good-natured when they grow up; and he was always
the sweetest-tempered, most generous-hearted, boy in the
world.”

Elizabeth almost stared at her. -- “Can this be Mr.
Darcy!” thought she.

“His father was an excellent man,” said Mrs. Gardiner.

“Yes, Ma’am, that he was indeed; and his son will be
just like him -- just as affable to the poor.”

Elizabeth listened, wondered, doubted, and was impatient
for more. Mrs. Reynolds could interest her on no
other point. She related the subject of the pictures, the
dimensions of the rooms, and the price of the furniture,
in vain. Mr. Gardiner, highly amused by the kind of
family prejudice, to which he attributed her excessive
commendation of her master, soon led again to the subject;
and she dwelt with energy on his many merits, as they
proceeded together up the great staircase.

“He is the best landlord, and the best master,” said
she, “that ever lived. Not like the wild young men
now-a-days, who think of nothing but themselves. There
is not one of his tenants or servants but what will give
him a good name. Some people call him proud; but
I am sure I never saw any thing of it. To my fancy, it is
only because he does not rattle away like other young
men.”

“In what an amiable light does this place him!”
thought Elizabeth.

“This fine account of him,” whispered her aunt, as they
walked, “is not quite consistent with his behaviour to our
poor friend.”

“Perhaps we might be deceived.”

“That is not very likely; our authority was too good.”

On reaching the spacious lobby above, they were shewn
%%249%%
into a very pretty sitting-room, lately fitted up with
greater elegance and lightness than the apartments below;
and were informed that it was but just done, to give
pleasure to Miss Darcy, who had taken a liking to the room,
when last at Pemberley.

“He is certainly a good brother,” said Elizabeth, as
she walked towards one of the windows.

Mrs. Reynolds anticipated Miss Darcy’s delight, when
she should enter the room. “And this is always the way
with him,” she added. -- “Whatever can give his sister
any pleasure, is sure to be done in a moment. There is
nothing he would not do for her.”

The picture gallery, and two or three of the principal
bed-rooms, were all that remained to be shewn. In the
former were many good paintings; but Elizabeth knew
nothing of the art; and from such as had been already
visible below, she had willingly turned to look at some
drawings of Miss Darcy’s, in crayons, whose subjects were
usually more interesting, and also more intelligible.

In the gallery there were many family portraits, but
they could have little to fix the attention of a stranger.
Elizabeth walked on in quest of the only face whose
features would be known to her. At last it arrested her -- and
she beheld a striking resemblance of Mr. Darcy, with
such a smile over the face, as she remembered to have
sometimes seen, when he looked at her. She stood several
minutes before the picture in earnest contemplation, and
returned to it again before they quitted the gallery.
Mrs. Reynolds informed them, that it had been taken in
his father’s life time.

There was certainly at this moment, in Elizabeth’s
mind, a more gentle sensation towards the original, than
she had ever felt in the height of their acquaintance.
The commendation bestowed on him by Mrs. Reynolds
was of no trifling nature. What praise is more valuable
than the praise of an intelligent servant? As a brother,
a landlord, a master, she considered how many people’s
happiness were in his guardianship! -- How much of
%%250%%
pleasure or pain it was in his power to bestow! -- How
much of good or evil must be done by him! Every idea
that had been brought forward by the housekeeper was
favourable to his character, and as she stood before the
canvas, on which he was represented, and fixed his eyes
upon herself, she thought of his regard with a deeper
sentiment of gratitude than it had ever raised before;
she remembered its warmth, and softened its impropriety
of expression.

When all of the house that was open to general inspection
had been seen, they returned down stairs, and taking
leave of the housekeeper, were consigned over to the
gardener, who met them at the hall door.

As they walked across the lawn towards the river,
Elizabeth turned back to look again; her uncle and aunt
stopped also, and while the former was conjecturing as
to the date of the building, the owner of it himself suddenly
came forward from the road, which led behind it to the
stables.

They were within twenty yards of each other, and so
abrupt was his appearance, that it was impossible to avoid
his sight. Their eyes instantly met, and the cheeks of
each were overspread with the deepest blush. He absolutely
started, and for a moment seemed immoveable from
surprise; but shortly recovering himself, advanced towards
the party, and spoke to Elizabeth, if not in terms of perfect
composure, at least of perfect civility.

She had instinctively turned away; but, stopping on
his approach, received his compliments with an embarrassment
impossible to be overcome. Had his first
appearance, or his resemblance to the picture they had
just been examining, been insufficient to assure the other
two that they now saw Mr. Darcy, the gardener’s expression
of surprise, on beholding his master, must immediately
have told it. They stood a little aloof while he was talking
to their niece, who, astonished and confused, scarcely
dared lift her eyes to his face, and knew not what answer
she returned to his civil enquiries after her family. Amazed
%%251%%
at the alteration in his manner since they last parted,
every sentence that he uttered was increasing her embarrassment;
and every idea of the impropriety of her being
found there, recurring to her mind, the few minutes in
which they continued together, were some of the most
uncomfortable of her life. Nor did he seem much more
at ease; when he spoke, his accent had none of its usual
sedateness; and he repeated his enquiries as to the time
of her having left Longbourn, and of her stay in Derbyshire,
so often, and in so hurried a way, as plainly spoke
the distraction of his thoughts.

At length, every idea seemed to fail him; and, after
standing a few moments without saying a word, he suddenly
recollected himself, and took leave.

The others then joined her, and expressed their admiration
of his figure; but Elizabeth heard not a word, and,
wholly engrossed by her own feelings, followed them in
silence. She was overpowered by shame and vexation.
Her coming there was the most unfortunate, the most
ill-judged thing in the world! How strange must it
appear to him! In what a disgraceful light might it not
strike so vain a man! It might seem as if she had purposely
thrown herself in his way again! Oh! why did she
come? or, why did he thus come a day before he was
expected? Had they been only ten minutes sooner, they
should have been beyond the reach of his discrimination,
for it was plain that he was that moment arrived, that
moment alighted from his horse or his carriage. She
blushed again and again over the perverseness of the
meeting. And his behaviour, so strikingly altered, -- what
could it mean? That he should even speak to her
was amazing! -- but to speak with such civility, to enquire
after her family! Never in her life had she seen his
manners so little dignified, never had he spoken with such
gentleness as on this unexpected meeting. What a contrast
did it offer to his last address in Rosing’s Park, when he
put his letter into her hand! She knew not what to think,
nor how to account for it.
%%252%%

They had now entered a beautiful walk by the side of
the water, and every step was bringing forward a nobler
fall of ground, or a finer reach of the woods to which they
were approaching; but it was some time before Elizabeth
was sensible of any of it; and, though she answered
mechanically to the repeated appeals of her uncle and
aunt, and seemed to direct her eyes to such objects as
they pointed out, she distinguished no part of the scene.
Her thoughts were all fixed on that one spot of Pemberley
House, whichever it might be, where Mr. Darcy then was.
She longed to know what at that moment was passing
in his mind; in what manner he thought of her, and
whether, in defiance of every thing, she was still dear to
him. Perhaps he had been civil, only because he felt
himself at ease; yet there had been \textit{that} in his voice,
which was not like ease. Whether he had felt more of
pain or of pleasure in seeing her, she could not tell, but he
certainly had not seen her with composure.

At length, however, the remarks of her companions on
her absence of mind roused her, and she felt the necessity
of appearing more like herself.

They entered the woods, and bidding adieu to the river
for a while, ascended some of the higher grounds; whence,
in spots where the opening of the trees gave the eye power
to wander, were many charming views of the valley, the
opposite hills, with the long range of woods overspreading
many, and occasionally part of the stream. Mr. Gardiner
expressed a wish of going round the whole Park, but
feared it might be beyond a walk. With a triumphant
smile, they were told, that it was ten miles round. It
settled the matter; and they pursued the accustomed
circuit; which brought them again, after some time, in
a descent among hanging woods, to the edge of the water,
in one of its narrowest parts. They crossed it by a simple
bridge, in character with the general air of the scene;
it was a spot less adorned than any they had yet visited;
and the valley, here contracted into a glen, allowed room
only for the stream, and a narrow walk amidst the rough
%%253%%
coppice-wood which bordered it. Elizabeth longed to
explore its windings; but when they had crossed the
bridge, and perceived their distance from the house,
Mrs. Gardiner, who was not a great walker, could go no
farther, and thought only of returning to the carriage as
quickly as possible. Her niece was, therefore, obliged to
submit, and they took their way towards the house on
the opposite side of the river, in the nearest direction;
but their progress was slow, for Mr. Gardiner, though
seldom able to indulge the taste, was very fond of fishing,
and was so much engaged in watching the occasional
appearance of some trout in the water, and talking to
the man about them, that he advanced but little. Whilst
wandering on in this slow manner, they were again surprised,
and Elizabeth’s astonishment was quite equal to
what it had been at first, by the sight of Mr. Darcy
approaching them, and at no great distance. The walk
being here less sheltered than on the other side, allowed
them to see him before they met. Elizabeth, however
astonished, was at least more prepared for an interview
than before, and resolved to appear and to speak with
calmness, if he really intended to meet them. For a few
moments, indeed, she felt that he would probably strike
into some other path. This idea lasted while a turning
in the walk concealed him from their view; the turning
past, he was immediately before them. With a glance
she saw, that he had lost none of his recent civility; and,
to imitate his politeness, she began, as they met, to admire
the beauty of the place; but she had not got beyond the
words “delightful,” and “charming,” when some unlucky
recollections obtruded, and she fancied that praise
of Pemberley from her, might be mischievously construed.
Her colour changed, and she said no more.

Mrs. Gardiner was standing a little behind; and on
her pausing, he asked her, if she would do him the honour
of introducing him to her friends. This was a stroke of
civility for which she was quite unprepared; and she
could hardly suppress a smile, at his being now seeking
%%254%%
the acquaintance of some of those very people, against
whom his pride had revolted, in his offer to herself. “What
will be his surprise,” thought she, “when he knows who
they are! He takes them now for people of fashion.”

The introduction, however, was immediately made;
and as she named their relationship to herself, she stole
a sly look at him, to see how he bore it; and was not
without the expectation of his decamping as fast as he
could from such disgraceful companions. That he was
\textit{surprised} by the connexion was evident; he sustained it
however with fortitude, and so far from going away,
turned back with them, and entered into conversation
with Mr. Gardiner. Elizabeth could not but be pleased,
could not but triumph. It was consoling, that he should
know she had some relations for whom there was no need
to blush. She listened most attentively to all that passed
between them, and gloried in every expression, every
sentence of her uncle, which marked his intelligence, his
taste, or his good manners.

The conversation soon turned upon fishing, and she
heard Mr. Darcy invite him, with the greatest civility, to
fish there as often as he chose, while he continued in the
neighbourhood, offering at the same time to supply him
with fishing tackle, and pointing out those parts of the
stream where there was usually most sport. Mrs. Gardiner,
who was walking arm in arm with Elizabeth, gave her
a look expressive of her wonder. Elizabeth said nothing,
but it gratified her exceedingly; the compliment must
be all for herself. Her astonishment, however, was
extreme; and continually was she repeating, “Why is
he so altered? From what can it proceed? It cannot
be for \textit{me}, it cannot be for \textit{my} sake that his manners are
thus softened. My reproofs at Hunsford could not work
such a change as this. It is impossible that he should
still love me.”

After walking some time in this way, the two ladies in
front, the two gentlemen behind, on resuming their places,
after descending to the brink of the river for the better
%%255%%
inspection of some curious water-plant, there chanced to
be a little alteration. It originated in Mrs. Gardiner, who,
fatigued by the exercise of the morning, found Elizabeth’s
arm inadequate to her support, and consequently preferred
her husband’s. Mr. Darcy took her place by her niece, and
they walked on together. After a short silence, the lady
first spoke. She wished him to know that she had been
assured of his absence before she came to the place, and
accordingly began by observing, that his arrival had been
very unexpected -- “for your housekeeper,” she added,
“informed us that you would certainly not be here till
to-morrow; and indeed, before we left Bakewell, we understood
that you were not immediately expected in the
country.” He acknowledged the truth of it all; and said
that business with his steward had occasioned his coming
forward a few hours before the rest of the party with
whom he had been travelling. “They will join me early
to-morrow,” he continued, “and among them are some
who will claim an acquaintance with you, -- Mr. Bingley
and his sisters.”

Elizabeth answered only by a slight bow. Her thoughts
were instantly driven back to the time when Mr. Bingley’s
name had been last mentioned between them; and if she
might judge from his complexion, \textit{his} mind was not very
differently engaged.

“There is also one other person in the party,” he continued
after a pause, “who more particularly wishes to
be known to you, -- Will you allow me, or do I ask too
much, to introduce my sister to your acquaintance during
your stay at Lambton?”

The surprise of such an application was great indeed;
it was too great for her to know in what manner she
acceded to it. She immediately felt that whatever desire
Miss Darcy might have of being acquainted with her,
must be the work of her brother, and without looking
farther, it was satisfactory; it was gratifying to know that
his resentment had not made him think really ill of her.

They now walked on in silence; each of them deep in
%%256%%
thought. Elizabeth was not comfortable; that was
impossible; but she was flattered and pleased. His wish
of introducing his sister to her, was a compliment of the
highest kind. They soon outstripped the others, and
when they had reached the carriage, Mr. and Mrs. Gardiner
were half a quarter of a mile behind.

He then asked her to walk into the house -- but she
declared herself not tired, and they stood together on the
lawn. At such a time, much might have been said, and
silence was very awkward. She wanted to talk, but
there seemed an embargo on every subject. At last she
recollected that she had been travelling, and they talked
of Matlock and Dove Dale with great perseverance. Yet
time and her aunt moved slowly -- and her patience and
her ideas were nearly worn out before the tete-a-tete was
over. On Mr. and Mrs. Gardiner’s coming up, they were
all pressed to go into the house and take some refreshment;
but this was declined, and they parted on each
side with the utmost politeness. Mr. Darcy handed the
ladies into the carriage, and when it drove off, Elizabeth
saw him walking slowly towards the house.

The observations of her uncle and aunt now began; and
each of them pronounced him to be infinitely superior to
any thing they had expected. “He is perfectly well
behaved, polite, and unassuming,” said her uncle.

“There \textit{is} something a little stately in him to be sure,”
replied her aunt, “but it is confined to his air, and is
not unbecoming. I can now say with the housekeeper,
that though some people may call him proud, \textit{I} have seen
nothing of it.”

“I was never more surprised than by his behaviour to
us. It was more than civil; it was really attentive; and
there was no necessity for such attention. His acquaintance
with Elizabeth was very trifling.”

“To be sure, Lizzy,” said her aunt, “he is not so handsome
as Wickham; or rather he has not Wickham’s
countenance, for his features are perfectly good. But how
came you to tell us that he was so disagreeable?”
%%257%%

Elizabeth excused herself as well as she could; said
that she had liked him better when they met in Kent than
before, and that she had never seen him so pleasant as
this morning.

“But perhaps he may be a little whimsical in his
civilities,” replied her uncle. “Your great men often are;
and therefore I shall not take him at his word about
fishing, as he might change his mind another day, and
warn me off his grounds.”

Elizabeth felt that they had entirely mistaken his
character, but said nothing.

“From what we have seen of him,” continued Mrs. Gardiner,
“I really should not have thought that he could
have behaved in so cruel a way by any body, as he has
done by poor Wickham. He has not an ill-natured look.
On the contrary, there is something pleasing about his
mouth when he speaks. And there is something of dignity
in his countenance, that would not give one an unfavourable
idea of his heart. But to be sure, the good lady who
shewed us the house, did give him a most flaming character!
I could hardly help laughing aloud sometimes. But he
is a liberal master, I suppose, and \textit{that} in the eye of a servant
comprehends every virtue.”

Elizabeth here felt herself called on to say something
in vindication of his behaviour to Wickham; and therefore
gave them to understand, in as guarded a manner
as she could, that by what she had heard from his relations
in Kent, his actions were capable of a very different
construction; and that his character was by no means
so faulty, nor Wickham’s so amiable, as they had been
considered in Hertfordshire. In confirmation of this, she
related the particulars of all the pecuniary transactions
in which they had been connected, without actually
naming her authority, but stating it to be such as might
be relied on.

Mrs. Gardiner was surprised and concerned; but as they
were now approaching the scene of her former pleasures,
every idea gave way to the charm of recollection; and
%%258%%
she was too much engaged in pointing out to her husband
all the interesting spots in its environs, to think of any
thing else. Fatigued as she had been by the morning’s
walk, they had no sooner dined than she set off again in
quest of her former acquaintance, and the evening was
spent in the satisfactions of an intercourse renewed after
many years discontinuance.

The occurrences of the day were too full of interest to
leave Elizabeth much attention for any of these new
friends; and she could do nothing but think, and think
with wonder, of Mr. Darcy’s civility, and above all, of his
wishing her to be acquainted with his sister.
%%259%%

\Chapter{CHAPTER II.}

Elizabeth had settled it that Mr. Darcy would bring
his sister to visit her, the very day after her reaching
Pemberley; and was consequently resolved not to be out
of sight of the inn the whole of that morning. But her
conclusion was false; for on the very morning after their
own arrival at Lambton, these visitors came. They had
been walking about the place with some of their new
friends, and were just returned to the inn to dress themselves
for dining with the same family, when the sound
of a carriage drew them to a window, and they saw
a gentleman and lady in a curricle, driving up the street.
Elizabeth immediately recognising the livery, guessed
what it meant, and imparted no small degree of surprise
to her relations, by acquainting them with the honour
which she expected. Her uncle and aunt were all amazement;
and the embarrassment of her manner as she
spoke, joined to the circumstance itself, and many of the
circumstances of the preceding day, opened to them a new
idea on the business. Nothing had ever suggested it
before, but they now felt that there was no other way of
accounting for such attentions from such a quarter, than
by supposing a partiality for their niece. While these
newly-born notions were passing in their heads, the perturbation
of Elizabeth’s feelings was every moment
increasing. She was quite amazed at her own discomposure;
but amongst other causes of disquiet, she dreaded
lest the partiality of the brother should have said too
much in her favour; and more than commonly anxious
to please, she naturally suspected that every power of
pleasing would fail her.

She retreated from the window, fearful of being seen;
and as she walked up and down the room, endeavouring
%%260%%
to compose herself, saw such looks of enquiring surprise
in her uncle and aunt, as made every thing worse.

Miss Darcy and her brother appeared, and this formidable
introduction took place. With astonishment did
Elizabeth see, that her new acquaintance was at least
as much embarrassed as herself. Since her being at
Lambton, she had heard that Miss Darcy was exceedingly
proud; but the observation of a very few minutes convinced
her, that she was only exceedingly shy. She found
it difficult to obtain even a word from her beyond a
monosyllable.

Miss Darcy was tall, and on a larger scale than Elizabeth;
and, though little more than sixteen, her figure
was formed, and her appearance womanly and graceful.
She was less handsome than her brother, but there was
sense and good humour in her face, and her manners were
perfectly unassuming and gentle. Elizabeth, who had
expected to find in her as acute and unembarrassed an
observer as ever Mr. Darcy had been, was much relieved
by discerning such different feelings.

They had not been long together, before Darcy told her
that Bingley was also coming to wait on her; and she
had barely time to express her satisfaction, and prepare
for such a visitor, when Bingley’s quick step was heard
on the stairs, and in a moment he entered the room. All
Elizabeth’s anger against him had been long done away;
but, had she still felt any, it could hardly have stood its
ground against the unaffected cordiality with which he
expressed himself, on seeing her again. He enquired in
a friendly, though general way, after her family, and looked
and spoke with the same good-humoured ease that he had
ever done.

To Mr. and Mrs. Gardiner he was scarcely a less interesting
personage than to herself. They had long wished to
see him. The whole party before them, indeed, excited
a lively attention. The suspicions which had just arisen
of Mr. Darcy and their niece, directed their observation
towards each with an earnest, though guarded, enquiry;
%%261%%
and they soon drew from those enquiries the full conviction
that one of them at least knew what it was to love. Of
the lady’s sensations they remained a little in doubt;
but that the gentleman was overflowing with admiration
was evident enough.

Elizabeth, on her side, had much to do. She wanted
to ascertain the feelings of each of her visitors, she wanted
to compose her own, and to make herself agreeable to all;
and in the latter object, where she feared most to fail,
she was most sure of success, for those to whom she
endeavoured to give pleasure were prepossessed in her
favour. Bingley was ready, Georgiana was eager, and
Darcy determined, to be pleased.

In seeing Bingley, her thoughts naturally flew to her
sister; and oh! how ardently did she long to know,
whether any of his were directed in a like manner.
Sometimes she could fancy, that he talked less than on
former occasions, and once or twice pleased herself with
the notion that as he looked at her, he was trying to trace
a resemblance. But, though this might be imaginary, she
could not be deceived as to his behaviour to Miss Darcy,
who had been set up as a rival of Jane. No look appeared
on either side that spoke particular regard. Nothing
occurred between them that could justify the hopes of
his sister. On this point she was soon satisfied; and two
or three little circumstances occurred ere they parted,
which, in her anxious interpretation, denoted a recollection
of Jane, not untinctured by tenderness, and a wish of
saying more that might lead to the mention of her, had
he dared. He observed to her, at a moment when the
others were talking together, and in a tone which had
something of real regret, that it “was a very long time
since he had had the pleasure of seeing her;” and, before
she could reply, he added, “It is above eight months.
We have not met since the 26th of November, when we
were all dancing together at Netherfield.”

Elizabeth was pleased to find his memory so exact;
and he afterwards took occasion to ask her, when
%%262%%
unattended to by any of the rest, whether \textit{all} her sisters
were at Longbourn. There was not much in the question,
nor in the preceding remark, but there was a look and
a manner which gave them meaning.

It was not often that she could turn her eyes on Mr.
Darcy himself; but, whenever she did catch a glimpse,
she saw an expression of general complaisance, and in all
that he said, she heard an accent so far removed from
hauteur or disdain of his companions, as convinced her
that the improvement of manners which she had yesterday
witnessed, however temporary its existence might prove,
had at least outlived one day. When she saw him thus
seeking the acquaintance, and courting the good opinion
of people, with whom any intercourse a few months ago
would have been a disgrace; when she saw him thus civil,
not only to herself, but to the very relations whom he had
openly disdained, and recollected their last lively scene
in Hunsford Parsonage, the difference, the change was
so great, and struck so forcibly on her mind, that she
could hardly restrain her astonishment from being visible.
Never, even in the company of his dear friends at Netherfield,
or his dignified relations at Rosings, had she seen
him so desirous to please, so free from self-consequence,
or unbending reserve as now, when no importance could
result from the success of his endeavours, and when even
the acquaintance of those to whom his attentions were
addressed, would draw down the ridicule and censure of
the ladies both of Netherfield and Rosings.

Their visitors staid with them above half an hour, and
when they arose to depart, Mr. Darcy called on his sister
to join him in expressing their wish of seeing Mr. and
Mrs. Gardiner, and Miss Bennet, to dinner at Pemberley,
before they left the country. Miss Darcy, though with
a diffidence which marked her little in the habit of giving
invitations, readily obeyed. Mrs. Gardiner looked at her
niece, desirous of knowing how \textit{she}, whom the invitation
most concerned, felt disposed as to its acceptance, but
Elizabeth had turned away her head. Presuming,
%%263%%
however, that this studied avoidance spoke rather a momentary
embarrassment, than any dislike of the proposal, and
seeing in her husband, who was fond of society, a perfect
willingness to accept it, she ventured to engage for her
attendance, and the day after the next was fixed on.

Bingley expressed great pleasure in the certainty of
seeing Elizabeth again, having still a great deal to say
to her, and many enquiries to make after all their Hertfordshire
friends. Elizabeth, construing all this into a wish
of hearing her speak of her sister, was pleased; and on
this account, as well as some others, found herself, when
their visitors left them, capable of considering the last
half hour with some satisfaction, though while it was
passing, the enjoyment of it had been little. Eager to
be alone, and fearful of enquiries or hints from her uncle
and aunt, she staid with them only long enough to hear
their favourable opinion of Bingley, and then hurried
away to dress.

But she had no reason to fear Mr. and Mrs. Gardiner’s
curiosity; it was not their wish to force her communication.
It was evident that she was much better acquainted
with Mr. Darcy than they had before any idea
of; it was evident that he was very much in love with
her. They saw much to interest, but nothing to justify
enquiry.

Of Mr. Darcy it was now a matter of anxiety to think
well; and, as far as their acquaintance reached, there was
no fault to find. They could not be untouched by his
politeness, and had they drawn his character from their
own feelings, and his servant’s report, without any reference
to any other account, the circle in Hertfordshire to
which he was known, would not have recognised it for
Mr. Darcy. There was now an interest, however, in
believing the housekeeper; and they soon became sensible,
that the authority of a servant who had known him since
he was four years old, and whose own manners indicated
respectability, was not to be hastily rejected. Neither
had any thing occurred in the intelligence of their Lambton
%%264%%
friends, that could materially lessen its weight. They had
nothing to accuse him of but pride; pride he probably
had, and if not, it would certainly be imputed by the
inhabitants of a small market-town, where the family
did not visit. It was acknowledged, however, that he
was a liberal man, and did much good among the poor.

With respect to Wickham, the travellers soon found that
he was not held there in much estimation; for though
the chief of his concerns, with the son of his patron, were
imperfectly understood, it was yet a well known fact that,
on his quitting Derbyshire, he had left many debts behind
him, which Mr. Darcy afterwards discharged.

As for Elizabeth, her thoughts were at Pemberley this
evening more than the last; and the evening, though as
it passed it seemed long, was not long enough to determine
her feelings towards \textit{one} in that mansion; and she
lay awake two whole hours, endeavouring to make them
out. She certainly did not hate him. No; hatred had
vanished long ago, and she had almost as long been
ashamed of ever feeling a dislike against him, that could
be so called. The respect created by the conviction of
his valuable qualities, though at first unwillingly admitted,
had for some time ceased to be repugnant to her feelings;
and it was now heightened into somewhat of a friendlier
nature, by the testimony so highly in his favour, and
bringing forward his disposition in so amiable a light,
which yesterday had produced. But above all, above
respect and esteem, there was a motive within her of
good will which could not be overlooked. It was
gratitude. -- Gratitude, not merely for having once loved her,
but for loving her still well enough, to forgive all the
petulance and acrimony of her manner in rejecting him,
and all the unjust accusations accompanying her rejection.
He who, she had been persuaded, would avoid her as his
greatest enemy, seemed, on this accidental meeting, most
eager to preserve the acquaintance, and without any indelicate
display of regard, or any peculiarity of manner,
where their two selves only were concerned, was soliciting
%%265%%
the good opinion of her friends, and bent on making her
known to his sister. Such a change in a man of so much
pride, excited not only astonishment but gratitude -- for
to love, ardent love, it must be attributed; and as such
its impression on her was of a sort to be encouraged, as
by no means unpleasing, though it could not be exactly
defined. She respected, she esteemed, she was grateful
to him, she felt a real interest in his welfare; and she
only wanted to know how far she wished that welfare to
depend upon herself, and how far it would be for the
happiness of both that she should employ the power,
which her fancy told her she still possessed, of bringing
on the renewal of his addresses.

It had been settled in the evening, between the aunt
and niece, that such a striking civility as Miss Darcy’s,
in coming to them on the very day of her arrival at
Pemberley, for she had reached it only to a late breakfast,
ought to be imitated, though it could not be equalled,
by some exertion of politeness on their side; and, consequently,
that it would be highly expedient to wait on
her at Pemberley the following morning. They were,
therefore, to go. -- Elizabeth was pleased, though, when she
asked herself the reason, she had very little to say in reply.

Mr. Gardiner left them soon after breakfast. The fishing
scheme had been renewed the day before, and a positive
engagement made of his meeting some of the gentlemen at
Pemberley by noon.
%%266%%

\Chapter{CHAPTER III.}

Convinced as Elizabeth now was that Miss Bingley’s
dislike of her had originated in jealousy, she could not
help feeling how very unwelcome her appearance at
Pemberley must be to her, and was curious to know with
how much civility on that lady’s side, the acquaintance
would now be renewed.

On reaching the house, they were shewn through the
hall into the saloon, whose northern aspect rendered it
delightful for summer. Its windows opening to the ground,
admitted a most refreshing view of the high woody hills
behind the house, and of the beautiful oaks and Spanish
chesnuts which were scattered over the intermediate lawn.

In this room they were received by Miss Darcy, who
was sitting there with Mrs. Hurst and Miss Bingley, and
the lady with whom she lived in London. Georgiana’s
reception of them was very civil; but attended with all
that embarrassment which, though proceeding from shyness
and the fear of doing wrong, would easily give to
those who felt themselves inferior, the belief of her being
proud and reserved. Mrs. Gardiner and her niece, however,
did her justice, and pitied her.

By Mrs. Hurst and Miss Bingley, they were noticed only
by a curtsey; and on their being seated, a pause, awkward
as such pauses must always be, succeeded for a few
moments. It was first broken by Mrs. Annesley, a genteel,
agreeable-looking woman, whose endeavour to introduce
some kind of discourse, proved her to be more truly well
bred than either of the others; and between her and
Mrs. Gardiner, with occasional help from Elizabeth, the
conversation was carried on. Miss Darcy looked as if she
wished for courage enough to join in it; and sometimes
did venture a short sentence, when there was least danger
of its being heard.
%%267%%

Elizabeth soon saw that she was herself closely watched
by Miss Bingley, and that she could not speak a word,
especially to Miss Darcy, without calling her attention.
This observation would not have prevented her from
trying to talk to the latter, had they not been seated at
an inconvenient distance; but she was not sorry to be
spared the necessity of saying much. Her own thoughts
were employing her. She expected every moment that
some of the gentlemen would enter the room. She wished,
she feared that the master of the house might be amongst
them; and whether she wished or feared it most, she could
scarcely determine. After sitting in this manner a quarter
of an hour, without hearing Miss Bingley’s voice, Elizabeth
was roused by receiving from her a cold enquiry after the
health of her family. She answered with equal indifference
and brevity, and the other said no more.

The next variation which their visit afforded was produced
by the entrance of servants with cold meat, cake,
and a variety of all the finest fruits in season; but this
did not take place till after many a significant look and
smile from Mrs. Annesley to Miss Darcy had been given,
to remind her of her post. There was now employment
for the whole party; for though they could not all talk,
they could all eat; and the beautiful pyramids of grapes,
nectarines, and peaches, soon collected them round the
table.

While thus engaged, Elizabeth had a fair opportunity
of deciding whether she most feared or wished for the
appearance of Mr. Darcy, by the feelings which prevailed
on his entering the room; and then, though but a moment
before she had believed her wishes to predominate, she
began to regret that he came.

He had been some time with Mr. Gardiner, who, with
two or three other gentlemen from the house, was engaged
by the river, and had left him only on learning that the
ladies of the family intended a visit to Georgiana that
morning. No sooner did he appear, than Elizabeth wisely
resolved to be perfectly easy and unembarrassed; --
%%268%%
a resolution the more necessary to be made, but perhaps
not the more easily kept, because she saw that the suspicions
of the whole party were awakened against them,
and that there was scarcely an eye which did not watch
his behaviour when he first came into the room. In no
countenance was attentive curiosity so strongly marked
as in Miss Bingley’s, in spite of the smiles which overspread
her face whenever she spoke to one of its objects; for
jealousy had not yet made her desperate, and her attentions
to Mr. Darcy were by no means over. Miss Darcy,
on her brother’s entrance, exerted herself much more to
talk; and Elizabeth saw that he was anxious for his
sister and herself to get acquainted, and forwarded, as
much as possible, every attempt at conversation on either
side. Miss Bingley saw all this likewise; and, in the
imprudence of anger, took the first opportunity of saying,
with sneering civility,

“Pray, Miss Eliza, are not the ------shire militia removed
from Meryton? They must be a great loss to \textit{your}
family.”

In Darcy’s presence she dared not mention Wickham’s
name; but Elizabeth instantly comprehended that he was
uppermost in her thoughts; and the various recollections
connected with him gave her a moment’s distress; but,
exerting herself vigorously to repel the ill-natured attack,
she presently answered the question in a tolerably disengaged
tone. While she spoke, an involuntary glance
shewed her Darcy with an heightened complexion, earnestly
looking at her, and his sister overcome with confusion,
and unable to lift up her eyes. Had Miss Bingley known
what pain she was then giving her beloved friend, she
undoubtedly would have refrained from the hint; but she
had merely intended to discompose Elizabeth, by bringing
forward the idea of a man to whom she believed her
partial, to make her betray a sensibility which might
injure her in Darcy’s opinion, and perhaps to remind the
latter of all the follies and absurdities, by which some part
of her family were connected with that corps. Not a
%%269%%
syllable had ever reached her of Miss Darcy’s meditated
elopement. To no creature had it been revealed, where
secresy was possible, except to Elizabeth; and from all
Bingley’s connections her brother was particularly anxious
to conceal it, from that very wish which Elizabeth had
long ago attributed to him, of their becoming hereafter
her own. He had certainly formed such a plan, and
without meaning that it should affect his endeavour to
separate him from Miss Bennet, it is probable that it
might add something to his lively concern for the welfare
of his friend.

Elizabeth’s collected behaviour, however, soon quieted
his emotion; and as Miss Bingley, vexed and disappointed,
dared not approach nearer to Wickham, Georgiana also
recovered in time, though not enough to be able to speak
any more. Her brother, whose eye she feared to meet,
scarcely recollected her interest in the affair, and the very
circumstance which had been designed to turn his thoughts
from Elizabeth, seemed to have fixed them on her more,
and more cheerfully.

Their visit did not continue long after the question
and answer above-mentioned; and while Mr. Darcy was
attending them to their carriage, Miss Bingley was venting
her feelings in criticisms on Elizabeth’s person, behaviour,
and dress. But Georgiana would not join her. Her
brother’s recommendation was enough to ensure her
favour: his judgment could not err, and he had spoken
in such terms of Elizabeth, as to leave Georgiana without
the power of finding her otherwise than lovely and amiable.
When Darcy returned to the saloon, Miss Bingley could
not help repeating to him some part of what she had been
saying to his sister.

“How very ill Eliza Bennet looks this morning, Mr.
Darcy,” she cried; “I never in my life saw any one so
much altered as she is since the winter. She is grown
so brown and coarse! Louisa and I were agreeing that
we should not have known her again.”

However little Mr. Darcy might have liked such an
%%270%%
address, he contented himself with coolly replying, that
he perceived no other alteration than her being rather
tanned, -- no miraculous consequence of travelling in the
summer.

“For my own part,” she rejoined, “I must confess that
I never could see any beauty in her. Her face is too thin;
her complexion has no brilliancy; and her features are
not at all handsome. Her nose wants character; there is
nothing marked in its lines. Her teeth are tolerable, but
not out of the common way; and as for her eyes, which
have sometimes been called so fine, I never could perceive
any thing extraordinary in them. They have a sharp,
shrewish look, which I do not like at all; and in her
air altogether, there is a self-sufficiency without fashion,
which is intolerable.”

Persuaded as Miss Bingley was that Darcy admired
Elizabeth, this was not the best method of recommending
herself; but angry people are not always wise; and in
seeing him at last look somewhat nettled, she had all
the success she expected. He was resolutely silent however;
and, from a determination of making him speak,
she continued,

“I remember, when we first knew her in Hertfordshire,
how amazed we all were to find that she was a reputed
beauty; and I particularly recollect your saying one
night, after they had been dining at Netherfield, ‘\textit{She}
a beauty! -- I should as soon call her mother a wit.’ But
afterwards she seemed to improve on you, and I believe
you thought her rather pretty at one time.”

“Yes,” replied Darcy, who could contain himself no
longer, “but \textit{that} was only when I first knew her, for it is
many months since I have considered her as one of the
handsomest women of my acquaintance.”

He then went away, and Miss Bingley was left to all
the satisfaction of having forced him to say what gave
no one any pain but herself.

Mrs. Gardiner and Elizabeth talked of all that had
occurred, during their visit, as they returned, except what
%%271%%
had particularly interested them both. The looks and
behaviour of every body they had seen were discussed,
except of the person who had mostly engaged their attention.
They talked of his sister, his friends, his house, his
fruit, of every thing but himself; yet Elizabeth was
longing to know what Mrs. Gardiner thought of him,
and Mrs. Gardiner would have been highly gratified by
her niece’s beginning the subject.
%%272%%

\Chapter{CHAPTER IV.}

Elizabeth had been a good deal disappointed in not
finding a letter from Jane, on their first arrival at Lambton;
and this disappointment had been renewed on each of
the mornings that had now been spent there; but on the
third, her repining was over, and her sister justified by
the receipt of two letters from her at once, on one of which
was marked that it had been missent elsewhere. Elizabeth
was not surprised at it, as Jane had written the direction
remarkably ill.

They had just been preparing to walk as the letters
came in; and her uncle and aunt, leaving her to enjoy
them in quiet, set off by themselves. The one missent
must be first attended to; it had been written five days
ago. The beginning contained an account of all their
little parties and engagements, with such news as the
country afforded; but the latter half, which was dated
a day later, and written in evident agitation, gave more
important intelligence. It was to this effect:

“Since writing the above, dearest Lizzy, something has
occurred of a most unexpected and serious nature; but
I am afraid of alarming you -- be assured that we are all
well. What I have to say relates to poor Lydia. An
express came at twelve last night, just as we were all gone
to bed, from Colonel Forster, to inform us that she was
gone off to Scotland with one of his officers; to own the
truth, with Wickham! -- Imagine our surprise. To Kitty,
however, it does not seem so wholly unexpected. I am
very, very sorry. So imprudent a match on both sides! -- But
I am willing to hope the best, and that his character
has been misunderstood. Thoughtless and indiscreet I can
easily believe him, but this step (and let us rejoice over it)
marks nothing bad at heart. His choice is disinterested
at least, for he must know my father can give her nothing.
%%273%%
Our poor mother is sadly grieved. My father bears it
better. How thankful am I, that we never let them know
what has been said against him; we must forget it ourselves.
They were off Saturday night about twelve, as
is conjectured, but were not missed till yesterday morning
at eight. The express was sent off directly. My dear
Lizzy, they must have passed within ten miles of us.
Colonel Forster gives us reason to expect him here
soon. Lydia left a few lines for his wife, informing
her of their intention. I must conclude, for I cannot be
long from my poor mother. I am afraid you will not
be able to make it out, but I hardly know what I have
written.”

Without allowing herself time for consideration, and
scar\-cely knowing what she felt, Elizabeth on finishing this
letter, instantly seized the other, and opening it with the
utmost impatience, read as follows: it had been written
a day later than the conclusion of the first.

“By this time, my dearest sister, you have received
my hurried letter; I wish this may be more intelligible,
but though not confined for time, my head is so bewildered
that I cannot answer for being coherent. Dearest Lizzy,
I hardly know what I would write, but I have bad news
for you, and it cannot be delayed. Imprudent as a marriage
between Mr. Wickham and our poor Lydia would be, we
are now anxious to be assured it has taken place, for
there is but too much reason to fear they are not gone to
Scotland. Colonel Forster came yesterday, having left
Brighton the day before, not many hours after the express.
Though Lydia’s short letter to Mrs. F. gave them to understand
that they were going to Gretna Green, something
was dropped by Denny expressing his belief that W. never
intended to go there, or to marry Lydia at all, which was
repeated to Colonel F. who instantly taking the alarm,
set off from B. intending to trace their route. He did
trace them easily to Clapham, but no farther; for on
entering that place they removed into a hackney-coach
and dismissed the chaise that brought them from Epsom.
%%274%%
All that is known after this is, that they were seen to
continue the London road. I know not what to think.
After making every possible enquiry on that side London,
Colonel F. came on into Hertfordshire, anxiously renewing
them at all the turnpikes, and at the inns in Barnet and
Hatfield, but without any success, no such people had
been seen to pass through. With the kindest concern he
came on to Longbourn, and broke his apprehensions to us
in a manner most creditable to his heart. I am sincerely
grieved for him and Mrs. F. but no one can throw any
blame on them. Our distress, my dear Lizzy, is very great.
My father and mother believe the worst, but I cannot
think so ill of him. Many circumstances might make it
more eligible for them to be married privately in town than
to pursue their first plan; and even if \textit{he} could form such
a design against a young woman of Lydia’s connections,
which is not likely, can I suppose her so lost to every
thing? -- Impossible. I grieve to find, however, that
Colonel F. is not disposed to depend upon their marriage;
he shook his head when I expressed my hopes, and said
he feared W. was not a man to be trusted. My poor
mother is really ill and keeps her room. Could she exert
herself it would be better, but this is not to be expected;
and as to my father, I never in my life saw him so affected.
Poor Kitty has anger for having concealed their attachment;
but as it was a matter of confidence one cannot
wonder. I am truly glad, dearest Lizzy, that you have
been spared something of these distressing scenes; but
now as the first shock is over, shall I own that I long for
your return? I am not so selfish, however, as to press
for it, if inconvenient. Adieu. I take up my pen again
to do, what I have just told you I would not, but circumstances
are such, that I cannot help earnestly begging
you all to come here, as soon as possible. I know my dear
uncle and aunt so well, that I am not afraid of requesting
it, though I have still something more to ask of the former.
My father is going to London with Colonel Forster instantly,
to try to discover her. What he means to do,
%%275%%
I am sure I know not; but his excessive distress will not
allow him to pursue any measure in the best and safest
way, and Colonel Forster is obliged to be at Brighton
again to-morrow evening. In such an exigence my uncle’s
advice and assistance would be every thing in the world;
he will immediately comprehend what I must feel, and
I rely upon his goodness.”

“Oh! where, where is my uncle?” cried Elizabeth,
darting from her seat as she finished the letter, in eagerness
to follow him, without losing a moment of the time so
precious; but as she reached the door, it was opened by
a servant, and Mr. Darcy appeared. Her pale face and
impetuous manner made him start, and before he could
recover himself enough to speak, she, in whose mind every
idea was superseded by Lydia’s situation, hastily exclaimed,
“I beg your pardon, but I must leave you. I must find
Mr. Gardiner this moment, on business that cannot be
delayed; I have not an instant to lose.”

“Good God! what is the matter?” cried he, with
more feeling than politeness; then recollecting himself,
“I will not detain you a minute, but let me, or let the
servant, go after Mr. and Mrs. Gardiner. You are not well
enough; -- you cannot go yourself.”

Elizabeth hesitated, but her knees trembled under her,
and she felt how little would be gained by her attempting
to pursue them. Calling back the servant, therefore, she
commissioned him, though in so breathless an accent as
made her almost unintelligible, to fetch his master and
mistress home, instantly.

On his quitting the room, she sat down, unable to
support herself, and looking so miserably ill, that it was
impossible for Darcy to leave her, or to refrain from
saying, in a tone of gentleness and commiseration, “Let
me call your maid. Is there nothing you could take, to
give you present relief? -- A glass of wine; -- shall I get
you one? -- You are very ill.”

“No, I thank you;” she replied, endeavouring to
recover herself. “There is nothing the matter with me.
%%276%%
I am quite well. I am only distressed by some dreadful
news which I have just received from Longbourn.”

She burst into tears as she alluded to it, and for a few
minutes could not speak another word. Darcy, in wretched
suspense, could only say something indistinctly of his
concern, and observe her in compassionate silence. At
length, she spoke again. “I have just had a letter from
Jane, with such dreadful news. It cannot be concealed
from any one. My youngest sister has left all her friends -- has
eloped; -- has thrown herself into the power of -- of
Mr. Wickham. They are gone off together from Brighton.
\textit{You} know him too well to doubt the rest. She has no
money, no connections, nothing that can tempt him to -- she
is lost for ever.”

Darcy was fixed in astonishment. “When I consider,”
she added, in a yet more agitated voice, “that \textit{I} might
have prevented it! -- \textit{I} who knew what he was. Had I but
explained some part of it only -- some part of what I learnt,
to my own family! Had his character been known, this
could not have happened. But it is all, all too late
now.”

“I am grieved, indeed,” cried Darcy; “grieved -- shocked.
But is it certain, absolutely certain?”

“Oh yes! -- They left Brighton together on Sunday
night, and were traced almost to London, but not beyond;
they are certainly not gone to Scotland.”

“And what has been done, what has been attempted,
to recover her?”

“My father is gone to London, and Jane has written
to beg my uncle’s immediate assistance, and we shall be
off, I hope, in half an hour. But nothing can be done;
I know very well that nothing can be done. How is such
a man to be worked on? How are they even to be discovered?
I have not the smallest hope. It is every way
horrible!”

Darcy shook his head in silent acquiesence.

“When \textit{my} eyes were opened to his real character. -- Oh!
had I known what I ought, what I dared, to do!
%%277%%
But I knew not -- I was afraid of doing too much.
Wretched, wretched, mistake!”

Darcy made no answer. He seemed scarcely to hear
her, and was walking up and down the room in earnest
meditation; his brow contracted, his air gloomy. Elizabeth
soon observed, and instantly understood it. Her power
was sinking; every thing \textit{must} sink under such a proof
of family weakness, such an assurance of the deepest
disgrace. She could neither wonder nor condemn, but the
belief of his self-conquest brought nothing consolatory to
her bosom, afforded no palliation of her distress. It was,
on the contrary, exactly calculated to make her understand
her own wishes; and never had she so honestly felt that
she could have loved him, as now, when all love must
be vain.

But self, though it would intrude, could not engross
her. Lydia -- the humiliation, the misery, she was bringing
on them all, soon swallowed up every private care; and
covering her face with her handkerchief, Elizabeth was
soon lost to every thing else; and, after a pause of several
minutes, was only recalled to a sense of her situation by
the voice of her companion, who, in a manner, which
though it spoke compassion, spoke likewise restraint, said,
“I am afraid you have been long desiring my absence,
nor have I any thing to plead in excuse of my stay, but
real, though unavailing, concern. Would to heaven that
any thing could be either said or done on my part, that
might offer consolation to such distress. -- But I will not
torment you with vain wishes, which may seem purposely
to ask for your thanks. This unfortunate affair will,
I fear, prevent my sister’s having the pleasure of seeing
you at Pemberley to day.”

“Oh, yes. Be so kind as to apologize for us to Miss
Darcy. Say that urgent business calls us home immediately.
Conceal the unhappy truth as long as it is possible. -- I
know it cannot be long.”

He readily assured her of his secrecy -- again expressed
his sorrow for her distress, wished it a happier conclusion
%%278%%
than there was at present reason to hope, and leaving
his compliments for her relations, with only one serious,
parting, look, went away.

As he quitted the room, Elizabeth felt how improbable
it was that they should ever see each other again on such
terms of cordiality as had marked their several meetings
in Derbyshire; and as she threw a retrospective glance
over the whole of their acquaintance, so full of contradictions
and varieties, sighed at the perverseness of those
feelings which would now have promoted its continuance,
and would formerly have rejoiced in its termination.

If gratitude and esteem are good foundations of affection,
Elizabeth’s change of sentiment will be neither
improbable nor faulty. But if otherwise, if the regard
springing from such sources is unreasonable or unnatural,
in comparison of what is so often described as arising on
a first interview with its object, and even before two
words have been exchanged, nothing can be said in her
defence, except that she had given somewhat of a trial
to the latter method, in her partiality for Wickham, and
that its ill-success might perhaps authorise her to seek
the other less interesting mode of attachment. Be that as
it may, she saw him go with regret; and in this early
example of what Lydia’s infamy must produce, found
additional anguish as she reflected on that wretched
business. Never, since reading Jane’s second letter, had
she entertained a hope of Wickham’s meaning to marry
her. No one but Jane, she thought, could flatter herself
with such an expectation. Surprise was the least of her
feelings on this developement. While the contents of the
first letter remained on her mind, she was all surprise -- all
astonishment that Wickham should marry a girl, whom
it was impossible he could marry for money; and how
Lydia could ever have attached him, had appeared
incomprehensible. But now it was all too natural. For such
an attachment as this, she might have sufficient charms;
and though she did not suppose Lydia to be deliberately
engaging in an elopement, without the intention of
%%279%%
marriage, she had no difficulty in believing that neither her
virtue nor her understanding would preserve her from
falling an easy prey.

She had never perceived, while the regiment was in
Hertfordshire, that Lydia had any partiality for him, but
she was convinced that Lydia had wanted only encouragement
to attach herself to any body. Sometimes one
officer, sometimes another had been her favourite, as their
attentions raised them in her opinion. Her affections had
been continually fluctuating, but never without an object.
The mischief of neglect and mistaken indulgence towards
such a girl. -- Oh! how acutely did she now feel it.

She was wild to be at home -- to hear, to see, to be
upon the spot, to share with Jane in the cares that must
now fall wholly upon her, in a family so deranged; a father
absent, a mother incapable of exertion, and requiring
constant attendance; and though almost persuaded that
nothing could be done for Lydia, her uncle’s interference
seemed of the utmost importance, and till he entered the
room, the misery of her impatience was severe. Mr. and
Mrs. Gardiner had hurried back in alarm, supposing, by
the servant’s account, that their niece was taken suddenly
ill; -- but satisfying them instantly on that head, she
eagerly communicated the cause of their summons,
reading the two letters aloud, and dwelling on the postscript
of the last, with trembling energy. -- Though Lydia
had never been a favourite with them, Mr. and Mrs.
Gardiner could not but be deeply affected. Not Lydia
only, but all were concerned in it; and after the first
exclamations of surprise and horror, Mr. Gardiner readily
promised every assistance in his power. -- Elizabeth,
though expecting no less, thanked him with tears of
gratitude; and all three being actuated by one spirit,
every thing relating to their journey was speedily settled.
They were to be off as soon as possible. “But what is
to be done about Pemberley?” cried Mrs. Gardiner.
“John told us Mr. Darcy was here when you sent for
us; -- was it so?”
%%280%%

“Yes; and I told him we should not be able to keep
our engagement. \textit{That} is all settled.”

“That is all settled;” repeated the other, as she ran
into her room to prepare. “And are they upon such
terms as for her to disclose the real truth! Oh, that
I knew how it was!”

But wishes were vain; or at best could serve only to
amuse her in the hurry and confusion of the following
hour. Had Elizabeth been at leisure to be idle, she
would have remained certain that all employment was
impossible to one so wretched as herself; but she had
her share of business as well as her aunt, and amongst
the rest there were notes to be written to all their friends
in Lambton, with false excuses for their sudden departure.
An hour, however, saw the whole completed; and Mr.
Gardiner meanwhile having settled his account at the inn,
nothing remained to be done but to go; and Elizabeth,
after all the misery of the morning, found herself, in
a shorter space of time than she could have supposed,
seated in the carriage, and on the road to Longbourn.
%%281%%

\Chapter{CHAPTER V.}

“I have been thinking it over again, Elizabeth,” said
her uncle, as they drove from the town; “and really,
upon serious consideration, I am much more inclined
than I was to judge as your eldest sister does of the matter.
It appears to me so very unlikely, that any young man
should form such a design against a girl who is by no
means unprotected or friendless, and who was actually
staying in his colonel’s family, that I am strongly inclined
to hope the best. Could he expect that her friends would
not step forward? Could he expect to be noticed again
by the regiment, after such an affront to Colonel Forster?
His temptation is not adequate to the risk.”

“Do you really think so?” cried Elizabeth, brightening
up for a moment.

“Upon my word,” said Mrs. Gardiner, “I begin to be
of your uncle’s opinion. It is really too great a violation
of decency, honour, and interest, for him to be guilty of
it. I cannot think so very ill of Wickham. Can you,
yourself, Lizzy, so wholly give him up, as to believe him
capable of it?”

“Not perhaps of neglecting his own interest. But of
every other neglect I can believe him capable. If, indeed,
it should be so! But I dare not hope it. Why should
they not go on to Scotland, if that had been the case?”

“In the first place,” replied Mr. Gardiner, “there is
no absolute proof that they are not gone to Scotland.”

“Oh! but their removing from the chaise into an
hackney coach is such a presumption! And, besides, no
traces of them were to be found on the Barnet road.”

“Well, then -- supposing them to be in London. They
may be there, though for the purpose of concealment,
for no more exceptionable purpose. It is not likely that
money should be very abundant on either side; and it
%%282%%
might strike them that they could be more economically,
though less expeditiously, married in London, than in
Scotland.”

“But why all this secrecy? Why any fear of detection?
Why must their marriage be private? Oh! no, no, this
is not likely. His most particular friend, you see by
Jane’s account, was persuaded of his never intending to
marry her. Wickham will never marry a woman without
some money. He cannot afford it. And what claims has
Lydia, what attractions has she beyond youth, health, and
good humour, that could make him for her sake, forego
every chance of benefiting himself by marrying well? As
to what restraint the apprehension of disgrace in the corps
might throw on a dishonourable elopement with her, I am
not able to judge; for I know nothing of the effects that
such a step might produce. But as to your other objection,
I am afraid it will hardly hold good. Lydia has no
brothers to step forward; and he might imagine, from
my father’s behaviour, from his indolence and the little
attention he has ever seemed to give to what was going
forward in his family, that \textit{he} would do as little, and
think as little about it, as any father could do, in such
a matter.”

“But can you think that Lydia is so lost to every thing
but love of him, as to consent to live with him on any
other terms than marriage?”

“It does seem, and it is most shocking indeed,” replied
Elizabeth, with tears in her eyes, “that a sister’s sense
of decency and virtue in such a point should admit of
doubt. But, really, I know not what to say. Perhaps
I am not doing her justice. But she is very young; she
has never been taught to think on serious subjects; and
for the last half year, nay, for a twelvemonth, she has
been given up to nothing but amusement and vanity.
She has been allowed to dispose of her time in the most
idle and frivolous manner, and to adopt any opinions
that came in her way. Since the ------shire were first
quartered in Meryton, nothing but love, flirtation, and
%%283%%
officers, have been in her head. She has been doing
every thing in her power by thinking and talking on the
subject, to give greater -- what shall I call it? susceptibility
to her feelings; which are naturally lively enough.
And we all know that Wickham has every charm of
person and address that can captivate a woman.”

“But you see that Jane,” said her aunt, “does not
think so ill of Wickham, as to believe him capable of the
attempt.”

“Of whom does Jane ever think ill? And who is there,
whatever might be their former conduct, that she would
believe capable of such an attempt, till it were proved
against them? But Jane knows, as well as I do, what
Wickham really is. We both know that he has been
profligate in every sense of the word. That he has neither
integrity nor honour. That he is as false and deceitful,
as he is insinuating.”

“And do you really know all this?” cried Mrs.
Gardiner, whose curiosity as to the mode of her intelligence
was all alive.

“I do, indeed,” replied Elizabeth, colouring. “I told
you the other day, of his infamous behaviour to Mr. Darcy;
and you, yourself, when last at Longbourn, heard in what
manner he spoke of the man, who had behaved with such
forbearance and liberality towards him. And there are
other circumstances which I am not at liberty -- which it
is not worth while to relate; but his lies about the whole
Pemberley family are endless. From what he said of
Miss Darcy, I was thoroughly prepared to see a proud,
reserved, disagreeable girl. Yet he knew to the contrary
himself. He must know that she was as amiable and
unpretending as we have found her.”

“But does Lydia know nothing of this? Can she be
ignorant of what you and Jane seem so well to
understand?”

“Oh, yes! -- that, that is the worst of all. Till I was
in Kent, and saw so much both of Mr. Darcy and his
relation, Colonel Fitzwilliam, I was ignorant of the truth
%%284%%
myself. And when I returned home, the ------shire was
to leave Meryton in a week or fortnight’s time. As that
was the case, neither Jane, to whom I related the whole,
nor I, thought it necessary to make our knowledge public;
for of what use could it apparently be to any one, that
the good opinion which all the neighbourhood had of
him, should then be overthrown? And even when it was
settled that Lydia should go with Mrs. Forster, the necessity
of opening her eyes to his character never occurred
to me. That \textit{she} could be in any danger from the deception
never entered my head. That such a consequence as \textit{this}
should ensue, you may easily believe was far enough from
my thoughts.”

“When they all removed to Brighton, therefore, you
had no reason, I suppose, to believe them fond of each
other.”

“Not the slightest. I can remember no symptom of
affection on either side; and had any thing of the kind
been perceptible, you must be aware that ours is not a
family, on which it could be thrown away. When first
he entered the corps, she was ready enough to admire
him; but so we all were. Every girl in, or near Meryton,
was out of her senses about him for the first two months;
but he never distinguished \textit{her} by any particular attention,
and, consequently, after a moderate period of extravagant
and wild admiration, her fancy for him gave way, and others
of the regiment, who treated her with more distinction,
again became her favourites.”

\strut

It may be easily believed, that however little of novelty
could be added to their fears, hopes, and conjectures, on
this interesting subject, by its repeated discussion, no
other could detain them from it long, during the whole
of the journey. From Elizabeth’s thoughts it was never
absent. Fixed there by the keenest of all anguish, self
reproach, she could find no interval of ease or
forgetfulness.

They travelled as expeditiously as possible; and sleeping
%%285%%
one night on the road, reached Longbourn by dinner-time
the next day. It was a comfort to Elizabeth to consider
that Jane could not have been wearied by long
expectations.

The little Gardiners, attracted by the sight of a chaise,
were standing on the steps of the house, as they entered
the paddock; and when the carriage drove up to the door,
the joyful surprise that lighted up their faces, and displayed
itself over their whole bodies, in a variety of capers
and frisks, was the first pleasing earnest of their welcome.

Elizabeth jumped out; and, after giving each of them
an hasty kiss, hurried into the vestibule, where Jane, who
came running down stairs from her mother’s apartment,
immediately met her.

Elizabeth, as she affectionately embraced her, whilst
tears filled the eyes of both, lost not a moment in asking
whether any thing had been heard of the fugitives.

“Not yet,” replied Jane. “But now that my dear
uncle is come, I hope every thing will be well.”

“Is my father in town?”

“Yes, he went on Tuesday as I wrote you word.”

“And have you heard from him often?”

“We have heard only once. He wrote me a few lines
on Wednesday, to say that he had arrived in safety, and
to give me his directions, which I particularly begged him
to do. He merely added, that he should not write again,
till he had something of importance to mention.”

“And my mother -- How is she? How are you all?”

“My mother is tolerably well, I trust; though her
spirits are greatly shaken. She is up stairs, and will have
great satisfaction in seeing you all. She does not yet
leave her dressing-room. Mary and Kitty, thank Heaven!
are quite well.”

“But you -- How are you?” cried Elizabeth. “You
look pale. How much you must have gone through!”

Her sister, however, assured her, of her being perfectly
well; and their conversation, which had been passing
while Mr. and Mrs. Gardiner were engaged with their
%%286%%
children, was now put an end to, by the approach of the
whole party. Jane ran to her uncle and aunt, and welcomed
and thanked them both, with alternate smiles and
tears.

When they were all in the drawing room, the questions
which Elizabeth had already asked, were of course
repeated by the others, and they soon found that Jane
had no intelligence to give. The sanguine hope of good,
however, which the benevolence of her heart suggested,
had not yet deserted her; she still expected that it would
all end well, and that every morning would bring some
letter, either from Lydia or her father, to explain their
proceedings, and perhaps announce the marriage.

Mrs. Bennet, to whose apartment they all repaired,
after a few minutes conversation together, received them
exactly as might be expected; with tears and lamentations
of regret, invectives against the villanous conduct
of Wickham, and complaints of her own sufferings and
ill usage; blaming every body but the person to whose
ill judging indulgence the errors of her daughter must be
principally owing.

“If I had been able,” said she, “to carry my point of
going to Brighton, with all my family, \textit{this} would not
have happened; but poor dear Lydia had nobody to
take care of her. Why did the Forsters ever let her go
out of their sight? I am sure there was some great
neglect or other on their side, for she is not the kind of
girl to do such a thing, if she had been well looked after.
I always thought they were very unfit to have the charge
of her; but I was over-ruled, as I always am. Poor
dear child! And now here’s Mr. Bennet gone away, and
I know he will fight Wickham, wherever he meets him,
and then he will be killed, and what is to become of us all?
The Collinses will turn us out, before he is cold in his
grave; and if you are not kind to us, brother, I do not
know what we shall do.”

They all exclaimed against such terrific ideas; and
Mr. Gardiner, after general assurances of his affection for
%%287%%
her and all her family, told her that he meant to be in
London the very next day, and would assist Mr. Bennet
in every endeavour for recovering Lydia.

“Do not give way to useless alarm,” added he, “though
it is right to be prepared for the worst, there is no occasion
to look on it as certain. It is not quite a week since they
left Brighton. In a few days more, we may gain some
news of them, and till we know that they are not married,
and have no design of marrying, do not let us give the
matter over as lost. As soon as I get to town, I shall
go to my brother, and make him come home with me to
Gracechurch Street, and then we may consult together as
to what is to be done.”

“Oh! my dear brother,” replied Mrs. Bennet, “that
is exactly what I could most wish for. And now do,
when you get to town, find them out, wherever they may
be; and if they are not married already, \textit{make} them
marry. And as for wedding clothes, do not let them wait
for that, but tell Lydia she shall have as much money
as she chuses, to buy them, after they are married. And,
above all things, keep Mr. Bennet from fighting. Tell
him what a dreadful state I am in, -- that I am frightened
out of my wits; and have such tremblings, such flutterings,
all over me, such spasms in my side, and pains in
my head, and such beatings at heart, that I can get no
rest by night nor by day. And tell my dear Lydia, not
to give any directions about her clothes, till she has seen
me, for she does not know which are the best warehouses.
Oh, brother, how kind you are! I know you will contrive
it all.”

But Mr. Gardiner, though he assured her again of his
earnest endeavours in the cause, could not avoid recommending
moderation to her, as well in her hopes as her
fears; and, after talking with her in this manner till
dinner was on table, they left her to vent all her feelings
on the housekeeper, who attended, in the absence of her
daughters.

Though her brother and sister were persuaded that
%%288%%
there was no real occasion for such a seclusion from the
family, they did not attempt to oppose it, for they knew
that she had not prudence enough to hold her tongue
before the servants, while they waited at table, and
judged it better that \textit{one} only of the household, and the
one whom they could most trust, should comprehend all
her fears and solicitude on the subject.

In the dining-room they were soon joined by Mary and
Kitty, who had been too busily engaged in their separate
apartments, to make their appearance before. One came
from her books, and the other from her toilette. The
faces of both, however, were tolerably calm; and no
change was visible in either, except that the loss of her
favourite sister, or the anger which she had herself incurred
in the business, had given something more of fretfulness
than usual, to the accents of Kitty. As for Mary, she was
mistress enough of herself to whisper to Elizabeth with
a countenance of grave reflection, soon after they were
seated at table,

“This is a most unfortunate affair; and will probably
be much talked of. But we must stem the tide of malice,
and pour into the wounded bosoms of each other, the
balm of sisterly consolation.”

Then, perceiving in Elizabeth no inclination of replying,
she added, “Unhappy as the event must be for Lydia,
we may draw from it this useful lesson; that loss of
virtue in a female is irretrievable -- that one false step
involves her in endless ruin -- that her reputation is no
less brittle than it is beautiful, -- and that she cannot be
too much guarded in her behaviour towards the undeserving
of the other sex.”

Elizabeth lifted up her eyes in amazement, but was too
much oppressed to make any reply. Mary, however,
continued to console herself with such kind of moral
extractions from the evil before them.

In the afternoon, the two elder Miss Bennets were able
to be for half an hour by themselves; and Elizabeth
instantly availed herself of the opportunity of making many
%%289%%
enquiries, which Jane was equally eager to satisfy. After
joining in general lamentations over the dreadful sequel
of this event, which Elizabeth considered as all but
certain, and Miss Bennet could not assert to be wholly
impossible; the former continued the subject, by saying,
“But tell me all and every thing about it, which I have
not already heard. Give me farther particulars. What
did Colonel Forster say? Had they no apprehension of
any thing before the elopement took place? They must
have seen them together for ever.”

“Colonel Forster did own that he had often suspected
some partiality, especially on Lydia’s side, but nothing
to give him any alarm. I am so grieved for him. His
behaviour was attentive and kind to the utmost. He \textit{was}
coming to us, in order to assure us of his concern, before he
had any idea of their not being gone to Scotland: when
that apprehension first got abroad, it hastened his journey.”

“And was Denny convinced that Wickham would not
marry? Did he know of their intending to go off? Had
Colonel Forster seen Denny himself?”

“Yes; but when questioned by \textit{him} Denny denied
knowing any thing of their plan, and would not give his
real opinion about it. He did not repeat his persuasion
of their not marrying -- and from \textit{that}, I am inclined to
hope, he might have been misunderstood before.”

“And till Colonel Forster came himself, not one of you
entertained a doubt, I suppose, of their being really
married?”

“How was it possible that such an idea should enter
our brains! I felt a little uneasy -- a little fearful of my
sister’s happiness with him in marriage, because I knew
that his conduct had not been always quite right. My
father and mother knew nothing of that, they only felt
how imprudent a match it must be. Kitty then owned,
with a very natural triumph on knowing more than the
rest of us, that in Lydia’s last letter, she had prepared
her for such a step. She had known, it seems, of their
being in love with each other, many weeks.”
%%290%%

“But not before they went to Brighton?”

“No, I believe not.”

“And did Colonel Forster appear to think ill of Wickham
himself? Does he know his real character?”

“I must confess that he did not speak so well of Wickham
as he formerly did. He believed him to be imprudent
and extravagant. And since this sad affair has taken
place, it is said, that he left Meryton greatly in debt; but
I hope this may be false.”

“Oh, Jane, had we been less secret, had we told what
we knew of him, this could not have happened!”

“Perhaps it would have been better;” replied her
sister. “But to expose the former faults of any person,
without knowing what their present feelings were, seemed
unjustifiable. We acted with the best intentions.”

“Could Colonel Forster repeat the particulars of Lydia’s
note to his wife?”

“He brought it with him for us to see.”

Jane then took it from her pocket-book, and gave it
to Elizabeth. These were the contents:

\begin{letter}
“\textsc{My dear Harriet},

“You will laugh when you know where I am gone,
and I cannot help laughing myself at your surprise to-morrow
morning, as soon as I am missed. I am going to
Gretna Green, and if you cannot guess with who, I shall
think you a simpleton, for there is but one man in the
world I love, and he is an angel. I should never be happy
without him, so think it no harm to be off. You need
not send them word at Longbourn of my going, if you
do not like it, for it will make the surprise the greater,
when I write to them, and sign my name Lydia Wickham.
What a good joke it will be! I can hardly write for
laughing. Pray make my excuses to Pratt, for not keeping
my engagement, and dancing with him to-night. Tell
him I hope he will excuse me when he knows all, and tell
him I will dance with him at the next ball we meet, with
great pleasure. I shall send for my clothes when I get
%%291%%
to Longbourn; but I wish you would tell Sally to mend
a great slit in my worked muslin gown, before they are
packed up. Good bye. Give my love to Colonel Forster,
I hope you will drink to our good journey.

“Your affectionate friend,

\LetterSig{“Lydia Bennet.”}
\end{letter}

\begin{sloppypar}
“Oh! thoughtless, thoughtless Lydia!” cried Elizabeth
when she had finished it. “What a letter is this,
to be written at such a moment. But at least it shews,
that \textit{she} was serious in the object of her journey. Whatever
he might afterwards persuade her to, it was not on her
side a \textit{scheme} of infamy. My poor father! how he must
have felt it!”
\end{sloppypar}

“I never saw any one so shocked. He could not speak
a word for full ten minutes. My mother was taken ill
immediately, and the whole house in such confusion!”

“Oh! Jane,” cried Elizabeth, “was there a servant
belonging to it, who did not know the whole story before
the end of the day?”

“I do not know. -- I hope there was. -- But to be guarded
at such a time, is very difficult. My mother was in
hysterics, and though I endeavoured to give her every
assistance in my power, I am afraid I did not do so much
as I might have done! But the horror of what might
possibly happen, almost took from me my faculties.”

“Your attendance upon her, has been too much for
you. You do not look well. Oh! that I had been with
you, you have had every care and anxiety upon yourself
alone.”

“Mary and Kitty have been very kind, and would
have shared in every fatigue, I am sure, but I did not
think it right for either of them. Kitty is slight and
delicate, and Mary studies so much, that her hours of
repose should not be broken in on. My aunt Phillips
came to Longbourn on Tuesday, after my father went
away; and was so good as to stay till Thursday with me.
She was of great use and comfort to us all, and lady
%%292%%
Lucas has been very kind; she walked here on Wednesday
morning to condole with us, and offered her services,
or any of her daughters, if they could be of use to us.”

“She had better have stayed at home,” cried Elizabeth;
“perhaps she \textit{meant} well, but, under such a misfortune
as this, one cannot see too little of one’s neighbours.
Assistance is impossible; condolence, insufferable. Let
them triumph over us at a distance, and be satisfied.”

She then proceeded to enquire into the measures which
her father had intended to pursue, while in town, for
the recovery of his daughter.

“He meant, I believe,” replied Jane, “to go to Epsom,
the place where they last changed horses, see the postilions,
and try if any thing could be made out from them. His
principal object must be, to discover the number of the
hackney coach which took them from Clapham. It had
come with a fare from London; and as he thought the
circumstance of a gentleman and lady’s removing from
one carriage into another, might be remarked, he meant
to make enquiries at Clapham. If he could any how discover
at what house the coachman had before set down
his fare, he determined to make enquiries there, and hoped
it might not be impossible to find out the stand and number
of the coach. I do not know of any other designs that
he had formed: but he was in such a hurry to be gone,
and his spirits so greatly discomposed, that I had difficulty
in finding out even so much as this.”
%%293%%

\Chapter{CHAPTER VI.}

The whole party were in hopes of a letter from Mr.
Bennet the next morning, but the post came in without
bringing a single line from him. His family knew him to
be on all common occasions, a most negligent and dilatory
correspondent, but at such a time, they had hoped for
exertion. They were forced to conclude, that he had no
pleasing intelligence to send, but even of \textit{that} they would
have been glad to be certain. Mr. Gardiner had waited
only for the letters before he set off.

When he was gone, they were certain at least of receiving
constant information of what was going on, and their
uncle promised, at parting, to prevail on Mr. Bennet to
return to Longbourn, as soon as he could, to the great
consolation of his sister, who considered it as the only
security for her husband’s not being killed in a duel.

Mrs. Gardiner and the children were to remain in Hertfordshire
a few days longer, as the former thought her
presence might be serviceable to her nieces. She shared
in their attendance on Mrs. Bennet, and was a great
comfort to them, in their hours of freedom. Their other
aunt also visited them frequently, and always, as she said,
with the design of cheering and heartening them up,
though as she never came without reporting some fresh
instance of Wickham’s extravagance or irregularity, she
seldom went away without leaving them more dispirited
than she found them.

All Meryton seemed striving to blacken the man, who,
but three months before, had been almost an angel of
light. He was declared to be in debt to every tradesman
in the place, and his intrigues, all honoured with the title
of seduction, had been extended into every tradesman’s
family. Every body declared that he was the wickedest
young man in the world; and every body began to find
%%294%%
out, that they had always distrusted the appearance of
his goodness. Elizabeth, though she did not credit above
half of what was said, believed enough to make her former
assurance of her sister’s ruin still more certain; and even
Jane, who believed still less of it, became almost hopeless,
more especially as the time was now come, when if they
had gone to Scotland, which she had never before entirely
despaired of, they must in all probability have gained
some news of them.

Mr. Gardiner left Longbourn on Sunday; on Tuesday,
his wife received a letter from him; it told them, that
on his arrival, he had immediately found out his brother,
and persuaded him to come to Gracechurch street. That
Mr. Bennet had been to Epsom and Clapham, before his
arrival, but without gaining any satisfactory information;
and that he was now determined to enquire at all the
principal hotels in town, as Mr. Bennet thought it possible
they might have gone to one of them, on their first coming
to London, before they procured lodgings. Mr. Gardiner
himself did not expect any success from this measure,
but as his brother was eager in it, he meant to assist him
in pursuing it. He added, that Mr. Bennet seemed wholly
disinclined at present, to leave London, and promised to
write again very soon. There was also a postscript to this
effect.

“I have written to Colonel Forster to desire him to
find out, if possible, from some of the young man’s intimates
in the regiment, whether Wickham has any relations
or connections, who would be likely to know in what part
of the town he has now concealed himself. If there were
any one, that one could apply to, with a probability of
gaining such a clue as that, it might be of essential consequence.
At present we have nothing to guide us. Colonel
Forster will, I dare say, do every thing in his power to
satisfy us on this head. But, on second thoughts, perhaps
Lizzy could tell us, what relations he has now living,
better than any other person.”

Elizabeth was at no loss to understand from whence
%%295%%
this deference for her authority proceeded; but it was
not in her power to give any information of so satisfactory
a nature, as the compliment deserved.

She had never heard of his having had any relations,
except a father and mother, both of whom had been dead
many years. It was possible, however, that some of his
companions in the ------shire, might be able to give
more information; and, though she was not very sanguine
in expecting it, the application was a something to look
forward to.

Every day at Longbourn was now a day of anxiety;
but the most anxious part of each was when the post
was expected. The arrival of letters was the first grand
object of every morning’s impatience. Through letters,
whatever of good or bad was to be told, would be communicated,
and every succeeding day was expected to
bring some news of importance.

But before they heard again from Mr. Gardiner, a letter
arrived for their father, from a different quarter, from
Mr. Collins; which, as Jane had received directions to
open all that came for him in his absence, she accordingly
read; and Elizabeth, who knew what curiosities his letters
always were, looked over her, and read it likewise. It was
as follows:

\begin{letter}
“\textsc{My dear Sir},

“I feel myself called upon, by our relationship, and
my situation in life, to condole with you on the grievous
affliction you are now suffering under, of which we were
yesterday informed by a letter from Hertfordshire. Be
assured, my dear Sir, that Mrs. Collins and myself sincerely
sympathise with you, and all your respectable family, in
your present distress, which must be of the bitterest kind,
because proceeding from a cause which no time can remove.
No arguments shall be wanting on my part, that can
alleviate so severe a misfortune; or that may comfort
you, under a circumstance that must be of all others most
afflicting to a parent’s mind. The death of your daughter
%%296%%
would have been a blessing in comparison of this. And
it is the more to be lamented, because there is reason to
suppose, as my dear Charlotte informs me, that this
licentiousness of behaviour in your daughter, has proceeded
from a faulty degree of indulgence, though, at the
same time, for the consolation of yourself and Mrs. Bennet,
I am inclined to think that her own disposition must be
naturally bad, or she could not be guilty of such an
enormity, at so early an age. Howsoever that may be,
you are grievously to be pitied, in which opinion I am not
only joined by Mrs. Collins, but likewise by lady Catherine
and her daughter, to whom I have related the affair.
They agree with me in apprehending that this false step
in one daughter, will be injurious to the fortunes of all
the others, for who, as lady Catherine herself condescendingly
says, will connect themselves with such a family.
And this consideration leads me moreover to reflect with
augmented satisfaction on a certain event of last November,
for had it been otherwise, I must have been involved
in all your sorrow and disgrace. Let me advise you then,
my dear Sir, to console yourself as much as possible, to
throw off your unworthy child from your affection for
ever, and leave her to reap the fruits of her own heinous
offence.

\raggedleft “I am, dear Sir, \&c. \&c.”
\end{letter}

Mr. Gardiner did not write again, till he had received
an answer from Colonel Forster; and then he had nothing
of a pleasant nature to send. It was not known that
Wickham had a single relation, with whom he kept up
any connection, and it was certain that he had no near
one living. His former acquaintance had been numerous;
but since he had been in the militia, it did not appear that
he was on terms of particular friendship with any of them.
There was no one therefore who could be pointed out, as
likely to give any news of him. And in the wretched state
of his own finances, there was a very powerful motive for
secrecy, in addition to his fear of discovery by Lydia’s
relations, for it had just transpired that he had left gaming
%%297%%
debts behind him, to a very considerable amount. Colonel
Forster believed that more than a thousand pounds would
be necessary to clear his expences at Brighton. He owed
a good deal in the town, but his debts of honour were still
more formidable. Mr. Gardiner did not attempt to conceal
these particulars from the Longbourn family; Jane heard
them with horror. “A gamester!” she cried. “This is
wholly unexpected. I had not an idea of it.”

Mr. Gardiner added in his letter, that they might expect
to see their father at home on the following day, which was
Saturday. Rendered spiritless by the ill-success of all
their endeavours, he had yielded to his brother-in-law’s
intreaty that he would return to his family, and leave it
to him to do, whatever occasion might suggest to be
advisable for continuing their pursuit. When Mrs. Bennet
was told of this, she did not express so much satisfaction
as her children expected, considering what her anxiety
for his life had been before.

“What, is he coming home, and without poor Lydia!”
she cried. “Sure he will not leave London before he has
found them. Who is to fight Wickham, and make him
marry her, if he comes away?”

As Mrs. Gardiner began to wish to be at home, it was
settled that she and her children should go to London,
at the same time that Mr. Bennet came from it. The
coach, therefore, took them the first stage of their journey,
and brought its master back to Longbourn.

Mrs. Gardiner went away in all the perplexity about
Elizabeth and her Derbyshire friend, that had attended
her from that part of the world. His name had never
been voluntarily mentioned before them by her niece;
and the kind of half-expectation which Mrs. Gardiner had
formed, of their being followed by a letter from him,
had ended in nothing. Elizabeth had received none since
her return, that could come from Pemberley.

The present unhappy state of the family, rendered any
other excuse for the lowness of her spirits unnecessary;
nothing, therefore, could be fairly conjectured from \textit{that},
%%298%%
though Elizabeth, who was by this time tolerably well
acquainted with her own feelings, was perfectly aware,
that, had she known nothing of Darcy, she could have
borne the dread of Lydia’s infamy somewhat better. It
would have spared her, she thought, one sleepless night
out of two.

When Mr. Bennet arrived, he had all the appearance
of his usual philosophic composure. He said as little as
he had ever been in the habit of saying; made no mention
of the business that had taken him away, and it was
some time before his daughters had courage to speak of it.

It was not till the afternoon, when he joined them at
tea, that Elizabeth ventured to introduce the subject;
and then, on her briefly expressing her sorrow for what
he must have endured, he replied, “Say nothing of that.
Who should suffer but myself? It has been my own
doing, and I ought to feel it.”

“You must not be too severe upon yourself,” replied
Elizabeth.

“You may well warn me against such an evil. Human
nature is so prone to fall into it! No, Lizzy, let me once
in my life feel how much I have been to blame. I am not
afraid of being overpowered by the impression. It will
pass away soon enough.”

“Do you suppose them to be in London?”

“Yes; where else can they be so well concealed?”

“And Lydia used to want to go to London,” added
Kitty.

“She is happy, then,” said her father, drily; “and
her residence there will probably be of some duration.”

Then, after a short silence, he continued, “Lizzy, I bear
you no ill-will for being justified in your advice to me
last May, which, considering the event, shews some
greatness of mind.”

They were interrupted by Miss Bennet, who came to
fetch her mother’s tea.

“This is a parade,” cried he, “which does one good;
it gives such an elegance to misfortune! Another day
%%299%%
I will do the same; I will sit in my library, in my night cap
and powdering gown, and give as much trouble as I can, -- or,
perhaps, I may defer it, till Kitty runs away.”

“I am not going to run away, Papa,” said Kitty,
fretfully; “if \textit{I} should ever go to Brighton, I would
behave better than Lydia.”

“\textit{You} go to Brighton! -- I would not trust you so near
it as East Bourne for fifty pounds! No, Kitty, I have
at last learnt to be cautious, and you will feel the effects
of it. No officer is ever to enter my house again, nor even
to pass through the village. Balls will be absolutely
prohibited, unless you stand up with one of your sisters.
And you are never to stir out of doors, till you can prove,
that you have spent ten minutes of every day in a rational
manner.”

Kitty, who took all these threats in a serious light,
began to cry.

“Well, well,” said he, “do not make yourself unhappy.
If you are a good girl for the next ten years, I will take
you to a review at the end of them.”
%%300%%

\Chapter{CHAPTER VII.}

Two days after Mr. Bennet’s return, as Jane and
Elizabeth were walking together in the shrubbery behind
the house, they saw the housekeeper coming towards
them, and, concluding that she came to call them to their
mother, went forward to meet her; but, instead of the
expected summons, when they approached her, she said
to Miss Bennet, “I beg your pardon, madam, for interrupting
you, but I was in hopes you might have got some
good news from town, so I took the liberty of coming
to ask.”

“What do you mean, Hill? We have heard nothing
from town.”

“Dear madam,” cried Mrs. Hill, in great astonishment,
“don’t you know there is an express come for master
from Mr. Gardiner? He has been here this half hour, and
master has had a letter.”

Away ran the girls, too eager to get in to have time for
speech. They ran through the vestibule into the breakfast
room; from thence to the library; -- their father was in
neither; and they were on the point of seeking him up
stairs with their mother, when they were met by the butler,
who said,

“If you are looking for my master, ma’am, he is walking
towards the little copse.”

Upon this information, they instantly passed through
the hall once more, and ran across the lawn after their
father, who was deliberately pursuing his way towards
a small wood on one side of the paddock.

Jane, who was not so light, nor so much in the habit of
running as Elizabeth, soon lagged behind, while her sister,
panting for breath, came up with him, and eagerly cried out,

“Oh, Papa, what news? what news? have you heard
from my uncle?”
%%301%%

“Yes, I have had a letter from him by express.”

“Well, and what news does it bring? good or bad?”

“What is there of good to be expected?” said he,
taking the letter from his pocket; “but perhaps you
would like to read it.”

Elizabeth impatiently caught it from his hand. Jane
now came up.

“Read it aloud,” said their father, “for I hardly know
myself what it is about.”

\begin{letter}
\LetterDate{“Gracechurch-street, Monday,\\
August 2.}

“\textsc{My dear Brother},

“At last I am able to send you some tidings of my
niece, and such as, upon the whole, I hope will give you
satisfaction. Soon after you left me on Saturday, I was
fortunate enough to find out in what part of London they
were. The particulars, I reserve till we meet. It is enough
to know they are discovered, I have seen them both------”

“Then it is, as I always hoped,” cried Jane; “they
are married!”

Elizabeth read on; “I have seen them both. They
are not married, nor can I find there was any intention
of being so; but if you are willing to perform the engagements
which I have ventured to make on your side, I hope
it will not be long before they are. All that is required
of you is, to assure to your daughter, by settlement, her
equal share of the five thousand pounds, secured among
your children after the decease of yourself and my sister;
and, moreover, to enter into an engagement of allowing
her, during your life, one hundred pounds per annum.
These are conditions, which, considering every thing, I had
no hesitation in complying with, as far as I thought
myself privileged, for you. I shall send this by express,
that no time may be lost in bringing me your answer.
You will easily comprehend, from these particulars, that
Mr. Wickham’s circumstances are not so hopeless as they
are generally believed to be. The world has been deceived
%%302%%
in that respect; and I am happy to say, there will be
some little money, even when all his debts are discharged,
to settle on my niece, in addition to her own fortune.
If, as I conclude will be the case, you send me full powers
to act in your name, throughout the whole of this business,
I will immediately give directions to Haggerston for preparing
a proper settlement. There will not be the smallest
occasion for your coming to town again; therefore, stay
quietly at Longbourn, and depend on my diligence and care.
Send back your answer as soon as you can, and be careful
to write explicitly. We have judged it best, that my niece
should be married from this house, of which I hope you
will approve. She comes to us to-day. I shall write again
as soon as any thing more is determined on. Your’s, \&c.

\LetterSig{“Edw. Gardiner.”}
\end{letter}

“Is it possible!” cried Elizabeth, when she had
finished. “Can it be possible that he will marry her?”

“Wickham is not so undeserving, then, as we have
th\-ought him;” said her sister. “My dear father, I congratulate
you.”

“And have you answered the letter?” said Elizabeth.

“No; but it must be done soon.”

Most earnestly did she then intreat him to lose no more
time before he wrote.

“Oh! my dear father,” she cried, “come back, and
write immediately. Consider how important every
moment is, in such a case.”

“Let me write for you,” said Jane, “if you dislike the
trouble yourself.”

“I dislike it very much,” he replied; “but it must
be done.”

And so saying, he turned back with them, and walked
towards the house.

“And may I ask?” said Elizabeth, “but the terms,
I suppose, must be complied with.”

“Complied with! I am only ashamed of his asking so
little.”
%%303%%

“And they \textit{must} marry! Yet he is \textit{such} a man!”

“Yes, yes, they must marry. There is nothing else to
be done. But there are two things that I want very much
to know:-- one is, how much money your uncle has laid
down, to bring it about; and the other, how I am ever
to pay him.”

“Money! my uncle!” cried Jane, “what do you
mean, Sir?”

“I mean, that no man in his senses, would marry Lydia
on so slight a temptation as one hundred a-year during
my life, and fifty after I am gone.”

“That is very true,” said Elizabeth; “though it had
not occurred to me before. His debts to be discharged,
and something still to remain! Oh! it must be my
uncle’s doings! Generous, good man, I am afraid he has
distressed himself. A small sum could not do all this.”

“No,” said her father, “Wickham’s a fool, if he
takes her with a farthing less than ten thousand pounds.
I should be sorry to think so ill of him, in the very beginning
of our relationship.”

“Ten thousand pounds! Heaven forbid! How is half
such a sum to be repaid?”

Mr. Bennet made no answer, and each of them, deep
in thought, continued silent till they reached the house.
Their father then went to the library to write, and the
girls walked into the breakfast-room.

“And they are really to be married!” cried Elizabeth,
as soon as they were by themselves. “How strange this
is! And for \textit{this} we are to be thankful. That they should
marry, small as is their chance of happiness, and wretched
as is his character, we are forced to rejoice! Oh, Lydia!”

“I comfort myself with thinking,” replied Jane, “that
he certainly would not marry Lydia, if he had not a real
regard for her. Though our kind uncle has done something
towards clearing him, I cannot believe that ten thousand
pounds, or any thing like it, has been advanced. He has
children of his own, and may have more. How could he
spare half ten thousand pounds?”
%%304%%

“If we are ever able to learn what Wickham’s debts
have been,” said Elizabeth, “and how much is settled on his
side on our sister, we shall exactly know what Mr. Gardiner
has done for them, because Wickham has not sixpence
of his own. The kindness of my uncle and aunt can never
be requited. Their taking her home, and affording her
their personal protection and countenance, is such a
sacrifice to her advantage, as years of gratitude cannot
enough acknowledge. By this time she is actually with
them! If such goodness does not make her miserable
now, she will never deserve to be happy! What a meeting
for her, when she first sees my aunt!”

“We must endeavour to forget all that has passed on
either side,” said Jane: “I hope and trust they will yet
be happy. His consenting to marry her is a proof, I will
believe, that he is come to a right way of thinking. Their
mutual affection will steady them; and I flatter myself
they will settle so quietly, and live in so rational a manner,
as may in time make their past imprudence forgotten.”

“Their conduct has been such,” replied Elizabeth, “as
neither you, nor I, nor any body, can ever forget. It is
useless to talk of it.”

It now occurred to the girls that their mother was in
all likelihood perfectly ignorant of what had happened.
They went to the library, therefore, and asked their father,
whether he would not wish them to make it known to
her. He was writing, and, without raising his head, coolly
replied,

“Just as you please.”

“May we take my uncle’s letter to read to her?”

“Take whatever you like, and get away.”

Elizabeth took the letter from his writing table, and
they went up stairs together. Mary and Kitty were both
with Mrs. Bennet: one communication would, therefore,
do for all. After a slight preparation for good news, the
letter was read aloud. Mrs. Bennet could hardly contain
herself. As soon as Jane had read Mr. Gardiner’s hope
of Lydia’s being soon married, her joy burst forth, and
%%305%%
every following sentence added to its exuberance. She
was now in an irritation as violent from delight, as she
had ever been fidgetty from alarm and vexation. To know
that her daughter would be married was enough. She
was disturbed by no fear for her felicity, nor humbled
by any remembrance of her misconduct.

“My dear, dear Lydia!” she cried: “This is delightful
indeed! -- She will be married! -- I shall see her again! -- She
will be married at sixteen! -- My good, kind brother! -- I
knew how it would be -- I knew he would manage every
thing. How I long to see her! and to see dear Wickham
too! But the clothes, the wedding clothes! I will write
to my sister Gardiner about them directly. Lizzy, my
dear, run down to your father, and ask him how much
he will give her. Stay, stay, I will go myself. Ring the
bell, Kitty, for Hill. I will put on my things in a moment.
My dear, dear Lydia! -- How merry we shall be together
when we meet!”

Her eldest daughter endeavoured to give some relief to
the violence of these transports, by leading her thoughts
to the obligations which Mr. Gardiner’s behaviour laid
them all under.

“For we must attribute this happy conclusion,” she
added, “in a great measure, to his kindness. We are
persuaded that he has pledged himself to assist Mr. Wickham
with money.”

“Well,” cried her mother, “it is all very right; who
should do it but her own uncle? If he had not had
a family of his own, I and my children must have had all
his money you know, and it is the first time we have ever
had any thing from him, except a few presents. Well!
I am so happy. In a short time, I shall have a daughter
married. Mrs. Wickham! How well it sounds. And she
was only sixteen last June. My dear Jane, I am in such
a flutter, that I am sure I can’t write; so I will dictate,
and you write for me. We will settle with your father
about the money afterwards; but the things should be
ordered immediately.”
%%306%%

She was then proceeding to all the particulars of calico,
muslin, and cambric, and would shortly have dictated
some very plentiful orders, had not Jane, though with
some difficulty, persuaded her to wait, till her father was
at leisure to be consulted. One day’s delay she observed,
would be of small importance; and her mother was too
happy, to be quite so obstinate as usual. Other schemes
too came into her head.

“I will go to Meryton,” said she, “as soon as I am
dressed, and tell the good, good news to my sister Phillips.
And as I come back, I can call on Lady Lucas and Mrs.
Long. Kitty, run down and order the carriage. An
airing would do me a great deal of good, I am sure. Girls,
can I do any thing for you in Meryton? Oh! here comes
Hill. My dear Hill, have you heard the good news? Miss
Lydia is going to be married; and you shall all have a
bowl of punch, to make merry at her wedding.”

Mrs. Hill began instantly to express her joy. Elizabeth
received her congratulations amongst the rest, and then,
sick of this folly, took refuge in her own room, that she
might think with freedom.

Poor Lydia’s situation must, at best, be bad enough;
but that it was no worse, she had need to be thankful.
She felt it so; and though, in looking forward, neither
rational happiness nor worldly prosperity, could be justly
expected for her sister; in looking back to what they
had feared, only two hours ago, she felt all the advantages
of what they had gained.
%%307%%

\Chapter{CHAPTER VIII.}

Mr. Bennet had very often wished, before this period
of his life, that, instead of spending his whole income, he
had laid by an annual sum, for the better provision of
his children, and of his wife, if she survived him. He now
wished it more than ever. Had he done his duty in that
respect, Lydia need not have been indebted to her uncle,
for whatever of honour or credit could now be purchased
for her. The satisfaction of prevailing on one of the most
worthless young men in Great Britain to be her husband,
might then have rested in its proper place.

He was seriously concerned, that a cause of so little
advantage to any one, should be forwarded at the sole
expence of his brother-in-law, and he was determined,
if possible, to find out the extent of his assistance, and to
discharge the obligation as soon as he could.

When first Mr. Bennet had married, economy was held
to be perfectly useless; for, of course, they were to have
a son. This son was to join in cutting off the entail, as
soon as he should be of age, and the widow and younger
children would by that means be provided for. Five
daughters successively entered the world, but yet the son
was to come; and Mrs. Bennet, for many years after
Lydia’s birth, had been certain that he would. This event
had at last been despaired of, but it was then too late to
be saving. Mrs. Bennet had no turn for economy, and her
husband’s love of independence had alone prevented their
exceeding their income.

Five thousand pounds was settled by marriage articles
on Mrs. Bennet and the children. But in what proportions
it should be divided amongst the latter, depended
on the will of the parents. This was one point, with
regard to Lydia at least, which was now to be settled,
and Mr. Bennet could have no hesitation in acceding to
%%308%%
the proposal before him. In terms of grateful acknowledgment
for the kindness of his brother, though expressed
most concisely, he then delivered on paper his perfect
approbation of all that was done, and his willingness to
fulfil the engagements that had been made for him. He
had never before supposed that, could Wickham be prevailed
on to marry his daughter, it would be done with
so little inconvenience to himself, as by the present
arrangement. He would scarcely be ten pounds a-year
the loser, by the hundred that was to be paid them;
for, what with her board and pocket allowance, and the
continual presents in money, which passed to her, through
her mother’s hands, Lydia’s expences had been very little
within that sum.

That it would be done with such trifling exertion on
his side, too, was another very welcome surprise; for his
chief wish at present, was to have as little trouble in the
business as possible. When the first transports of rage
which had produced his activity in seeking her were over,
he naturally returned to all his former indolence. His
letter was soon dispatched; for though dilatory in undertaking
business, he was quick in its execution. He begged
to know farther particulars of what he was indebted to
his brother; but was too angry with Lydia, to send any
message to her.

The good news quickly spread through the house; and
with proportionate speed through the neighbourhood. It
was borne in the latter with decent philosophy. To be
sure it would have been more for the advantage of conversation,
had Miss Lydia Bennet come upon the town;
or, as the happiest alternative, been secluded from the
world, in some distant farm house. But there was much
to be talked of, in marrying her; and the good-natured
wishes for her well-doing, which had proceeded before,
from all the spiteful old ladies in Meryton, lost but
little of their spirit in this change of circumstances,
because with such an husband, her misery was considered
certain.
%%309%%

It was a fortnight since Mrs. Bennet had been down
stairs, but on this happy day, she again took her seat at
the head of her table, and in spirits oppressively high.
No sentiment of shame gave a damp to her triumph.
The marriage of a daughter, which had been the first
object of her wishes, since Jane was sixteen, was now on
the point of accomplishment, and her thoughts and her
words ran wholly on those attendants of elegant nuptials,
fine muslins, new carriages, and servants. She was busily
searching through the neighbourhood for a proper situation
for her daughter, and, without knowing or considering
what their income might be, rejected many as deficient
in size and importance.

“Haye-Park might do,” said she, “if the Gouldings
would quit it, or the great house at Stoke, if the drawing-room
were larger; but Ashworth is too far off! I could
not bear to have her ten miles from me; and as for Purvis
Lodge, the attics are dreadful.”

Her husband allowed her to talk on without interruption,
while the servants remained. But when they had
withdrawn, he said to her, “Mrs. Bennet, before you take
any, or all of these houses, for your son and daughter,
let us come to a right understanding. Into \textit{one} house in
this neighbourhood, they shall never have admittance.
I will not encourage the impudence of either, by receiving
them at Longbourn.”

A long dispute followed this declaration; but Mr.
Bennet was firm: it soon led to another; and Mrs.
Bennet found, with amazement and horror, that her
husband would not advance a guinea to buy clothes for
his daughter. He protested that she should receive from
him no mark of affection whatever, on the occasion.
Mrs. Bennet could hardly comprehend it. That his anger
could be carried to such a point of inconceivable resentment,
as to refuse his daughter a privilege, without which
her marriage would scarcely seem valid, exceeded all that
she could believe possible. She was more alive to the
disgrace, which the want of new clothes must reflect on
%%310%%
her daughter’s nuptials, than to any sense of shame at
her eloping and living with Wickham, a fortnight before
they took place.

Elizabeth was now most heartily sorry that she had,
from the distress of the moment, been led to make Mr.
Darcy acquainted with their fears for her sister; for
since her marriage would so shortly give the proper
termination to the elopement, they might hope to conceal
its unfavourable beginning, from all those who were not
immediately on the spot.

She had no fear of its spreading farther, through his
means. There were few people on whose secrecy she
would have more confidently depended; but at the same
time, there was no one, whose knowledge of a sister’s
frailty would have mortified her so much. Not, however,
from any fear of disadvantage from it, individually to
herself; for at any rate, there seemed a gulf impassable
between them. Had Lydia’s marriage been concluded on
the most honourable terms, it was not to be supposed
that Mr. Darcy would connect himself with a family,
where to every other objection would now be added, an
alliance and relationship of the nearest kind with the
man whom he so justly scorned.

From such a connection she could not wonder that he
should shrink. The wish of procuring her regard, which
she had assured herself of his feeling in Derbyshire, could
not in rational expectation survive such a blow as this.
She was humbled, she was grieved; she repented, though
she hardly knew of what. She became jealous of his
esteem, when she could no longer hope to be benefited
by it. She wanted to hear of him, when there seemed
the least chance of gaining intelligence. She was convinced
that she could have been happy with him; when it was
no longer likely they should meet.

What a triumph for him, as she often thought, could
he know that the proposals which she had proudly spurned
only four months ago, would now have been gladly and
gratefully received! He was as generous, she doubted not,
%%311%%
as the most generous of his sex. But while he was mortal,
there must be a triumph.

She began now to comprehend that he was exactly the
man, who, in disposition and talents, would most suit her.
His understanding and temper, though unlike her own,
would have answered all her wishes. It was an union that
must have been to the advantage of both; by her ease
and liveliness, his mind might have been softened, his
manners improved, and from his judgment, information,
and knowledge of the world, she must have received
benefit of greater importance.

But no such happy marriage could now teach the
admiring multitude what connubial felicity really was.
An union of a different tendency, and precluding the possibility
of the other, was soon to be formed in their family.

How Wickham and Lydia were to be supported in
tolerable independence, she could not imagine. But how
little of permanent happiness could belong to a couple
who were only brought together because their passions
were stronger than their virtue, she could easily
conjecture.

\strut

Mr. Gardiner soon wrote again to his brother. To
Mr. Bennet’s acknowledgments he briefly replied, with
assurances of his eagerness to promote the welfare of any
of his family; and concluded with intreaties that the
subject might never be mentioned to him again. The
principal purport of his letter was to inform them, that
Mr. Wickham had resolved on quitting the Militia.

“It was greatly my wish that he should do so,” he
added, “as soon as his marriage was fixed on. And
I think you will agree with me, in considering a removal
from that corps as highly advisable, both on his account
and my niece’s. It is Mr. Wickham’s intention to go into
the regulars; and, among his former friends, there are
still some who are able and willing to assist him in the
army. He has the promise of an ensigncy in General
------’s regiment, now quartered in the North. It is an
%%312%%
advantage to have it so far from this part of the kingdom.
He promises fairly, and I hope among different people,
where they may each have a character to preserve, they
will both be more prudent. I have written to Colonel
Forster, to inform him of our present arrangements, and
to request that he will satisfy the various creditors of
Mr. Wickham in and near Brighton, with assurances of
speedy payment, for which I have pledged myself. And
will you give yourself the trouble of carrying similar
assurances to his creditors in Meryton, of whom I shall
subjoin a list, according to his information. He has
given in all his debts; I hope at least he has not deceived
us. Haggerston has our directions, and all will be completed
in a week. They will then join his regiment, unless
they are first invited to Longbourn; and I understand
from Mrs. Gardiner, that my niece is very desirous of
seeing you all, before she leaves the South. She is well,
and begs to be dutifully remembered to you and her
mother. -- Your’s, \&c.

\LetterSig{“E. Gardiner.”}

\strut

Mr. Bennet and his daughters saw all the advantages
of Wickham’s removal from the ------shire, as clearly as
Mr. Gardiner could do. But Mrs. Bennet, was not so well
pleased with it. Lydia’s being settled in the North, just
when she had expected most pleasure and pride in her
company, for she had by no means given up her plan of
their residing in Hertfordshire, was a severe disappointment;
and besides, it was such a pity that Lydia should
be taken from a regiment where she was acquainted with
every body, and had so many favourites.

“She is so fond of Mrs. Forster,” said she, “it will be
quite shocking to send her away! And there are several
of the young men, too, that she likes very much. The
officers may not be so pleasant in General ------’s
regiment.”

His daughter’s request, for such it might be considered,
of being admitted into her family again, before she set
%%313%%
off for the North, received at first an absolute negative.
But Jane and Elizabeth, who agreed in wishing, for the
sake of their sister’s feelings and consequence, that she
should be noticed on her marriage by her parents, urged
him so earnestly, yet so rationally and so mildly, to receive
her and her husband at Longbourn, as soon as they were
married, that he was prevailed on to think as they thought,
and act as they wished. And their mother had the satisfaction
of knowing, that she should be able to shew her
married daughter in the neighbourhood, before she was
banished to the North. When Mr. Bennet wrote again
to his brother, therefore, he sent his permission for them
to come; and it was settled, that as soon as the ceremony
was over, they should proceed to Longbourn. Elizabeth
was surprised, however, that Wickham should consent to
such a scheme, and, had she consulted only her own inclination,
any meeting with him would have been the last
object of her wishes.
%%314%%

\Chapter{CHAPTER IX.}

Their sister’s wedding day arrived; and Jane and
Elizabeth felt for her probably more than she felt for
herself. The carriage was sent to meet them at ------,
and they were to return in it, by dinner-time. Their
arrival was dreaded by the elder Miss Bennets; and Jane
more especially, who gave Lydia the feelings which would
have attended herself, had \textit{she} been the culprit, was
wretched in the thought of what her sister must endure.

They came. The family were assembled in the breakfast
room, to receive them. Smiles decked the face of
Mrs. Bennet, as the carriage drove up to the door; her
husband looked impenetrably grave; her daughters,
alarmed, anxious, uneasy.

Lydia’s voice was heard in the vestibule; the door was
thrown open, and she ran into the room. Her mother
stepped forwards, embraced her, and welcomed her with
rapture; gave her hand with an affectionate smile to
Wickham, who followed his lady, and wished them both
joy, with an alacrity which shewed no doubt of their
happiness.

Their reception from Mr. Bennet, to whom they then
turned, was not quite so cordial. His countenance rather
gained in austerity; and he scarcely opened his lips.
The easy assurance of the young couple, indeed, was
enough to provoke him. Elizabeth was disgusted, and
even Miss Bennet was shocked. Lydia was Lydia still;
untamed, unabashed, wild, noisy, and fearless. She turned
from sister to sister, demanding their congratulations, and
when at length they all sat down, looked eagerly round
the room, took notice of some little alteration in it, and
observed, with a laugh, that it was a great while since
she had been there.

Wickham was not at all more distressed than herself,
%%315%%
but his manners were always so pleasing, that had his
character and his marriage been exactly what they ought,
his smiles and his easy address, while he claimed their
relationship, would have delighted them all. Elizabeth
had not before believed him quite equal to such assurance;
but she sat down, resolving within herself, to draw no
limits in future to the impudence of an impudent man.
\textit{She} blushed, and Jane blushed; but the cheeks of the
two who caused their confusion, suffered no variation of
colour.

There was no want of discourse. The bride and her
mother could neither of them talk fast enough; and
Wickham, who happened to sit near Elizabeth, began
enquiring after his acquaintance in that neighbourhood,
with a good humoured ease, which she felt very unable
to equal in her replies. They seemed each of them to have
the happiest memories in the world. Nothing of the past
was recollected with pain; and Lydia led voluntarily to
subjects, which her sisters would not have alluded to for
the world.

“Only think of its being three months,” she cried,
“since I went away; it seems but a fortnight I declare;
and yet there have been things enough happened in the
time. Good gracious! when I went away, I am sure
I had no more idea of being married till I came back
again! though I thought it would be very good fun if
I was.”

Her father lifted up his eyes. Jane was distressed.
Elizabeth looked expressively at Lydia; but she, who
never heard nor saw any thing of which she chose to be
insensible, gaily continued, “Oh! mamma, do the people
here abouts know I am married to-day? I was afraid
they might not; and we overtook William Goulding in
his curricle, so I was determined he should know it, and
so I let down the side glass next to him, and took off my
glove, and let my hand just rest upon the window frame,
so that he might see the ring, and then I bowed and
smiled like any thing.”
%%316%%

Elizabeth could bear it no longer. She got up, and ran
out of the room; and returned no more, till she heard
them passing through the hall to the dining parlour.
She then joined them soon enough to see Lydia, with
anxious parade, walk up to her mother’s right hand, and
hear her say to her eldest sister, “Ah! Jane, I take
your place now, and you must go lower, because I am a
married woman.”

It was not to be supposed that time would give Lydia
that embarrassment, from which she had been so wholly
free at first. Her ease and good spirits increased. She
longed to see Mrs. Phillips, the Lucasses, and all their
other neighbours, and to hear herself called “Mrs. Wickham,”
by each of them; and in the mean time, she went
after dinner to shew her ring and boast of being married,
to Mrs. Hill and the two housemaids.

“Well, mamma,” said she, when they were all returned
to the breakfast room, “and what do you think of my
husband? Is not he a charming man? I am sure my
sisters must all envy me. I only hope they may have
half my good luck. They must all go to Brighton. That
is the place to get husbands. What a pity it is, mamma,
we did not all go.”

“Very true; and if I had my will, we should. But
my dear Lydia, I don’t at all like your going such a way
off. Must it be so?”

“Oh, lord! yes; -- there is nothing in that. I shall
like it of all things. You and papa, and my sisters, must
come down and see us. We shall be at Newcastle all the
winter, and I dare say there will be some balls, and I will
take care to get good partners for them all.”

“I should like it beyond any thing!” said her mother.

“And then when you go away, you may leave one
or two of my sisters behind you; and I dare say I shall
get husbands for them before the winter is over.”

“I thank you for my share of the favour,” said Elizabeth;
“but I do not particularly like your way of getting
husbands.”
%%317%%

Their visitors were not to remain above ten days with
them. Mr. Wickham had received his commission before
he left London, and he was to join his regiment at the end
of a fortnight.

No one but Mrs. Bennet, regretted that their stay
would be so short; and she made the most of the time,
by visiting about with her daughter, and having very
frequent parties at home. These parties were acceptable
to all; to avoid a family circle was even more desirable
to such as did think, than such as did not.

Wickham’s affection for Lydia, was just what Elizabeth
had expected to find it; not equal to Lydia’s for him.
She had scarcely needed her present observation to be
satisfied, from the reason of things, that their elopement
had been brought on by the strength of her love, rather
than by his; and she would have wondered why, without
violently caring for her, he chose to elope with her at
all, had she not felt certain that his flight was rendered
necessary by distress of circumstances; and if that were
the case, he was not the young man to resist an opportunity
of having a companion.

Lydia was exceedingly fond of him. He was her dear
Wickham on every occasion; no one was to be put in
competition with him. He did every thing best in the
world; and she was sure he would kill more birds on the
first of September, than any body else in the country.

One morning, soon after their arrival, as she was sitting
with her two elder sisters, she said to Elizabeth,

“Lizzy, I never gave \textit{you} an account of my wedding,
I believe. You were not by, when I told mamma, and the
others, all about it. Are not you curious to hear how it
was managed?”

“No really,” replied Elizabeth; “I think there cannot
be too little said on the subject.”

“La! You are so strange! But I must tell you how
it went off. We were married, you know, at St. Clement’s,
because Wickham’s lodgings were in that parish. And
it was settled that we should all be there by eleven o’clock.
%%318%%
My uncle and aunt and I were to go together; and the
others were to meet us at the church. Well, Monday
morning came, and I was in such a fuss! I was so afraid
you know that something would happen to put it off, and
then I should have gone quite distracted. And there was
my aunt, all the time I was dressing, preaching and
talking away just as if she was reading a sermon. However,
I did not hear above one word in ten, for I was
thinking, you may suppose, of my dear Wickham. I longed
to know whether he would be married in his blue coat.

“Well, and so we breakfasted at ten as usual; I
thought it would never be over; for, by the bye, you are
to understand, that my uncle and aunt were horrid
unpleasant all the time I was with them. If you’ll believe
me, I did not once put my foot out of doors, though I was
there a fortnight. Not one party, or scheme, or any thing.
To be sure London was rather thin, but however the
little Theatre was open. Well, and so just as the carriage
came to the door, my uncle was called away upon business
to that horrid man Mr. Stone. And then, you know,
when once they get together, there is no end of it. Well,
I was so frightened I did not know what to do, for my
uncle was to give me away; and if we were beyond the
hour, we could not be married all day. But, luckily, he
came back again in ten minutes time, and then we all
set out. However, I recollected afterwards, that if he
\textit{had} been prevented going, the wedding need not be put
off, for Mr. Darcy might have done as well.”

“Mr. Darcy!” repeated Elizabeth, in utter amazement.

“Oh, yes! -- he was to come there with Wickham, you
know. But gracious me! I quite forgot! I ought not to
have said a word about it. I promised them so faithfully!
What will Wickham say? It was to be such a secret!”

“If it was to be secret,” said Jane, “say not another
word on the subject. You may depend upon my seeking
no further.”

“Oh! certainly,” said Elizabeth, though burning with
curiosity; “we will ask you no questions.”
%%319%%

“Thank you,” said Lydia, “for if you did, I should
certainly tell you all, and then Wickham would be
angry.”

On such encouragement to ask, Elizabeth was forced
to put it out of her power, by running away.

But to live in ignorance on such a point was impossible;
or at least it was impossible not to try for information.
Mr. Darcy had been at her sister’s wedding. It was
exactly a scene, and exactly among people, where he had
apparently least to do, and least temptation to go. Conjectures
as to the meaning of it, rapid and wild, hurried
into her brain; but she was satisfied with none. Those
that best pleased her, as placing his conduct in the noblest
light, seemed most improbable. She could not bear such
suspense; and hastily seizing a sheet of paper, wrote
a short letter to her aunt, to request an explanation of
what Lydia had dropt, if it were compatible with the
secrecy which had been intended.

“You may readily comprehend,” she added, “what my
curiosity must be to know how a person unconnected with
any of us, and (comparatively speaking) a stranger to our
family, should have been amongst you at such a time.
Pray write instantly, and let me understand it -- unless it
is, for very cogent reasons, to remain in the secrecy which
Lydia seems to think necessary; and then I must endeavour
to be satisfied with ignorance.”

“Not that I \textit{shall} though,” she added to herself, as
she finished the letter; “and my dear aunt, if you do
not tell me in an honourable manner, I shall certainly
be reduced to tricks and stratagems to find it out.”

Jane’s delicate sense of honour would not allow her to
speak to Elizabeth privately of what Lydia had let fall;
Elizabeth was glad of it; -- till it appeared whether her
inquiries would receive any satisfaction, she had rather
be without a confidante.
%%320%%

\Chapter{CHAPTER X.}

Elizabeth had the satisfaction of receiving an answer
to her letter, as soon as she possibly could. She was no
sooner in possession of it, than hurrying into the little
copse, where she was least likely to be interrupted, she
sat down on one of the benches, and prepared to be happy;
for the length of the letter convinced her that it did not
contain a denial.

\begin{letter}
\LetterDate{“Gracechurch-street, Sept. 6.}

“\textsc{My dear niece},

“I have just received your letter, and shall devote this
whole morning to answering it, as I foresee that a \textit{little}
writing will not comprise what I have to tell you. I must
confess myself surprised by your application; I did not
expect it from \textit{you}. Don’t think me angry, however, for I
only mean to let you know, that I had not imagined such
enquiries to be necessary on \textit{your} side. If you do not
choose to understand me, forgive my impertinence. Your
uncle is as much surprised as I am -- and nothing but the
belief of your being a party concerned, would have allowed
him to act as he has done. But if you are really innocent
and ignorant, I must be more explicit. On the very day
of my coming home from Longbourn, your uncle had a
most unexpected visitor. Mr. Darcy called, and was shut
up with him several hours. It was all over before I arrived;
so my curiosity was not so dreadfully racked as \textit{your’s}
seems to have been. He came to tell Mr. Gardiner that
he had found out where your sister and Mr. Wickham were,
and that he had seen and talked with them both, Wickham
repeatedly, Lydia once. From what I can collect, he left
Derbyshire only one day after ourselves, and came to
town with the resolution of hunting for them. The motive
professed, was his conviction of its being owing to himself
that Wickham’s worthlessness had not been so well known,
%%321%%
as to make it impossible for any young woman of character,
to love or confide in him. He generously imputed the
whole to his mistaken pride, and confessed that he
had before thought it beneath him, to lay his private
actions open to the world. His character was to speak
for itself. He called it, therefore, his duty to step
forward, and endeavour to remedy an evil, which had
been brought on by himself. If he \textit{had another} motive,
I am sure it would never disgrace him. He had
been some days in town, before he was able to discover
them; but he had something to direct his search, which
was more than \textit{we} had; and the consciousness of this,
was another reason for his resolving to follow us. There
is a lady, it seems, a Mrs. Younge, who was some
time ago governess to Miss Darcy, and was dismissed
from her charge on some cause of disapprobation, though
he did not say what. She then took a large house in
Edward-street, and has since maintained herself by letting
lodgings. This Mrs. Younge was, he knew, intimately
acquainted with Wickham; and he went to her for intelligence
of him, as soon as he got to town. But it was
two or three days before he could get from her what he
wanted. She would not betray her trust, I suppose,
without bribery and corruption, for she really did know
where her friend was to be found. Wickham indeed had
gone to her, on their first arrival in London, and had she
been able to receive them into her house, they would
have taken up their abode with her. At length, however,
our kind friend procured the wished-for direction. They
were in ------ street. He saw Wickham, and afterwards
insisted on seeing Lydia. His first object with her, he
acknowledged, had been to persuade her to quit her
present disgraceful situation, and return to her friends as
soon as they could be prevailed on to receive her, offering
his assistance, as far as it would go. But he found Lydia
absolutely resolved on remaining where she was. She
cared for none of her friends, she wanted no help of his,
she would not hear of leaving Wickham. She was sure
%%322%%
they should be married some time or other, and it did
not much signify when. Since such were her feelings,
it only remained, he thought, to secure and expedite
a marriage, which, in his very first conversation with
Wickham, he easily learnt, had never been \textit{his} design.
He confessed himself obliged to leave the regiment, on
account of some debts of honour, which were very pressing;
and scrupled not to lay all the ill-consequences of Lydia’s
flight, on her own folly alone. He meant to resign his
commission immediately; and as to his future situation,
he could conjecture very little about it. He must go
somewhere, but he did not know where, and he knew he
should have nothing to live on. Mr. Darcy asked him
why he had not married your sister at once. Though
Mr. Bennet was not imagined to be very rich, he would
have been able to do something for him, and his situation
must have been benefited by marriage. But he found,
in reply to this question, that Wickham still cherished
the hope of more effectually making his fortune by marriage,
in some other country. Under such circumstances,
however, he was not likely to be proof against the temptation
of immediate relief. They met several times, for
there was much to be discussed. Wickham of course
wanted more than he could get; but at length was
reduced to be reasonable. Every thing being settled
between \textit{them}, Mr. Darcy’s next step was to make your
uncle acquainted with it, and he first called in Gracechurch-%
street the evening before I came home. But Mr. Gardiner
could not be seen, and Mr. Darcy found, on further enquiry,
that your father was still with him, but would quit town
the next morning. He did not judge your father to be
a person whom he could so properly consult as your uncle,
and therefore readily postponed seeing him, till after the
departure of the former. He did not leave his name, and
till the next day, it was only known that a gentleman had
called on business. On Saturday he came again. Your
father was gone, your uncle at home, and, as I said before,
they had a great deal of talk together. They met again on
%%323%%
Sunday, and then \textit{I} saw him too. It was not all settled
before Monday: as soon as it was, the express was sent
off to Longbourn. But our visitor was very obstinate.
I fancy, Lizzy, that obstinacy is the real defect of his
character after all. He has been accused of many faults
at different times; but \textit{this} is the true one. Nothing was
to be done that he did not do himself; though I am
sure (and I do not speak it to be thanked, therefore say
nothing about it,) your uncle would most readily have
settled the whole. They battled it together for a long
time, which was more than either the gentleman or lady
concerned in it deserved. But at last your uncle was
forced to yield, and instead of being allowed to be of use
to his niece, was forced to put up with only having the
probable credit of it, which went sorely against the grain;
and I really believe your letter this morning gave him
great pleasure, because it required an explanation that
would rob him of his borrowed feathers, and give the
praise where it was due. But, Lizzy, this must go no
farther than yourself, or Jane at most. You know pretty
well, I suppose, what has been done for the young people.
His debts are to be paid, amounting, I believe, to considerably
more than a thousand pounds, another thousand in
addition to her own settled upon \textit{her}, and his commission
purchased. The reason why all this was to be done by
him alone, was such as I have given above. It was owing
to him, to his reserve, and want of proper consideration,
that Wickham’s character had been so misunderstood,
and consequently that he had been received and noticed
as he was. Perhaps there was some truth in \textit{this}; though
I doubt whether \textit{his} reserve, or \textit{anybody’s} reserve, can be
answerable for the event. But in spite of all this fine
talking, my dear Lizzy, you may rest perfectly assured,
that your uncle would never have yielded, if we had not
given him credit for \textit{another interest} in the affair. When
all this was resolved on, he returned again to his friends,
who were still staying at Pemberley; but it was agreed
that he should be in London once more when the wedding
%%324%%
took place, and all money matters were then to receive
the last finish. I believe I have now told you every
thing. It is a relation which you tell me is to give you
great surprise; I hope at least it will not afford you any
displeasure. Lydia came to us; and Wickham had
constant admission to the house. \textit{He} was exactly what
he had been, when I knew him in Hertfordshire; but
I would not tell you how little I was satisfied with \textit{her}
behaviour while she staid with us, if I had not perceived,
by Jane’s letter last Wednesday, that her conduct on
coming home was exactly of a piece with it, and therefore
what I now tell you, can give you no fresh pain. I talked
to her repeatedly in the most serious manner, representing to
her all the wickedness of what she had done, and all the
unhappiness she had brought on her family. If she heard
me, it was by good luck, for I am sure she did not listen.
I was sometimes quite provoked, but then I recollected
my dear Elizabeth and Jane, and for their sakes had
patience with her. Mr. Darcy was punctual in his return,
and as Lydia informed you, attended the wedding. He
dined with us the next day, and was to leave town again
on Wednesday or Thursday. Will you be very angry with
me, my dear Lizzy, if I take this opportunity of saying
(what I was never bold enough to say before) how much
I like him. His behaviour to us has, in every respect, been
as pleasing as when we were in Derbyshire. His understanding
and opinions all please me; he wants nothing
but a little more liveliness, and \textit{that}, if he marry \textit{prudently},
his wife may teach him. I thought him very sly; -- he
hardly ever mentioned your name. But slyness seems the
fashion. Pray forgive me, if I have been very presuming,
or at least do not punish me so far, as to exclude me from
P. I shall never be quite happy till I have been all round
the park. A low phaeton, with a nice little pair of ponies,
would be the very thing. But I must write no more.
The children have been wanting me this half hour. Your’s,
very sincerely,

\LetterSig{“M. Gardiner.”}
\end{letter}
%%325%%

The contents of this letter threw Elizabeth into a flutter
of spirits, in which it was difficult to determine whether
pleasure or pain bore the greatest share. The vague and
unsettled suspicions which uncertainty had produced of
what Mr. Darcy might have been doing to forward her
sister’s match, which she had feared to encourage, as an
exertion of goodness too great to be probable, and at the
same time dreaded to be just, from the pain of obligation,
were proved beyond their greatest extent to be true!
He had followed them purposely to town, he had taken
on himself all the trouble and mortification attendant on
such a research; in which supplication had been necessary
to a woman whom he must abominate and despise, and
where he was reduced to meet, frequently meet, reason
with, persuade, and finally bribe, the man whom he always
most wished to avoid, and whose very name it was punishment
to him to pronounce. He had done all this for
a girl whom he could neither regard nor esteem. Her
heart did whisper, that he had done it for her. But it
was a hope shortly checked by other considerations, and
she soon felt that even her vanity was insufficient, when
required to depend on his affection for her, for a woman
who had already refused him, as able to overcome a sentiment
so natural as abhorrence against relationship with
Wickham. Brother-in-law of Wickham! Every kind of
pride must revolt from the connection. He had to be sure
done much. She was ashamed to think how much. But
he had given a reason for his interference, which asked
no extraordinary stretch of belief. It was reasonable that
he should feel he had been wrong; he had liberality, and
he had the means of exercising it; and though she would
not place herself as his principal inducement, she could,
perhaps, believe, that remaining partiality for her, might
assist his endeavours in a cause where her peace of mind
must be materially concerned. It was painful, exceedingly
painful, to know that they were under obligations
to a person who could never receive a return. They owed
the restoration of Lydia, her character, every thing to
%%326%%
him. Oh! how heartily did she grieve over every ungracious
sensation she had ever encouraged, every saucy
speech she had ever directed towards him. For herself
she was humbled; but she was proud of him. Proud that
in a cause of compassion and honour, he had been able
to get the better of himself. She read over her aunt’s
commendation of him again and again. It was hardly
enough; but it pleased her. She was even sensible of
some pleasure, though mixed with regret, on finding how
steadfastly both she and her uncle had been persuaded
that affection and confidence subsisted between Mr. Darcy
and herself.

She was roused from her seat, and her reflections, by
some one’s approach; and before she could strike into
another path, she was overtaken by Wickham.

“I am afraid I interrupt your solitary ramble, my dear
sister?” said he, as he joined her.

“You certainly do,” she replied with a smile; “but
it does not follow that the interruption must be
unwelcome.”

“I should be sorry indeed, if it were. \textit{We} were always
good friends; and now we are better.”

“True. Are the others coming out?”

“I do not know. Mrs. Bennet and Lydia are going in
the carriage to Meryton. And so, my dear sister, I find
from our uncle and aunt, that you have actually seen
Pemberley.”

She replied in the affirmative.

“I almost envy you the pleasure, and yet I believe
it would be too much for me, or else I could take it in my
way to Newcastle. And you saw the old housekeeper,
I suppose? Poor Reynolds, she was always very fond of
me. But of course she did not mention my name to you.”

“Yes, she did.”

“And what did she say?”

“That you were gone into the army, and she was
afraid had -- not turned out well. At such a distance as
\textit{that}, you know, things are strangely misrepresented.”
%%327%%

“Certainly,” he replied, biting his lips. Elizabeth
hoped she had silenced him; but he soon afterwards
said,

“I was surprised to see Darcy in town last month. We
passed each other several times. I wonder what he can
be doing there.”

“Perhaps preparing for his marriage with Miss de
Bourgh,” said Elizabeth. “It must be something particular,
to take him there at this time of year.”

“Undoubtedly. Did you see him while you were at
Lamb\-ton? I thought I understood from the Gardiners
that you had.”

“Yes; he introduced us to his sister.”

“And do you like her?”

“Very much.”

“I have heard, indeed, that she is uncommonly improved
within this year or two. When I last saw her, she
was not very promising. I am very glad you liked her.
I hope she will turn out well.”

“I dare say she will; she has got over the most trying
age.”

“Did you go by the village of Kympton?”

“I do not recollect that we did.”

“I mention it, because it is the living which I ought to
have had. A most delightful place! -- Excellent Parsonage
House! It would have suited me in every respect.”

“How should you have liked making sermons?”

“Exceedingly well. I should have considered it as part
of my duty, and the exertion would soon have been
nothing. One ought not to repine; -- but, to be sure, it
would have been such a thing for me! The quiet, the
retirement of such a life, would have answered all my
ideas of happiness! But it was not to be. Did you ever
hear Darcy mention the circumstance, when you were in
Kent?”

“I \textit{have} heard from authority, which I thought \textit{as good},
that it was left you conditionally only, and at the will
of the present patron.”
%%328%%

“You have. Yes, there was something in \textit{that}; I told
you so from the first, you may remember.”

“I \textit{did} hear, too, that there was a time, when sermon-making
was not so palatable to you as it seems to be at
present; that you actually declared your resolution of
never taking orders, and that the business had been
compromised accordingly.”

“You did! and it was not wholly without foundation.
You may remember what I told you on that point, when
first we talked of it.”

They were now almost at the door of the house, for
she had walked fast to get rid of him; and unwilling for
her sister’s sake, to provoke him, she only said in reply,
with a good-humoured smile,

“Come, Mr. Wickham, we are brother and sister, you
know. Do not let us quarrel about the past. In future,
I hope we shall be always of one mind.”

She held out her hand; he kissed it with affectionate
gallantry, though he hardly knew how to look, and they
entered the house.
%%329%%

\Chapter{CHAPTER XI.}

Mr. Wickham was so perfectly satisfied with this conversation,
that he never again distressed himself, or
provoked his dear sister Elizabeth, by introducing the
subject of it; and she was pleased to find that she had
said enough to keep him quiet.

The day of his and Lydia’s departure soon came, and
Mrs. Bennet was forced to submit to a separation, which,
as her husband by no means entered into her scheme of
their all going to Newcastle, was likely to continue at
least a twelvemonth.

“Oh! my dear Lydia,” she cried, “when shall we meet
again?”

“Oh, lord! I don’t know. Not these two or three
years perhaps.”

“Write to me very often, my dear.”

“As often as I can. But you know married women
have never much time for writing. My sisters may write
to \textit{me}. They will have nothing else to do.”

Mr. Wickham’s adieus were much more affectionate than
his wife’s. He smiled, looked handsome, and said many
pretty things.

“He is as fine a fellow,” said Mr. Bennet, as soon as
they were out of the house, “as ever I saw. He simpers,
and smirks, and makes love to us all. I am prodigiously
proud of him. I defy even Sir William Lucas himself,
to produce a more valuable son-in-law.”

The loss of her daughter made Mrs. Bennet very dull
for several days.

“I often think,” said she, “that there is nothing so
bad as parting with one’s friends. One seems so forlorn
without them.”

“This is the consequence you see, Madam, of marrying
a daughter,” said Elizabeth. “It must make you better
satisfied that your other four are single.”
%%330%%

“It is no such thing. Lydia does not leave me because
she is married; but only because her husband’s regiment
happens to be so far off. If that had been nearer, she
would not have gone so soon.”

But the spiritless condition which this event threw her
into, was shortly relieved, and her mind opened again to
the agitation of hope, by an article of news, which then
began to be in circulation. The housekeeper at Netherfield
had received orders to prepare for the arrival of her
master, who was coming down in a day or two, to shoot
there for several weeks. Mrs. Bennet was quite in the
fidgets. She looked at Jane, and smiled, and shook her
head by turns.

“Well, well, and so Mr. Bingley is coming down, sister,”
(for Mrs. Phillips first brought her the news.) “Well, so
much the better. Not that I care about it, though. He
is nothing to us, you know, and I am sure \textit{I} never want
to see him again. But, however, he is very welcome to
come to Netherfield, if he likes it. And who knows what
\textit{may} happen? But that is nothing to us. You know,
sister, we agreed long ago never to mention a word about
it. And so, is it quite certain he is coming?”

“You may depend on it,” replied the other, “for
Mrs. Nich\-olls was in Meryton last night; I saw her
passing by, and went out myself on purpose to know the
truth of it; and she told me that it was certain true.
He comes down on Thursday at the latest, very likely
on Wednesday. She was going to the butcher’s, she told
me, on purpose to order in some meat on Wednesday, and
she has got three couple of ducks, just fit to be killed.”

Miss Bennet had not been able to hear of his coming,
without changing colour. It was many months since she
had mentioned his name to Elizabeth; but now, as soon
as they were alone together, she said,

“I saw you look at me to day, Lizzy, when my aunt
told us of the present report; and I know I appeared
distressed. But don’t imagine it was from any silly cause.
I was only confused for the moment, because I felt that
%%331%%
I \textit{should} be looked at. I do assure you, that the news does
not affect me either with pleasure or pain. I am glad
of one thing, that he comes alone; because we shall see
the less of him. Not that I am afraid of \textit{myself}, but I dread
other people’s remarks.”

Elizabeth did not know what to make of it. Had she
not seen him in Derbyshire, she might have supposed him
capable of coming there, with no other view than what
was acknowledged; but she still thought him partial to
Jane, and she wavered as to the greater probability of
his coming there \textit{with} his friend’s permission, or being bold
enough to come without it.

“Yet it is hard,” she sometimes thought, “that this
poor man cannot come to a house, which he has legally
hired, without raising all this speculation! I \textit{will} leave him
to himself.”

In spite of what her sister declared, and really believed
to be her feelings, in the expectation of his arrival, Elizabeth
could easily perceive that her spirits were affected
by it. They were more disturbed, more unequal, than she
had often seen them.

The subject which had been so warmly canvassed
between their parents, about a twelvemonth ago, was now
brought forward again.

“As soon as ever Mr. Bingley comes, my dear,” said
Mrs. Bennet, “you will wait on him of course.”

“No, no. You forced me into visiting him last year,
and promised if I went to see him, he should marry one
of my daughters. But it ended in nothing, and I will
not be sent on a fool’s errand again.”

His wife represented to him how absolutely necessary
such an attention would be from all the neighbouring
gentlemen, on his returning to Netherfield.

“’Tis an etiquette I despise,” said he. “If he wants
our society, let him seek it. He knows where we live.
I will not spend \textit{my} hours in running after my neighbours
every time they go away, and come back again.”

“Well, all I know is, that it will be abominably rude
%%332%%
if you do not wait on him. But, however, that shan’t
prevent my asking him to dine here, I am determined.
We must have Mrs. Long and the Gouldings soon. That
will make thirteen with ourselves, so there will be just
room at table for him.”

Consoled by this resolution, she was the better able to
bear her husband’s incivility; though it was very mortifying
to know that her neighbours might all see Mr. Bingley
in consequence of it, before \textit{they} did. As the day of his
arrival drew near,

“I begin to be sorry that he comes at all,” said Jane
to her sister. “It would be nothing; I could see him
with perfect indifference, but I can hardly bear to hear
it thus perpetually talked of. My mother means well;
but she does not know, no one can know how much
I suffer from what she says. Happy shall I be, when his
stay at Netherfield is over!”

“I wish I could say any thing to comfort you,” replied
Elizabeth; “but it is wholly out of my power. You
must feel it; and the usual satisfaction of preaching
patience to a sufferer is denied me, because you have
always so much.”

Mr. Bingley arrived. Mrs. Bennet, through the assistance
of servants, contrived to have the earliest tidings
of it, that the period of anxiety and fretfulness on her
side, might be as long as it could. She counted the days
that must intervene before their invitation could be sent;
hopeless of seeing him before. But on the third morning
after his arrival in Hertfordshire, she saw him from her
dressing-room window, enter the paddock, and ride towards
the house.

Her daughters were eagerly called to partake of her joy.
Jane resolutely kept her place at the table; but Elizabeth,
to satisfy her mother, went to the window -- she looked, -- she
saw Mr. Darcy with him, and sat down again by her
sister.

“There is a gentleman with him, mamma,” said Kitty;
“who can it be?”
%%333%%

“Some acquaintance or other, my dear, I suppose;
I am sure I do not know.”

“La!” replied Kitty, “it looks just like that man that
used to be with him before. Mr. what’s his name. That
tall, proud man.”

“Good gracious! Mr. Darcy! -- and so it does I vow.
Well, any friend of Mr. Bingley’s will always be welcome
here to be sure; but else I must say that I hate the very
sight of him.”

Jane looked at Elizabeth with surprise and concern.
She knew but little of their meeting in Derbyshire, and
therefore felt for the awkwardness which must attend
her sister, in seeing him almost for the first time after
receiving his explanatory letter. Both sisters were uncomfortable
enough. Each felt for the other, and of course
for themselves; and their mother talked on, of her dislike
of Mr. Darcy, and her resolution to be civil to him only
as Mr. Bingley’s friend, without being heard by either of
them. But Elizabeth had sources of uneasiness which
could not be suspected by Jane, to whom she had never
yet had courage to shew Mrs. Gardiner’s letter, or to
relate her own change of sentiment towards him. To
Jane, he could be only a man whose proposals she had
refused, and whose merit she had undervalued; but to
her own more extensive information, he was the person,
to whom the whole family were indebted for the first of
benefits, and whom she regarded herself with an interest,
if not quite so tender, at least as reasonable and just, as
what Jane felt for Bingley. Her astonishment at his
coming -- at his coming to Netherfield, to Longbourn, and
voluntarily seeking her again, was almost equal to what
she had known on first witnessing his altered behaviour
in Derbyshire.

The colour which had been driven from her face, returned
for half a minute with an additional glow, and a smile
of delight added lustre to her eyes, as she thought for that
space of time, that his affection and wishes must still be
unshaken. But she would not be secure.
%%334%%

“Let me first see how he behaves,” said she; “it will
then be early enough for expectation.”

She sat intently at work, striving to be composed, and
without daring to lift up her eyes, till anxious curiosity
carried them to the face of her sister, as the servant was
approaching the door. Jane looked a little paler than
usual, but more sedate than Elizabeth had expected.
On the gentlemen’s appearing, her colour increased; yet
she received them with tolerable ease, and with a propriety
of behaviour equally free from any symptom of
resentment, or any unnecessary complaisance.

Elizabeth said as little to either as civility would allow,
and sat down again to her work, with an eagerness which
it did not often command. She had ventured only one
glance at Darcy. He looked serious as usual; and she
thought, more as he had been used to look in Hertfordshire,
than as she had seen him at Pemberley. But, perhaps
he could not in her mother’s presence be what he was
before her uncle and aunt. It was a painful, but not an
improbable, conjecture.

Bingley, she had likewise seen for an instant, and in
that short period saw him looking both pleased and
embarrassed. He was received by Mrs. Bennet with
a degree of civility, which made her two daughters
ashamed, especially when contrasted with the cold and
ceremonious politeness of her curtsey and address to his
friend.

Elizabeth particularly, who knew that her mother owed
to the latter the preservation of her favourite daughter
from irremediable infamy, was hurt and distressed to
a most painful degree by a distinction so ill applied.

Darcy, after enquiring of her how Mr. and Mrs. Gardiner
did, a question which she could not answer without confusion,
said scarcely any thing. He was not seated by her;
perhaps that was the reason of his silence; but it had
not been so in Derbyshire. There he had talked to her
friends, when he could not to herself. But now several
%%335%%
minutes elapsed, without bringing the sound of his voice;
and when occasionally, unable to resist the impulse of
curiosity, she raised her eyes to his face, she as often
found him looking at Jane, as at herself, and frequently
on no object but the ground. More thoughtfulness, and
less anxiety to please than when they last met, were
plainly expressed. She was disappointed, and angry with
herself for being so.

“Could I expect it to be otherwise!” said she. “Yet
why did he come?”

She was in no humour for conversation with any one
but himself; and to him she had hardly courage to speak.

She enquired after his sister, but could do no more.

“It is a long time, Mr. Bingley, since you went away,”
said Mrs. Bennet.

He readily agreed to it.

“I began to be afraid you would never come back
again. People \textit{did} say, you meant to quit the place entirely
at Mich\-aelmas; but, however, I hope it is not true.
A great many changes have happened in the neighbourhood,
since you went away. Miss Lucas is married and
settled. And one of my own daughters. I suppose you
have heard of it; indeed, you must have seen it in the
papers. It was in the Times and the Courier, I know;
though it was not put in as it ought to be. It was only said,
‘Lately, George Wickham, Esq. to Miss Lydia Bennet,’
without there being a syllable said of her father, or the
place where she lived, or any thing. It was my brother
Gardiner’s drawing up too, and I wonder how he came
to make such an awkward business of it. Did you see it?”

Bingley replied that he did, and made his congratulations.
Elizabeth dared not lift up her eyes. How Mr.
Darcy looked, therefore, she could not tell.

“It is a delightful thing, to be sure, to have a daughter
well married,” continued her mother, “but at the same
time, Mr. Bingley, it is very hard to have her taken such
a way from me. They are gone down to Newcastle, a place
quite northward, it seems, and there they are to stay,
%%336%%
I do not know how long. His regiment is there; for I
suppose you have heard of his leaving the ------shire, and
of his being gone into the regulars. Thank Heaven! he
has \textit{some} friends, though perhaps not so many as he
deserves.”

Elizabeth, who knew this to be levelled at Mr. Darcy,
was in such misery of shame, that she could hardly keep
her seat. It drew from her, however, the exertion of
speaking, which nothing else had so effectually done
before; and she asked Bingley, whether he meant to
make any stay in the country at present. A few weeks,
he believed.

“When you have killed all your own birds, Mr. Bingley,”
said her mother, “I beg you will come here, and shoot
as many as you please, on Mr. Bennet’s manor. I am
sure he will be vastly happy to oblige you, and will save
all the best of the covies for you.”

Elizabeth’s misery increased, at such unnecessary, such
officious attention! Were the same fair prospect to arise
at present, as had flattered them a year ago, every thing,
she was persuaded, would be hastening to the same
vexatious conclusion. At that instant she felt, that years
of happiness could not make Jane or herself amends, for
moments of such painful confusion.

“The first wish of my heart,” said she to herself, “is
never more to be in company with either of them. Their
society can afford no pleasure, that will atone for such
wretchedness as this! Let me never see either one or the
other again!”

Yet the misery, for which years of happiness were to
offer no compensation, received soon afterwards material
relief, from observing how much the beauty of her sister
re-kindled the admiration of her former lover. When
first he came in, he had spoken to her but little; but
every five minutes seemed to be giving her more of his
attention. He found her as handsome as she had been
last year; as good natured, and as unaffected, though
not quite so chatty. Jane was anxious that no difference
%%337%%
should be perceived in her at all, and was really persuaded
that she talked as much as ever. But her mind was so
busily engaged, that she did not always know when she
was silent.

When the gentlemen rose to go away, Mrs. Bennet was
mindful of her intended civility, and they were invited
and engaged to dine at Longbourn in a few days time.

“You are quite a visit in my debt, Mr. Bingley,” she
added, “for when you went to town last winter, you
promised to take a family dinner with us, as soon as you
returned. I have not forgot, you see; and I assure you,
I was very much disappointed that you did not come back
and keep your engagement.”

Bingley looked a little silly at this reflection, and said
something of his concern, at having been prevented by
business. They then went away.

Mrs. Bennet had been strongly inclined to ask them to
stay and dine there, that day; but, though she always
kept a very good table, she did not think any thing less
than two courses, could be good enough for a man, on
whom she had such anxious designs, or satisfy the appetite
and pride of one who had ten thousand a-year.
%%338%%

\Chapter{CHAPTER XII.}

As soon as they were gone, Elizabeth walked out to
recover her spirits; or in other words, to dwell without
interruption on those subjects that must deaden them
more. Mr. Darcy’s behaviour astonished and vexed her.

“Why, if he came only to be silent, grave, and indifferent,”
said she, “did he come at all?”

She could settle it in no way that gave her pleasure.

“He could be still amiable, still pleasing, to my uncle
and aunt, when he was in town; and why not to me?
If he fears me, why come hither? If he no longer cares
for me, why silent? Teazing, teazing, man! I will think
no more about him.”

Her resolution was for a short time involuntarily kept
by the approach of her sister, who joined her with a cheerful
look, which shewed her better satisfied with their
visitors, than Elizabeth.

“Now,” said she, “that this first meeting is over, I feel
perfectly easy. I know my own strength, and I shall never
be embarrassed again by his coming. I am glad he dines
here on Tuesday. It will then be publicly seen, that on
both sides, we meet only as common and indifferent
acquaintance.”

“Yes, very indifferent indeed,” said Elizabeth, laughingly.
“Oh, Jane, take care.”

“My dear Lizzy, you cannot think me so weak, as to
be in danger now.”

“I think you are in very great danger of making him
as much in love with you as ever.”

\strut

They did not see the gentlemen again till Tuesday;
and Mrs. Bennet, in the meanwhile, was giving way to all
the happy schemes, which the good humour, and common
politeness of Bingley, in half an hour’s visit, had revived.
%%339%%

On Tuesday there was a large party assembled at Longbourn;
and the two, who were most anxiously expected,
to the credit of their punctuality as sportsmen, were in
very good time. When they repaired to the dining-room,
Elizabeth eagerly watched to see whether Bingley
would take the place, which, in all their former parties,
had belonged to him, by her sister. Her prudent mother,
occupied by the same ideas, forbore to invite him to sit
by herself. On entering the room, he seemed to hesitate;
but Jane happened to look round, and happened to smile:
it was decided. He placed himself by her.

Elizabeth, with a triumphant sensation, looked towards
his friend. He bore it with noble indifference, and she
would have imagined that Bingley had received his
sanction to be happy, had she not seen his eyes likewise
turned towards Mr. Darcy, with an expression of half-laughing
alarm.

His behaviour to her sister was such, during dinner
time, as shewed an admiration of her, which, though more
guarded than formerly, persuaded Elizabeth, that if left
wholly to himself, Jane’s happiness, and his own, would
be speedily secured. Though she dared not depend upon
the consequence, she yet received pleasure from observing
his behaviour. It gave her all the animation that her
spirits could boast; for she was in no cheerful humour.
Mr. Darcy was almost as far from her, as the table could
divide them. He was on one side of her mother. She
knew how little such a situation would give pleasure to
either, or make either appear to advantage. She was not
near enough to hear any of their discourse, but she could
see how seldom they spoke to each other, and how formal
and cold was their manner, whenever they did. Her
mother’s ungraciousness, made the sense of what they
owed him more painful to Elizabeth’s mind; and she
would, at times, have given any thing to be privileged to
tell him, that his kindness was neither unknown nor unfelt
by the whole of the family.

She was in hopes that the evening would afford some
%%340%%
opportunity of bringing them together; that the whole
of the visit would not pass away without enabling them
to enter into something more of conversation, than the
mere ceremonious salutation attending his entrance.
Anxious and uneasy, the period which passed in the
drawing-room, before the gentlemen came, was wearisome
and dull to a degree, that almost made her uncivil. She
looked forward to their entrance, as the point on
which all her chance of pleasure for the evening must
depend.

“If he does not come to me, \textit{then},” said she, “I shall
give him up for ever.”

The gentlemen came; and she thought he looked as
if he would have answered her hopes; but, alas! the
ladies had crowded round the table, where Miss Bennet
was making tea, and Elizabeth pouring out the coffee, in
so close a confederacy, that there was not a single vacancy
near her, which would admit of a chair. And on the
gentlemen’s approaching, one of the girls moved closer
to her than ever, and said, in a whisper,

“The men shan’t come and part us, I am determined.
We want none of them; do we?”

Darcy had walked away to another part of the room.
She followed him with her eyes, envied every one to whom
he spoke, had scarcely patience enough to help anybody
to coffee; and then was enraged against herself for being
so silly!

“A man who has once been refused! How could I ever
be foolish enough to expect a renewal of his love? Is there
one among the sex, who would not protest against such
a weakness as a second proposal to the same woman?
There is no indignity so abhorrent to their feelings!”

She was a little revived, however, by his bringing back
his coffee cup himself; and she seized the opportunity
of saying,

“Is your sister at Pemberley still?”

“Yes, she will remain there till Christmas.”

“And quite alone? Have all her friends left her?”
%%341%%

“Mrs. Annesley is with her. The others have been gone
on to Scarborough, these three weeks.”

She could think of nothing more to say; but if he
wished to converse with her, he might have better success.
He stood by her, however, for some minutes, in silence;
and, at last, on the young lady’s whispering to Elizabeth
again, he walked away.

When the tea-things were removed, and the card tables
placed, the ladies all rose, and Elizabeth was then hoping
to be soon joined by him, when all her views were overthrown,
by seeing him fall a victim to her mother’s rapacity
for whist players, and in a few moments after seated with
the rest of the party. She now lost every expectation of
pleasure. They were confined for the evening at different
tables, and she had nothing to hope, but that his eyes
were so often turned towards her side of the room, as to
make him play as unsuccessfully as herself.

Mrs. Bennet had designed to keep the two Netherfield
gentlemen to supper; but their carriage was unluckily
ordered before any of the others, and she had no opportunity
of detaining them.

“Well girls,” said she, as soon as they were left to
themselves, “What say you to the day? I think every
thing has passed off uncommonly well, I assure you. The
dinner was as well dressed as any I ever saw. The venison
was roasted to a turn -- and everybody said, they never
saw so fat a haunch. The soup was fifty times better
than what we had at the Lucas’s last week; and even
Mr. Darcy acknowledged, that the partridges were remarkably
well done; and I suppose he has two or three French
cooks at least. And, my dear Jane, I never saw you look
in greater beauty. Mrs. Long said so too, for I asked her
whether you did not. And what do you think she said
besides? ‘Ah! Mrs. Bennet, we shall have her at Netherfield
at last.’ She did indeed. I do think Mrs. Long is
as good a creature as ever lived -- and her nieces are very
pretty behaved girls, and not at all handsome: I like
them prodigiously.”
%%342%%

Mrs. Bennet, in short, was in very great spirits; she
had seen enough of Bingley’s behaviour to Jane, to be
convinced that she would get him at last; and her expectations
of advantage to her family, when in a happy
humour, were so far beyond reason, that she was quite
disappointed at not seeing him there again the next day,
to make his proposals.

“It has been a very agreeable day,” said Miss Bennet
to Elizabeth. “The party seemed so well selected, so
suitable one with the other. I hope we may often meet
again.”

Elizabeth smiled.

“Lizzy, you must not do so. You must not suspect
me. It mortifies me. I assure you that I have now learnt
to enjoy his conversation as an agreeable and sensible
young man, without having a wish beyond it. I am perfectly
satisfied from what his manners now are, that he never
had any design of engaging my affection. It is only that
he is blessed with greater sweetness of address, and a
stronger desire of generally pleasing than any other man.”

“You are very cruel,” said her sister, “you will not
let me smile, and are provoking me to it every moment.”

“How hard it is in some cases to be believed!”

“And how impossible in others!”

“But why should you wish to persuade me that I feel
more than I acknowledge?”

“That is a question which I hardly know how to answer.
We all love to instruct, though we can teach only what
is not worth knowing. Forgive me; and if you persist
in indifference, do not make \textit{me} your confidante.”
%%343%%

\Chapter{CHAPTER XIII.}

A few days after this visit, Mr. Bingley called again,
and alone. His friend had left him that morning for
London, but was to return home in ten days time. He
sat with them above an hour, and was in remarkably good
spirits. Mrs. Bennet invited him to dine with them; but,
with many expressions of concern, he confessed himself
engaged elsewhere.

“Next time you call,” said she, “I hope we shall be
more lucky.”

He should be particularly happy at any time, \&c. \&c.;
and if she would give him leave, would take an early
opportunity of waiting on them.

“Can you come to-morrow?”

Yes, he had no engagement at all for to-morrow; and
her invitation was accepted with alacrity.

He came, and in such very good time, that the ladies
were none of them dressed. In ran Mrs. Bennet to her
daughter’s room, in her dressing gown, and with her hair
half finished, crying out,

“My dear Jane, make haste and hurry down. He is
come -- Mr. Bingley is come. -- He is, indeed. Make haste,
make haste. Here, Sarah, come to Miss Bennet this
moment, and help her on with her gown. Never mind
Miss Lizzy’s hair.”

“We will be down as soon as we can,” said Jane;
“but I dare say Kitty is forwarder than either of us, for
she went up stairs half an hour ago.”

“Oh! hang Kitty! what has she to do with it? Come
be quick, be quick! where is your sash my dear?”

But when her mother was gone, Jane would not be
prevailed on to go down without one of her sisters.

The same anxiety to get them by themselves, was
visible again in the evening. After tea, Mr. Bennet retired
to the library, as was his custom, and Mary went up stairs
%%344%%
to her instrument. Two obstacles of the five being thus
removed, Mrs. Bennet sat looking and winking at Elizabeth
and Catherine for a considerable time, without making
any impression on them. Elizabeth would not observe
her; and when at last Kitty did, she very innocently
said, “What is the matter mamma? What do you keep
winking at me for? What am I to do?”

“Nothing child, nothing. I did not wink at you.”
She then sat still five minutes longer; but unable to
waste such a precious occasion, she suddenly got up, and
saying to Kitty,

“Come here, my love, I want to speak to you,” took
her out of the room. Jane instantly gave a look at Elizabeth,
which spoke her distress at such premeditation, and
her intreaty that \textit{she} would not give into it. In a few
minutes, Mrs. Bennet half opened the door and called out,

“Lizzy, my dear, I want to speak with you.”

Elizabeth was forced to go.

“We may as well leave them by themselves you
know;” said her mother as soon as she was in the hall.
“Kitty and I are going up stairs to sit in my dressing room.”

Elizabeth made no attempt to reason with her mother,
but remained quietly in the hall, till she and Kitty were
out of sight, then returned into the drawing room.

Mrs. Bennet’s schemes for this day were ineffectual.
Bingley was every thing that was charming, except the
professed lover of her daughter. His ease and cheerfulness
rendered him a most agreeable addition to their evening
party; and he bore with the ill-judged officiousness of
the mother, and heard all her silly remarks with a forbearance
and command of countenance, particularly
grateful to the daughter.

He scarcely needed an invitation to stay supper; and
before he went away, an engagement was formed, chiefly
through his own and Mrs. Bennet’s means, for his coming
next morning to shoot with her husband.

After this day, Jane said no more of her indifference.
Not a word passed between the sisters concerning Bingley;
%%345%%
but Elizabeth went to bed in the happy belief that all
must speedily be concluded, unless Mr. Darcy returned
within the stated time. Seriously, however, she felt
tolerably persuaded that all this must have taken place
with that gentleman’s concurrence.

Bingley was punctual to his appointment; and he and
Mr. Bennet spent the morning together, as had been agreed
on. The latter was much more agreeable than his companion
expected. There was nothing of presumption or
folly in Bingley, that could provoke his ridicule, or disgust
him into silence; and he was more communicative, and
less eccentric than the other had ever seen him. Bingley
of course returned with him to dinner; and in the evening
Mrs. Bennet’s invention was again at work to get every
body away from him and her daughter. Elizabeth, who
had a letter to write, went into the breakfast room for
that purpose soon after tea; for as the others were all
going to sit down to cards, she could not be wanted to
counteract her mother’s schemes.

But on returning to the drawing room, when her letter
was finished, she saw, to her infinite surprise, there was
reason to fear that her mother had been too ingenious for
her. On opening the door, she perceived her sister and
Bingley standing together over the hearth, as if engaged in
earnest conversation; and had this led to no suspicion,
the faces of both as they hastily turned round, and moved
away from each other, would have told it all. \textit{Their}
situation was awkward enough; but \textit{her’s} she thought was
still worse. Not a syllable was uttered by either; and Elizabeth
was on the point of going away again, when Bingley,
who as well as the other had sat down, suddenly rose, and
whispering a few words to her sister, ran out of the room.

Jane could have no reserves from Elizabeth, where
confidence would give pleasure; and instantly embracing
her, acknowledged, with the liveliest emotion, that she
was the happiest creature in the world.

“’Tis too much!” she added, “by far too much. I do
not deserve it. Oh! why is not every body as happy?”
%%346%%

Elizabeth’s congratulations were given with a sincerity,
a warmth, a delight, which words could but poorly express.
Every sentence of kindness was a fresh source of happiness
to Jane. But she would not allow herself to stay with her
sister, or say half that remained to be said, for the present.

“I must go instantly to my mother;” she cried.
“I would not on any account trifle with her affectionate
solicitude; or allow her to hear it from any one but myself.
He is gone to my father already. Oh! Lizzy, to know
that what I have to relate will give such pleasure to all
my dear family! how shall I bear so much happiness!”

She then hastened away to her mother, who had purposely
broken up the card party, and was sitting up stairs
with Kitty.

Elizabeth, who was left by herself, now smiled at the
rapidity and ease with which an affair was finally settled,
that had given them so many previous months of suspense
and vexation.

“And this,” said she, “is the end of all his friend’s
anxious circumspection! of all his sister’s falsehood
and contrivance! the happiest, wisest, most reasonable
end!”

In a few minutes she was joined by Bingley, whose
conference with her father had been short and to the
purpose.

“Where is your sister?” said he hastily, as he opened
the door.

“With my mother up stairs. She will be down in a
moment I dare say.”

He then shut the door, and coming up to her, claimed
the good wishes and affection of a sister. Elizabeth
honestly and heartily expressed her delight in the prospect
of their relationship. They shook hands with great
cordiality; and then till her sister came down, she had
to listen to all he had to say, of his own happiness, and
of Jane’s perfections; and in spite of his being a lover,
Elizabeth really believed all his expectations of felicity,
to be rationally founded, because they had for basis the
%%347%%
excellent understanding, and super-excellent disposition
of Jane, and a general similarity of feeling and taste
between her and himself.

It was an evening of no common delight to them all;
the satisfaction of Miss Bennet’s mind gave a glow of
such sweet animation to her face, as made her look handsomer
than ever. Kitty simpered and smiled, and hoped
her turn was coming soon. Mrs. Bennet could not give
her consent, or speak her approbation in terms warm
enough to satisfy her feelings, though she talked to Bingley
of nothing else, for half an hour; and when Mr. Bennet
joined them at supper, his voice and manner plainly
shewed how really happy he was.

Not a word, however, passed his lips in allusion to it,
till their visitor took his leave for the night; but as soon
as he was gone, he turned to his daughter and said,

\begin{sloppypar}
“Jane, I congratulate you. You will be a very happy
woman.”
\end{sloppypar}

Jane went to him instantly, kissed him, and thanked
him for his goodness.

“You are a good girl;” he replied, “and I have great
pleasure in thinking you will be so happily settled. I have
not a doubt of your doing very well together. Your
tempers are by no means unlike. You are each of you so
complying, that nothing will ever be resolved on; so
easy, that every servant will cheat you; and so generous,
that you will always exceed your income.”

“I hope not so. Imprudence or thoughtlessness in
money matters, would be unpardonable in \textit{me}.”

“Exceed their income! My dear Mr. Bennet,” cried
his wife, “what are you talking of? Why, he has four
or five thousand a-year, and very likely more.” Then
addressing her daughter, “Oh! my dear, dear Jane, I am
so happy! I am sure I sha’nt get a wink of sleep all night.
I knew how it would be. I always said it must be so,
at last. I was sure you could not be so beautiful for
nothing! I remember, as soon as ever I saw him, when
he first came into Hertfordshire last year, I thought how
%%348%%
likely it was that you should come together. Oh! he is
the handsomest young man that ever was seen!”

Wickham, Lydia, were all forgotten. Jane was beyond
competition her favourite child. At that moment, she
cared for no other. Her younger sisters soon began to
make interest with her for objects of happiness which she
might in future be able to dispense.

Mary petitioned for the use of the library at Netherfield;
and Kitty begged very hard for a few balls there every
winter.

Bingley, from this time, was of course a daily visitor
at Longbourn; coming frequently before breakfast, and
always remaining till after supper; unless when some
barbarous neighbour, who could not be enough detested,
had given him an invitation to dinner, which he thought
himself obliged to accept.

Elizabeth had now but little time for conversation with
her sister; for while he was present, Jane had no attention
to bestow on any one else; but she found herself considerably
useful to both of them, in those hours of separation
that must sometimes occur. In the absence of Jane,
he always attached himself to Elizabeth, for the pleasure
of talking of her; and when Bingley was gone, Jane
constantly sought the same means of relief.

“He has made me so happy,” said she, one evening,
“by telling me, that he was totally ignorant of my
being in town last spring! I had not believed it
possible.”

“I suspected as much,” replied Elizabeth. “But how
did he account for it?”

“It must have been his sister’s doing. They were
certainly no friends to his acquaintance with me, which
I cannot wonder at, since he might have chosen so much
more advantageously in many respects. But when they
see, as I trust they will, that their brother is happy with
me, they will learn to be contented, and we shall be on
good terms again; though we can never be what we once
were to each other.”
%%349%%

“That is the most unforgiving speech,” said Elizabeth,
“that I ever heard you utter. Good girl! It would vex
me, indeed, to see you again the dupe of Miss Bingley’s
pretended regard.”

“Would you believe it, Lizzy, that when he went to
town last November, he really loved me, and nothing but
a persuasion of \textit{my} being indifferent, would have prevented
his coming down again!”

“He made a little mistake to be sure; but it is to
the credit of his modesty.”

This naturally introduced a panegyric from Jane on
his diffidence, and the little value he put on his own good
qualities.

Elizabeth was pleased to find, that he had not betrayed
the interference of his friend, for, though Jane had the
most generous and forgiving heart in the world, she knew it
was a circumstance which must prejudice her against him.

“I am certainly the most fortunate creature that ever
existed!” cried Jane. “Oh! Lizzy, why am I thus
singled from my family, and blessed above them all!
If I could but see \textit{you} as happy! If there \textit{were} but such
another man for you!”

“If you were to give me forty such men, I never could
be so happy as you. Till I have your disposition, your
goodness, I never can have your happiness. No, no, let
me shift for myself; and, perhaps, if I have very good
luck, I may meet with another Mr. Collins in time.”

The situation of affairs in the Longbourn family could
not be long a secret. Mrs. Bennet was privileged to whisper
it to Mrs. Philips, and \textit{she} ventured, without any permission,
to do the same by all her neighbours in Meryton.

The Bennets were speedily pronounced to be the luckiest
family in the world, though only a few weeks before, when
Lydia had first run away, they had been generally proved
to be marked out for misfortune.
%%350%%

\Chapter{CHAPTER XIV.}

One morning, about a week after Bingley’s engagement
with Jane had been formed, as he and the females of the
family were sitting together in the dining room, their
attention was suddenly drawn to the window, by the
sound of a carriage; and they perceived a chaise and four
driving up the lawn. It was too early in the morning for
visitors, and besides, the equipage did not answer to that
of any of their neighbours. The horses were post; and
neither the carriage, nor the livery of the servant who
preceded it, were familiar to them. As it was certain,
however, that somebody was coming, Bingley instantly
prevailed on Miss Bennet to avoid the confinement of
such an intrusion, and walk away with him into the
shrubbery. They both set off, and the conjectures of the
remaining three continued, though with little satisfaction,
till the door was thrown open, and their visitor entered.
It was lady Catherine de Bourgh.

They were of course all intending to be surprised; but
their astonishment was beyond their expectation; and on
the part of Mrs. Bennet and Kitty, though she was perfectly
unknown to them, even inferior to what Elizabeth felt.

She entered the room with an air more than usually
ungracious, made no other reply to Elizabeth’s salutation,
than a slight inclination of the head, and sat down without
saying a word. Elizabeth had mentioned her name to
her mother, on her ladyship’s entrance, though no request
of introduction had been made.

Mrs. Bennet all amazement, though flattered by having
a guest of such high importance, received her with the
utmost politeness. After sitting for a moment in silence,
she said very stiffly to Elizabeth,

“I hope you are well, Miss Bennet. That lady I suppose
is your mother.”
%%351%%

Elizabeth replied very concisely that she was.

“And \textit{that} I suppose is one of your sisters.”

“Yes, madam,” said Mrs. Bennet, delighted to speak
to a lady Catherine. “She is my youngest girl but one.
My youngest of all, is lately married, and my eldest is
some-where about the grounds, walking with a young man,
who I believe will soon become a part of the family.”

“You have a very small park here,” returned lady
Catherine after a short silence.

“It is nothing in comparison of Rosings, my lady,
I dare say; but I assure you it is much larger than
Sir William Lucas’s.”

“This must be a most inconvenient sitting room for
the evening, in summer; the windows are full west.”

Mrs. Bennet assured her that they never sat there after
dinner; and then added,

“May I take the liberty of asking your ladyship
whether you left Mr. and Mrs. Collins well.”

“Yes, very well. I saw them the night before last.”

Elizabeth now expected that she would produce a letter
for her from Charlotte, as it seemed the only probable
motive for her calling. But no letter appeared, and she
was completely puzzled.

Mrs. Bennet, with great civility, begged her ladyship
to take some refreshment; but Lady Catherine very
resolutely, and not very politely, declined eating any
thing; and then rising up, said to Elizabeth,

“Miss Bennet, there seemed to be a prettyish kind of
a little wilderness on one side of your lawn. I should be
glad to take a turn in it, if you will favour me with your
company.”

“Go, my dear,” cried her mother, “and shew her
ladyship about the different walks. I think she will be
pleased with the hermitage.”

Elizabeth obeyed, and running into her own room for
her parasol, attended her noble guest down stairs. As
they passed through the hall, Lady Catherine opened the
doors into the dining-parlour and drawing-room, and
%%352%%
pronouncing them, after a short survey, to be decent
looking rooms, walked on.

Her carriage remained at the door, and Elizabeth saw
that her waiting-woman was in it. They proceeded in
silence along the gravel walk that led to the copse; Elizabeth
was determined to make no effort for conversation
with a woman, who was now more than usually insolent
and disagreeable.

“How could I ever think her like her nephew?” said
she, as she looked in her face.

As soon as they entered the copse, Lady Catherine began
in the following manner:--

“You can be at no loss, Miss Bennet, to understand the
reason of my journey hither. Your own heart, your own
conscience, must tell you why I come.”

Elizabeth looked with unaffected astonishment.

“Indeed, you are mistaken, Madam. I have not been
at all able to account for the honour of seeing you here.”

“Miss Bennet,” replied her ladyship, in an angry tone,
“you ought to know, that I am not to be trifled with.
But however insincere \textit{you} may choose to be, you shall
not find \textit{me} so. My character has ever been celebrated
for its sincerity and frankness, and in a cause of such
moment as this, I shall certainly not depart from it.
A report of a most alarming nature, reached me two days
ago. I was told, that not only your sister was on the
point of being most advantageously married, but that \textit{you},
that Miss Elizabeth Bennet, would, in all likelihood, be
soon afterwards united to my nephew, my own nephew,
Mr. Darcy. Though I \textit{know} it must be a scandalous
falsehood; though I would not injure him so much as
to suppose the truth of it possible, I instantly resolved
on setting off for this place, that I might make my sentiments
known to you.”

“If you believed it impossible to be true,” said Elizabeth,
colouring with astonishment and disdain, “I wonder you
took the trouble of coming so far. What could your
ladyship propose by it?”
%%353%%

“At once to insist upon having such a report universally
contradicted.”

“Your coming to Longbourn, to see me and my family,”
said Elizabeth, coolly, “will be rather a confirmation of
it; if, indeed, such a report is in existence.”

“If! do you then pretend to be ignorant of it? Has
it not been industriously circulated by yourselves? Do
you not know that such a report is spread abroad?”

“I never heard that it was.”

“And can you likewise declare, that there is no \textit{foundation}
for it?”

“I do not pretend to possess equal frankness with your
ladyship. \textit{You} may ask questions, which \textit{I} shall not choose
to answer.”

“This is not to be borne. Miss Bennet, I insist on being
satisfied. Has he, has my nephew, made you an offer of
marriage?”

“Your ladyship has declared it to be impossible.”

“It ought to be so; it must be so, while he retains the
use of his reason. But \textit{your} arts and allurements may,
in a moment of infatuation, have made him forget what
he owes to himself and to all his family. You may have
drawn him in.”

“If I have, I shall be the last person to confess it.”

“Miss Bennet, do you know who I am? I have not
been accustomed to such language as this. I am almost
the nearest relation he has in the world, and am entitled
to know all his dearest concerns.”

“But you are not entitled to know \textit{mine}; nor will
such behaviour as this, ever induce me to be explicit.”

“Let me be rightly understood. This match, to which
you have the presumption to aspire, can never take place.
No, never. Mr. Darcy is engaged to \textit{my daughter}. Now
what have you to say?”

“Only this; that if he is so, you can have no reason
to suppose he will make an offer to me.”

Lady Catherine hesitated for a moment, and then
replied,
%%354%%

“The engagement between them is of a peculiar kind.
From their infancy, they have been intended for each
other. It was the favourite wish of \textit{his} mother, as well
as of her’s. While in their cradles, we planned the union:
and now, at the moment when the wishes of both sisters
would be accomplished, in their marriage, to be prevented
by a young woman of inferior birth, of no importance in
the world, and wholly unallied to the family! Do you
pay no regard to the wishes of his friends? To his tacit
engagement with Miss De Bourgh? Are you lost to every
feeling of propriety and delicacy? Have you not heard
me say, that from his earliest hours he was destined for
his cousin?”

“Yes, and I had heard it before. But what is that
to me? If there is no other objection to my marrying your
nephew, I shall certainly not be kept from it, by knowing
that his mother and aunt wished him to marry Miss
De Bourgh. You both did as much as you could, in planning
the marriage. Its completion depended on others.
If Mr. Darcy is neither by honour nor inclination confined
to his cousin, why is not he to make another choice? And
if I am that choice, why may not I accept him?”

“Because honour, decorum, prudence, nay, interest,
forbid it. Yes, Miss Bennet, interest; for do not expect
to be noticed by his family or friends, if you wilfully act
against the inclinations of all. You will be censured,
slighted, and despised, by every one connected with him.
Your alliance will be a disgrace; your name will never
even be mentioned by any of us.”

“These are heavy misfortunes,” replied Elizabeth.
“But the wife of Mr. Darcy must have such extraordinary
sources of happiness necessarily attached to her situation,
that she could, upon the whole, have no cause to repine.”

“Obstinate, headstrong girl! I am ashamed of you!
Is this your gratitude for my attentions to you last
spring? Is nothing due to me on that score?

“Let us sit down. You are to understand, Miss Bennet,
that I came here with the determined resolution of
%%355%%
carrying my purpose; nor will I be dissuaded from it.
I have not been used to submit to any person’s whims.
I have not been in the habit of brooking disappointment.”

“\textit{That} will make your ladyship’s situation at present
more pitiable; but it will have no effect on \textit{me}.”

“I will not be interrupted. Hear me in silence. My
daughter and my nephew are formed for each other. They
are descended on the maternal side, from the same noble
line; and, on the father’s, from respectable, honourable,
and ancient, though untitled families. Their fortune on
both sides is splendid. They are destined for each other
by the voice of every member of their respective houses;
and what is to divide them? The upstart pretensions of
a young woman without family, connections, or fortune.
Is this to be endured! But it must not, shall not be. If
you were sensible of your own good, you would not wish
to quit the sphere, in which you have been brought up.”

“In marrying your nephew, I should not consider myself
as quitting that sphere. He is a gentleman; I am a gentleman’s
daughter; so far we are equal.”

“True. You \textit{are} a gentleman’s daughter. But who was
your mother? Who are your uncles and aunts? Do not
imagine me ignorant of their condition.”

“Whatever my connections may be,” said Elizabeth,
“if your nephew does not object to them, they can be
nothing to \textit{you}.”

“Tell me once for all, are you engaged to him?”

Though Elizabeth would not, for the mere purpose of
obliging Lady Catherine, have answered this question;
she could not but say, after a moment’s deliberation,

“I am not.”

Lady Catherine seemed pleased.

“And will you promise me, never to enter into such an
engagement?”

“I will make no promise of the kind.”

“Miss Bennet I am shocked and astonished. I expected
to find a more reasonable young woman. But do not
deceive yourself into a belief that I will ever recede.
%%356%%
I shall not go away, till you have given me the assurance
I require.”

“And I certainly \textit{never} shall give it. I am not to be
intimidated into anything so wholly unreasonable. Your
ladyship wants Mr. Darcy to marry your daughter; but
would my giving you the wished-for promise, make \textit{their}
marriage at all more probable? Supposing him to be
attached to me, would \textit{my} refusing to accept his hand,
make him wish to bestow it on his cousin? Allow me to
say, Lady Catherine, that the arguments with which you
have supported this extraordinary application, have been
as frivolous as the application was ill-judged. You have
widely mistaken my character, if you think I can be
worked on by such persuasions as these. How far your
nephew might approve of your interference in \textit{his} affairs,
I cannot tell; but you have certainly no right to concern
yourself in mine. I must beg, therefore, to be importuned
no farther on the subject.”

“Not so hasty, if you please. I have by no means
done. To all the objections I have already urged, I have
still another to add. I am no stranger to the particulars
of your youngest sister’s infamous elopement. I know
it all; that the young man’s marrying her, was a patched-up
business, at the expence of your father and uncles.
And is \textit{such} a girl to be my nephew’s sister? Is \textit{her} husband,
is the son of his late father’s steward, to be his brother?
Heaven and earth! -- of what are you thinking? Are the
shades of Pemberley to be thus polluted?”

“You can \textit{now} have nothing farther to say,” she resentfully
answered. “You have insulted me, in every possible
method. I must beg to return to the house.”

And she rose as she spoke. Lady Catherine rose also,
and they turned back. Her ladyship was highly
incensed.

“You have no regard, then, for the honour and credit
of my nephew! Unfeeling, selfish girl! Do you not consider
that a connection with you, must disgrace him in
the eyes of everybody?”
%%357%%

“Lady Catherine, I have nothing farther to say. You
know my sentiments.”

“You are then resolved to have him?”

“I have said no such thing. I am only resolved to act
in that manner, which will, in my own opinion, constitute
my happiness, without reference to \textit{you}, or to any person
so wholly unconnected with me.”

“It is well. You refuse, then, to oblige me. You refuse
to obey the claims of duty, honour, and gratitude. You
are determined to ruin him in the opinion of all his friends,
and make him the contempt of the world.”

“Neither duty, nor honour, nor gratitude,” replied
Elizabeth, “have any possible claim on me, in the present
instance. No principle of either, would be violated by
my marriage with Mr. Darcy. And with regard to the
resentment of his family, or the indignation of the world,
if the former \textit{were} excited by his marrying me, it would
not give me one moment’s concern -- and the world in
general would have too much sense to join in the scorn.”

“And this is your real opinion! This is your final
resolve! Very well. I shall now know how to act. Do
not imagine, Miss Bennet, that your ambition will ever
be gratified. I came to try you. I hoped to find you
reasonable; but depend upon it I will carry my point.”

In this manner Lady Catherine talked on, till they were
at the door of the carriage, when turning hastily round,
she added,

“I take no leave of you, Miss Bennet. I send no compliments
to your mother. You deserve no such attention.
I am most seriously displeased.”

Elizabeth made no answer; and without attempting
to persuade her ladyship to return into the house, walked
quietly into it herself. She heard the carriage drive away
as she proceeded up stairs. Her mother impatiently met
her at the door of the dressing-room, to ask why Lady
Catherine would not come in again and rest herself.

“She did not choose it,” said her daughter, “she
would go.”
%%358%%

“She is a very fine-looking woman! and her calling
here was prodigiously civil! for she only came, I suppose,
to tell us the Collinses were well. She is on her road
somewhere, I dare say, and so passing through Meryton,
thought she might as well call on you. I suppose she had
nothing particular to say to you, Lizzy?”

Elizabeth was forced to give into a little falsehood
here; for to acknowledge the substance of their conversation
was impossible.
%%359%%

\Chapter{CHAPTER XV.}

The discomposure of spirits, which this extraordinary
visit threw Elizabeth into, could not be easily overcome;
nor could she for many hours, learn to think of it less
than incessantly. Lady Catherine it appeared, had
actually taken the trouble of this journey from Rosings,
for the sole purpose of breaking off her supposed engagement
with Mr. Darcy. It was a rational scheme to be
sure! but from what the report of their engagement could
originate, Elizabeth was at a loss to imagine; till she
recollected that \textit{his} being the intimate friend of Bingley,
and \textit{her} being the sister of Jane, was enough, at a time
when the expectation of one wedding, made every body
eager for another, to supply the idea. She had not herself
forgotten to feel that the marriage of her sister must bring
them more frequently together. And her neighbours at
Lucas lodge, therefore, (for through their communication
with the Collinses, the report she concluded had reached
lady Catherine) had only set \textit{that} down, as almost certain
and immediate, which \textit{she} had looked forward to as possible,
at some future time.

In revolving lady Catherine’s expressions, however, she
could not help feeling some uneasiness as to the possible
consequence of her persisting in this interference. From
what she had said of her resolution to prevent their
marriage, it occurred to Elizabeth that she must meditate
an application to her nephew; and how \textit{he} might take
a similar representation of the evils attached to a connection
with her, she dared not pronounce. She knew not
the exact degree of his affection for his aunt, or his dependence
on her judgment, but it was natural to suppose
that he thought much higher of her ladyship than \textit{she}
could do; and it was certain, that in enumerating the
%%360%%
miseries of a marriage with \textit{one}, whose immediate connections
were so unequal to his own, his aunt would address
him on his weakest side. With his notions of dignity, he
would probably feel that the arguments, which to Elizabeth
had appeared weak and ridiculous, contained much
good sense and solid reasoning.

If he had been wavering before, as to what he should
do, which had often seemed likely, the advice and intreaty
of so near a relation might settle every doubt, and determine
him at once to be as happy, as dignity unblemished
could make him. In that case he would return no more.
Lady Catherine might see him in her way through town;
and his engagement to Bingley of coming again to Netherfield
must give way.

“If, therefore, an excuse for not keeping his promise,
should come to his friend within a few days,” she added,
“I shall know how to understand it. I shall then give
over every expectation, every wish of his constancy. If
he is satisfied with only regretting me, when he might
have obtained my affections and hand, I shall soon cease
to regret him at all.”

\strut

The surprise of the rest of the family, on hearing who
their visitor had been, was very great; but they obligingly
satisfied it, with the same kind of supposition, which had
appeased Mrs. Bennet’s curiosity; and Elizabeth was
spared from much teazing on the subject.

The next morning, as she was going down stairs, she
was met by her father, who came out of his library with
a letter in his hand.

“Lizzy,” said he, “I was going to look for you; come
into my room.”

She followed him thither; and her curiosity to know
what he had to tell her, was heightened by the supposition
of its being in some manner connected with the letter he
held. It suddenly struck her that it might be from lady
Catherine; and she anticipated with dismay all the
consequent explanations.
%%361%%

She followed her father to the fire place, and they both
sat down. He then said,

“I have received a letter this morning that has astonished
me exceedingly. As it principally concerns yourself,
you ought to know its contents. I did not know before,
that I had \textit{two} daughters on the brink of matrimony. Let
me congratulate you, on a very important conquest.”

The colour now rushed into Elizabeth’s cheeks in the
instantaneous conviction of its being a letter from the
nephew, instead of the aunt; and she was undetermined
whether most to be pleased that he explained himself at
all, or offended that his letter was not rather addressed to
herself; when her father continued,

“You look conscious. Young ladies have great penetration
in such matters as these; but I think I may defy
even \textit{your} sagacity, to discover the name of your admirer.
This letter is from Mr. Collins.”

“From Mr. Collins! and what can \textit{he} have to say?”

“Something very much to the purpose of course. He
begins with congratulations on the approaching nuptials
of my eldest daughter, of which it seems he has been told,
by some of the good-natured, gossiping Lucases. I shall
not sport with your impatience, by reading what he says
on that point. What relates to yourself, is as follows.
“Having thus offered you the sincere congratulations of
Mrs. Collins and myself on this happy event, let me now
add a short hint on the subject of another: of which
we have been advertised by the same authority. Your
daughter Elizabeth, it is presumed, will not long bear the
name of Bennet, after her elder sister has resigned it,
and the chosen partner of her fate, may be reasonably
looked up to, as one of the most illustrious personages
in this land.”

“Can you possibly guess, Lizzy, who is meant by
this?” “This young gentleman is blessed in a peculiar
way, with every thing the heart of mortal can most
desire, -- splendid property, noble kindred, and extensive
patronage. Yet in spite of all these temptations, let me
%%362%%
warn my cousin Elizabeth, and yourself, of what evils you
may incur, by a precipitate closure with this gentleman’s
proposals, which, of course, you will be inclined to take
immediate advantage of.”

“Have you any idea, Lizzy, who this gentleman is?
But now it comes out.”

“My motive for cautioning you, is as follows. We have
reason to imagine that his aunt, lady Catherine de Bourgh,
does not look on the match with a friendly eye.”

“\textit{Mr. Darcy}, you see, is the man! Now, Lizzy, I think
I \textit{have} surprised you. Could he, or the Lucases, have
pitched on any man, within the circle of our acquaintance,
whose name would have given the lie more effectually to
what they related? Mr. Darcy, who never looks at any
woman but to see a blemish, and who probably never
looked at \textit{you} in his life! It is admirable!”

Elizabeth tried to join in her father’s pleasantry, but
could only force one most reluctant smile. Never had his
wit been directed in a manner so little agreeable to her.

“Are you not diverted?”

“Oh! yes. Pray read on.”

“After mentioning the likelihood of this marriage to
her ladyship last night, she immediately, with her usual
condescension, expressed what she felt on the occasion;
when it became apparent, that on the score of some family
objections on the part of my cousin, she would never give
her consent to what she termed so disgraceful a match.
I thought it my duty to give the speediest intelligence of
this to my cousin, that she and her noble admirer may be
aware of what they are about, and not run hastily into
a marriage which has not been properly sanctioned.”
“Mr. Collins moreover adds,” “I am truly rejoiced that
my cousin Lydia’s sad business has been so well hushed
up, and am only concerned that their living together
before the marriage took place, should be so generally
known. I must not, however, neglect the duties of my
station, or refrain from declaring my amazement, at
hearing that you received the young couple into your
%%363%%
house as soon as they were married. It was an encouragement
of vice; and had I been the rector of Longbourn,
I should very strenuously have opposed it. You ought
certainly to forgive them as a christian, but never to
admit them in your sight, or allow their names to be
mentioned in your hearing.” “\textit{That} is his notion of
christian forgiveness! The rest of his letter is only about
his dear Charlotte’s situation, and his expectation of a
young olive-branch. But, Lizzy, you look as if you did
not enjoy it. You are not going to be \textit{Missish}, I hope,
and pretend to be affronted at an idle report. For what
do we live, but to make sport for our neighbours, and
laugh at them in our turn?”

“Oh!” cried Elizabeth, “I am excessively diverted.
But it is so strange!”

“Yes -- \textit{that} is what makes it amusing. Had they fixed
on any other man it would have been nothing; but \textit{his}
perfect indifference, and \textit{your} pointed dislike, make it so
delightfully absurd! Much as I abominate writing, I would
not give up Mr. Collins’s correspondence for any consideration.
Nay, when I read a letter of his, I cannot
help giving him the preference even over Wickham, much
as I value the impudence and hypocrisy of my son-in-law.
And pray, Lizzy, what said Lady Catherine about this
report? Did she call to refuse her consent?”

To this question his daughter replied only with a laugh;
and as it had been asked without the least suspicion, she
was not distressed by his repeating it. Elizabeth had
never been more at a loss to make her feelings appear
what they were not. It was necessary to laugh, when
she would rather have cried. Her father had most cruelly
mortified her, by what he said of Mr. Darcy’s indifference,
and she could do nothing but wonder at such a want of
penetration, or fear that perhaps, instead of his seeing
too \textit{little}, she might have fancied too \textit{much}.
%%364%%

\Chapter{CHAPTER XVI.}

Instead of receiving any such letter of excuse from his
friend, as Elizabeth half expected Mr. Bingley to do, he
was able to bring Darcy with him to Longbourn before
many days had passed after Lady Catherine’s visit. The
gentlemen arrived early; and, before Mrs. Bennet had
time to tell him of their having seen his aunt, of which
her daughter sat in momentary dread, Bingley, who
wanted to be alone with Jane, proposed their all walking
out. It was agreed to. Mrs. Bennet was not in the habit
of walking, Mary could never spare time, but the remaining
five set off together. Bingley and Jane, however, soon
allowed the others to outstrip them. They lagged behind,
while Elizabeth, Kitty, and Darcy, were to entertain each
other. Very little was said by either; Kitty was too
much afraid of him to talk; Elizabeth was secretly
forming a desperate resolution; and perhaps he might be
doing the same.

They walked towards the Lucases, because Kitty wished
to call upon Maria; and as Elizabeth saw no occasion
for making it a general concern, when Kitty left them, she
went boldly on with him alone. Now was the moment
for her resolution to be executed, and, while her courage
was high, she immediately said,

“Mr. Darcy, I am a very selfish creature; and, for the
sake of giving relief to my own feelings, care not how
much I may be wounding your’s. I can no longer help
thanking you for your unexampled kindness to my poor
sister. Ever since I have known it, I have been most
anxious to acknowledge to you how gratefully I feel it.
Were it known to the rest of my family, I should not have
merely my own gratitude to express.”

“I am sorry, exceedingly sorry,” replied Darcy, in
a tone of surprise and emotion, “that you have ever
been informed of what may, in a mistaken light, have
%%365%%
given you uneasiness. I did not think Mrs. Gardiner was
so little to be trusted.”

“You must not blame my aunt. Lydia’s thoughtlessness
first betrayed to me that you had been concerned
in the matter; and, of course, I could not rest till I knew
the particulars. Let me thank you again and again, in
the name of all my family, for that generous compassion
which induced you to take so much trouble, and bear so
many mortifications, for the sake of discovering them.”

“If you \textit{will} thank me,” he replied, “let it be for yourself
alone. That the wish of giving happiness to you, might
add force to the other inducements which led me on, I shall
not attempt to deny. But your \textit{family} owe me nothing.
Much as I respect them, I believe, I thought only of \textit{you}.”

Elizabeth was too much embarrassed to say a word.
After a short pause, her companion added, “You are too
generous to trifle with me. If your feelings are still what
they were last April, tell me so at once. \textit{My} affections
and wishes are unchanged, but one word from you will
silence me on this subject for ever.”

Elizabeth feeling all the more than common awkwardness
and anxiety of his situation, now forced herself to
speak; and immediately, though not very fluently, gave
him to understand, that her sentiments had undergone so
material a change, since the period to which he alluded,
as to make her receive with gratitude and pleasure, his
present assurances. The happiness which this reply
produced, was such as he had probably never felt before;
and he expressed himself on the occasion as sensibly and
as warmly as a man violently in love can be supposed to
do. Had Elizabeth been able to encounter his eye, she
might have seen how well the expression of heart-felt
delight, diffused over his face, became him; but, though
she could not look, she could listen, and he told her of
feelings, which, in proving of what importance she was
to him, made his affection every moment more valuable.

They walked on, without knowing in what direction.
There was too much to be thought; and felt, and said,
%%366%%
for attention to any other objects. She soon learnt that
they were indebted for their present good understanding
to the efforts of his aunt, who \textit{did} call on him in her
return through London, and there relate her journey to
Longbourn, its motive, and the substance of her conversation
with Elizabeth; dwelling emphatically on every
expression of the latter, which, in her ladyship’s apprehension,
peculiarly denoted her perverseness and assurance,
in the belief that such a relation must assist her
endeavours to obtain that promise from her nephew,
which \textit{she} had refused to give. But, unluckily for her
ladyship, its effect had been exactly contrariwise.

“It taught me to hope,” said he, “as I had scarcely
ever allowed myself to hope before. I knew enough of
your disposition to be certain, that, had you been absolutely,
irrevocably decided against me, you would have
acknowledged it to Lady Catherine, frankly and openly.”

Elizabeth coloured and laughed as she replied, “Yes,
you know enough of my \textit{frankness} to believe me capable
of \textit{that}. After abusing you so abominably to your face,
I could have no scruple in abusing you to all your relations.”

“What did you say of me, that I did not deserve?
For, though your accusations were ill-founded, formed
on mistaken premises, my behaviour to you at the time,
had merited the severest reproof. It was unpardonable.
I cannot think of it without abhorrence.”

“We will not quarrel for the greater share of blame
annexed to that evening,” said Elizabeth. “The conduct
of neither, if strictly examined, will be irreproachable;
but since then, we have both, I hope, improved in civility.”

“I cannot be so easily reconciled to myself. The
recollection of what I then said, of my conduct, my
manners, my expressions during the whole of it, is now,
and has been many months, inexpressibly painful to me.
Your reproof, so well applied, I shall never forget: ‘had
you behaved in a more gentleman-like manner.’ Those
were your words. You know not, you can scarcely conceive,
how they have tortured me; -- though it was some
%%367%%
time, I confess, before I was reasonable enough to allow
their justice.”

“I was certainly very far from expecting them to make
so strong an impression. I had not the smallest idea of
their being ever felt in such a way.”

“I can easily believe it. You thought me then devoid
of every proper feeling, I am sure you did. The turn of
your countenance I shall never forget, as you said that
I could not have addressed you in any possible way, that
would induce you to accept me.”

“Oh! do not repeat what I then said. These recollections
will not do at all. I assure you, that I have long
been most heartily ashamed of it.”

Darcy mentioned his letter. “Did it,” said he, “did
it \textit{soon} make you think better of me? Did you, on reading
it, give any credit to its contents?”

She explained what its effect on her had been, and how
gradually all her former prejudices had been removed.

“I knew,” said he, “that what I wrote must give you
pain, but it was necessary. I hope you have destroyed
the letter. There was one part especially, the opening
of it, which I should dread your having the power of
reading again. I can remember some expressions which
might justly make you hate me.”

“The letter shall certainly be burnt, if you believe it
essential to the preservation of my regard; but, though
we have both reason to think my opinions not entirely
unalterable, they are not, I hope, quite so easily changed
as that implies.”

“When I wrote that letter,” replied Darcy, “I believed
myself perfectly calm and cool, but I am since convinced
that it was written in a dreadful bitterness of spirit.”

“The letter, perhaps, began in bitterness, but it did
not end so. The adieu is charity itself. But think no
more of the letter. The feelings of the person who wrote,
and the person who received it, are now so widely different
from what they were then, that every unpleasant circumstance
attending it, ought to be forgotten. You must
%%368%%
learn some of my philosophy. Think only of the past as
its remembrance gives you pleasure.”

“I cannot give you credit for any philosophy of the
kind. \textit{Your} retrospections must be so totally void of
reproach, that the contentment arising from them, is not
of philosophy, but what is much better, of ignorance.
But with \textit{me}, it is not so. Painful recollections will intrude,
which cannot, which ought not to be repelled. I have
been a selfish being all my life, in practice, though not in
principle. As a child I was taught what was \textit{right}, but
I was not taught to correct my temper. I was given good
principles, but left to follow them in pride and conceit.
Unfortunately an only son, (for many years an only \textit{child})
I was spoilt by my parents, who though good themselves,
(my father particularly, all that was benevolent and
amiable,) allowed, encouraged, almost taught me to be
selfish and overbearing, to care for none beyond my own
family circle, to think meanly of all the rest of the world,
to \textit{wish} at least to think meanly of their sense and worth
compared with my own. Such I was, from eight to eight
and twenty; and such I might still have been but for
you, dearest, loveliest Elizabeth! What do I not owe you!
You taught me a lesson, hard indeed at first, but most
advantageous. By you, I was properly humbled. I came
to you without a doubt of my reception. You shewed
me how insufficient were all my pretensions to please
a woman worthy of being pleased.”

“Had you then persuaded yourself that I sh\-ould?”

“Indeed I had. What will you think of my vanity?
I believed you to be wishing, expecting my addresses.”

“My manners must have been in fault, but not intentionally
I assure you. I never meant to deceive you, but
my spirits might often lead me wrong. How you must
have hated me after \textit{that} evening?”

“Hate you! I was angry perhaps at first, but my
anger soon began to take a proper direction.”

“I am almost afraid of asking what you thought of me;
when we met at Pemberley. You blamed me for coming?”
%%369%%

“No indeed; I felt nothing but surprise.”

“Your surprise could not be greater than \textit{mine} in being
noticed by you. My conscience told me that I deserved
no extraordinary politeness, and I confess that I did not
expect to receive \textit{more} than my due.”

“My object \textit{then},” replied Darcy, “was to shew you,
by every civility in my power, that I was not so mean as
to resent the past; and I hoped to obtain your forgiveness,
to lessen your ill opinion, by letting you see
that your reproofs had been attended to. How soon
any other wishes introduced themselves I can hardly tell,
but I believe in about half an hour after I had seen you.”

He then told her of Georgiana’s delight in her acquaintance,
and of her disappointment at its sudden interruption;
which naturally leading to the cause of that
interruption, she soon learnt that his resolution of following
her from Derbyshire in quest of her sister, had been
formed before he quitted the inn, and that his gravity
and thoughtfulness there, had arisen from no other
struggles than what such a purpose must comprehend.

She expressed her gratitude again, but it was too
painful a subject to each, to be dwelt on farther.

After walking several miles in a leisurely manner, and
too busy to know any thing about it, they found at last, on
examining their watches, that it was time to be at home.

“What could become of Mr. Bingley and Jane!”
was a wonder which introduced the discussion of \textit{their}
affairs. Darcy was delighted with their engagement; his
friend had given him the earliest information of it.

“I must ask whether you were surprised?” said
Elizabeth.

“Not at all. When I went away, I felt that it would
soon happen.”

“That is to say, you had given your permission. I
guessed as much.” And though he exclaimed at the term,
she found that it had been pretty much the case.

“On the evening before my going to London,” said he
“I made a confession to him, which I believe I ought to
%%370%%
have made long ago. I told him of all that had occurred
to make my former interference in his affairs, absurd
and impertinent. His surprise was great. He had never
had the slightest suspicion. I told him, moreover, that
I believed myself mistaken in supposing, as I had done,
that your sister was indifferent to him; and as I could
easily perceive that his attachment to her was unabated,
I felt no doubt of their happiness together.”

Elizabeth could not help smiling at his easy manner of
directing his friend.

“Did you speak from your own observation,” said she,
“when you told him that my sister loved him, or merely
from my information last spring?”

“From the former. I had narrowly observed her during
the two visits which I had lately made her here; and
I was convinced of her affection.”

“And your assurance of it, I suppose, carried immediate
conviction to him.”

“It did. Bingley is most unaffectedly modest. His
diffidence had prevented his depending on his own judgment
in so anxious a case, but his reliance on mine, made
every thing easy. I was obliged to confess one thing,
which for a time, and not unjustly, offended him. I could
not allow myself to conceal that your sister had been in
town three months last winter, that I had known it, and
purposely kept it from him. He was angry. But his
anger, I am persuaded, lasted no longer than he remained
in any doubt of your sister’s sentiments. He has heartily
forgiven me now.”

Elizabeth longed to observe that Mr. Bingley had been
a most delightful friend; so easily guided that his worth
was invaluable; but she checked herself. She remembered
that he had yet to learn to be laught at, and it
was rather too early to begin. In anticipating the happiness
of Bingley, which of course was to be inferior only
to his own, he continued the conversation till they reached
the house. In the hall they parted.
%%371%%

\Chapter{CHAPTER XVII.}

“My dear Lizzy, where can you have been walking
to?” was a question which Elizabeth received from Jane
as soon as she entered the room, and from all the others
when they sat down to table. She had only to say in
reply, that they had wandered about, till she was beyond
her own knowledge. She coloured as she spoke; but
neither that, nor any thing else, awakened a suspicion of
the truth.

The evening passed quietly, unmarked by any thing
extraordinary. The acknowledged lovers tal\-ked and
laughed, the unacknowledged were silent. Darcy was not
of a disposition in which happiness overflows in mirth;
and Elizabeth, agitated and confused, rather \textit{knew} that
she was happy, than \textit{felt} herself to be so; for, besides the
immediate embarrassment, there were other evils before
her. She anticipated what would be felt in the family when
her situation became known; she was aware that no one
liked him but Jane; and even feared that with the others
it was a \textit{dislike} which not all his fortune and consequence
might do away.

At night she opened her heart to Jane. Though suspicion
was very far from Miss Bennet’s general habits, she
was absolutely incredulous here.

“You are joking, Lizzy. This cannot be! -- engaged to
Mr. Darcy! No, no, you shall not deceive me. I know
it to be impossible.”

“This is a wretched beginning indeed! My sole dependence
was on you; and I am sure nobody else will believe
me, if you do not. Yet, indeed, I am in earnest. I speak
nothing but the truth. He still loves me, and we are
engaged.”

Jane looked at her doubtingly. “Oh, Lizzy! it cannot
be. I know how much you dislike him.”
%%372%%

“You know nothing of the matter. \textit{That} is all to be
forgot. Perhaps I did not always love him so well as
I do now. But in such cases as these, a good memory is
unpardonable. This is the last time I shall ever remember
it myself.”

Miss Bennet still looked all amazement. Elizabeth
again, and more seriously assured her of its truth.

“Good Heaven! can it be really so! Yet now I must
believe you,” cried Jane. “My dear, dear Lizzy, I would -- I
do congratulate you -- but are you certain? forgive the
question -- are you quite certain that you can be happy
with him?”

“There can be no doubt of that. It is settled between
us already, that we are to be the happiest couple in the
world. But are you pleased, Jane? Shall you like to have
such a brother?”

“Very, very much. Nothing could give either Bingley
or myself more delight. But we considered it, we talked
of it as impossible. And do you really love him quite well
enough? Oh, Lizzy! do any thing rather than marry
without affection. Are you quite sure that you feel what
you ought to do?”

“Oh, yes! You will only think I feel \textit{more} than I ought
to do, when I tell you all.”

“What do you mean?”

“Why, I must confess, that I love him better than I do
Bingley. I am afraid you will be angry.”

“My dearest sister, now \textit{be} serious. I want to talk
very seriously. Let me know every thing that I am to
know, without delay. Will you tell me how long you have
loved him?”

“It has been coming on so gradually, that I hardly
know when it began. But I believe I must date it from
my first seeing his beautiful grounds at Pemberley.”

Another intreaty that she would be serious, however,
produced the desired effect; and she soon satisfied Jane
by her solemn assurances of attachment. When convinced
on that article, Miss Bennet had nothing farther to wish.
%%373%%

“Now I am quite happy,” said she, “for you will be as
happy as myself. I always had a value for him. Were
it for nothing but his love of you, I must always have
esteemed him; but now, as Bingley’s friend and your
husband, there can be only Bingley and yourself more
dear to me. But Lizzy, you have been very sly, very
reserved with me. How little did you tell me of what
passed at Pemberley and Lambton! I owe all that
I know of it, to another, not to you.”

Elizabeth told her the motives of her secrecy. She had
been unwilling to mention Bingley; and the unsettled
state of her own feelings had made her equally avoid the
name of his friend. But now she would no longer conceal
from her, his share in Lydia’s marriage. All was acknowledged,
and half the night spent in conversation.

\strut

“Good gracious!” cried Mrs. Bennet, as she stood at
a window the next morning, “if that disagreeable Mr.
Darcy is not coming here again with our dear Bingley!
What can he mean by being so tiresome as to be always
coming here? I had no notion but he would go a shooting,
or something or other, and not disturb us with his company.
What shall we do with him? Lizzy, you must
walk out with him again, that he may not be in Bingley’s
way.”

Elizabeth could hardly help laughing at so convenient
a proposal; yet was really vexed that her mother should
be always giving him such an epithet.

As soon as they entered, Bingley looked at her so
expressively, and shook hands with such warmth, as left
no doubt of his good information; and he soon afterwards
said aloud, “Mr. Bennet, have you no more lanes hereabouts
in which Lizzy may lose her way again to-day?”

“I advise Mr. Darcy, and Lizzy, and Kitty,” said
Mrs. Bennet, “to walk to Oakham Mount this morning.
It is a nice long walk, and Mr. Darcy has never seen the
view.”

“It may do very well for the others,” replied Mr.
%%374%%
Bingley; “but I am sure it will be too much for Kitty.
Wont it, Kitty?”

Kitty owned that she had rather stay at home. Darcy
professed a great curiosity to see the view from the Mount,
and Elizabeth silently consented. As she went up stairs
to get ready, Mrs. Bennet followed her, saying,

“I am quite sorry, Lizzy, that you should be forced
to have that disagreeable man all to yourself. But I hope
you will not mind it: it is all for Jane’s sake, you know;
and there is no occasion for talking to him, except just
now and then. So, do not put yourself to inconvenience.”

During their walk, it was resolved that Mr. Bennet’s
consent should be asked in the course of the evening.
Elizabeth reserved to herself the application for her
mother’s. She could not determine how her mother
would take it; sometimes doubting whether all his wealth
and grandeur would be enough to overcome her abhorrence
of the man. But whether she were violently set
against the match, or violently delighted with it, it was
certain that her manner would be equally ill adapted to
do credit to her sense; and she could no more bear that
Mr. Darcy should hear the first raptures of her joy, than
the first vehemence of her disapprobation.

\strut

In the evening, soon after Mr. Bennet withdrew to the
library, she saw Mr. Darcy rise also and follow him, and
her agitation on seeing it was extreme. She did not fear
her father’s opposition, but he was going to be made
unhappy, and that it should be through her means, that
\textit{she}, his favourite child, should be distressing him by her
choice, should be filling him with fears and regrets in
disposing of her, was a wretched reflection, and she sat
in misery till Mr. Darcy appeared again, when, looking
at him, she was a little relieved by his smile. In a few
minutes he approached the table where she was sitting
with Kitty; and, while pretending to admire her work,
said in a whisper, “Go to your father, he wants you in
the library.” She was gone directly.
%%375%%

Her father was walking about the room, looking grave
and anxious. “Lizzy,” said he, “what are you doing?
Are you out of your senses, to be accepting this man?
Have not you always hated him?”

How earnestly did she then wish that her former
opinions had been more reasonable, her expressions more
moderate! It would have spared her from explanations
and professions which it was exceedingly awkward to
give; but they were now necessary, and she assured him
with some confusion, of her attachment to Mr. Darcy.

“Or in other words, you are determined to have him.
He is rich, to be sure, and you may have more fine clothes
and fine carriages than Jane. But will they make you
happy?”

“Have you any other objection,” said Elizabeth, “than
your belief of my indifference?”

“None at all. We all know him to be a proud, unpleasant
sort of man; but this would be nothing if you
really liked him.”

“I do, I do like him,” she replied, with tears in her eyes,
“I love him. Indeed he has no improper pride. He is
perfectly amiable. You do not know what he really is;
then pray do not pain me by speaking of him in such
terms.”

“Lizzy,” said her father, “I have given him my consent.
He is the kind of man, indeed, to whom I should never
dare refuse any thing, which he condescended to ask.
I now give it to \textit{you}, if you are resolved on having him.
But let me advise you to think better of it. I know your
disposition, Lizzy. I know that you could be neither
happy nor respectable, unless you truly esteemed your
husband; unless you looked up to him as a superior.
Your lively talents would place you in the greatest danger
in an unequal marriage. You could scarcely escape discredit
and misery. My child, let me not have the grief
of seeing \textit{you} unable to respect your partner in life. You
know not what you are about.”

Elizabeth, still more affected, was earnest and solemn
%%376%%
in her reply; and at length, by repeated assurances that
Mr. Darcy was really the object of her choice, by explaining
the gradual change which her estimation of him had
undergone, relating her absolute certainty that his affection
was not the work of a day, but had stood the test
of many months suspense, and enumerating with energy
all his good qualities, she did conquer her father’s incredulity,
and reconcile him to the match.

“Well, my dear,” said he, when she ceased speaking,
“I have no more to say. If this be the case, he deserves
you. I could not have parted with you, my Lizzy, to
any one less worthy.”

To complete the favourable impression, she then told
him what Mr. Darcy had voluntarily done for Lydia.
He heard her with astonishment.

“This is an evening of wonders, indeed! And so,
Darcy did every thing; made up the match, gave the
money, paid the fellow’s debts, and got him his commission!
So much the better. It will save me a world
of trouble and economy. Had it been your uncle’s doing,
I must and \textit{would} have paid him; but these violent young
lovers carry every thing their own way. I shall offer to
pay him to-morrow; he will rant and storm about his
love for you, and there will be an end of the matter.”

He then recollected her embarrassment a few days
before, on his reading Mr. Collins’s letter; and after
laughing at her some time, allowed her at last to go -- saying,
as she quitted the room, “If any young men come
for Mary or Kitty, send them in, for I am quite at leisure.”

Elizabeth’s mind was now relieved from a very heavy
weight; and, after half an hour’s quiet reflection in her
own room, she was able to join the others with tolerable
composure. Every thing was too recent for gaiety, but
the evening passed tranquilly away; there was no longer
any thing material to be dreaded, and the comfort of
ease and familiarity would come in time.

When her mother went up to her dressing-room at
night, she followed her, and made the important
%%377%%
communication. Its effect was most extraordinary; for on
first hearing it, Mrs. Bennet sat quite still, and unable to
utter a syllable. Nor was it under many, many minutes,
that she could comprehend what she heard; though not
in general backward to credit what was for the advantage
of her family, or that came in the shape of a lover to any
of them. She began at length to recover, to fidget about
in her chair, get up, sit down again, wonder, and bless
herself.

“Good gracious! Lord bless me! only think! dear
me! Mr. Darcy! Who would have thought it! And
is it really true? Oh! my sweetest Lizzy! how rich and
how great you will be! What pin-money, what jewels,
what carriages you will have! Jane’s is nothing to it -- nothing
at all. I am so pleased -- so happy. Such a charming
man! -- so handsome! so tall! -- Oh, my dear Lizzy!
pray apologise for my having disliked him so much before.
I hope he will overlook it. Dear, dear Lizzy. A house
in town! Every thing that is charming! Three daughters
married! Ten thousand a year! Oh, Lord! What will
become of me. I shall go distracted.”

This was enough to prove that her approbation need
not be doubted: and Elizabeth, rejoicing that such an
effusion was heard only by herself, soon went away. But
before she had been three minutes in her own room, her
mother followed her.

“My dearest child,” she cried, “I can think of nothing
else! Ten thousand a year, and very likely more! ’Tis
as good as a Lord! And a special licence. You must and
shall be married by a special licence. But my dearest
love, tell me what dish Mr. Darcy is particularly fond of,
that I may have it to-morrow.”

This was a sad omen of what her mother’s behaviour
to the gentleman himself might be; and Elizabeth found,
that though in the certain possession of his warmest
affection, and secure of her relations’ consent, there was
still something to be wished for. But the morrow passed
off much better than she expected; for Mrs. Bennet
%%378%%
luckily stood in such awe of her intended son-in-law, that
she ventured not to speak to him, unless it was in her
power to offer him any attention, or mark her deference
for his opinion.

Elizabeth had the satisfaction of seeing her father
taking pains to get acquainted with him; and Mr. Bennet
soon assured her that he was rising every hour in his
esteem.

“I admire all my three sons-in-law highly,” said he.
“Wickham, perhaps, is my favourite; but I think I shall
like \textit{your} husband quite as well as Jane’s.”
%%379%%

\Chapter{CHAPTER XVIII.}

Elizabeth’s spirits soon rising to playfulness again, she
want\-ed Mr. Darcy to account for his having ever fallen
in love with her. “How could you begin?” said she.
“I can comprehend your going on charmingly, when you
had once made a beginning; but what could set you off
in the first place?”

“I cannot fix on the hour, or the spot, or the look, or
the words, which laid the foundation. It is too long ago.
I was in the middle before I knew that I \textit{had} begun.”

“My beauty you had early withstood, and as for my
manners -- my behaviour to \textit{you} was at least always
bordering on the uncivil, and I never spoke to you without
rather wishing to give you pain than not. Now be
sincere; did you admire me for my impertinence?”

“For the liveliness of your mind, I did.”

“You may as well call it impertinence at once. It was
very little less. The fact is, that you were sick of civility,
of deference, of officious attention. You were disgusted
with the women who were always speaking and looking,
and thinking for \textit{your} approbation alone. I roused, and
interested you, because I was so unlike \textit{them}. Had you
not been really amiable you would have hated me for it;
but in spite of the pains you took to disguise yourself,
your feelings were always noble and just; and in your
heart, you thoroughly despised the persons who so assiduously
courted you. There -- I have saved you the trouble
of accounting for it; and really, all things considered,
I begin to think it perfectly reasonable. To be sure, you
knew no actual good of me -- but nobody thinks of \textit{that}
when they fall in love.”

“Was there no good in your affectionate behaviour to
Jane, while she was ill at Netherfield?”
%%380%%

“Dearest Jane! who could have done less for her?
But make a virtue of it by all means. My good qualities
are under your protection, and you are to exaggerate them
as much as possible; and, in return, it belongs to me to
find occasions for teazing and quarrelling with you as often
as may be; and I shall begin directly by asking you what
made you so unwilling to come to the point at last. What
made you so shy of me, when you first called, and afterwards
dined here? Why, especially, when you called, did
you look as if you did not care about me?”

“Because you were grave and silent, and gave me no
encouragement.”

“But I was embarrassed.”

“And so was I.”

“You might have talked to me more when you came
to dinner.”

“A man who had felt less, might.”

“How unlucky that you should have a reasonable
answer to give, and that I should be so reasonable as to
admit it! But I wonder how long you \textit{would} have gone
on, if you had been left to yourself. I wonder when you
\textit{would} have spoken, if I had not asked you! My resolution
of thanking you for your kindness to Lydia had certainly
great effect. \textit{Too much}, I am afraid; for what becomes
of the moral, if our comfort springs from a breach of
promise, for I ought not to have mentioned the subject?
This will never do.”

“You need not distress yourself. The moral will be
perfectly fair. Lady Catherine’s unjustifiable endeavours
to separate us, were the means of removing all my doubts.
I am not indebted for my present happiness to your
eager desire of expressing your gratitude. I was not in
a humour to wait for any opening of your’s. My aunt’s
intelligence had given me hope, and I was determined at
once to know every thing.”

“Lady Catherine has been of infinite use, which ought
to make her happy, for she loves to be of use. But tell
me, what did you come down to Netherfield for? Was
%%381%%
it merely to ride to Longbourn and be embarrassed? or
had you intended any more serious consequence?”

“My real purpose was to see \textit{you}, and to judge, if I could,
whether I might ever hope to make you love me. My
avowed one, or what I avowed to myself, was to see
whether your sister were still partial to Bingley, and if she
were, to make the confession to him which I have since
made.”

“Shall you ever have courage to announce to Lady
Catherine, what is to befall her?”

“I am more likely to want time than courage, Elizabeth.
But it ought to be done, and if you will give me a sheet
of paper, it shall be done directly.”

“And if I had not a letter to write myself, I might sit
by you, and admire the evenness of your writing, as
another young lady once did. But I have an aunt, too,
who must not be longer neglected.”

From an unwillingness to confess how much her intimacy
with Mr. Darcy had been over-rated, Elizabeth had never
yet answered Mrs. Gardiner’s long letter, but now, having
\textit{that} to communicate which she knew would be most
welcome, she was almost ashamed to find, that her uncle
and aunt had already lost three days of happiness, and
immediately wrote as follows:

\begin{letter}
“I would have thanked you before, my dear aunt, as
I ought to have done, for your long, kind, satisfactory,
detail of particulars; but to say the truth, I was too
cross to write. You supposed more than really existed.
But \textit{now} suppose as much as you chuse; give a loose to
your fancy, indulge your imagination in every possible
flight which the subject will afford, and unless you believe
me actually married, you cannot greatly err. You must
write again very soon, and praise him a great deal more
than you did in your last. I thank you, again and again,
for not going to the Lakes. How could I be so silly as
to wish it! Your idea of the ponies is delightful. We
will go round the Park every day. I am the happiest
%%382%%
creature in the world. Perhaps other people have said
so before, but not one with such justice. I am happier
even than Jane; she only smiles, I laugh. Mr. Darcy
sends you all the love in the world, that he can spare from
me. You are all to come to Pemberley at Christmas.

\raggedleft Your’s, \&c.”
\end{letter}

Mr. Darcy’s letter to Lady Catherine, was in a different
style; and still different from either, was what Mr. Bennet
sent to Mr. Collins, in reply to his last.

\begin{letter}
“\textsc{Dear Sir},

“I must trouble you once more for congratulations.
Elizabeth will soon be the wife of Mr. Darcy. Console
Lady Catherine as well as you can. But, if I were you,
I would stand by the nephew. He has more to give.

\raggedleft “Your’s sincerely, \&c.”
\end{letter}

Miss Bingley’s congratulations to her brother, on his
approaching marriage, were all that was affectionate and
insincere. She wrote even to Jane on the occasion, to
express her delight, and repeat all her former professions
of regard. Jane was not deceived, but she was affected;
and though feeling no reliance on her, could not help
writing her a much kinder answer than she knew was
deserved.

The joy which Miss Darcy expressed on receiving similar
information, was as sincere as her brother’s in sending it.
Four sides of paper were insufficient to contain all her
delight, and all her earnest desire of being loved by her
sister.

Before any answer could arrive from Mr. Collins, or any
congratulations to Elizabeth, from his wife, the Longbourn
family heard that the Collinses were come themselves to
Lucas lodge. The reason of this sudden removal was soon
evident. Lady Catherine had been rendered so exceedingly
angry by the contents of her nephew’s letter, that
Charlotte, really rejoicing in the match, was anxious to
get away till the storm was blown over. At such a moment,
%%383%%
the arrival of her friend was a sincere pleasure to Elizabeth,
though in the course of their meetings she must sometimes
think the pleasure dearly bought, when she saw Mr. Darcy
exposed to all the parading and obsequious civility of her
husband. He bore it however with admirable calmness.
He could even listen to Sir William Lucas, when he complimented
him on carrying away the brightest jewel of
the country, and expressed his hopes of their all meeting
frequently at St. James’s, with very decent composure.
If he did shrug his shoulders, it was not till Sir William
was out of sight.

Mrs. Philips’s vulgarity was another, and perhaps a
great\-er tax on his forbearance; and though Mrs. Philips,
as well as her sister, stood in too much awe of him to
speak with the familiarity which Bingley’s good humour
encouraged, yet, whenever she \textit{did} speak, she must be
vulgar. Nor was her respect for him, though it made her
more quiet, at all likely to make her more elegant. Elizabeth
did all she could, to shield him from the frequent
notice of either, and was ever anxious to keep him to
herself, and to those of her family with whom he might
converse without mortification; and though the uncomfortable
feelings arising from all this took from the season
of courtship much of its pleasure, it added to the hope of the
future; and she looked forward with delight to the time
when they should be removed from society so little
pleasing to either, to all the comfort and elegance of their
family party at Pemberley.
%%384%%

\Chapter{CHAPTER XIX.}

Happy for all her maternal feelings was the day on
which Mrs. Bennet got rid of her two most deserving
daughters. With what delighted pride she afterwards
visited Mrs. Bingley and talked of Mrs. Darcy may be
guessed. I wish I could say, for the sake of her family,
that the accomplishment of her earnest desire in the
establishment of so many of her children, produced so
happy an effect as to make her a sensible, amiable, well-informed
woman for the rest of her life; though perhaps
it was lucky for her husband, who might not have relished
domestic felicity in so unusual a form, that she still was
occasionally nervous and invariably silly.

Mr. Bennet missed his second daughter exceedingly;
his affection for her drew him oftener from home than
any thing else could do. He delighted in going to Pemberley,
especially when he was least expected.

Mr. Bingley and Jane remained at Netherfield only
a twelvemonth. So near a vicinity to her mother and
Meryton relations was not desirable even to \textit{his} easy
temper, or \textit{her} affectionate heart. The darling wish of
his sisters was then gratified; he bought an estate in a
neighbouring county to Derbyshire, and Jane and Elizabeth,
in addition to every other source of happiness, were
within thirty miles of each other.

Kitty, to her very material advantage, spent the chief
of her time with her two elder sisters. In society so
superior to what she had generally known, her improvement
was great. She was not of so ungovernable a temper
as Lydia, and, removed from the influence of Lydia’s
example, she became, by proper attention and management,
less irritable, less ignorant, and less insipid. From
the farther disadvantage of Lydia’s society she was of
course carefully kept, and though Mrs. Wickham
%%385%%
frequently invited her to come and stay with her, with the
promise of balls and young men, her father would never
consent to her going.

Mary was the only daughter who remained at home;
and she was necessarily drawn from the pursuit of accomplishments
by Mrs. Bennet’s being quite unable to sit
alone. Mary was obliged to mix more with the world,
but she could still moralize over every morning visit;
and as she was no longer mortified by comparisons between
her sisters’ beauty and her own, it was suspected by her
father that she submitted to the change without much
reluctance.

As for Wickham and Lydia, their characters suffered
no revolution from the marriage of her sisters. He bore
with philosophy the conviction that Elizabeth must now
become acquainted with whatever of his ingratitude and
falsehood had before been unknown to her; and in spite
of every thing, was not wholly without hope that Darcy
might yet be prevailed on to make his fortune. The congratulatory
letter which Elizabeth received from Lydia on
her marriage, explained to her that, by his wife at least,
if not by himself, such a hope was cherished. The letter
was to this effect:

\begin{letter}
“\textsc{My dear Lizzy},

“I wish you joy. If you love Mr. Darcy half as well
as I do my dear Wickham, you must be very happy.
It is a great comfort to have you so rich, and when you
have nothing else to do, I hope you will think of us. I am
sure Wickham would like a place at court very much, and
I do not think we shall have quite money enough to live
upon without some help. Any place would do, of about
three or four hundred a year; but, however, do not
speak to Mr. Darcy about it, if you had rather not.

\raggedleft “Yours, \&c.”
\end{letter}

As it happened that Elizabeth had \textit{much} rather not,
she endeavoured in her answer to put an end to every
%%386%%
intreaty and expectation of the kind. Such relief, however,
as it was in her power to afford, by the practice of
what might be called economy in her own private expences,
she frequently sent them. It had always been evident
to her that such an income as theirs, under the direction
of two persons so extravagant in their wants, and heedless
of the future, must be very insufficient to their support;
and whenever they changed their quarters, either Jane
or herself were sure of being applied to, for some little
assistance towards discharging their bills. Their manner
of living, even when the restoration of peace dismissed
them to a home, was unsettled in the extreme. They were
always moving from place to place in quest of a cheap
situation, and always spending more than they ought.
His affection for her soon sunk into indifference; her’s
lasted a little longer; and in spite of her youth and her
manners, she retained all the claims to reputation which
her marriage had given her.

Though Darcy could never receive \textit{him} at Pemberley,
yet, for Elizabeth’s sake, he assisted him farther in his
profession. Lydia was occasionally a visitor there, when
her husband was gone to enjoy himself in London or
Bath; and with the Bingleys they both of them frequently
staid so long, that even Bingley’s good humour was overcome,
and he proceeded so far as to \textit{talk} of giving them
a hint to be gone.

Miss Bingley was very deeply mortified by Darcy’s
marriage; but as she thought it advisable to retain the
right of visiting at Pemberley, she dropt all her resentment;
was fonder than ever of Georgiana, almost as
attentive to Darcy as heretofore, and paid off every arrear
of civility to Elizabeth.

Pemberley was now Georgiana’s home; and the attachment
of the sisters was exactly what Darcy had hoped
to see. They were able to love each other, even as well
as they intended. Georgiana had the highest opinion in
the world of Elizabeth; though at first she often listened
with an astonishment bordering on alarm, at her lively,
%%387%%
sportive, manner of talking to her brother. He, who had
always inspired in herself a respect which almost overcame
her affection, she now saw the object of open pleasantry.
Her mind received knowledge which had never before
fallen in her way. By Elizabeth’s instructions she began
to comprehend that a woman may take liberties with her
husband, which a brother will not always allow in a sister
more than ten years younger than himself.

Lady Catherine was extremely indignant on the marriage
of her nephew; and as she gave way to all the genuine
frankness of her character, in her reply to the letter which
announced its arrangement, she sent him language so very
abusive, especially of Elizabeth, that for some time all
intercourse was at an end. But at length, by Elizabeth’s
persuasion, he was prevailed on to overlook the offence,
and seek a reconciliation; and, after a little farther
resistance on the part of his aunt, her resentment gave
way, either to her affection for him, or her curiosity to see
how his wife conducted herself; and she condescended
to wait on them at Pemberley, in spite of that pollution
which its woods had received, not merely from the presence
of such a mistress, but the visits of her uncle and aunt
from the city.

With the Gardiners, they were always on the most
intimate terms. Darcy, as well as Elizabeth, really loved
them; and they were both ever sensible of the warmest
gratitude towards the persons who, by bringing her into
Derbyshire, had been the means of uniting them.

\strut

\begin{center}
\textsc{finis.}
\end{center}
%%388%%
