\hyphenation{Hol-mes}

\frenchspacing

\newenvironment{letter}%
{\begin{list}{}{%X
\setlength{\leftmargin}{1em}
\setlength{\rightmargin}{\leftmargin}
\setlength{\parsep}{0em}
\setlength{\listparindent}{\parindent}
}\item[]}
{\end{list}}

\newcommand{\chsec}[1]{
\vspace{0em plus 5\baselineskip}
\section*{\centering #1}}

\newcommand{\Chapter}[1]{
\chapter{#1}
\vspace{0em plus 2\baselineskip}}

\renewcommand{\thechapter}{\Roman{chapter}}

\frontmatter
\title{Adventures of Sherlock Holmes}
\author{A. Conan Doyle}
\date{1892}

\maketitle
\tableofcontents

%% “THE GENTLEMAN IN THE PEW HANDED IT UP TO HER”
%%001
\mainmatter
\Chapter{A Scandal In Bohemia}

\chsec{I}

\textsc{To} Sherlock Holmes she is always \textit{the} woman. I
have seldom heard him mention her under any
other name. In his eyes she eclipses and predominates
the whole of her sex. It was not that he
felt any emotion akin to love for Irene Adler. All emotions,
and that one particularly, were abhorrent to his cold, precise,
but admirably balanced mind. He was, I take it, the most
perfect reasoning and observing machine that the world has
seen; but, as a lover, he would have placed himself in a false
position. He never spoke of the softer passions, save with
a gibe and a sneer. They were admirable things for the
observer -- excellent for drawing the veil from men’s motives
and actions. But for the trained reasoner to admit such intrusions
into his own delicate and finely adjusted temperament
was to introduce a distracting factor which might throw
a doubt upon all his mental results. Grit in a sensitive instrument,
or a crack in one of his own high-power lenses, would
not be more disturbing than a strong emotion in a nature
such as his. And yet there was but one woman to him, and
that woman was the late Irene Adler, of dubious and questionable
memory.

I had seen little of Holmes lately. My marriage had drifted
us away from each other. My own complete happiness, and
the home-centred interests which rise up around the man who
first finds himself master of his own establishment, were
%%010
sufficient to absorb all my attention; while Holmes, who loathed
every form of society with his whole Bohemian soul, remained
in our lodgings in Baker Street, buried among his old books,
and alternating from week to week between cocaine and ambition,
the drowsiness of the drug, and the fierce energy of
his own keen nature. He was still, as ever, deeply attracted
by the study of crime, and occupied his immense faculties
and extraordinary powers of observation in following out
those clues, and clearing up those mysteries, which had been
abandoned as hopeless by the official police. From time to
time I heard some vague account of his doings: of his summons
to Odessa in the case of the Trepoff murder, of his
clearing up of the singular tragedy of the Atkinson brothers
at Trincomalee, and finally of the mission which he had accomplished
so delicately and successfully for the reigning
family of Holland. Beyond these signs of his activity, however,
which I merely shared with all the readers of the daily
press, I knew little of my former friend and companion.

One night -- it was on the 20th of March, 1888 -- I was returning
from a journey to a patient (for I had now returned to
civil practice), when my way led me through Baker Street.
As I passed the well-remembered door, which must always be
associated in my mind with my wooing, and with the dark
incidents of the Study in Scarlet, I was seized with a keen
desire to see Holmes again, and to know how he was employing
his extraordinary powers. His rooms were brilliantly lit,
and, even as I looked up, I saw his tall, spare figure pass
twice in a dark silhouette against the blind. He was pacing
the room swiftly, eagerly, with his head sunk upon his chest
and his hands clasped behind him. To me, who knew his
every mood and habit, his attitude and manner told their own
story. He was at work again. He had arisen out of his drug-%
created dreams, and was hot upon the scent of some new
problem. I rang the bell, and was shown up to the chamber
which had formerly been in part my own.

His manner was not effusive. It seldom was; but he was
%%011
glad, I think, to see me. With hardly a word spoken, but
with a kindly eye, he waved me to an arm-chair, threw across
his case of cigars, and indicated a spirit case and a gasogene
in the corner. Then he stood before the fire, and looked me
over in his singular introspective fashion.

“Wedlock suits you,” he remarked. “I think, Watson, that
you have put on seven and a half pounds since I saw you.”

“Seven!” I answered.

“Indeed, I should have thought a little more. Just a trifle
more, I fancy, Watson. And in practice again, I observe.
You did not tell me that you intended to go into harness.”

“Then, how do you know?”

“I see it, I deduce it. How do I know that you have been
getting yourself very wet lately, and that you have a most
clumsy and careless servant girl?”

“My dear Holmes,” said I, “this is too much. You would
certainly have been burned, had you lived a few centuries ago.
It is true that I had a country walk on Thursday and came
home in a dreadful mess; but, as I have changed my clothes,
I can’t imagine how you deduce it. As to Mary Jane, she is
incorrigible, and my wife has given her notice; but there,
again, I fail to see how you work it out.”

He chuckled to himself and rubbed his long, nervous hands
together.

“It is simplicity itself,” said he; “my eyes tell me that on
the inside of your left shoe, just where the firelight strikes it,
the leather is scored by six almost parallel cuts. Obviously
they have been caused by some one who has very carelessly
scraped round the edges of the sole in order to remove
crusted mud from it. Hence, you see, my double deduction
that you had been out in vile weather, and that you had a
particularly malignant boot-slitting specimen of the London
slavey. As to your practice, if a gentleman walks into my
rooms smelling of iodoform, with a black mark of nitrate of
silver upon his right forefinger, and a bulge on the side of his
top-hat to show where he has secreted his stethoscope, I
%%012
must be dull, indeed, if I do not pronounce him to be an active
member of the medical profession.”

I could not help laughing at the ease with which he explained
his process of deduction. “When I hear you give
your reasons,” I remarked, “the thing always appears to me
to be so ridiculously simple that I could easily do it myself,
though at each successive instance of your reasoning I am
baffled, until you explain your process. And yet I believe
that my eyes are as good as yours.”

“Quite so,” he answered, lighting a cigarette, and throwing
himself down into an arm-chair. “You see, but you do not
observe. The distinction is clear. For example, you have frequently
seen the steps which lead up from the hall to this room.”

“Frequently.”

“How often?”

“Well, some hundreds of times.”

“Then how many are there?”

“How many? I don’t know.”

“Quite so! You have not observed. And yet you have
seen. That is just my point. Now, I know that there are
seventeen steps, because I have both seen and observed. By-the-way,
since you are interested in these little problems, and
since you are good enough to chronicle one or two of my
trifling experiences, you may be interested in this.” He threw
over a sheet of thick, pink-tinted note-paper which had been
lying open upon the table. “It came by the last post,” said
he. “Read it aloud.”

The note was undated, and without either signature or
address.

“There will call upon you to-night, at a quarter to eight
o’clock,” it said, “a gentleman who desires to consult you
upon a matter of the very deepest moment. Your recent services
to one of the royal houses of Europe have shown that
you are one who may safely be trusted with matters which are
of an importance which can hardly be exaggerated. This
account of you we have from all quarters received. Be in
%%013
your chamber then at that hour, and do not take it amiss if
your visitor wear a mask.”

“This is indeed a mystery,” I remarked. “What do you
imagine that it means?”

“I have no data yet. It is a capital mistake to theorize
before one has data. Insensibly one begins to twist facts to
suit theories, instead of theories to suit facts. But the note
itself. What do you deduce from it?”

I carefully examined the writing, and the paper upon which
it was written.

“The man who wrote it was presumably well to do,” I
remarked, endeavoring to imitate my companion’s processes.
“Such paper could not be bought under half a crown a
packet. It is peculiarly strong and stiff.”

“Peculiar -- that is the very word,” said Holmes. “It is
not an English paper at all. Hold it up to the light.”

I did so, and saw a large \textit{E} with a small \textit{g}, a \textit{P}, and a large
\textit{G} with a small \textit{t} woven into the texture of the paper.

“What do you make of that?” asked Holmes.

“The name of the maker, no doubt; or his monogram,
rather.”

“Not at all. The \textit{G} with the small \textit{t} stands for
‘Gesellschaft,’ which is the German for ‘Company.’ It is a
customary contraction like our ‘Co.’ \textit{P}, of course, stands for
‘Papier.’ Now for the \textit{Eg}. Let us glance at our Continental
Gazetteer.” He took down a heavy brown volume from his
shelves. “Eglow, Eglonitz -- here we are, Egria. It is in a
German-speaking country -- in Bohemia, not far from Carlsbad.
‘Remarkable as being the scene of the death of Wallenstein, and
for its numerous glass-factories and paper-mills.’ Ha, ha, my
boy, what do you make of that?” His eyes sparkled, and he
sent up a great blue triumphant cloud from his cigarette.

“The paper was made in Bohemia,” I said.

“Precisely. And the man who wrote the note is a German.
Do you note the peculiar construction of the sentence -- ‘This
account of you we have from all quarters received.’ A
%%014
Frenchman or Russian could not have written that. It is the German
who is so uncourteous to his verbs. It only remains,
therefore, to discover what is wanted by this German who
writes upon Bohemian paper, and prefers wearing a mask to
showing his face. And here he comes, if I am not mistaken,
to resolve all our doubts.”

As he spoke there was the sharp sound of horses’ hoofs and
grating wheels against the curb, followed by a sharp pull at
the bell. Holmes whistled.

“A pair, by the sound,” said he. “Yes,” he continued,
glancing out of the window. “A nice little brougham and a
pair of beauties. A hundred and fifty guineas apiece. There’s
money in this case, Watson, if there is nothing else.”

“I think that I had better go, Holmes.”

“Not a bit, doctor. Stay where you are. I am lost without
my Boswell. And this promises to be interesting. It
would be a pity to miss it.”

“But your client -- ”

“Never mind him. I may want your help, and so may he.
Here he comes. Sit down in that arm-chair, doctor, and give
us your best attention.”

A slow and heavy step, which had been heard upon the
stairs and in the passage, paused immediately outside the
door. Then there was a loud and authoritative tap.

“Come in!” said Holmes.

A man entered who could hardly have been less than six
feet six inches in height, with the chest and limbs of a
Hercules. His dress was rich with a richness which would,
in England, be looked upon as akin to bad taste. Heavy
bands of Astrakhan were slashed across the sleeves and
fronts of his double-breasted coat, while the deep blue cloak
which was thrown over his shoulders was lined with flame-%
colored silk, and secured at the neck with a brooch which
consisted of a single flaming beryl. Boots which extended
half-way up his calves, and which were trimmed at the tops
with rich brown fur, completed the impression of barbaric
%%015
%% “A MAN ENTERED”
%%016
opulence which was suggested by his whole appearance. He
carried a broad-brimmed hat in his hand, while he wore across
the upper part of his face, extending down past the cheekbones,
a black vizard mask, which he had apparently adjusted
that very moment, for his hand was still raised to it as he
entered. From the lower part of the face he appeared to be
a man of strong character, with a thick, hanging lip, and a
long, straight chin, suggestive of resolution pushed to the
length of obstinacy.

“You had my note?” he asked, with a deep harsh voice
and a strongly marked German accent. “I told you that I
would call.” He looked from one to the other of us, as if
uncertain which to address.

“Pray take a seat,” said Holmes. “This is my friend and
colleague, Dr. Watson, who is occasionally good enough to
help me in my cases. Whom have I the honor to address?”

“You may address me as the Count Von Kramm, a Bohemian
nobleman. I understand that this gentleman, your
friend, is a man of honor and discretion, whom I may trust
with a matter of the most extreme importance. If not, I
should much prefer to communicate with you alone.”

I rose to go, but Holmes caught me by the wrist and pushed
me back into my chair. “It is both, or none,” said he.
“You may say before this gentleman anything which you may
say to me.”

The count shrugged his broad shoulders. “Then I must
begin,” said he, “by binding you both to absolute secrecy for
two years, at the end of that time the matter will be of no
importance. At present it is not too much to say that it is of
such weight it may have an influence upon European history.”

“I promise,” said Holmes.

“And I.”

“You will excuse this mask,” continued our strange visitor.
“The august person who employs me wishes his agent to be
unknown to you, and I may confess at once that the title by
which I have just called myself is not exactly my own.”
%%018

“I was aware of it,” said Holmes, dryly.

“The circumstances are of great delicacy, and every precaution
has to be taken to quench what might grow to be
an immense scandal and seriously compromise one of the
reigning families of Europe. To speak plainly, the matter
implicates the great House of Ormstein, hereditary kings of
Bohemia.”

“I was also aware of that,” murmured Holmes, settling
himself down in his arm-chair and closing his eyes.

Our visitor glanced with some apparent surprise at the
languid, lounging figure of the man who had been no doubt
depicted to him as the most incisive reasoner and most energetic
agent in Europe. Holmes slowly reopened his eyes and
looked impatiently at his gigantic client.

“If your Majesty would condescend to state your case,” he
remarked, “I should be better able to advise you.”

The man sprang from his chair and paced up and down
the room in uncontrollable agitation. Then, with a gesture of
desperation, he tore the mask from his face and hurled it
upon the ground. “You are right,” he cried; “I am the
King. Why should I attempt to conceal it?”

“Why, indeed?” murmured Holmes. “Your Maj\-esty had
not spoken before I was aware that I was addressing Wilhelm
Gottsreich Sigismond von Ormstein, Grand Duke of Cassel-Felstein,
and hereditary King of Bohemia.”

“But you can understand,” said our strange visitor, sitting
down once more and passing his hand over his high, white
forehead, “you can understand that I am not accustomed to
doing such business in my own person. Yet the matter was
so delicate that I could not confide it to an agent without
putting myself in his power. I have come \textit{incognito} from
Prague for the purpose of consulting you.”

“Then, pray consult,” said Holmes, shutting his eyes once
more.

“The facts are briefly these: Some five years ago, during
a lengthy visit to Warsaw, I made the acquaintance of the
%%019
well-known adventuress, Irene Adler. The name is no doubt
familiar to you.”

“Kindly look her up in my index, doctor,” murmured
Holmes, without opening his eyes. For many years he had
adopted a system of docketing all paragraphs concerning men
and things, so that it was difficult to name a subject or a person
on which he could not at once furnish information. In
this case I found her biography sandwiched in between that
of a Hebrew Rabbi and that of a staff-commander who had
written a monograph upon the deep-sea fishes.

“Let me see!” said Holmes. “Hum! Born in New
Jersey in the year 1858. Contralto -- hum! La Scala, hum!
Prima donna Imperial Opera of Warsaw -- Yes! Retired from
operatic stage -- ha! Living in London -- quite so! Your
Majesty, as I understand, became entangled with this young
person, wrote her some compromising letters, and is now desirous
of getting those letters back.”

“Precisely so. But how -- ”

“Was there a secret marriage?”

“None.”

“No legal papers or certificates?”

“None.”

“Then I fail to follow your Majesty. If this young person
should produce her letters for blackmailing or other purposes,
how is she to prove their authenticity?”

“There is the writing.”

“Pooh, pooh! Forgery.”

“My private note-paper.”

“Stolen.”

“My own seal.”

“Imitated.”

“My photograph.”

“Bought.”

“We were both in the photograph.”

“Oh dear! That is very bad! Your Majesty has indeed
committed an indiscretion.”
%%020

“I was mad -- insane.”

“You have compromised yourself seriously.”

“I was only Crown Prince then. I was young. I am but
thirty now.”

“It must be recovered.”

“We have tried and failed.”

“Your Majesty must pay. It must be bought.”

“She will not sell.”

“Stolen, then.”

“Five attempts have been made. Twice burglars in my
pay ransacked her house. Once we diverted her luggage
when she travelled. Twice she has been waylaid. There has
been no result.”

“No sign of it?”

“Absolutely none.”

Holmes laughed. “It is quite a pretty little problem,”
said he.

“But a very serious one to me,” returned the King,
reproachfully.

“Very, indeed. And what does she propose to do with the
photograph?”

“To ruin me.”

“But how?”

“I am about to be married.”

“So I have heard.”

“To Clotilde Lothman von Saxe-Meningen, second daughter
of the King of Scandinavia. You may know the strict principles
of her family. She is herself the very soul of delicacy.
A shadow of a doubt as to my conduct would bring the matter
to an end.”

“And Irene Adler?”

“Threatens to send them the photograph. And she
will do it. I know that she will do it. You do not know
her, but she has a soul of steel. She has the face of
the most beautiful of women, and the mind of the most
resolute of men. Rather than I should marry another
%%021
woman, there are no lengths to which she would not go --
none.”

“You are sure that she has not sent it yet?”

“I am sure.”

“And why?”

“Because she has said that she would send it on the day
when the betrothal was publicly proclaimed. That will be
next Monday.”

“Oh, then, we have three days yet,” said Holmes, with a
yawn. “That is very fortunate, as I have one or two matters
of importance to look into just at present. Your Majesty will,
of course, stay in London for the present?”

“Certainly. You will find me at the Langham, under the
name of the Count Von Kramm.”

“Then I shall drop you a line to let you know how we
progress.”

“Pray do so. I shall be all anxiety.”

“Then, as to money?”

“You have \textit{carte blanche}.”

“Absolutely?”

“I tell you that I would give one of the provinces of my
kingdom to have that photograph.”

“And for present expenses?”

The king took a heavy chamois leather bag from under his
cloak and laid it on the table.

“There are three hundred pounds in gold and seven hundred
in notes,” he said.

Holmes scribbled a receipt upon a sheet of his note-book
and handed it to him.

“And mademoiselle’s address?” he asked.

“Is Briony Lodge, Serpentine Avenue, St.~John’s Wood.”

Holmes took a note of it. “One other question,” said he.
“Was the photograph a cabinet?”

“It was.”

“Then, good-night, your Majesty, and I trust that we shall
soon have some good news for you. And good-night, Watson,”
%%022
he added, as the wheels of the royal brougham rolled down
the street. “If you will be good enough to call to-morrow
afternoon, at three o’clock, I should like to chat this little
matter over with you.”

\chsec{II}

At three o’clock precisely I was at Baker Street, but Holmes
had not yet returned. The landlady informed me that he had
left the house shortly after eight o’clock in the morning. I
sat down beside the fire, however, with the intention of awaiting
him, however long he might be. I was already deeply
interested in his inquiry, for, though it was surrounded by
none of the grim and strange features which were associated
with the two crimes which I have already recorded, still, the
nature of the case and the exalted station of his client gave
it a character of its own. Indeed, apart from the nature of the
investigation which my friend had on hand, there was something
in his masterly grasp of a situation, and his keen, incisive
reasoning, which made it a pleasure to me to study his
system of work, and to follow the quick, subtle methods by
which he disentangled the most inextricable mysteries. So
accustomed was I to his invariable success that the very
possibility of his failing had ceased to enter into my head.

It was close upon four before the door opened, and a
drunken-looking groom, ill-kempt and side-whiskered, with an
inflamed face and disreputable clothes, walked into the room.
Accustomed as I was to my friend’s amazing powers in the
use of disguises, I had to look three times before I was certain
that it was indeed he. With a nod he vanished into the
bedroom, whence he emerged in five minutes tweed-suited and
respectable, as of old. Putting his hands into his pockets, he
stretched out his legs in front of the fire, and laughed heartily
for some minutes.

“Well, really!” he cried, and then he choked; and laughed
%%023
again until he was obliged to lie back, limp and helpless, in the
chair.

“What is it?”

“It’s quite too funny. I am sure you could never guess
how I employed my morning, or what I ended by doing.”

“I can’t imagine. I suppose that you have been watching
the habits, and perhaps the house, of Miss Irene Adler.”

“Quite so; but the sequel was rather unusual. I will tell
you, however. I left the house a little after eight o’clock this
morning, in the character of a groom out of work. There is
a wonderful sympathy and freemasonry among horsey men.
Be one of them, and you will know all that there is to know.
I soon found Briony Lodge. It is a \textit{bijou} villa, with a garden
at the back, but built out in front right up to the road, two
stories. Chubb lock to the door. Large sitting-room on the
right side, well furnished, with long windows almost to the
floor, and those preposterous English window fasteners which
a child could open. Behind there was nothing remarkable,
save that the passage window could be reached from the top
of the coach-house. I walked round it and examined it closely
from every point of view, but without noting anything else of
interest.

“I then lounged down the street, and found, as I expected,
that there was a mews in a lane which runs down by one wall
of the garden. I lent the ostlers a hand in rubbing down
their horses, and I received in exchange twopence, a glass of
half-and-half, two fills of shag tobacco, and as much information
as I could desire about Miss Adler, to say nothing of half
a dozen other people in the neighborhood in whom I was not
in the least interested, but whose biographies I was compelled
to listen to.”

“And what of Irene Adler?” I asked.

“Oh, she has turned all the men’s heads down in that part.
She is the daintiest thing under a bonnet on this planet. So
say the Serpentine-mews, to a man. She lives quietly, sings
at concerts, drives out at five every day, and returns at seven
%%024
sharp for dinner. Seldom goes out at other times, except
when she sings. Has only one male visitor, but a good deal
of him. He is dark, handsome, and dashing, never calls less
than once a day, and often twice. He is a Mr.~Godfrey Norton,
of the Inner Temple. See the advantages of a cabman
as a confidant. They had driven him home a dozen times
from Serpentine-mews, and knew all about him. When I had
listened to all that they had to tell, I began to walk up and
down near Briony Lodge once more, and to think over my
plan of campaign.

“This Godfrey Norton was evidently an important factor in
the matter. He was a lawyer. That sounded ominous. What
was the relation between them, and what the object of his
repeated visits? Was she his client, his friend, or his mistress?
If the former, she had probably transferred the photograph to
his keeping. If the latter, it was less likely. On the issue of
this question depended whether I should continue my work
at Briony Lodge, or turn my attention to the gentleman’s
chambers in the Temple. It was a delicate point, and it
widened the field of my inquiry. I fear that I bore you with
these details, but I have to let you see my little difficulties, if
you are to understand the situation.”

“I am following you closely,” I answered.

“I was still balancing the matter in my mind, when a hansom
cab drove up to Briony Lodge, and a gentleman sprang
out. He was a remarkably handsome man, dark, aquiline, and
mustached -- evidently the man of whom I had heard. He
appeared to be in a great hurry, shouted to the cabman to wait,
and brushed past the maid who opened the door with the air
of a man who was thoroughly at home.

“He was in the house about half an hour, and I could catch
glimpses of him in the windows of the sitting-room, pacing
up and down, talking excitedly, and waving his arms. Of
her I could see nothing. Presently he emerged, looking even
more flurried than before. As he stepped up to the cab, he
pulled a gold watch from his pocket and looked at it earnestly.
%%025
‘Drive like the devil,’ he shouted, ‘first to Gross \& Hankey’s
in Regent Street, and then to the church of St.~Monica in the
Edgware Road. Half a guinea if you do it in twenty
minutes!’

“Away they went, and I was just wondering whether I
should not do well to follow them, when up the lane came a
neat little landau, the coachman with his coat only half-buttoned,
and his tie under his ear, while all the tags of his
harness were sticking out of the buckles. It hadn’t pulled up
before she shot out of the hall door and into it. I only caught
a glimpse of her at the moment, but she was a lovely woman,
with a face that a man might die for.

“\,‘The Church of St.~Monica, John,’ she cried, ‘and half a
sovereign if you reach it in twenty minutes.’

“This was quite too good to lose, Watson. I was just
balancing whether I should run for it, or whether I should
perch behind her landau, when a cab came through the street.
The driver looked twice at such a shabby fare; but I jumped
in before he could object. ‘The Church of St.~Monica,’ said
I, ‘and half a sovereign if you reach it in twenty minutes.’
It was twenty-five minutes to twelve, and of course it was
clear enough what was in the wind.

“My cabby drove fast. I don’t think I ever drove faster,
but the others were there before us. The cab and the landau
with their steaming horses were in front of the door when I
arrived. I paid the man and hurried into the church. There
was not a soul there save the two whom I had followed and a
surpliced clergyman, who seemed to be expostulating with
them. They were all three standing in a knot in front of the
altar. I lounged up the side aisle like any other idler who has
dropped into a church. Suddenly, to my surprise, the three
at the altar faced round to me, and Godfrey Norton came
running as hard as he could towards me.”

“Thank God!” he cried. “You’ll do. Come! Come!”

“What then?” I asked.

“Come, man, come, only three minutes, or it won’t be legal.”
%%026

I was half-dragged up to the altar, and, before I knew
where I was, I found myself mumbling responses which were
whispered in my ear, and vouching for things of which I knew
nothing, and generally assisting in the secure tying up of
Irene Adler, spinster, to Godfrey Norton, bachelor. It was
all done in an instant, and there was the gentleman thanking
me on the one side and the lady on the other, while the
clergyman beamed on me in front. It was the most preposterous
position in which I ever found myself in my life, and it
was the thought of it that started me laughing just now. It
seems that there had been some informality about their license,
that the clergyman absolutely refused to marry them without
a witness of some sort, and that my lucky appearance saved
the bridegroom from having to sally out into the streets in
search of a best man. The bride gave me a sovereign, and I
mean to wear it on my watch-chain in memory of the
occasion.”

“This is a very unexpected turn of affairs,” said I; “and
what then?”

“Well, I found my plans very seriously menaced. It looked
as if the pair might take an immediate departure, and so
necessitate very prompt and energetic measures on my part.
At the church door, however, they separated, he driving back
to the Temple, and she to her own house. ‘I shall drive out
in the park at five as usual,’ she said, as she left him. I heard
no more. They drove away in different directions, and I
went off to make my own arrangements.”

“Which are?”

“Some cold beef and a glass of beer,” he answered, ringing
the bell. “I have been too busy to think of food, and I am
likely to be busier still this evening. By the way, doctor, I
shall want your co-operation.”

“I shall be delighted.”

“You don’t mind breaking the law?”

“Not in the least.”

“Nor running a chance of arrest?”
%%027

“Not in a good cause.”

“Oh, the cause is excellent!”

“Then I am your man.”

“I was sure that I might rely on you.”

“But what is it you wish?”

“When Mrs.~Turner has brought in the tray I will make it
clear to you. Now,” he said, as he turned hungrily on the
simple fare that our landlady had provided, “I must discuss
it while I eat, for I have not much time. It is nearly five
now. In two hours we must be on the scene of action.
Miss Irene, or Madame, rather, returns from her drive at
seven. We must be at Briony Lodge to meet her.”

“And what then?”

“You must leave that to me. I have already arranged
what is to occur. There is only one point on which I must
insist. You must not interfere, come what may. You understand?”

“I am to be neutral?”

“To do nothing whatever. There will probably be some
small unpleasantness. Do not join in it. It will end in my
being conveyed into the house. Four or five minutes afterwards
the sitting-room window will open. You are to station
yourself close to that open window.”

“Yes.”

“You are to watch me, for I will be visible to you.”

“Yes.”

“And when I raise my hand -- so -- you will throw into the
room what I give you to throw, and will, at the same time,
raise the cry of fire. You quite follow me?”

“Entirely.”

“It is nothing very formidable,” he said, taking a long
cigar-shaped roll from his pocket. “It is an ordinary plumber’s
smoke-rocket, fitted with a cap at either end to make it
self-lighting. Your task is confined to that. When you raise
your cry of fire, it will be taken up by quite a number of people.
You may then walk to the end of the street, and I will
%%028
rejoin you in ten minutes. I hope that I have made myself
clear?”

“I am to remain neutral, to get near the window, to watch
you, and, at the signal, to throw in this object, then to raise
the cry of fire, and to wait you at the corner of the street.”

“Precisely.”

“Then you may entirely rely on me.”

“That is excellent. I think, perhaps, it is almost time that
I prepare for the new role I have to play.”

He disappeared into his bedroom, and returned in a few
minutes in the character of an amiable and simple-minded
Nonconformist clergyman. His broad black hat, his baggy
trousers, his white tie, his sympathetic smile, and general
look of peering and benevolent curiosity were such as Mr.
John Hare alone could have equalled. It was not merely
that Holmes changed his costume. His expression, his manner,
his very soul seemed to vary with every fresh part that
he assumed. The stage lost a fine actor, even as science lost
an acute reasoner, when he became a specialist in crime.

It was a quarter past six when we left Baker Street, and it
still wanted ten minutes to the hour when we found ourselves
in Serpentine Avenue. It was already dusk, and the lamps
were just being lighted as we paced up and down in front of
Briony Lodge, waiting for the coming of its occupant. The
house was just such as I had pictured it from Sherlock
Holmes’ succinct description, but the locality appeared to be
less private that I expected. On the contrary, for a small
street in a quiet neighborhood, it was remarkably animated.
There was a group of shabbily-dressed men smoking and
laughing in a corner, a scissors-grinder with his wheel, two
guardsmen who were flirting with a nurse-girl, and several
well-dressed young men who were lounging up and down with
cigars in their mouths.

“You see,” remarked Holmes, as we paced to and fro in
front of the house, “this marriage rather simplifies matters.
The photograph becomes a double-edged weapon now. The
%%029
chances are that she would be as averse to its being seen by
Mr.~Godfrey Norton, as our client is to its coming to the
eyes of his princess. Now the question is, Where are we to
find the photograph?”

“Where, indeed?”

“It is most unlikely that she carries it about with her. It
is cabinet size. Too large for easy concealment about a
woman’s dress. She knows that the King is capable of having
her waylaid and searched. Two attempts of the sort have
already been made. We may take it, then, that she does not
carry it about with her.”

“Where, then?”

“Her banker or her lawyer. There is that double possibility.
But I am inclined to think neither. Women are
naturally secretive, and they like to do their own secreting.
Why should she hand it over to any one else? She could
trust her own guardianship, but she could not tell what indirect
or political influence might be brought to bear upon a
business man. Besides, remember that she had resolved to
use it within a few days. It must be where she can lay her
hands upon it. It must be in her own house.”

“But it has twice been burgled.”

“Pshaw! They did not know how to look.”

“But how will you look?”

“I will not look.”

“What then?”

“I will get her to show me.”

“But she will refuse.”

“She will not be able to. But I hear the rumble of wheels.
It is her carriage. Now carry out my orders to the letter.”

As he spoke the gleam of the side-lights of a carriage came
round the curve of the avenue. It was a smart little landau
which rattled up to the door of Briony Lodge. As it pulled
up, one of the loafing men at the corner dashed forward to
open the door in the hope of earning a copper, but was elbowed
away by another loafer, who had rushed up with the
%%030
same intention. A fierce quarrel broke out, which was
increased by the two guardsmen, who took sides with one of the
loungers, and by the scissors-grinder, who was equally hot
upon the other side. A blow was struck, and in an instant
the lady, who had stepped from her carriage, was the centre
of a little knot of flushed and struggling men, who struck
savagely at each other with their fists and sticks. Holmes
dashed into the crowd to protect the lady; but just as he
reached her he gave a cry and dropped to the ground, with
the blood running freely down his face. At his fall the
guardsmen took to their heels in one direction and the loungers
in the other, while a number of better dressed people,
who had watched the scuffle without taking part in it, crowded
in to help the lady and to attend to the injured man. Irene
Adler, as I will still call her, had hurried up the steps; but
she stood at the top with her superb figure outlined against
the lights of the hall, looking back into the street.

“Is the poor gentleman much hurt?” she asked.

“He is dead,” cried several voices.

“No, no, there’s life in him!” shouted another. “But he’ll
be gone before you can get him to hospital.”

“He’s a brave fellow,” said a woman. “They would have
had the lady’s purse and watch if it hadn’t been for him.
They were a gang, and a rough one, too. Ah, he’s breathing
now.”

“He can’t lie in the street. May we bring him in, marm?”

“Surely. Bring him into the sitting-room. There is a
comfortable sofa. This way, please!”

Slowly and solemnly he was borne into Briony Lodge and
laid out in the principal room, while I still observed the
proceedings from my post by the window. The lamps had been
lit, but the blinds had not been drawn, so that I could see
Holmes as he lay upon the couch. I do not know whether
he was seized with compunction at that moment for the part
he was playing, but I know that I never felt more heartily
ashamed of myself in my life than when I saw the beautiful
%%031
creature against whom I was conspiring, or the grace and
kindliness with which she waited upon the injured man. And
yet it would be the blackest treachery to Holmes to draw
back now from the part which he had intrusted to me. I
hardened my heart, and took the smoke-rocket from under
my ulster. After all, I thought, we are not injuring her. We
are but preventing her from injuring another.

Holmes had sat up upon the couch, and I saw him motion
like a man who is in need of air. A maid rushed across and
threw open the window. At the same instant I saw him raise
his hand, and at the signal I tossed my rocket into the room
with a cry of “Fire!” The word was no sooner out of my
mouth than the whole crowd of spectators, well dressed and
ill -- gentlemen, ostlers, and servant-maids -- joined in a general
shriek of “Fire!” Thick clouds of smoke curled through
the room and out at the open window. I caught a glimpse
of rushing figures, and a moment later the voice of Holmes
from within assuring them that it was a false alarm. Slipping
through the shouting crowd I made my way to the corner of
the street, and in ten minutes was rejoiced to find my friend’s
arm in mine, and to get away from the scene of uproar. He
walked swiftly and in silence for some few minutes, until we
had turned down one of the quiet streets which lead towards
the Edgware Road.

“You did it very nicely, doctor,” he remarked. “Nothing
could have been better. It is all right.”

“You have the photograph?”

“I know where it is.”

“And how did you find out?”

“She showed me, as I told you that she would.”

“I am still in the dark.”

“I do not wish to make a mystery,” said he, laughing.
“The matter was perfectly simple. You, of course, saw that
every one in the street was an accomplice. They were all engaged
for the evening.”

“I guessed as much.”
%%032

“Then, when the row broke out, I had a little moist red
paint in the palm of my hand. I rushed forward, fell down,
clapped my hand to my face, and became a piteous spectacle.
It is an old trick.”

“That also I could fathom.”

“Then they carried me in. She was bound to have me in.
What else could she do? And into her sitting-room, which
was the very room which I suspected. It lay between that
and her bedroom, and I was determined to see which. They
laid me on a couch, I motioned for air, they were compelled
to open the window, and you had your chance.”

“How did that help you?”

“It was all-important. When a woman thinks that her
house is on fire, her instinct is at once to rush to the thing
which she values most. It is a perfectly overpowering impulse,
and I have more than once taken advantage of it. In
the case of the Darlington Substitution Scandal it was of use
to me, and also in the Arnsworth Castle business. A married
woman grabs at her baby; an unmarried one reaches for her
jewel-box. Now it was clear to me that our lady of to-day
had nothing in the house more precious to her than what we
are in quest of. She would rush to secure it. The alarm of
fire was admirably done. The smoke and shouting were
enough to shake nerves of steel. She responded beautifully.
The photograph is in a recess behind a sliding panel just
above the right bell-pull. She was there in an instant, and I
caught a glimpse of it as she half-drew it out. When I cried
out that it was a false alarm, she replaced it, glanced at the
rocket, rushed from the room, and I have not seen her since.
I rose, and, making my excuses, escaped from the house. I
hesitated whether to attempt to secure the photograph at
once; but the coachman had come in, and as he was watching
me narrowly, it seemed safer to wait. A little over-precipitance
may ruin all.”

“And now?” I asked.

“Our quest is practically finished. I shall call with the
%%033
King to-morrow, and with you, if you care to come with us.
We will be shown into the sitting-room to wait for the lady,
but it is probable that when she comes she may find neither
us nor the photograph. It might be a satisfaction to His
Majesty to regain it with his own hands.”

“And when will you call?”

“At eight in the morning. She will not be up, so that we
shall have a clear field. Besides, we must be prompt, for this
marriage may mean a complete change in her life and habits.
I must wire to the King without delay.”

We had reached Baker Street, and had stopped at the door.
He was searching his pockets for the key, when some one
passing said:

“Good-night, Mister Sherlock Holmes.”

There were several people on the pavement at the time, but
the greeting appeared to come from a slim youth in an ulster
who had hurried by.

“I’ve heard that voice before,” said Holmes, staring down
the dimly-lit street. “Now, I wonder who the deuce that
could have been.”

\chsec{III}

I slept at Baker Street that night, and we were engaged
upon our toast and coffee in the morning when the King of
Bohemia rushed into the room.

“You have really got it!” he cried, grasping Sherlock
Holmes by either shoulder, and looking eagerly into his face.

“Not yet.”

“But you have hopes?”

“I have hopes.”

“Then, come. I am all impatience to be gone.”

“We must have a cab.”

“No, my brougham is waiting.”

“Then that will simplify matters.” We descended, and
started off once more for Briony Lodge.
%%034

“Irene Adler is married,” remarked Holmes.

“Married! When?”

“Yesterday.”

“But to whom?”

“To an English lawyer named Norton.”

“But she could not love him?”

“I am in hopes that she does.”

“And why in hopes?”

“Because it would spare your Majesty all fear of future annoyance.
If the lady loves her husband, she does not love
your Majesty. If she does not love your Majesty, there is no
reason why she should interfere with your Majesty’s plan.”

“It is true. And yet -- Well! I wish she had been of
my own station! What a queen she would have made!” He
relapsed into a moody silence, which was not broken until we
drew up in Serpentine Avenue.

The door of Briony Lodge was open, and an elderly woman
stood upon the steps. She watched us with a sardonic eye
as we stepped from the brougham.

“Mr.~Sherlock Holmes, I believe?” said she.

“I am Mr.~Holmes,” answered my companion, looking at
her with a questioning and rather startled gaze.

“Indeed! My mistress told me that you were likely to call.
She left this morning with her husband by the 5.15 train from
Charing Cross for the Continent.”

“What!” Sherlock Holmes staggered back, white with
chagrin and surprise. “Do you mean that she has left
England?”

“Never to return.”

“And the papers?” asked the King, hoarsely. “All is lost.”

“We shall see.” He pushed past the servant and rushed
into the drawing-room, followed by the King and myself. The
furniture was scattered about in every direction, with dismantled
shelves and open drawers, as if the lady had hurriedly
ransacked them before her flight. Holmes rushed at the
bell-pull, tore back a small sliding shutter, and, plunging in his
%%035
hand, pulled out a photograph and a letter. The photograph
was of Irene Adler herself in evening dress, the letter was
superscribed to “Sherlock Holmes, Esq. To be left till called
for.” My friend tore it open, and we all three read it together.
It was dated at midnight of the preceding night, and ran
in this way:

\begin{letter}
“\textsc{My Dear Mr.~Sherlock Holmes}, -- You really did it
very well. You took me in completely. Until after the alarm
of fire, I had not a suspicion. But then, when I found how I
had betrayed myself, I began to think. I had been warned
against you months ago. I had been told that, if the King
employed an agent, it would certainly be you. And your address
had been given me. Yet, with all this, you made me reveal
what you wanted to know. Even after I became suspicious,
I found it hard to think evil of such a dear, kind old
clergyman. But, you know, I have been trained as an actress
myself. Male costume is nothing new to me. I often take
advantage of the freedom which it gives. I sent John, the
coachman, to watch you, ran up-stairs, got into my walking-%
clothes, as I call them, and came down just as you departed.

“Well, I followed you to your door, and so made sure that
I was really an object of interest to the celebrated Mr.~Sherlock
Holmes. Then I, rather imprudently, wished you good-night,
and started for the Temple to see my husband.

“We both thought the best resource was flight, when pursued
by so formidable an antagonist; so you will find the
nest empty when you call to-morrow. As to the photograph,
your client may rest in peace. I love and am loved by a better
man than he. The King may do what he will without
hindrance from one whom he has cruelly wronged. I keep
it only to safeguard myself, and to preserve a weapon which
will always secure me from any steps which he might take in
the future. I leave a photograph which he might care to
possess; and I remain, dear Mr.~Sherlock Holmes, very truly
yours,

\hfill\textsc{Irene Norton, \textit{née} Adler}.”
\end{letter}
%%036

“What a woman -- oh, what a woman!” cried the King of
Bohemia, when we had all three read this epistle. “Did I
not tell you how quick and resolute she was? Would she not
have made an admirable queen? Is it not a pity that she
was not on my level?”

“From what I have seen of the lady she seems indeed to
be on a very different level to your Majesty,” said Holmes,
coldly. “I am sorry that I have not been able to bring your
Majesty’s business to a more successful conclusion.”

“On the contrary, my dear sir,” cried the King; “nothing
could be more successful. I know that her word is inviolate.
The photograph is now as safe as if it were in the fire.”

“I am glad to hear your Majesty say so.”

“I am immensely indebted to you. Pray tell me in what
way I can reward you. This ring -- ” He slipped an emerald
snake ring from his finger and held it out upon the palm of
his hand.

“Your Majesty has something which I should value even
more highly,” said Holmes.

“You have but to name it.”

“This photograph!”

The King stared at him in amazement.

“Irene’s photograph!” he cried. “Certainly, if you wish it.”

“I thank your Majesty. Then there is no more to be done
in the matter. I have the honor to wish you a very good-%
morning.” He bowed, and, turning away without observing
the hand which the King had stretched out to him, he set off
in my company for his chambers.

\strut

And that was how a great scandal threatened to affect the
kingdom of Bohemia, and how the best plans of Mr.~Sherlock
Holmes were beaten by a woman’s wit. He used to make
merry over the cleverness of women, but I have not heard
him do it of late. And when he speaks of Irene Adler, or
when he refers to her photograph, it is always under the
honorable title of \textit{the} woman.
%%037

\Chapter{The Red-Headed League}
\vspace{0em plus 1em}

I had called upon my friend, Mr.~Sherlock Holmes,
one day in the autumn of last year, and found him
in deep conversation with a very stout, florid-faced,
elderly gentleman, with fiery red hair. With an
apology for my intrusion, I was about to withdraw, when
Holmes pulled me abruptly into the room and closed the
door behind me.

“You could not possibly have come at a better time, my
dear Watson,” he said, cordially.

“I was afraid that you were engaged.”

“So I am. Very much so.”

“Then I can wait in the next room.”

“Not at all. This gentleman, Mr.~Wilson, has been my
partner and helper in many of my most successful cases, and
I have no doubt that he will be of the utmost use to me in
yours also.”

The stout gentleman half-rose from his chair and gave a
bob of greeting, with a quick, little, questioning glance from
his small, fat-encircled eyes.

“Try the settee,” said Holmes, relapsing into his arm-chair
and putting his finger-tips together, as was his custom when
in judicial moods. “I know, my dear Watson, that you share
my love of all that is bizarre and outside the conventions and
humdrum routine of every-day life. You have shown your
relish for it by the enthusiasm which has prompted you to
chronicle, and, if you will excuse my saying so, somewhat to
embellish so many of my own little adventures.”
%%038

“Your cases have indeed been of the greatest interest to
me,” I observed.

“You will remember that I remarked the other day, just
before we went into the very simple problem presented by
Miss Mary Sutherland, that for strange effects and extraordinary
combinations we must go to life itself, which is always
far more daring than any effort of the imagination.”

“A proposition which I took the liberty of doubting.”

“You did, doctor, but none the less you must come round
to my view, for otherwise I shall keep on piling fact upon fact
on you, until your reason breaks down under them and acknowledges
me to be right. Now, Mr.~Jabez Wilson here has
been good enough to call upon me this morning, and to begin
a narrative which promises to be one of the most singular
which I have listened to for some time. You have heard me
remark that the strangest and most unique things are very
often connected not with the larger but with the smaller
crimes, and occasionally, indeed, where there is room for
doubt whether any positive crime has been committed. As
far as I have heard it is impossible for me to say whether the
present case is an instance of crime or not, but the course of
events is certainly among the most singular that I have ever
listened to. Perhaps, Mr.~Wilson, you would have the great
kindness to recommence your narrative. I ask you, not
merely because my friend Dr. Watson has not heard the
opening part, but also because the peculiar nature of the story
makes me anxious to have every possible detail from your
lips. As a rule, when I have heard some slight indication of
the course of events, I am able to guide myself by the thousands
of other similar cases which occur to my memory. In
the present instance I am forced to admit that the facts are,
to the best of my belief, unique.”

The portly client puffed out his chest with an appearance of
some little pride, and pulled a dirty and wrinkled newspaper
from the inside pocket of his great-coat. As he glanced down
the advertisement column, with his head thrust forward, and
%%039
the paper flattened out upon his knee, I took a good look at
the man, and endeavored, after the fashion of my companion,
to read the indications which might be presented by his dress
or appearance.

I did not gain very much, however, by my inspection. Our
visitor bore every mark of being an average commonplace
British tradesman, obese, pompous, and slow. He wore rather
baggy gray shepherd’s check trousers, a not over-clean black
frock-coat, unbuttoned in the front, and a drab waistcoat with
a heavy brassy Albert chain, and a square pierced bit of metal
dangling down as an ornament. A frayed top-hat and a faded
brown overcoat with a wrinkled velvet collar lay upon a chair
beside him. Altogether, look as I would, there was nothing
remarkable about the man save his blazing red head, and the
expression of extreme chagrin and discontent upon his
features.

Sherlock Holmes’s quick eye took in my occupation, and he
shook his head with a smile as he noticed my questioning
glances. “Beyond the obvious facts that he has at some time
done manual labor, that he takes snuff, that he is a Freemason,
that he has been in China, and that he has done a considerable
amount of writing lately, I can deduce nothing else.”

Mr.~Jabez Wilson started up in his chair, with his forefinger
upon the paper, but his eyes upon my companion.

“How, in the name of good-fortune, did you know all that,
Mr.~Holmes?” he asked. “How did you know, for example,
that I did manual labor. It’s as true as gospel, for I began
as a ship’s carpenter.”

“Your hands, my dear sir. Your right hand is quite a size
larger than your left. You have worked with it, and the
muscles are more developed.”

“Well, the snuff, then, and the Freemasonry?”

“I won’t insult your intelligence by telling you how I read
that, especially as, rather against the strict rules of your order,
you use an arc-and-compass breastpin.”

“Ah, of course, I forgot that. But the writing?”
%%040

“What else can be indicated by that right cuff so very
shiny for five inches, and the left one with the smooth patch
near the elbow where you rest it upon the desk.”

“Well, but China?”

“The fish that you have tattooed immediately above your
right wrist could only have been done in China. I have made
a small study of tattoo marks, and have even contributed to
the literature of the subject. That trick of staining the fishes’
scales of a delicate pink is quite peculiar to China. When, in
addition, I see a Chinese coin hanging from your watch-chain,
the matter becomes even more simple.”

Mr.~Jabez Wilson laughed heavily. “Well, I never!” said
he. “I thought at first that you had done something clever,
but I see that there was nothing in it, after all.”

“I begin to think, Watson,” said Holmes, “that I make a
mistake in explaining. ‘Omne ignotum pro magnifico,’ you
know, and my poor little reputation, such as it is, will suffer
shipwreck if I am so candid. Can you not find the advertisement,
Mr.~Wilson?”

“Yes, I have got it now,” he answered, with his thick, red
finger planted half-way down the column. “Here it is. This
is what began it all. You just read it for yourself, sir.”

I took the paper from him, and read as follows:

“\textsc{To the Red-headed League}: On account of the bequest
of the late Ezekiah Hopkins, of Lebanon, Pa., U.S.A.,
there is now another vacancy open which entitles a member
of the League to a salary of £4 a week for purely nominal
services. All red-headed men who are sound in body and
mind, and above the age of twenty-one years, are eligible. Apply
in person on Monday, at eleven o’clock, to Duncan Ross,
at the offices of the League, 7 Pope’s Court, Fleet Street.”

“What on earth does this mean?” I ejaculated, after I had
twice read over the extraordinary announcement.

Holmes chuckled, and wriggled in his chair, as was his
habit when in high spirits. “It is a little off the beaten
%%041
track, isn’t it?” said he. “And now, Mr.~Wilson, off you go
at scratch, and tell us all about yourself, your household, and
the effect which this advertisement had upon your fortunes.
You will first make a note, doctor, of the paper and the
date.”

“It is \textit{The Morning Chronicle}, of April 27, 1890. Just two
months ago.”

“Very good. Now, Mr.~Wilson?”

“Well, it is just as I have been telling you, Mr.~Sherlock
Holmes,” said Jabez Wilson, mopping his forehead; “I have
a small pawnbroker’s business at Coburg Square, near the
city. It’s not a very large affair, and of late years it has not
done more than just give me a living. I used to be able to
keep two assistants, but now I only keep one; and I would
have a job to pay him, but that he is willing to come for half
wages, so as to learn the business.”

“What is the name of this obliging youth?” asked Sherlock
Holmes.

“His name is Vincent Spaulding, and he’s not such a youth,
either. It’s hard to say his age. I should not wish a smarter
assistant, Mr.~Holmes; and I know very well that he could
better himself, and earn twice what I am able to give him.
But, after all, if he is satisfied, why should I put ideas in his
head?”

“Why, indeed? You seem most fortunate in having an
\textit{employé} who comes under the full market price. It is not a
common experience among employers in this age. I don’t
know that your assistant is not as remarkable as your
advertisement.”

“Oh, he has his faults, too,” said Mr.~Wilson. “Never
was such a fellow for photography. Snapping away with a
camera when he ought to be improving his mind, and then
diving down into the cellar like a rabbit into its hole to
develope his pictures. That is his main fault; but, on the
whole, he’s a good worker. There’s no vice in him.”

“He is still with you, I presume?”
%%042

“Yes, sir. He and a girl of fourteen, who does a bit of
simple cooking, and keeps the place clean -- that’s all I have
in the house, for I am a widower, and never had any family.
We live very quietly, sir, the three of us; and we keep a roof
over our heads, and pay our debts, if we do nothing more.

“The first thing that put us out was that advertisement.
Spaulding, he came down into the office just this day eight
weeks, with this very paper in his hand, and he says:

“\,‘I wish to the Lord, Mr.~Wilson, that I was a red-headed
man.’

“\,‘Why that?’ I asks.

“\,‘Why,’ says he, ‘here’s another vacancy on the League of
the Red-headed Men. It’s worth quite a little fortune to any
man who gets it, and I understand that there are more vacancies
than there are men, so that the trustees are at their wits’
end what to do with the money. If my hair would only change
color, here’s a nice little crib all ready for me to step into.’

“\,‘Why, what is it, then?’ I asked. You see, Mr.~Holmes, I
am a very stay-at-home man, and as my business came to me
instead of my having to go to it, I was often weeks on end
without putting my foot over the door-mat. In that way I
didn’t know much of what was going on outside, and I was
always glad of a bit of news.

“\,‘Have you never heard of the League of the Red-headed
Men?’ he asked, with his eyes open.

“\,‘Never.’

“\,‘Why, I wonder at that, for you are eligible yourself for
one of the vacancies.’

“\,‘And what are they worth?’ I asked.

“\,‘Oh, merely a couple of hundred a year, but the work is
slight, and it need not interfere very much with one’s other
occupations.’

“Well, you can easily think that that made me prick up my
ears, for the business has not been over-good for some years,
and an extra couple of hundred would have been very handy.

“\,‘Tell me all about it,’ said I.
%%043

“\,‘Well,’ said he, showing me the advertisement, ‘you can
see for yourself that the League has a vacancy, and there is
the address where you should apply for particulars. As far as
I can make out, the League was founded by an American
millionaire, Ezekiah Hopkins, who was very peculiar in his
ways. He was himself red-headed, and he had a great sympathy
for all red-headed men; so, when he died, it was found
that he had left his enormous fortune in the hands of trustees,
with instructions to apply the interest to the providing of easy
berths to men whose hair is of that color. From all I hear it
is splendid pay, and very little to do.’

“\,‘But,’ said I, ‘there would be millions of red-headed men
who would apply.’

“\,‘Not so many as you might think,’ he answered. ‘You
see it is really confined to Londoners, and to grown men.
This American had started from London when he was young,
and he wanted to do the old town a good turn. Then, again,
I have heard it is no use your applying if your hair is light
red, or dark red, or anything but real bright, blazing, fiery red.
Now, if you cared to apply, Mr.~Wilson, you would just walk in;
but perhaps it would hardly be worth your while to put yourself
out of the way for the sake of a few hundred pounds.’

“Now, it is a fact, gentlemen, as you may see for yourselves,
that my hair is of a very full and rich tint, so that it seemed
to me that, if there was to be any competition in the matter, I
stood as good a chance as any man that I had ever met.
Vincent Spaulding seemed to know so much about it that I
thought he might prove useful, so I just ordered him to put
up the shutters for the day, and to come right away with me.
He was very willing to have a holiday, so we shut the business
up, and started off for the address that was given us in the
advertisement.

\begin{sloppypar}
“I never hope to see such a sight as that again, Mr.\ Holmes.
From north, south, east, and west every man who had a shade
of red in his hair had tramped into the city to answer the
advertisement. Fleet Street was choked with red-headed folk,
%%044
and Pope’s Court looked like a coster’s orange barrow. I
should not have thought there were so many in the whole
country as were brought together by that single advertisement.
Every shade of color they were -- straw, lemon, orange,
brick, Irish-setter, liv\-er, clay; but, as Spaulding said, there
were not many who had the real vivid flame-colored tint.
When I saw how many were waiting, I would have given it up
in despair; but Spaulding would not hear of it. How he did
it I could not imagine, but he pushed and pulled and butted
until he got me through the crowd, and right up to the steps
which led to the office. There was a double stream upon the
stair, some going up in hope, and some coming back dejected;
but we wedged in as well as we could, and soon found
ourselves in the office.”
\end{sloppypar}

“Your experience has been a most entertaining one,” remarked
Holmes, as his client paused and refreshed his memory
with a huge pinch of snuff. “Pray continue your very
interesting statement.”

“There was nothing in the office but a couple of wooden
chairs and a deal table, behind which sat a small man, with a
head that was even redder than mine. He said a few words
to each candidate as he came up, and then he always managed
to find some fault in them which would disqualify them. Getting
a vacancy did not seem to be such a very easy matter,
after all. However, when our turn came, the little man was
much more favorable to me than to any of the others, and he
closed the door as we entered, so that he might have a private
word with us.

“\,‘This is Mr.~Jabez Wilson,’ said my assistant, ‘and he is
willing to fill a vacancy in the League.’

“\,‘And he is admirably suited for it,’ the other answered.
‘He has every requirement. I cannot recall when I have seen
anything so fine.’ He took a step backward, cocked his head
on one side, and gazed at my hair until I felt quite bashful.
Then suddenly he plunged forward, wrung my hand, and congratulated
me warmly on my success.
%%045

“\,‘It would be injustice to hesitate,’ said he. ‘You will,
however, I am sure, excuse me for taking an obvious precaution.’
With that he seized my hair in both his hands, and
tugged until I yelled with the pain. ‘There is water in your
eyes,’ said he, as he released me. ‘I perceive that all is as it
should be. But we have to be careful, for we have twice been
deceived by wigs and once by paint. I could tell you tales of
cobbler’s wax which would disgust you with human nature.’
He stepped over to the window, and shouted through it at the
top of his voice that the vacancy was filled. A groan of
disappointment came up from below, and the folk all trooped
away in different directions, until there was not a red head to
be seen except my own and that of the manager.

“\,‘My name,’ said he, ‘is Mr.~Duncan Ross, and I am myself
one of the pensioners upon the fund left by our noble
benefactor. Are you a married man, Mr.~Wilson? Have you
a family?’

“I answered that I had not.

“His face fell immediately.

“\,‘Dear me!’ he said, gravely, ‘that is very serious indeed!
I am sorry to hear you say that. The fund was, of course,
for the propagation and spread of the red-heads as well as for
their maintenance. It is exceedingly unfortunate that you
should be a bachelor.’

“My face lengthened at this, Mr.~Holmes, for I thought
that I was not to have the vacancy after all; but, after
thinking it over for a few minutes, he said that it would be
all right.

“\,‘In the case of another,’ said he, ‘the objection might be
fatal, but we must stretch a point in favor of a man with such
a head of hair as yours. When shall you be able to enter
upon your new duties?’

“\,‘Well, it is a little awkward, for I have a business already,’
said I.

“\,‘Oh, never mind about that, Mr.~Wilson!’ said Vincent
Spaulding. ‘I shall be able to look after that for you.’
%%046

“\,‘What would be the hours?’ I asked.

“\,‘Ten to two.’

“Now a pawnbroker’s business is mostly done of an evening,
Mr.~Holmes, especially Thursday and Friday evening,
which is just before pay-day; so it would suit me very well
to earn a little in the mornings. Besides, I knew that my
assistant was a good man, and that he would see to anything
that turned up.

“\,‘That would suit me very well,’ said I. ‘And the pay?’

“\,‘Is £4 a week.’

“\,‘And the work?’

“\,‘Is purely nominal.’

“\,‘What do you call purely nominal?’

“\,‘Well, you have to be in the office, or at least in the
building, the whole time. If you leave, you forfeit your whole
position forever. The will is very clear upon that point.
You don’t comply with the conditions if you budge from the
office during that time.’

“\,‘It’s only four hours a day, and I should not think of
leaving,’ said I.

“\,‘No excuse will avail,’ said Mr.~Duncan Ross, ‘neither
sickness nor business nor anything else. There you must
stay, or you lose your billet.’

“\,‘And the work?’

“\,‘Is to copy out the “Encyclopædia Britannica.” There
is the first volume of it in that press. You must find your
own ink, pens, and blotting-paper, but we provide this table
and chair. Will you be ready to-morrow?’

“\,‘Certainly,’ I answered.

“\,‘Then, good-bye, Mr.~Jabez Wilson, and let me congratulate
you once more on the important position which you
have been fortunate enough to gain.’ He bowed me out of
the room, and I went home with my assistant, hardly knowing
what to say or do, I was so pleased at my own good
fortune.

“Well, I thought over the matter all day, and by evening I
%%047
was in low spirits again; for I had quite persuaded myself
that the whole affair must be some great hoax or fraud,
though what its object might be I could not imagine. It
seemed altogether past belief that any one could make such
a will, or that they would pay such a sum for doing anything
so simple as copying out the ‘Encyclopædia Britannica.’
Vincent Spaulding did what he could to cheer me up, but by
bedtime I had reasoned myself out of the whole thing.
However, in the morning I determined to have a look at it
anyhow, so I bought a penny bottle of ink, and with a quill-%
pen, and seven sheets of foolscap paper, I started off for
Pope’s Court.

“Well, to my surprise and delight, everything was as right
as possible. The table was set out ready for me, and Mr.
Duncan Ross was there to see that I got fairly to work. He
started me off upon the letter A, and then he left me; but he
would drop in from time to time to see that all was right with
me. At two o’clock he bade me good-day, complimented me
upon the amount that I had written, and locked the door of
the office after me.

“This went on day after day, Mr.~Holmes, and on Saturday
the manager came in and planked down four golden sovereigns
for my week’s work. It was the same next week, and
the same the week after. Every morning I was there at ten,
and every afternoon I left at two. By degrees Mr.~Duncan
Ross took to coming in only once of a morning, and then, after
a time, he did not come in at all. Still, of course, I never
dared to leave the room for an instant, for I was not sure
when he might come, and the billet was such a good one,
and suited me so well, that I would not risk the loss of it.

“Eight weeks passed away like this, and I had written
about Abbots and Archery and Armor and Architecture and
Attica, and hoped with diligence that I might get on to the
B’s before very long. It cost me something in foolscap, and I
had pretty nearly filled a shelf with my writings. And then
suddenly the whole business came to an end.”
%%048

“To an end?”

“Yes, sir. And no later than this morning. I went to my
work as usual at ten o’clock, but the door was shut and
locked, with a little square of card-board hammered on to the
middle of the panel with a tack. Here it is, and you can read
for yourself.”

He held up a piece of white card-board about the size of a
sheet of note-paper. It read in this fashion:

\begin{center}
\scshape “The Red-Headed League

is

Dissolved.
\end{center}
\begin{flushright}
October 9, 1890.”\hspace*{3em}
\end{flushright}

Sherlock Holmes and I surveyed this curt announcement
and the rueful face behind it, until the comical side of the
affair so completely overtopped every other consideration
that we both burst out into a roar of laughter.

“I cannot see that there is anything very funny,” cried
our client, flushing up to the roots of his flaming head. “If
you can do nothing better than laugh at me, I can go elsewhere.”

“No, no,” cried Holmes, shoving him back into the chair
from which he had half risen. “I really wouldn’t miss your
case for the world. It is most refreshingly unusual. But
there is, if you will excuse my saying so, something just a
little funny about it. Pray what steps did you take when
you found the card upon the door?”

“I was staggered, sir. I did not know what to do. Then
I called at the offices round, but none of them seemed to
know anything about it. Finally, I went to the landlord, who
is an accountant living on the ground-floor, and I asked him
if he could tell me what had become of the Red-headed
League. He said that he had never heard of any such body.
Then I asked him who Mr.~Duncan Ross was. He answered
that the name was new to him.
%%049
%%“THE DOOR WAS SHUT AND LOCKED”
%%050

“\,‘Well,’ said I, ‘the gentleman at No. 4.’

“\,‘What, the red-headed man?’

“\,‘Yes.’

“\,‘Oh,’ said he, ‘his name was William Morris. He was a
solicitor, and was using my room as a temporary convenience
until his new premises were ready. He moved out yesterday.’

“\,‘Where could I find him?’

“\,‘Oh, at his new offices. He did tell me the address. Yes,
17 King Edward Street, near St.~Paul’s.’

“I started off, Mr.~Holmes, but when I got to that address
it was a manufactory of artificial knee-caps, and no one in it
had ever heard of either Mr.~William Morris or Mr.~Duncan
Ross.”

“And what did you do then?” asked Holmes.

“I went home to Saxe-Coburg Square, and I took the advice
of my assistant. But he could not help me in any way.
He could only say that if I waited I should hear by post.
But that was not quite good enough, Mr.~Holmes. I did not
wish to lose such a place without a struggle, so, as I had
heard that you were good enough to give advice to poor folk
who were in need of it, I came right away to you.”

“And you did very wisely,” said Holmes. “Your case is
an exceedingly remarkable one, and I shall be happy to look
into it. From what you have told me I think that it is possible
that graver issues hang from it than might at first sight
appear.”

“Grave enough!” said Mr.~Jabez Wilson. “Why, I have
lost four pound a week.”

“As far as you are personally concerned,” remark\-ed Hol\-mes,
“I do not see that you have any grievance against this extraordinary
league. On the contrary, you are, as I understand,
richer by some £30, to say nothing of the minute knowledge
which you have gained on every subject which comes
under the letter A. You have lost nothing by them.”

“No, sir. But I want to find out about them, and who
they are, and what their object was in playing this prank -- if
%%052
it was a prank -- upon me. It was a pretty expensive joke for
them, for it cost them two and thirty pounds.”

“We shall endeavor to clear up these points for you. And,
first, one or two questions, Mr.~Wilson. This assistant of
yours who first called your attention to the advertisement -- how
long had he been with you?”

“About a month then.”

“How did he come?”

“In answer to an advertisement.”

“Was he the only applicant?”

“No, I had a dozen.”

“Why did you pick him?”

“Because he was handy, and would come cheap.”

“At half-wages, in fact.”

“Yes.”

“What is he like, this Vincent Spaulding?”

“Small, stout-built, very quick in his ways, no hair on his
face, though he’s not short of thirty. Has a white splash of
acid upon his forehead.”

Holmes sat up in his chair in considerable excitement. “I
thought as much,” said he. “Have you ever observed that
his ears are pierced for earrings?”

“Yes, sir. He told me that a gypsy had done it for him
when he was a lad.”

“Hum!” said Holmes, sinking back in deep thought. “He
is still with you?”

“Oh yes, sir; I have only just left him.”

“And has your business been attended to in your absence?”

“Nothing to complain of, sir. There’s never very much to
do of a morning.”

“That will do, Mr.~Wilson. I shall be happy to give you
an opinion upon the subject in the course of a day or two.
To-day is Saturday, and I hope that by Monday we may
come to a conclusion.”

“Well, Watson,” said Holmes, when our visitor had left us,
“what do you make of it all?”
%%053

“I make nothing of it,” I answered, frankly. “It is a
most mysterious business.”

“As a rule,” said Holmes, “the more bizarre a thing is the
less mysterious it proves to be. It is your commonplace,
featureless crimes which are really puzzling, just as a commonplace
face is the most difficult to identify. But I must
be prompt over this matter.”

“What are you going to do, then?” I asked.

“To smoke,” he answered. “It is quite a three-pipe
problem, and I beg that you won’t speak to me for fifty minutes.”
He curled himself up in his chair, with his thin knees
drawn up to his hawk-like nose, and there he sat with his eyes
closed and his black clay pipe thrusting out like the bill of
some strange bird. I had come to the conclusion that he
had dropped asleep, and indeed was nodding myself, when he
suddenly sprang out of his chair with the gesture of a man
who has made up his mind, and put his pipe down upon the
mantel-piece.

“Sarasate plays at the St.~James’s Hall this afternoon,” he
remarked. “What do you think, Watson? Could your patients
spare you for a few hours?”

“I have nothing to do to-day. My practice is never very
absorbing.”

“Then put on your hat and come. I am going through
the city first, and we can have some lunch on the way. I
observe that there is a good deal of German music on the
programme, which is rather more to my taste than Italian or
French. It is introspective, and I want to introspect. Come
along!”

We travelled by the Underground as far as Aldersgate;
and a short walk took us to Saxe-Coburg Square, the scene
of the singular story which we had listened to in the morning.
It was a pokey, little, shabby-genteel place, where four
lines of dingy two-storied brick houses looked out into a
small railed-in enclosure, where a lawn of weedy grass and a
few clumps of faded laurel-bushes made a hard fight against
%%054
a smoke-laden and uncongenial atmosphere. Three gilt balls
and a brown board with “\textsc{Jabez Wilson}” in white letters,
upon a corner house, announced the place where our red-headed
client carried on his business. Sherlock Holmes
stopped in front of it with his head on one side, and looked
it all over, with his eyes shining brightly between puckered
lids. Then he walked slowly up the street, and then down
again to the corner, still looking keenly at the houses. Finally
he returned to the pawnbroker’s, and, having thumped vigorously
upon the pavement with his stick two or three times,
he went up to the door and knocked. It was instantly
opened by a bright-looking, clean-shaven young fellow, who
asked him to step in.

“Thank you,” said Holmes, “I only wished to ask you
how you would go from here to the Strand.”

“Third right, fourth left,” answered the assistant, promptly,
closing the door.

“Smart fellow, that,” observed Holmes, as we walked away.
“He is, in my judgment, the fourth smartest man in London,
and for daring I am not sure that he has not a claim to be
third. I have known something of him before.”

“Evidently,” said I, “Mr.~Wilson’s assistant counts for a
good deal in this mystery of the Red-headed League. I am
sure that you inquired your way merely in order that you
might see him.”

“Not him.”

“What then?”

“The knees of his trousers.”

“And what did you see?”

“What I expected to see.”

“Why did you beat the pavement?”

“My dear doctor, this is a time for observation, not for
talk. We are spies in an enemy’s country. We know something
of Saxe-Coburg Square. Let us now explore the parts
which lie behind it.”

The road in which we found ourselves as we turned round
%%055
the corner from the retired Saxe-Coburg Square presented as
great a contrast to it as the front of a picture does to the
back. It was one of the main arteries which convey the
traffic of the city to the north and west. The roadway was
blocked with the immense stream of commerce flowing in a
double tide inward and outward, while the foot-paths were
black with the hurrying swarm of pedestrians. It was difficult
to realize as we looked at the line of fine shops and
stately business premises that they really abutted on the
other side upon the faded and stagnant square which we had
just quitted.

“Let me see,” said Holmes, standing at the corner, and
glancing along the line, “I should like just to remember the
order of the houses here. It is a hobby of mine to have an
exact knowledge of London. There is Mortimer’s, the tobacconist,
the little newspaper shop, the Coburg branch of the
City and Suburban Bank, the Vegetarian Restaurant, and
McFarlane’s carriage-building depot. That carries us right
on to the other block. And now, doctor, we’ve done our
work, so it’s time we had some play. A sandwich and a cup
of coffee, and then off to violin-land, where all is sweetness
and delicacy and harmony, and there are no red-headed clients
to vex us with their conundrums.”

My friend was an enthusiastic musician, being himself not
only a very capable performer, but a composer of no ordinary
merit. All the afternoon he sat in the stalls wrapped in the
most perfect happiness, gently waving his long, thin fingers
in time to the music, while his gently smiling face and his languid,
dreamy eyes were as unlike those of Holmes, the sleuth-%
hound, Holmes the relentless, keen-witted, ready-handed criminal
agent, as it was possible to conceive. In his singular
character the dual nature alternately asserted itself, and his
extreme exactness and astuteness represented, as I have often
thought, the reaction against the poetic and contemplative
mood which occasionally predominated in him. The swing
of his nature took him from extreme languor to devouring
%%056
energy; and, as I knew well, he was never so truly formidable
as when, for days on end, he had been lounging in his
arm-chair amid his improvisations and his black-letter editions.
Then it was that the lust of the chase would suddenly
come upon him, and that his brilliant reasoning power would
rise to the level of intuition, until those who were unacquainted
with his methods would look askance at him as on a man
whose knowledge was not that of other mortals. When I
saw him that afternoon so enwrapped in the music at St.
James’s Hall I felt that an evil time might be coming upon
those whom he had set himself to hunt down.

“You want to go home, no doubt, doctor,” he remarked,
as we emerged.

“Yes, it would be as well.”

“And I have some business to do which will take some
hours. This business at Coburg Square is serious.”

“Why serious?”

“A considerable crime is in contemplation. I have every
reason to believe that we shall be in time to stop it. But to-%
day being Saturday rather complicates matters. I shall want
your help to-night.”

“At what time?”

“Ten will be early enough.”

“I shall be at Baker Street at ten.”

“Very well. And, I say, doctor, there may be some little
danger, so kindly put your army revolver in your pocket.”
He waved his hand, turned on his heel, and disappeared in
an instant among the crowd.

I trust that I am not more dense than my neighbors, but I
was always oppressed with a sense of my own stupidity in my
dealings with Sherlock Holmes. Here I had heard what he
had heard, I had seen what he had seen, and yet from his
words it was evident that he saw clearly not only what had
happened, but what was about to happen, while to me the
whole business was still confused and grotesque. As I drove
home to my house in Kensington I thought over it all, from
%%057
%%“ALL AFTERNOON HE SAT IN THE STALLS”
%%058
the extraordinary story of the red-headed copier of the
“Encyclopædia” down to the visit to Saxe-Coburg Square, and
the ominous words with which he had parted from me. What
was this nocturnal expedition, and why should I go armed?
Where were we going, and what were we to do? I had the
hint from Holmes that this smooth-faced pawnbroker’s assistant
was a formidable man -- a man who might play a
deep game. I tried to puzzle it out, but gave it up in
despair, and set the matter aside until night should bring an
explanation.

It was a quarter past nine when I started from home and
made my way across the Park, and so through Oxford Street
to Baker Street. Two hansoms were standing at the door,
and, as I entered the passage, I heard the sound of voices
from above. On entering his room I found Holmes in animated
conversation with two men, one of whom I recognized
as Peter Jones, the official police agent, while the other was
a long, thin, sad-faced man, with a very shiny hat and oppressively
respectable frock-coat.

“Ha! our party is complete,” said Holmes, buttoning up
his pea-jacket, and taking his heavy hunting crop from the
rack. “Watson, I think you know Mr.~Jones, of Scotland
Yard? Let me introduce you to Mr.~Merryweather, who is
to be our companion in to-night’s adventure.”

“We’re hunting in couples again, doctor, you see,” said
Jones, in his consequential way. “Our friend here is a wonderful
man for starting a chase. All he wants is an old dog
to help him to do the running down.”

“I hope a wild goose may not prove to be the end of our
chase,” observed Mr.~Merryweather, gloomily.

“You may place considerable confidence in Mr.~Holmes,
sir,” said the police agent, loftily. “He has his own little
methods, which are, if he won’t mind my saying so, just a little
too theoretical and fantastic, but he has the makings of a
detective in him. It is not too much to say that once or
twice, as in that business of the Sholto murder and the Agra
%%060
treasure, he has been more nearly correct than the official
force.”

“Oh, if you say so, Mr.~Jones, it is all right,” said the
stranger, with deference. “Still, I confess that I miss my
rubber. It is the first Saturday night for seven-and-twenty
years that I have not had my rubber.”

“I think you will find,” said Sherlock Holmes, “that you
will play for a higher stake to-night than you have ever done
yet, and that the play will be more exciting. For you, Mr.
Merryweather, the stake will be some £30,000; and for you,
Jones, it will be the man upon whom you wish to lay your
hands.”

“John Clay, the murderer, thief, smasher, and forger. He’s
a young man, Mr.~Merryweather, but he is at the head of his
profession, and I would rather have my bracelets on him than
on any criminal in London. He’s a remarkable man, is young
John Clay. His grandfather was a royal duke, and he himself
has been to Eton and Oxford. His brain is as cunning
as his fingers, and though we meet signs of him at every turn,
we never know where to find the man himself. He’ll crack a
crib in Scotland one week, and be raising money to build an
orphanage in Cornwall the next. I’ve been on his track for
years, and have never set eyes on him yet.”

“I hope that I may have the pleasure of introducing you
to-night. I’ve had one or two little turns also with Mr.~John
Clay, and I agree with you that he is at the head of his profession.
It is past ten, however, and quite time that we started.
If you two will take the first hansom, Watson and I will
follow in the second.”

Sherlock Holmes was not very communicative during the
long drive, and lay back in the cab humming the tunes which
he had heard in the afternoon. We rattled through an endless
labyrinth of gas-lit streets until we emerged into
Farringdon Street.

“We are close there now,” my friend remarked. “This fellow
Merryweather is a bank director, and personally interested
%%061
in the matter. I thought it as well to have Jones with us
also. He is not a bad fellow, though an absolute imbecile in
his profession. He has one positive virtue. He is as brave
as a bull-dog, and as tenacious as a lobster if he gets his
claws upon any one. Here we are, and they are waiting for
us.”

We had reached the same crowded thoroughfare in which
we had found ourselves in the morning. Our cabs were dismissed,
and, following the guidance of Mr.~Merryweather, we
passed down a narrow passage and through a side door, which
he opened for us. Within there was a small corridor, which
ended in a very massive iron gate. This also was opened,
and led down a flight of winding stone steps, which terminated
at another formidable gate. Mr.~Merryweather stopped
to light a lantern, and then conducted us down a dark, earth-%
smelling passage, and so, after opening a third door, into a
huge vault or cellar, which was piled all round with crates and
massive boxes.

“You are not very vulnerable from above,” Holmes
remarked, as he held up the lantern and gazed about him.

“Nor from below,” said Mr.~Merryweather, striking his
stick upon the flags which lined the floor. “Why, dear me,
it sounds quite hollow!” he remarked, looking up in surprise.

“I must really ask you to be a little more quiet,” said
Holmes, severely. “You have already imperilled the whole
success of our expedition. Might I beg that you would have
the goodness to sit down upon one of those boxes, and not
to interfere?”

The solemn Mr.~Merryweather perched himself upon a
crate, with a very injured expression upon his face, while
Holmes fell upon his knees upon the floor, and, with the lantern
and a magnifying lens, began to examine minutely the
cracks between the stones. A few seconds sufficed to satisfy
him, for he sprang to his feet again, and put his glass in his
pocket.

“We have at least an hour before us,” he remarked; “for
%%062
they can hardly take any steps until the good pawnbroker is
safely in bed. Then they will not lose a minute, for the
sooner they do their work the longer time they will have for
their escape. We are at present, doctor -- as no doubt you
have divined -- in the cellar of the city branch of one of the
principal London banks. Mr.~Merryweather is the chairman
of directors, and he will explain to you that there are reasons
why the more daring criminals of London should take a
considerable interest in this cellar at present.”

“It is our French gold,” whispered the director. “We
have had several warnings that an attempt might be made
upon it.”

“Your French gold?”

“Yes. We had occasion some months ago to strengthen
our resources, and borrowed, for that purpose, 30,000
nap\-ol\-eons from the Bank of France. It has become known that
we have never had occasion to unpack the money, and that
it is still lying in our cellar. The crate upon which I sit
contains 2000 napoleons packed between layers of lead foil.
Our reserve of bullion is much larger at present than is usually
kept in a single branch office, and the directors have had
misgivings upon the subject.”

“Which were very well justified,” observed Holmes. “And
now it is time that we arranged our little plans. I expect
that within an hour matters will come to a head. In the
mean time, Mr.~Merryweather, we must put the screen over
that dark lantern.”

“And sit in the dark?”

“I am afraid so. I had brought a pack of cards in my
pocket, and I thought that, as we were a \textit{partie carrée}, you
might have your rubber after all. But I see that the enemy’s
preparations have gone so far that we cannot risk the presence
of a light. And, first of all, we must choose our positions.
These are daring men, and though we shall take them
at a disadvantage, they may do us some harm unless we are
careful. I shall stand behind this crate, and do you conceal
%%063
yourselves behind those. Then, when I flash a light upon
them, close in swiftly. If they fire, Watson, have no compunction
about shooting them down.”

I placed my revolver, cocked, upon the top of the wooden
case behind which I crouched. Holmes shot the slide across
the front of his lantern, and left us in pitch darkness -- such
an absolute darkness as I have never before experienced.
The smell of hot metal remained to assure us that the light
was still there, ready to flash out at a moment’s notice. To
me, with my nerves worked up to a pitch of expectancy, there
was something depressing and subduing in the sudden gloom,
and in the cold, dank air of the vault.

“They have but one retreat,” whispered Holmes. “That
is back through the house into Saxe-Coburg Square. I hope
that you have done what I asked you, Jones?”

“I have an inspector and two officers waiting at the front
door.”

“Then we have stopped all the holes. And now we must
be silent and wait.”

What a time it seemed! From comparing notes afterwards
it was but an hour and a quarter, yet it appeared to me that
the night must have almost gone, and the dawn be breaking
above us. My limbs were weary and stiff, for I feared to
change my position; yet my nerves were worked up to the
highest pitch of tension, and my hearing was so acute that I
could not only hear the gentle breathing of my companions,
but I could distinguish the deeper, heavier in-breath of the
bulky Jones from the thin, sighing note of the bank director.
From my position I could look over the case in the direction
of the floor. Suddenly my eyes caught the glint of a light.

At first it was but a lurid spark upon the stone pavement.
Then it lengthened out until it became a yellow line, and
then, without any warning or sound, a gash seemed to open
and a hand appeared; a white, almost womanly hand, which
felt about in the centre of the little area of light. For a minute
or more the hand, with its writhing fingers, protruded out
%%064
of the floor. Then it was withdrawn as suddenly as it appeared,
and all was dark again save the single lurid spark
which marked a chink between the stones.

Its disappearance, however, was but momentary. With a
rending, tearing sound, one of the broad, white stones turned
over upon its side, and left a square, gaping hole, through
which streamed the light of a lantern. Over the edge there
peeped a clean-cut, boyish face, which looked keenly about it,
and then, with a hand on either side of the aperture, drew itself
shoulder-high and waist-high, until one knee rested upon
the edge. In another instant he stood at the side of the
hole, and was hauling after him a companion, lithe and small
like himself, with a pale face and a shock of very red hair.

“It’s all clear,” he whispered. “Have you the chisel and
the bags. Great Scott! Jump, Archie, jump, and I’ll swing
for it!”

Sherlock Holmes had sprung out and seized the intruder
by the collar. The other dived down the hole, and I heard
the sound of rending cloth as Jones clutched at his skirts.
The light flashed upon the barrel of a revolver, but Holmes’s
hunting crop came down on the man’s wrist, and the pistol
clinked upon the stone floor.

“It’s no use, John Clay,” said Holmes, blandly. “You have
no chance at all.”

“So I see,” the other answered, with the utmost coolness.
“I fancy that my pal is all right, though I see you have got
his coat-tails.”

“There are three men waiting for him at the door,” said
Holmes.

“Oh, indeed! You seem to have done the thing very
completely. I must compliment you.”

“And I you,” Holmes answered. “Your red-headed idea
was very new and effective.”

“You’ll see your pal again presently,” said Jones. “He’s
quicker at climbing down holes than I am. Just hold out
while I fix the derbies.”
%%065

“I beg that you will not touch me with your filthy hands,”
remarked our prisoner, as the handcuffs clattered upon his
wrists. “You may not be aware that I have royal blood in
my veins. Have the goodness, also, when you address me always
to say ‘sir’ and ‘please.’\,”

“All right,” said Jones, with a stare and a snigger. “Well,
would you please, sir, march up-stairs, where we can get a cab
to carry your highness to the police-station?”

“That is better,” said John Clay, serenely. He made a
sweeping bow to the three of us, and walked quietly off in the
custody of the detective.

“Really Mr.~Holmes,” said Mr.~Merryweather, as we followed
them from the cellar, “I do not know how the bank can
thank you or repay you. There is no doubt that you have
detected and defeated in the most complete manner one of
the most determined attempts at bank robbery that have ever
come within my experience.”

“I have had one or two little scores of my own to settle
with Mr.~John Clay,” said Holmes. “I have been at some
small expense over this matter, which I shall expect the bank
to refund, but beyond that I am amply repaid by having had
an experience which is in many ways unique, and by hearing
the very remarkable narrative of the Red-headed League.”

\strut

“You see, Watson,” he explained, in the early hours of the
morning, as we sat over a glass of whiskey-and-soda in Baker
Street, “it was perfectly obvious from the first that the only
possible object of this rather fantastic business of the
advertisement of the League, and the copying of the ‘Encyclopædia,’
must be to get this not over-bright pawnbroker out of the
way for a number of hours every day. It was a curious way
of managing it, but, really, it would be difficult to suggest a
better. The method was no doubt suggested to Clay’s ingenious
mind by the color of his accomplice’s hair. The £4 a
week was a lure which must draw him, and what was it to
them, who were playing for thousands? They put in the
%%066
advertisement, one rogue has the temporary office, the other
rogue incites the man to apply for it, and together they manage
to secure his absence every morning in the week. From
the time that I heard of the assistant having come for half
wages, it was obvious to me that he had some strong motive
for securing the situation.”

“But how could you guess what the motive was?”

“Had there been women in the house, I should have suspected
a mere vulgar intrigue. That, however, was out of the
question. The man’s business was a small one, and there
was nothing in his house which could account for such elaborate
preparations, and such an expenditure as they were at.
It must, then, be something out of the house. What could it
be? I thought of the assistant’s fondness for photography,
and his trick of vanishing into the cellar. The cellar! There
was the end of this tangled clue. Then I made inquiries as
to this mysterious assistant, and found that I had to deal with
one of the coolest and most daring criminals in London. He
was doing something in the cellar -- something which took
many hours a day for months on end. What could it be, once
more? I could think of nothing save that he was running a
tunnel to some other building.

“So far I had got when we went to visit the scene of
action. I surprised you by beating upon the pavement with
my stick. I was ascertaining whether the cellar stretched out
in front or behind. It was not in front. Then I rang the bell,
and, as I hoped, the assistant answered it. We have had
some skirmishes, but we had never set eyes upon each other
before. I hardly looked at his face. His knees were what I
wished to see. You must yourself have remarked how worn,
wrinkled, and stained they were. They spoke of those hours
of burrowing. The only remaining point was what they were
burrowing for. I walked round the corner, saw that the City
and Suburban Bank abutted on our friend’s premises, and felt
that I had solved my problem. When you drove home after
the concert I called upon Scotland Yard, and upon the
%%067
chairman of the bank directors, with the result that you have
seen.”

“And how could you tell that they would make their attempt
to-night?” I asked.

“Well, when they closed their League offices that was a
sign that they cared no longer about Mr.~Jabez Wilson’s
presence -- in other words, that they had completed their tunnel.
But it was essential that they should use it soon, as it might
be discovered, or the bullion might be removed. Saturday
would suit them better than any other day, as it would give
them two days for their escape. For all these reasons I expected
them to come to-night.”

“You reasoned it out beautifully,” I exclaimed, in unfeigned
admiration. “It is so long a chain, and yet every link rings
true.”

“It saved me from ennui,” he answered, yawning. “Alas!
I already feel it closing in upon me. My life is spent in one
long effort to escape from the commonplaces of existence.
These little problems help me to do so.”

“And you are a benefactor of the race,” said I.

He shrugged his shoulders. “Well, perhaps, after all, it is
of some little use,” he remarked. “\,‘L’homme c’est rien --
l’œuvre c’est tout,’ as Gustave Flaubert wrote to Georges
Sand.”
%%068

\Chapter{A Case Of Identity}

“\textsc{My} dear fellow,” said Sherlock Holmes, as we sat on
either side of the fire in his lodgings at Baker
Street, “life is infinitely stranger than anything
which the mind of man could invent. We would
not dare to conceive the things which are really mere commonplaces
of existence. If we could fly out of that window
hand in hand, hover over this great city, gently remove the
roofs, and peep in at the queer things which are going on, the
strange coincidences, the plannings, the cross-purposes, the
wonderful chains of events, working through generations, and
leading to the most \textit{outré} results, it would make all fiction
with its conventionalities and foreseen conclusions most stale
and unprofitable.”

“And yet I am not convinced of it,” I answered. “The
cases which come to light in the papers are, as a rule, bald
enough, and vulgar enough. We have in our police reports
realism pushed to its extreme limits, and yet the result is, it
must be confessed, neither fascinating nor artistic.”

“A certain selection and discretion must be used in producing
a realistic effect,” remarked Holmes. “This is wanting
in the police report, where more stress is laid, perhaps,
upon the platitudes of the magistrate than upon the details,
which to an observer contain the vital essence of the whole
matter. Depend upon it there is nothing so unnatural as the
commonplace.”

I smiled and shook my head. “I can quite understand you
thinking so,” I said. “Of course, in your position of
%%069
unofficial adviser and helper to everybody who is absolutely
puzzled, throughout three continents, you are brought in contact
with all that is strange and bizarre. But here” -- I picked
up the morning paper from the ground -- “let us put it to a
practical test. Here is the first heading upon which I come.
‘A husband’s cruelty to his wife.’ There is half a column
of print, but I know without reading it that it is all perfectly
familiar to me. There is, of course, the other woman, the
drink, the push, the blow, the bruise, the sympathetic sister
or landlady. The crudest of writers could invent nothing
more crude.”

“Indeed, your example is an unfortunate one for your argument,”
said Holmes, taking the paper and glancing his eye
down it. “This is the Dundas separation case, and, as it
happens, I was engaged in clearing up some small points in
connection with it. The husband was a teetotaler, there was no
other woman, and the conduct complained of was that he had
drifted into the habit of winding up every meal by taking out
his false teeth and hurling them at his wife, which, you will
allow, is not an action likely to occur to the imagination of the
average story-teller. Take a pinch of snuff, doctor, and
acknowledge that I have scored over you in your example.”

He held out his snuffbox of old gold, with a great amethyst
in the centre of the lid. Its splendor was in such contrast to
his homely ways and simple life that I could not help commenting
upon it.

“Ah,” said he, “I forgot that I had not seen you for some
weeks. It is a little souvenir from the King of Bohemia in
return for my assistance in the case of the Irene Adler
papers.”

“And the ring?” I asked, glancing at a remarkable brilliant
which sparkled upon his finger.

“It was from the reigning family of Holland, though the
matter in which I served them was of such delicacy that I
cannot confide it even to you, who have been good enough to
chronicle one or two of my little problems.”
%%070

“And have you any on hand just now?” I asked, with
interest.

“Some ten or twelve, but none which present any feature
of interest. They are important, you understand, without being
interesting. Indeed, I have found that it is usually in unimportant
matters that there is a field for the observation,
and for the quick analysis of cause and effect which gives the
charm to an investigation. The larger crimes are apt to be
the simpler, for the bigger the crime, the more obvious, as a
rule, is the motive. In these cases, save for one rather intricate
matter which has been referred to me from Marseilles,
there is nothing which presents any features of interest. It is
possible, however, that I may have something better before
very many minutes are over, for this is one of my clients, or I
am much mistaken.”

He had risen from his chair, and was standing between the
parted blinds, gazing down into the dull, neutral-tinted London
street. Looking over his shoulder, I saw that on the pavement
opposite there stood a large woman with a heavy fur
boa round her neck, and a large curling red feather in a broad-%
brimmed hat which was tilted in a coquettish Duchess-of-%
Devonshire fashion over her ear. From under this great
panoply she peeped up in a nervous, hesitating fashion at our
windows, while her body oscillated backward and forward,
and her fingers fidgeted with her glove buttons. Suddenly,
with a plunge, as of the swimmer who leaves the bank, she
hurried across the road, and we heard the sharp clang of the
bell.

“I have seen those symptoms before,” said Holmes, throwing
his cigarette into the fire. “Oscillation upon the pavement
always means an \textit{affaire de cœur}. She would like advice, but
is not sure that the matter is not too delicate for communication.
And yet even here we may discriminate. When a
woman has been seriously wronged by a man she no longer
oscillates, and the usual symptom is a broken bell wire. Here
we may take it that there is a love matter, but that the maiden
%%071
is not so much angry as perplexed, or grieved. But here she
comes in person to resolve our doubts.”

As he spoke there was a tap at the door, and the boy in
buttons entered to announce Miss Mary Sutherland, while the
lady herself loomed behind his small black figure like a full-%
sailed merchant-man behind a tiny pilot boat. Sherlock
Holmes welcomed her with the easy courtesy for which he was
remarkable, and having closed the door, and bowed her into
an arm-chair, he looked her over in the minute, and yet abstracted
fashion which was peculiar to him.

“Do you not find,” he said, “that with your short sight it
is a little trying to do so much type-writing?”

“I did at first,” she answered, “but now I know where the
letters are without looking.” Then, suddenly realizing the
full purport of his words, she gave a violent start and looked
up, with fear and astonishment upon her broad, good-humored
face. “You’ve heard about me, Mr.~Holmes,” she cried, “else
how could you know all that?”

“Never mind,” said Holmes, laughing; “it is my business
to know things. Perhaps I have trained myself to see what
others overlook. If not, why should you come to consult me?”

“I came to you, sir, because I heard of you from Mrs.
Etherege, whose husband you found so easy when the police
and every one had given him up for dead. Oh, Mr.~Holmes,
I wish you would do as much for me. I’m not rich, but still
I have a hundred a year in my own right, besides the little
that I make by the machine, and I would give it all to know
what has become of Mr.~Hosmer Angel.”

“Why did you come away to consult me in such a hurry?”
asked Sherlock Holmes, with his finger-tips together, and his
eyes to the ceiling.

Again a startled look came over the somewhat vacuous face
of Miss Mary Sutherland. “Yes, I did bang out of the house,”
she said, “for it made me angry to see the easy way in which
Mr.~Windibank -- that is, my father -- took it all. He would
not go to the police, and he would not go to you, and so at
%%072
last, as he would do nothing, and kept on saying that there
was no harm done, it made me mad, and I just on with my
things and came right away to you.”

“Your father,” said Holmes, “your step-father, surely, since
the name is different.”

“Yes, my step-father. I call him father, though it sounds
funny, too, for he is only five years and two months older
than myself.”

“And your mother is alive?”

“Oh yes, mother is alive and well. I wasn’t best pleased,
Mr.~Holmes, when she married again so soon after father’s
death, and a man who was nearly fifteen years younger than
herself. Father was a plumber in the Tottenham Court Road,
and he left a tidy business behind him, which mother carried
on with Mr.~Hardy, the foreman; but when Mr.~Windibank
came he made her sell the business, for he was very superior,
being a traveller in wines. They got £4700 for the
goodwill and interest, which wasn’t near as much as father could
have got if he had been alive.”

I had expected to see Sherlock Holmes impatient under this
rambling and inconsequential narrative, but, on the contrary,
he had listened with the greatest concentration of attention.

“Your own little income,” he asked, “does it come out of
the business?”

“Oh no, sir. It is quite separate, and was left me by my
Uncle Ned in Auckland. It is in New Zealand stock, paying
$\mathsf{4\frac{1}{2}}$ per cent. Two thousand five hundred pounds was the
amount, but I can only touch the interest.”

“You interest me extremely,” said Holmes. “And since
you draw so large a sum as a hundred a year, with what you
earn into the bargain, you no doubt travel a little, and indulge
yourself in every way. I believe that a single lady can get on
very nicely upon an income of about £60.”

“I could do with much less than that, Mr.~Holmes, but you
understand that as long as I live at home I don’t wish to be
a burden to them, and so they have the use of the money just
%%073
%%“SHERLOCK HOLMES WELCOMED HER”
%%074
while I am staying with them. Of course, that is only just for
the time. Mr.~Windibank draws my interest every quarter,
and pays it over to mother, and I find that I can do pretty
well with what I earn at type-writing. It brings me twopence
a sheet, and I can often do from fifteen to twenty sheets in a
day.”

“You have made your position very clear to me,” said
Holmes. “This is my friend, Dr. Watson, before whom you
can speak as freely as before myself. Kindly tell us now all
about your connection with Mr.~Hosmer Angel.”

A flush stole over Miss Sutherland’s face, and she picked
nervously at the fringe of her jacket. “I met him first at the
gasfitters’ ball,” she said. “They used to send father tickets
when he was alive, and then afterwards they remembered us,
and sent them to mother. Mr.~Windibank did not wish us to
go. He never did wish us to go anywhere. He would get quite
mad if I wanted so much as to join a Sunday-school treat.
But this time I was set on going, and I would go; for what
right had he to prevent? He said the folk were not fit for us
to know, when all father’s friends were to be there. And he
said that I had nothing fit to wear, when I had my purple
plush that I had never so much as taken out of the drawer.
At last, when nothing else would do, he went off to France
upon the business of the firm, but we went, mother and I, with
Mr.~Hardy, who used to be our foreman, and it was there I
met Mr.~Hosmer Angel.”

“I suppose,” said Holmes, “that when Mr.~Wind\-ibank came
back from France he was very annoyed at your having gone
to the ball.”

“Oh, well, he was very good about it. He laughed, I remember,
and shrugged his shoulders, and said there was no
use denying anything to a woman, for she would have her
way.”

“I see. Then at the gasfitters’ ball you met, as I understand,
a gentleman called Mr.~Hosmer Angel.”

“Yes, sir. I met him that night, and he called next day to
%%076
ask if we had got home all safe, and after that we met him --
that is to say, Mr.~Holmes, I met him twice for walks, but
after that father came back again, and Mr.~Hosmer Angel
could not come to the house any more.”

“No?”

“Well, you know, father didn’t like anything of the sort.
He wouldn’t have any visitors if he could help it, and he used
to say that a woman should be happy in her own family circle.
But then, as I used to say to mother, a woman wants her own
circle to begin with, and I had not got mine yet.”

“But how about Mr.~Hosmer Angel? Did he make no attempt
to see you?”

“Well, father was going off to France again in a week, and
Hosmer wrote and said that it would be safer and better not
to see each other until he had gone. We could write in the
mean time, and he used to write every day. I took the letters
in in the morning, so there was no need for father to
know.”

“Were you engaged to the gentleman at this time?”

“Oh yes, Mr.~Holmes. We were engaged after the first
walk that we took. Hosmer -- Mr.~Angel -- was a cashier in
an office in Leadenhall Street -- and -- ”

“What office?”

“That’s the worst of it, Mr.~Holmes, I don’t know.”

“Where did he live, then?”

“He slept on the premises.”

“And you don’t know his address?”

“No -- except that it was Leadenhall Street.”

“Where did you address your letters, then?”

“To the Leadenhall Street Post-office, to be left till called
for. He said that if they were sent to the office he would be
chaffed by all the other clerks about having letters from a
lady, so I offered to type-write them, like he did his, but he
wouldn’t have that, for he said that when I wrote them they
seemed to come from me, but when they were type-written he
always felt that the machine had come between us. That
%%077
will just show you how fond he was of me, Mr.~Holmes, and
the little things that he would think of.”

“It was most suggestive,” said Holmes. “It has long been
an axiom of mine that the little things are infinitely the most
important. Can you remember any other little things about
Mr.~Hosmer Angel?”

“He was a very shy man, Mr.~Holmes. He would rather
walk with me in the evening than in the daylight, for he said
that he hated to be conspicuous. Very retiring and gentlemanly
he was. Even his voice was gentle. He’d had the
quinsy and swollen glands when he was young, he told me,
and it had left him with a weak throat, and a hesitating,
whispering fashion of speech. He was always well dressed,
very neat and plain, but his eyes were weak, just as mine are,
and he wore tinted glasses against the glare.”

“Well, and what happened when Mr.~Windibank, your step-father,
returned to France?”

“Mr.~Hosmer Angel came to the house again, and proposed
that we should marry before father came back. He was in
dreadful earnest, and made me swear, with my hands on the
Testament, that whatever happened I would always be true
to him. Mother said he was quite right to make me swear,
and that it was a sign of his passion. Mother was all in his
favor from the first, and was even fonder of him than I was.
Then, when they talked of marrying within the week, I began
to ask about father; but they both said never to mind about
father, but just to tell him afterwards, and mother said she
would make it all right with him. I didn’t quite like that, Mr.
Holmes. It seemed funny that I should ask his leave, as he
was only a few years older than me; but I didn’t want to do
anything on the sly, so I wrote to father at Bordeaux, where
the company has its French offices, but the letter came back
to me on the very morning of the wedding.”

“It missed him, then?”

“Yes, sir; for he had started to England just before it arrived.”
%%078

“Ha! that was unfortunate. Your wedding was arranged,
then, for the Friday. Was it to be in church?”

“Yes, sir, but very quietly. It was to be at St.~Saviour’s,
near King’s Cross, and we were to have breakfast afterwards
at the St.~Pancras Hotel. Hosmer came for us in a hansom,
but as there were two of us, he put us both into it, and stepped
himself into a four-wheeler, which happened to be the only
other cab in the street. We got to the church first, and when
the four-wheeler drove up we waited for him to step out, but
he never did, and when the cabman got down from the box
and looked, there was no one there! The cabman said that
he could not imagine what had become of him, for he had
seen him get in with his own eyes. That was last Friday, Mr.
Holmes, and I have never seen or heard anything since then
to throw any light upon what became of him.”

“It seems to me that you have been very shamefully treated,”
said Holmes.

“Oh no, sir! He was too good and kind to leave me so.
Why, all the morning he was saying to me that, whatever
happened, I was to be true; and that even if something quite
unforeseen occurred to separate us, I was always to remember
that I was pledged to him, and that he would claim his pledge
sooner or later. It seemed strange talk for a wedding-morning,
but what has happened since gives a meaning to it.”

“Most certainly it does. Your own opinion is, then, that
some unforeseen catastrophe has occurred to him?”

“Yes, sir. I believe that he foresaw some danger, or else he
would not have talked so. And then I think that what he
foresaw happened.”

“But you have no notion as to what it could have been?”

“None.”

“One more question. How did your mother take the
matter?”

“She was angry, and said that I was never to speak of the
matter again.”

“And your father? Did you tell him?”
%%079

“Yes; and he seemed to think, with me, that something had
happened, and that I should hear of Hosmer again. As he
said, what interest could any one have in bringing me to the
doors of the church, and then leaving me? Now, if he had
borrowed my money, or if he had married me and got my money
settled on him, there might be some reason; but Hosmer
was very independent about money, and never would look at
a shilling of mine. And yet, what could have happened? And
why could he not write? Oh, it drives me half-mad to think
of! and I can’t sleep a wink at night.” She pulled a little
handkerchief out of her muff, and began to sob heavily into it.

“I shall glance into the case for you,” said Holmes, rising;
“and I have no doubt that we shall reach some definite result.
Let the weight of the matter rest upon me now, and do
not let your mind dwell upon it further. Above all, try to let
Mr.~Hosmer Angel vanish from your memory, as he has done
from your life.”

“Then you don’t think I’ll see him again?”

“I fear not.”

“Then what has happened to him?”

“You will leave that question in my hands. I should like
an accurate description of him, and any letters of his which
you can spare.”

“I advertised for him in last Saturday’s \textit{Chronicle},” said
she. “Here is the slip, and here are four letters from him.”

“Thank you. And your address?”

“No. 31 Lyon Place, Camberwell.”

“Mr.~Angel’s address you never had, I understand. Where
is your father’s place of business?”

“He travels for Westhouse \& Marbank, the great claret importers
of Fenchurch Street.”

“Thank you. You have made your statement very clearly.
You will leave the papers here, and remember the advice
which I have given you. Let the whole incident be a sealed
book, and do not allow it to affect your life.”

“You are very kind, Mr.~Holmes, but I cannot do that. I
%%080
shall be true to Hosmer. He shall find me ready when he
comes back.”

For all the preposterous hat and the vacuous face, there was
something noble in the simple faith of our visitor which compelled
our respect. She laid her little bundle of papers upon
the table, and went her way, with a promise to come again
whenever she might be summoned.

Sherlock Holmes sat silent for a few minutes with his fingertips
still pressed together, his legs stretched out in front of
him, and his gaze directed upward to the ceiling. Then he
took down from the rack the old and oily clay pipe, which was
to him as a counsellor, and, having lit it, he leaned back in
his chair, with the thick blue cloud-wreaths spinning up from
him, and a look of infinite languor in his face.

“Quite an interesting study, that maiden,” he observed. “I
found her more interesting than her little problem, which, by
the way, is rather a trite one. You will find parallel cases, if
you consult my index, in Andover in ’77, and there was something
of the sort at The Hague last year. Old as is the idea,
however, there were one or two details which were new to me.
But the maiden herself was most instructive.”

“You appeared to read a good deal upon her which was
quite invisible to me,” I remarked.

“Not invisible, but unnoticed, Watson. You did not know
where to look, and so you missed all that was important. I
can never bring you to realize the importance of sleeves, the
suggestiveness of thumb-nails, or the great issues that may
hang from a boot-lace. Now, what did you gather from that
woman’s appearance? Describe it.”

“Well, she had a slate-colored, broad-brimmed straw hat,
with a feather of a brickish red. Her jacket was black, with
black beads sewn upon it, and a fringe of little black jet
ornaments. Her dress was brown, rather darker than coffee color,
with a little purple plush at the neck and sleeves. Her
gloves were grayish, and were worn through at the right
forefinger. Her boots I didn’t observe. She had small, round,
%%081
hanging gold ear-rings, and a general air of being fairly well-%
to-do, in a vulgar, comfortable, easy-going way.”

Sherlock Holmes clapped his hands softly together and
chuckled.

“’Pon my word, Watson, you are coming along wonderfully.
You have really done very well indeed. It is true that you
have missed everything of importance, but you have hit upon
the method, and you have a quick eye for color. Never trust
to general impressions, my boy, but concentrate yourself upon
details. My first glance is always at a woman’s sleeve. In a
man it is perhaps better first to take the knee of the trouser.
As you observe, this woman had plush upon her sleeves,
which is a most useful material for showing traces. The
double line a little above the wrist, where the type-writist
presses against the table, was beautifully defined. The sewing-%
machine, of the hand type, leaves a similar mark, but only
on the left arm, and on the side of it farthest from the thumb,
instead of being right across the broadest part, as this was.
I then glanced at her face, and observing the dint of a pince-nez
at either side of her nose, I ventured a remark upon short
sight and type-writing, which seemed to surprise her.”

“It surprised me.”

“But, surely, it was very obvious. I was then much surprised
and interested on glancing down to observe that,
though the boots which she was wearing were not unlike each
other, they were really odd ones; the one having a slightly
decorated toe-cap, and the other a plain one. One was buttoned
only in the two lower buttons out of five, and the other
at the first, third, and fifth. Now, when you see that a young
lady, otherwise neatly dressed, has come away from home
with odd boots, half-buttoned, it is no great deduction to say
that she came away in a hurry.”

“And what else?” I asked, keenly interested, as I always
was, by my friend’s incisive reasoning.

“I noted, in passing, that she had written a note before
leaving home, but after being fully dressed. You observed
%%082
that her right glove was torn at the forefinger, but you did not
apparently see that both glove and finger were stained with
violet ink. She had written in a hurry, and dipped her pen
too deep. It must have been this morning, or the mark would
not remain clear upon the finger. All this is amusing, though
rather elementary, but I must go back to business, Watson.
Would you mind reading me the advertised description of Mr.
Hosmer Angel?”

I held the little printed slip to the light. “Missing,” it
said, “on the morning of the 14th, a gentleman named Hosmer
Angel. About 5 ft. 7 in. in height; strongly built, sallow
complexion, black hair, a little bald in the centre, bushy,
black side-whiskers and mustache; tinted glasses, slight infirmity
of speech. Was dressed, when last seen, in black
frock-coat faced with silk, black waistcoat, gold Albert chain,
and gray Harris tweed trousers, with brown gaiters over elastic-%
sided boots. Known to have been employed in an office in
Leadenhall Street. Anybody bringing,” etc., etc.

“That will do,” said Holmes. “As to the letters,” he continued,
glancing over them, “they are very commonplace.
Absolutely no clew in them to Mr.~Angel, save that he quotes
Balzac once. There is one remarkable point, however, which
will no doubt strike you.”

“They are type-written,” I remarked.

“Not only that, but the signature is type-written. Look at
the neat little ‘Hosmer Angel’ at the bottom. There is a
date, you see, but no superscription except Leadenhall Street,
which is rather vague. The point about the signature is very
suggestive -- in fact, we may call it conclusive.”

“Of what?”

“My dear fellow, is it possible you do not see how strongly
it bears upon the case?”

“I cannot say that I do, unless it were that he wished to
be able to deny his signature if an action for breach of
promise were instituted.”

“No, that was not the point. However, I shall write two
%%083
letters, which should settle the matter. One is to a firm in
the city, the other is to the young lady’s step-father, Mr.
Windibank, asking him whether he could meet us here at six
o’clock to-morrow evening. It is just as well that we should
do business with the male relatives. And now, doctor, we
can do nothing until the answers to those letters come, so we
may put our little problem upon the shelf for the interim.”

I had had so many reasons to believe in my friend’s subtle
powers of reasoning, and extraordinary energy in action, that
I felt that he must have some solid grounds for the assured
and easy demeanor with which he treated the singular mystery
which he had been called upon to fathom. Once only had
I known him to fail, in the case of the King of Bohemia and
of the Irene Adler photograph; but when I looked back to the
weird business of the Sign of Four, and the extraordinary
circumstances connected with the Study in Scarlet, I felt that it
would be a strange tangle indeed which he could not unravel.

I left him then, still puffing at his black clay pipe, with the
conviction that when I came again on the next evening I would
find that he held in his hands all the clews which would lead
up to the identity of the disappearing bridegroom of Miss
Mary Sutherland.

A professional case of great gravity was engaging my own
attention at the time, and the whole of next day I was busy
at the bedside of the sufferer. It was not until close upon
six o’clock that I found myself free, and was able to spring
into a hansom and drive to Baker Street, half afraid that I
might be too late to assist at the \textit{dénouement} of the little
mystery. I found Sherlock Holmes alone, however, half asleep,
with his long, thin form curled up in the recesses of his
armchair. A formidable array of bottles and test-tubes, with the
pungent cleanly smell of hydrochloric acid, told me that he
had spent his day in the chemical work which was so dear to
him.

“Well, have you solved it?” I asked, as I entered.

“Yes. It was the bisulphate of baryta.”
%%084

“No, no, the mystery!” I cried.

“Oh, that! I thought of the salt that I have been working
upon. There was never any mystery in the matter, though,
as I said yesterday, some of the details are of interest. The
only drawback is that there is no law, I fear, that can touch
the scoundrel.”

“Who was he, then, and what was his object in deserting
Miss Sutherland?”

The question was hardly out of my mouth, and Holmes
had not yet opened his lips to reply, when we heard a heavy
footfall in the passage, and a tap at the door.

“This is the girl’s step-father, Mr.~James Windi\-bank,” said
Holmes. “He has written to me to say that he would be
here at six. Come in!”

The man who entered was a sturdy, middle-sized fellow,
some thirty years of age, clean shaven, and sallow skinned,
with a bland, insinuating manner, and a pair of wonderfully
sharp and penetrating gray eyes. He shot a questioning
glance at each of us, placed his shiny top hat upon the
sideboard, and with a slight bow sidled down into the nearest
chair.

“Good-evening, Mr.~James Windibank,” said Holmes. “I
think that this type-written letter is from you, in which you
made an appointment with me for six o’clock?”

“Yes, sir. I am afraid that I am a little late, but I am not
quite my own master, you know. I am sorry that Miss
Sutherland has troubled you about this little matter, for I
think it is far better not to wash linen of the sort in public.
It was quite against my wishes that she came, but she is a
very excitable, impulsive girl, as you may have noticed, and
she is not easily controlled when she has made up her mind
on a point. Of course, I did not mind you so much, as you
are not connected with the official police, but it is not pleasant
to have a family misfortune like this noised abroad. Besides,
it is a useless expense, for how could you possibly find
this Hosmer Angel?”
%%085

“On the contrary,” said Holmes, quietly; “I have every
reason to believe that I will succeed in discovering Mr.~Hosmer
Angel.”

Mr.~Windibank gave a violent start, and dropped his gloves.
“I am delighted to hear it,” he said.

“It is a curious thing,” remarked Holmes, “that a type-writer
has really quite as much individuality as a man’s
handwriting. Unless they are quite new, no two of them
write exactly alike. Some letters get more worn than others,
and some wear only on one side. Now, you remark in this
note of yours, Mr.~Windibank, that in every case there is some
little slurring over of the ‘e,’ and a slight defect in the tail of
the ‘r.’ There are fourteen other characteristics, but those
are the more obvious.”

“We do all our correspondence with this machine at the
office, and no doubt it is a little worn,” our visitor answered,
glancing keenly at Holmes with his bright little eyes.

“And now I will show you what is really a very interesting
study, Mr.~Windibank,” Holmes continued. “I think of writing
another little monograph some of these days on the type-writer
and its relation to crime. It is a subject to which I
have devoted some little attention. I have here four letters
which purport to come from the missing man. They are all
type-written. In each case, not only are the ‘e’s’ slurred and
the ‘r’s’ tailless, but you will observe, if you care to use my
magnifying lens, that the fourteen other characteristics to
which I have alluded are there as well.”

Mr.~Windibank sprang out of his chair, and picked up his
hat. “I cannot waste time over this sort of fantastic talk,
Mr.~Holmes,” he said. “If you can catch the man, catch
him, and let me know when you have done it.”

“Certainly,” said Holmes, stepping over and turning the
key in the door. “I let you know, then, that I have caught
him!”

“What! where?” shouted Mr.~Windibank, turning white to
his lips, and glancing about him like a rat in a trap.
%%086

“Oh, it won’t do -- really it won’t,” said Holmes, suavely.
“There is no possible getting out of it, Mr.~Windibank. It is
quite too transparent, and it was a very bad compliment when
you said that it was impossible for me to solve so simple a
question. That’s right! Sit down, and let us talk it over.”

Our visitor collapsed into a chair, with a ghastly face, and
a glitter of moisture on his brow. “It -- it’s not actionable,”
he stammered.

“I am very much afraid that it is not. But between
ourselves, Windibank, it was as cruel and selfish and heartless a
trick in a petty way as ever came before me. Now, let me
just run over the course of events, and you will contradict me
if I go wrong.”

The man sat huddled up in his chair, with his head sunk
upon his breast, like one who is utterly crushed. Holmes
stuck his feet up on the corner of the mantel-piece, and, leaning
back with his hands in his pockets, began talking, rather
to himself, as it seemed, than to us.

“The man married a woman very much older than himself
for her money,” said he, “and he enjoyed the use of the
money of the daughter as long as she lived with them. It
was a considerable sum, for people in their position, and the
loss of it would have made a serious difference. It was worth
an effort to preserve it. The daughter was of a good, amiable
disposition, but affectionate and warm-hearted in her ways, so
that it was evident that with her fair personal advantages,
and her little income, she would not be allowed to remain
single long. Now her marriage would mean, of course, the
loss of a hundred a year, so what does her step-father do to
prevent it? He takes the obvious course of keeping her at
home, and forbidding her to seek the company of people of
her own age. But soon he found that that would not answer
forever. She became restive, insisted upon her rights, and
finally announced her positive intention of going to a certain
ball. What does her clever step-father do then? He conceives
an idea more creditable to his head than to his heart.
%%087
%%“GLANCING ABOUT HIM LIKE A RAT IN A TRAP”
%%088
With the connivance and assistance of his wife he disguised
himself, covered those keen eyes with tinted glasses, masked
the face with a mustache and a pair of bushy whiskers, sunk
that clear voice into an insinuating whisper, and doubly secure
on account of the girl’s short sight, he appears as Mr.~Hosmer
Angel, and keeps off other lovers by making love himself.”

“It was only a joke at first,” groaned our visitor. “We
never thought that she would have been so carried away.”

“Very likely not. However that may be, the young lady
was very decidedly carried away, and having quite made up
her mind that her step-father was in France, the suspicion of
treachery never for an instant entered her mind. She was
flattered by the gentleman’s attentions, and the effect was
increased by the loudly expressed admiration of her mother.
Then Mr.~Angel began to call, for it was obvious that the
matter should be pushed as far as it would go, if a real effect
were to be produced. There were meetings, and an engagement,
which would finally secure the girl’s affections from
turning towards any one else. But the deception could not
be kept up forever. These pretended journeys to France
were rather cumbrous. The thing to do was clearly to bring
the business to an end in such a dramatic manner that it
would leave a permanent impression upon the young lady’s
mind, and prevent her from looking upon any other suitor for
some time to come. Hence those vows of fidelity exacted
upon a Testament, and hence also the allusions to a possibility
of something happening on the very morning of the
wedding. James Windibank wished Miss Sutherland to be
so bound to Hosmer Angel, and so uncertain as to his fate,
that for ten years to come, at any rate, she would not listen
to another man. As far as the church door he brought her,
and then, as he could go no farther, he conveniently vanished
away by the old trick of stepping in at one door of a four-%
wheeler, and out at the other. I think that that was the chain
of events, Mr.~Windibank!”

Our visitor had recovered something of his assurance while
%%090
Holmes had been talking, and he rose from his chair now
with a cold sneer upon his pale face.

“It may be so, or it may not, Mr.~Holmes,” said he, “but if
you are so very sharp you ought to be sharp enough to know
that it is you who are breaking the law now, and not me. I
have done nothing actionable from the first, but as long as
you keep that door locked you lay yourself open to an action
for assault and illegal constraint.”

“The law cannot, as you say, touch you,” said Holmes, unlocking
and throwing open the door, “yet there never was a
man who deserved punishment more. If the young lady has
a brother or a friend, he ought to lay a whip across your
shoulders. By Jove!” he continued, flushing up at the sight
of the bitter sneer upon the man’s face, “it is not part of my
duties to my client, but here’s a hunting crop handy, and I
think I shall just treat myself to -- ” He took two swift steps
to the whip, but before he could grasp it there was a wild
clatter of steps upon the stairs, the heavy hall door banged,
and from the window we could see Mr.~James Windibank
running at the top of his speed down the road.

“There’s a cold-blooded scoundrel!” said Holmes, laughing,
as he threw himself down into his chair once more. “That
fellow will rise from crime to crime until he does something
very bad, and ends on a gallows. The case has, in some respects,
been not entirely devoid of interest.”

“I cannot now entirely see all the steps of your reasoning,”
I remarked.

“Well, of course it was obvious from the first that this Mr.
Hosmer Angel must have some strong object for his curious
conduct, and it was equally clear that the only man who really
profited by the incident, as far as we could see, was the
stepfather. Then the fact that the two men were never together,
but that the one always appeared when the other was away,
was suggestive. So were the tinted spectacles and the curious
voice, which both hinted at a disguise, as did the bushy
whiskers. My suspicions were all confirmed by his peculiar
%%091
action in type-writing his signature, which, of course, inferred
that his handwriting was so familiar to her that she would
recognize even the smallest sample of it. You see all these
isolated facts, together with many minor ones, all pointed in
the same direction.”

“And how did you verify them?”

“Having once spotted my man, it was easy to get corroboration.
I knew the firm for which this man worked. Having
taken the printed description, I eliminated everything from it
which could be the result of a disguise -- the whiskers, the
glasses, the voice, and I sent it to the firm, with a request that
they would inform me whether it answered to the description
of any of their travellers. I had already noticed the peculiarities
of the type-writer, and I wrote to the man himself at his
business address, asking him if he would come here. As I
expected, his reply was type-written, and revealed the same
trivial but characteristic defects. The same post brought me
a letter from Westhouse \& Marbank, of Fenchurch Street, to
say that the description tallied in every respect with that of
their employé, James Windibank. \textit{Voila tout!}”

“And Miss Sutherland?”

“If I tell her she will not believe me. You may remember
the old Persian saying, ‘There is danger for him who taketh
the tiger cub, and danger also for whoso snatches a delusion
from a woman.’ There is as much sense in Hafiz as in Horace,
and as much knowledge of the world.”
%%092

\Chapter{The Boscombe Valley Mystery}

\textsc{We} were seated at breakfast one morning, my wife
and I, when the maid brought in a telegram. It
was from Sherlock Holmes, and ran in this way:

“Have you a couple of days to spare? Have
just been wired for from the West of England in connection
with Bos\-combe Valley tragedy. Shall be glad if you will
come with me. Air and scenery perfect. Leave Paddington
by the 11.15.”

“What do you say, dear?” said my wife, looking across at
me. “Will you go?”

“I really don’t know what to say. I have a fairly long list
at present.”

“Oh, Anstruther would do your work for you. You have
been looking a little pale lately. I think that the change
would do you good, and you are always so interested in Mr.
Sherlock Holmes’s cases.”

“I should be ungrateful if I were not, seeing what I gained
through one of them,” I answered. “But if I am to go, I
must pack at once, for I have only half an hour.”

My experience of camp life in Afghanistan had at least had
the effect of making me a prompt and ready traveller. My
wants were few and simple, so that in less than the time
stated I was in a cab with my valise, rattling away to Paddington
Station. Sherlock Holmes was pacing up and down
the platform, his tall, gaunt figure made even gaunter and
taller by his long gray travelling-cloak and close-fitting cloth
cap.
%%093

“It is really very good of you to come, Watson,” said he.
“It makes a considerable difference to me, having some one
with me on whom I can thoroughly rely. Local aid is always
either worthless or else biassed. If you will keep the two
corner seats I shall get the tickets.”

We had the carriage to ourselves save for an immense litter
of papers which Holmes had brought with him. Among
these he rummaged and read, with intervals of note-taking
and of meditation, until we were past Reading. Then he
suddenly rolled them all into a gigantic ball, and tossed them
up onto the rack.

“Have you heard anything of the case?” he asked.

“Not a word. I have not seen a paper for some days.”

“The London press has not had very full accounts. I
have just been looking through all the recent papers in order
to master the particulars. It seems, from what I gather,
to be one of those simple cases which are so extremely
difficult.”

“That sounds a little paradoxical.”

“But it is profoundly true. Singularity is almost invariably
a clew. The more featureless and commonplace a crime is,
the more difficult is it to bring it home. In this case, however,
they have established a very serious case against the son
of the murdered man.”

“It is a murder, then?”

“Well, it is conjectured to be so. I shall take nothing for
granted until I have the opportunity of looking personally
into it. I will explain the state of things to you, as far as I
have been able to understand it, in a very few words.

“Boscombe Valley is a country district not very far from
Ross, in Herefordshire. The largest landed proprietor in
that part is a Mr.~John Turner, who made his money in Australia,
and returned some years ago to the old country. One
of the farms which he held, that of Hatherley, was let to Mr.
Charles McCarthy, who was also an ex-Australian. The men
had known each other in the colonies, so that it was not
%%094
unnatural that when they came to settle down they should do so
as near each other as possible. Turner was apparently the
richer man, so McCarthy became his tenant, but still remained,
it seems, upon terms of perfect equality, as they were
frequently together. McCarthy had one son, a lad of eighteen,
and Turner had an only daughter of the same age, but
neither of them had wives living. They appear to have avoided
the society of the neighboring English families, and to have
led retired lives, though both the McCarthys were fond of
sport, and were frequently seen at the race-meetings of the
neighborhood. McCarthy kept two servants -- a man and a
girl. Turner had a considerable household, some half-dozen
at the least. That is as much as I have been able to gather
about the families. Now for the facts.

“On June 3, that is, on Monday last, McCarthy left his
house at Hatherley about three in the afternoon, and walked
down to the Boscombe Pool, which is a small lake formed by
the spreading out of the stream which runs down the Boscombe
Valley. He had been out with his serving-man in the
morning at Ross, and he had told the man that he must hurry,
as he had an appointment of importance to keep at three.
From that appointment he never came back alive.

“From Hatherley Farm-house to the Boscombe Pool is a
quarter of a mile, and two people saw him as he passed over
this ground. One was an old woman, whose name is not
mentioned, and the other was William Crowder, a game-keeper
in the employ of Mr.~Turner. Both these witnesses depose
that Mr.~McCarthy was walking alone. The game-keeper adds
that within a few minutes of his seeing Mr.~McCarthy pass he
had seen his son, Mr.~James McCarthy, going the same way
with a gun under his arm. To the best of his belief, the father
was actually in sight at the time, and the son was following
him. He thought no more of the matter until he heard
in the evening of the tragedy that had occurred.

“The two McCarthys were seen after the time when William
Crowder, the game-keeper, lost sight of them. The
%%095
Boscombe Pool is thickly-wooded round, with just a fringe of
grass and of reeds round the edge. A girl of fourteen, Patience
Moran, who is the daughter of the lodge-keeper of
the Boscombe Valley estate, was in one of the woods picking
flowers. She states that while she was there she saw, at
the border of the wood and close by the lake, Mr.~McCarthy
and his son, and that they appeared to be having a violent
quarrel. She heard Mr.~McCarthy the elder using very strong
language to his son, and she saw the latter raise up his hand
as if to strike his father. She was so frightened by their violence
that she ran away, and told her mother when she reached
home that she had left the two McCarthys quarrelling near
Boscombe Pool, and that she was afraid that they were going
to fight. She had hardly said the words when young Mr.
McCarthy came running up to the lodge to say that he had
found his father dead in the wood, and to ask for the help of
the lodge-keeper. He was much excited, without either his
gun or his hat, and his right hand and sleeve were observed
to be stained with fresh blood. On following him they found
the dead body stretched out upon the grass beside the Pool.
The head had been beaten in by repeated blows of some heavy
and blunt weapon. The injuries were such as might very
well have been inflicted by the butt-end of his son’s gun,
which was found lying on the grass within a few paces of the
body. Under these circumstances the young man was instantly
arrested, and a verdict of ‘Wilful Murder’ having been
returned at the inquest on Tuesday, he was on Wednesday
brought before the magistrates at Ross, who have referred the
case to the next assizes. Those are the main facts of the case
as they came out before the coroner and at the police-court.”

“I could hardly imagine a more damning case,” I remarked.
“If ever circumstantial evidence pointed to a criminal
it does so here.”

“Circumstantial evidence is a very tricky thing,” answered
Holmes, thoughtfully. “It may seem to point very straight
to one thing, but if you shift your own point of view a little,
%%096
you may find it pointing in an equally uncompromising manner
to something entirely different. It must be confessed,
however, that the case looks exceedingly grave against the
young man, and it is very possible that he is indeed the culprit.
There are several people in the neighborhood, however,
and among them Miss Turner, the daughter of the neighboring
land-owner, who believe in his innocence, and who have
retained Lestrade, whom you may recollect in connection
with the Study in Scarlet, to work out the case in his interest.
Lestrade, being rather puzzled, has referred the case to me,
and hence it is that two middle-aged gentlemen are flying
westward at fifty miles an hour, instead of quietly digesting
their breakfasts at home.”

“I am afraid,” said I, “that the facts are so obvious that
you will find little credit to be gained out of this case.”

“There is nothing more deceptive than an obvious fact,” he
answered, laughing. “Besides, we may chance to hit upon
some other obvious facts which may have been by no means
obvious to Mr.~Lestrade. You know me too well to think that
I am boasting when I say that I shall either confirm or destroy
his theory by means which he is quite incapable of employing,
or even of understanding. To take the first example
to hand, I very clearly perceive that in your bedroom the window
is upon the right-hand side, and yet I question whether
Mr.~Lestrade would have noted even so self-evident a thing
as that.”

“How on earth -- ”

“My dear fellow, I know you well. I know the military
neatness which characterizes you. You shave every morning,
and in this season you shave by the sunlight; but since your
shaving is less and less complete as we get farther back on
the left side, until it becomes positively slovenly as we get
round the angle of the jaw, it is surely very clear that that side
is less well illuminated than the other. I could not imagine
a man of your habits looking at himself in an equal light, and
being satisfied with such a result. I only quote this as a
%%097
%%“THEY FOUND THE BODY”
%%098
trivial example of observation and inference. Therein lies my
\textit{métier}, and it is just possible that it may be of some service in
the investigation which lies before us. There are one or two
minor points which were brought out in the inquest, and which
are worth considering.”

“What are they?”

“It appears that his arrest did not take place at once, but
after the return to Hatherley Farm. On the inspector of
constabulary informing him that he was a prisoner, he remarked
that he was not surprised to hear it, and that it was no more
than his deserts. This observation of his had the natural
effect of removing any traces of doubt which might have remained
in the minds of the coroner’s jury.”

“It was a confession,” I ejaculated.

“No, for it was followed by a protestation of innocence.”

“Coming on the top of such a damning series of events, it
was at least a most suspicious remark.”

“On the contrary,” said Holmes, “it is the brightest rift
which I can at present see in the clouds. However innocent
he might be, he could not be such an absolute imbecile as
not to see that the circumstances were very black against him.
Had he appeared surprised at his own arrest, or feigned indignation
at it, I should have looked upon it as highly suspicious,
because such surprise or anger would not be natural
under the circumstances, and yet might appear to be the best
policy to a scheming man. His frank acceptance of the situation
marks him as either an innocent man, or else as a man
of considerable self-restraint and firmness. As to his remark
about his deserts, it was also not unnatural if you consider
that he stood beside the dead body of his father, and that
there is no doubt that he had that very day so far forgotten
his filial duty as to bandy words with him, and even, according
to the little girl whose evidence is so important, to raise
his hand as if to strike him. The self-reproach and contrition
which are displayed in his remark appear to me to be the
signs of a healthy mind, rather than of a guilty one.”
%%100

I shook my head. “Many men have been hanged on far
slighter evidence,” I remarked.

“So they have. And many men have been wrongfully
hanged.”

“What is the young man’s own account of the matter?”

“It is, I am afraid, not very encouraging to his supporters,
though there are one or two points in it which are suggestive.
You will find it here, and may read it for yourself.”

He picked out from his bundle a copy of the local Herefordshire
paper, and having turned down the sheet, he pointed
out the paragraph in which the unfortunate young man
had given his own statement of what had occurred. I settled
myself down in the corner of the carriage, and read it
very carefully. It ran in this way:

“Mr.~James McCarthy, the only son of the deceased, was
then called, and gave evidence as follows: ‘I had been
away from home for three days at Bristol, and had only just
returned upon the morning of last Monday, the 3rd. My father
was absent from home at the time of my arrival, and I
was informed by the maid that he had driven over to Ross
with John Cobb, the groom. Shortly after my return I heard
the wheels of his trap in the yard, and, looking out of my window,
I saw him get out and walk rapidly out of the yard,
though I was not aware in which direction he was going. I
then took my gun, and strolled out in the direction of the
Boscombe Pool, with the intention of visiting the rabbit-warren
which is upon the other side. On my way I saw William
Crowder, the game-keeper, as he had stated in his evidence;
but he is mistaken in thinking that I was following my father.
I had no idea that he was in front of me. When about a
hundred yards from the Pool I heard a cry of “Cooee!”
which was a usual signal between my father and myself. I
then hurried forward, and found him standing by the Pool.
He appeared to be much surprised at seeing me, and asked
me rather roughly what I was doing there. A conversation
ensued which led to high words, and almost to blows, for my
%%101
father was a man of a very violent temper. Seeing that his
passion was becoming ungovernable, I left him, and returned
towards Hatherley Farm. I had not gone more than 150
yards, however, when I heard a hideous outcry behind me,
which caused me to run back again. I found my father expiring
upon the ground, with his head terribly injured. I
dropped my gun, and held him in my arms, but he almost instantly
expired. I knelt beside him for some minutes, and
then made my way to Mr.~Turner’s lodge-keeper, his house
being the nearest, to ask for assistance. I saw no one near
my father when I returned, and I have no idea how he came
by his injuries. He was not a popular man, being somewhat
cold and forbidding in his manners; but he had, as far as I
know, no active enemies. I know nothing further of the
matter.’

“The Coroner: Did your father make any statement to you
before he died?

“Witness: He mumbled a few words, but I could only catch
some allusion to a rat.

“The Coroner: What did you understand by that?

“Witness: It conveyed no meaning to me. I thought that
he was delirious.

“The Coroner: What was the point upon which you and
your father had this final quarrel?

“Witness: I should prefer not to answer.

“The Coroner: I am afraid that I must press it.

“Witness: It is really impossible for me to tell you. I can
assure you that it has nothing to do with the sad tragedy
which followed.

“The Coroner: That is for the court to decide. I need
not point out to you that your refusal to answer will prejudice
your case considerably in any future proceedings which may
arise.

“Witness: I must still refuse.

“The Coroner: I understand that the cry of ‘Cooee’ was
a common signal between you and your father?
%%102

“Witness: It was.

“The Coroner: How was it, then, that he uttered it before
he saw you, and before he even knew that you had returned
from Bristol?

“Witness (with considerable confusion): I do not know.

“A Juryman: Did you see nothing which aroused your
suspicions when you returned on hearing the cry, and found
your father fatally injured?

“Witness: Nothing definite.

“The Coroner: What do you mean?

“Witness: I was so disturbed and excited as I rushed
out into the open, that I could think of nothing except of
my father. Yet I have a vague impression that as I ran forward
something lay upon the ground to the left of me. It
seemed to me to be something gray in color, a coat of some
sort, or a plaid perhaps. When I rose from my father I
looked round for it, but it was gone.

“\,‘Do you mean that it disappeared before you went for
help?’

“\,‘Yes, it was gone.’

“\,‘You cannot say what it was?’

“\,‘No, I had a feeling something was there.’

“\,‘How far from the body?’

“\,‘A dozen yards or so.’

“\,‘And how far from the edge of the wood?’

“\,‘About the same.’

“\,‘Then if it was removed it was while you were within a
dozen yards of it?’

“\,‘Yes, but with my back towards it.’

“This concluded the examination of the witness.”

“I see,” said I, as I glanced down the column, “that the
coroner in his concluding remarks was rather severe upon
young McCarthy. He calls attention, and with reason, to the
discrepancy about his father having signalled to him before
seeing him, also to his refusal to give details of his conversation
with his father, and his singular account of his father’s
%%103
dying words. They are all, as he remarks, very much against
the son.”

Holmes laughed softly to himself, and stretched himself out
upon the cushioned seat. “Both you and the coroner have
been at some pains,” said he, “to single out the very strongest
points in the young man’s favor. Don’t you see that you
alternately give him credit for having too much imagination
and too little. Too little, if he could not invent a cause of
quarrel which would give him the sympathy of the jury; too
much, if he evolved from his own inner consciousness anything
so \textit{outré} as a dying reference to a rat, and the incident
of the vanishing cloth. No, sir, I shall approach this case
from the point of view that what this young man says is true,
and we shall see whither that hypothesis will lead us. And
now here is my pocket Petrarch, and not another word shall
I say of this case until we are on the scene of action. We
lunch at Swindon, and I see that we shall be there in twenty
minutes.”

It was nearly four o’clock when we at last, after passing
through the beautiful Stroud Valley, and over the broad gleaming
Severn, found ourselves at the pretty little country-town
of Ross. A lean, ferret-like man, furtive and sly-looking, was
waiting for us upon the platform. In spite of the light brown
dustcoat and leather-leggings which he wore in deference to
his rustic surroundings, I had no difficulty in recognizing
Lestrade, of Scotland Yard. With him we drove to the
Hereford Arms, where a room had already been engaged
for us.

“I have ordered a carriage,” said Lestrade, as we sat over
a cup of tea. “I knew your energetic nature, and that you
would not be happy until you had been on the scene of the
crime.”

“It was very nice and complimentary of you,” Holmes
answered. “It is entirely a question of barometric pressure.”

Lestrade looked startled. “I do not quite follow,” he
said.
%%104

“How is the glass? Twenty-nine, I see. No wind, and
not a cloud in the sky. I have a caseful of cigarettes here
which need smoking, and the sofa is very much superior to
the usual country hotel abomination. I do not think that it
is probable that I shall use the carriage to-night.”

Lestrade laughed indulgently. “You have, no doubt, already
formed your conclusions from the newspapers,” he said.
“The case is as plain as a pikestaff, and the more one goes
into it the plainer it becomes. Still, of course, one can’t refuse
a lady, and such a very positive one, too. She had heard
of you, and would have your opinion, though I repeatedly told
her that there was nothing which you could do which I had
not already done. Why, bless my soul! here is her carriage
at the door.”

He had hardly spoken before there rushed into the room
one of the most lovely young women that I have ever seen
in my life. Her violet eyes shining, her lips parted, a pink
flush upon her cheeks, all thought of her natural reserve lost
in her overpowering excitement and concern.

“Oh, Mr.~Sherlock Holmes!” she cried, glancing from one
to the other of us, and finally, with a woman’s quick intuition,
fastening upon my companion, “I am so glad that you have
come. I have driven down to tell you so. I know that James
didn’t do it. I know it, and I want you to start upon your
work knowing it, too. Never let yourself doubt upon that
point. We have known each other since we were little children,
and I know his faults as no one else does; but he is too
tender-hearted to hurt a fly. Such a charge is absurd to any
one who really knows him.”

“I hope we may clear him, Miss Turner,” said Sherlock
Holmes. “You may rely upon my doing all that I can.”

“But you have read the evidence. You have formed some
conclusion? Do you not see some loophole, some flaw? Do
you not yourself think that he is innocent?”

“I think that it is very probable.”

“There, now!” she cried, throwing back her head, and
%%105
looking defiantly at Lestrade. “You hear! He gives me
hopes.”

Lestrade shrugged his shoulders. “I am afraid that my
colleague has been a little quick in forming his conclusions,”
he said.

“But he is right. Oh! I know that he is right. James
never did it. And about his quarrel with his father, I am sure
that the reason why he would not speak about it to the coroner
was because I was concerned in it.”

“In what way?” asked Holmes.

“It is no time for me to hide anything. James and his
father had many disagreements about me. Mr.~McCarthy was
very anxious that there should be a marriage between us.
James and I have always loved each other as brother and sister;
but of cou\-rse he is young, and has seen very little of life
yet, and -- and -- well, he naturally did not wish to do anything
like that yet. So there were quarrels, and this, I am sure, was
one of them.”

“And your father?” asked Holmes. “Was he in favor of
such a union?”

“No, he was averse to it also. No one but Mr.~McCarthy
was in favor of it.” A quick blush passed over her fresh
young face as Holmes shot one of his keen, questioning
glances at her.

“Thank you for this information,” said he. “May I see
your father if I call to-morrow?”

“I am afraid the doctor won’t allow it.”

“The doctor?”

“Yes, have you not heard? Poor father has never been
strong for years back, but this has broken him down completely.
He has taken to his bed, and Dr. Willows says that
he is a wreck, and that his nervous system is shattered. Mr.
McCarthy was the only man alive who had known dad in the
old days in Victoria.”

“Ha! In Victoria! That is important.”

“Yes, at the mines.”
%%106

“Quite so; at the gold-mines, where, as I understand, Mr.
Turner made his money.”

“Yes, certainly.”

“Thank you, Miss Turner. You have been of material
assistance to me.”

“You will tell me if you have any news to-morrow. No
doubt you will go to the prison to see James. Oh, if you do,
Mr.~Holmes, do tell him that I know him to be innocent.”

“I will, Miss Turner.”

“I must go home now, for dad is very ill, and he misses
me so if I leave him. Good-bye, and God help you in your
undertaking.” She hurried from the room as impulsively as
she had entered, and we heard the wheels of her carriage rattle
off down the street.

“I am ashamed of you, Holmes,” said Lestrade, with dignity,
after a few minutes’ silence. “Why should you raise up
hopes which you are bound to disappoint? I am not over-tender
of heart, but I call it cruel.”

“I think that I see my way to clearing James McCarthy,”
said Holmes. “Have you an order to see him in prison?”

“Yes, but only for you and me.”

“Then I shall reconsider my resolution about going out.
We have still time to take a train to Hereford and see him
to-night?”

“Ample.”

“Then let us do so. Watson, I fear that you will find it
very slow, but I shall only be away a couple of hours.”

I walked down to the station with them, and then wandered
through the streets of the little town, finally returning to the
hotel, where I lay upon the sofa and tried to interest myself in
a yellow-backed novel. The puny plot of the story was so thin,
however, when compared to the deep mystery through which
we were groping, and I found my attention wander so continually
from the fiction to the fact, that I at last flung it across
the room, and gave myself up entirely to a consideration of
the events of the day. Supposing that this unhappy young
%%107
man’s story was absolutely true, then what hellish thing, what
absolutely unforeseen and extraordinary calamity could have
occurred between the time when he parted from his father,
and the moment when, drawn back by his screams, he rushed
into the glade? It was something terrible and deadly. What
could it be? Might not the nature of the injuries reveal
something to my medical instincts? I rang the bell, and called
for the weekly county paper, which contained a verbatim account
of the inquest. In the surgeon’s deposition it was stated
that the posterior third of the left parietal bone and the
left half of the occipital bone had been shattered by a heavy
blow from a blunt weapon. I marked the spot upon my
own head. Clearly such a blow must have been struck from
behind. That was to some extent in favor of the accused, as
when seen quarrelling he was face to face with his father.
Still, it did not go for very much, for the older man might have
turned his back before the blow fell. Still, it might be worth
while to call Holmes’s attention to it. Then there was the
peculiar dying reference to a rat. What could that mean? It
could not be delirium. A man dying from a sudden blow
does not commonly become delirious. No, it was more likely
to be an attempt to explain how he met his fate. But what
could it indicate? I cudgelled my brains to find some possible
explanation. And then the incident of the gray cloth,
seen by young McCarthy. If that were true, the murderer
must have dropped some part of his dress, presumably his
overcoat, in his flight, and must have had the hardihood to
return and to carry it away at the instant when the son was
kneeling with his back turned not a dozen paces off. What a
tissue of mysteries and improbabilities the whole thing was!
I did not wonder at Lestrade’s opinion, and yet I had so much
faith in Sherlock Holmes’s insight that I could not lose hope
as long as every fresh fact seemed to strengthen his conviction
of young McCarthy’s innocence.

It was late before Sherlock Holmes returned. He came
back alone, for Lestrade was staying in lodgings in the town.
%%108

“The glass still keeps very high,” he remarked, as he sat
down. “It is of importance that it should not rain before we
are able to go over the ground. On the other hand, a man
should be at his very best and keenest for such nice work as
that, and I did not wish to do it when fagged by a long
journey. I have seen young McCarthy.”

“And what did you learn from him?”

“Nothing.”

“Could he throw no light?”

“None at all. I was inclined to think at one time that he
knew who had done it, and was screening him or her, but I
am convinced now that he is as puzzled as every one else. He
is not a very quick-witted youth, though comely to look at,
and, I should think, sound at heart.”

“I cannot admire his taste,” I remarked, “if it is indeed a
fact that he was averse to a marriage with so charming a
young lady as this Miss Turner.”

“Ah, thereby hangs a rather painful tale. This fellow is
madly, insanely in love with her, but some two years ago,
when he was only a lad, and before he really knew her, for
she had been away five years at a boarding-school, what does
the idiot do but get into the clutches of a barmaid in Bristol,
and marry her at a registry office? No one knows a word of
the matter, but you can imagine how maddening it must be to
him to be upbraided for not doing what he would give his
very eyes to do, but what he knows to be absolutely impossible.
It was sheer frenzy of this sort which made him throw
his hands up into the air when his father, at their last interview,
was goading him on to propose to Miss Turner. On the
other hand, he had no means of supporting himself, and his
father, who was by all accounts a very hard man, would have
thrown him over utterly had he known the truth. It was with
his barmaid wife that he had spent the last three days in
Bristol, and his father did not know where he was. Mark
that point. It is of importance. Good has come out of evil,
however, for the barmaid, finding from the papers that he is
%%109
in serious trouble, and likely to be hanged, has thrown him
over utterly, and has written to him to say that she has a
husband already in the Bermuda Dockyard, so that there is
really no tie between them. I think that that bit of news has
consoled young McCarthy for all that he has suffered.”

“But if he is innocent, who has done it?”

“Ah! who? I would call your attention very particularly
to two points. One is that the murdered man had an appointment
with some one at the Pool, and that the some one
could not have been his son, for his son was away, and he did
not know when he would return. The second is that the
murdered man was heard to cry ‘Cooee!’ before he knew
that his son had returned. Those are the crucial points upon
which the case depends. And now let us talk about George
Meredith, if you please, and we shall leave all minor matters
until to-morrow.”

There was no rain, as Holmes had foretold, and the morning
broke bright and cloudless. At nine o’clock Lestrade
called for us with the carriage, and we set off for Hatherley
Farm and the Boscombe Pool.

“There is serious news this morning,” Lestrade observed.
“It is said that Mr.~Turner, of the Hall, is so ill that his life
is despaired of.”

“An elderly man, I presume?” said Holmes.

“About sixty; but his constitution has been shattered by
his life abroad, and he has been in failing health for some
time. This business has had a very bad effect upon him.
He was an old friend of McCarthy’s, and, I may add, a great
benefactor to him, for I have learned that he gave him
Hatherley Farm rent free.”

“Indeed! That is interesting,” said Holmes.

“Oh yes! In a hundred other ways he has helped him.
Everybody about here speaks of his kindness to him.”

“Really! Does it not strike you as a little singular that
this McCarthy, who appears to have had little of his own, and
to have been under such obligations to Turner, should still
%%110
talk of marrying his son to Turner’s daughter, who is, presumably,
heiress to the estate, and that in such a very cocksure
manner, as if it were merely a case of a proposal and all else
would follow? It is the more strange, since we know that
Turner himself was averse to the idea. The daughter told us
as much. Do you not deduce something from that?”

“We have got to the deductions and the inferences,” said
Lestrade, winking at me. “I find it hard enough to tackle
facts, Holmes, without flying away after theories and fancies.”

“You are right,” said Holmes, demurely; “you do find it
very hard to tackle the facts.”

“Anyhow, I have grasped one fact which you seem to find it
difficult to get hold of,” replied Lestrade, with some warmth.

“And that is -- ”

“That McCarthy, senior, met his death from McCarthy,
junior, and that all theories to the contrary are the merest
moonshine.”

“Well, moonshine is a brighter thing than fog,” said
Holmes, laughing. “But I am very much mistaken if this
is not Hatherley Farm upon the left.”

“Yes, that is it.” It was a wide-spread, com\-fortable-looking
building, two-storied, slate roofed, with great yellow blotches of
lichen upon the gray walls. The drawn blinds and the smokeless
chimneys, however, gave it a stricken look, as though the
weight of this horror still lay heavy upon it. We called at the
door, when the maid, at Holmes’s request, showed us the boots
which her master wore at the time of his death, and also a
pair of the son’s, though not the pair which he had then had.
Having measured these very carefully from seven or eight
different points, Holmes desired to be led to the court-yard,
from which we all followed the winding track which led to
Boscombe Pool.

Sherlock Holmes was transformed when he was hot upon
such a scent at this. Men who had only known the quiet
thinker and logician of Baker Street would have failed to
recognize him. His face flushed and darkened. His brows
%%111
%%“THE MAID SHOWED US THE BOOTS.”
%%112
were drawn into two hard, black lines, while his eyes shone
out from beneath them with a steely glitter. His face was
bent downward, his shoulders bowed, his lips compressed,
and the veins stood out like whip-cord in his long, sinewy
neck. His nostrils seemed to dilate with a purely animal lust
for the chase, and his mind was so absolutely concentrated
upon the matter before him, that a question or remark fell unheeded
upon his ears, or, at the most, only provoked a quick,
impatient snarl in reply. Swiftly and silently he made his
way along the track which ran through the meadows, and so
by way of the woods to the Boscombe Pool. It was damp,
marshy ground, as is all that district, and there were marks
of many feet, both upon the path and amid the short grass
which bounded it on either side. Sometimes Holmes would
hurry on, sometimes stop dead, and once he made quite a little
\textit{détour} into the meadow. Lestrade and I walked behind
him, the detective indifferent and contemptuous, while I
watched my friend with the interest which sprang from the
conviction that every one of his actions was directed towards
a definite end.

The Boscombe Pool, which is a little reed-girt sheet of
water some fifty yards across, is situated at the boundary between
the Hatherley Farm and the private park of the wealthy
Mr.~Turner. Above the woods which lined it upon the farther
side we could see the red, jutting pinnacles which marked the
site of the rich land-owner’s dwelling. On the Hatherley side
of the Pool the woods grew very thick, and there was a narrow
belt of sodden grass twenty paces across between the edge
of the trees and the reeds which lined the lake. Lestrade
showed us the exact spot at which the body had been found,
and, indeed, so moist was the ground, that I could plainly see
the traces which had been left by the fall of the stricken man.
To Holmes, as I could see by his eager face and peering eyes,
very many other things were to be read upon the trampled
grass. He ran round, like a dog who is picking up a scent,
and then turned upon my companion.
%%114

“What did you go into the Pool for?” he asked.

“I fished about with a rake. I thought there might be some
weapon or other trace. But how on earth -- ”

“Oh, tut, tut! I have no time! That left foot of yours
with its inward twist is all over the place. A mole could trace
it, and there it vanishes among the reeds. Oh, how simple it
would all have been had I been here before they came like a
herd of buffalo, and wallowed all over it. Here is where the
party with the lodge-keeper came, and they have covered all
tracks for six or eight feet round the body. But here are
three separate tracks of the same feet.” He drew out a lens,
and lay down upon his waterproof to have a better view, talking
all the time rather to himself than to us. “These are
young McCarthy’s feet. Twice he was walking, and once he
ran swiftly so that the soles are deeply marked, and the heels
hardly visible. That bears out his story. He ran when he
saw his father on the ground. Then here are the father’s feet
as he paced up and down. What is this, then? It is the butt-end
of the gun as the son stood listening. And this? Ha,
ha! What have we here? Tiptoes! tiptoes! Square, too,
quite unusual boots! They come, they go, they come again -- of
course that was for the cloak. Now where did they come
from?” He ran up and down, sometimes losing, sometimes
finding the track until we were well within the edge of the
wood, and under the shadow of a great beech, the largest tree
in the neighborhood. Holmes traced his way to the farther
side of this, and lay down once more upon his face with a little
cry of satisfaction. For a long time he remained there,
turning over the leaves and dried sticks, gathering up what
seemed to me to be dust into an envelope, and examining
with his lens not only the ground, but even the bark of the
tree as far as he could reach. A jagged stone was lying
among the moss, and this also he carefully examined and retained.
Then he followed a pathway through the wood until
he came to the high-road, where all traces were lost.

“It has been a case of considerable interest,” he remarked,
%%115
returning to his natural manner. “I fancy that this gray
house on the right must be the lodge. I think that I will go
in and have a word with Moran, and perhaps write a little
note. Having done that, we may drive back to our luncheon.
You may walk to the cab, and I shall be with you presently.”

It was about ten minutes before we regained our cab, and
drove back into Ross, Holmes still carrying with him the
stone which he had picked up in the wood.

“This may interest you, Lestrade,” he remarked, holding it
out. “The murder was done with it.”

“I see no marks.”

“There are none.”

“How do you know, then?”

“The grass was growing under it. It had only lain there a
few days. There was no sign of a place whence it had been
taken. It corresponds with the injuries. There is no sign of
any other weapon.”

“And the murderer?”

“Is a tall man, left-handed, limps with the right leg, wears
thick-soled shooting-boots and a gray cloak, smokes Indian
cigars, uses a cigar-holder, and carries a blunt penknife in his
pocket. There are several other indications, but these may
be enough to aid us in our search.”

Lestrade laughed. “I am afraid that I am still a sceptic,”
he said. “Theories are all very well, but we have to deal
with a hard-headed British jury.”

“\textit{Nous verrons},” answered Holmes, calmly. “You work
your own method, and I shall work mine. I shall be busy
this afternoon, and shall probably return to London by the
evening train.”

“And leave your case unfinished?”

“No, finished.”

“But the mystery?”

“It is solved.”

“Who was the criminal, then?”

“The gentleman I describe.”
%%116

“But who is he?”

“Surely it would not be difficult to find out. This is not
such a populous neighborhood.”

Lestrade shrugged his shoulders. “I am a practical man,”
he said, “and I really cannot undertake to go about the country
looking for a left-handed gentleman with a game-leg. I
should become the laughing-stock of Scotland Yard.”

“All right,” said Holmes, quietly. “I have given you the
chance. Here are your lodgings. Good-bye. I shall drop
you a line before I leave.”

Having left Lestrade at his rooms, we drove to our hotel,
where we found lunch upon the table. Holmes was silent
and buried in thought with a pained expression upon his face,
as one who finds himself in a perplexing position.

“Look here, Watson,” he said, when the cloth was cleared;
“just sit down in this chair and let me preach to you for a
little. I don’t quite know what to do, and I should value your
advice. Light a cigar, and let me expound.”

“Pray do so.”

“Well, now, in considering this case there are two points
about young McCarthy’s narrative which struck us both instantly,
although they impressed me in his favor and you
against him. One was the fact that his father should, according
to his account, cry ‘Cooee!’ before seeing him. The
other was his singular dying reference to a rat. He mumbled
several words, you understand, but that was all that caught
the son’s ear. Now from this double point our research must
commence, and we will begin it by presuming that what the
lad says is absolutely true.”

“What of this ‘Cooee!’ then?”

“Well, obviously it could not have been meant for the son.
The son, as far as he knew, was in Bristol. It was mere
chance that he was within ear-shot. The ‘Cooee!’ was meant
to attract the attention of whoever it was that he had the appointment
with. But ‘Cooee’ is a distinctly Australian cry,
and one which is used between Australians. There is a strong
%%117
presumption that the person whom McCarthy expected to
meet him at Boscombe Pool was some one who had been in
Australia.”

“What of the rat, then?”

Sherlock Holmes took a folded paper from his pocket and
flattened it out on the table. “This is a map of the Colony
of Victoria,” he said. “I wired to Bristol for it last night.”
He put his hand over part of the map. “What do you read?”
he asked.

“ARAT,” I read.

“And now?” He raised his hand.

“BALLARAT.”

“Quite so. That was the word the man uttered, and of
which his son only caught the last two syllables. He was trying
to utter the name of his murderer. So-and-so, of Ballarat.”

“It is wonderful!” I exclaimed.

“It is obvious. And now, you see, I had narrowed the field
down considerably. The possession of a gray garment was a
third point which, granting the son’s statement to be correct,
was a certainty. We have come now out of mere vagueness
to the definite conception of an Australian from Ballarat with
a gray cloak.”

“Certainly.”

“And one who was at home in the district, for the Pool
can only be approached by the farm or by the estate, where
strangers could hardly wander.”

“Quite so.”

“Then comes our expedition of to-day. By an examination
of the ground I gained the trifling details which I gave to that
imbecile Lestrade, as to the personality of the criminal.”

“But how did you gain them?”

“You know my method. It is founded upon the observance
of trifles.”

“His height I know that you might roughly judge from the
length of his stride. His boots, too, might be told from their
traces.”
%%118

“Yes, they were peculiar boots.”

“But his lameness?”

“The impression of his right foot was always less distinct
than his left. He put less weight upon it. Why? Because
he limped -- he was lame.”

“But his left-handedness.”

“You were yourself struck by the nature of the injury as
recorded by the surgeon at the inquest. The blow was struck
from immediately behind, and yet was upon the left side.
Now, how can that be unless it were by a left-handed man?
He had stood behind that tree during the interview between
the father and son. He had even smoked there. I found the
ash of a cigar, which my special knowledge of tobacco ashes
enabled me to pronounce as an Indian cigar. I have, as you
know, devoted some attention to this, and written a little
monograph on the ashes of 140 different varieties of pipe,
cigar, and cigarette tobacco. Having found the ash, I then
looked round and discovered the stump among the moss
where he had tossed it. It was an Indian cigar, of the variety
which are rolled in Rotterdam.”

“And the cigar-holder?”

“I could see that the end had not been in his mouth.
Therefore he used a holder. The tip had been cut off, not
bitten off, but the cut was not a clean one, so I deduced a
blunt pen-knife.”

“Holmes,” I said, “you have drawn a net round this man
from which he cannot escape, and you have saved an innocent
human life as truly as if you had cut the cord which was hanging
him. I see the direction in which all this points. The
culprit is -- ”

“Mr.~John Turner,” cried the hotel waiter, opening the door
of our sitting-room, and ushering in a visitor.

The man who entered was a strange and impressive figure.
His slow, limping step and bowed shoulders gave the appearance
of decrepitude, and yet his hard, deep-lined, craggy features,
and his enormous limbs showed that he was possessed
%%119
of unusual strength of body and of character. His tangled
beard, grizzled hair, and outstanding, drooping eyebrows combined
to give an air of dignity and power to his appearance,
but his face was of an ashen white, while his lips and the corners
of his nostrils were tinged with a shade of blue. It was
clear to me at a glance that he was in the grip of some deadly
and chronic disease.

“Pray sit down on the sofa,” said Holmes, gently. “You
had my note?”

“Yes, the lodge-keeper brought it up. You said that you
wished to see me here to avoid scandal.”

“I thought people would talk if I went to the Hall.”

“And why did you wish to see me?” He looked across at
my companion with despair in his weary eyes, as though his
question was already answered.

“Yes,” said Holmes, answering the look rather than the
words. “It is so. I know all about McCarthy.”

The old man sank his face in his hands. “God help me!”
he cried. “But I would not have let the young man come to
harm. I give you my word that I would have spoken out if it
went against him at the Assizes.”

“I am glad to hear you say so,” said Holmes, gravely.

“I would have spoken now had it not been for my dear
girl. It would break her heart -- it will break her heart when
she hears that I am arrested.”

“It may not come to that,” said Holmes.

“What!”

“I am no official agent. I understand that it was your
daughter who required my presence here, and I am acting in
her interests. Young McCarthy must be got off, however.”

“I am a dying man,” said old Turner. “I have had diabetes
for years. My doctor says it is a question whether I
shall live a month. Yet I would rather die under my own
roof than in a jail.”

Holmes rose and sat down at the table with his pen in his
hand and a bundle of paper before him. “Just tell us the
%%120
truth,” he said. “I shall jot down the facts. You will sign
it, and Watson here can witness it. Then I could produce
your confession at the last extremity to save young McCarthy.
I promise you that I shall not use it unless it is absolutely
needed.”

“It’s as well,” said the old man; “it’s a question whether
I shall live to the Assizes, so it matters little to me, but I
should wish to spare Alice the shock. And now I will make
the thing clear to you; it has been a long time in the acting,
but will not take me long to tell.

“You didn’t know this dead man, McCarthy. He was a
devil incarnate. I tell you that. God keep you out of the
clutches of such a man as he. His grip has been upon me
these twenty years, and he has blasted my life. I’ll tell you
first how I came to be in his power.

“It was in the early sixties at the diggings. I was a young
chap then, hot-blooded and reckless, ready to turn my hand
at anything; I got among bad companions, took to drink, had
no luck with my claim, took to the bush, and in a word became
what you would call over here a highway robber. There
were six of us, and we had a wild, free life of it, sticking up a
station from time to time, or stopping the wagons on the road
to the diggings. Black Jack of Ballarat was the name I went
under, and our party is still remembered in the colony as the
Ballarat Gang.

“One day a gold convoy came down from Ballarat to Melbourne,
and we lay in wait for it and attacked it. There were
six troopers and six of us, so it was a close thing, but we emptied
four of their saddles at the first volley. Three of our
boys were killed, however, before we got the swag. I put my
pistol to the head of the wagon-driver, who was this very man
McCarthy. I wish to the Lord that I had shot him then, but
I spared him, though I saw his wicked little eyes fixed on my
face, as though to remember every feature. We got away with
the gold, became wealthy men, and made our way over to
England without being suspected. There I parted from my
%%121
old pals, and determined to settle down to a quiet and respectable
life. I bought this estate, which chanced to be in the
market, and I set myself to do a little good with my money,
to make up for the way in which I had earned it. I married,
too, and though my wife died young, she left me my dear little
Alice. Even when she was just a baby her wee hand seemed
to lead me down the right path as nothing else had ever done.
In a word, I turned over a new leaf, and did my best to make
up for the past. All was going well when McCarthy laid his
grip upon me.

“I had gone up to town about an investment, and I met
him in Regent Street with hardly a coat to his back or a boot
to his foot.

“\,‘Here we are, Jack,’ says he, touching me on the arm;
‘we’ll be as good as a family to you. There’s two of us, me
and my son, and you can have the keeping of us. If you
don’t -- it’s a fine, law-abiding country is England, and there’s
always a policeman within hail.’

“Well, down they came to the West country, there was no
shaking them off, and there they have lived rent free on my
best land ever since. There was no rest for me, no peace, no
forgetfulness; turn where I would, there was his cunning,
grinning face at my elbow. It grew worse as Alice grew up,
for he soon saw I was more afraid of her knowing my past
than of the police. Whatever he wanted he must have, and
whatever it was I gave him without question, land, money,
houses, until at last he asked a thing which I could not give.
He asked for Alice.

“His son, you see, had grown up, and so had my girl, and
as I was known to be in weak health, it seemed a fine stroke
to him that his lad should step into the whole property. But
there I was firm. I would not have his cursed stock mixed
with mine; not that I had any dislike to the lad, but his blood
was in him, and that was enough. I stood firm. McCarthy
threatened. I braved him to do his worst. We were to meet
at the Pool midway between our houses to talk it over.
%%122

“When I went down there I found him talking with his
son, so I smoked a cigar, and waited behind a tree until he
should be alone. But as I listened to his talk all that was
black and bitter in me seemed to come uppermost. He was
urging his son to marry my daughter with as little regard for
what she might think as if she were a slut from off the streets.
It drove me mad to think that I and all that I held most dear
should be in the power of such a man as this. Could I not
snap the bond? I was already a dying and a desperate man.
Though clear of mind and fairly strong of limb, I knew that
my own fate was sealed. But my memory and my girl!
Both could be saved, if I could but silence that foul tongue.
I did it, Mr.~Holmes. I would do it again. Deeply as I have
sinned, I have led a life of martyrdom to atone for it. But
that my girl should be entangled in the same meshes which
held me was more than I could suffer. I struck him down
with no more compunction than if he had been some foul and
venomous beast. His cry brought back his son; but I had
gained the cover of the wood, though I was forced to go back
to fetch the cloak which I had dropped in my flight. That is
the true story, gentlemen, of all that occurred.”

“Well, it is not for me to judge you,” said Holmes, as the
old man signed the statement which had been drawn out.
“I pray that we may never be exposed to such a temptation.”

“I pray not, sir. And what do you intend to do?”

“In view of your health, nothing. You are yourself aware
that you will soon have to answer for your deed at a higher
court than the Assizes. I will keep your confession, and, if
McCarthy is condemned, I shall be forced to use it. If not,
it shall never be seen by mortal eye; and your secret, whether
you be alive or dead, shall be safe with us.”

“Farewell, then,” said the old man, solemnly. “Your own
death-beds, when they come, will be the easier for the thought
of the peace which you have given to mine.” Tottering and
shaking in all his giant frame, he stumbled slowly from the
room.
%%123

“God help us!” said Holmes, after a long silence. “Why
does fate play such tricks with poor, helpless worms? I never
hear of such a case as this that I do not think of Baxter’s
words, and say, ‘There, but for the grace of God, goes Sherlock
Holmes.’\,”

James McCarthy was acquitted at the Assizes, on the
strength of a number of objections which had been drawn out
by Holmes, and submitted to the defending counsel. Old
Turner lived for seven months after our interview, but he is
now dead; and there is every prospect that the son and
daughter may come to live happily together, in ignorance of
the black cloud which rests upon their past.
%%124

\Chapter{The Five Orange Pips}

\textsc{When} I glance over my notes and records of the
Sherlock Holmes cases between the years ’82 and
’90, I am faced by so many which present strange
and interesting features that it is no easy matter
to know which to choose and which to leave. Some, however,
have already gained publicity through the papers, and
others have not offered a field for those peculiar qualities
which my friend possessed in so high a degree, and which it
is the object of these papers to illustrate. Some, too, have
baffled his analytical skill, and would be, as narratives,
beginnings without an ending, while others have been but partially
cleared up, and have their explanations founded rather upon
conjecture and surmise than on that absolute logical proof
which was so dear to him. There is, however, one of these
last which was so remarkable in its details and so startling in
its results that I am tempted to give some account of it, in
spite of the fact that there are points in connection with it
which never have been, and probably never will be, entirely
cleared up.

The year ’87 furnished us with a long series of cases of
greater or less interest, of which I retain the records. Among
my headings under this one twelve months I find an account
of the adventure of the Paradol Chamber, of the Amateur
Mendicant Society, who held a luxurious club in the lower
vault of a furniture warehouse, of the facts connected with the
loss of the British bark \textit{Sophy Anderson}, of the singular
adventures of the Grice Patersons in the island of Uffa, and
%%125
finally of the Camberwell poisoning case. In the latter, as
may be remembered, Sherlock Holmes was able, by winding
up the dead man’s watch, to prove that it had been wound up
two hours before, and that therefore the deceased had gone to
bed within that time -- a deduction which was of the greatest
importance in clearing up the case. All these I may sketch
out at some future date, but none of them present such singular
features as the strange train of circumstances which I have
now taken up my pen to describe.

It was in the latter days of September, and the equinoctial
gales had set in with exceptional violence. All day the wind
had screamed and the rain had beaten against the windows,
so that even here in the heart of great, hand-made London we
were forced to raise our minds for the instant from the routine
of life, and to recognize the presence of those great elemental
forces which shriek at mankind through the bars of
his civilization, like untamed beasts in a cage. As evening
drew in, the storm grew higher and louder, and the wind cried
and sobbed like a child in the chimney. Sherlock Holmes
sat moodily at one side of the fireplace cross-indexing his
records of crime, while I at the other was deep in one of Clark
Russell’s fine sea-stories, until the howl of the gale from without
seemed to blend with the text, and the splash of the rain
to lengthen out into the long swash of the sea waves. My
wife was on a visit to her mother’s, and for a few days I was
a dweller once more in my old quarters at Baker Street.

“Why,” said I, glancing up at my companion, “that was
surely the bell. Who could come to-night? Some friend of
yours, perhaps?”

“Except yourself I have none,” he answered. “I do not encourage
visitors.”

“A client, then?”

“If so, it is a serious case. Nothing less would bring a
man out on such a day and at such an hour. But I take it
that it is more likely to be some crony of the landlady’s.”

Sherlock Holmes was wrong in his conjecture, however, for
%%126
there came a step in the passage and a tapping at the door.
He stretched out his long arm to turn the lamp away from
himself and towards the vacant chair upon which a new-comer
must sit. “Come in!” said he.

The man who entered was young, some two-and-twenty at
the outside, well-groomed and trimly clad, with something of
refinement and delicacy in his bearing. The steaming umbrella
which he held in his hand, and his long shining waterproof
told of the fierce weather through which he had come.
He looked about him anxiously in the glare of the lamp, and
I could see that his face was pale and his eyes heavy, like
those of a man who is weighed down with some great anxiety.

“I owe you an apology,” he said, raising his golden \textit{pince-nez}
to his eyes. “I trust that I am not intruding. I fear
that I have brought some traces of the storm and rain into
your snug chamber.”

“Give me your coat and umbrella,” said Holmes. “They
may rest here on the hook, and will be dry presently. You
have come up from the south-west, I see.”

“Yes, from Horsham.”

“That clay and chalk mixture which I see upon your toe-caps
is quite distinctive.”

“I have come for advice.”

“That is easily got.”

“And help.”

“That is not always so easy.”

“I have heard of you, Mr.~Holmes. I heard from Major
Prendergast how you saved him in the Tankerville Club
Scandal.”

“Ah, of course. He was wrongfully accused of cheating at
cards.”

“He said that you could solve anything.”

“He said too much.”

“That you are never beaten.”

“I have been beaten four times -- three times by men, and
once by a woman.”
%%127

“But what is that compared with the number of your
successes?”

“It is true that I have been generally successful.”

“Then you may be so with me.”

“I beg that you will draw your chair up to the fire, and favor
me with some details as to your case.”

“It is no ordinary one.”

“None of those which come to me are. I am the last court
of appeal.”

“And yet I question, sir, whether, in all your experience,
you have ever listened to a more mysterious and inexplicable
chain of events than those which have happened in my own
family.”

“You fill me with interest,” said Holmes. “Pray give us
the essential facts from the commencement, and I can afterwards
question you as to those details which seem to me to be
most important.”

The young man pulled his chair up, and pushed his wet
feet out towards the blaze.

“My name,” said he, “is John Openshaw, but my own affairs
have, as far as I can understand it, little to do with this
awful business. It is an hereditary matter; so in order to give
you an idea of the facts, I must go back to the commencement
of the affair.

“You must know that my grandfather had two sons -- my
uncle Elias and my father Joseph. My father had a small
factory at Coventry, which he enlarged at the time of the invention
of bicycling. He was the patentee of the Openshaw
unbreakable tire, and his business met with such success that
he was able to sell it, and to retire upon a handsome
competence.

“My uncle Elias emigrated to America when he was a
young man, and became a planter in Florida, where he was
reported to have done very well. At the time of the war
he fought in Jackson’s army, and afterwards under Hood,
where he rose to be a colonel. When Lee laid down his arms
%%128
my uncle returned to his plantation, where he remained for
three or four years. About 1869 or 1870 he came back to
Europe, and took a small estate in Sussex, near Horsham.
He had made a very considerable fortune in the States, and
his reason for leaving them was his aversion to the negroes,
and his dislike of the Republican policy in extending the
franchise to them. He was a singular man, fierce and quick-tempered,
very foul-mouthed when he was angry, and of a
most retiring disposition. During all the years that he lived
at Horsham I doubt if ever he set foot in the town. He had
a garden and two or three fields round his house, and there
he would take his exercise, though very often for weeks on
end he would never leave his room. He drank a great deal
of brandy, and smoked very heavily, but he would see no
society, and did not want any friends, not even his own brother.

“He didn’t mind me, in fact, he took a fancy to me, for at
the time when he saw me first I was a youngster of twelve or
so. This would be in the year 1878, after he had been eight
or nine years in England. He begged my father to let me
live with him, and he was very kind to me in his way. When
he was sober he used to be fond of playing backgammon and
draughts with me, and he would make me his representative
both with the servants and with the tradespeople, so that by
the time that I was sixteen I was quite master of the house.
I kept all the keys, and could go where I liked and do what I
liked, so long as I did not disturb him in his privacy. There
was one singular exception, however, for he had a single room,
a lumber-room up among the attics, which was invariably
locked, and which he would never permit either me or anyone
else to enter. With a boy’s curiosity I have peeped through
the key-hole, but I was never able to see more than such a
collection of old trunks and bundles as would be expected in
such a room.

“One day -- it was in March, 1883 -- a letter with a foreign
stamp lay upon the table in front of the Colonel’s plate. It
was not a common thing for him to receive letters, for his bills
%%129
were all paid in ready money, and he had no friends of any
sort. ‘From India!’ said he, as he took it up, ‘Pondicherry
postmark! What can this be?’ Opening it hurriedly, out
there jumped five little dried orange pips, which pattered
down upon his plate. I began to laugh at this, but the laugh
was struck from my lips at the sight of his face. His lip had
fallen, his eyes were protruding, his skin the color of putty,
and he glared at the envelope which he still held in his trembling
hand. ‘K. K. K.!’ he shrieked, and then, ‘My God, my
God, my sins have overtaken me!’

“\,‘What is it, uncle?’ I cried.

“\,‘Death,’ said he, and rising from the table he retired to
his room, leaving me palpitating with horror. I took up the
envelope, and saw scrawled in red ink upon the inner flap,
just above the gum, the letter K three times repeated. There
was nothing else save the five dried pips. What could be the
reason of his overpowering terror? I left the breakfast-table,
and as I ascended the stair I met him coming down with an
old rusty key, which must have belonged to the attic, in one
hand, and a small brass box, like a cash-box, in the other.

“\,‘They may do what they like, but I’ll checkmate them
still,’ said he, with an oath. ‘Tell Mary that I shall want a
fire in my room to-day, and send down to Fordham, the Horsham
lawyer.’

“I did as he ordered, and when the lawyer arrived I was
asked to step up to the room. The fire was burning brightly,
and in the grate there was a mass of black, fluffy ashes, as of
burned paper, while the brass box stood open and empty beside
it. As I glanced at the box I noticed, with a start, that
upon the lid were printed the treble K which I had read in the
morning upon the envelope.

“\,‘I wish you, John,’ said my uncle, ‘to witness my will. I
leave my estate, with all its advantages and all its disadvantages
to my brother, your father, whence it will, no doubt, descend
to you. If you can enjoy it in peace, well and good!
If you find you cannot, take my advice, my boy, and leave it
%%130
to your deadliest enemy. I am sorry to give you such a two-edged
thing, but I can’t say what turn things are going to
take. Kindly sign the paper where Mr.~Fordham shows you.’

“I signed the paper as directed, and the lawyer took it
away with him. The singular incident made, as you may
think, the deepest impression upon me, and I pondered over
it, and turned it every way in my mind without being able to
make anything of it. Yet I could not shake off the vague
feeling of dread which it left behind though the sensation
grew less keen as the weeks passed, and nothing happened to
disturb the usual routine of our lives. I could see a change
in my uncle, however. He drank more than ever, and he was
less inclined for any sort of society. Most of his time he
would spend in his room, with the door locked upon the inside,
but sometimes he would emerge in a sort of drunken
frenzy, and would burst out of the house and tear about the
garden with a revolver in his hand, screaming out that he was
afraid of no man, and that he was not to be cooped up, like a
sheep in a pen, by man or devil. When these hot fits were
over, however, he would rush tumultuously in at the door, and
lock and bar it behind him, like a man who can brazen it out
no longer against the terror which lies at the roots of his soul.
At such times I have seen his face, even on a cold day, glisten
with moisture, as though it were new raised from a
basin.

“Well, to come to an end of the matter, Mr.~Holmes, and
not to abuse your patience, there came a night when he made
one of those drunken sallies from which he never came back.
We found him, when we went to search for him, face downward
in a little green-scummed pool, which lay at the foot of
the garden. There was no sign of any violence, and the water
was but two feet deep, so that the jury, having regard to
his known eccentricity, brought in a verdict of suicide. But
I, who knew how he winced from the very thought of death,
had much ado to persuade myself that he had gone out of his
way to meet it. The matter passed, however, and my father
%%131
entered into possession of the estate, and of some £14,000,
which lay to his credit at the bank.”

“One moment,” Holmes interposed. “Your statement is,
I foresee, one of the most remarkable to which I have ever
listened. Let me have the date of the reception by your uncle
of the letter, and the date of his supposed suicide.”

“The latter arrived on March 10, 1883. His death was
seven weeks later, upon the night of May 2d.”

“Thank you. Pray proceed.”

“When my father took over the Horsham property, he, at
my request, made a careful examination of the attic, which
had been always locked up. We found the brass box there,
although its contents had been destroyed. On the inside of
the cover was a paper label, with the initials of K. K. K. repeated
upon it, and ‘Letters, memoranda, receipts, and a register’
written beneath. These, we presume, indicated the
nature of the papers which had been destroyed by Colonel
Openshaw. For the rest, there was nothing of much importance
in the attic, save a great many scattered papers and
note-books bearing upon my uncle’s life in America. Some of
them were of the war time, and showed that he had done his
duty well, and had borne the repute of a brave soldier. Others
were of a date during the reconstruction of the Southern
States, and were mostly concerned with politics, for he had
evidently taken a strong part in opposing the carpet-bag politicians
who had been sent down from the North.

“Well, it was the beginning of ’84 when my father came to
live at Horsham, and all went as well as possible with us until
the January of ’85. On the fourth day after the new year I
heard my father give a sharp cry of surprise as we sat together
at the breakfast-table. There he was, sitting with a
newly-opened envelope in one hand and five dried orange pips
in the outstretched palm of the other one. He had always
laughed at what he called my cock-and-a-bull story about the
Colonel, but he looked very scared and puzzled now that the
same thing had come upon himself.
%%132

“\,‘Why, what on earth does this mean, John?’ he
stammered.

“My heart had turned to lead. ‘It is K. K. K.,’ said I.

“He looked inside the envelope. ‘So it is,’ he cried.
‘Here are the very letters. But what is this written above
them?’

“\,‘Put the papers on the sundial,’ I read, peeping over his
shoulder.

“\,‘What papers? What sundial?’ he asked.

“\,‘The sundial in the garden. There is no other,’ said I;
‘but the papers must be those that are destroyed.’

“\,‘Pooh! said he, gripping hard at his courage. ‘We are
in a civilized land here, and we can’t have tomfoolery of this
kind. Where does the thing come from?’

“\,‘From Dundee,’ I answered, glancing at the post-mark.

“\,‘Some preposterous practical joke,’ said he. ‘What have
I to do with sundials and papers? I shall take no notice of
such nonsense.’

“\,‘I should certainly speak to the police,’ I said.

“\,‘And be laughed at for my pains. Nothing of the sort.’

“\,‘Then let me do so?’

“\,‘No, I forbid you. I won’t have a fuss made about such
nonsense.’

“It was in vain to argue with him, for he was a very obstinate
man. I went about, however, with a heart which was
full of forebodings.

“On the third day after the coming of the letter my father
went from home to visit an old friend of his, Major Freebody,
who is in command of one of the forts upon Portsdown Hill.
I was glad that he should go, for it seemed to me that he was
farther from danger when he was away from home. In that,
however, I was in error. Upon the second day of his absence
I received a telegram from the Major, imploring me to come
at once. My father had fallen over one of the deep chalk-%
pits which abound in the neighborhood, and was lying senseless,
with a shattered skull. I hurried to him, but he passed
%%133
away without having ever recovered his consciousness. He
had, as it appears, been returning from Fareham in the twilight,
and as the country was unknown to him, and the chalk-%
pit unfenced, the jury had no hesitation in bringing in a verdict
of ‘Death from accidental causes.’ Carefully as I examined
every fact connected with his death, I was unable to
find anything which could suggest the idea of murder. There
were no signs of violence, no footmarks, no robbery, no record
of strangers having been seen upon the roads. And yet I
need not tell you that my mind was far from at ease, and that
I was well-nigh certain that some foul plot had been woven
round him.

“In this sinister way I came into my inheritance. You will
ask me why I did not dispose of it? I answer, because I was
well convinced that our troubles were in some way dependent
upon an incident in my uncle’s life, and that the danger would
be as pressing in one house as in another.

“It was in January, ’85, that my poor father met his end,
and two years and eight months have elapsed since then.
During that time I have lived happily at Horsham, and I had
begun to hope that this curse had passed away from the family,
and that it had ended with the last generation. I had begun
to take comfort too soon, however; yesterday morning the
blow fell in the very shape in which it had come upon my
father.”

The young man took from his waistcoat a crumpled envelope,
and, turning to the table, he shook out upon it five
little dried orange pips.

“This is the envelope,” he continued. “The post-mark is
London -- eastern division. Within are the very words which
were upon my father’s last message: ‘K. K. K.’; and then
‘Put the papers on the sundial.’\,”

“What have you done?” asked Holmes.

“Nothing.”

“Nothing?”

“To tell the truth” -- he sank his face into his thin, white
%%134
hands -- “I have felt helpless. I have felt like one of those
poor rabbits when the snake is writhing towards it. I seem
to be in the grasp of some resistless, inexorable evil, which
no foresight and no precautions can guard against.”

“Tut! tut!” cried Sherlock Holmes. “You must act, man,
or you are lost. Nothing but energy can save you. This is
no time for despair.”

“I have seen the police.”

“Ah!”

“But they listened to my story with a smile. I am convinced
that the inspector has formed the opinion that the
letters are all practical jokes, and that the deaths of my relations
were really accidents, as the jury stated, and were not
to be connected with the warnings.”

Holmes shook his clenched hands in the air. “Incredible
imbecility!” he cried.

“They have, however, allowed me a policeman, who may
remain in the house with me.”

“Has he come with you to-night?”

“No. His orders were to stay in the house.”

Again Holmes raved in the air.

“Why did you come to me?” he said; “and, above all,
why did you not come at once?”

“I did not know. It was only to-day that I spoke to Major
Prendergast about my troubles, and was advised by him to
come to you.”

“It is really two days since you had the letter. We should
have acted before this. You have no further evidence, I suppose,
than that which you have placed before us -- no suggestive
detail which might help us?”

“There is one thing,” said John Openshaw. He rummaged
in his coat pocket, and drawing out a piece of discolored,
blue-tinted paper, he laid it out upon the table. “I have
some remembrance,” said he, “that on the day when my uncle
burned the papers I observed that the small, unburned margins
which lay amid the ashes were of this particular color.
%%135
I found this single sheet upon the floor of his room, and I am
inclined to think that it may be one of the papers which has,
perhaps, fluttered out from among the others, and in that way
have escaped destruction. Beyond the mention of pips, I do
not see that it helps us much. I think myself that it is a
page from some private diary. The writing is undoubtedly
my uncle’s.”

Holmes moved the lamp, and we both bent over the sheet
of paper, which showed by its ragged edge that it had indeed
been torn from a book. It was headed, “March, 1869,” and
beneath were the following enigmatical notices:

“4th. Hudson came. Same old platform.

“7th. Set the pips on McCauley, Paramore, and John
Swain, of St.~Augustine.

“9th. McCauley cleared.

“10th. John Swain cleared.

“12th. Visited Paramore. All well.”

“Thank you!” said Holmes, folding up the paper, and returning
it to our visitor. “And now you must on no account
lose another instant. We cannot spare time even to discuss
what you have told me. You must get home instantly and
act.”

“What shall I do?”

“There is but one thing to do. It must be done at once.
You must put this piece of paper which you have shown us
into the brass box which you have described. You must also
put in a note to say that all the other papers were burned by
your uncle, and that this is the only one which remains. You
must assert that in such words as will carry conviction with
them. Having done this, you must at once put the box out
upon the sundial, as directed. Do you understand?”

“Entirely.”

“Do not think of revenge, or anything of the sort, at present.
I think that we may gain that by means of the law; but
we have our web to weave, while theirs is already woven. The
first consideration is to remove the pressing danger which
%%136
threatens you. The second is to clear up the mystery and to
punish the guilty parties.”

“I thank you,” said the young man, rising, and pulling on
his overcoat. “You have given me fresh life and hope. I
shall certainly do as you advise.”

“Do not lose an instant. And, above all, take care of yourself
in the meanwhile, for I do not think that there can be a
doubt that you are threatened by a very real and imminent
danger. How do you go back?”

“By train from Waterloo.”

“It is not yet nine. The streets will be crowded, so I trust
that you may be in safety. And yet you cannot guard yourself
too closely.”

“I am armed.”

“That is well. To-morrow I shall set to work upon your
case.”

“I shall see you at Horsham, then?”

“No, your secret lies in London. It is there that I shall
seek it.”

“Then I shall call upon you in a day, or in two days, with
news as to the box and the papers. I shall take your advice
in every particular.” He shook hands with us, and took his
leave. Outside the wind still screamed, and the rain splashed
and pattered against the windows. This strange, wild story
seemed to have come to us from amid the mad elements -- blown
in upon us like a sheet of sea-weed in a gale -- and now
to have been reabsorbed by them once more.

Sherlock Holmes sat for some time in silence, with his head
sunk forward and his eyes bent upon the red glow of the fire.
Then he lit his pipe, and leaning back in his chair he watched
the blue smoke-rings as they chased each other up to the
ceiling.

“I think, Watson,” he remarked at last, “that of all our
cases we have had none more fantastic than this.”

“Save, perhaps, the Sign of Four.”

“Well, yes. Save, perhaps, that. And yet this John
%%137
Openshaw seems to me to be walking amid even greater perils than
did the Sholtos.”

“But have you,” I asked, “formed any definite conception
as to what these perils are?”

“There can be no question as to their nature,” he answered.

“Then what are they? Who is this K. K. K., and why
does he pursue this unhappy family?”

Sherlock Holmes closed his eyes and placed his elbows
upon the arms of his chair, with his finger-tips together.
“The ideal reasoner,” he remarked, “would, when he had
once been shown a single fact in all its bearings, deduce
from it not only all the chain of events which led up to it, but
also all the results which would follow from it. As Cuvier
could correctly describe a whole animal by the contemplation
of a single bone, so the observer who has thoroughly understood
one link in a series of incidents, should be able to accurately
state all the other ones, both before and after. We
have not yet grasped the results which the reason alone can
attain to. Problems may be solved in the study which have
baffled all those who have sought a solution by the aid of
their senses. To carry the art, however, to its highest pitch,
it is necessary that the reasoner should be able to utilize all
the facts which have come to his knowledge; and this in itself
implies, as you will readily see, a possession of all knowledge,
which, even in these days of free education and encyclopædias,
is a somewhat rare accomplishment. It is not so impossible,
however, that a man should possess all knowledge which is
likely to be useful to him in his work, and this I have endeavored
in my case to do. If I remember rightly, you on one
occasion, in the early days of our friendship, defined my limits
in a very precise fashion.”

“Yes,” I answered, laughing. “It was a singular document.
Philosophy, astronomy, and politics were marked at
zero, I remember. Botany variable, geology profound as regards
the mud-stains from any region within fifty miles of
town, chemistry eccentric, anatomy unsystematic, sensational
%%138
literature and crime records unique, violin-player, boxer,
swordsman, lawyer, and self-poisoner by cocaine and tobacco.
Those, I think, were the main points of my analysis.”

Holmes grinned at the last item. “Well,” he said, “I say
now, as I said then, that a man should keep his little brain-attic
stocked with all the furniture that he is likely to use, and
the rest he can put away in the lumber-room of his library,
where he can get it if he wants it. Now, for such a case as
the one which has been submitted to us to-night, we need certainly
to muster all our resources. Kindly hand me down the
letter K of the American Encyclopædia which stands upon
the shelf beside you. Thank you. Now let us consider the
situation, and see what may be deduced from it. In the first
place, we may start with a strong presumption that Colonel
Openshaw had some very strong reason for leaving America.
Men at his time of life do not change all their habits, and
exchange willingly the charming climate of Florida for the lonely
life of an English provincial town. His extreme love of solitude
in England suggests the idea that he was in fear of some
one or something, so we may assume as a working hypothesis
that it was fear of some one or something which drove him
from America. As to what it was he feared, we can only deduce
that by considering the formidable letters which were
received by himself and his successors. Did you remark the
post-marks of those letters?”

“The first was from Pondicherry, the second from Dundee,
and the third from London.”

“From East London. What do you deduce from that?”

“They are all seaports. That the writer was on board of
a ship.”

“Excellent. We have already a clew. There can be no
doubt that the probability -- the strong probability -- is that the
writer was on board of a ship. And now let us consider another
point. In the case of Pondicherry, seven weeks elapsed
between the threat and its fulfilment, in Dundee it was only
some three or four days. Does that suggest anything?”
%%139

“A greater distance to travel.”

“But the letter had also a greater distance to come.”

“Then I do not see the point.”

“There is at least a presumption that the vessel in which
the man or men are is a sailing-ship. It looks as if they always
sent their singular warning or token before them when
starting upon their mission. You see how quickly the deed
followed the sign when it came from Dundee. If they had
come from Pondicherry in a steamer they would have arrived
almost as soon as their letter. But as a matter of fact seven
weeks elapsed. I think that those seven weeks represented
the difference between the mail-boat which brought the letter,
and the sailing-vessel which brought the writer.”

“It is possible.”

“More than that. It is probable. And now you see the
deadly urgency of this new case, and why I urged young Openshaw
to caution. The blow has always fallen at the end of
the time which it would take the senders to travel the distance.
But this one comes from London, and therefore we cannot
count upon delay.”

“Good God!” I cried; “what can it mean, this relentless
persecution?”

“The papers which Openshaw carried are obviously of vital
importance to the person or persons in the sailing-ship. I
think that it is quite clear that there must be more than one
of them. A single man could not have carried out two deaths
in such a way as to deceive a coroner’s jury. There must
have been several in it, and they must have been men of resource
and determination. Their papers they mean to have,
be the holder of them who it may. In this way you see K.
K. K. ceases to be the initials of an individual, and becomes
the badge of a society.”

“But of what society?”

“Have you never -- ” said Sherlock Holmes, bending forward
and sinking his voice -- “have you never heard of the
Ku Klux Klan?”
%%140

“I never have.”

Holmes turned over the leaves of the book upon his knee.
“Here it is,” said he, presently, “\,‘Ku Klux Klan. A name
derived from the fanciful resemblance to the sound produced
by cocking a rifle. This terrible secret society was formed by
some ex-Confederate soldiers in the Southern States after the
Civil War, and it rapidly formed local branches in different
parts of the country, notably in Tennessee, Louisiana, the
Carolinas, Georgia, and Florida. Its power was used for political
purposes, principally for the terrorizing of the negro
voters, and the murdering and driving from the country of
those who were opposed to its views. Its outrages were usually
preceded by a warning sent to the marked man in some
fantastic but generally recognized shape -- a sprig of oak-leaves
in some parts, melon seeds or orange pips in others. On receiving
this the victim might either openly abjure his former
ways, or might fly from the country. If he braved the matter
out, death would unfailingly come upon him, and usually in
some strange and unforeseen manner. So perfect was the organization
of the society, and so systematic its methods, that
there is hardly a case upon record where any man succeeded
in braving it with impunity, or in which any of its outrages
were traced home to the perpetrators. For some years
the organization flourished, in spite of the efforts of the
United States Government and of the better classes of the
community in the South. Eventually, in the year 1869,
the movement rather suddenly collapsed, although there
have been sporadic outbreaks of the same sort since that
date.’

“You will observe,” said Holmes, laying down the volume,
“that the sudden breaking up of the society was coincident
with the disappearance of Openshaw from America with their
papers. It may well have been cause and effect. It is no
wonder that he and his family have some of the more implacable
spirits upon their track. You can understand that this
register and diary may implicate some of the first men in the
%%141
South, and that there may be many who will not sleep easy at
night until it is recovered.”

“Then the page we have seen -- ”

“Is such as we might expect. It ran, if I remember right,
‘sent the pips to A, B, and C,’ -- that is, sent the society’s
warning to them. Then there are successive entries that A
and B cleared, or left the country, and finally that C was visited,
with, I fear, a sinister result for C. Well, I think, Doctor,
that we may let some light into this dark place, and I
believe that the only chance young Openshaw has in the
mean time is to do what I have told him. There is nothing
more to be said or to be done to-night, so hand me over my
violin, and let us try to forget for half an hour the miserable
weather and the still more miserable ways of our fellow-%
men.”

\strut

It had cleared in the morning, and the sun was shining
with a subdued brightness through the dim veil which hangs
over the great city. Sherlock Holmes was already at breakfast
when I came down.

“You will excuse me for not waiting for you,” said he; “I
have, I foresee, a very busy day before me in looking into this
case of young Openshaw’s.”

“What steps will you take?” I asked.

“It will very much depend upon the results of my first inquiries.
I may have to go down to Horsham, after all.”

“You will not go there first?”

“No, I shall commence with the city. Just ring the bell,
and the maid will bring up your coffee.”

As I waited, I lifted the unopened newspaper from the table
and glanced my eye over it. It rested upon a heading which
sent a chill to my heart.

“Holmes,” I cried, “you are too late.”

“Ah!” said he, laying down his cup, “I feared as much.
How was it done?” He spoke calmly, but I could see that
he was deeply moved.
%%142

“My eye caught the name of Openshaw, and the heading,
‘Tragedy near Waterloo Bridge.’ Here is the account: ‘Between
nine and ten last night Police-constable Cook, of the H
Division, on duty near Waterloo Bridge, heard a cry for help
and a splash in the water. The night, however, was extremely
dark and stormy, so that, in spite of the help of several
passers-by, it was quite impossible to effect a rescue. The
alarm, however, was given, and, by the aid of the water-police,
the body was eventually recovered. It proved to be that of a
young gentleman whose name, as it appears from an envelope
which was found in his pocket, was John Openshaw, and
whose residence is near Horsham. It is conjectured that he
may have been hurrying down to catch the last train from
Waterloo Station, and that in his haste and the extreme darkness
he missed his path and walked over the edge of one of
the small landing-places for river steamboats. The body exhibited
no traces of violence, and there can be no doubt that
the deceased had been the victim of an unfortunate accident,
which should have the effect of calling the attention of the
authorities to the condition of the river-side landing-%
stages.’\,”

We sat in silence for some minutes, Holmes more depressed
and shaken than I had ever seen him.

“That hurts my pride, Watson,” he said, at last. “It is a
petty feeling, no doubt, but it hurts my pride. It becomes a
personal matter with me now, and, if God sends me health, I
shall set my hand upon this gang. That he should come
to me for help, and that I should send him away to his
death -- !” He sprang from his chair and paced about the
room in uncontrollable agitation, with a flush upon his sallow
cheeks, and a nervous clasping and unclasping of his long,
thin hands.

“They must be cunning devils,” he exclaimed, at last.
“How could they have decoyed him down there? The Embankment
is not on the direct line to the station. The bridge,
no doubt, was too crowded, even on such a night, for their
%%143
%%“\,‘HOLMES,’ I CRIED, ‘YOU ARE TOO LATE’\,”
%%144
purpose. Well, Watson, we shall see who will win in the
long run. I am going out now!”

“To the police?”

“No; I shall be my own police. When I have spun the
web they may take the flies, but not before.”

All day I was engaged in my professional work, and it was
late in the evening before I returned to Baker Street. Sherlock
Holmes had not come back yet. It was nearly ten o’clock
before he entered, looking pale and worn. He walked up to
the sideboard, and, tearing a piece from the loaf, he devoured
it voraciously, washing it down with a long draught of water.

“You are hungry,” I remarked.

“Starving. It had escaped my memory. I have had nothing
since breakfast.”

“Nothing?”

“Not a bite. I had no time to think of it.”

“And how have you succeeded?”

“Well.”

“You have a clew?”

“I have them in the hollow of my hand. Young Openshaw
shall not long remain unavenged. Why, Watson, let us put
their own devilish trade-mark upon them. It is well thought
of!”

“What do you mean?”

He took an orange from the cupboard, and, tearing it to
pieces, he squeezed out the pips upon the table. Of these he
took five, and thrust them into an envelope. On the inside of
the flap he wrote “S. H. for J. O.” Then he sealed it and
addressed it to “Captain James Calhoun, Bark \textit{Lone Star},
Savannah, Georgia.”

“That will await him when he enters port,” said he, chuckling.
“It may give him a sleepless night. He will find it as
sure a precursor of his fate as Openshaw did before him.”

“And who is this Captain Calhoun?”

“The leader of the gang. I shall have the others, but he
first.”
%%146

“How did you trace it, then?”

He took a large sheet of paper from his pocket, all covered
with dates and names.

“I have spent the whole day,” said he, “over Lloyd’s registers
and the files of the old papers, following the future career
of every vessel which touched at Pondicherry in January
and February in ’83. There were thirty-six ships of fair tonnage
which were reported there during those months. Of
these, one, the \textit{Lone Star}, instantly attracted my attention,
since, although it was reported as having cleared from London,
the name is that which is given to one of the States of
the Union.”

“Texas, I think.”

“I was not and am not sure which; but I knew that the
ship must have an American origin.”

“What then?”

“I searched the Dundee records, and when I found that the
bark \textit{Lone Star} was there in January, ’85, my suspicion became
a certainty. I then inquired as to the vessels which lay at
present in the port of London.”

“Yes?”

“The \textit{Lone Star} had arrived here last week. I went down
to the Albert Dock, and found that she had been taken down
the river by the early tide this morning, homeward bound to
Savannah. I wired to Gravesend, and learned that she had
passed some time ago; and as the wind is easterly, I have no
doubt that she is now past the Goodwins, and not very far
from the Isle of Wight.”

“What will you do, then?”

“Oh, I have my hand upon him. He and the two mates,
are, as I learn, the only native-born Americans in the ship.
The others are Finns and Germans. I know, also, that they
were all three away from the ship last night. I had it from
the stevedore who has been loading their cargo. By the time
that their sailing-ship reaches Savannah the mail-boat will
have carried this letter, and the cable will have informed the
%%147
police of Savannah that these three gentlemen are badly wanted
here upon a charge of murder.”

There is ever a flaw, however, in the best laid of human
plans, and the murderers of John Openshaw were never to receive
the orange pips which would show them that another, as
cunning and as resolute as themselves, was upon their track.
Very long and very severe were the equinoctial gales that
year. We waited long for news of the \textit{Lone Star} of Savannah,
but none ever reached us. We did at last hear that
somewhere far out in the Atlantic a shattered stern-post of a
boat was seen swinging in the trough of a wave, with the letters
“L. S.” carved upon it, and that is all which we shall
ever know of the fate of the \textit{Lone Star}.
%%148

\Chapter{The Man With The Twisted Lip}

\textsc{Isa} Whitney, brother of the late Elias Whitney,
D.D., Principal of the Theological College
of St.~George’s, was much addicted to opium.
The habit grew upon him, as I understand, from
some foolish freak when he was at college; for having read
De Quincey’s description of his dreams and sensations, he
had drenched his tobacco with laudanum in an attempt to
produce the same effects. He found, as so many more have
done, that the practice is easier to attain than to get rid of,
and for many years he continued to be a slave to the drug,
an object of mingled horror and pity to his friends and relatives.
I can see him now, with yellow, pasty face, drooping
lids, and pin-point pupils, all huddled in a chair, the wreck
and ruin of a noble man.

One night -- it was in June, ’89 -- there came a ring to my
bell, about the hour when a man gives his first yawn and
glances at the clock. I sat up in my chair, and my wife laid
her needle-work down in her lap and made a little face of
disappointment.

“A patient!” said she. “You’ll have to go out.”

I groaned, for I was newly come back from a weary day.

We heard the door open, a few hurried words, and then
quick steps upon the linoleum. Our own door flew open, and
a lady, clad in some dark-colored stuff, with a black veil, entered
the room.

“You will excuse my calling so late,” she began, and then,
suddenly losing her self-control, she ran forward, threw her
%%149
arms about my wife’s neck, and sobbed upon her shoulder.
“Oh, I’m in such trouble!” she cried; “I do so want a little
help.”

“Why,” said my wife, pulling up her veil, “it is Kate Whitney.
How you startled me, Kate! I had not an idea who
you were when you came in.”

“I didn’t know what to do, so I came straight to you.”
That was always the way. Folk who were in grief came to
my wife like birds to a light-house.

“It was very sweet of you to come. Now, you must have
some wine and water, and sit here comfortably and tell us all
about it. Or should you rather that I sent James off to bed?”

“Oh, no, no! I want the Doctor’s advice and help, too.
It’s about Isa. He has not been home for two days. I am
so frightened about him!”

It was not the first time that she had spoken to us of her
husband’s trouble, to me as a doctor, to my wife as an old
friend and school companion. We soothed and comforted
her by such words as we could find. Did she know where
her husband was? Was it possible that we could bring him
back to her?

It seemed that it was. She had the surest information that
of late he had, when the fit was on him, made use of an opium
den in the farthest east of the city. Hitherto his orgies had
always been confined to one day, and he had come back,
twitching and shattered, in the evening. But now the spell
had been upon him eight-and-forty hours, and he lay there,
doubtless among the dregs of the docks, breathing in the
poison or sleeping off the effects. There he was to be found,
she was sure of it, at the “Bar of Gold,” in Upper Swandam
Lane. But what was she to do? How could she, a young
and timid woman, make her way into such a place, and pluck
her husband out from among the ruffians who surrounded
him?

There was the case, and of course there was but one way
out of it. Might I not escort her to this place? And then,
%%150
as a second thought, why should she come at all? I was Isa
Whitney’s medical adviser, and as such I had influence over
him. I could manage it better if I were alone. I promised
her on my word that I would send him home in a cab within
two hours if he were indeed at the address which she had
given me. And so in ten minutes I had left my arm-chair
and cheery sitting-room behind me, and was speeding eastward
in a hansom on a strange errand, as it seemed to me at
the time, though the future only could show how strange it
was to be.

But there was no great difficulty in the first stage of my
adventure. Upper Swandam Lane is a vile alley lurking behind
the high wharves which line the north side of the river
to the east of London Bridge. Between a slop-shop and a
gin-shop, approached by a steep flight of steps leading down
to a black gap like the mouth of a cave, I found the den of
which I was in search. Ordering my cab to wait, I passed
down the steps, worn hollow in the centre by the ceaseless
tread of drunken feet, and by the light of a flickering oil-lamp
above the door I found the latch, and made my way into a
long, low room, thick and heavy with the brown opium smoke,
and terraced with wooden berths, like the forecastle of an
emigrant ship.

Through the gloom one could dimly catch a glimpse of
bodies lying in strange fantastic poses, bowed shoulders, bent
knees, heads thrown back and chins pointing upward, with
here and there a dark, lack-lustre eye turned upon the newcomer.
Out of the black shadows there glimmered little red
circles of light, now bright, now faint, as the burning poison
waxed or waned in the bowls of the metal pipes. The most
lay silent, but some muttered to themselves, and others talked
together in a strange, low, monotonous voice, their conversation
coming in gushes, and then suddenly tailing off into silence,
each mumbling out his own thoughts, and paying little
heed to the words of his neighbor. At the farther end was a
small brazier of burning charcoal, beside which on a three-%
%%151
legged wooden stool there sat a tall, thin old man, with his
jaw resting upon his two fists, and his elbows upon his knees,
staring into the fire.

As I entered, a sallow Malay attendant had hurried up with
a pipe for me and a supply of the drug, beckoning me to an
empty berth.

“Thank you. I have not come to stay,” said I. “There
is a friend of mine here, Mr.~Isa Whitney, and I wish to speak
with him.”

There was a movement and an exclamation from my right,
and, peering through the gloom, I saw Whitney, pale, haggard,
and unkempt, staring out at me.

“My God! It’s Watson,” said he. He was in a pitiable
state of reaction, with every nerve in a twitter. “I say, Watson,
what o’clock is it?”

“Nearly eleven.”

“Of what day?”

“Of Friday, June 19th.”

“Good heavens! I thought it was Wednesday. It is Wednesday.
What d’you want to frighten a chap for?” He sank
his face onto his arms, and began to sob in a high treble key.

“I tell you that it is Friday, man. Your wife has been
waiting this two days for you. You should be ashamed of
yourself!”

“So I am. But you’ve got mixed, Watson, for I have only
been here a few hours, three pipes, four pipes -- I forget how
many. But I’ll go home with you. I wouldn’t frighten Kate -- poor
little Kate. Give me your hand! Have you a cab?”

“Yes, I have one waiting.”

“Then I shall go in it. But I must owe something. Find
what I owe, Watson. I am all off color. I can do nothing
for myself.”

I walked down the narrow passage between the double row
of sleepers, holding my breath to keep out the vile, stupefying
fumes of the drug, and looking about for the manager. As I
passed the tall man who sat by the brazier I felt a sudden
%%152
pluck at my skirt, and a low voice whispered, “Walk past me,
and then look back at me.” The words fell quite distinctly
upon my ear. I glanced down. They could only have come
from the old man at my side, and yet he sat now as absorbed
as ever, very thin, very wrinkled, bent with age, an opium pipe
dangling down from between his knees, as though it had
dropped in sheer lassitude from his fingers. I took two steps
forward and looked back. It took all my self-control to prevent
me from breaking out into a cry of astonishment. He
had turned his back so that none could see him but I. His
form had filled out, his wrinkles were gone, the dull eyes had
regained their fire, and there, sitting by the fire, and grinning
at my surprise, was none other than Sherlock Holmes. He
made a slight motion to me to approach him, and instantly,
as he turned his face half round to the company once more,
subsided into a doddering, loose-lipped senility.

“Holmes!” I whispered, “what on earth are you doing in
this den?”

“As low as you can,” he answered; “I have excellent ears.
If you would have the great kindness to get rid of that sottish
friend of yours I should be exceedingly glad to have a little
talk with you.”

“I have a cab outside.”

“Then pray send him home in it. You may safely trust
him, for he appears to be too limp to get into any mischief.
I should recommend you also to send a note by the cabman
to your wife to say that you have thrown in your lot with me.
If you will wait outside, I shall be with you in five minutes.”

It was difficult to refuse any of Sherlock Holmes’s requests,
for they were always so exceedingly definite, and put forward
with such a quiet air of mastery. I felt, however, that when
Whitney was once confined in the cab my mission was practically
accomplished, and for the rest, I could not wish anything
better than to be associated with my friend in one of
those singular adventures which were the normal condition of
his existence. In a few minutes I had written my note, paid
%%153
Whitney’s bill, led him out to the cab, and seen him driven
through the darkness. In a very short time a decrepit figure
had emerged from the opium den, and I was walking down
the street with Sherlock Holmes. For two streets he shuffled
along with a bent back and an uncertain foot. Then, glancing
quickly round, he straightened himself out and burst into a
hearty fit of laughter.

“I suppose, Watson,” said he, “that you imagine that I
have added opium-smoking to cocaine injections, and all the
other little weaknesses on which you have favored me with
your medical views.”

“I was certainly surprised to find you there.”

“But not more so than I to find you.”

“I came to find a friend.”

“And I to find an enemy.”

“An enemy?”

“Yes; one of my natural enemies, or, shall I say, my natural
prey. Briefly, Watson, I am in the midst of a very remarkable
inquiry, and I have hoped to find a clew in the incoherent
ramblings of these sots, as I have done before now.
Had I been recognized in that den my life would not have
been worth an hour’s purchase; for I have used it before now
for my own purposes, and the rascally Lascar who runs it has
sworn to have vengeance upon me. There is a trap-door at
the back of that building, near the corner of Paul’s Wharf,
which could tell some strange tales of what has passed through
it upon the moonless nights.”

“What! You do not mean bodies?”

“Aye, bodies, Watson. We should be rich men if we had
£1000 for every poor devil who has been done to death in
that den. It is the vilest murder-trap on the whole river-side,
and I fear that Neville St.~Clair has entered it never to leave
it more. But our trap should be here.” He put his two forefingers
between his teeth and whistled shrilly -- a signal which
was answered by a similar whistle from the distance, followed
shortly by the rattle of wheels and the clink of horses’ hoofs.
%%154

“Now, Watson,” said Holmes, as a tall dog-cart dashed up
through the gloom, throwing out two golden tunnels of yellow
light from its side lanterns. “You’ll come with me, won’t
you?”

“If I can be of use.”

“Oh, a trusty comrade is always of use; and a chronicler
still more so. My room at ‘The Cedars’ is a double-bedded
one.”

“\,‘The Cedars?’\,”

“Yes; that is Mr.~St.~Clair’s house. I am staying there
while I conduct the inquiry.”

“Where is it, then?”

“Near Lee, in Kent. We have a seven-mile drive before
us.”

“But I am all in the dark.”

“Of course you are. You’ll know all about it presently.
Jump up here. All right, John; we shall not need you.
Here’s half a crown. Look out for me to-morrow, about
eleven. Give her her head. So long, then!”

He flicked the horse with his whip, and we dashed away
through the endless succession of sombre and deserted streets,
which widened gradually, until we were flying across a broad
balustraded bridge, with the murky river flowing sluggishly
beneath us. Beyond lay another dull wilderness of bricks
and mortar, its silence broken only by the heavy, regular footfall
of the policeman, or the songs and shouts of some belated
party of revellers. A dull wrack was drifting slowly across
the sky, and a star or two twinkled dimly here and there
through the rifts of the clouds. Holmes drove in silence,
with his head sunk upon his breast, and the air of a man who
is lost in thought, while I sat beside him, curious to learn
what this new quest might be which seemed to tax his powers
so sorely, and yet afraid to break in upon the current of his
thoughts. We had driven several miles, and were beginning
to get to the fringe of the belt of suburban villas, when he
shook himself, shrugged his shoulders, and lit up his pipe
%%155
with the air of a man who has satisfied himself that he is acting
for the best.

“You have a grand gift of silence, Watson,” said he. “It
makes you quite invaluable as a companion. ’Pon my word,
it is a great thing for me to have some one to talk to, for my
own thoughts are not over pleasant. I was wondering what I
should say to this dear little woman to-night when she meets
me at the door.”

“You forget that I know nothing about it.”

“I shall just have time to tell you the facts of the case before
we get to Lee. It seems absurdly simple, and yet, somehow,
I can get nothing to go upon. There’s plenty of thread,
no doubt, but I can’t get the end of it into my hand. Now,
I’ll state the case clearly and concisely to you, Watson, and
maybe you can see a spark where all is dark to me.”

“Proceed, then.”

“Some years ago -- to be definite, in May, 1884 -- there came
to Lee a gentleman, Neville St.~Clair by name, who appeared
to have plenty of money. He took a large villa, laid out the
grounds very nicely, and lived generally in good style. By
degrees he made friends in the neighborhood, and in 1887 he
married the daughter of a local brewer, by whom he now has
two children. He had no occupation, but was interested in
several companies, and went into town as a rule in the morning,
returning by the 5.14 from Cannon Street every night.
Mr.~St.~Clair is now thirty-seven years of age, is a man of
temperate habits, a good husband, a very affectionate father,
and a man who is popular with all who know him. I may
add that his whole debts at the present moment, as far as we
have been able to ascertain, amount to £88 10\textit{s.}, while he
has £220 standing to his credit in the Capital and Counties
Bank. There is no reason, therefore, to think that money
troubles have been weighing upon his mind.

“Last Monday Mr.~Neville St.~Clair went into town rather
earlier than usual, remarking before he started that he had
two important commissions to perform, and that he would
%%156
bring his little boy home a box of bricks. Now, by the merest
chance, his wife received a telegram upon this same Monday,
very shortly after his departure, to the effect that a small parcel
of considerable value which she had been expecting was
waiting for her at the offices of the Aberdeen Shipping Company.
Now, if you are well up in your London, you will
know that the offices of the company is in Fresno Street,
which branches out of Upper Swandam Lane, where you
found me to-night. Mrs.~St.~Clair had her lunch, started for
the city, did some shopping, proceeded to the company’s office,
got her packet, and found herself at exactly 4.35 walking
through Swandam Lane on her way back to the station.
Have you followed me so far?”

“It is very clear.”

“If you remember, Monday was an exceedingly hot day,
and Mrs.~St.~Clair walked slowly, glancing about in the hope
of seeing a cab, as she did not like the neighborhood in
which she found herself. While she was walking in this way
down Swandam Lane, she suddenly heard an ejaculation or
cry, and was struck cold to see her husband looking down at
her, and, as it seemed to her, beckoning to her from a second-floor
window. The window was open, and she distinctly saw
his face, which she describes as being terribly agitated. He
waved his hands frantically to her, and then vanished from
the window so suddenly that it seemed to her that he had
been plucked back by some irresistible force from behind.
One singular point which struck her quick feminine eye was
that, although he wore some dark coat, such as he had started
to town in, he had on neither collar nor necktie.

“Convinced that something was amiss with him, she rushed
down the steps -- for the house was none other than the opium
den in which you found me to-night -- and, running through
the front room, she attempted to ascend the stairs which led
to the first floor. At the foot of the stairs, however, she met
this Lascar scoundrel of whom I have spoken, who thrust her
back, and, aided by a Dane, who acts as assistant there,
%%157
%%“AT THE FOOT OF THE STAIRS SHE MET THIS LASCAR SCOUNDREL.”
%%158
pushed her out into the street. Filled with the most maddening
doubts and fears, she rushed down the lane, and, by
rare good-fortune, met, in Fresno Street, a number of constables
with an inspector, all on their way to their beat. The
inspector and two men accompanied her back, and, in spite of
the continued resistance of the proprietor, they made their
way to the room in which Mr.~St.~Clair had last been seen.
There was no sign of him there. In fact, in the whole of that
floor there was no one to be found, save a crippled wretch of
hideous aspect, who, it seems, made his home there. Both he
and the Lascar stoutly swore that no one else had been in the
front room during the afternoon. So determined was their
denial that the inspector was staggered, and had almost come
to believe that Mrs.~St.~Clair had been deluded, when, with a
cry, she sprang at a small deal box which lay upon the table,
and tore the lid from it. Out there fell a cascade of children’s
bricks. It was the toy which he had promised to
bring home.

“This discovery, and the evident confusion which the cripple
showed, made the inspector realize that the matter was
serious. The rooms were carefully examined, and results all
pointed to an abominable crime. The front room was plainly
furnished as a sitting-room, and led into a small bedroom,
which looked out upon the back of one of the wharves. Between
the wharf and the bedroom window is a narrow strip,
which is dry at low tide, but is covered at high tide with at
least four and a half feet of water. The bedroom window was
a broad one, and opened from below. On examination traces
of blood were to be seen upon the window-sill, and several
scattered drops were visible upon the wooden floor of the bedroom.
Thrust away behind a curtain in the front room were
all the clothes of Mr.~Neville St.~Clair, with the exception of
his coat. His boots, his socks, his hat, and his watch -- all
were there. There were no signs of violence upon any of
these garments, and there were no other traces of Mr.~Neville
St.~Clair. Out of the window he must apparently have gone,
%%160
for no other exit could be discovered, and the ominous bloodstains
upon the sill gave little promise that he could save himself
by swimming, for the tide was at its very highest at the
moment of the tragedy.

“And now as to the villains who seemed to be immediately
implicated in the matter. The Lascar was known to be a
man of the vilest antecedents, but as, by Mrs.~St.~Clair’s story,
he was known to have been at the foot of the stair within
a very few seconds of her husband’s appearance at the window,
he could hardly have been more than an accessory to
the crime. His defense was one of absolute ignorance, and
he protested that he had no knowledge as to the doings of
Hugh Boone, his lodger, and that he could not account in any
way for the presence of the missing gentleman’s clothes.

“So much for the Lascar manager. Now for the sinister
cripple who lives upon the second floor of the opium den,
and who was certainly the last human being whose eyes rested
upon Neville St.~Clair. His name is Hugh Boone, and his
hideous face is one which is familiar to every man who goes
much to the city. He is a professional beggar, though, in
order to avoid the police regulations, he pretends to a small
trade in wax vestas. Some little distance down Threadneedle
Street, upon the left-hand side, there is, as you may have remarked,
a small angle in the wall. Here it is that this creature
takes his daily seat, cross-legged, with his tiny stock of
matches on his lap, and, as he is a piteous spectacle, a small
rain of charity descends into the greasy leather cap which lies
upon the pavement beside him. I have watched the fellow
more than once, before ever I thought of making his professional
acquaintance, and I have been surprised at the harvest
which he has reaped in a short time. His appearance, you
see, is so remarkable that no one can pass him without observing
him. A shock of orange hair, a pale face disfigured
by a horrible scar, which, by its contraction, has turned up
the outer edge of his upper lip, a bull-dog chin, and a pair of
very penetrating dark eyes, which present a singular contrast
%%161
to the color of his hair, all mark him out from amid the common
crowd of mendicants, and so, too, does his wit, for he is
ever ready with a reply to any piece of chaff which may be
thrown at him by the passers-by. This is the man whom we
now learn to have been the lodger at the opium den, and to
have been the last man to see the gentleman of whom we are
in quest.”

“But a cripple!” said I. “What could he have done
single-handed against a man in the prime of life?”

“He is a cripple in the sense that he walks with a limp;
but in other respects he appears to be a powerful and well-nurtured
man. Surely your medical experience would tell
you, Watson, that weakness in one limb is often compensated
for by exceptional strength in the others.”

“Pray continue your narrative.”

“Mrs.~St.~Clair had fainted at the sight of the blood upon
the window, and she was escorted home in a cab by the police,
as her presence could be of no help to them in their investigations.
Inspector Barton, who had charge of the case,
made a very careful examination of the premises, but without
finding anything which threw any light upon the matter. One
mistake had been made in not arresting Boone instantly, as
he was allowed some few minutes during which he might have
communicated with his friend the Lascar, but this fault was
soon remedied, and he was seized and searched, without anything
being found which could incriminate him. There were,
it is true, some blood-stains upon his right shirt-sleeve, but he
pointed to his ring-finger, which had been cut near the nail,
and explained that the bleeding came from there, adding that
he had been to the window not long before, and that the stains
which had been observed there came doubtless from the same
source. He denied strenuously having ever seen Mr.~Neville
St.~Clair, and swore that the presence of the clothes in his
room was as much a mystery to him as to the police. As to
Mrs.~St.~Clair’s assertion that she had actually seen her husband
at the window, he declared that she must have been either
%%162
mad or dreaming. He was removed, loudly protesting, to the
police-station, while the inspector remained upon the premises
in the hope that the ebbing tide might afford some fresh clew.

“And it did, though they hardly found upon the mud-bank
what they had feared to find. It was Neville St.~Clair’s coat,
and not Neville St.~Clair, which lay uncovered as the tide receded.
And what do you think they found in the pockets?”

“I cannot imagine.”

“No, I don’t think you would guess. Every pocket stuffed
with pennies and half-pennies -- 421 pennies and 270 half-pennies.
It was no wonder that it had not been swept away
by the tide. But a human body is a different matter. There
is a fierce eddy between the wharf and the house. It seemed
likely enough that the weighted coat had remained when the
stripped body had been sucked away into the river.”

“But I understand that all the other clothes were found in
the room. Would the body be dressed in a coat alone?”

“No, sir, but the facts might be met speciously enough.
Suppose that this man Boone had thrust Neville St.~Clair
through the window, there is no human eye which could have
seen the deed. What would he do then? It would of course
instantly strike him that he must get rid of the tell-tale garments.
He would seize the coat, then, and be in the act of
throwing it out, when it would occur to him that it would swim
and not sink. He has little time, for he has heard the scuffle
down-stairs when the wife tried to force her way up, and perhaps
he has already heard from his Lascar confederate that
the police are hurrying up the street. There is not an instant
to be lost. He rushes to some secret horde, where he has
accumulated the fruits of his beggary, and he stuffs all the
coins upon which he can lay his hands into the pockets to
make sure of the coat’s sinking. He throws it out, and would
have done the same with the other garments had not he heard
the rush of steps below, and only just had time to close the
window when the police appeared.”

“It certainly sounds feasible.”
%%163

“Well, we will take it as a working hypothesis for want
of a better. Boone, as I have told you, was arrested and
taken to the station, but it could not be shown that there had
ever before been anything against him. He had for years
been known as a professional beggar, but his life appeared to
have been a very quiet and innocent one. There the matter
stands at present, and the questions which have to be solved -- what
Neville St.~Clair was doing in the opium den, what happened
to him when there, where is he now, and what Hugh
Boone had to do with his disappearance -- are all as far from
a solution as ever. I confess that I cannot recall any case
within my experience which looked at the first glance so
simple, and yet which presented such difficulties.”

While Sherlock Holmes had been detailing this singular
series of events, we had been whirling through the outskirts of
the great town until the last straggling houses had been left
behind, and we rattled along with a country hedge upon
either side of us. Just as he finished, however, we drove
through two scattered villages, where a few lights still glimmered
in the windows.

“We are on the outskirts of Lee,” said my companion.
“We have touched on three English counties in our short
drive, starting in Middlesex, passing over an angle of Surrey,
and ending in Kent. See that light among the trees? That
is ‘The Cedars,’ and beside that lamp sits a woman whose
anxious ears have already, I have little doubt, caught the clink
of our horse’s feet.”

“But why are you not conducting the case from Baker
Street?” I asked.

“Because there are many inquiries which must be made
out here. Mrs.~St.~Clair has most kindly put two rooms at
my disposal, and you may rest assured that she will have
nothing but a welcome for my friend and colleague. I hate
to meet her, Watson, when I have no news of her husband.
Here we are. Whoa, there, whoa!”

We had pulled up in front of a large villa which stood
%%164
within its own grounds. A stable-boy had run out to the horse’s
head, and, springing down, I followed Holmes up the small,
winding gravel-drive which led to the house. As we approached,
the door flew open, and a little blonde woman stood
in the opening, clad in some sort of light mousseline de soie,
with a touch of fluffy pink chiffon at her neck and wrists.
She stood with her figure outlined against the flood of
light, one hand upon the door, one half-raised in her
eagerness, her body slightly bent, her head and face protruded,
with eager eyes and parted lips, a standing
question.

“Well?” she cried, “well?” And then, seeing that there
were two of us, she gave a cry of hope which sank into a
groan as she saw that my companion shook his head and
shrugged his shoulders.

“No good news?”

“None.”

“No bad?”

“No.”

“Thank God for that. But come in. You must be weary,
for you have had a long day.”

“This is my friend, Dr. Watson. He has been of most
vital use to me in several of my cases, and a lucky chance
has made it possible for me to bring him out and associate
him with this investigation.”

“I am delighted to see you,” said she, pressing my hand
warmly. “You will, I am sure, forgive anything that may be
wanting in our arrangements, when you consider the blow
which has come so suddenly upon us.”

“My dear madam,” said I, “I am an old campaigner, and
if I were not, I can very well see that no apology is needed.
If I can be of any assistance, either to you or to my friend
here, I shall be indeed happy.”

“Now, Mr.~Sherlock Holmes,” said the lady, as we entered
a well-lit dining-room, upon the table of which a cold supper
had been laid out, “I should very much like to ask you one
%%165
or two plain questions, to which I beg that you will give a
plain answer.”

“Certainly, madam.”

“Do not trouble about my feelings. I am not hysterical,
nor given to fainting. I simply wish to hear your real, real
opinion.”

“Upon what point?”

“In your heart of hearts do you think that Neville is alive?”

Sherlock Holmes seemed to be embarrassed by the question.
“Frankly, now!” she repeated, standing upon the rug
and looking keenly down at him as he leaned back in a
basket-chair.

“Frankly, then, madam, I do not.”

“You think that he is dead?”

“I do.”

“Murdered?”

“I don’t say that. Perhaps.”

“And on what day did he meet his death?”

“On Monday.”

“Then perhaps, Mr.~Holmes, you will be good enough
to explain how it is that I have received a letter from him
to-day.”

Sherlock Holmes sprang out of his chair as if he had been
galvanized.

“What!” he roared.

“Yes, to-day.” She stood smiling, holding up a little slip
of paper in the air.

“May I see it?”

“Certainly.”

He snatched it from her in his eagerness, and smoothing it
out upon the table, he drew over the lamp, and examined it
intently. I had left my chair, and was gazing at it over his
shoulder. The envelope was a very coarse one, and was
stamped with the Gravesend post-mark, and with the date of
that very day, or rather of the day before, for it was considerably
after midnight.
%%166

“Coarse writing,” murmured Holmes. “Surely this is not
your husband’s writing, madam.”

“No, but the enclosure is.”

“I perceive also that whoever addressed the envelope had
to go and inquire as to the address.”

“How can you tell that?”

“The name, you see, is in perfectly black ink, which has
dried itself. The rest is of the grayish color, which shows that
blotting-paper has been used. If it had been written straight
off, and then blotted, none would be of a deep black shade.
This man has written the name, and there has then been a
pause before he wrote the address, which can only mean that
he was not familiar with it. It is, of course, a trifle, but there
is nothing so important as trifles. Let us now see the letter.
Ha! there has been an enclosure here!”

“Yes, there was a ring. His signet-ring.”

“And you are sure that this is your husband’s hand?”

“One of his hands.”

“One?”

“His hand when he wrote hurriedly. It is very unlike his
usual writing, and yet I know it well.”

“\,‘Dearest do not be frightened. All will come well. There
is a huge error which it may take some little time to rectify.
Wait in patience. -- Neville.’ Written in pencil upon the fly-%
leaf of a book, octavo size, no water-mark. Hum! Posted
to-day in Gravesend by a man with a dirty thumb. Ha!
And the flap has been gummed, if I am not very much
in error, by a person who had been chewing tobacco.
And you have no doubt that it is your husband’s hand,
madam?”

“None. Neville wrote those words.”

“And they were posted to-day at Gravesend. Well, Mrs.
St.~Clair, the clouds lighten, though I should not venture to
say that the danger is over.”

“But he must be alive, Mr.~Holmes.”

“Unless this is a clever forgery to put us on the wrong
%%167
scent. The ring, after all, proves nothing. It may have been
taken from him.”

“No, no; it is, it is, it is his very own writing!”

“Very well. It may, however, have been written on Monday,
and only posted to-day.”

“That is possible.”

“If so, much may have happened between.”

“Oh, you must not discourage me, Mr.~Holmes. I know
that all is well with him. There is so keen a sympathy between
us that I should know if evil came upon him. On the
very day that I saw him last he cut himself in the bedroom,
and yet I in the dining-room rushed up-stairs instantly with
the utmost certainty that something had happened. Do you
think that I would respond to such a trifle, and yet be ignorant
of his death?”

“I have seen too much not to know that the impression of
a woman may be more valuable than the conclusion of an analytical
reasoner. And in this letter you certainly have a very
strong piece of evidence to corroborate your view. But if
your husband is alive, and able to write letters, why should he
remain away from you?”

“I cannot imagine. It is unthinkable.”

“And on Monday he made no remarks before leaving
you?”

“No.”

“And you were surprised to see him in Swandam Lane?”

“Very much so.”

“Was the window open?”

“Yes.”

“Then he might have called to you?”

“He might.”

“He only, as I understand, gave an inarticulate cry?”

“Yes.”

“A call for help, you thought?”

“Yes. He waved his hands.”

“But it might have been a cry of surprise. Astonishment
%%168
at the unexpected sight of you might cause him to throw up
his hands?”

“It is possible.”

“And you thought he was pulled back?”

“He disappeared so suddenly.”

“He might have leaped back. You did not see any one
else in the room?”

“No, but this horrible man confessed to having been there,
and the Lascar was at the foot of the stairs.”

“Quite so. Your husband, as far as you could see, had his
ordinary clothes on?”

“But without his collar or tie. I distinctly saw his bare
throat.”

“Had he ever spoken of Swandam Lane?”

“Never.”

“Had he ever showed any signs of having taken opium?”

“Never.”

“Thank you, Mrs.~St.~Clair. Those are the principal points
about which I wished to be absolutely clear. We shall now
have a little supper and then retire, for we may have a very
busy day to-morrow.”

A large and comfortable double-bedded room had been
placed at our disposal, and I was quickly between the sheets,
for I was weary after my night of adventure. Sherlock
Holmes was a man, however, who, when he had an unsolved
problem upon his mind, would go for days, and even for a
week, without rest, turning it over, rearranging his facts, looking
at it from every point of view, until he had either fathomed
it, or convinced himself that his data were insufficient.
It was soon evident to me that he was now preparing for an
all-night sitting. He took off his coat and waistcoat, put on
a large blue dressing-gown, and then wandered about the
room collecting pillows from his bed and cushions from the
sofa and arm-chairs. With these he constructed a sort of
Eastern divan, upon which he perched himself cross-legged,
with an ounce of shag tobacco and a box of matches laid out
%%169
in front of him. In the dim light of the lamp I saw him sitting
there, an old briar pipe between his lips, his eyes fixed
vacantly upon the corner of the ceiling, the blue smoke curling
up from him, silent, motionless, with the light shining upon
his strong-set aquiline features. So he sat as I dropped off
to sleep, and so he sat when a sudden ejaculation caused me
to wake up, and I found the summer sun shining into the
apartment. The pipe was still between his lips, the smoke
still curled upward, and the room was full of a dense tobacco
haze, but nothing remained of the heap of shag which I had
seen upon the previous night.

“Awake, Watson?” he asked.

“Yes.”

“Game for a morning drive?”

“Certainly.”

“Then dress. No one is stirring yet, but I know where the
stable-boy sleeps, and we shall soon have the trap out.” He
chuckled to himself as he spoke, his eyes twinkled, and he
seemed a different man to the sombre thinker of the previous
night.

As I dressed I glanced at my watch. It was no wonder
that no one was stirring. It was twenty-five minutes past
four. I had hardly finished when Holmes returned with the
news that the boy was putting in the horse.

“I want to test a little theory of mine,” said he, pulling on
his boots. “I think, Watson, that you are now standing in
the presence of one of the most absolute fools in Europe. I
deserve to be kicked from here to Charing Cross. But I think
I have the key of the affair now.”

“And where is it?” I asked, smiling.

“In the bath-room,” he answered. “Oh yes, I am not joking,”
he continued, seeing my look of incredulity. “I have
just been there, and I have taken it out, and I have got it
in this Gladstone bag. Come on, my boy, and we shall see
whether it will not fit the lock.”

We made our way down-stairs as quietly as possible, and
%%170
out into the bright morning sunshine. In the road stood our
horse and trap, with the half-clad stable-boy waiting at the
head. We both sprang in, and away we dashed down the
London Road. A few country carts were stirring, bearing in
vegetables to the metropolis, but the lines of villas on either
side were as silent and lifeless as some city in a dream.

“It has been in some points a singular case,” said Holmes,
flicking the horse on into a gallop. “I confess that I have
been as blind as a mole, but it is better to learn wisdom late
than never to learn it at all.”

In town the earliest risers were just beginning to look sleepily
from their windows as we drove through the streets of the
Surrey side. Passing down the Waterloo Bridge Road we
crossed over the river, and dashing up Wellington Street
wheeled sharply to the right, and found ourselves in Bow
Street. Sherlock Holmes was well known to the Force, and
the two constables at the door saluted him. One of them
held the horse’s head while the other led us in.

“Who is on duty?” asked Holmes.

“Inspector Bradstreet, sir.”

“Ah, Bradstreet, how are you?” A tall, stout official had
come down the stone-flagged passage, in a peaked cap and
frogged jacket. “I wish to have a quiet word with you,
Bradstreet.”

“Certainly, Mr.~Holmes. Step into my room here.”

It was a small, office-like room, with a huge ledger upon the
table, and a telephone projecting from the wall. The inspector
sat down at his desk.

“What can I do for you, Mr.~Holmes?”

“I called about that beggarman, Boone -- the one who was
charged with being concerned in the disappearance of Mr.
Neville St.~Clair, of Lee.”

“Yes. He was brought up and remanded for further
inquiries.”

“So I heard. You have him here?”

“In the cells.”
%%171

“Is he quiet?”

“Oh, he gives no trouble. But he is a dirty scoundrel.”

“Dirty?”

“Yes, it is all we can do to make him wash his hands, and
his face is as black as a tinker’s. Well, when once his case
has been settled, he will have a regular prison bath; and I
think, if you saw him, you would agree with me that he needed
it.”

“I should like to see him very much.”

“Would you? That is easily done. Come this way. You
can leave your bag.”

“No, I think that I’ll take it.”

“Very good. Come this way, if you please.” He led us
down a passage, opened a barred door, passed down a winding
stair, and brought us to a whitewashed corridor with a
line of doors on each side.

“The third on the right is his,” said the inspector. “Here
it is!” He quietly shot back a panel in the upper part of the
door and glanced through.

“He is asleep,” said he. “You can see him very well.”

We both put our eyes to the grating. The prisoner lay
with his face towards us, in a very deep sleep, breathing slowly
and heavily. He was a middle-sized man, coarsely clad as
became his calling, with a colored shirt protruding through the
rent in his tattered coat. He was, as the inspector had said,
extremely dirty, but the grime which covered his face could
not conceal its repulsive ugliness. A broad wheal from an
old scar ran right across it from eye to chin, and by its
contraction had turned up one side of the upper lip, so that
three teeth were exposed in a perpetual snarl. A shock of
very bright red hair grew low over his eyes and forehead.

“He’s a beauty, isn’t he?” said the inspector.

“He certainly needs a wash,” remarked Holmes. “I had
an idea that he might, and I took the liberty of bringing the
tools with me.” He opened the Gladstone bag as he spoke,
and took out, to my astonishment, a very large bath-sponge.
%%172

“He! he! You are a funny one,” chuckled the
inspector.

“Now, if you will have the great goodness to open that
door very quietly, we will soon make him cut a much more
respectable figure.”

“Well, I don’t know why not,” said the inspector. “He
doesn’t look a credit to the Bow Street cells, does he?” He
slipped his key into the lock, and we all very quietly entered
the cell. The sleeper half turned, and then settled down once
more into a deep slumber. Holmes stooped to the water-jug,
moistened his sponge, and then rubbed it twice vigorously
across and down the prisoner’s face.

“Let me introduce you,” he shouted, “to Mr.~Nev\-ille St.
Clair, of Lee, in the county of Kent.”

Never in my life have I seen such a sight. The man’s face
peeled off under the sponge like the bark from a tree. Gone
was the coarse brown tint! Gone, too, was the horrid scar
which had seamed it across, and the twisted lip which had
given the repulsive sneer to the face! A twitch brought away
the tangled red hair, and there, sitting up in his bed, was a
pale, sad-faced, refined-looking man, black-haired and smooth-%
skinned, rubbing his eyes, and staring about him with sleepy
bewilderment. Then suddenly realizing the exposure, he broke
into a scream, and threw himself down with his face to the
pillow.

“Great heavens!” cried the inspector, “it is, indeed, the
missing man. I know him from the photograph.”

The prisoner turned with the reckless air of a man who
abandons himself to his destiny. “Be it so,” said he. “And
pray, what am I charged with?”

“With making away with Mr.~Neville St.~--------- Oh, come,
you can’t be charged with that, unless they make a case of
attempted suicide of it,” said the inspector, with a grin.
“Well, I have been twenty-seven years in the force, but this
really takes the cake.”

“If I am Mr.~Neville St.~Clair, then it is obvious that no
%%173
crime has been committed, and that, therefore, I am illegally
detained.”

“No crime, but a very great error has been committed,”
said Holmes. “You would have done better to have trusted
your wife.”

“It was not the wife, it was the children,” groaned the prisoner.
“God help me, I would not have them ashamed of
their father. My God! What an exposure! What can I
do?”

Sherlock Holmes sat down beside him on the couch and
patted him kindly on the shoulder.

“If you leave it to a court of law to clear the matter up,”
said he, “of course you can hardly avoid publicity. On the
other hand, if you convince the police authorities that there is
no possible case against you, I do not know that there is any
reason that the details should find their way into the papers.
Inspector Bradstreet would, I am sure, make notes upon anything
which you might tell us, and submit it to the proper authorities.
The case would then never go into court at all.”

“God bless you!” cried the prisoner, passionately. “I
would have endured imprisonment, aye, even execution, rather
than have left my miserable secret as a family blot to my
children.

“You are the first who have ever heard my story. My
father was a school-master in Chesterfield, where I received
an excellent education. I travelled in my youth, took to the
stage, and finally became a reporter on an evening paper in
London. One day my editor wished to have a series of articles
upon begging in the metropolis, and I volunteered to
supply them. There was the point from which all my adventures
started. It was only by trying begging as an amateur
that I could get the facts upon which to base my articles.
When an actor I had, of course, learned all the secrets of
making up, and had been famous in the greenroom for my
skill. I took advantage now of my attainments. I painted
my face, and to make myself as pitiable as possible I made a
%%174
good scar and fixed one side of my lip in a twist by the aid of
a small slip of flesh-colored plaster. Then with a red head of
hair, and an appropriate dress, I took my station in the busiest
part of the city, ostensibly as a match-seller, but really as
a beggar. For seven hours I plied my trade, and when I returned
home in the evening I found, to my surprise, that I
had received no less than 26\textit{s.} 4\textit{d.}

“I wrote my articles, and thought little more of the matter
until, some time later, I backed a bill for a friend, and had a
writ served upon me for £25. I was at my wits’ end where
to get the money, but a sudden idea came to me. I begged a
fortnight’s grace from the creditor, asked for a holiday from
my employers, and spent the time in begging in the city under
my disguise. In ten days I had the money, and had paid the
debt.

“Well, you can imagine how hard it was to settle down to
arduous work at £2 a week, when I knew that I could earn
as much in a day by smearing my face with a little paint, laying
my cap on the ground, and sitting still. It was a long
fight between my pride and the money, but the dollars won at
last, and I threw up reporting, and sat day after day in the
corner which I had first chosen, inspiring pity by my ghastly
face, and filling my pockets with coppers. Only one man
knew my secret. He was the keeper of a low den in which I
used to lodge in Swandam Lane, where I could every morning
emerge as a squalid beggar, and in the evenings transform
myself into a well-dressed man about town. This fellow, a
Lascar, was well paid by me for his rooms, so that I knew
that my secret was safe in his possession.

“Well, very soon I found that I was saving considerable
sums of money. I do not mean that any beggar in the streets
of London could earn £700 a year -- which is less than my
average takings -- but I had exceptional advantages in my
power of making up, and also in a facility of repartee, which
improved by practice, and made me quite a recognized character
in the city. All day a stream of pennies, varied by
%%175
silver, poured in upon me, and it was a very bad day in which I
failed to take £2.

“As I grew richer I grew more ambitious, took a house in
the country, and eventually married, without any one having a
suspicion as to my real occupation. My dear wife knew that
I had business in the city. She little knew what.

“Last Monday I had finished for the day, and was dressing
in my room above the opium den, when I looked out of my
window, and saw, to my horror and astonishment, that my
wife was standing in the street, with her eyes fixed full upon
me. I gave a cry of surprise, threw up my arms to cover my
face, and, rushing to my confidant, the Lascar, entreated him
to prevent any one from coming up to me. I heard her voice
down-stairs, but I knew that she could not ascend. Swiftly I
threw off my clothes, pulled on those of a beggar, and put on
my pigments and wig. Even a wife’s eyes could not pierce so
complete a disguise. But then it occurred to me that there
might be a search in the room, and that the clothes might betray
me. I threw open the window, reopening by my violence
a small cut which I had inflicted upon myself in the bedroom
that morning. Then I seized my coat, which was weighted
by the coppers which I had just transferred to it from the
leather bag in which I carried my takings. I hurled it out of
the window, and it disappeared into the Thames. The other
clothes would have followed, but at that moment there was a
rush of constables up the stair, and a few minutes after I
found, rather, I confess, to my relief, that instead of being
identified as Mr.~Neville St.~Clair, I was arrested as his
murderer.

“I do not know that there is anything else for me to explain.
I was determined to preserve my disguise as long as
possible, and hence my preference for a dirty face. Knowing
that my wife would be terribly anxious, I slipped off my
ring, and confided it to the Lascar at a moment when no
constable was watching me, together with a hurried scrawl,
telling her that she had no cause to fear.”
%%176

“That note only reached her yesterday,” said Holmes.

“Good God! What a week she must have spent!”

“The police have watched this Lascar,” said Inspector
Bradstreet, “and I can quite understand that he might find it
difficult to post a letter unobserved. Probably he handed it
to some sailor customer of his, who forgot all about it for some
days.”

“That was it,” said Holmes, nodding approvingly; “I have
no doubt of it. But have you never been prosecuted for
begging?”

“Many times; but what was a fine to me?”

“It must stop here, however,” said Bradstreet. “If the
police are to hush this thing up, there must be no more of
Hugh Boone.”

“I have sworn it by the most solemn oaths which a man
can take.”

“In that case I think that it is probable that no further
steps may be taken. But if you are found again, then all must
come out. I am sure, Mr.~Holmes, that we are very much indebted
to you for having cleared the matter up. I wish I
knew how you reach your results.”

“I reached this one,” said my friend, “by sitting upon five
pillows and consuming an ounce of shag. I think, Watson,
that if we drive to Baker Street we shall just be in time for
breakfast.”
%%177

\Chapter{The Adventure Of The Blue Carbuncle}

\textsc{I had} called upon my friend Sherlock Holmes
upon the second morning after Christmas, with
the intention of wishing him the compliments of
the season. He was lounging upon the sofa in a
purple dressing-gown, a pipe-rack within his reach upon the
right, and a pile of crumpled morning papers, evidently newly
studied, near at hand. Beside the couch was a wooden chair,
and on the angle of the back hung a very seedy and disreputable
hard-felt hat, much the worse for wear, and cracked in
several places. A lens and a forceps lying upon the seat of
the chair suggested that the hat had been suspended in this
manner for the purpose of examination.

“You are engaged,” said I; “perhaps I interrupt you.”

“Not at all. I am glad to have a friend with whom I can
discuss my results. The matter is a perfectly trivial one” (he
jerked his thumb in the direction of the old hat), “but there
are points in connection with it which are not entirely devoid
of interest and even of instruction.”

I seated myself in his arm-chair and warmed my hands
before his crackling fire, for a sharp frost had set in, and the
windows were thick with the ice crystals. “I suppose,” I
remarked, “that, homely as it looks, this thing has some
deadly story linked on to it -- that it is the clew which will
guide you in the solution of some mystery and the punishment
of some crime.”

“No, no. No crime,” said Sherlock Holmes, laughing.
“Only one of those whimsical little incidents which will
%%178
happen when you have four million human beings all jostling
each other within the space of a few square miles. Amid the
action and reaction of so dense a swarm of humanity, every
possible combination of events may be expected to take place,
and many a little problem will be presented which may be
striking and bizarre without being criminal. We have already
had experience of such.”

“So much so,” I remarked, “that of the last six cases
which I have added to my notes, three have been entirely
free of any legal crime.”

“Precisely. You allude to my attempt to recover the Irene
Adler papers, to the singular case of Miss Mary Sutherland,
and to the adventure of the man with the twisted lip.
Well, I have no doubt that this small matter will fall into the
same innocent category. You know Peterson, the
commissionaire?”

“Yes.”

“It is to him that this trophy belongs.”

“It is his hat.”

“No, no; he found it. Its owner is unknown. I beg that
you will look upon it, not as a battered billycock, but as an
intellectual problem. And, first, as to how it came here. It
arrived upon Christmas morning, in company with a good fat
goose, which is, I have no doubt, roasting at this moment in
front of Peterson’s fire. The facts are these: about four
o’clock on Christmas morning, Peterson, who, as you know,
is a very honest fellow, was returning from some small
jollification, and was making his way homeward down Tottenham
Court Road. In front of him he saw, in the gaslight, a
tallish man, walking with a slight stagger, and carrying a white
goose slung over his shoulder. As he reached the corner of
Goodge Street, a row broke out between this stranger and a
little knot of roughs. One of the latter knocked off the man’s
hat, on which he raised his stick to defend himself, and, swinging
it over his head, smashed the shop window behind him.
Peterson had rushed forward to protect the stranger from his
%%179
assailants; but the man, shocked at having broken the window,
and seeing an official-looking person in uniform rushing towards
him, dropped his goose, took to his heels, and vanished
amid the labyrinth of small streets which lie at the back of
Tottenham Court Road. The roughs had also fled at the
appearance of Peterson, so that he was left in possession of
the field of battle, and also of the spoils of victory in the
shape of this battered hat and a most unimpeachable Christmas
goose.”

“Which surely he restored to their owner?”

“My dear fellow, there lies the problem. It is true that
‘For Mrs.~Henry Baker’ was printed upon a small card which
was tied to the bird’s left leg, and it is also true that the
initials ‘H. B.’ are legible upon the lining of this hat; but as
there are some thousands of Bakers, and some hundreds of
Henry Bakers in this city of ours, it is not easy to restore lost
property to any one of them.”

“What, then, did Peterson do?”

“He brought round both hat and goose to me on Christmas
morning, knowing that even the smallest problems are of
interest to me. The goose we retained until this morning,
when there were signs that, in spite of the slight frost, it
would be well that it should be eaten without unnecessary
delay. Its finder has carried it off, therefore, to fulfil the
ultimate destiny of a goose, while I continue to retain
the hat of the unknown gentleman who lost his Christmas
dinner.”

“Did he not advertise?”

“No.”

“Then, what clue could you have as to his identity?”

“Only as much as we can deduce.”

“From his hat?”

“Precisely.”

“But you are joking. What can you gather from this old
battered felt?”

“Here is my lens. You know my methods. What can you
%%180
gather yourself as to the individuality of the man who has worn
this article?”

I took the tattered object in my hands and turned it over
rather ruefully. It was a very ordinary black hat of the usual
round shape, hard, and much the worse for wear. The lining
had been of red silk, but was a good deal discolored. There
was no maker’s name; but, as Holmes had remarked, the
initials “H. B.” were scrawled upon one side. It was pierced
in the brim for a hat-securer, but the elastic was missing.
For the rest, it was cracked, exceedingly dusty, and spotted
in several places, although there seemed to have been some
attempt to hide the discolored patches by smearing them with
ink.

“I can see nothing,” said I, handing it back to my friend.

“On the contrary, Watson, you can see everything. You
fail, however, to reason from what you see. You are too timid
in drawing your inferences.”

“Then, pray tell me what it is that you can infer from this
hat?”

He picked it up and gazed at it in the peculiar introspective
fashion which was characteristic of him. “It is perhaps
less suggestive than it might have been,” he remarked, “and
yet there are a few inferences which are very distinct, and a
few others which represent at least a strong balance of probability.
That the man was highly intellectual is of course
obvious upon the face of it, and also that he was fairly well-to-do
within the last three years, although he has now fallen
upon evil days. He had foresight, but has less now than formerly,
pointing to a moral retrogression, which, when taken
with the decline of his fortunes, seems to indicate some evil
influence, probably drink, at work upon him. This may account
also for the obvious fact that his wife has ceased to love
him.”

“My dear Holmes!”

“He has, however, retained some degree of self-respect,”
he continued, disregarding my remonstrance. “He is a man
%%181
who leads a sedentary life, goes out little, is out of training
entirely, is middle-aged, has grizzled hair which he has had
cut within the last few days, and which he anoints with lime-cream.
These are the more patent facts which are to be deduced
from his hat. Also, by-the-way, that it is extremely
improbable that he has gas laid on in his house.”

“You are certainly joking, Holmes.”

“Not in the least. Is it possible that even now, when I
give you these results, you are unable to see how they are
attained?”

“I have no doubt that I am very stupid; but I must confess
that I am unable to follow you. For example, how did
you deduce that this man was intellectual?”

For answer Holmes clapped the hat upon his head. It
came right over the forehead and settled upon the bridge of
his nose. “It is a question of cubic capacity,” said he; “a
man with so large a brain must have something in it.”

“The decline of his fortunes, then?”

“This hat is three years old. These flat brims curled at
the edge came in then. It is a hat of the very best quality.
Look at the band of ribbed silk and the excellent lining. If
this man could afford to buy so expensive a hat three years
ago, and has had no hat since, then he has assuredly gone
down in the world.”

“Well, that is clear enough, certainly. But how about the
foresight and the moral retrogression?”

Sherlock Holmes laughed. “Here is the foresight,” said
he, putting his finger upon the little disk and loop of the hat-%
securer. “They are never sold upon hats. If this man ordered
one, it is a sign of a certain amount of foresight, since
he went out of his way to take this precaution against the
wind. But since we see that he has broken the elastic, and
has not troubled to replace it, it is obvious that he has less
foresight now than formerly, which is a distinct proof of a
weakening nature. On the other hand, he has endeavored to
conceal some of these stains upon the felt by daubing them
%%182
with ink, which is a sign that he has not entirely lost his self-%
respect.”

“Your reasoning is certainly plausible.”

“The further points, that he is middle-aged, that his hair is
grizzled, that it has been recently cut, and that he uses lime-%
cream, are all to be gathered from a close examination of the
lower part of the lining. The lens discloses a large number
of hair-ends, clean cut by the scissors of the barber. They
all appear to be adhesive, and there is a distinct odor of lime-%
cream. This dust, you will observe, is not the gritty, gray
dust of the street, but the fluffy brown dust of the house,
showing that it has been hung up in-doors most of the time;
while the marks of moisture upon the inside are proof positive
that the wearer perspired very freely, and could, therefore,
hardly be in the best of training.”

“But his wife -- you said that she had ceased to love
him.”

“This hat has not been brushed for weeks. When I see
you, my dear Watson, with a week’s accumulation of dust
upon your hat, and when your wife allows you to go out in
such a state, I shall fear that you also have been unfortunate
enough to lose your wife’s affection.”

“But he might be a bachelor.”

“Nay, he was bringing home the goose as a peace-offer\-ing
to his wife. Remember the card upon the bird’s leg.”

“You have an answer to everything. But how on earth do
you deduce that the gas is not laid on in his house?”

“One tallow stain, or even two, might come by chance;
but when I see no less than five, I think that there can be
little doubt that the individual must be brought into frequent
contact with burning tallow -- walks up-stairs at night probably
with his hat in one hand and a guttering candle in the other.
Anyhow, he never got tallow-stains from a gas-jet. Are you
satisfied?”

“Well, it is very ingenious,” said I, laughing; “but since,
as you said just now, there has been no crime committed, and
%%183
no harm done, save the loss of a goose, all this seems to be
rather a waste of energy.”

Sherlock Holmes had opened his mouth to reply, when the
door flew open, and Peterson, the commissionaire, rush\-ed into
the apartment with flushed cheeks and the face of a man who
is dazed with astonishment.

“The goose, Mr.~Holmes! The goose, sir!” he gasped.

“Eh? What of it, then? Has it returned to life and
flapped off through the kitchen window?” Holmes twisted
himself round upon the sofa to get a fairer view of the man’s
excited face.

“See here, sir! See what my wife found in its crop!” He
held out his hand and displayed upon the centre of the palm
a brilliantly scintillating blue stone, rather smaller than a bean
in size, but of such purity and radiance that it twinkled like
an electric point in the dark hollow of his hand.

Sherlock Holmes sat up with a whistle. “By Jove, Peterson!”
said he, “this is treasure trove indeed. I suppose you
know what you have got?”

“A diamond, sir? A precious stone. It cuts into glass as
though it were putty.”

“It’s more than a precious stone. It is \textit{the} precious stone.”

“Not the Countess of Morcar’s blue carbuncle!” I
ejaculated.

“Precisely so. I ought to know its size and shape, seeing
that I have read the advertisement about it in \textit{The Times} every
day lately. It is absolutely unique, and its value can only be
conjectured, but the reward offered of £1000 is certainly not
within a twentieth part of the market price.”

“A thousand pounds! Great Lord of mercy!” The commissionaire
plumped down into a chair, and stared from one
to the other of us.

“That is the reward, and I have reason to know that there
are sentimental considerations in the background which would
induce the countess to part with half her fortune if she could
but recover the gem.”
%%184

“It was lost, if I remember aright, at the ‘Hotel Cosmopolitan,’\,”
I remarked.

“Precisely so, on December 22d, just five days ago. John
Horner, a plumber, was accused of having abstracted it from
the lady’s jewel-case. The evidence against him was so strong
that the case has been referred to the Assizes. I have some
account of the matter here, I believe.” He rummaged amid
his newspapers, glancing over the dates, until at last he
smoothed one out, doubled it over, and read the following
paragraph:

“\,‘Hotel Cosmopolitan Jewel Robbery. John Horner, 26,
plumber, was brought up upon the charge of having upon the
22d inst. abstracted from the jewel-case of the Countess of
Morcar the valuable gem known as the blue carbuncle. James
Ryder, upper-attendant at the hotel, gave his evidence to the
effect that he had shown Horner up to the dressing-room of
the Countess of Morcar upon the day of the robbery, in order
that he might solder the second bar of the grate, which was
loose. He had remained with Horner some little time, but
had finally been called away. On returning, he found that
Horner had disappeared, that the bureau had been forced
open, and that the small morocco casket in which, as it afterwards
transpired, the countess was accustomed to keep her
jewel, was lying empty upon the dressing-table. Ryder instantly
gave the alarm, and Horner was arrested the same
evening; but the stone could not be found either upon his
person or in his rooms. Catherine Cusack, maid to the
countess, deposed to having heard Ryder’s cry of dismay on
discovering the robbery, and to having rushed into the room,
where she found matters as described by the last witness.
Inspector Bradstreet, B division, gave evidence as to the arrest
of Horner, who struggled frantically, and protested his
innocence in the strongest terms. Evidence of a previous
conviction for robbery having been given against the prisoner,
the magistrate refused to deal summarily with the offence, but
referred it to the Assizes. Horner, who had shown signs of
%%185
intense emotion during the proceedings, fainted away at the
conclusion, and was carried out of court.’

“Hum! So much for the police-court,” said Holmes,
thou\-ghtfully, tossing aside the paper. “The question for us
now to solve is the sequence of events leading from a rifled
jewel-case at one end to the crop of a goose in Tottenham
Court Road at the other. You see, Watson, our little deductions
have suddenly assumed a much more important and less
innocent aspect. Here is the stone; the stone came from the
goose, and the goose came from Mr.~Henry Baker, the gentleman
with the bad hat and all the other characteristics with
which I have bored you. So now we must set ourselves very
seriously to finding this gentleman, and ascertaining what part
he has played in this little mystery. To do this, we must try
the simplest means first, and these lie undoubtedly in an
advertisement in all the evening papers. If this fail, I shall have
recourse to other methods.”

“What will you say?”

“Give me a pencil and that slip of paper. Now, then:
‘Found at the corner of Goodge Street, a goose and a black felt
hat. Mr.~Henry Baker can have the same by applying at 6.30
this evening at 221\textsc{B}, Baker Street.’ That is clear and concise.”

“Very. But will he see it?”

“Well, he is sure to keep an eye on the papers, since, to a
poor man, the loss was a heavy one. He was clearly so scared
by his mischance in breaking the window and by the approach
of Peterson, that he thought of nothing but flight; but
since then he must have bitterly regretted the impulse which
caused him to drop his bird. Then, again, the introduction of
his name will cause him to see it, for every one who knows
him will direct his attention to it. Here you are, Peterson,
run down to the advertising agency, and have this put in the
evening papers.”

“In which, sir?”

“Oh, in the \textit{Globe}, \textit{Star}, \textit{Pall Mall}, \textit{St.~James’s}, \textit{Evening
News}, \textit{Standard}, \textit{Echo}, and any others that occur to you.”
%%186

“Very well, sir. And this stone?”

“Ah, yes, I shall keep the stone. Thank you. And, I say,
Peterson, just buy a goose on your way back, and leave it
here with me, for we must have one to give to this gentleman
in place of the one which your family is now devouring.”

When the commissionaire had gone, Holmes took up the
stone and held it against the light. “It’s a bonny thing,”
said he. “Just see how it glints and sparkles. Of course it
is a nucleus and focus of crime. Every good stone is. They
are the devil’s pet baits. In the larger and older jewels every
facet may stand for a bloody deed. This stone is not yet
twenty years old. It was found in the banks of the Amoy
River in Southern China, and is remarkable in having every
characteristic of the carbuncle, save that it is blue in shade,
instead of ruby red. In spite of its youth, it has already a
sinister history. There have been two murders, a vitriol-throwing,
a suicide, and several robberies brought about for
the sake of this forty-grain weight of crystallized charcoal.
Who would think that so pretty a toy would be a purveyor to
the gallows and the prison? I’ll lock it up in my strong
box now, and drop a line to the countess to say that we
have it.”

“Do you think that this man Horner is innocent?”

“I cannot tell.”

“Well, then, do you imagine that this other one, Henry
Baker, had anything to do with the matter?”

“It is, I think, much more likely that Henry Baker is an
absolutely innocent man, who had no idea that the bird which
he was carrying was of considerably more value than if it
were made of solid gold. That, however, I shall determine
by a very simple test, if we have an answer to our advertisement.”

“And you can do nothing until then?”

“Nothing.”

“In that case I shall continue my professional round. But
I shall come back in the evening at the hour you have
%%187
mentioned, for I should like to see the solution of so tangled a
business.”

“Very glad to see you. I dine at seven. There is a
woodcock, I believe. By-the-way, in view of recent occurrences,
perhaps I ought to ask Mrs.~Hudson to examine its crop.”

I had been delayed at a case, and it was a little after half-past
six when I found myself in Baker Street once more. As
I approached the house I saw a tall man in a Scotch bonnet
with a coat which was buttoned up to his chin, waiting outside
in the bright semicircle which was thrown from the fanlight.
Just as I arrived, the door was opened, and we were
shown up together to Holmes’s room.

“Mr.~Henry Baker, I believe,” said he, rising from his arm-%
chair, and greeting his visitor with the easy air of geniality
which he could so readily assume. “Pray take this chair by
the fire, Mr.~Baker. It is a cold night, and I observe that
your circulation is more adapted for summer than for winter.
Ah, Watson, you have just come at the right time. Is that
your hat, Mr.~Baker?”

“Yes, sir, that is undoubtedly my hat.”

He was a large man, with rounded shoulders, a massive
head, and a broad, intelligent face, sloping down to a pointed
beard of grizzled brown. A touch of red in nose and cheeks,
with a slight tremor of his extended hand, recalled Holmes’s
surmise as to his habits. His rusty black frock-coat was buttoned
right up in front, with the collar turned up, and his lank
wrists protruded from his sleeves without a sign of cuff or
shirt. He spoke in a slow staccato fashion, choosing his
words with care, and gave the impression generally of a man
of learning and letters who had had ill-usage at the hands of
fortune.

“We have retained these things for some days,” said
Holmes, “because we expected to see an advertisement from
you giving your address. I am at a loss to know now why
you did not advertise.”

Our visitor gave a rather shamefaced laugh. “Shillings
%%188
have not been so plentiful with me as they once were,” he remarked.
“I had no doubt that the gang of roughs who assaulted
me had carried off both my hat and the bird. I did
not care to spend more money in a hopeless attempt at recovering
them.”

“Very naturally. By-the-way, about the bird, we were compelled
to eat it.”

“To eat it!” Our visitor half rose from his chair in his
excitement.

“Yes, it would have been of no use to any one had we not
done so. But I presume that this other goose upon the sideboard,
which is about the same weight and perfectly fresh,
will answer your purpose equally well?”

“Oh, certainly, certainly;” answered Mr.~Baker, with a sigh
of relief.

“Of course, we still have the feathers, legs, crop, and so on
of your own bird, so if you wish -- ”

The man burst into a hearty laugh. “They might be useful
to me as relics of my adventure,” said he, “but beyond
that I can hardly see what use the \textit{disjecta membra} of my late
acquaintance are going to be to me. No, sir, I think that,
with your permission, I will confine my attentions to the excellent
bird which I perceive upon the sideboard.”

Sherlock Holmes glanced sharply across at me with a slight
shrug of his shoulders.

“There is your hat, then, and there your bird,” said he.
“By-the-way, would it bore you to tell me where you got the
other one from? I am somewhat of a fowl fancier, and I
have seldom seen a better grown goose.”

“Certainly, sir,” said Baker, who had risen and tucked his
newly-gained property under his arm. “There are a few of
us who frequent the ‘Alpha Inn,’ near the Museum -- we are
to be found in the Museum itself during the day, you understand.
This year our good host, Windigate by name, instituted
a goose club, by which, on consideration of some few
pence every week, we were each to receive a bird at
%%189
Christmas. My pence were duly paid, and the rest is familiar to
you. I am much indebted to you, sir, for a Scotch bonnet is
fitted neither to my years nor my gravity.” With a comical
pomposity of manner he bowed solemnly to both of us and
strode off upon his way.

“So much for Mr.~Henry Baker,” said Holmes, when he
had closed the door behind him. “It is quite certain that he
knows nothing whatever about the matter. Are you hungry,
Watson?”

“Not particularly.”

“Then I suggest that we turn our dinner into a supper, and
follow up this clew while it is still hot.”

“By all means.”

It was a bitter night, so we drew on our ulsters and wrapped
cravats about our throats. Outside, the stars were shining
coldly in a cloudless sky, and the breath of the passers-by
blew out into smoke like so many pistol shots. Our footfalls
rang out crisply and loudly as we swung through the Doctors’
quarter, Wimpole Street, Harley Street, and so through Wigmore
Street into Oxford Street. In a quarter of an hour we
were in Bloomsbury at the “Alpha Inn,” which is a small
public-house at the corner of one of the streets which runs
down into Holborn. Holmes pushed open the door of the
private bar, and ordered two glasses of beer from the ruddy-faced,
white-aproned landlord.

“Your beer should be excellent if it is as good as your
geese,” said he.

“My geese!” The man seemed surprised.

“Yes. I was speaking only half an hour ago to Mr.~Henry
Baker, who was a member of your goose club.”

“Ah! yes, I see. But you see, sir, them’s not \textit{our} geese.”

“Indeed! Whose, then?”

“Well, I got the two dozen from a salesman in Covent Garden.”

“Indeed? I know some of them. Which was it?”

“Breckinridge is his name.”
%%190

“Ah! I don’t know him. Well, here’s your good health,
landlord, and prosperity to your house. Good-night?”

“Now for Mr.~Breckinridge,” he continued, buttoning up
his coat, as we came out into the frosty air. “Remember,
Watson, that though we have so homely a thing as a goose at
one end of this chain, we have at the other a man who will
certainly get seven years’ penal servitude unless we can establish
his innocence. It is possible that our inquiry may
but confirm his guilt; but, in any case, we have a line of
investigation which has been missed by the police, and which a
singular chance has placed in our hands. Let us follow it
out to the bitter end. Faces to the south, then, and quick
march!”

We passed across Holborn, down Endell Street, and so
through a zigzag of slums to Covent Garden Market. One
of the largest stalls bore the name of Breckinridge upon it,
and the proprietor, a horsey-looking man, with a sharp face
and trim side-whiskers, was helping a boy to put up the
shutters.

“Good-evening. It’s a cold night,” said Holmes.

The salesman nodded, and shot a questioning glance at my
companion.

“Sold out of geese, I see,” continued Holmes, pointing at
the bare slabs of marble.

“Let you have 500 to-morrow morning.”

“That’s no good.”

“Well, there are some on the stall with the gas-flare.”

“Ah, but I was recommended to you.”

“Who by?”

“The landlord of the ‘Alpha.’\,”

“Oh, yes; I sent him a couple of dozen.”

“Fine birds they were, too. Now where did you get them
from?”

To my surprise the question provoked a burst of anger
from the salesman.

“Now, then, mister,” said he, with his head cocked and his
%%191
arms akimbo, “what are you driving at? Let’s have it
straight, now.”

“It is straight enough. I should like to know who sold
you the geese which you supplied to the ‘Alpha.’\,”

“Well, then, I sha’n’t tell you. So now!”

“Oh, it is a matter of no importance; but I don’t know
why you should be so warm over such a trifle.”

“Warm! You’d be as warm, maybe, if you were as pestered
as I am. When I pay good money for a good article
there should be an end of the business; but it’s ‘Where are
the geese?’ and ‘Who did you sell the geese to?’ and ‘What will
you take for the geese?’ One would think they were the only
geese in the world, to hear the fuss that is made over them.”

“Well, I have no connection with any other people who
have been making inquiries,” said Holmes, carelessly. “If
you won’t tell us the bet is off, that is all. But I’m always
ready to back my opinion on a matter of fowls, and I have a
fiver on it that the bird I ate is country bred.”

“Well, then, you’ve lost your fiver, for it’s town bred,”
snapped the salesman.

“It’s nothing of the kind.”

“I say it is.”

“I don’t believe it.”

“D’you think you know more about fowls than I, who have
handled them ever since I was a nipper? I tell you, all those
birds that went to the ‘Alpha’ were town bred.”

“You’ll never persuade me to believe that.”

“Will you bet, then?”

“It’s merely taking your money, for I know that I am right.
But I’ll have a sovereign on with you, just to teach you not to
be obstinate.”

The salesman chuckled grimly. “Bring me the books,
Bill,” said he.

The small boy brought round a small thin volume and a
great greasy-backed one, laying them out together beneath
the hanging lamp.
%%192

“Now then, Mr.~Cocksure,” said the salesman, “I thought
that I was out of geese, but before I finish you’ll find that
there is still one left in my shop. You see this little book?”

“Well?”

“That’s the list of the folk from whom I buy. D’you see?
Well, then, here on this page are the country folk, and the
numbers after their names are where their accounts are in the
big ledger. Now, then! You see this other page in red ink?
Well, that is a list of my town suppliers. Now, look at that
third name. Just read it out to me.”

“Mrs.~Oakshott, 117, Brixton Road -- 249,” read Holmes.

“Quite so. Now turn that up in the ledger.”

Holmes turned to the page indicated. “Here you are,
‘Mrs.~Oakshott, 117, Brixton Road, egg and poultry
supplier.’\,”

“Now, then, what’s the last entry?”

“\,‘December 22. Twenty-four geese at 7\textit{s.} 6\textit{d.}’\,”

“Quite so. There you are. And underneath?”

“\,‘Sold to Mr.~Windigate of the ‘Alpha,’ at 12\textit{s.}’\,”

“What have you to say now?”

Sherlock Holmes looked deeply chagrined. He drew a
sovereign from his pocket and threw it down upon the slab,
turning away with the air of a man whose disgust is too deep
for words. A few yards off he stopped under a lamp-post,
and laughed in the hearty, noiseless fashion which was peculiar
to him.

“When you see a man with whiskers of that cut and the
‘pink ’un’ protruding out of his pocket, you can always
draw him by a bet,” said he. “I dare say that if I had put
£100 down in front of him, that man would not have given
me such complete information as was drawn from him by
the idea that he was doing me on a wager. Well, Watson,
we are, I fancy, nearing the end of our quest, and the only
point which remains to be determined is whether we should
go on to this Mrs.~Oakshott to-night, or whether we should reserve
it for to-morrow. It is clear from what that surly fellow
%%193
said that there are others besides ourselves who are anxious
about the matter, and I should -- ”

His remarks were suddenly cut short by a loud hubbub
which broke out from the stall which we had just left. Turning
round we saw a little rat-faced fellow standing in the centre
of the circle of yellow light which was thrown by the
swinging lamp, while Breckinridge the salesman, framed in
the door of his stall, was shaking his fists fiercely at the
cringing figure.

“I’ve had enough of you and your geese,” he shouted. “I
wish you were all at the devil together. If you come pestering
me any more with your silly talk I’ll set the dog at you.
You bring Mrs.~Oakshott here and I’ll answer her, but what
have you to do with it? Did I buy the geese off you?”

“No; but one of them was mine all the same,” whined the
little man.

“Well, then, ask Mrs.~Oakshott for it.”

“She told me to ask you.”

“Well, you can ask the King of Proosia, for all I care.
I’ve had enough of it. Get out of this!” He rushed fiercely
forward, and the inquirer flitted away into the darkness.

“Ha! this may save us a visit to Brixton Road,” whispered
Holmes. “Come with me, and we will see what is to be
made of this fellow.” Striding through the scattered knots
of people who lounged round the flaring stalls, my companion
speedily overtook the little man and touched him upon the
shoulder. He sprang round, and I could see in the gaslight
that every vestige of color had been driven from his
face.

“Who are you, then? What do you want?” he asked, in a
quavering voice.

“You will excuse me,” said Holmes, blandly, “but I could
not help overhearing the questions which you put to the salesman
just now. I think that I could be of assistance to you.”

“You? Who are you? How could you know anything of
the matter?”
%%194

“My name is Sherlock Holmes. It is my business to know
what other people don’t know.”

“But you can know nothing of this?”

“Excuse me, I know everything of it. You are endeavoring
to trace some geese which were sold by Mrs.~Oakshott, of
Brixton Road, to a salesman named Breckinridge, by him in
turn to Mr.~Windigate, of the ‘Alpha,’ and by him to his club,
of which Mr.~Henry Baker is a member.”

“Oh, sir, you are the very man whom I have longed to
meet,” cried the little fellow, with outstretched hands and
quivering fingers. “I can hardly explain to you how interested
I am in this matter.”

Sherlock Holmes hailed a four-wheeler which was passing.
“In that case we had better discuss it in a cosey room rather
than in this windswept market-place,” said he. “But pray
tell me, before we go farther, who it is that I have the pleasure
of assisting.”

The man hesitated for an instant. “My name is John
Robinson,” he answered, with a sidelong glance.

“No, no; the real name,” said Holmes, sweetly. “It is
always awkward doing business with an \textit{alias}.”

A flush sprang to the white cheeks of the stranger. “Well,
then,” said he, “my real name is James Ryder.”

“Precisely so. Head attendant at the ‘Hotel Cosmopolitan.’
Pray step into the cab, and I shall soon be able to tell you
everything which you would wish to know.”

The little man stood glancing from one to the other of us
with half-frightened, half-hopeful eyes, as one who is not sure
whether he is on the verge of a windfall or of a catastrophe.
Then he stepped into the cab, and in half an hour we were
back in the sitting-room at Baker Street. Nothing had been
said during our drive, but the high, thin breathing of our new
companion, and the claspings and unclaspings of his hands,
spoke of the nervous tension within him.

“Here we are!” said Holmes, cheerily, as we filed into the
room. “The fire looks very seasonable in this weather. You
%%195
look cold, Mr.~Ryder. Pray take the basket-chair. I will just
put on my slippers before we settle this little matter of yours.
Now, then! You want to know what became of those
geese?”

“Yes, sir.”

“Or rather, I fancy, of that goose. It was one bird, I imagine,
in which you were interested -- white, with a black bar
across the tail.”

Ryder quivered with emotion. “Oh, sir,” he cried, “can you
tell me where it went to?”

“It came here.”

“Here?”

“Yes, and a most remarkable bird it proved. I don’t wonder
that you should take an interest in it. It laid an egg after
it was dead -- the bonniest, brightest little blue egg that ever
was seen. I have it here in my museum.”

Our visitor staggered to his feet and clutched the mantelpiece
with his right hand. Holmes unlocked his strong-box,
and held up the blue carbuncle, which shone out like a star,
with a cold, brilliant, many-pointed radiance. Ryder stood
glaring with a drawn face, uncertain whether to claim or to
disown it.

“The game’s up, Ryder,” said Holmes, quietly. “Hold
up, man, or you’ll be into the fire! Give him an arm back
into his chair, Watson. He’s not got blood enough to go in
for felony with impunity. Give him a dash of brandy. So!
Now he looks a little more human. What a shrimp it is, to
be sure!”

For a moment he had staggered and nearly fallen, but the
brandy brought a tinge of color into his cheeks, and he sat
staring with frightened eyes at his accuser.

“I have almost every link in my hands, and all the proofs
which I could possibly need, so there is little which you need
tell me. Still, that little may as well be cleared up to make
the case complete. You had heard, Ryder, of this blue stone
of the Countess of Morcar’s?”
%%196

“It was Catherine Cusack who told me of it,” said he, in a
crackling voice.

“I see -- her ladyship’s waiting-maid. Well, the temptation
of sudden wealth so easily acquired was too much for you, as
it has been for better men before you; but you were not very
scrupulous in the means you used. It seems to me, Ryder,
that there is the making of a very pretty villain in you. You
knew that this man Horner, the plumber, had been concerned
in some such matter before, and that suspicion would rest the
more readily upon him. What did you do, then? You made
some small job in my lady’s room -- you and your confederate
Cusack -- and you managed that he should be the man sent
for. Then, when he had left, you rifled the jewel-case, raised
the alarm, and had this unfortunate man arrested. You
then -- ”

Ryder threw himself down suddenly upon the rug and
clutched at my companion’s knees. “For God’s sake, have
mercy!” he shrieked. “Think of my father! of my mother!
It would break their hearts. I never went wrong before! I
never will again. I swear it. I’ll swear it on a Bible. Oh,
don’t bring it into court! For Christ’s sake, don’t!”

“Get back into your chair!” said Holmes, sternly. “It is
very well to cringe and crawl now, but you thought little
enough of this poor Horner in the dock for a crime of which
he knew nothing.”

“I will fly, Mr.~Holmes. I will leave the country, sir. Then
the charge against him will break down.”

“Hum! We will talk about that. And now let us hear a
true account of the next act. How came the stone into the
goose, and how came the goose into the open market? Tell
us the truth, for there lies your only hope of safety.”

Ryder passed his tongue over his parched lips. “I will tell
you it just as it happened, sir,” said he. “When Horner had
been arrested, it seemed to me that it would be best for me
to get away with the stone at once, for I did not know at what
moment the police might not take it into their heads to search
%%197
%%“\,‘HAVE MERCY!’ HE SHRIEKED”
%%198
me and my room. There was no place about the hotel where
it would be safe. I went out, as if on some commission, and
I made for my sister’s house. She had married a man named
Oakshott, and lived in Brixton Road, where she fattened fowls
for the market. All the way there every man I met seemed to
me to be a policeman or a detective; and, for all that it was
a cold night, the sweat was pouring down my face before I
came to the Brixton Road. My sister asked me what was the
matter, and why I was so pale; but I told her that I had been
upset by the jewel robbery at the hotel. Then I went into the
back yard and smoked a pipe, and wondered what it would
be best to do.

“I had a friend once called Maudsley, who went to the bad,
and has just been serving his time in Pentonville. One day
he had met me, and fell into talk about the ways of thieves,
and how they could get rid of what they stole. I knew that
he would be true to me, for I knew one or two things about
him; so I made up my mind to go right on to Kilburn, where
he lived, and take him into my confidence. He would show
me how to turn the stone into money. But how to get to him
in safety? I thought of the agonies I had gone through in
coming from the hotel. I might at any moment be seized and
searched, and there would be the stone in my waistcoat pocket.
I was leaning against the wall at the time, and looking at the
geese which were waddling about round my feet, and suddenly
an idea came into my head which showed me how I could
beat the best detective that ever lived.

“My sister had told me some weeks before that I might
have the pick of her geese for a Christmas present, and I knew
that she was always as good as her word. I would take my
goose now, and in it I would carry my stone to Kilburn.
There was a little shed in the yard, and behind this I drove
one of the birds -- a fine big one, white, with a barred tail. I
caught it, and, prying its bill open, I thrust the stone down
its throat as far as my finger could reach. The bird gave a
gulp, and I felt the stone pass along its gullet and down into
%%200
its crop. But the creature flapped and struggled, and out
came my sister to know what was the matter. As I turned
to speak to her the brute broke loose and fluttered off among
the others.

“\,‘Whatever were you doing with that bird, Jem?’ says she.

“\,‘Well,’ said I, ‘you said you’d give me one for Christmas,
and I was feeling which was the fattest.’

“\,‘Oh,’ says she, ‘we’ve set yours aside for you -- Jem’s bird,
we call it. It’s the big white one over yonder. There’s twenty-six
of them, which makes one for you, and one for us, and
two dozen for the market.’

“\,‘Thank you, Maggie,’ says I; ‘but if it is all the same to
you, I’d rather have that one I was handling just now.’

“\,‘The other is a good three pound heavier,’ said she, ‘and
we fattened it expressly for you.’

“\,‘Never mind. I’ll have the other, and I’ll take it now,’
said I.

“\,‘Oh, just as you like,’ said she, a little huffed. ‘Which is
it you want, then?’

“\,‘That white one with the barred tail, right in the middle
of the flock.’

“\,‘Oh, very well. Kill it and take it with you.’

“Well, I did what she said, Mr.~Holmes, and I carried the
bird all the way to Kilburn. I told my pal what I had done,
for he was a man that it was easy to tell a thing like that to.
He laughed until he choked, and we got a knife and opened
the goose. My heart turned to water, for there was no sign
of the stone, and I knew that some terrible mistake had occurred.
I left the bird, rushed back to my sister’s, and hurried
into the back yard. There was not a bird to be seen there.

“\,‘Where are they all, Maggie?’ I cried.

“\,‘Gone to the dealer’s, Jem.’

“\,‘Which dealer’s?’

“\,‘Breckinridge, of Covent Garden.’

“\,‘But was there another with a barred tail?’ I asked, ‘the
same as the one I chose?’
%%201

“\,‘Yes, Jem; there were two barred-tailed ones, and I could
never tell them apart.’

“Well, then, of course I saw it all, and I ran off as hard as
my feet would carry me to this man Breckinridge; but he had
sold the lot at once, and not one word would he tell me as to
where they had gone. You heard him yourselves to-night.
Well, he has always answered me like that. My sister thinks
that I am going mad. Sometimes I think that I am myself.
And now -- and now I am myself a branded thief, without ever
having touched the wealth for which I sold my character.
God help me! God help me!” He burst into convulsive
sobbing, with his face buried in his hands.

There was a long silence, broken only by his heavy breathing,
and by the measured tapping of Sherlock Holmes’s finger-tips
upon the edge of the table. Then my friend rose and
threw open the door.

“Get out!” said he.

“What, sir! Oh, heaven bless you!”

“No more words. Get out!”

And no more words were needed. There was a rush, a
clatter upon the stairs, the bang of a door, and the crisp rattle
of running footfalls from the street.

“After all, Watson,” said Holmes, reaching up his hand
for his clay pipe, “I am not retained by the police to supply
their deficiencies. If Horner were in danger it would be another
thing; but this fellow will not appear against him, and
the case must collapse. I suppose that I am commuting a
felony, but it is just possible that I am saving a soul. This
fellow will not go wrong again; he is too terribly frightened.
Send him to jail now, and you make him a jail-bird for life.
Besides, it is the season of forgiveness. Chance has put in
our way a most singular and whimsical problem, and its solution
is its own reward. If you will have the goodness to
touch the bell, doctor, we will begin another investigation, in
which, also, a bird will be the chief feature.”
%%202

\Chapter{The Adventure Of The Speckled Band}

\textsc{On} glancing over my notes of the seventy odd cases
in which I have during the last eight years studied
the methods of my friend Sherlock Holmes, I find
many tragic, some comic, a large number merely
strange, but none commonplace; for, working as he did rather
for the love of his art than for the acquirement of wealth,
he refused to associate himself with any investigation which
did not tend towards the unusual, and even the fantastic.
Of all these varied cases, however, I cannot recall any which
presented more singular features than that which was associated
with the well-known Surrey family of the Roylotts of
Stoke Moran. The events in question occurred in the early
days of my association with Holmes, when we were sharing
rooms as bachelors in Baker Street. It is possible that I
might have placed them upon record before, but a promise of
secrecy was made at the time, from which I have only been
freed during the last month by the untimely death of the lady
to whom the pledge was given. It is perhaps as well that the
facts should now come to light, for I have reasons to know
that there are wide-spread rumors as to the death of Dr.
Grimesby Roylott which tend to make the matter even more
terrible than the truth.

It was early in April in the year ’83 that I woke one morning
to find Sherlock Holmes standing, fully dressed, by the
side of my bed. He was a late riser as a rule, and as the
clock on the mantel-piece showed me that it was only a quarter
past seven, I blinked up at him in some surprise, and
%%203
perhaps just a little resentment, for I was myself regular in my
habits.

“Very sorry to knock you up, Watson,” said he, “but it’s
the common lot this morning. Mrs.~Hudson has been knocked
up, she retorted upon me, and I on you.”

“What is it, then -- a fire?”

“No; a client. It seems that a young lady has arrived in
a considerable state of excitement, who insists upon seeing
me. She is waiting now in the sitting-room. Now, when
young ladies wander about the metropolis at this hour of the
morning, and knock sleepy people up out of their beds, I presume
that it is something very pressing which they have to
communicate. Should it prove to be an interesting case, you
would, I am sure, wish to follow it from the outset. I thought,
at any rate, that I should call you and give you the chance.”

“My dear fellow, I would not miss it for anything.”

I had no keener pleasure than in following Holmes in his
professional investigations, and in admiring the rapid deductions,
as swift as intuitions, and yet always founded on a logical
basis, with which he unravelled the problems which were
submitted to him. I rapidly threw on my clothes, and was
ready in a few minutes to accompany my friend down to the
sitting-room. A lady dressed in black and heavily veiled,
who had been sitting in the window, rose as we entered.

\begin{sloppypar}
“Good-morning, madam,” said Holmes, cheerily. “My
name is Sherlock Holmes. This is my intimate friend and
associate, Dr. Watson, before whom you can speak as freely
as before myself. Ha! I am glad to see that Mrs.~Hudson
has had the good sense to light the fire. Pray draw up to it,
and I shall order you a cup of hot coffee, for I observe that
you are shivering.”
\end{sloppypar}

“It is not cold which makes me shiver,” said the woman,
in a low voice, changing her seat as requested.

“What, then?”

“It is fear, Mr.~Holmes. It is terror.” She raised her
veil as she spoke, and we could see that she was indeed in a
%%204
pitiable state of agitation, her face all drawn and gray, with
restless, frightened eyes, like those of some hunted animal.
Her features and figure were those of a woman of thirty, but
her hair was shot with premature gray, and her expression
was weary and haggard. Sherlock Holmes ran her over with
one of his quick, all-comprehensive glances.

“You must not fear,” said he, soothingly, bending forward
and patting her forearm. “We shall soon set matters right,
I have no doubt. You have come in by train this morning, I
see.”

“You know me, then?”

“No, but I observe the second half of a return ticket in the
palm of your left glove. You must have started early, and
yet you had a good drive in a dog-cart, along heavy roads, before
you reached the station.”

The lady gave a violent start, and stared in bewilderment
at my companion.

“There is no mystery, my dear madam,” said he, smiling.
“The left arm of your jacket is spattered with mud in no less
than seven places. The marks are perfectly fresh. There is
no vehicle save a dog-cart which throws up mud in that way,
and then only when you sit on the left-hand side of the
driver.”

“Whatever your reasons may be, you are perfectly correct,”
said she. “I started from home before six, reached Leatherhead
at twenty past, and came in by the first train to Waterloo.
Sir, I can stand this strain no longer; I shall go mad if
it continues. I have no one to turn to -- none, save only one,
who cares for me, and he, poor fellow, can be of little aid. I
have heard of you, Mr.~Holmes; I have heard of you from
Mrs.~Farintosh, whom you helped in the hour of her sore need.
It was from her that I had your address. Oh, sir, do you not
think that you could help me, too, and at least throw a little
light through the dense darkness which surrounds me? At
present it is out of my power to reward you for your services,
but in a month or six weeks I shall be married, with the
%%205
control of my own income, and then at least you shall not find
me ungrateful.”

Holmes turned to his desk, and unlocking it, drew out a
small case-book, which he consulted.

“Farintosh,” said he. “Ah yes, I recall the case; it was
concerned with an opal tiara. I think it was before your
time, Watson. I can only say, madam, that I shall be happy
to devote the same care to your case as I did to that of your
friend. As to reward, my profession is its own reward; but
you are at liberty to defray whatever expenses I may be put
to, at the time which suits you best. And now I beg that you
will lay before us everything that may help us in forming an
opinion upon the matter.”

“Alas!” replied our visitor, “the very horror of my situation
lies in the fact that my fears are so vague, and my suspicions
depend so entirely upon small points, which might seem
trivial to another, that even he to whom of all others I have a
right to look for help and advice looks upon all that I tell him
about it as the fancies of a nervous woman. He does not
say so, but I can read it from his soothing answers and averted
eyes. But I have heard, Mr.~Holmes, that you can see
deeply into the manifold wickedness of the human heart.
You may advise me how to walk amid the dangers which encompass
me.”

“I am all attention, madam.”

“My name is Helen Stoner, and I am living with my step-father,
who is the last survivor of one of the oldest Saxon families
in England, the Roylotts of Stoke Moran, on the western
border of Surrey.”

Holmes nodded his head. “The name is familiar to me,”
said he.

“The family was at one time among the richest in England,
and the estates extended over the borders into Berkshire in
the north, and Hampshire in the west. In the last century,
however, four successive heirs were of a dissolute and wasteful
disposition, and the family ruin was eventually completed
%%206
by a gambler in the days of the Regency. Nothing was left
save a few acres of ground, and the two-hundred-year-old
house, which is itself crushed under a heavy mortgage. The
last squire dragged out his existence there, living the horrible
life of an aristocratic pauper; but his only son, my step-father,
seeing that he must adapt himself to the new conditions, obtained
an advance from a relative, which enabled him to take
a medical degree, and went out to Calcutta, where, by his
professional skill and his force of character, he established a
large practice. In a fit of anger, however, caused by some
robberies which had been perpetrated in the house, he beat
his native butler to death, and narrowly escaped a capital
sentence. As it was, he suffered a long term of imprisonment,
and afterwards returned to England a morose and disappointed
man.

“When Dr. Roylott was in India he married my mother,
Mrs.~Stoner, the young widow of Major-general Stoner, of the
Bengal Artillery. My sister Julia and I were twins, and we
were only two years old at the time of my mother’s re-marriage.
She had a considerable sum of money -- not less than
£1000 a year -- and this she bequeathed to Dr. Roylott
entirely while we resided with him, with a provision that a
certain annual sum should be allowed to each of us in the
event of our marriage. Shortly after our return to England
my mother died -- she was killed eight years ago in a railway
accident near Crewe. Dr. Roylott then abandoned his attempts
to establish himself in practice in London, and took
us to live with him in the old ancestral house at Stoke Moran.
The money which my mother had left was enough for all our
wants, and there seemed to be no obstacle to our happiness.

“But a terrible change came over our step-father about this
time. Instead of making friends and exchanging visits with our
neighbors, who had at first been overjoyed to see a Roylott of
Stoke Moran back in the old family seat, he shut himself up
in his house, and seldom came out save to indulge in ferocious
quarrels with whoever might cross his path. Violence of
%%207
temper approaching to mania has been hereditary in the men of
the family, and in my step-father’s case it had, I believe, been
intensified by his long residence in the tropics. A series of
disgraceful brawls took place, two of which ended in the
police-court, until at last he became the terror of the village,
and the folks would fly at his approach, for he is a man of immense
strength, and absolutely uncontrollable in his anger.

“Last week he hurled the local blacksmith over a parapet
into a stream, and it was only by paying over all the money
which I could gather together that I was able to avert another
public exposure. He had no friends at all save the wandering
gypsies, and he would give these vagabonds leave to encamp
upon the few acres of bramble-covered land which
represent the family estate, and would accept in return the
hospitality of their tents, wandering away with them sometimes
for weeks on end. He has a passion also for Indian
animals, which are sent over to him by a correspondent, and
he has at this moment a cheetah and a baboon, which wander
freely over his grounds, and are feared by the villagers almost
as much as their master.

“You can imagine from what I say that my poor sister
Julia and I had no great pleasure in our lives. No servant
would stay with us, and for a long time we did all the work of
the house. She was but thirty at the time of her death, and
yet her hair had already begun to whiten, even as mine has.”

“Your sister is dead, then?”

“She died just two years ago, and it is of her death that I
wish to speak to you. You can understand that, living the
life which I have described, we were little likely to see anyone
of our own age and position. We had, however, an aunt, my
mother’s maiden sister, Miss Honoria Westphail, who lives
near Harrow, and we were occasionally allowed to pay short
visits at this lady’s house. Julia went there at Christmas two
years ago, and met there a half-pay major of marines, to
whom she became engaged. My step-father learned of the
engagement when my sister returned, and offered no
%%208
objection to the marriage; but within a fortnight of the day which
had been fixed for the wedding, the terrible event occurred
which has deprived me of my only companion.”

Sherlock Holmes had been leaning back in his chair with
his eyes closed and his head sunk in a cushion, but he half
opened his lids now and glanced across at his visitor.

“Pray be precise as to details,” said he.

“It is easy for me to be so, for every event of that dreadful
time is seared into my memory. The manor-house is, as
I have already said, very old, and only one wing is now inhabited.
The bedrooms in this wing are on the ground floor,
the sitting-rooms being in the central block of the buildings.
Of these bedrooms the first is Dr. Roylott’s, the second my
sister’s, and the third my own. There is no communication
between them, but they all open out into the same corridor.
Do I make myself plain?”

“Perfectly so.”

“The windows of the three rooms open out upon the lawn.
That fatal night Dr. Roylott had gone to his room early,
though we knew that he had not retired to rest, for my sister
was troubled by the smell of the strong Indian cigars which
it was his custom to smoke. She left her room, therefore,
and came into mine, where she sat for some time, chatting
about her approaching wedding. At eleven o’clock she rose
to leave me but she paused at the door and looked back.

“\,‘Tell me, Helen,’ said she, ‘have you ever heard any one
whistle in the dead of the night?’

“\,‘Never,’ said I.

“\,‘I suppose that you could not possibly whistle, yourself,
in your sleep?’

“\,‘Certainly not. But why?’

“\,‘Because during the last few nights I have always, about
three in the morning, heard a low, clear whistle. I am a light
sleeper, and it has awakened me. I cannot tell where it came
from -- perhaps from the next room, perhaps from the lawn. I
thought that I would just ask you whether you had heard it.
%%209

“\,‘No, I have not. It must be those wretched gypsies in
the plantation.’

“\,‘Very likely. And yet if it were on the lawn, I wonder
that you did not hear it also.’

“\,‘Ah, but I sleep more heavily than you.’

“\,‘Well, it is of no great consequence, at any rate.’ She
smiled back at me, closed my door, and a few moments later
I heard her key turn in the lock.”

“Indeed,” said Holmes. “Was it your custom always to
lock yourselves in at night?”

“Always.”

“And why?”

“I think that I mentioned to you that the doctor kept a
cheetah and a baboon. We had no feeling of security unless
our doors were locked.”

“Quite so. Pray proceed with your statement.”

“I could not sleep that night. A vague feeling of impending
misfortune impressed me. My sister and I, you will recollect,
were twins, and you know how subtle are the links
which bind two souls which are so closely allied. It was a
wild night. The wind was howling outside, and the rain was
beating and splashing against the windows. Suddenly, amid
all the hubbub of the gale, there burst forth the wild scream
of a terrified woman. I knew that it was my sister’s voice.
I sprang from my bed, wrapped a shawl round me, and rushed
into the corridor. As I opened my door I seemed to hear a
low whistle, such as my sister described, and a few moments
later a clanging sound, as if a mass of metal had fallen. As
I ran down the passage, my sister’s door was unlocked, and
revolved slowly upon its hinges. I stared at it horror-stricken,
not knowing what was about to issue from it. By the light of
the corridor-lamp I saw my sister appear at the opening, her
face blanched with terror, her hands groping for help, her
whole figure swaying to and fro like that of a drunkard. I ran
to her and threw my arms round her, but at that moment her
knees seemed to give way and she fell to the ground. She
%%210
writhed as one who is in terrible pain, and her limbs were
dreadfully convulsed. At first I thought that she had not
recognized me, but as I bent over her she suddenly shrieked
out in a voice which I shall never forget, ‘Oh, my God!
Helen! It was the band! The speckled band!’ There was
something else which she would fain have said, and she
stabbed with her finger into the air in the direction of the
doctor’s room, but a fresh convulsion seized her and choked
her words. I rushed out, calling loudly for my step-father,
and I met him hastening from his room in his dressing-gown.
When he reached my sister’s side she was unconscious, and
though he poured brandy down her throat and sent for medical
aid from the village, all efforts were in vain, for she slowly
sank and died without having recovered her consciousness.
Such was the dreadful end of my beloved sister.”

“One moment,” said Holmes; “are you sure about this
whistle and metallic sound? Could you swear to it?”

“That was what the county coroner asked me at the inquiry.
It is my strong impression that I heard it, and yet, among
the crash of the gale and the creaking of an old house, I may
possibly have been deceived.”

“Was your sister dressed?”

“No, she was in her night-dress. In her right hand was
found the charred stump of a match, and in her left a
matchbox.”

“Showing that she had struck a light and looked about her
when the alarm took place. That is important. And what
conclusions did the coroner come to?”

“He investigated the case with great care, for Dr. Roylott’s
conduct had long been notorious in the county, but he was
unable to find any satisfactory cause of death. My evidence
showed that the door had been fastened upon the inner side,
and the windows were blocked by old-fashioned shutters with
broad iron bars, which were secured every night. The walls
were carefully sounded, and were shown to be quite solid all
round, and the flooring was also thoroughly examined, with
%%211
the same result. The chimney is wide, but is barred up by
four large staples. It is certain, therefore, that my sister was
quite alone when she met her end. Besides, there were no
marks of any violence upon her.”

“How about poison?”

“The doctors examined her for it, but without success.”

“What do you think that this unfortunate lady died of,
then?”

“It is my belief that she died of pure fear and nervous
shock, though what it was that frightened her I cannot
imagine.”

“Were there gypsies in the plantation at the time?”

“Yes, there are nearly always some there.”

“Ah, and what did you gather from this allusion to a band -- a
speckled band?”

“Sometimes I have thought that it was merely the wild talk
of delirium, sometimes that it may have referred to some band
of people, perhaps to these very gypsies in the plantation. I
do not know whether the spotted handkerchiefs which so
many of them wear over their heads might have suggested the
strange adjective which she used.”

Holmes shook his head like a man who is far from being
satisfied.

“These are very deep waters,” said he; “pray go on with
your narrative.”

“Two years have passed since then, and my life has been
until lately lonelier than ever. A month ago, however, a dear
friend, whom I have known for many years, has done me the
honor to ask my hand in marriage. His name is Armitage -- Percy
Armitage -- the second son of Mr.~Armitage, of Crane
Water, near Reading. My step-father has offered no opposition
to the match, and we are to be married in the course of
the spring. Two days ago some repairs were started in the
west wing of the building, and my bedroom wall has been
pierced, so that I have had to move into the chamber in which
my sister died, and to sleep in the very bed in which she
%%212
slept. Imagine, then, my thrill of terror when last night, as I
lay awake, thinking over her terrible fate, I suddenly heard in
the silence of the night the low whistle which had been the
herald of her own death. I sprang up and lit the lamp, but
nothing was to be seen in the room. I was too shaken to go
to bed again, however, so I dressed, and as soon as it was
daylight I slipped down, got a dog-cart at the ‘Crown Inn,’
which is opposite, and drove to Leatherhead, from whence
I have come on this morning with the one object of seeing
you and asking your advice.”

“You have done wisely,” said my friend. “But have you
told me all?”

“Yes, all.”

“Miss Roylott, you have not. You are screening your step-%
father.”

“Why, what do you mean?”

For answer Holmes pushed back the frill of black lace
which fringed the hand that lay upon our visitor’s knee. Five
little livid spots, the marks of four fingers and a thumb, were
printed upon the white wrist.

“You have been cruelly used,” said Holmes.

The lady colored deeply and covered over her injured
wrist. “He is a hard man,” she said, “and perhaps he hardly
knows his own strength.”

There was a long silence, during which Holmes leaned
his chin upon his hands and stared into the crackling
fire.

“This is a very deep business,” he said, at last. “There
are a thousand details which I should desire to know before I
decide upon our course of action. Yet we have not a moment
to lose. If we were to come to Stoke Moran to-day,
would it be possible for us to see over these rooms without
the knowledge of your step-father?”

“As it happens, he spoke of coming into town to-day upon
some most important business. It is probable that he will be
away all day, and that there would be nothing to disturb you.
%%213
We have a house-keeper now, but she is old and foolish, and I
could easily get her out of the way.”

“Excellent. You are not averse to this trip, Watson?”

“By no means.”

“Then we shall both come. What are you going to do
yourself?”

“I have one or two things which I would wish to do now
that I am in town. But I shall return by the twelve o’clock
train, so as to be there in time for your coming.”

“And you may expect us early in the afternoon. I have
myself some small business matters to attend to. Will you
not wait and breakfast?”

“No, I must go. My heart is lightened already since I
have confided my trouble to you. I shall look forward to seeing
you again this afternoon.” She dropped her thick black
veil over her face and glided from the room.

“And what do you think of it all, Watson?” asked Sherlock
Holmes, leaning back in his chair.

“It seems to me to be a most dark and sinister
business.”

“Dark enough and sinister enough.”

“Yet if the lady is correct in saying that the flooring and
walls are sound, and that the door, window, and chimney are
impassable, then her sister must have been undoubtedly alone
when she met her mysterious end.”

“What becomes, then, of these nocturnal whistles, and what
of the very peculiar words of the dying woman?”

“I cannot think.”

“When you combine the ideas of whistles at night, the presence
of a band of gypsies who are on intimate terms with this
old doctor, the fact that we have every reason to believe that
the doctor has an interest in preventing his step-daughter’s
marriage, the dying allusion to a band, and, finally, the fact
that Miss Helen Stoner heard a metallic clang, which might
have been caused by one of those metal bars which secured
the shutters falling back into their place, I think that there is
%%214
good ground to think that the mystery may be cleared along
those lines.”

“But what, then, did the gypsies do?”

“I cannot imagine.”

“I see many objections to any such theory.”

“And so do I. It is precisely for that reason that we are
going to Stoke Moran this day. I want to see whether the
objections are fatal, or if they may be explained away. But
what in the name of the devil!”

The ejaculation had been drawn from my companion by
the fact that our door had been suddenly dashed open, and
that a huge man had framed himself in the aperture. His
costume was a peculiar mixture of the professional and of the
agricultural, having a black top-hat, a long frock-coat, and a
pair of high gaiters, with a hunting-crop swinging in his hand.
So tall was he that his hat actually brushed the cross bar of
the doorway, and his breadth seemed to span it across from
side to side. A large face, seared with a thousand wrinkles,
burned yellow with the sun, and marked with every evil passion,
was turned from one to the other of us, while his deep-set,
bile-shot eyes, and his high, thin, fleshless nose, gave him
somewhat the resemblance to a fierce old bird of prey.

“Which of you is Holmes?” asked this apparition.

“My name, sir; but you have the advantage of me,” said
my companion, quietly.

“I am Dr. Grimesby Roylott, of Stoke Moran.”

“Indeed, doctor,” said Holmes, blandly. “Pray take a
seat.”

“I will do nothing of the kind. My step-daughter has been
here. I have traced her. What has she been saying to you?”

“It is a little cold for the time of the year,” said Holmes.

“What has she been saying to you?” screamed the old man,
furiously.

“But I have heard that the crocuses promise well,” continued
my companion, imperturbably.

“Ha! You put me off, do you?” said our new visitor,
%%215
taking a step forward and shaking his hunting-crop. “I
know you, you scoundrel! I have heard of you before. You
are Holmes, the meddler.”

My friend smiled.

“Holmes, the busybody!”

His smile broadened.

“Holmes, the Scotland-yard Jack-in-office!”

Holmes chuckled heartily. “Your conversation is most
entertaining,” said he. “When you go out close the door, for
there is a decided draught.”

“I will go when I have said my say. Don’t you dare to
meddle with my affairs. I know that Miss Stoner has been
here. I traced her! I am a dangerous man to fall foul of!
See here.” He stepped swiftly forward, seized the poker, and
bent it into a curve with his huge brown hands.

“See that you keep yourself out of my grip,” he snarled,
and hurling the twisted poker into the fireplace, he strode out
of the room.

“He seems a very amiable person,” said Holmes, laughing.
“I am not quite so bulky, but if he had remained I might
have shown him that my grip was not much more feeble than
his own.” As he spoke he picked up the steel poker, and
with a sudden effort straightened it out again.

“Fancy his having the insolence to confound me with the
official detective force! This incident gives zest to our
investigation, however, and I only trust that our little friend will
not suffer from her imprudence in allowing this brute to trace
her. And now, Watson, we shall order breakfast, and afterwards
I shall walk down to Doctors’ Commons, where I hope
to get some data which may help us in this matter.”

\strut

It was nearly one o’clock when Sherlock Holmes returned
from his excursion. He held in his hand a sheet of blue paper,
scrawled over with notes and figures.

“I have seen the will of the deceased wife,” said he. “To
determine its exact meaning I have been obliged to work out
%%216
the present prices of the investments with which it is concerned.
The total income, which at the time of the wife’s
death was little short of £1100, is now, through the fall in
agricultural prices, not more than £750. Each daughter can
claim an income of £250, in case of marriage. It is evident,
therefore, that if both girls had married, this beauty would
have had a mere pittance, while even one of them would cripple
him to a very serious extent. My morning’s work has not
been wasted, since it has proved that he has the very strongest
motives for standing in the way of anything of the sort. And
now, Watson, this is too serious for dawdling, especially as
the old man is aware that we are interesting ourselves in his
affairs; so if you are ready, we shall call a cab and drive to
Waterloo. I should be very much obliged if you would slip
your revolver into your pocket. An Eley’s No. 2 is an excellent
argument with gentlemen who can twist steel pokers into
knots. That and a tooth-brush are, I think, all that we need.”

At Waterloo we were fortunate in catching a train for
Leatherhead, where we hired a trap at the station inn, and
drove for four or five miles through the lovely Surrey lanes.
It was a perfect day, with a bright sun and a few fleecy clouds
in the heavens. The trees and way-side hedges were just
throwing out their first green shoots, and the air was full of
the pleasant smell of the moist earth. To me at least there
was a strange contrast between the sweet promise of the
spring and this sinister quest upon which we were engaged.
My companion sat in the front of the trap, his arms folded,
his hat pulled down over his eyes, and his chin sunk upon his
breast, buried in the deepest thought. Suddenly, however,
he started, tapped me on the shoulder, and pointed over the
meadows.

“Look there!” said he.

A heavily-timbered park stretched up in a gentle slope,
thickening into a grove at the highest point. From amid the
branches there jutted out the gray gables and high roof-tree
of a very old mansion.
%%217

“Stoke Moran?” said he.

“Yes, sir, that be the house of Dr. Grimesby Roylott,” remarked
the driver.

“There is some building going on there,” said Holmes;
“that is where we are going.”

“There’s the village,” said the driver, pointing to a cluster
of roofs some distance to the left; “but if you want to get
to the house, you’ll find it shorter to get over this stile, and so
by the foot-path over the fields. There it is, where the lady
is walking.”

“And the lady, I fancy, is Miss Stoner,” observed Holmes,
shading his eyes. “Yes, I think we had better do as you
suggest.”

We got off, paid our fare, and the trap rattled back on its
way to Leatherhead.

“I thought it as well,” said Holmes, as we climbed the
stile, “that this fellow should think we had come here as
architects, or on some definite business. It may stop his
gossip. Good-afternoon, Miss Stoner. You see that we have
been as good as our word.”

Our client of the morning had hurried forward to meet us
with a face which spoke her joy. “I have been waiting so
eagerly for you,” she cried, shaking hands with us warmly.
“All has turned out splendidly. Dr. Roylott has gone to
town, and it is unlikely that he will be back before evening.”

“We have had the pleasure of making the doctor’s acquaintance,”
said Holmes, and in a few words he sketched out what
had occurred. Miss Stoner turned white to the lips as she
listened.

“Good heavens!” she cried, “he has followed me, then.”

“So it appears.”

“He is so cunning that I never know when I am safe from
him. What will he say when he returns?”

“He must guard himself, for he may find that there is some
one more cunning than himself upon his track. You must
lock yourself up from him to-night. If he is violent, we shall
%%218
take you away to your aunt’s at Harrow. Now, we must make
the best use of our time, so kindly take us at once to the rooms
which we are to examine.”

The building was of gray, lichen-blotched stone, with a high
central portion, and two curving wings, like the claws of a
crab, thrown out on each side. In one of these wings the
windows were broken, and blocked with wooden boards, while
the roof was partly caved in, a picture of ruin. The central
portion was in little better repair, but the right-hand block
was comparatively modern, and the blinds in the windows,
with the blue smoke curling up from the chimneys, showed
that this was where the family resided. Some scaffolding had
been erected against the end wall, and the stone-work had
been broken into, but there were no signs of any workmen at
the moment of our visit. Holmes walked slowly up and down
the ill-trimmed lawn, and examined with deep attention the
outsides of the windows.

“This, I take it, belongs to the room in which you used to
sleep, the centre one to your sister’s, and the one next to the
main building to Dr. Roylott’s chamber?”

“Exactly so. But I am now sleeping in the middle one.”

“Pending the alterations, as I understand. By-the-way,
there does not seem to be any very pressing need for repairs
at that end wall.”

“There were none. I believe that it was an excuse to move
me from my room.”

“Ah! that is suggestive. Now, on the other side of this
narrow wing runs the corridor from which these three rooms
open. There are windows in it, of course?”

“Yes, but very small ones. Too narrow for any one to pass
through.”

“As you both locked your doors at night, your rooms were
unapproachable from that side. Now, would you have the
kindness to go into your room and bar your shutters.”

Miss Stoner did so, and Holmes, after a careful examination
through the open window, endeavored in every way to
%%219
force the shutter open, but without success. There was no
slit through which a knife could be passed to raise the bar.
Then with his lens he tested the hinges, but they were of
solid iron, built firmly into the massive masonry. “Hum!”
said he, scratching his chin in some perplexity; “my theory
certainly presents some difficulties. No one could pass these
shutters if they were bolted. Well, we shall see if the inside
throws any light upon the matter.”

A small side door led into the whitewashed corridor from
which the three bedrooms opened. Holmes refused to examine
the third chamber, so we passed at once to the second,
that in which Miss Stoner was now sleeping, and in which her
sister had met with her fate. It was a homely little room,
with a low ceiling and a gaping fireplace, after the fashion of
old country-houses. A brown chest of drawers stood in one
corner, a narrow white-counterpaned bed in another, and a
dressing-table on the left-hand side of the window. These
articles, with two small wicker-work chairs, made up all the
furniture in the room, save for a square of Wilton carpet in
the centre. The boards round and the panelling of the walls
were of brown, worm-eaten oak, so old and discolored that it
may have dated from the original building of the house.
Holmes drew one of the chairs into a corner and sat silent,
while his eyes travelled round and round and up and down,
taking in every detail of the apartment.

“Where does that bell communicate with?” he asked, at
last, pointing to a thick bell-rope which hung down beside the
bed, the tassel actually lying upon the pillow.

“It goes to the house-keeper’s room.”

“It looks newer than the other things?”

“Yes, it was only put there a couple of years ago.”

“Your sister asked for it, I suppose?”

“No, I never heard of her using it. We used always to get
what we wanted for ourselves.”

“Indeed, it seemed unnecessary to put so nice a bell-pull
there. You will excuse me for a few minutes while I satisfy
%%220
myself as to this floor.” He threw himself down upon his
face with his lens in his hand, and crawled swiftly backward
and forward, examining minutely the cracks between the
boards. Then he did the same with the wood-work with
which the chamber was panelled. Finally he walked over to
the bed, and spent some time in staring at it, and in running
his eye up and down the wall. Finally he took the bell-rope
in his hand and gave it a brisk tug.

“Why, it’s a dummy,” said he.

“Won’t it ring?”

“No, it is not even attached to a wire. This is very interesting.
You can see now that it is fastened to a hook just
above where the little opening for the ventilator is.”

“How very absurd! I never noticed that before.”

“Very strange!” muttered Holmes, pulling at the rope.
“There are one or two very singular points about this room.
For example, what a fool a builder must be to open a ventilator
into another room, when, with the same trouble, he might
have communicated with the outside air!”

“That is also quite modern,” said the lady.

“Done about the same time as the bell-rope?” remarked
Holmes.

“Yes, there were several little changes carried out about
that time.”

“They seem to have been of a most interesting character -- dummy
bell-ropes, and ventilators which do not ventilate.
With your permission, Miss Stoner, we shall now carry our
researches into the inner apartment.”

Dr. Grimesby Roylott’s chamber was larger than that of his
step-daughter, but was as plainly furnished. A camp-bed, a
small wooden shelf full of books, mostly of a technical character,
an arm-chair beside the bed, a plain wooden chair against
the wall, a round table, and a large iron safe were the principal
things which met the eye. Holmes walked slowly round
and examined each and all of them with the keenest interest.

“What’s in here?” he asked, tapping the safe.
%%221

“My step-father’s business papers.”

“Oh! you have seen inside, then?”

“Only once, some years ago. I remember that it was full
of papers.”

“There isn’t a cat in it, for example?”

“No. What a strange idea!”

“Well, look at this!” He took up a small saucer of milk
which stood on the top of it.

“No; we don’t keep a cat. But there is a cheetah and a
baboon.”

“Ah, yes, of course! Well, a cheetah is just a big cat, and
yet a saucer of milk does not go very far in satisfying its
wants, I dare say. There is one point which I should wish to
determine.” He squatted down in front of the wooden chair,
and examined the seat of it with the greatest attention.

“Thank you. That is quite settled,” said he, rising and
putting his lens in his pocket. “Hello! Here is something
interesting!”

The object which had caught his eye was a small dog-lash
hung on one corner of the bed. The lash, however, was curled
upon itself, and tied so as to make a loop of whip-cord.

“What do you make of that, Watson?”

“It’s a common enough lash. But I don’t know why it
should be tied.”

“That is not quite so common, is it? Ah, me! it’s a wicked
world, and when a clever man turns his brains to crime it is
the worst of all. I think that I have seen enough now, Miss
Stoner, and with your permission we shall walk out upon the
lawn.”

I had never seen my friend’s face so grim or his brow so
dark as it was when we turned from the scene of this investigation.
We had walked several times up and down the lawn,
neither Miss Stoner nor myself liking to break in upon his
thoughts before he roused himself from his reverie.

“It is very essential, Miss Stoner,” said he, “that you
should absolutely follow my advice in every respect.”
%%222

“I shall most certainly do so.”

“The matter is too serious for any hesitation. Your life
may depend upon your compliance.”

“I assure you that I am in your hands.”

“In the first place, both my friend and I must spend the
night in your room.”

Both Miss Stoner and I gazed at him in astonishment.

“Yes, it must be so. Let me explain. I believe that that
is the village inn over there?”

“Yes, that is the ‘Crown.’\,”

“Very good. Your windows would be visible from there?”

“Certainly.”

“You must confine yourself to your room, on pretence of a
headache, when your step-father comes back. Then when you
hear him retire for the night, you must open the shutters of
your window, undo the hasp, put your lamp there as a signal
to us, and then withdraw quietly with everything which you
are likely to want into the room which you used to occupy.
I have no doubt that, in spite of the repairs, you could manage
there for one night.”

“Oh yes, easily.”

“The rest you will leave in our hands.”

“But what will you do?”

“We shall spend the night in your room, and we shall investigate
the cause of this noise which has disturbed you.”

“I believe, Mr.~Holmes, that you have already made up
your mind,” said Miss Stoner, laying her hand upon my companion’s
sleeve.

“Perhaps I have.”

“Then for pity’s sake tell me what was the cause of my
sister’s death.”

“I should prefer to have clearer proofs before I speak.”

“You can at least tell me whether my own thought is correct,
and if she died from some sudden fright.”

“No, I do not think so. I think that there was probably
some more tangible cause. And now, Miss Stoner, we must
%%223
%%“\,‘GOOD-BYE, AND BE BRAVE’\,”
%%224
leave you, for if Dr. Roylott returned and saw us, our journey
would be in vain. Good-bye, and be brave, for if you will do
what I have told you, you may rest assured that we shall soon
drive away the dangers that threaten you.”

Sherlock Holmes and I had no difficulty in engaging a bedroom
and sitting-room at the “Crown Inn.” They were on
the upper floor, and from our window we could command a
view of the avenue gate, and of the inhabited wing of Stoke
Moran Manor House. At dusk we saw Dr. Grimesby Roylott
drive past, his huge form looming up beside the little figure
of the lad who drove him. The boy had some slight
difficulty in undoing the heavy iron gates, and we heard the
hoarse roar of the doctor’s voice, and saw the fury with which
he shook his clinched fists at him. The trap drove on, and a
few minutes later we saw a sudden light spring up among the
trees as the lamp was lit in one of the sitting-rooms.

“Do you know, Watson,” said Holmes, as we sat together
in the gathering darkness, “I have really some scruples as to
taking you to-night. There is a distinct element of danger.”

“Can I be of assistance?”

“Your presence might be invaluable.”

“Then I shall certainly come.”

“It is very kind of you.”

“You speak of danger. You have evidently seen more in
these rooms than was visible to me.”

“No, but I fancy that I may have deduced a little more.
I imagine that you saw all that I did.”

“I saw nothing remarkable save the bell-rope, and what
purpose that could answer I confess is more than I can
imagine.”

“You saw the ventilator, too?”

“Yes, but I do not think that it is such a very unusual thing
to have a small opening between two rooms. It was so small
that a rat could hardly pass through.”

“I knew that we should find a ventilator before ever we
came to Stoke Moran.”
%%226

“My dear Holmes!”

“Oh yes, I did. You remember in her statement she said
that her sister could smell Dr. Roylott’s cigar. Now, of course
that suggested at once that there must be a communication
between the two rooms. It could only be a small one, or it
would have been remarked upon at the coroner’s inquiry. I
deduced a ventilator.”

“But what harm can there be in that?”

“Well, there is at least a curious coincidence of dates. A
ventilator is made, a cord is hung, and a lady who sleeps in
the bed dies. Does not that strike you?”

“I cannot as yet see any connection.”

“Did you observe anything very peculiar about that bed?”

“No.”

“It was clamped to the floor. Did you ever see a bed
fastened like that before?”

“I cannot say that I have.”

“The lady could not move her bed. It must always be in
the same relative position to the ventilator and to the rope --
for so we may call it, since it was clearly never meant for a
bell-pull.”

“Holmes,” I cried, “I seem to see dimly what you are
hinting at. We are only just in time to prevent some subtle
and horrible crime.”

“Subtle enough and horrible enough. When a doctor
does go wrong, he is the first of criminals. He has nerve
and he has knowledge. Palmer and Pritchard were among
the heads of their profession. This man strikes even deeper,
but I think, Watson, that we shall be able to strike deeper
still. But we shall have horrors enough before the night is
over; for goodness’ sake let us have a quiet pipe, and turn
our minds for a few hours to something more cheerful.”

\strut

About nine o’clock the light among the trees was extinguished,
and all was dark in the direction of the Manor
House. Two hours passed slowly away, and then, suddenly,
%%227
just at the stroke of eleven, a single bright light shone out
right in front of us.

“That is our signal,” said Holmes, springing to his feet;
“it comes from the middle window.”

As we passed out he exchanged a few words with the
landlord, explaining that we were going on a late visit to an
acquaintance, and that it was possible that we might spend the
night there. A moment later we were out on the dark road,
a chill wind blowing in our faces, and one yellow light twinkling
in front of us through the gloom to guide us on our sombre
errand.

There was little difficulty in entering the grounds, for unrepaired
breaches gaped in the old park wall. Making our way
among the trees, we reached the lawn, crossed it, and were
about to enter through the window, when out from a clump
of laurel bushes there darted what seemed to be a hideous
and distorted child, who threw itself upon the grass with
writhing limbs, and then ran swiftly across the lawn into the
darkness.

“My God!” I whispered; “did you see it?”

Holmes was for the moment as startled as I. His hand
closed like a vice upon my wrist in his agitation. Then he
broke into a low laugh, and put his lips to my ear.

“It is a nice household,” he murmured. “That is the
baboon.”

I had forgotten the strange pets which the doctor affected.
There was a cheetah, too; perhaps we might find it upon our
shoulders at any moment. I confess that I felt easier in my
mind when, after following Holmes’s example and slipping off
my shoes, I found myself inside the bedroom. My companion
noiselessly closed the shutters, moved the lamp onto the
table, and cast his eyes round the room. All was as we had
seen it in the daytime. Then creeping up to me and making
a trumpet of his hand, he whispered into my ear again so
gently that it was all that I could do to distinguish the words:

“The least sound would be fatal to our plans.”
%%228

I nodded to show that I had heard.

“We must sit without light. He would see it through the
ventilator.”

I nodded again.

“Do not go asleep; your very life may depend upon it.
Have your pistol ready in case we should need it. I will sit
on the side of the bed, and you in that chair.”

I took out my revolver and laid it on the corner of the table.

Holmes had brought up a long thin cane, and this he
placed upon the bed beside him. By it he laid the box of
matches and the stump of a candle. Then he turned down
the lamp, and we were left in darkness.

How shall I ever forget that dreadful vigil? I could not
hear a sound, not even the drawing of a breath, and yet I
knew that my companion sat open-eyed, within a few feet of
me, in the same state of nervous tension in which I was myself.
The shutters cut off the least ray of light, and we
waited in absolute darkness. From outside came the occasional
cry of a night-bird, and once at our very window a long
drawn cat-like whine, which told us that the cheetah was indeed
at liberty. Far away we could hear the deep tones of
the parish clock, which boomed out every quarter of an hour.
How long they seemed, those quarters! Twelve struck, and
one and two and three, and still we sat waiting silently for
whatever might befall.

Suddenly there was the momentary gleam of a light up in
the direction of the ventilator, which vanished immediately,
but was succeeded by a strong smell of burning oil and heated
metal. Some one in the next room had lit a dark-lantern.
I heard a gentle sound of movement, and then all was silent
once more, though the smell grew stronger. For half an hour
I sat with straining ears. Then suddenly another sound became
audible -- a very gentle, soothing sound, like that of a
small jet of steam escaping continually from a kettle. The
instant that we heard it, Holmes sprang from the bed, struck
a match, and lashed furiously with his cane at the bell-pull.
%%229

“You see it, Watson?” he yelled. “You see it?”

But I saw nothing. At the moment when Holmes struck
the light I heard a low, clear whistle, but the sudden glare
flashing into my weary eyes made it impossible for me to tell
what it was at which my friend lashed so savagely. I could,
however, see that his face was deadly pale, and filled with
horror and loathing.

He had ceased to strike, and was gazing up at the ventilator,
when suddenly there broke from the silence of the night
the most horrible cry to which I have ever listened. It
swelled up louder and louder, a hoarse yell of pain and fear
and anger all mingled in the one dreadful shriek. They say
that away down in the village, and even in the distant parsonage,
that cry raised the sleepers from their beds. It struck
cold to our hearts, and I stood gazing at Holmes, and he at
me, until the last echoes of it had died away into the silence
from which it rose.

“What can it mean?” I gasped.

“It means that it is all over,” Holmes answered. “And
perhaps, after all, it is for the best. Take your pistol, and we
will enter Dr. Roylott’s room.”

With a grave face he lit the lamp and led the way down
the corridor. Twice he struck at the chamber door without
any reply from within. Then he turned the handle and entered,
I at his heels, with the cocked pistol in my hand.

It was a singular sight which met our eyes. On the table
stood a dark-lantern with the shutter half open, throwing a
brilliant beam of light upon the iron safe, the door of which
was ajar. Beside this table, on the wooden chair, sat Dr.
Grimesby Roylott, clad in a long gray dressing-gown, his bare
ankles protruding beneath, and his feet thrust into red heelless
Turkish slippers. Across his lap lay the short stock with
the long lash which we had noticed during the day. His chin
was cocked upward and his eyes were fixed in a dreadful,
rigid stare at the corner of the ceiling. Round his brow he
had a peculiar yellow band, with brownish speckles, which
%%230
seemed to be bound tightly round his head. As we entered
he made neither sound nor motion.

“The band! the speckled band!” whispered Holmes.

I took a step forward. In an instant his strange head-gear
began to move, and there reared itself from among his hair
the squat diamond-shaped head and puffed neck of a loathsome
serpent.

“It is a swamp adder!” cried Holmes; “the deadliest
snake in India. He has died within ten seconds of being
bitten. Violence does, in truth, recoil upon the violent, and
the schemer falls into the pit which he digs for another. Let
us thrust this creature back into its den, and we can then remove
Miss Stoner to some place of shelter, and let the county
police know what has happened.”

As he spoke he drew the dog-whip swiftly from the dead
man’s lap, and throwing the noose round the reptile’s neck, he
drew it from its horrid perch, and carrying it at arm’s length,
threw it into the iron safe, which he closed upon it.

\strut

Such are the true facts of the death of Dr. Grimesby Roylott,
of Stoke Moran. It is not necessary that I should prolong
a narrative which has already run to too great a length,
by telling how we broke the sad news to the terrified girl, how
we conveyed her by the morning train to the care of her good
aunt at Harrow, of how the slow process of official inquiry
came to the conclusion that the doctor met his fate while
indiscreetly playing with a dangerous pet. The little which I
had yet to learn of the case was told me by Sherlock Holmes
as we travelled back next day.

“I had,” said he, “come to an entirely erroneous conclusion,
which shows, my dear Watson, how dangerous it always is to
reason from insufficient data. The presence of the gypsies,
and the use of the word ‘band,’ which was used by the poor
girl, no doubt to explain the appearance which she had caught
a hurried glimpse of by the light of her match, were sufficient
to put me upon an entirely wrong scent. I can only claim
%%231
the merit that I instantly reconsidered my position when,
however, it became clear to me that whatever danger threatened
an occupant of the room could not come either from the
window or the door. My attention was speedily drawn, as I
have already remarked to you, to this ventilator, and to the
bell-rope which hung down to the bed. The discovery that
this was a dummy, and that the bed was clamped to the floor,
instantly gave rise to the suspicion that the rope was there as
bridge for something passing through the hole, and coming to
the bed. The idea of a snake instantly occurred to me, and
when I coupled it with my knowledge that the doctor was
furnished with a supply of creatures from India, I felt that I
was probably on the right track. The idea of using a form
of poison which could not possibly be discovered by any
chemical test was just such a one as would occur to a clever
and ruthless man who had had an Eastern training. The
rapidity with which such a poison would take effect would
also, from his point of view, be an advantage. It would be a
sharp-eyed coroner, indeed, who could distinguish the two little
dark punctures which would show where the poison fangs
had done their work. Then I thought of the whistle. Of
course he must recall the snake before the morning light revealed
it to the victim. He had trained it, probably by the
use of the milk which we saw, to return to him when summoned.
He would put it through this ventilator at the hour
that he thought best, with the certainty that it would crawl
down the rope and land on the bed. It might or might not
bite the occupant, perhaps she might escape every night for a
week, but sooner or later she must fall a victim.

“I had come to these conclusions before ever I had entered
his room. An inspection of his chair showed me that he had
been in the habit of standing on it, which of course would be
necessary in order that he should reach the ventilator. The
sight of the safe, the saucer of milk, and the loop of
whipcord were enough to finally dispel any doubts which may
have remained. The metallic clang heard by Miss Stoner
%%232
was obviously caused by her step-father hastily closing the door
of his safe upon its terrible occupant. Having once made up
my mind, you know the steps which I took in order to put the
matter to the proof. I heard the creature hiss, as I have no
doubt that you did also, and I instantly lit the light and attacked
it.”

“With the result of driving it through the ventilator.”

“And also with the result of causing it to turn upon its
master at the other side. Some of the blows of my cane
came home, and roused its snakish temper, so that it flew
upon the first person it saw. In this way I am no doubt indirectly
responsible for Dr. Grimesby Roylott’s death, and I
cannot say that it is likely to weigh very heavily upon my
conscience.”
%%233

\Chapter{The Adventure Of The Engineer’s Thumb}

\textsc{Of} all the problems which have been submitted to
my friend Mr.~Sherlock Holmes for solution during
the years of our intimacy, there were only
two which I was the means of introducing to his
notice -- that of Mr.~Hatherley’s thumb, and that of Colonel
Warburton’s madness. Of these the latter may have afforded
a finer field for an acute and original observer, but the other
was so strange in its inception and so dramatic in its details,
that it may be the more worthy of being placed upon record,
even if it gave my friend fewer openings for those deductive
methods of reasoning by which he achieved such remarkable
results. The story has, I believe, been told more than once
in the newspapers, but, like all such narratives, its effect is
much less striking when set forth \textit{en bloc} in a single half-%
column of print than when the facts slowly evolve before your
own eyes, and the mystery clears gradually away as each new
discovery furnishes a step which leads on to the complete
truth. At the time the circumstances made a deep impression
upon me, and the lapse of two years has hardly served
to weaken the effect.

It was in the summer of ’89, not long after my marriage,
that the events occurred which I am now about to summarize.
I had returned to civil practice, and had finally abandoned
Holmes in his Baker Street rooms, although I continually
visited him, and occasionally even persuaded him to forego
his Bohemian habits so far as to come and visit us. My practice
had steadily increased, and as I happened to live at no
%%234
very great distance from Paddington Station, I got a few
patients from among the officials. One of these, whom I
had cured of a painful and lingering disease, was never
weary of advertising my virtues, and of endeavoring to
send me on every sufferer over whom he might have any
influence.

One morning, at a little before seven o’clock, I was awakened
by the maid tapping at the door, to announce that two
men had come from Paddington, and were waiting in the
consulting-room. I dressed hurriedly, for I knew by experience
that railway cases were seldom trivial, and hastened downstairs.
As I descended, my old ally, the guard, came out of
the room and closed the door tightly behind him.

“I’ve got him here,” he whispered, jerking his thumb over
his shoulder; “he’s all right.”

“What is it, then?” I asked, for his manner suggested that
it was some strange creature which he had caged up in my
room.

“It’s a new patient,” he whispered. “I thought I’d bring
him round myself; then he couldn’t slip away. There he is,
all safe and sound. I must go now, doctor; I have my dooties,
just the same as you.” And off he went, this trusty
tout, without even giving me time to thank him.

I entered my consulting-room and found a gentleman seated
by the table. He was quietly dressed in a suit of heather
tweed, with a soft cloth cap, which he had laid down upon
my books. Round one of his hands he had a handkerchief
wrapped, which was mottled all over with blood-stains. He
was young, not more than five-and-twenty, I should say, with
a strong, masculine face; but he was exceedingly pale, and
gave me the impression of a man who was suffering from
some strong agitation, which it took all his strength of mind
to control.

“I am sorry to knock you up so early, doctor,” said he,
“but I have had a very serious accident during the night. I
came in by train this morning, and on inquiring at
%%235
Paddington as to where I might find a doctor, a worthy fellow very
kindly escorted me here. I gave the maid a card, but I see
that she has left it upon the side-table.”

I took it up and glanced at it. “Mr.~Victor Hatherley, hydraulic
engineer, 16\textsc{A}, Victoria Street (3d floor).” That was
the name, style, and abode of my morning visitor. “I regret
that I have kept you waiting,” said I, sitting down in my
library-chair. “You are fresh from a night journey, I understand,
which is in itself a monotonous occupation.”

“Oh, my night could not be called monotonous,” said he,
and laughed. He laughed very heartily, with a high, ringing
note, leaning back in his chair and shaking his sides. All my
medical instincts rose up against that laugh.

“Stop it!” I cried; “pull yourself together!” and I poured
out some water from a caraffe.

It was useless, however. He was off in one of those hysterical
outbursts which come upon a strong nature when some
great crisis is over and gone. Presently he came to himself
once more, very weary and blushing hotly.

“I have been making a fool of myself,” he gasped.

“Not at all. Drink this.” I dashed some brandy into the
water, and the color began to come back to his bloodless
cheeks.

“That’s better!” said he. “And now, doctor, perhaps you
would kindly attend to my thumb, or rather to the place where
my thumb used to be.”

He unwound the handkerchief and held out his hand. It
gave even my hardened nerves a shudder to look at it. There
were four protruding fingers and a horrid red, spongy surface
where the thumb should have been. It had been hacked or
torn right out from the roots.

“Good heavens!” I cried, “this is a terrible injury. It
must have bled considerably.”

“Yes, it did. I fainted when it was done, and I think that
I must have been senseless for a long time. When I came
to I found that it was still bleeding, so I tied one end of my
%%236
handkerchief very tightly round the wrist, and braced it up
with a twig.”

“Excellent! You should have been a surgeon.”

“It is a question of hydraulics, you see, and came within
my own province.”

“This has been done,” said I, examining the wound, “by
a very heavy and sharp instrument.”

“A thing like a cleaver,” said he.

“An accident, I presume?”

“By no means.”

“What! a murderous attack?”

“Very murderous indeed.”

“You horrify me.”

I sponged the wound, cleaned it, dressed it, and finally covered
it over with cotton wadding and carbolized bandages.
He lay back without wincing, though he bit his lip from time
to time.

“How is that?” I asked, when I had finished.

“Capital! Between your brandy and your bandage, I feel
a new man. I was very weak, but I have had a good deal to
go through.”

“Perhaps you had better not speak of the matter. It is evidently
trying to your nerves.”

“Oh no, not now. I shall have to tell my tale to the police;
but, between ourselves, if it were not for the convincing
evidence of this wound of mine, I should be surprised if they
believed my statement; for it is a very extraordinary one, and
I have not much in the way of proof with which to back it up;
and, even if they believe me, the clews which I can give them
are so vague that it is a question whether justice will be
done.”

“Ha!” cried I, “if it is anything in the nature of a problem
which you desire to see solved, I should strongly recommend
you to come to my friend Mr.~Sherlock Holmes before
you go to the official police.”

“Oh, I have heard of that fellow,” answered my visitor,
%%237
“and I should be very glad if he would take the matter up,
though of course I must use the official police as well. Would
you give me an introduction to him?”

“I’ll do better. I’ll take you round to him myself.”

“I should be immensely obliged to you.”

“We’ll call a cab and go together. We shall just be in time
to have a little breakfast with him. Do you feel equal to it?”

“Yes; I shall not feel easy until I have told my story.”

“Then my servant will call a cab, and I shall be with you
in an instant.” I rushed up-stairs, explained the matter shortly
to my wife, and in five minutes was inside a hansom, driving
with my new acquaintance to Baker Street.

Sherlock Holmes was, as I expected, lounging about his
sitting-room in his dressing-gown, reading the agony column
of \textit{The Times}, and smoking his before-breakfast pipe, which
was composed of all the plugs and dottels left from his smokes
of the day before, all carefully dried and collected on the corner
of the mantel-piece. He received us in his quietly genial
fashion, ordered fresh rashers and eggs, and joined us in a
hearty meal. When it was concluded he settled our new acquaintance
upon the sofa, placed a pillow beneath his head,
and laid a glass of brandy-and-water within his reach.

“It is easy to see that your experience has been no common
one, Mr.~Hatherley,” said he. “Pray, lie down there
and make yourself absolutely at home. Tell us what you can,
but stop when you are tired, and keep up your strength with
a little stimulant.”

“Thank you,” said my patient, “but I have felt another
man since the doctor bandaged me, and I think that your
breakfast has completed the cure. I shall take up as little of
your valuable time as possible, so I shall start at once upon
my peculiar experiences.”

Holmes sat in his big arm-chair with the weary, heavy-lidded
expression which veiled his keen and eager nature, while
I sat opposite to him, and we listened in silence to the strange
story which our visitor detailed to us.
%%238

“You must know,” said he, “that I am an orphan and a
bachelor, residing alone in lodgings in London. By profession
I am an hydraulic engineer, and I have had considerable
experience of my work during the seven years that I was
apprenticed to Venner \& Matheson, the well-known firm, of
Greenwich. Two years ago, having served my time, and having
also come into a fair sum of money through my poor
father’s death, I determined to start in business for myself,
and took professional chambers in Victoria Street.

“I suppose that every one finds his first independent start
in business a dreary experience. To me it has been exceptionally
so. During two years I have had three consultations
and one small job, and that is absolutely all that my profession
has brought me. My gross takings amount to £27 10\textit{s.}
Every day, from nine in the morning until four in the afternoon,
I waited in my little den, until at last my heart began
to sink, and I came to believe that I should never have any
practice at all.

“Yesterday, however, just as I was thinking of leaving the
office, my clerk entered to say there was a gentleman waiting
who wished to see me upon business. He brought up a card,
too, with the name of ‘Colonel Lysander Stark’ engraved
upon it. Close at his heels came the colonel himself, a man
rather over the middle size, but of an exceeding thinness. I
do not think that I have ever seen so thin a man. His whole
face sharpened away into nose and chin, and the skin of his
cheeks was drawn quite tense over his outstanding bones. Yet
this emaciation seemed to be his natural habit, and due to no
disease, for his eye was bright, his step brisk, and his bearing
assured. He was plainly but neatly dressed, and his age, I
should judge, would be nearer forty than thirty.

“\,‘Mr.~Hatherley?’ said he, with something of a German
accent. ‘You have been recommended to me, Mr.~Hatherley,
as being a man who is not only proficient in his profession,
but is also discreet and capable of preserving a secret.’

“I bowed, feeling as flattered as any young man would at
%%239
such an address. ‘May I ask who it was who gave me so
good a character?’

“\,‘Well, perhaps it is better that I should not tell you that
just at this moment. I have it from the same source that
you are both an orphan and a bachelor, and are residing
alone in London.’

“\,‘That is quite correct,’ I answered, ‘but you will excuse
me if I say that I cannot see how all this bears upon my
professional qualifications. I understood that it was on a
professional matter that you wished to speak to me?’

“\,‘Undoubtedly so. But you will find that all I say is really
to the point. I have a professional commission for you, but
absolute secrecy is quite essential -- \textit{absolute} secrecy, you
understand, and of course we may expect that more from a man
who is alone than from one who lives in the bosom of his
family.’

“\,‘If I promise to keep a secret,’ said I, ‘you may absolutely
depend upon my doing so.’

“He looked very hard at me as I spoke, and it seemed to
me that I had never seen so suspicious and questioning an
eye.

“\,‘Do you promise, then?’ said he, at last.

“\,‘Yes, I promise.’

“\,‘Absolute and complete silence before, during, and after?
No reference to the matter at all, either in word or writing?’

“\,‘I have already given you my word.’

“\,‘Very good.’ He suddenly sprang up, and darting like
lightning across the room, he flung open the door. The passage
outside was empty.

“\,‘That’s all right,’ said he, coming back. ‘I know that clerks
are sometimes curious as to their master’s affairs. Now we
can talk in safety.’ He drew up his chair very close to mine,
and began to stare at me again with the same questioning and
thoughtful look.

“A feeling of repulsion, and of something akin to fear had
begun to rise within me at the strange antics of this fleshless
%%240
man. Even my dread of losing a client could not restrain
me from showing my impatience.

“\,‘I beg that you will state your business, sir,’ said I; ‘my
time is of value.’ Heaven forgive me for that last sentence,
but the words came to my lips.

“\,‘How would fifty guineas for a night’s work suit you?’
he asked.

“\,‘Most admirably.’

“\,‘I say a night’s work, but an hour’s would be nearer the
mark. I simply want your opinion about a hydraulic stamping
machine which has got out of gear. If you show us what
is wrong we shall soon set it right ourselves. What do you
think of such a commission as that?’

“\,‘The work appears to be light and the pay munificent.’

“\,‘Precisely so. We shall want you to come to-night by
the last train.’

“\,‘Where to?’

“\,‘To Eyford, in Berkshire. It is a little place near the
borders of Oxfordshire, and within seven miles of Reading.
There is a train from Paddington which would bring you
there at about 11.15.

“\,‘Very good.’

“\,‘I shall come down in a carriage to meet you.’

“\,‘There is a drive, then?’

“\,‘Yes, our little place is quite out in the country. It is a
good seven miles from Eyford Station.’

“\,‘Then we can hardly get there before midnight. I suppose
there would be no chance of a train back. I should be
compelled to stop the night.’

“\,‘Yes, we could easily give you a shake-down.’

“\,‘That is very awkward. Could I not come at some more
convenient hour?’

“\,‘We have judged it best that you should come late. It is
to recompense you for any inconvenience that we are paying
to you, a young and unknown man, a fee which would buy an
opinion from the very heads of your profession. Still, of
%%241
course, if you would like to draw out of the business, there is
plenty of time to do so.’

“I thought of the fifty guineas, and of how very useful they
would be to me. ‘Not at all,’ said I, ‘I shall be very happy
to accommodate myself to your wishes. I should like, however,
to understand a little more clearly what it is that you
wish me to do.’

“\,‘Quite so. It is very natural that the pledge of secrecy
which we have exacted from you should have aroused your
curiosity. I have no wish to commit you to anything without
your having it all laid before you. I suppose that we are
absolutely safe from eavesdroppers?’

“\,‘Entirely.’

“\,‘Then the matter stands thus. You are probably aware
that fuller’s-earth is a valuable product, and that it is only
found in one or two places in England?’

“\,‘I have heard so.’

“\,‘Some little time ago I bought a small place -- a very small
place -- within ten miles of Reading. I was fortunate enough
to discover that there was a deposit of fuller’s-earth in one of
my fields. On examining it, however, I found that this deposit
was a comparatively small one, and that it formed a link
between two very much larger ones upon the right and left -- both
of them, however, in the grounds of my neighbors. These
good people were absolutely ignorant that their land contained
that which was quite as valuable as a gold-mine. Naturally,
it was to my interest to buy their land before they discovered
its true value; but, unfortunately, I had no capital by which
I could do this. I took a few of my friends into the secret,
however, and they suggested that we should quietly and secretly
work our own little deposit, and that in this way we
should earn the money which would enable us to buy the
neighboring fields. This we have now been doing for some
time, and in order to help us in our operations we erected an
hydraulic press. This press, as I have already explained, has
got out of order, and we wish your advice upon the subject.
%%242
We guard our secret very jealously, however, and if it once
became known that we had hydraulic engineers coming to our
little house, it would soon rouse inquiry, and then, if the facts
came out, it would be good-bye to any chance of getting these
fields and carrying out our plans. That is why I have made
you promise me that you will not tell a human being that you
are going to Eyford to-night. I hope that I make it all plain?’

“\,‘I quite follow you,’ said I. ‘The only point which I
could not quite understand, was what use you could make of
an hydraulic press in excavating fuller’s-earth, which, as I
understand, is dug out like gravel from a pit.’

“\,‘Ah!’ said he, carelessly, ‘we have our own process. We
compress the earth into bricks, so as to remove them without
revealing what they are. But that is a mere detail. I have
taken you fully into my confidence now, Mr.~Hatherley, and I
have shown you how I trust you.’ He rose as he spoke. ‘I
shall expect you, then, at Eyford at 11.15.’

“\,‘I shall certainly be there.’

“\,‘And not a word to a soul.’ He looked at me with a last,
long, questioning gaze, and then, pressing my hand in a cold,
dank grasp, he hurried from the room.

“Well, when I came to think it all over in cool blood I was
very much astonished, as you may both think, at this sudden
commission which had been intrusted to me. On the one
hand, of course, I was glad, for the fee was at least tenfold
what I should have asked had I set a price upon my own
services, and it was possible that this order might lead to
other ones. On the other hand, the face and manner of my
patron had made an unpleasant impression upon me, and I
could not think that his explanation of the fuller’s-earth was
sufficient to explain the necessity for my coming at midnight,
and his extreme anxiety lest I should tell any one of my errand.
However, I threw all fears to the winds, ate a hearty
supper, drove to Paddington, and started off, having obeyed
to the letter the injunction as to holding my tongue.

“At Reading I had to change not only my carriage, but my
%%243
%%“\,‘NOT A WORD TO A SOUL’\,”
%%244
station. However, I was in time for the last train to Eyford,
and I reached the little dim-lit station after eleven o’clock. I
was the only passenger who got out there, and there was no
one upon the platform save a single sleepy porter with a lantern.
As I passed out through the wicket gate, however, I
found my acquaintance of the morning waiting in the shadow
upon the other side. Without a word he grasped my arm and
hurried me into a carriage, the door of which was standing
open. He drew up the windows on either side, tapped on the
wood-work, and away we went as fast as the horse could go.”

“One horse?” interjected Holmes.

“Yes, only one.”

“Did you observe the color?”

“Yes, I saw it by the side-lights when I was stepping into
the carriage. It was a chestnut.”

“Tired-looking or fresh?”

“Oh, fresh and glossy.”

“Thank you. I am sorry to have interrupted you. Pray
continue your most interesting statement.”

“Away we went then, and we drove for at least an hour.
Colonel Lysander Stark had said that it was only seven miles,
but I should think, from the rate that we seemed to go, and
from the time that we took, that it must have been nearer
twelve. He sat at my side in silence all the time, and I was
aware, more than once when I glanced in his direction, that
he was looking at me with great intensity. The country roads
seem to be not very good in that part of the world, for we
lurched and jolted terribly. I tried to look out of the windows
to see something of where we were, but they were made
of frosted glass, and I could make out nothing save the occasional
bright blur of a passing light. Now and then I hazarded
some remark to break the monotony of the journey,
but the colonel answered only in monosyllables, and the conversation
soon flagged. At last, however, the bumping of the
road was exchanged for the crisp smoothness of a gravel-drive,
and the carriage came to a stand. Colonel Lysander
%%246
Stark sprang out, and, as I followed after him, pulled me
swiftly into a porch which gaped in front of us. We stepped,
as it were, right out of the carriage and into the hall, so
that I failed to catch the most fleeting glance of the front of
the house. The instant that I had crossed the threshold the
door slammed heavily behind us, and I heard faintly the rattle
of the wheels as the carriage drove away.

“It was pitch dark inside the house, and the colonel fumbled
about looking for matches, and muttering under his
breath. Suddenly a door opened at the other end of the
passage, and a long, golden bar of light shot out in our direction.
It grew broader, and a woman appeared with a lamp in
her hand, which she held above her head, pushing her face
forward and peering at us. I could see that she was pretty,
and from the gloss with which the light shone upon her dark
dress I knew that it was a rich material. She spoke a few
words in a foreign tongue in a tone as though asking a question,
and when my companion answered in a gruff monosyllable
she gave such a start that the lamp nearly fell from her
hand. Colonel Stark went up to her, whispered something in
her ear, and then, pushing her back into the room from whence
she had come, he walked towards me again with the lamp in
his hand.

“\,‘Perhaps you will have the kindness to wait in this room
for a few minutes,’ said he, throwing open another door. It
was a quiet, little, plainly-furnished room, with a round table
in the centre, on which several German books were scattered.
Colonel Stark laid down the lamp on the top of a harmonium
beside the door. ‘I shall not keep you waiting an instant,’
said he, and vanished into the darkness.

“I glanced at the books upon the table, and in spite of my
ignorance of German I could see that two of them were treatises
on science, the others being volumes of poetry. Then I
walked across to the window, hoping that I might catch some
glimpse of the country-side, but an oak shutter, heavily barred,
was folded across it. It was a wonderfully silent house.
%%247
There was an old clock ticking loudly somewhere in the passage,
but otherwise everything was deadly still. A vague
feeling of uneasiness began to steal over me. Who were
these German people, and what were they doing, living in this
strange, out-of-the-way place? And where was the place? I
was ten miles or so from Eyford, that was all I knew, but
whether north, south, east, or west I had no idea. For that
matter, Reading, and possibly other large towns, were within
that radius, so the place might not be so secluded, after all.
Yet it was quite certain, from the absolute stillness, that we
were in the country. I paced up and down the room, humming
a tune under my breath to keep up my spirits, and feeling
that I was thoroughly earning my fifty-guinea fee.

“Suddenly, without any preliminary sound in the midst of
the utter stillness, the door of my room swung slowly open.
The woman was standing in the aperture, the darkness of the
hall behind her, the yellow light from my lamp beating upon
her eager and beautiful face. I could see at a glance that she
was sick with fear, and the sight sent a chill to my own heart.
She held up one shaking finger to warn me to be silent, and
she shot a few whispered words of broken English at me, her
eyes glancing back, like those of a frightened horse, into the
gloom behind her.

“\,‘I would go,’ said she, trying hard, as it seemed to me, to
speak calmly; ‘I would go. I should not stay here. There
is no good for you to do.’

“\,‘But, madam,’ said I, ‘I have not yet done what I came
for. I cannot possibly leave until I have seen the machine.’

“\,‘It is not worth your while to wait,’ she went on. ‘You
can pass through the door; no one hinders.’ And then, seeing
that I smiled and shook my head, she suddenly threw
aside her constraint and made a step forward, with her hands
wrung together. ‘For the love of Heaven!’ she whispered,
‘get away from here before it is too late!’

“But I am somewhat headstrong by nature, and the more
ready to engage in an affair when there is some obstacle in
%%248
the way. I thought of my fifty-guinea fee, of my wearisome
journey, and of the unpleasant night which seemed to be before
me. Was it all to go for nothing? Why should I slink
away without having carried out my commission, and without
the payment which was my due? This woman might, for all
I knew, be a monomaniac. With a stout bearing, therefore,
though her manner had shaken me more than I cared to confess,
I still shook my head, and declared my intention of remaining
where I was. She was about to renew her entreaties,
when a door slammed overhead, and the sound of several
footsteps were heard upon the stairs. She listened for an instant,
threw up her hands with a despairing gesture, and vanished
as suddenly and as noiselessly as she had come.

“The new-comers were Colonel Lysander Stark and a short,
thick man with a chinchilla beard growing out of the creases
of his double chin, who was introduced to me as Mr.
Ferguson.

“\,‘This is my secretary and manager,’ said the colonel.
‘By-the-way, I was under the impression that I left this door
shut just now. I fear that you have felt the draught.’

“\,‘On the contrary,’ said I, ‘I opened the door myself, because
I felt the room to be a little close.’

“He shot one of his suspicious looks at me. ‘Perhaps we
had better proceed to business, then,’ said he. ‘Mr.~Ferguson
and I will take you up to see the machine.’

“\,‘I had better put my hat on, I suppose.’

“\,‘Oh no, it is in the house.’

“\,‘What, you dig fuller’s-earth in the house?’

“\,‘No, no. This is only where we compress it. But never
mind that. All we wish you to do is to examine the
machine, and to let us know what is wrong with it.’

“We went up-stairs together, the colonel first with the
lamp, the fat manager, and I behind him. It was a labyrinth
of an old house, with corridors, passages, narrow winding
staircases, and little low doors, the thresholds of which were
hollowed out by the generations who had crossed them.
%%249
There were no carpets and no signs of any furniture above
the ground floor, while the plaster was peeling off the walls,
and the damp was breaking through in green, unhealthy
blotches. I tried to put on as unconcerned an air as possible,
but I had not forgotten the warnings of the lady, even
though I disregarded them, and I kept a keen eye upon my
two companions. Ferguson appeared to be a morose and
silent man, but I could see from the little that he said that
he was at least a fellow-countryman.

“Colonel Lysander Stark stopped at last before a low door,
which he unlocked. Within was a small, square room, in
which the three of us could hardly get at one time. Ferguson
remained outside, and the colonel ushered me in.

“\,‘We are now,’ said he, ‘actually within the hydraulic
press, and it would be a particularly unpleasant thing for us
if any one were to turn it on. The ceiling of this small chamber
is really the end of the descending piston, and it comes
down with the force of many tons upon this metal floor.
There are small lateral columns of water outside which receive
the force, and which transmit and multiply it in the
manner which is familiar to you. The machine goes readily
enough, but there is some stiffness in the working of it, and
it has lost a little of its force. Perhaps you will have the
goodness to look it over and to show us how we can set it
right.’

“I took the lamp from him, and I examined the machine
very thoroughly. It was indeed a gigantic one, and capable
of exercising enormous pressure. When I passed outside,
however, and pressed down the levers which controlled it, I
knew at once by the whishing sound that there was a slight
leakage, which allowed a regurgitation of water through one
of the side cylinders. An examination showed that one of
the india-rubber bands which was round the head of a driving-rod
had shrunk so as not quite to fill the socket along which
it worked. This was clearly the cause of the loss of power,
and I pointed it out to my companions, who followed my
%%250
remarks very carefully, and asked several practical questions
as to how they should proceed to set it right. When I had
made it clear to them, I returned to the main chamber of the
machine and took a good look at it to satisfy my own curiosity.
It was obvious at a glance that the story of the fuller’s-earth
was the merest fabrication, for it would be absurd to suppose
that so powerful an engine could be designed for so inadequate
a purpose. The walls were of wood, but the floor
consisted of a large iron trough, and when I came to examine
it I could see a crust of metallic deposit all over it. I had
stooped and was scraping at this to see exactly what it was,
when I heard a muttered exclamation in German, and saw the
cadaverous face of the colonel looking down at me.

“\,‘What are you doing there?’ he asked.

“I felt angry at having been tricked by so elaborate a story
as that which he had told me. ‘I was admiring your fuller’s-%
earth,’ said I; I think that I should be better able to advise
you as to your machine if I knew what the exact purpose was
for which it was used.’

“The instant that I uttered the words I regretted the rashness
of my speech. His face set hard, and a baleful light
sprang up in his gray eyes.

“\,‘Very well,’ said he, ‘you shall know all about the machine.’
He took a step backward, slammed the little door,
and turned the key in the lock. I rushed towards it and
pulled at the handle, but it was quite secure, and did not give
in the least to my kicks and shoves. ‘Hello!’ I yelled.
‘Hello! Colonel! Let me out!’

“And then suddenly in the silence I heard a sound which
sent my heart into my mouth. It was the clank of the levers
and the swish of the leaking cylinder. He had set the engine
at work. The lamp still stood upon the floor where I had
placed it when examining the trough. By its light I saw that
the black ceiling was coming down upon me, slowly, jerkily,
but, as none knew better than myself, with a force which must
within a minute grind me to a shapeless pulp. I threw
%%251
myself, screaming, against the door, and dragged with my nails
at the lock. I implored the colonel to let me out, but the
remorseless clanking of the levers drowned my cries. The
ceiling was only a foot or two above my head, and with my
hand upraised I could feel its hard, rough surface. Then it
flashed through my mind that the pain of my death would
depend very much upon the position in which I met it. If I
lay on my face the weight would come upon my spine, and I
shuddered to think of that dreadful snap. Easier the other
way, perhaps; and yet, had I the nerve to lie and look up at
that deadly black shadow wavering down upon me? Already
I was unable to stand erect, when my eye caught something
which brought a gush of hope back to my heart.

“I have said that though the floor and ceiling were of iron,
the walls were of wood. As I gave a last hurried glance
around, I saw a thin line of yellow light between two of the
boards, which broadened and broadened as a small panel was
pushed backward. For an instant I could hardly believe that
here was indeed a door which led away from death. The
next instant I threw myself through, and lay half-fainting upon
the other side. The panel had closed again behind me, but
the crash of the lamp, and a few moments afterwards the clang
of the two slabs of metal, told me how narrow had been my
escape.

“I was recalled to myself by a frantic plucking at my wrist,
and I found myself lying upon the stone floor of a narrow corridor,
while a woman bent over me and tugged at me with her
left hand, while she held a candle in her right. It was the
same good friend whose warning I had so foolishly rejected.

“\,‘Come! come!’ she cried, breathlessly. ‘They will be
here in a moment. They will see that you are not there. Oh,
do not waste the so-precious time, but come!’

“This time, at least, I did not scorn her advice. I staggered
to my feet and ran with her along the corridor and
down a winding stair. The latter led to another broad passage,
and, just as we reached it, we heard the sound of running
%%252
feet and the shouting of two voices, one answering the other,
from the floor on which we were and from the one beneath.
My guide stopped and looked about her like one who is at
her wits’ end. Then she threw open a door which led into a
bedroom, through the window of which the moon was shining
brightly.

“\,‘It is your only chance,’ said she. ‘It is high, but it may
be that you can jump it.’

“As she spoke a light sprang into view at the further end
of the passage, and I saw the lean figure of Colonel Lysander
Stark rushing forward with a lantern in one hand and a
weapon like a butcher’s cleaver in the other. I rushed across
the bedroom, flung open the window, and looked out. How
quiet and sweet and wholesome the garden looked in the
moonlight, and it could not be more than thirty feet down. I
clambered out upon the sill, but I hesitated to jump until I
should have heard what passed between my savior and the
ruffian who pursued me. If she were ill-used, then at any
risks I was determined to go back to her assistance. The
thought had hardly flashed through my mind before he was
at the door, pushing his way past her; but she threw her arms
round him and tried to hold him back.

“\,‘Fritz! Fritz!’ she cried, in English, ‘remember your
prom\-ise after the last time. You said it should not be again.
He will be silent! Oh, he will be silent!’

“\,‘You are mad, Elise!’ he shouted, struggling to break
away from her. ‘You will be the ruin of us. He has seen
too much. Let me pass, I say!’ He dashed her to one side,
and, rushing to the window, cut at me with his heavy weapon.
I had let myself go, and was hanging by the hands to the sill,
when his blow fell. I was conscious of a dull pain, my grip
loosened, and I fell into the garden below.

“I was shaken but not hurt by the fall; so I picked myself
up and rushed off among the bushes as hard as I could run,
for I understood that I was far from being out of danger yet.
Suddenly, however, as I ran, a deadly dizziness and sickness
%%253
came over me. I glanced down at my hand, which was throbbing
painfully, and then, for the first time, saw that my thumb
had been cut off and that the blood was pouring from my
wound. I endeavored to tie my handkerchief round it, but
there came a sudden buzzing in my ears, and next moment I
fell in a dead faint among the rose-bushes.

“How long I remained unconscious I cannot tell. It must
have been a very long time, for the moon had sunk, and a
bright morning was breaking when I came to myself. My
clothes were all sodden with dew, and my coat-sleeve was
drenched with blood from my wounded thumb. The smarting
of it recalled in an instant all the particulars of my night’s
adventure, and I sprang to my feet with the feeling that I
might hardly yet be safe from my pursuers. But, to my astonishment,
when I came to look round me, neither house nor
garden were to be seen. I had been lying in an angle of the
hedge close by the high-road, and just a little lower down was
a long building, which proved, upon my approaching it, to be
the very station at which I had arrived upon the previous
night. Were it not for the ugly wound upon my hand, all
that had passed during those dreadful hours might have been
an evil dream.

“Half dazed, I went into the station and asked about the
morning train. There would be one to Reading in less than
an hour. The same porter was on duty, I found, as had been
there when I arrived. I inquired of him whether he had ever
heard of Colonel Lysander Stark. The name was strange to
him. Had he observed a carriage the night before waiting
for me? No, he had not. Was there a police-station anywhere
near? There was one about three miles off.

“It was too far for me to go, weak and ill as I was. I determined
to wait until I got back to town before telling my
story to the police. It was a little past six when I arrived, so
I went first to have my wound dressed, and then the doctor
was kind enough to bring me along here. I put the case into
your hands, and shall do exactly what you advise.”
%%254

We both sat in silence for some little time after, listening to
this extraordinary narrative. Then Sherlock Holmes pulled
down from the shelf one of the ponderous commonplace
books in which he placed his cuttings.

“Here is an advertisement which will interest you,” said
he. “It appeared in all the papers about a year ago. Listen
to this: ‘Lost, on the 9th inst., Mr.~Jeremiah Hayling, aged
twenty-six, an hydraulic engineer. Left his lodgings at ten
o’clock at night, and has not been heard of since. Was
dressed in,’ etc., etc. Ha! That represents the last time
that the colonel needed to have his machine overhauled, I
fancy.”

“Good heavens!” cried my patient. “Then that explains
what the girl said.”

“Undoubtedly. It is quite clear that the colonel was a
cool and desperate man, who was absolutely determined
that nothing should stand in the way of his little game, like
those out-and-out pirates who will leave no survivor from a
captured ship. Well, every moment now is precious, so if
you feel equal to it, we shall go down to Scotland Yard at
once as a preliminary to starting for Eyford.”

Some three hours or so afterwards we were all in the train
together, bound from Reading to the little Berkshire village.
There were Sherlock Holmes, the hydraulic engineer, Inspector
Bradstreet, of Scotland Yard, a plain-clothes man, and myself.
Bradstreet had spread an ordnance map of the county
out upon the seat, and was busy with his compasses drawing
a circle with Eyford for its centre.

“There you are,” said he. “That circle is drawn at a radius
of ten miles from the village. The place we want must
be somewhere near that line. You said ten miles, I think, sir.”

“It was an hour’s good drive.”

“And you think that they brought you back all that way
when you were unconscious?”

“They must have done so. I have a confused memory,
too, of having been lifted and conveyed somewhere.”
%%255

“What I cannot understand,” said I, “is why they should
have spared you when they found you lying fainting in the
garden. Perhaps the villain was softened by the woman’s
entreaties.”

“I hardly think that likely. I never saw a more inexorable
face in my life.”

“Oh, we shall soon clear up all that,” said Bradstreet.
“Well, I have drawn my circle, and I only wish I knew at
what point upon it the folk that we are in search of are to be
found.”

“I think I could lay my finger on it,” said Holmes, quietly.

“Really, now!” cried the inspector, “you have formed your
opinion! Come, now, we shall see who agrees with you. I
say it is south, for the country is more deserted there.”

“And I say east,” said my patient.

“I am for west,” remarked the plain-clothes man. “There
are several quiet little villages up there.”

“And I am for north,” said I, “because there are no hills
there, and our friend says that he did not notice the carriage
go up any.”

“Come,” cried the inspector, laughing; “it’s a very pretty
diversity of opinion. We have boxed the compass among us.
Who do you give your casting vote to?”

“You are all wrong.”

“But we can’t \textit{all} be.”

“Oh yes, you can. This is my point;” he placed his finger
in the centre of the circle. “This is where we shall find
them.”

“But the twelve-mile drive?” gasped Hatherley.

“Six out and six back. Nothing simpler. You say yourself
that the horse was fresh and glossy when you got in.
How could it be that if it had gone twelve miles over heavy
roads?”

“Indeed, it is a likely ruse enough,” observed Bradstreet,
thoughtfully. “Of course there can be no doubt as to the
nature of this gang.”
%%256

“None at all,” said Holmes. “They are coiners on a large
scale, and have used the machine to form the amalgam which
has taken the place of silver.”

“We have known for some time that a clever gang was at
work,” said the inspector. “They have been turning out
half-crowns by the thousand. We even traced them as far as
Reading, but could get no farther, for they had covered their
traces in a way that showed that they were very old hands.
But now, thanks to this lucky chance, I think that we have
got them right enough.”

But the inspector was mistaken, for those criminals were
not destined to fall into the hands of justice. As we rolled
into Eyford Station we saw a gigantic column of smoke which
streamed up from behind a small clump of trees in the neighborhood,
and hung like an immense ostrich feather over the
landscape.

“A house on fire?” asked Bradstreet, as the train steamed
off again on its way.

“Yes, sir!” said the station-master.

“When did it break out?”

“I hear that it was during the night, sir, but it has got
worse, and the whole place is in a blaze.”

“Whose house is it?”

“Dr. Becher’s.”

“Tell me,” broke in the engineer, “is Dr. Becher a German,
very thin, with a long, sharp nose?”

The station-master laughed heartily. “No, sir, Dr. Becher
is an Englishman, and there isn’t a man in the parish who has
a better-lined waistcoat. But he has a gentleman staying with
him, a patient, as I understand, who is a foreigner, and he
looks as if a little good Berkshire beef would do him no
harm.”

The station-master had not finished his speech before we
were all hastening in the direction of the fire. The road
topped a low hill, and there was a great wide-spread whitewashed
building in front of us, spouting fire at every chink
%%257
and window, while in the garden in front three fire-engines
were vainly striving to keep the flames under.

“That’s it!” cried Hatherley, in intense excitement. “There
is the gravel-drive, and there are the rose-bushes where I lay.
That second window is the one that I jumped from.”

“Well, at least,” said Holmes, “you have had your revenge
upon them. There can be no question that it was your oil-%
lamp which, when it was crushed in the press, set fire to the
wooden walls, though no doubt they were too excited in the
chase after you to observe it at the time. Now keep your
eyes open in this crowd for your friends of last night, though
I very much fear that they are a good hundred miles off by
now.”

And Holmes’s fears came to be realized, for from that day
to this no word has ever been heard either of the beautiful
woman, the sinister German, or the morose Englishman. Early
that morning a peasant had met a cart containing several
people and some very bulky boxes driving rapidly in the direction
of Reading, but there all traces of the fugitives disappeared,
and even Holmes’s ingenuity failed ever to discover
the least clew as to their whereabouts.

The firemen had been much perturbed at the strange arrangements
which they had found within, and still more so
by discovering a newly severed human thumb upon a window-%
sill of the second floor. About sunset, however, their efforts
were at last successful, and they subdued the flames, but not
before the roof had fallen in, and the whole place been reduced
to such absolute ruin that, save some twisted cylinders and
iron piping, not a trace remained of the machinery which had
cost our unfortunate acquaintance so dearly. Large masses
of nickel and of tin were discovered stored in an out-house,
but no coins were to be found, which may have explained the
presence of those bulky boxes which have been already referred
to.

How our hydraulic engineer had been conveyed from the
garden to the spot where he recovered his senses might have
%%258
remained forever a mystery were it not for the soft mould,
which told us a very plain tale. He had evidently been carried
down by two persons, one of whom had remarkably small
feet and the other unusually large ones. On the whole, it was
most probable that the silent Englishman, being less bold or
less murderous than his companion, had assisted the woman
to bear the unconscious man out of the way of danger.

“Well,” said our engineer ruefully, as we took our seats to
return once more to London, “it has been a pretty business
for me! I have lost my thumb and I have lost a fifty-guinea
fee, and what have I gained?”

“Experience,” said Holmes, laughing. “Indirectly it may
be of value, you know; you have only to put it into words to
gain the reputation of being excellent company for the remainder
of your existence.”
%%259

\Chapter{The Adventure Of The Noble Bachelor}

\textsc{The} Lord St.~Simon marriage, and its curious termination,
have long ceased to be a subject of interest
in those exalted circles in which the unfortunate
bridegroom moves. Fresh scandals have
eclipsed it, and their more piquant details have drawn the
gossips away from this four-year-old drama. As I have reason
to believe, however, that the full facts have never been
revealed to the general public, and as my friend Sherlock
Holmes had a considerable share in clearing the matter up,
I feel that no memoir of him would be complete without some
little sketch of this remarkable episode.

It was a few weeks before my own marriage, during the
days when I was still sharing rooms with Holmes in Baker
Street, that he came home from an afternoon stroll to find a
letter on the table waiting for him. I had remained in-doors
all day, for the weather had taken a sudden turn to rain, with
high autumnal winds, and the jezail bullet which I had brought
back in one of my limbs as a relic of my Afghan campaign,
throbbed with dull persistency. With my body in one easy-%
chair and my legs upon another, I had surrounded myself
with a cloud of newspapers, until at last, saturated with the
news of the day, I tossed them all aside and lay listless,
watching the huge crest and monogram upon the envelope
upon the table, and wondering lazily who my friend’s noble
correspondent could be.

“Here is a very fashionable epistle,” I remarked, as he
entered. “Your morning letters, if I remember right, were
from a fish-monger and a tide-waiter.”
%%260

“Yes, my correspondence has certainly the charm of variety,”
he answered, smiling, “and the humbler are usually
the more interesting. This looks like one of those unwelcome
social summonses which call upon a man either to be
bored or to lie.”

He broke the seal and glanced over the contents.

“Oh, come, it may prove to be something of interest after
all.”

“Not social, then?”

“No, distinctly professional.”

“And from a noble client?”

“One of the highest in England.”

“My dear fellow, I congratulate you.”

“I assure you, Watson, without affectation, that the status
of my client is a matter of less moment to me than the interest
of his case. It is just possible, however, that that also
may not be wanting in this new investigation. You have
been reading the papers diligently of late, have you not?”

“It looks like it,” said I, ruefully, pointing to a huge bundle
in the corner. “I have had nothing else to do.”

“It is fortunate, for you will perhaps be able to post me up.
I read nothing except the criminal news and the agony column.
The latter is always instructive. But if you have followed
recent events so closely you must have read about Lord St.
Simon and his wedding?”

“Oh yes, with the deepest interest.”

“That is well. The letter which I hold in my hand is
from Lord St.~Simon. I will read it to you, and in return
you must turn over these papers and let me have whatever
bears upon the matter. This is what he says:

\begin{letter}
“\,‘\textsc{My dear Mr.~Sherlock Holmes}, -- Lord Backwater
tells me that I may place implicit reliance upon your judgment
and discretion. I have determined, therefore, to call
upon you, and to consult you in reference to the very painful
event which has occurred in connection with my wedding.
%%261
Mr.~Lestrade, of Scotland Yard, is acting already in the matter,
but he assures me that he sees no objection to your co-%
operation, and that he even thinks that it might be of some
assistance. I will call at four o’clock in the afternoon, and,
should you have any other engagement at that time, I hope
that you will postpone it, as this matter is of paramount
importance. Yours faithfully, \textsc{St.~Simon}.’
\end{letter}

“It is dated from Grosvenor Mansions, written with a quill
pen, and the noble lord has had the misfortune to get a smear
of ink upon the outer side of his right little finger,” remarked
Holmes, as he folded up the epistle.

“He says four o’clock. It is three now. He will be here
in an hour.”

“Then I have just time, with your assistance, to get clear
upon the subject. Turn over those papers, and arrange the
extracts in their order of time, while I take a glance as to
who our client is.” He picked a red-covered volume from
a line of books of reference beside the mantel-piece. “Here
he is,” said he, sitting down and flattening it out upon his
knee. “Lord Robert Walsingham de Vere St.~Simon, second
son of the Duke of Balmoral -- Hum! Arms: Azure, three
caltrops in chief over a fess sable. Born in 1846. He’s forty-%
one years of age, which is mature for marriage. Was Undersecretary
for the Colonies in a late Administration. The
Duke, his father, was at one time Secretary for Foreign Affairs.
They inherit Plantagenet blood by direct descent, and
Tudor on the distaff side. Ha! Well, there is nothing very
instructive in all this. I think that I must turn to you, Watson,
for something more solid.”

“I have very little difficulty in finding what I want,” said
I, “for the facts are quite recent, and the matter struck me
as remarkable. I feared to refer them to you, however, as I
knew that you had an inquiry on hand, and that you disliked
the intrusion of other matters.”

“Oh, you mean the little problem of the Grosve\-nor Square
%%262
furniture van. That is quite cleared up now -- though, indeed,
it was obvious from the first. Pray give me the results
of your newspaper selections.”

“Here is the first notice which I can find. It is in the
personal column of \textit{The Morning Post}, and dates, as you see,
some weeks back. ‘A marriage has been arranged,’ it says,
‘and will, if rumor is correct, very shortly take place, between
Lord Robert St.~Simon, second son of the Duke of Balmoral,
and Miss Hatty Doran, the only daughter of Aloysius Doran,
Esq., of San Francisco, Cal., U.S.A.’ That is all.”

“Terse and to the point,” remarked Holmes, stretching
his long, thin legs towards the fire.

“There was a paragraph amplifying this in one of the society
papers of the same week. Ah, here it is. ‘There will
soon be a call for protection in the marriage market, for the
present free-trade principle appears to tell heavily against
our home product. One by one the management of the noble
houses of Great Britain is passing into the hands of our fair
cousins from across the Atlantic. An important addition has
been made during the last week to the list of the prizes which
have been borne away by these charming invaders. Lord
St.~Simon, who has shown himself for over twenty years proof
against the little god’s arrows, has now definitely announced
his approaching marriage with Miss Hatty Doran, the fascinating
daughter of a California millionaire. Miss Doran,
whose graceful figure and striking face attracted much attention
at the Westbury House festivities, is an only child, and
it is currently reported that her dowry will run to considerably
over the six figures, with expectancies for the future. As it
is an open secret that the Duke of Balmoral has been compelled
to sell his pictures within the last few years, and as
Lord St.~Simon has no property of his own, save the small
estate of Birchmoor, it is obvious that the Californian heiress
is not the only gainer by an alliance which will enable her to
make the easy and common transition from a Republican
lady to a British peeress.’\,”
%%263

“Anything else?” asked Holmes, yawning.

“Oh yes; plenty. Then there is another note in \textit{The Morning
Post} to say that the marriage would be an absolutely quiet
one, that it would be at St.~George’s, Hanover Square, that
only half a dozen intimate friends would be invited, and that
the party would return to the furnished house at Lancaster
Gate which has been taken by Mr.~Aloysius Doran. Two
days later -- that is, on Wednesday last -- there is a curt
announcement that the wedding had taken place, and that the
honey-moon would be passed at Lord Backwater’s place, near
Petersfield. Those are all the notices which appeared before
the disappearance of the bride.”

“Before the what?” asked Holmes, with a start.

“The vanishing of the lady.”

“When did she vanish, then?”

“At the wedding breakfast.”

“Indeed. This is more interesting than it promi\-sed to be;
quite dramatic, in fact.”

“Yes; it struck me as being a little out of the common.”

“They often vanish before the ceremony, and occasionally
during the honey-moon; but I cannot call to mind anything
quite so prompt as this. Pray let me have the details.”

“I warn you that they are very incomplete.”

“Perhaps we may make them less so.”

“Such as they are, they are set forth in a single article of
a morning paper of yesterday, which I will read to you. It
is headed, ‘Singular Occurrence at a Fashionable
Wedding’:

“\,‘The family of Lord Robert St.~Simon has been thrown
into the greatest consternation by the strange and painful
episodes which have taken place in connection with his wedding.
The ceremony, as shortly announced in the papers of
yesterday, occurred on the previous morning; but it is only
now that it has been possible to confirm the strange rumors
which have been so persistently floating about. In spite of
the attempts of the friends to hush the matter up, so much
%%264
public attention has now been drawn to it that no good purpose
can be served by affecting to disregard what is a common
subject for conversation.

“\,‘The ceremony, which was performed at St.\ George’s,
Hanover Square, was a very quiet one, no one being present
save the father of the bride, Mr.~Aloysius Doran, the Duchess
of Balmoral, Lord Backwater, Lord Eustace, and Lady Clara
St.~Simon (the younger brother and sister of the bridegroom),
and Lady Alicia Whittington. The whole party proceeded
afterwards to the house of Mr.~Aloysius Doran, at Lancaster
Gate, where breakfast had been prepared. It appears that
some little trouble was caused by a woman, whose name has
not been ascertained, who endeavored to force her way into
the house after the bridal party, alleging that she had some
claim upon Lord St.~Simon. It was only after a painful and
prolonged scene that she was ejected by the butler and the
footman. The bride, who had fortunately entered the house
before this unpleasant interruption, had sat down to breakfast
with the rest, when she complained of a sudden indisposition,
and retired to her room. Her prolonged absence having
caused some comment, her father followed her, but learned
from her maid that she had only come up to her chamber
for an instant, caught up an ulster and bonnet, and hurried
down to the passage. One of the footmen declared that he
had seen a lady leave the house thus apparelled, but had
refused to credit that it was his mistress, believing her to
be with the company. On ascertaining that his daughter
had disappeared, Mr.~Aloysius Doran, in conjunction with the
bridegroom, instantly put themselves into communication
with the police, and very energetic inquiries are being made,
which will probably result in a speedy clearing up of this very
singular business. Up to a late hour last night, however,
nothing had transpired as to the whereabouts of the missing
lady. There are rumors of foul play in the matter, and it is
said that the police have caused the arrest of the woman who
had caused the original disturbance, in the belief that, from
%%265
jealousy or some other motive, she may have been concerned
in the strange disappearance of the bride.’\,”

“And is that all?”

“Only one little item in another of the morning papers, but
it is a suggestive one.”

“And it is -- ”

“That Miss Flora Millar, the lady who had caused the disturbance,
has actually been arrested. It appears that she
was formerly a \textit{danseuse} at the ‘Allegro,’ and that she has
known the bridegroom for some years. There are no further
particulars, and the whole case is in your hands now -- so far
as it has been set forth in the public press.”

“And an exceedingly interesting case it appears to be. I
would not have missed it for worlds. But there is a ring at
the bell, Watson, and as the clock makes it a few minutes
after four, I have no doubt that this will prove to be our noble
client. Do not dream of going, Watson, for I very much prefer
having a witness, if only as a check to my own memory.”

“Lord Robert St.~Simon,” announced our page-boy, throwing
open the door. A gentleman entered, with a pleasant,
cultured face, high-nosed and pale, with something perhaps
of petulance about the mouth, and with the steady, well-opened
eye of a man whose pleasant lot it had ever been to command
and to be obeyed. His manner was brisk, and yet his
general appearance gave an undue impression of age, for he
had a slight forward stoop and a little bend of the knees as
he walked. His hair, too, as he swept off his very curly-brimmed
hat, was grizzled round the edges and thin upon the top.
As to his dress, it was careful to the verge of foppishness,
with high collar, black frock-coat, white waistcoat, yellow
gloves, patent-leather shoes, and light-colored gaiters. He
advanced slowly into the room, turning his head from left to
right, and swinging in his right hand the cord which held his
golden eye-glasses.

“Good-day, Lord St.~Simon,” said Holmes, rising and bowing.
“Pray take the basket-chair. This is my friend and
%%266
colleague, Dr. Watson. Draw up a little to the fire, and we
will talk this matter over.”

“A most painful matter to me, as you can most readily
imagine, Mr.~Holmes. I have been cut to the quick. I understand
that you have already managed several delicate cases
of this sort, sir, though I presume that they were hardly from
the same class of society.”

“No, I am descending.”

“I beg pardon.”

“My last client of the sort was a king.”

“Oh, really! I had no idea. And which king?”

“The King of Scandinavia.”

“What! Had he lost his wife?”

“You can understand,” said Holmes, suavely, “that I extend
to the affairs of my other clients the same secrecy which
I promise to you in yours.”

“Of course! Very right! very right! I’m sure I beg pardon.
As to my own case, I am ready to give you any information
which may assist you in forming an opinion.”

“Thank you. I have already learned all that is in the
public prints, nothing more. I presume that I may take it as
correct -- this article, for example, as to the disappearance of
the bride.”

Lord St.~Simon glanced over it. “Yes, it is correct, as far
as it goes.”

“But it needs a great deal of supplementing before any one
could offer an opinion. I think that I may arrive at my
facts most directly by questioning you.”

“Pray do so.”

“When did you first meet Miss Hatty Doran?”

“In San Francisco, a year ago.”

“You were travelling in the States?”

“Yes.”

“Did you become engaged then?”

“No.”

“But you were on a friendly footing?”
%%267

“I was amused by her society, and she could see that I
was amused.”

“Her father is very rich?”

“He is said to be the richest man on the Pacific slope.”

“And how did he make his money?”

“In mining. He had nothing a few years ago. Then he
struck gold, invested it, and came up by leaps and bounds.”

“Now, what is your own impression as to the young lady’s -- your
wife’s character?”

The nobleman swung his glasses a little faster and stared
down into the fire. “You see, Mr.~Holmes,” said he, “my
wife was twenty before her father became a rich man. During
that time she ran free in a mining camp, and wandered
through woods or mountains, so that her education has
come from Nature rather than from the school-master. She
is what we call in England a tomboy, with a strong nature,
wild and free, unfettered by any sort of traditions. She
is impetuous -- volcanic, I was about to say. She is swift in
making up her mind, and fearless in carrying out her resolutions.
On the other hand, I would not have given her the
name which I have the honor to bear” -- he gave a little
stately cough -- “had not I thought her to be at bottom a
noble woman. I believe that she is capable of heroic self-%
sacrifice, and that anything dishonorable would be repugnant
to her.”

“Have you her photograph?”

“I brought this with me.” He opened a locket, and showed
us the full face of a very lovely woman. It was not a
photograph, but an ivory miniature, and the artist had brought
out the full effect of the lustrous black hair, the large dark
eyes, and the exquisite mouth. Holmes gazed long and earnestly
at it. Then he closed the locket and handed it back
to Lord St.~Simon.

“The young lady came to London, then, and you renewed
your acquaintance?”

“Yes, her father brought her over for this last London
%%268
season. I met her several times, became engaged to her, and
have now married her.”

“She brought, I understand, a considerable dowry?”

“A fair dowry. Not more than is usual in my family.”

“And this, of course, remains to you, since the marriage is
a \textit{fait accompli?}”

“I really have made no inquiries on the subject.”

“Very naturally not. Did you see Miss Doran on the day
before the wedding?”

“Yes.”

“Was she in good spirits?”

“Never better. She kept talking of what we should do in
our future lives.”

“Indeed! That is very interesting. And on the morning
of the wedding?”

“She was as bright as possible -- at least, until after the
ceremony.”

“And did you observe any change in her then?”

“Well, to tell the truth, I saw then the first signs that I
had ever seen that her temper was just a little sharp. The
incident, however, was too trivial to relate, and can have no
possible bearing upon the case.”

“Pray let us have it, for all that.”

“Oh, it is childish. She dropped her bouquet as we went
towards the vestry. She was passing the front pew at the
time, and it fell over into the pew. There was a moment’s
delay, but the gentleman in the pew handed it up to her
again, and it did not appear to be the worse for the fall.
Yet, when I spoke to her of the matter, she answered me
abruptly; and in the carriage, on our way home, she seemed
absurdly agitated over this trifling cause.”

“Indeed! You say that there was a gentleman in the pew.
Some of the general public were present, then?”

“Oh yes. It is impossible to exclude them when the church
is open.”

“This gentleman was not one of your wife’s friends?”
%%269

“No, no; I call him a gentleman by courtesy, but he was
quite a common-looking person. I hardly noticed his appearance.
But really I think that we are wandering rather far
from the point.”

“Lady St.~Simon, then, returned from the wedding in a
less cheerful frame of mind than she had gone to it. What
did she do on re-entering her father’s house?”

“I saw her in conversation with her maid.”

“And who is her maid?”

“Alice is her name. She is an American, and came from
California with her.”

“A confidential servant?”

“A little too much so. It seemed to me that her mistress
allowed her to take great liberties. Still, of course, in America
they look upon these things in a different way.”

“How long did she speak to this Alice?”

“Oh, a few minutes. I had something else to think of.”

“You did not overhear what they said?”

“Lady St.~Simon said something about ‘jumping a claim.’
She was accustomed to use slang of the kind. I have no idea
what she meant.”

“American slang is very expressive sometimes. And what
did your wife do when she finished speaking to her maid?”

“She walked into the breakfast-room.”

“On your arm?”

“No, alone. She was very independent in little matters
like that. Then, after we had sat down for ten minutes or
so, she rose hurriedly, muttered some words of apology, and
left the room. She never came back.”

“But this maid, Alice, as I understand, deposes that she
went to her room, covered her bride’s dress with a long ulster,
put on a bonnet, and went out.”

“Quite so. And she was afterwards seen walking into
Hyde Park in company with Flora Millar, a woman who is
now in custody, and who had already made a disturbance at
Mr.~Doran’s house that morning.”
%%270

“Ah, yes. I should like a few particulars as to this young
lady, and your relations to her.”

Lord St.~Simon shrugged his shoulders and raised his eyebrows.
“We have been on a friendly footing for some years -- I
may say on a \textit{very} friendly footing. She used to be at
the ‘Allegro.’ I have not treated her ungenerously, and she
has no just cause of complaint against me; but you know
what women are, Mr.~Holmes. Flora was a dear little thing,
but exceedingly hot-headed, and devotedly attached to me.
She wrote me dreadful letters when she heard that I was
about to be married; and, to tell the truth, the reason why I
had the marriage celebrated so quietly was that I feared lest
there might be a scandal in the church. She came to Mr.
Doran’s door just after we returned, and she endeavored to
push her way in, uttering very abusive expressions towards
my wife, and even threatening her, but I had foreseen the
possibility of something of the sort, and I had two police fellows
there in private clothes, who soon pushed her out again.
She was quiet when she saw that there was no good in making
a row.”

“Did your wife hear all this?”

“No, thank goodness, she did not.”

“And she was seen walking with this very woman
afterwards?”

“Yes. That is what Mr.~Lestrade, of Scotland Yard, looks
upon as so serious. It is thought that Flora decoyed my
wife out, and laid some terrible trap for her.”

“Well, it is a possible supposition.”

“You think so, too?”

“I did not say a probable one. But you do not yourself
look upon this as likely?”

“I do not think Flora would hurt a fly.”

“Still, jealousy is a strange transformer of characters. Pray
what is your own theory as to what took place?”

“Well, really, I came to seek a theory, not to propound one.
I have given you all the facts. Since you ask me, however,
%%271
I may say that it has occurred to me as possible that the
excitement of this affair, the consciousness that she had made
so immense a social stride, had the effect of causing some little
nervous disturbance in my wife.”

“In short, that she had become suddenly deranged?”

“Well, really, when I consider that she has turned her
back -- I will not say upon me, but upon so much that many
have aspired to without success -- I can hardly explain it in
any other fashion.”

“Well, certainly that is also a conceivable hypothesis,” said
Holmes, smiling. “And now, Lord St.~Simon, I think that I
have nearly all my data. May I ask whether you were seated
at the breakfast-table so that you could see out of the
window?”

“We could see the other side of the road and the Park.”

“Quite so. Then I do not think that I need to detain you
longer. I shall communicate with you.”

“Should you be fortunate enough to solve this problem,”
said our client, rising.

“I have solved it.”

“Eh? What was that?”

“I say that I have solved it.”

“Where, then, is my wife?”

“That is a detail which I shall speedily supply.”

Lord St.~Simon shook his head. “I am afraid that it will
take wiser heads than yours or mine,” he remarked, and bowing
in a stately, old-fashioned manner, he departed.

“It is very good of Lord St.~Simon to honor my head by
putting it on a level with his own,” said Sherlock Holmes,
laughing. “I think that I shall have a whiskey-and-soda and
a cigar after all this cross-questioning. I had formed my
conclusions as to the case before our client came into the room.”

“My dear Holmes!”

“I have notes of several similar cases, though none, as I
remarked before, which were quite as prompt. My whole examination
served to turn my conjecture into a certainty.
%%272
Circumstantial evidence is occasionally very convincing, as when
you find a trout in the milk, to quote Thoreau’s example.”

“But I have heard all that you have heard.”

“Without, however, the knowledge of pre-exist\-ing cases
which serves me so well. There was a parallel instance in
Aberdeen some years back, and something on very much the
same lines at Munich the year after the Franco-Prussian war.
It is one of these cases -- but, hello, here is Lestrade! Good-%
afternoon, Lestrade! You will find an extra tumbler upon
the sideboard, and there are cigars in the box.”

The official detective was attired in a pea-jacket and cravat,
which gave him a decidedly nautical appearance, and he
carried a black canvas bag in his hand. With a short greeting
he seated himself and lit the cigar which had been offered
to him.

“What’s up, then?” asked Holmes, with a twinkle in his
eye. “You look dissatisfied.”

“And I feel dissatisfied. It is this infernal St.~Simon marriage
case. I can make neither head nor tail of the
business.”

“Really! You surprise me.”

“Who ever heard of such a mixed affair? Every clew
seems to slip through my fingers. I have been at work upon
it all day.”

“And very wet it seems to have made you,” said Holmes,
laying his hand upon the arm of the pea-jacket.

“Yes, I have been dragging the Serpentine.”

“In Heaven’s name, what for?”

“In search of the body of Lady St.~Simon.”

Sherlock Holmes leaned back in his chair and laughed
heartily.

“Have you dragged the basin of Trafalgar Sq\-uare fountain?”
he asked.

“Why? What do you mean?”

“Because you have just as good a chance of finding this
lady in the one as in the other.”
%%273

Lestrade shot an angry glance at my companion. “I suppose
you know all about it,” he snarled.

“Well, I have only just heard the facts, but my mind is
made up.”

“Oh, indeed! Then you think that the Serpentine plays
no part in the matter?”

“I think it very unlikely.”

“Then perhaps you will kindly explain how it is that we
found this in it?” He opened his bag as he spoke, and tumbled
onto the floor a wedding-dress of watered silk, a pair of
white satin shoes, and a bride’s wreath and veil, all discolored
and soaked in water. “There,” said he, putting a new wedding-ring
upon the top of the pile. “There is a little nut for
you to crack, Master Holmes.”

“Oh, indeed!” said my friend, blowing blue rings into the
air. “You dragged them from the Serpentine?”

“No. They were found floating near the margin by a park-%
keeper. They have been identified as her clothes, and it
seemed to me that if the clothes were there the body would
not be far off.”

“By the same brilliant reasoning, every man’s body is to be
found in the neighborhood of his wardrobe. And pray what
did you hope to arrive at through this?”

“At some evidence implicating Flora Millar in the
disappearance.”

“I am afraid that you will find it difficult.”

“Are you, indeed, now?” cried Lestrade, with some bitterness.
“I am afraid, Holmes, that you are not very practical
with your deductions and your inferences. You have made
two blunders in as many minutes. This dress does implicate
Miss Flora Millar.”

“And how?”

“In the dress is a pocket. In the pocket is a card-case.
In the card-case is a note. And here is the very note.” He
slapped it down upon the table in front of him. “Listen to
this: ‘You will see me when all is ready. Come at once.
%%274
F. H. M.’ Now my theory all along has been that Lady St.
Simon was decoyed away by Flora Millar, and that she, with
confederates, no doubt, was responsible for her disappearance.
Here, signed with her initials, is the very note which was no
doubt quietly slipped into her hand at the door, and which
lured her within their reach.”

“Very good, Lestrade,” said Holmes, laughing. “You
really are very fine indeed. Let me see it.” He took up the
paper in a listless way, but his attention instantly became
riveted, and he gave a little cry of satisfaction. “This is indeed
important,” said he.

“Ha! you find it so?”

“Extremely so. I congratulate you warmly.”

Lestrade rose in his triumph and bent his head to look.
“Why,” he shrieked, “you’re looking at the wrong side!”

“On the contrary, this is the right side.”

“The right side? You’re mad! Here is the note written
in pencil over here.”

“And over here is what appears to be the fragment of a
hotel bill, which interests me deeply.”

“There’s nothing in it. I looked at it before,” said Lestrade.
“\,‘Oct. 4th, rooms 8\textit{s.}, breakfast 2\textit{s.} 6\textit{d.}, cocktail 1\textit{s.},
lunch 2\textit{s.} 6\textit{d.}, glass sherry, 8\textit{d.}’ I see nothing in that.”

“Very likely not. It is most important, all the same. As
to the note, it is important also, or at least the initials are, so
I congratulate you again.”

“I’ve wasted time enough,” said Lestrade, rising. “I believe
in hard work, and not in sitting by the fire spinning fine
theories. Good-day, Mr.~Holmes, and we shall see which
gets to the bottom of the matter first.” He gathered up the
garments, thrust them into the bag, and made for the door.

“Just one hint to you, Lestrade,” drawled Holmes, before
his rival vanished; “I will tell you the true solution of the
matter. Lady St.~Simon is a myth. There is not, and there
never has been, any such person.”

Lestrade looked sadly at my companion. Then he turned
%%275
to me, tapped his forehead three times, shook his head solemnly,
and hurried away.

He had hardly shut the door behind him when Holmes rose
and put on his overcoat. “There is something in what the
fellow says about out-door work,” he remarked, “so I think,
Watson, that I must leave you to your papers for a little.”

It was after five o’clock when Sherlock Holmes left me, but
I had no time to be lonely, for within an hour there arrived
a confectioner’s man with a very large flat box. This he unpacked
with the help of a youth whom he had brought with
him, and presently, to my very great astonishment, a quite
epicurean little cold supper began to be laid out upon our
humble lodging-house mahogany. There were a couple of
brace of cold woodcock, a pheasant, a \textit{pâté de foie gras} pie,
with a group of ancient and cobwebby bottles. Having laid
out all these luxuries, my two visitors vanished away, like the
genii of the Arabian Nights, with no explanation save that
the things had been paid for and were ordered to this address.

Just before nine o’clock Sherlock Holmes stepp\-ed briskly
into the room. His features were gravely set, but there was a
light in his eye which made me think that he had not been
disappointed in his conclusions.

“They have laid the supper, then,” he said, rubbing his
hands.

“You seem to expect company. They have laid for five.”

“Yes, I fancy we may have some company dropping in,”
said he. “I am surprised that Lord St.~Simon has not already
arrived. Ha! I fancy that I hear his step now upon
the stairs.”

It was indeed our visitor of the morning who came bustling
in, dangling his glasses more vigorously than ever, and with a
very perturbed expression upon his aristocratic features.

“My messenger reached you, then?” asked Holmes.

“Yes, and I confess that the contents startled me beyond
measure. Have you good authority for what you say?”

“The best possible.”
%%276

Lord St.~Simon sank into a chair and passed his hand over
his forehead.

“What will the duke say,” he murmured, “when he hears
that one of the family has been subjected to such
humiliation?”

“It is the purest accident. I cannot allow that there is any
humiliation.”

“Ah, you look on these things from another stand-point.”

“I fail to see that any one is to blame. I can hardly see
how the lady could have acted otherwise, though her abrupt
method of doing it was undoubtedly to be regretted. Having
no mother, she had no one to advise her at such a crisis.”

“It was a slight, sir, a public slight,” said Lord St.~Simon,
tapping his fingers upon the table.

“You must make allowance for this poor girl, placed in so
unprecedented a position.”

“I will make no allowance. I am very angry indeed, and I
have been shamefully used.”

“I think that I heard a ring,” said Holmes. “Yes, there
are steps on the landing. If I cannot persuade you to take a
lenient view of the matter, Lord St.~Simon, I have brought an
advocate here who may be more successful.” He opened the
door and ushered in a lady and gentleman. “Lord St.~Simon,”
said he, “allow me to introduce you to Mr.~and Mrs.
Francis Hay Moulton. The lady, I think, you have already
met.”

At the sight of these new-comers our client had sprung
from his seat and stood very erect, with his eyes cast down
and his hand thrust into the breast of his frock-coat, a picture
of offended dignity. The lady had taken a quick step forward
and had held out her hand to him, but he still refused to
raise his eyes. It was as well for his resolution, perhaps, for
her pleading face was one which it was hard to resist.

“You’re angry, Robert,” said she. “Well, I guess you
have every cause to be.”

“Pray make no apology to me,” said Lord St.~Simon, bitterly.
%%277

“Oh yes, I know that I have treated you real bad, and that
I should have spoken to you before I went; but I was kind of
rattled, and from the time when I saw Frank here again I
just didn’t know what I was doing or saying. I only wonder
I didn’t fall down and do a faint right there before the
altar.”

“Perhaps, Mrs.~Moulton, you would like my friend and me
to leave the room while you explain this matter?”

“If I may give an opinion,” remarked the strange gentleman,
“we’ve had just a little too much secrecy over this business
already. For my part, I should like all Europe and
America to hear the rights of it.” He was a small, wiry,
sunburnt man, clean shaven, with a sharp face and alert
manner.

“Then I’ll tell our story right away,” said the lady. “Frank
here and I met in ’84, in McQuire’s camp, near the Rockies,
where pa was working a claim. We were engaged to each
other, Frank and I; but then one day father struck a rich
pocket and made a pile, while poor Frank here had a claim
that petered out and came to nothing. The richer pa grew,
the poorer was Frank; so at last pa wouldn’t hear of our engagement
lasting any longer, and he took me away to ’Frisco.
Frank wouldn’t throw up his hand, though; so he followed
me there, and he saw me without pa knowing anything about
it. It would only have made him mad to know, so we just
fixed it all up for ourselves. Frank said that he would go
and make his pile, too, and never come back to claim me
until he had as much as pa. So then I promised to
wait for him to the end of time, and pledged myself not to
marry any one else while he lived. ‘Why shouldn’t we be
married right away, then,’ said he, ‘and then I will feel sure
of you; and I won’t claim to be your husband until I come
back?’ Well, we talked it over, and he had fixed it all up so
nicely, with a clergyman all ready in waiting, that we just did
it right there; and then Frank went off to seek his fortune,
and I went back to pa.
%%278

“The next I heard of Frank was that he was in Montana,
and then he went prospecting in Arizona, and then I heard of
him from New Mexico. After that came a long newspaper
story about how a miners’ camp had been attacked by Apache
Indians, and there was my Frank’s name among the killed.
I fainted dead away, and I was very sick for months after.
Pa thought I had a decline, and took me to half the doctors
in ’Frisco. Not a word of news came for a year and more, so
that I never doubted that Frank was really dead. Then
Lord St.~Simon came to ’Frisco, and we came to London, and
a marriage was arranged, and pa was very pleased, but I felt
all the time that no man on this earth would ever take the
place in my heart that had been given to my poor Frank.

“Still, if I had married Lord St.~Simon, of course I’d have
done my duty by him. We can’t command our love, but we
can our actions. I went to the altar with him with the intention
to make him just as good a wife as it was in me to be.
But you may imagine what I felt when, just as I came to the
altar rails, I glanced back and saw Frank standing and looking
at me out of the first pew. I thought it was his ghost at
first; but when I looked again, there he was still, with a kind
of question in his eyes as if to ask me whether I were glad or
sorry to see him. I wonder I didn’t drop. I know that everything
was turning round, and the words of the clergyman
were just like the buzz of a bee in my ear. I didn’t know
what to do. Should I stop the service and make a scene in
the church? I glanced at him again, and he seemed to know
what I was thinking, for he raised his finger to his lips to tell
me to be still. Then I saw him scribble on a piece of paper,
and I knew that he was writing me a note. As I passed his
pew on the way out I dropped my bouquet over to him, and
he slipped the note into my hand when he returned me the
flowers. It was only a line asking me to join him when he
made the sign to me to do so. Of course I never doubted
for a moment that my first duty was now to him, and I determined
to do just whatever he might direct.
%%279

“When I got back I told my maid, who had known him in
California, and had always been his friend. I ordered her to
say nothing, but to get a few things packed and my ulster
ready. I know I ought to have spoken to Lord St.~Simon, but
it was dreadful hard before his mother and all those great people.
I just made up my mind to run away and explain afterwards.
I hadn’t been at the table ten minutes before I saw
Frank out of the window at the other side of the road. He
beckoned to me, and then began walking into the Park. I
slipped out, put on my things, and followed him. Some woman
came talking something or other about Lord St.~Simon to me -- seemed
to me from the little I heard as if he had a little
secret of his own before marriage also -- but I managed to get
away from her, and soon overtook Frank. We got into a cab
together, and away we drove to some lodgings he had taken
in Gordon Square, and that was my true wedding after all
those years of waiting. Frank had been a prisoner among
the Apaches, had escaped, came on to ’Frisco, found that I
had given him up for dead and had gone to England, followed
me there, and had come upon me at last on the very morning
of my second wedding.”

“I saw it in a paper,” explained the American. “It gave
the name and the church, but not where the lady lived.”

“Then we had a talk as to what we should do, and Frank
was all for openness, but I was so ashamed of it all that I felt
as if I should like to vanish away and never see any of them
again -- just sending a line to pa, perhaps, to show him that I
was alive. It was awful to me to think of all those lords and
ladies sitting round that breakfast-table and waiting for me to
come back. So Frank took my wedding-clothes and things
and made a bundle of them, so that I should not be traced,
and dropped them away somewhere where no one could find
them. It is likely that we should have gone on to Paris to-%
morrow, only that this good gentleman, Mr.~Holmes, came
round to us this evening, though how he found us is more
than I can think, and he showed us very clearly and kindly
%%280
that I was wrong and that Frank was right, and that we
should be putting ourselves in the wrong if we were so secret.
Then he offered to give us a chance of talking to Lord St.
Simon alone, and so we came right away round to his rooms
at once. Now, Robert, you have heard it all, and I am very
sorry if I have given you pain, and I hope that you do not
think very meanly of me.”

Lord St.~Simon had by no means relaxed his rigid attitude,
but had listened with a frowning brow and a compressed lip
to this long narrative.

“Excuse me,” he said, “but it is not my custom to discuss
my most intimate personal affairs in this public manner.”

“Then you won’t forgive me? You won’t shake hands before
I go?”

“Oh, certainly, if it would give you any pleasure.” He put
out his hand and coldly grasped that which she extended to
him.

“I had hoped,” suggested Holmes, “that you would have
joined us in a friendly supper.”

“I think that there you ask a little too much,” responded
his lordship. “I may be forced to acquiesce in these recent
developments, but I can hardly be expected to make merry
over them. I think that, with your permission, I will now wish
you all a very good-night.” He included us all in a sweeping
bow and stalked out of the room.

“Then I trust that you at least will honor me with your
company,” said Sherlock Holmes. “It is always a joy to
meet an American, Mr.~Moulton, for I am one of those who
believe that the folly of a monarch and the blundering of a
minister in far-gone years will not prevent our children from
being some day citizens of the same world-wide country under
a flag which shall be a quartering of the Union Jack with the
Stars and Stripes.”

\strut

“The case has been an interesting one,” remarked Holmes,
when our visitors had left us, “because it serves to show very
%%281
%%“\,‘I WILL WISH YOU ALL A VERY GOOD NIGHT’\,”
%%282
clearly how simple the explanation may be of an affair which
at first sight seems to be almost inexplicable. Nothing could
be more natural than the sequence of events as narrated by
this lady, and nothing stranger than the result when viewed,
for instance, by Mr.~Lestrade, of Scotland Yard.”

“You were not yourself at fault at all, then?”

“From the first, two facts were very obvious to me, the one
that the lady had been quite willing to undergo the wedding
ceremony, the other that she had repented of it within a few
minutes of returning home. Obviously something had occurred
during the morning, then, to cause her to change her
mind. What could that something be? She could not have
spoken to any one when she was out, for she had been in the
company of the bridegroom. Had she seen some one, then?
If she had, it must be some one from America, because she
had spent so short a time in this country that she could hardly
have allowed any one to acquire so deep an influence over her
that the mere sight of him would induce her to change her
plans so completely. You see we have already arrived, by a
process of exclusion, at the idea that she might have seen an
American. Then who could this American be, and why should
he possess so much influence over her? It might be a lover;
it might be a husband. Her young womanhood had, I knew,
been spent in rough scenes and under strange conditions. So
far I had got before I ever heard Lord St.~Simon’s narrative.
When he told us of a man in a pew, of the change in the
bride’s manner, of so transparent a device for obtaining a note
as the dropping of a bouquet, of her resort to her confidential
maid, and of her very significant allusion to claim-jumping -- which
in miners’ parlance means taking possession of that
which another person has a prior claim to -- the whole situation
became absolutely clear. She had gone off with a man, and
the man was either a lover or was a previous husband -- the
chances being in favor of the latter.”

“And how in the world did you find them?”

“It might have been difficult, but friend Lestrade held
%%284
information in his hands the value of which he did not himself
know. The initials were of course of the highest importance,
but more valuable still was it to know that within a week he
had settled his bill at one of the most select London hotels.”

“How did you deduce the select?”

“By the select prices. Eight shillings for a bed and
eightpence for a glass of sherry pointed to one of the most
expensive hotels. There are not many in London which charge at
that rate. In the second one which I visited in Northumberland
Avenue, I learned by an inspection of the book that Francis
H. Moulton, an American gentleman, had left only the day
before, and on looking over the entries against him, I came
upon the very items which I had seen in the duplicate bill.
His letters were to be forwarded to 226 Gordon Square; so
thither I travelled, and being fortunate enough to find the
loving couple at home, I ventured to give them some paternal
advice, and to point out to them that it would be better in
every way that they should make their position a little clearer
both to the general public and to Lord St.~Simon in particular.
I invited them to meet him here, and, as you see, I
made him keep the appointment.”

“But with no very good result,” I remarked. “His conduct
was certainly not very gracious.”

“Ah, Watson,” said Holmes, smiling, “perhaps you would
not be very gracious either, if, after all the trouble of wooing
and wedding, you found yourself deprived in an instant of
wife and of fortune. I think that we may judge Lord St.~Simon
very mercifully, and thank our stars that we are never likely
to find ourselves in the same position. Draw your chair up,
and hand me my violin, for the only problem we have still to
solve is how to while away these bleak autumnal evenings.”
%%285

\Chapter{The Adventure Of The Beryl Coronet}

“\textsc{Holmes},” said I, as I stood one morning in our
bow-window looking down the street, “here is a
madman coming along. It seems rather sad that
his relatives should allow him to come out alone.”

My friend rose lazily from his arm-chair and stood with
his hands in the pockets of his dressing-gown, looking over
my shoulder. It was a bright, crisp February morning, and
the snow of the day before still lay deep upon the ground,
shimmering brightly in the wintry sun. Down the centre of
Baker Street it had been ploughed into a brown crumbly band
by the traffic, but at either side and on the heaped-up edges of
the foot-paths it still lay as white as when it fell. The gray
pavement had been cleaned and scraped, but was still dangerously
slippery, so that there were fewer passengers than usual.
Indeed, from the direction of the Metropolitan Station no one
was coming save the single gentleman whose eccentric conduct
had drawn my attention.

He was a man of about fifty, tall, portly, and imposing,
with a massive, strongly marked face and a commanding
figure. He was dressed in a sombre yet rich style, in black
frock-coat, shining hat, neat brown gaiters, and well-cut pearl-%
gray trousers. Yet his actions were in absurd contrast to the
dignity of his dress and features, for he was running hard, with
occasional little springs, such as a weary man gives who is
little accustomed to set any tax upon his legs. As he ran he
jerked his hands up and down, waggled his head, and writhed
his face into the most extraordinary contortions.
%%286

“What on earth can be the matter with him?” I asked.
“He is looking up at the numbers of the houses.”

“I believe that he is coming here,” said Holmes, rubbing
his hands.

“Here?”

“Yes; I rather think he is coming to consult me professionally.
I think that I recognize the symptoms. Ha! did I
not tell you?” As he spoke, the man, puffing and blowing,
rushed at our door and pulled at our bell until the whole
house resounded with the clanging.

A few moments later he was in our room, still puffing, still
gesticulating, but with so fixed a look of grief and despair in
his eyes that our smiles were turned in an instant to horror
and pity. For a while he could not get his words out, but
swayed his body and plucked at his hair like one who has
been driven to the extreme limits of his reason. Then, suddenly
springing to his feet, he beat his head against the wall
with such force that we both rushed upon him and tore him
away to the centre of the room. Sherlock Holmes pushed
him down into the easy-chair, and, sitting beside him, patted
his hand, and chatted with him in the easy, soothing tones
which he knew so well how to employ.

“You have come to me to tell your story, have you not?”
said he. “You are fatigued with your haste. Pray wait until
you have recovered yourself, and then I shall be most happy
to look into any little problem which you may submit to me.”

The man sat for a minute or more with a heaving chest,
fighting against his emotion. Then he passed his handkerchief
over his brow, set his lips tight, and turned his face
towards us.

“No doubt you think me mad?” said he.

“I see that you have had some great trouble,” responded
Holmes.

“God knows I have! -- a trouble which is enough to unseat
my reason, so sudden and so terrible is it. Public disgrace I
might have faced, although I am a man whose character has
%%287
never yet borne a stain. Private affliction also is the lot of
every man; but the two coming together, and in so frightful a
form, have been enough to shake my very soul. Besides, it
is not I alone. The very noblest in the land may suffer, unless
some way be found out of this horrible affair.”

“Pray compose yourself, sir,” said Holmes, “and let me
have a clear account of who you are, and what it is that has
befallen you.”

“My name,” answered our visitor, “is probably familiar to
your ears. I am Alexander Holder, of the banking firm of
Holder \& Stevenson, of Threadneedle Street.”

The name was indeed well known to us as belonging to the
senior partner in the second largest private banking concern
in the City of London. What could have happened, then, to
bring one of the foremost citizens of London to this most
pitiable pass? We waited, all curiosity, until with another
effort he braced himself to tell his story.

“I feel that time is of value,” said he; “that is why I hastened
here when the police inspector suggested that I should secure
your co-operation. I came to Baker Street by the Underground,
and hurried from there on foot, for the cabs go slowly
through this snow. That is why I was so out of breath, for
I am a man who takes very little exercise. I feel better now,
and I will put the facts before you as shortly and yet as clearly
as I can.

“It is, of course, well known to you that in a successful
banking business as much depends upon our being able to
find remunerative investments for our funds as upon our increasing
our connection and the number of our depositors.
One of our most lucrative means of laying out money is in
the shape of loans, where the security is unimpeachable. We
have done a good deal in this direction during the last few
years, and there are many noble families to whom we have
advanced large sums upon the security of their pictures, libraries,
or plate.

“Yesterday morning I was seated in my office at the bank
%%288
when a card was brought in to me by one of the clerks. I
started when I saw the name, for it was that of none other
than -- well, perhaps even to you I had better say no more
than that it was a name which is a household word all over
the earth -- one of the highest, noblest, most exalted names in
England. I was overwhelmed by the honor, and attempted,
when he entered, to say so, but he plunged at once into business
with the air of a man who wishes to hurry quickly through
a disagreeable task.

“\,‘Mr.~Holder,’ said he, ‘I have been informed that you are
in the habit of advancing money.’

“\,‘The firm do so when the security is good,’ I answered.

“\,‘It is absolutely essential to me,’ said he, ‘that I should
have £50,000 at once. I could of course borrow so trifling a
sum ten times over from my friends, but I much prefer to make
it a matter of business, and to carry out that business myself.
In my position you can readily understand that it is unwise to
place one’s self under obligations.’

“\,‘For how long, may I ask, do you want this sum?’ I asked.

“\,‘Next Monday I have a large sum due to me, and I shall
then most certainly repay what you advance, with whatever
interest you think it right to charge. But it is very essential
to me that the money should be paid at once.’

“\,‘I should be happy to advance it without further parley
from my own private purse,’ said I, ‘were it not that the
strain would be rather more than it could bear. If, on the
other hand, I am to do it in the name of the firm, then in
justice to my partner I must insist that, even in your case,
every business-like precaution should be taken.’

“\,‘I should much prefer to have it so,’ said he, raising up a
square, black morocco case which he had laid beside his chair.
‘You have doubtless heard of the Beryl Coronet?’

“\,‘One of the most precious public possessions of the empire,’
said I.

“\,‘Precisely.’ He opened the case, and there, imbedded
in soft, flesh-colored velvet, lay the magnificent piece of
%%289
jewelry which he had named. ‘There are thirty-nine enormous
beryls,’ said he, ‘and the price of the gold chasing is
incalculable. The lowest estimate would put the worth of the
coronet at double the sum which I have asked. I am prepared
to leave it with you as my security.’

“I took the precious case into my hands and looked in
some perplexity from it to my illustrious client.

“\,‘You doubt its value?’ he asked.

“\,‘Not at all. I only doubt -- ’

“\,‘The propriety of my leaving it. You may set your mind
at rest about that. I should not dream of doing so were
it not absolutely certain that I should be able in four days
to reclaim it. It is a pure matter of form. Is the security
sufficient?’

“\,‘Ample.’

“\,‘You understand, Mr.~Holder, that I am giving you a
strong proof of the confidence which I have in you, founded
upon all that I have heard of you. I rely upon you not
only to be discreet and to refrain from all gossip upon the
matter, but, above all, to preserve this coronet with every
possible precaution, because I need not say that a great
public scandal would be caused if any harm were to befall it.
Any injury to it would be almost as serious as its complete
loss, for there are no beryls in the world to match these, and
it would be impossible to replace them. I leave it with you,
however, with every confidence, and I shall call for it in person
on Monday morning.’

“Seeing that my client was anxious to leave, I said no
more; but, calling for my cashier, I ordered him to pay over
fifty £1000 notes. When I was alone once more, however,
with the precious case lying upon the table in front of me,
I could not but think with some misgivings of the immense
responsibility which it entailed upon me. There could be
no doubt that, as it was a national possession, a horrible
scandal would ensue if any misfortune should occur to it. I
already regretted having ever consented to take charge of it.
%%290
However, it was too late to alter the matter now, so I locked
it up in my private safe, and turned once more to my work.

“When evening came I felt that it would be an imprudence
to leave so precious a thing in the office behind me. Bankers’
safes had been forced before now, and why should not mine
be? If so, how terrible would be the position in which I
should find myself! I determined, therefore, that for the
next few days I would always carry the case backward and
forward with me, so that it might never be really out of my
reach. With this intention, I called a cab, and drove out to
my house at Streatham, carrying the jewel with me. I did
not breathe freely until I had taken it up-stairs and locked it
in the bureau of my dressing-room.

“And now a word as to my household, Mr.~Holmes, for I
wish you to thoroughly understand the situation. My groom
and my page sleep out of the house, and may be set aside
altogether. I have three maid-servants who have been with
me a number of years, and whose absolute reliability is quite
above suspicion. Another, Lucy Parr, the second waiting-maid,
has only been in my service a few months. She came
with an excellent character, however, and has always given me
satisfaction. She is a very pretty girl, and has attracted admirers
who have occasionally hung about the place. That is
the only drawback which we have found to her, but we believe
her to be a thoroughly good girl in every way.

“So much for the servants. My family itself is so small
that it will not take me long to describe it. I am a widower,
and have an only son, Arthur. He has been a disappointment
to me, Mr.~Holmes -- a grievous disappointment. I have
no doubt that I am myself to blame. People tell me that I
have spoiled him. Very likely I have. When my dear wife
died I felt that he was all I had to love. I could not bear to
see the smile fade even for a moment from his face. I have
never denied him a wish. Perhaps it would have been better
for both of us had I been sterner, but I meant it for the best.

“It was naturally my intention that he should succeed me
%%291
in my business, but he was not of a business turn. He was
wild, wayward, and, to speak the truth, I could not trust him
in the handling of large sums of money. When he was young
he became a member of an aristocratic club, and there, having
charming manners, he was soon the intimate of a number of
men with long purses and expensive habits. He learned to
play heavily at cards and to squander money on the turf,
until he had again and again to come to me and implore me
to give him an advance upon his allowance, that he might
settle his debts of honor. He tried more than once to break
away from the dangerous company which he was keeping, but
each time the influence of his friend Sir George Burnwell was
enough to draw him back again.

“And, indeed, I could not wonder that such a man as Sir
George Burnwell should gain an influence over him, for he
has frequently brought him to my house, and I have found myself
that I could hardly resist the fascination of his manner.
He is older than Arthur, a man of the world to his finger-tips,
one who had been everywhere, seen everything, a brilliant
talker, and a man of great personal beauty. Yet when I think
of him in cold blood, far away from the glamour of his presence,
I am convinced from his cynical speech, and the look
which I have caught in his eyes, that he is one who should
be deeply distrusted. So I think, and so, too, thinks my
little Mary, who has a woman’s quick insight into character.

“And now there is only she to be described. She is my
niece; but when my brother died five years ago and left her
alone in the world I adopted her, and have looked upon her
ever since as my daughter. She is a sunbeam in my house -- sweet,
loving, beautiful, a wonderful manager and
housekeeper, yet as tender and quiet and gentle as a woman could
be. She is my right hand. I do not know what I could do
without her. In only one matter has she ever gone against
my wishes. Twice my boy has asked her to marry him, for
he loves her devotedly, but each time she has refused him. I
think that if any one could have drawn him into the right path
%%292
it would have been she, and that his marriage might have
changed his whole life; but now, alas! it is too late -- for ever
too late!

“Now, Mr.~Holmes, you know the people who live under
my roof, and I shall continue with my miserable story.

“When we were taking coffee in the drawing-room that
night, after dinner, I told Arthur and Mary my experience,
and of the precious treasure which we had under our roof,
suppressing only the name of my client. Lucy Parr, who had
brought in the coffee, had, I am sure, left the room; but I
cannot swear that the door was closed. Mary and Arthur
were much interested, and wished to see the famous coronet,
but I thought it better not to disturb it.

“\,‘Where have you put it?’ asked Arthur.

“\,‘In my own bureau.’

“\,‘Well, I hope to goodness the house won’t be burgled
during the night,’ said he.

“\,‘It is locked up,’ I answered.

“\,‘Oh, any old key will fit that bureau. When I was a
youngster I have opened it myself with the key of the box-%
room cupboard.’

“He often had a wild way of talking, so that I thought little
of what he said. He followed me to my room, however, that
night with a very grave face.

“\,‘Look here, dad,’ said he, with his eyes cast down, ‘can
you let me have £200?’

“\,‘No, I cannot!’ I answered, sharply. ‘I have been far
too generous with you in money matters.’

“\,‘You have been very kind,’ said he: ‘but I must have
this money, or else I can never show my face inside the club
again.’

“\,‘And a very good thing, too!’ I cried.

“\,‘Yes, but you would not have me leave it a dishonored
man,’ said he. ‘I could not bear the disgrace. I must raise
the money in some way, and if you will not let me have it,
then I must try other means.’
%%293

“I was very angry, for this was the third demand during
the month. ‘You shall not have a farthing from me,’ I cried;
on which he bowed and left the room without another word.

“When he was gone I unlocked my bureau, made sure that
my treasure was safe, and locked it again. Then I started to
go round the house to see that all was secure -- a duty which
I usually leave to Mary, but which I thought it well to perform
myself that night. As I came down the stairs I saw
Mary herself at the side window of the hall, which she closed
and fastened as I approached.

“\,‘Tell me, dad,’ said she, looking, I thought, a little disturbed,
‘did you give Lucy, the maid, leave to go out to-night?’

“\,‘Certainly not.’

“\,‘She came in just now by the back door. I have no
doubt that she has only been to the side gate to see some one;
but I think that it is hardly safe, and should be stopped.”

“\,‘You must speak to her in the morning, or I will, if you
prefer it. Are you sure that everything is fastened?’

“\,‘Quite sure, dad.’

“\,‘Then, good-night.’ I kissed her, and went up to my bedroom
again, where I was soon asleep.

“I am endeavoring to tell you everything, Mr.~Holmes,
which may have any bearing upon the case, but I beg that you
will question me upon any point which I do not make clear.”

“On the contrary, your statement is singularly lucid.”

“I come to a part of my story now in which I should wish
to be particularly so. I am not a very heavy sleeper, and the
anxiety in my mind tended, no doubt, to make me even less
so than usual. About two in the morning, then, I was awakened
by some sound in the house. It had ceased ere I was
wide awake, but it had left an impression behind it as though
a window had gently closed somewhere. I lay listening with
all my ears. Suddenly, to my horror, there was a distinct
sound of footsteps moving softly in the next room. I slipped
out of bed, all palpitating with fear, and peeped round the
corner of my dressing-room door.
%%294

“\,‘Arthur!’ I screamed, ‘you villain! you thief! How
dare you touch that coronet?’

“The gas was half up, as I had left it, and my unhappy
boy, dressed only in his shirt and trousers, was standing beside
the light, holding the coronet in his hands. He appeared
to be wrenching at it, or bending it with all his strength. At
my cry he dropped it from his grasp, and turned as pale as
death. I snatched it up and examined it. One of the gold
corners, with three of the beryls in it, was missing.

“\,‘You blackguard!’ I shouted, beside myself with rage.
‘You have destroyed it! You have dishonored me for ever!
Where are the jewels which you have stolen?’

“\,‘Stolen!’ he cried.

“\,‘Yes, you thief!’ I roared, shaking him by the shoulder.

“\,‘There are none missing. There cannot be any missing,’
said he.

“\,‘There are three missing. And you know where they are.
Must I call you a liar as well as a thief? Did I not see you
trying to tear off another piece?’

“\,‘You have called me names enough,’ said he; ‘I will not
stand it any longer. I shall not say another word about this
business since you have chosen to insult me. I will leave
your house in the morning and make my own way in the
world.’

“\,‘You shall leave it in the hands of the police!’ I cried,
half-mad with grief and rage. ‘I shall have this matter
probed to the bottom.’

“\,‘You shall learn nothing from me,’ said he, with a passion
such as I should not have thought was in his nature.
‘If you choose to call the police, let the police find what
they can.’

“By this time the whole house was astir, for I had raised
my voice in my anger. Mary was the first to rush into my
room, and, at the sight of the coronet and of Arthur’s face,
she read the whole story, and, with a scream, fell down senseless
on the ground. I sent the house-maid for the police, and
%%295
put the investigation into their hands at once. When the
inspector and a constable entered the house, Arthur, who had
stood sullenly with his arms folded, asked me whether it was
my intention to charge him with theft. I answered that it
had ceased to be a private matter, but had become a public
one, since the ruined coronet was national property. I was
determined that the law should have its way in everything.

“\,‘At least,’ said he, ‘you will not have me arrested at once.
It would be to your advantage as well as mine if I might leave
the house for five minutes.’

“\,‘That you may get away, or perhaps that you may conceal
what you have stolen,’ said I. And then realizing the
dreadful position in which I was placed, I implored him to
remember that not only my honor, but that of one who was
far greater than I was at stake; and that he threatened to
raise a scandal which would convulse the nation. He might
avert it all if he would but tell me what he had done with the
three missing stones.

“\,‘You may as well face the matter,’ said I; ‘you have
been caught in the act, and no confession could make your
guilt more heinous. If you but make such reparation as is in
your power, by telling us where the beryls are, all shall be
forgiven and forgotten.’

“\,‘Keep your forgiveness for those who ask for it,’ he answered,
turning away from me, with a sneer. I saw that he
was too hardened for any words of mine to influence him.
There was but one way for it. I called in the inspector, and
gave him into custody. A search was made at once, not only
of his person, but of his room, and of every portion of the
house where he could possibly have concealed the gems; but
no trace of them could be found, nor would the wretched boy
open his mouth for all our persuasions and our threats. This
morning he was removed to a cell, and I, after going through
all the police formalities, have hurried round to you, to implore
you to use your skill in unravelling the matter. The
police have openly confessed that they can at present make
%%296
nothing of it. You may go to any expense which you think
necessary. I have already offered a reward of £1000. My
God, what shall I do! I have lost my honor, my gems, and
my son in one night. Oh, what shall I do!”

He put a hand on either side of his head, and rocked himself
to and fro, droning to himself like a child whose grief has
got beyond words.

Sherlock Holmes sat silent for some few minutes, with his
brows knitted and his eyes fixed upon the fire.

“Do you receive much company?” he asked.

“None, save my partner with his family, and an occasional
friend of Arthur’s. Sir George Burnwell has been several
times lately. No one else, I think.”

“Do you go out much in society?”

“Arthur does. Mary and I stay at home. We neither of
us care for it.”

“That is unusual in a young girl.”

“She is of a quiet nature. Besides, she is not so very
young. She is four-and-twenty.”

“This matter, from what you say, seems to have been a
shock to her also.”

“Terrible! She is even more affected than I.”

“You have neither of you any doubt as to your son’s
guilt?”

“How can we have, when I saw him with my own eyes
with the coronet in his hands.”

“I hardly consider that a conclusive proof. Was the remainder
of the coronet at all injured?”

“Yes, it was twisted.”

“Do you not think, then, that he might have been trying to
straighten it?”

“God bless you! You are doing what you can for him and
for me. But it is too heavy a task. What was he doing there
at all? If his purpose were innocent, why did he not say so?”

“Precisely. And if it were guilty, why did he not invent a
lie? His silence appears to me to cut both ways. There are
%%297
several singular points about the case. What did the police
think of the noise which awoke you from your sleep?”

“They considered that it might be caused by Arthur’s closing
his bedroom door.”

“A likely story! As if a man bent on felony would slam
his door so as to wake a household. What did they say, then,
of the disappearance of these gems?”

“They are still sounding the planking and probing the
furniture in the hope of finding them.”

“Have they thought of looking outside the house?”

“Yes, they have shown extraordinary energy. The whole
garden has already been minutely examined.”

“Now, my dear sir,” said Holmes, “is it not obvious to
you now that this matter really strikes very much deeper than
either you or the police were at first inclined to think? It
appeared to you to be a simple case; to me it seems exceedingly
complex. Consider what is involved by your theory.
You suppose that your son came down from his bed, went, at
great risk, to your dressing-room, opened your bureau, took
out your coronet, broke off by main force a small portion of
it, went off to some other place, concealed three gems out of
the thirty-nine, with such skill that nobody can find them,
and then returned with the other thirty-six into the room in
which he exposed himself to the greatest danger of being discovered.
I ask you now, is such a theory tenable?”

“But what other is there?” cried the banker, with a gesture
of despair. “If his motives were innocent, why does he not
explain them?”

“It is our task to find that out,” replied Holmes; “so now,
if you please, Mr.~Holder, we will set off for Streatham together,
and devote an hour to glancing a little more closely
into details.”

My friend insisted upon my accompanying them in their
expedition, which I was eager enough to do, for my curiosity
and sympathy were deeply stirred by the story to which we
had listened. I confess that the guilt of the banker’s son
%%298
appeared to me to be as obvious as it did to his unhappy
father, but still I had such faith in Holmes’s judgment that I
felt that there must be some grounds for hope as long as he
was dissatisfied with the accepted explanation. He hardly
spoke a word the whole way out to the southern suburb, but
sat with his chin upon his breast and his hat drawn over his
eyes, sunk in the deepest thought. Our client appeared to
have taken fresh heart at the little glimpse of hope which had
been presented to him, and he even broke into a desultory
chat with me over his business affairs. A short railway journey
and a shorter walk brought us to Fairbank, the modest
residence of the great financier.

Fairbank was a good-sized square house of white stone,
standing back a little from the road. A double carriage-sweep,
with a snow-clad lawn, stretched down in front to two large
iron gates which closed the entrance. On the right side was
a small wooden thicket, which led into a narrow path between
two neat hedges stretching from the road to the kitchen door,
and forming the tradesmen’s entrance. On the left ran a lane
which led to the stables, and was not itself within the grounds
at all, being a public, though little used, thoroughfare. Holmes
left us standing at the door, and walked slowly all round the
house, across the front, down the tradesmen’s path, and so
round by the garden behind into the stable lane. So long was
he that Mr.~Holder and I went into the dining-room and
waited by the fire until he should return. We were sitting
there in silence when the door opened and a young lady came
in. She was rather above the middle height, slim, with dark
hair and eyes, which seemed the darker against the absolute
pallor of her skin. I do not think that I have ever seen such
deadly paleness in a woman’s face. Her lips, too, were bloodless,
but her eyes were flushed with crying. As she swept
silently into the room she impressed me with a greater sense
of grief than the banker had done in the morning, and it was
the more striking in her as she was evidently a woman of
strong character, with immense capacity for self-restraint.
%%299
Disregarding my presence, she went straight to her uncle, and
passed her hand over his head with a sweet womanly caress.

“You have given orders that Arthur should be liberated,
have you not, dad?” she asked.

“No, no, my girl, the matter must be probed to the bottom.”

“But I am so sure that he is innocent. You know what
women’s instincts are. I know that he has done no harm
and that you will be sorry for having acted so harshly.”

“Why is he silent, then, if he is innocent?”

“Who knows? Perhaps because he was so angry that you
should suspect him.”

“How could I help suspecting him, when I actually saw
him with the coronet in his hand?”

“Oh, but he had only picked it up to look at it. Oh do,
do take my word for it that he is innocent. Let the matter
drop and say no more. It is so dreadful to think of our dear
Arthur in prison!”

“I shall never let it drop until the gems are found -- never,
Mary! Your affection for Arthur blinds you as to the awful
consequences to me. Far from hushing the thing up, I have
brought a gentleman down from London to inquire more
deeply into it.”

“This gentleman?” she asked, facing round to me.

“No, his friend. He wished us to leave him alone. He is
round in the stable lane now.”

“The stable lane?” She raised her dark eyebrows. “What
can he hope to find there? Ah! this, I suppose, is he. I
trust, sir, that you will succeed in proving, what I feel sure is
the truth, that my cousin Arthur is innocent of this crime.”

“I fully share your opinion, and I trust, with you, that we
may prove it,” returned Holmes, going back to the mat to
knock the snow from his shoes. “I believe I have the honor
of addressing Miss Mary Holder. Might I ask you a question
or two?”

“Pray do, sir, if it may help to clear this horrible affair up.”

“You heard nothing yourself last night?”
%%300

“Nothing, until my uncle here began to speak loudly. I
heard that, and I came down.”

“You shut up the windows and doors the night before.
Did you fasten all the windows?”

“Yes.”

“Were they all fastened this morning?”

“Yes.”

“You have a maid who has a sweetheart? I think that you
remarked to your uncle last night that she had been out to
see him?”

“Yes, and she was the girl who waited in the drawing-room,
and who may have heard uncle’s remarks about the coronet.”

“I see. You infer that she may have gone out to tell her
sweetheart, and that the two may have planned the robbery.”

“But what is the good of all these vague theories,” cried
the banker, impatiently, “when I have told you that I saw
Arthur with the coronet in his hands?”

“Wait a little, Mr.~Holder. We must come back to that.
About this girl, Miss Holder. You saw her return by the
kitchen door, I presume?”

“Yes; when I went to see if the door was fastened for the
night I met her slipping in. I saw the man, too, in the gloom.”

“Do you know him?”

“Oh yes; he is the green-grocer who brings our vegetables
round. His name is Francis Prosper.”

“He stood,” said Holmes, “to the left of the door -- that
is to say, farther up the path than is necess\-ary to reach the
door?”

“Yes, he did.”

“And he is a man with a wooden leg?”

Something like fear sprang up in the young lady’s expressive
black eyes. “Why, you are like a magician,” said she.
“How do you know that?” She smiled, but there was no
answering smile in Holmes’s thin, eager face.

“I should be very glad now to go up-stairs,” said he. “I
shall probably wish to go over the outside of the house again.
%%301
Perhaps I had better take a look at the lower windows before
I go up.”

He walked swiftly round from one to the other, pausing
only at the large one which looked from the hall onto the
stable lane. This he opened, and made a very careful examination
of the sill with his powerful magnifying lens. “Now
we shall go up-stairs,” said he, at last.

The banker’s dressing-room was a plainly furnished little
chamber, with a gray carpet, a large bureau, and a long mirror.
Holmes went to the bureau first and looked hard at the
lock.

“Which key was used to open it?” he asked.

“That which my son himself indicated -- that of the cupboard
of the lumber-room.”

“Have you it here?”

“That is it on the dressing-table.”

Sherlock Holmes took it up and opened the bureau.

“It is a noiseless lock,” said he. “It is no wonder that it
did not wake you. This case, I presume, contains the coronet.
We must have a look at it.” He opened the case, and, taking
out the diadem, he laid it upon the table. It was a magnificent
specimen of the jeweller’s art, and the thirty-six stones
were the finest that I have ever seen. At one side of the
coronet was a cracked edge, where a corner holding three
gems had been torn away.

“Now, Mr.~Holder,” said Holmes, “here is the corner
which corresponds to that which has been so unfortunately
lost. Might I beg that you will break it off.”

The banker recoiled in horror. “I should not dream of
trying,” said he.

“Then I will.” Holmes suddenly bent his strength upon
it, but without result. “I feel it give a little,” said he; “but,
though I am exceptionally strong in the fingers, it would take
me all my time to break it. An ordinary man could not do it.
Now, what do you think would happen if I did break it, Mr.
Holder? There would be a noise like a pistol shot. Do you
%%302
tell me that all this happened within a few yards of your bed,
and that you heard nothing of it?”

“I do not know what to think. It is all dark to me.”

“But perhaps it may grow lighter as we go. What do you
think, Miss Holder?”

“I confess that I still share my uncle’s perplexity.”

“Your son had no shoes or slippers on when you saw
him?”

“He had nothing on save only his trousers and shirt.”

“Thank you. We have certainly been favored with extraordinary
luck during this inquiry, and it will be entirely
our own fault if we do not succeed in clearing the matter up.
With your permission, Mr.~Holder, I shall now continue my
investigations outside.”

He went alone, at his own request, for he explained that
any unnecessary footmarks might make his task more difficult.
For an hour or more he was at work, returning at last
with his feet heavy with snow and his features as inscrutable
as ever.

“I think that I have seen now all that there is to see, Mr.
Holder,” said he; “I can serve you best by returning to
my rooms.”

“But the gems, Mr.~Holmes. Where are they?”

“I cannot tell.”

The banker wrung his hands. “I shall never see them
again!” he cried. “And my son? You give me hopes?”

“My opinion is in no way altered.”

“Then, for God’s sake, what was this dark business which
was acted in my house last night?”

“If you can call upon me at my Baker Street rooms
to-morrow morning between nine and ten I shall be happy to do
what I can to make it clearer. I understand that you give
me \textit{carte blanche} to act for you, provided only that I get back
the gems, and that you place no limit on the sum I may draw.”

“I would give my fortune to have them back.”

“Very good. I shall look into the matter between this and
%%303
then. Good-bye; it is just possible that I may have to come
over here again before evening.”

It was obvious to me that my companion’s mind was now
made up about the case, although what his conclusions were
was more than I could even dimly imagine. Several times
during our homeward journey I endeavored to sound him
upon the point, but he always glided away to some other topic,
until at last I gave it over in despair. It was not yet three
when we found ourselves in our room once more. He hurried
to his chamber, and was down again in a few minutes dressed
as a common loafer. With his collar turned up, his shiny,
seedy coat, his red cravat, and his worn boots, he was a perfect
sample of the class.

“I think that this should do,” said he, glancing into the
glass above the fireplace. “I only wish that you could come
with me, Watson, but I fear that it won’t do. I may be on
the trail in this matter, or I may be following a will-of-the-%
wisp, but I shall soon know which it is. I hope that I may be
back in a few hours.” He cut a slice of beef from the joint
upon the sideboard, sandwiched it between two rounds of
bread, and, thrusting this rude meal into his pocket, he started
off upon his expedition.

I had just finished my tea when he returned, evidently in
excellent spirits, swinging an old elastic-sided boot in his
hand. He chucked it down into a corner and helped himself
to a cup of tea.

“I only looked in as I passed,” said he. “I am going
right on.”

“Where to?”

“Oh, to the other side of the West End. It may be some
time before I get back. Don’t wait up for me in case I should
be late.”

“How are you getting on?”

“Oh, so so. Nothing to complain of. I have been out to
Streatham since I saw you last, but I did not call at the house.
It is a very sweet little problem, and I would not have missed
%%304
it for a good deal. However, I must not sit gossiping here,
but must get these disreputable clothes off and return to my
highly respectable self.”

I could see by his manner that he had stronger reasons for
satisfaction than his words alone would imply. His eyes
twinkled, and there was even a touch of color upon his sallow
cheeks. He hastened up-stairs, and a few minutes later I
heard the slam of the hall door, which told me that he was off
once more upon his congenial hunt.

I waited until midnight, but there was no sign of his return,
so I retired to my room. It was no uncommon thing for him
to be away for days and nights on end when he was hot upon
a scent, so that his lateness caused me no surprise. I do not
know at what hour he came in, but when I came down to
breakfast in the morning, there he was with a cup of coffee
in one hand and the paper in the other, as fresh and trim as
possible.

“You will excuse my beginning without you, Watson,” said
he; “but you remember that our client has rather an early
appointment this morning.”

“Why, it is after nine now,” I answered. “I should not be
surprised if that were he. I thought I heard a ring.”

It was, indeed, our friend the financier. I was shocked by
the change which had come over him, for his face, which was
naturally of a broad and massive mould, was now pinched and
fallen in, while his hair seemed to me at least a shade whiter.
He entered with a weariness and lethargy which was even
more painful than his violence of the morning before, and he
dropped heavily into the arm-chair which I pushed forward
for him.

“I do not know what I have done to be so severely tried,”
said he. “Only two days ago I was a happy and prosperous
man, without a care in the world. Now I am left to a lonely
and dishonored age. One sorrow comes close upon the heels
of another. My niece, Mary, has deserted me.”

“Deserted you?”
%%305

“Yes. Her bed this morning had not been slept in, her
room was empty, and a note for me lay upon the hall table.
I had said to her last night, in sorrow and not in anger, that
if she had married my boy all might have been well with him.
Perhaps it was thoughtless of me to say so. It is to that remark
that she refers in this note:

\begin{letter}
“\,‘\textsc{My dearest Uncle}, -- I feel that I have brought trouble
upon you, and that if I had acted differently this terrible
misfortune might never have occurred. I cannot, with this thought
in my mind, ever again be happy under your roof, and I feel
that I must leave you for ever. Do not worry about my future,
for that is provided for; and, above all, do not search for
me, for it will be fruitless labor and an ill-service to me. In
life or in death, I am ever your loving \hfill\textsc{Mary}.’
\end{letter}

“What could she mean by that note, Mr.~Holmes? Do
you think it points to suicide?”

“No, no, nothing of the kind. It is perhaps the best possible
solution. I trust, Mr.~Holder, that you are nearing the
end of your troubles.”

“Ha! You say so! You have heard something, Mr.~Holmes;
you have learned something! Where are the gems?”

“You would not think £1000 apiece an excessive sum for
them?”

“I would pay ten.”

“That would be unnecessary. Three thousand will cover
the matter. And there is a little reward, I fancy. Have you
your check-book? Here is a pen. Better make it out for
£4000 pounds.”

With a dazed face the banker made out the required check.
Holmes walked over to his desk, took out a little triangular
piece of gold with three gems in it, and threw it down upon
the table.

With a shriek of joy our client clutched it up.

“You have it!” he gasped. “I am saved! I am saved!”
%%306

The reaction of joy was as passionate as his grief had been,
and he hugged his recovered gems to his bosom.

“There is one other thing you owe, Mr.~Holder,” said
Sherlock Holmes, rather sternly.

“Owe!” He caught up a pen. “Name the sum, and I
will pay it.”

“No, the debt is not to me. You owe a very humble
apology to that noble lad, your son, who has carried himself
in this matter as I should be proud to see my own son do,
should I ever chance to have one.”

“Then it was not Arthur who took them?”

“I told you yesterday, and I repeat to-day, that it was not.”

“You are sure of it! Then let us hurry to him at once, to
let him know that the truth is known.”

“He knows it already. When I had cleared it all up I had
an interview with him, and, finding that he would not tell me
the story, I told it to him, on which he had to confess that I
was right, and to add the very few details which were not yet
quite clear to me. Your news of this morning, however, may
open his lips.”

“For Heaven’s sake, tell me, then, what is this extraordinary
mystery!”

“I will do so, and I will show you the steps by which I
reached it. And let me say to you, first, that which it is
hardest for me to say and for you to hear: there has been
an understanding between Sir George Burnwell and your niece
Mary. They have now fled together.”

“My Mary? Impossible!”

“It is, unfortunately, more than possible; it is certain.
Neither you nor your son knew the true character of this man
when you admitted him into your family circle. He is one of
the most dangerous men in England -- a ruined gambler, an
absolutely desperate villain, a man without heart or conscience.
Your niece knew nothing of such men. When he
breathed his vows to her, as he had done to a hundred before
her, she flattered herself that she alone had touched his
%%307
heart. The devil knows best what he said, but at least she
became his tool, and was in the habit of seeing him nearly
every evening.”

“I cannot, and I will not, believe it!” cried the banker, with
an ashen face.

“I will tell you, then, what occurred in your house last
night. Your niece, when you had, as she thought, gone to your
room, slipped down and talked to her lover through the window
which leads into the stable lane. His footmarks had pressed
right through the snow, so long had he stood there. She told
him of the coronet. His wicked lust for gold kindled at the
news, and he bent her to his will. I have no doubt that she
loved you, but there are women in whom the love of a lover
extinguishes all other loves, and I think that she must have
been one. She had hardly listened to his instructions when
she saw you coming down-stairs, on which she closed the window
rapidly, and told you about one of the servants’ escapade
with her wooden-legged lover, which was all perfectly true.

“Your boy, Arthur, went to bed after his interview with
you, but he slept badly on account of his uneasiness about his
club debts. In the middle of the night he heard a soft tread
pass his door, so he rose, and looking out, was surprised to see
his cousin walking very stealthily along the passage, until she
disappeared into your dressing-room. Petrified with astonishment,
the lad slipped on some clothes, and waited there in the
dark to see what would come of this strange affair. Presently
she emerged from the room again, and in the light of the
passage-lamp your son saw that she carried the precious coronet
in her hands. She passed down the stairs, and he, thrilling
with horror, ran along and slipped behind the curtain near
your door, whence he could see what passed in the hall beneath.
He saw her stealthily open the window, hand out the
coronet to some one in the gloom, and then closing it once
more hurry back to her room, passing quite close to where he
stood hid behind the curtain.

“As long as she was on the scene he could not take any
%%308
action without a horrible exposure of the woman whom he
loved. But the instant that she was gone he realized how
crushing a misfortune this would be for you, and how all-important
it was to set it right. He rushed down, just as he
was, in his bare feet, opened the window, sprang out into the
snow, and ran down the lane, where he could see a dark figure
in the moonlight. Sir George Burnwell tried to get away, but
Arthur caught him, and there was a struggle between them,
your lad tugging at one side of the coronet, and his opponent
at the other. In the scuffle, your son struck Sir George, and
cut him over the eye. Then something suddenly snapped, and
your son, finding that he had the coronet in his hands, rushed
back, closed the window, ascended to your room, and had just
observed that the coronet had been twisted in the struggle,
and was endeavoring to straighten it when you appeared upon
the scene.”

“Is it possible?” gasped the banker.

“You then roused his anger by calling him names at a moment
when he felt that he had deserved your warmest thanks.
He could not explain the true state of affairs without betraying
one who certainly deserved little enough consideration at
his hands. He took the more chivalrous view, however, and
preserved her secret.”

“And that was why she shrieked and fainted when she saw
the coronet,” cried Mr.~Holder. “Oh, my God! what a blind
fool I have been! And his asking to be allowed to go out for
five minutes! The dear fellow wanted to see if the missing
piece were at the scene of the struggle. How cruelly I have
misjudged him!”

“When I arrived at the house,” continued Holmes, “I at
once went very carefully round it to observe if there were any
traces in the snow which might help me. I knew that none
had fallen since the evening before, and also that there had
been a strong frost to preserve impressions. I passed along
the tradesmen’s path, but found it all trampled down and
indistinguishable. Just beyond it, however, at the far side of
%%309
the kitchen door, a woman had stood and talked with a man,
whose round impressions on one side showed that he had a
wooden leg. I could even tell that they had been disturbed,
for the woman had run back swiftly to the door, as was shown
by the deep toe and light heel marks, while Wooden-leg had
waited a little, and then had gone away. I thought at the
time that this might be the maid and her sweetheart, of whom
you had already spoken to me, and inquiry showed it was so.
I passed round the garden without seeing anything more than
random tracks, which I took to be the police; but when I got
into the stable lane a very long and complex story was written
in the snow in front of me.

“There was a double line of tracks of a booted man, and
a second double line which I saw with delight belonged to a
man with naked feet. I was at once convinced from what you
had told me that the latter was your son. The first had
walked both ways, but the other had run swiftly, and, as his
tread was marked in places over the depression of the boot, it
was obvious that he had passed after the other. I followed
them up, and found that they led to the hall window, where
Boots had worn all the snow away while waiting. Then I
walked to the other end, which was a hundred yards or more
down the lane. I saw where Boots had faced round, where
the snow was cut up as though there had been a struggle, and,
finally, where a few drops of blood had fallen, to show me
that I was not mistaken. Boots had then run down the lane,
and another little smudge of blood showed that it was he who
had been hurt. When he came to the high-road at the other
end, I found that the pavement had been cleared, so there
was an end to that clew.

“On entering the house, however, I examined, as you remember,
the sill and framework of the hall window with my
lens, and I could at once see that some one had passed out.
I could distinguish the outline of an instep where the wet foot
had been placed in coming in. I was then beginning to be
able to form an opinion as to what had occurred. A man had
%%310
waited outside the window, some one had brought the gems;
the deed had been overseen by your son, he had pursued the
thief, had struggled with him, they had each tugged at the
coronet, their united strength causing injuries which neither
alone could have effected. He had returned with the prize,
but had left a fragment in the grasp of his opponent. So far
I was clear. The question now was, who was the man, and
who was it brought him the coronet?

“It is an old maxim of mine that when you have excluded
the impossible, whatever remains, however improbable, must
be the truth. Now, I knew that it was not you who had
brought it down, so there only remained your niece and the
maids. But if it were the maids, why should your son allow
himself to be accused in their place? There could be no
possible reason. As he loved his cousin, however, there was
an excellent explanation why he should retain her secret -- the
more so as the secret was a disgraceful one. When I remembered
that you had seen her at that window, and how she
had fainted on seeing the coronet again, my conjecture became
a certainty.

“And who could it be who was her confederate? A lover
evidently, for who else could outweigh the love and gratitude
which she must feel to you? I knew that you went out little,
and that your circle of friends was a very limited one. But
among them was Sir George Burnwell. I had heard of him
before as being a man of evil reputation among women. It
must have been he who wore those boots and retained the
missing gems. Even though he knew that Arthur had discovered
him, he might still flatter himself that he was safe,
for the lad could not say a word without compromising his
own family.

“Well, your own good sense will suggest what measures I
took next. I went in the shape of a loafer to Sir George’s
house, managed to pick up an acquaintance with his valet,
learned that his master had cut his head the night before, and,
finally, at the expense of six shillings, made all sure by buying
%%311
%%“I CLAPPED A PISTOL TO HIS HEAD”
%%312
a pair of his cast-off shoes. With these I journeyed down to
Streatham, and saw that they exactly fitted the tracks.”

“I saw an ill-dressed vagabond in the lane yesterday evening,”
said Mr.~Holder.

“Precisely. It was I. I found that I had my man, so I
came home and changed my clothes. It was a delicate part
which I had to play then, for I saw that a prosecution must
be avoided to avert scandal, and I knew that so astute a villain
would see that our hands were tied in the matter. I went
and saw him. At first, of course, he denied everything. But
when I gave him every particular that had occurred, he tried
to bluster, and took down a life-preserver from the wall. I
knew my man, however, and I clapped a pistol to his head
before he could strike. Then he became a little more reasonable.
I told him that we would give him a price for the
stones he held -- £1000 apiece. That brought out the first
signs of grief that he had shown. ‘Why, dash it all!’ said he,
‘I’ve let them go at six hundred for the three!’ I soon managed
to get the address of the receiver who had them, on
promising him that there would be no prosecution. Off I set
to him, and after much chaffering I got our stones at £1000
apiece. Then I looked in upon your son, told him that all
was right, and eventually got to my bed about two o’clock,
after what I may call a really hard day’s work.”

“A day which has saved England from a great public
scandal,” said the banker, rising. “Sir, I cannot find words
to thank you, but you shall not find me ungrateful for what
you have done. Your skill has indeed exceeded all that I
have heard of it. And now I must fly to my dear boy to
apologize to him for the wrong which I have done him. As
to what you tell me of poor Mary, it goes to my very heart.
Not even your skill can inform me where she is now.”

“I think that we may safely say,” returned Holmes, “that
she is wherever Sir George Burnwell is. It is equally certain,
too, that whatever her sins are, they will soon receive a more
than sufficient punishment.”
%%314

\Chapter{The Adventure Of The Copper Beeches}

“\textsc{To} the man who loves art for its own sake,” remarked
Sherlock Holmes, tossing aside the advertisement
sheet of \textit{The Daily Telegraph}, “it is
frequently in its least important and lowliest
manifestations that the keenest pleasure is to be derived. It
is pleasant to me to observe, Watson, that you have so far
grasped this truth that in these little records of our cases
which you have been good enough to draw up, and, I am
bound to say, occasionally to embellish, you have given prominence
not so much to the many \textit{causes célèbres} and sensational
trials in which I have figured, but rather to those incidents
which may have been trivial in themselves, but which have
given room for those faculties of deduction and of logical
synthesis which I have made my special province.”

“And yet,” said I, smiling, “I cannot quite hold myself
absolved from the charge of sensationalism which has been
urged against my records.”

“You have erred, perhaps,” he observed, taking up a glowing
cinder with the tongs, and lighting with it the long cherry-wood
pipe which was wont to replace his clay when he was in
a disputatious, rather than a meditative mood -- “you have
erred perhaps in attempting to put color and life into each of
your statements, instead of confining yourself to the task of
placing upon record that severe reasoning from cause to effect
which is really the only notable feature about the thing.”

“It seems to me that I have done you full justice in the
matter,” I remarked, with some coldness, for I was repelled
%%315
by the egotism which I had more than once observed to be a
strong factor in my friend’s singular character.

“No, it is not selfishness or conceit,” said he, answering, as
was his wont, my thoughts rather than my words. “If I claim
full justice for my art, it is because it is an impersonal thing
-- a thing beyond myself. Crime is common. Logic is rare.
Therefore it is upon the logic rather than upon the crime that
you should dwell. You have degraded what should have been
a course of lectures into a series of tales.”

It was a cold morning of the early spring, and we sat after
breakfast on either side of a cheery fire in the old room at
Baker Street. A thick fog rolled down between the lines of
dun-colored houses, and the opposing windows loomed like
dark, shapeless blurs through the heavy yellow wreaths. Our
gas was lit, and shone on the white cloth and glimmer of
china and metal, for the table had not been cleared yet. Sherlock
Holmes had been silent all the morning, dipping continuously
into the advertisement columns of a succession of papers,
until at last, having apparently given up his search, he had
emerged in no very sweet temper to lecture me upon my literary
shortcomings.

“At the same time,” he remarked, after a pause, during
which he had sat puffing at his long pipe and gazing down
into the fire, “you can hardly be open to a charge of
sensationalism, for out of these cases which you have been so kind
as to interest yourself in, a fair proportion do not treat of
crime, in its legal sense, at all. The small matter in which I
endeavored to help the King of Bohemia, the singular experience
of Miss Mary Sutherland, the problem connected with
the man with the twisted lip, and the incident of the noble
bachelor, were all matters which are outside the pale of the
law. But in avoiding the sensational, I fear that you may
have bordered on the trivial.”

“The end may have been so,” I answered, “but the methods
I hold to have been novel and of interest.”

“Pshaw, my dear fellow, what do the public, the great
%%316
unobservant public, who could hardly tell a weaver by his tooth or
a compositor by his left thumb, care about the finer shades of
analysis and deduction! But, indeed, if you are trivial, I cannot
blame you, for the days of the great cases are past. Man,
or at least criminal man, has lost all enterprise and originality.
As to my own little practice, it seems to be degenerating into
an agency for recovering lost lead pencils and giving advice
to young ladies from boarding-schools. I think that I have
touched bottom at last, however. This note I had this morning
marks my zero-point, I fancy. Read it!” He tossed a
crumpled letter across to me.

It was dated from Montague Place upon the preceding
evening, and ran thus:

\begin{letter}
“\textsc{Dear Mr.~Holmes}, -- I am very anxious to consult you as
to whether I should or should not accept a situation which
has been offered to me as governess. I shall call at half-past
ten to-morrow, if I do not inconvenience you.

\hspace{2em}“Yours faithfully, \hfill\textsc{Violet~Hunter}.”
\end{letter}

“Do you know the young lady?” I asked.

“Not I.”

“It is half-past ten now.”

“Yes, and I have no doubt that is her ring.”

“It may turn out to be of more interest than you think.
You remember that the affair of the blue carbuncle, which
appeared to be a mere whim at first, developed into a serious
investigation. It may be so in this case, also.”

“Well, let us hope so. But our doubts will very soon be
solved, for here, unless I am much mistaken, is the person in
question.”

As he spoke the door opened and a young lady entered the
room. She was plainly but neatly dressed, with a bright,
quick face, freckled like a plover’s egg, and with the brisk
manner of a woman who has had her own way to make in the
world.
%%317

“You will excuse my troubling you, I am sure,” said she,
as my companion rose to greet her; “but I have had a very
strange experience, and as I have no parents or relations of
any sort from whom I could ask advice, I thought that perhaps
you would be kind enough to tell me what I should do.”

“Pray take a seat, Miss Hunter. I shall be happy to do
anything that I can to serve you.”

I could see that Holmes was favorably impressed by the
manner and speech of his new client. He looked her over in
his searching fashion, and then composed himself, with his lids
drooping and his finger tips together, to listen to her story.

“I have been a governess for five years,” said she, “in the
family of Colonel Spence Munro, but two months ago the
colonel received an appointment at Halifax, in Nova Scotia,
and took his children over to America with him, so that I
found myself without a situation. I advertised, and I answered
advertisements, but without success. At last the little money
which I had saved began to run short, and I was at my wits’
end as to what I should do.

“There is a well-known agency for governesses in the West
End called Westaway’s, and there I used to call about once a
week in order to see whether anything had turned up which
might suit me. Westaway was the name of the founder of
the business, but it is really managed by Miss Stoper. She
sits in her own little office, and the ladies who are seeking
employment wait in an ante-room, and are then shown in one
by one, when she consults her ledgers, and sees whether she
has anything which would suit them.

“Well, when I called last week I was shown into the little
office as usual, but I found that Miss Stoper was not alone.
A prodigiously stout man with a very smiling face, and a great
heavy chin which rolled down in fold upon fold over his throat,
sat at her elbow with a pair of glasses on his nose, looking
very earnestly at the ladies who entered. As I came in he
gave quite a jump in his chair, and turned quickly to Miss
Stoper:
%%318

“\,‘That will do,’ said he; ‘I could not ask for anything
better. Capital! capital!’ He seemed quite enthusiastic, and
rubbed his hands together in the most genial fashion. He
was such a comfortable-looking man that it was quite a pleasure
to look at him.

“\,‘You are looking for a situation, miss?’ he asked.

“\,‘Yes, sir.’

“\,‘As governess?’

“\,‘Yes, sir.’

“\,‘And what salary do you ask?’

“\,‘I had £4 a month in my last place with Colonel Spence
Munro.’

“\,‘Oh, tut, tut! sweating -- rank sweating!’ he cried, throwing
his fat hands out into the air like a man who is in a
boiling passion. ‘How could any one offer so pitiful a sum to
a lady with such attractions and accomplishments?’

“\,‘My accomplishments, sir, may be less than you imagine,’
said I. ‘A little French, a little German, music, and drawing -- ’

“\,‘Tut, tut!’ he cried. ‘This is all quite beside the question.
The point is, have you or have you not the bearing and deportment
of a lady? There it is in a nutshell. If you have
not, you are not fitted for the rearing of a child who may some
day play a considerable part in the history of the country.
But if you have, why, then, how could any gentleman ask you
to condescend to accept anything under the three figures?
Your salary with me, madam, would commence at £100 a
year.’

“You may imagine, Mr.~Holmes, that to me, destitute as I
was, such an offer seemed almost too good to be true. The
gentleman, however, seeing perhaps the look of incredulity
upon my face, opened a pocket-book and took out a note.

“\,‘It is also my custom,’ said he, smiling in the most pleasant
fashion until his eyes were just two little shining slits
amid the white creases of his face, ‘to advance to my young
ladies half their salary beforehand, so that they may meet any
little expenses of their journey and their wardrobe.’
%%319

“It seemed to me that I had never met so fascinating and
so thoughtful a man. As I was already in debt to my tradesmen,
the advance was a great convenience, and yet there was
something unnatural about the whole transaction which made
me wish to know a little more before I quite committed myself.

“\,‘May I ask where you live, sir?’ said I.

“\,‘Hampshire. Charming rural place. The Copper Beeches,
five miles on the far side of Winchester. It is the most lovely
country, my dear young lady, and the dearest old country-%
house.’

“\,‘And my duties, sir? I should be glad to know what they
would be.’

“\,‘One child -- one dear little romper just six years old.
Oh, if you could see him killing cockroaches with a slipper!
Smack! smack! smack! Three gone before you could wink!’
He leaned back in his chair and laughed his eyes into his
head again.

“I was a little startled at the nature of the child’s amusement,
but the father’s laughter made me think that perhaps he
was joking.

“\,‘My sole duties, then,’ I asked, ‘are to take charge of a
single child?’

“\,‘No, no, not the sole, not the sole, my dear young lady,’
he cried. ‘Your duty would be, as I am sure your good sense
would suggest, to obey any little commands my wife might
give, provided always that they were such commands as a lady
might with propriety obey. You see no difficulty, heh?’

“\,‘I should be happy to make myself useful.’

“\,‘Quite so. In dress now, for example. We are faddy
people, you know -- faddy but kind-hearted. If you were asked
to wear any dress which we might give you, you would not
object to our little whim. Heh?’

“\,‘No,’ said I, considerably astonished at his words.

“\,‘Or to sit here, or sit there, that would not be offensive
to you?’

“\,‘Oh, no.’
%%320

“\,‘Or to cut your hair quite short before you come to us?’

“I could hardly believe my ears. As you may observe, Mr.
Holmes, my hair is somewhat luxuriant, and of a rather peculiar
tint of chestnut. It has been considered artistic. I could
not dream of sacrificing it in this off-hand fashion.

“\,‘I am afraid that that is quite impossible,’ said I. He
had been watching me eagerly out of his small eyes, and I
could see a shadow pass over his face as I spoke.

“\,‘I am afraid that it is quite essential,’ said he. ‘It is a
little fancy of my wife’s, and ladies’ fancies, you know, madam,
ladies’ fancies must be consulted. And so you won’t cut your
hair?’

“\,‘No, sir, I really could not,’ I answered, firmly.

“\,‘Ah, very well; then that quite settles the matter. It is
a pity, because in other respects you would really have done
very nicely. In that case, Miss Stoper, I had best inspect a
few more of your young ladies.’

“The manageress had sat all this while busy with her
papers without a word to either of us, but she glanced at me
now with so much annoyance upon her face that I could not
help suspecting that she had lost a handsome commission
through my refusal.

“\,‘Do you desire your name to be kept upon the books?’
she asked.

“\,‘If you please, Miss Stoper.’

“\,‘Well, really, it seems rather useless, since you refuse the
most excellent offers in this fashion,’ said she, sharply. ‘You
can hardly expect us to exert ourselves to find another such
opening for you. Good-day to you, Miss Hunter.’ She struck
a gong upon the table, and I was shown out by the page.

“Well, Mr.~Holmes, when I got back to my lodgings and
found little enough in the cupboard, and two or three bills
upon the table, I began to ask myself whether I had not done
a very foolish thing. After all, if these people had strange
fads, and expected obedience on the most extraordinary matters,
they were at least ready to pay for their eccentricity.
%%321
Very few governesses in England are getting £100 a year.
Besides, what use was my hair to me? Many people are improved
by wearing it short, and perhaps I should be among
the number. Next day I was inclined to think that I had
made a mistake, and by the day after I was sure of it. I had
almost overcome my pride, so far as to go back to the agency
and inquire whether the place was still open, when I received
this letter from the gentleman himself. I have it here, and I
will read it to you:

“\,‘The Copper Beeches, near Winchester.

\begin{letter}
“\,‘\textsc{Dear Miss Hunter}, -- Miss Stoper has very kindly given
me your address, and I write from here to ask you whether
you have reconsidered your decision. My wife is very anxious
that you should come, for she has been much attracted by my
description of you. We are willing to give £30 a quarter, or
£120 a year, so as to recompense you for any little inconvenience
which our fads may cause you. They are not very
exacting, after all. My wife is fond of a particular shade of
electric blue, and would like you to wear such a dress in-doors
in the morning. You need not, however, go to the expense of
purchasing one, as we have one belonging to my dear daughter
Alice (now in Philadelphia), which would, I should think, fit
you very well. Then, as to sitting here or there, or amusing
yourself in any manner indicated, that need cause you no
inconvenience. As regards your hair, it is no doubt a pity,
especially as I could not help remarking its beauty during our
short interview, but I am afraid that I must remain firm upon
this point, and I only hope that the increased salary may
recompense you for the loss. Your duties, as far as the child
is concerned, are very light. Now do try to come, and I shall
meet you with the dog-cart at Winchester. Let me know your
train.

\hspace{1em}Yours faithfully, \hfill\textsc{Jephro~Rucastle}.’
\end{letter}

“That is the letter which I have just received, Mr.~Holmes,
and my mind is made up that I will accept it. I thought,
%%322
however, that before taking the final step I should like to
submit the whole matter to your consideration.”

“Well, Miss Hunter, if your mind is made up, that settles
the question,” said Holmes, smiling.

“But you would not advise me to refuse?”

“I confess that it is not the situation which I should like to
see a sister of mine apply for.”

“What is the meaning of it all, Mr.~Holmes?”

“Ah, I have no data. I cannot tell. Perhaps you have
yourself formed some opinion?”

“Well, there seems to me to be only one possible solution.
Mr.~Rucastle seemed to be a very kind, good-natured man. Is
it not possible that his wife is a lunatic, that he desires to
keep the matter quiet for fear she should be taken to an asylum,
and that he humors her fancies in every way in order to
prevent an outbreak.”

“That is a possible solution -- in fact, as matters stand, it is
the most probable one. But in any case it does not seem to
be a nice household for a young lady.”

“But the money, Mr.~Holmes, the money!”

“Well, yes, of course the pay is good -- too good. That is
what makes me uneasy. Why should they give you £120 a
year, when they could have their pick for £40? There must
be some strong reason behind.”

“I thought that if I told you the circumstances you would
understand afterwards if I wanted your help. I should feel so
much stronger if I felt that you were at the back of me.”

“Oh, you may carry that feeling away with you. I assure
you that your little problem promises to be the most interesting
which has come my way for some months. There is
something distinctly novel about some of the features. If
you should find yourself in doubt or in danger -- ”

“Danger! What danger do you foresee?”

Holmes shook his head gravely. “It would cease to be a
danger if we could define it,” said he. “But at any time, day
or night, a telegram would bring me down to your help.”
%%323

“That is enough.” She rose briskly from her chair with
the anxiety all swept from her face. “I shall go down to
Hampshire quite easy in my mind now. I shall write to Mr.
Rucastle at once, sacrifice my poor hair to-night, and start
for Winchester to-morrow.” With a few grateful words to
Holmes she bade us both good-night and bustled off upon
her way.

“At least,” said I, as we heard her quick, firm step descending
the stairs, “she seems to be a young lady who is very well
able to take care of herself.”

“And she would need to be,” said Holmes, gravely; “I am
much mistaken if we do not hear from her before many days
are past.”

It was not very long before my friend’s prediction was fulfilled.
A fortnight went by, during which I frequently found
my thoughts turning in her direction, and wondering what
strange side-alley of human experience this lonely woman had
strayed into. The unusual salary, the curious conditions, the
light duties, all pointed to something abnormal, though whether
a fad or a plot, or whether the man were a philanthropist or a
villain, it was quite beyond my powers to determine. As to
Holmes, I observed that he sat frequently for half an hour on
end, with knitted brows and an abstracted air, but he swept
the matter away with a wave of his hand when I mentioned it.
“Data! data! data!” he cried, impatiently. “I can’t make
bricks without clay.” And yet he would always wind up by
muttering that no sister of his should ever have accepted such
a situation.

The telegram which we eventually received came late one
night, just as I was thinking of turning in, and Holmes was
settling down to one of those all-night chemical researches which
he frequently indulged in, when I would leave him stooping
over a retort and a test-tube at night, and find him in the same
position when I came down to breakfast in the morning. He
opened the yellow envelope, and then, glancing at the message,
threw it across to me.
%%324

“Just look up the trains in Bradshaw,” said he, and turned
back to his chemical studies.

The summons was a brief and urgent one.

“Please be at the ‘Black Swan’ Hotel at Winchester at
mid-day to-morrow,” it said. “Do come! I am at my wits’
end. \hfill\textsc{Hunter}.”

“Will you come with me?” asked Holmes, glancing up.

“I should wish to.”

“Just look it up, then.”

“There is a train at half-past nine,” said I, glancing over
my Bradshaw. “It is due at Winchester at 11.30.”

“That will do very nicely. Then perhaps I had better
postpone my analysis of the acetones, as we may need to be
at our best in the morning.”

By eleven o’clock the next day we were well upon our way
to the old English capital. Holmes had been buried in the
morning papers all the way down, but after we had passed the
Hampshire border he threw them down, and began to admire
the scenery. It was an ideal spring day, a light blue sky,
flecked with little fleecy white clouds drifting across from
west to east. The sun was shining very brightly, and yet
there was an exhilarating nip in the air, which set an edge to
a man’s energy. All over the country-side, away to the rolling
hills around Aldershot, the little red and gray roofs of the
farm-steadings peeped out from amid the light green of the
new foliage.

“Are they not fresh and beautiful?” I cried, with all the
enthusiasm of a man fresh from the fogs of Baker Street.

But Holmes shook his head gravely.

“Do you know, Watson,” said he, “that it is one of the
curses of a mind with a turn like mine that I must look at
everything with reference to my own special subject. You
look at these scattered houses, and you are impressed by their
beauty. I look at them, and the only thought which comes
%%325
to me is a feeling of their isolation and of the impunity with
which crime may be committed there.”

“Good heavens!” I cried. “Who would associate crime
with these dear old homesteads?”

“They always fill me with a certain horror. It is my belief,
Watson, founded upon my experience, that the lowest and
vilest alleys in London do not present a more dreadful record
of sin than does the smiling and beautiful country-side.”

“You horrify me!”

“But the reason is very obvious. The pressure of public
opinion can do in the town what the law cannot accomplish.
There is no lane so vile that the scream of a tortured child, or
the thud of a drunkard’s blow, does not beget sympathy and
indignation among the neighbors, and then the whole machinery
of justice is ever so close that a word of complaint can set
it going, and there is but a step between the crime and the
dock. But look at these lonely houses, each in its own fields,
filled for the most part with poor ignorant folk who know little
of the law. Think of the deeds of hellish cruelty, the hidden
wickedness which may go on, year in, year out, in such places,
and none the wiser. Had this lady who appeals to us for help
gone to live in Winchester, I should never have had a fear for
her. It is the five miles of country which makes the danger.
Still, it is clear that she is not personally threatened.”

“No. If she can come to Winchester to meet us she can
get away.”

“Quite so. She has her freedom.”

“What \textit{can} be the matter, then? Can you suggest no explanation?”

“I have devised seven separate explanations, each of which
would cover the facts as far as we know them. But which of
these is correct can only be determined by the fresh information
which we shall no doubt find waiting for us. Well, there
is the tower of the cathedral, and we shall soon learn all that
Miss Hunter has to tell.”

The “Black Swan” is an inn of repute in the High Street,
%%326
at no distance from the station, and there we found the young
lady waiting for us. She had engaged a sitting-room, and our
lunch awaited us upon the table.

“I am so delighted that you have come,” she said, earnes\-tly.
“It is so very kind of you both; but indeed I do not know
what I should do. Your advice will be altogether invaluable
to me.”

“Pray tell us what has happened to you.”

“I will do so, and I must be quick, for I have promised Mr.
Rucastle to be back before three. I got his leave to come
into town this morning, though he little knew for what
purpose.”

“Let us have everything in its due order.” Holmes thrust
his long thin legs out towards the fire and composed himself
to listen.

“In the first place, I may say that I have met, on the whole,
with no actual ill-treatment from Mr.~and Mrs.~Rucastle. It
is only fair to them to say that. But I cannot understand
them, and I am not easy in my mind about them.”

“What can you not understand?”

“Their reasons for their conduct. But you shall have it all
just as it occurred. When I came down, Mr.~Rucastle met me
here, and drove me in his dog-cart to the Copper Beeches. It
is, as he said, beautifully situated, but it is not beautiful in
itself, for it is a large square block of a house, whitewashed,
but all stained and streaked with damp and bad weather.
There are grounds round it, woods on three sides, and on the
fourth a field which slopes down to the Southampton high-road,
which curves past about a hundred yards from the front door.
This ground in front belongs to the house, but the woods all
round are part of Lord Southerton’s preserves. A clump of
copper beeches immediately in front of the hall door has given
its name to the place.

“I was driven over by my employer, who was as amiable as
ever, and was introduced by him that evening to his wife and
the child. There was no truth, Mr.~Holmes, in the conjecture
%%327
%%“\,‘I AM SO DELIGHTED THAT YOU HAVE COME’\,”
%%328
which seemed to us to be probable in your rooms at Baker
Street. Mrs.~Rucastle is not mad. I found her to be a silent,
pale-faced woman, much younger than her husband, not more
than thirty, I should think, while he can hardly be less than
forty-five. From their conversation I have gathered that they
have been married about seven years, that he was a widower,
and that his only child by the first wife was the daughter who
has gone to Philadelphia. Mr.~Rucastle told me in private
that the reason why she had left them was that she had an
unreasoning aversion to her step-mother. As the daughter
could not have been less than twenty, I can quite imagine that
her position must have been uncomfortable with her father’s
young wife.

“Mrs.~Rucastle seemed to me to be colorless in mind as
well as in feature. She impressed me neither favorably nor
the reverse. She was a nonentity. It was easy to see that she
was passionately devoted both to her husband and to her little
son. Her light gray eyes wandered continually from one to
the other, noting every little want and forestalling it if possible.
He was kind to her also in his bluff, boisterous fashion,
and on the whole they seemed to be a happy couple. And
yet she had some secret sorrow, this woman. She would often
be lost in deep thought, with the saddest look upon her face.
More than once I have surprised her in tears. I have thought
sometimes that it was the disposition of her child which
weighed upon her mind, for I have never met so utterly spoilt
and so ill-natured a little creature. He is small for his age,
with a head which is quite disproportionately large. His
whole life appears to be spent in an alternation between
savage fits of passion and gloomy intervals of sulking.
Giving pain to any creature weaker than himself seems to
be his one idea of amusement, and he shows quite remarkable
talent in planning the capture of mice, little birds,
and insects. But I would rather not talk about the creature,
Mr.~Holmes, and, indeed, he has little to do with my
story.”
%%330

“I am glad of all details,” remarked my friend, “whether
they seem to you to be relevant or not.”

“I shall try not to miss anything of importance. The one
unpleasant thing about the house, which struck me at once,
was the appearance and conduct of the servants. There are
only two, a man and his wife. Toller, for that is his name, is
a rough, uncouth man, with grizzled hair and whiskers, and a
perpetual smell of drink. Twice since I have been with them
he has been quite drunk, and yet Mr.~Rucastle seemed to take
no notice of it. His wife is a very tall and strong woman
with a sour face, as silent as Mrs.~Rucastle, and much less
amiable. They are a most unpleasant couple, but fortunately
I spend most of my time in the nursery and my own room,
which are next to each other in one corner of the building.

“For two days after my arrival at the Copper Beeches my
life was very quiet; on the third, Mrs.~Rucastle came down
just after breakfast and whispered something to her husband.

“\,‘Oh yes,’ said he, turning to me; ‘we are very much obliged
to you, Miss Hunter, for falling in with our whims so far as to
cut your hair. I assure you that it has not detracted in the
tiniest iota from your appearance. We shall now see how the
electric-blue dress will become you. You will find it laid out
upon the bed in your room, and if you would be so good as to
put it on we should both be extremely obliged.’

“The dress which I found waiting for me was of a peculiar
shade of blue. It was of excellent material, a sort of beige,
but it bore unmistakable signs of having been worn before.
It could not have been a better fit if I had been measured for
it. Both Mr.~and Mrs.~Rucastle expressed a delight at the
look of it, which seemed quite exaggerated in its vehemence.
They were waiting for me in the drawing-room, which is a
very large room, stretching along the entire front of the house,
with three long windows reaching down to the floor. A chair
had been placed close to the central window, with its back
turned towards it. In this I was asked to sit, and then Mr.
Rucastle, walking up and down on the other side of the room,
%%331
began to tell me a series of the funniest stories that I have
ever listened to. You cannot imagine how comical he was,
and I laughed until I was quite weary. Mrs.~Rucastle, however,
who has evidently no sense of humor, never so much as
smiled, but sat with her hands in her lap, and a sad, anxious
look upon her face. After an hour or so, Mr.~Rucastle suddenly
remarked that it was time to commence the duties of
the day, and that I might change my dress and go to little
Edward in the nursery.

“Two days later this same performance was gone through
under exactly similar circumstances. Again I changed my
dress, again I sat in the window, and again I laughed very
heartily at the funny stories of which my employer had an
immense \textit{répertoire}, and which he told inimitably. Then he
handed me a yellow-backed novel, and, moving my chair a little
sideways, that my own shadow might not fall upon the page,
he begged me to read aloud to him. I read for about ten
minutes, beginning in the heart of a chapter, and then suddenly,
in the middle of a sentence, he ordered me to cease
and to change my dress.

“You can easily imagine, Mr.~Holmes, how curious I became
as to what the meaning of this extraordinary performance
could possibly be. They were always very careful, I
observed, to turn my face away from the window, so that I
became consumed with the desire to see what was going on
behind my back. At first it seemed to be impossible, but I
soon devised a means. My hand-mirror had been broken, so a
happy thought seized me, and I concealed a piece of the glass
in my handkerchief. On the next occasion, in the midst of
my laughter, I put my handkerchief up to my eyes, and was
able with a little management to see all that there was behind
me. I confess that I was disappointed. There was
nothing. At least that was my first impression. At the second
glance, however, I perceived that there was a man standing
in the Southampton Road, a small bearded man in a gray
suit, who seemed to be looking in my direction. The road is
%%332
an important highway, and there are usually people there.
This man, however, was leaning against the railings which
bordered our field, and was looking earnestly up. I lowered
my handkerchief and glanced at Mrs.~Rucastle, to find her
eyes fixed upon me with a most searching gaze. She said
nothing, but I am convinced that she had divined that I had
a mirror in my hand, and had seen what was behind me. She
rose at once.

“\,‘Jephro,’ said she, ‘there is an impertinent fellow upon the
road there who stares up at Miss Hunter.’

“\,‘No friend of yours, Miss Hunter?’ he asked.

“\,‘No; I know no one in these parts.’

“\,‘Dear me! How very impertinent! Kindly turn round
and motion to him to go away.’

“\,‘Surely it would be better to take no notice.’

“\,‘No, no, we should have him loitering here always. Kindly
turn round and wave him away like that.’

“I did as I was told, and at the same instant Mrs.~Rucastle
drew down the blind. That was a week ago, and from that
time I have not sat again in the window, nor have I worn the
blue dress, nor seen the man in the road.”

“Pray continue,” said Holmes. “Your narrative promises
to be a most interesting one.”

“You will find it rather disconnected, I fear, and there may
prove to be little relation between the different incidents of
which I speak. On the very first day that I was at the Copper
Beeches, Mr.~Rucastle took me to a small out-house which
stands near the kitchen door. As we approached it I heard
the sharp rattling of a chain, and the sound as of a large
animal moving about.

“\,‘Look in here!’ said Mr.~Rucastle, showing me a slit between
two planks. ‘Is he not a beauty?’

“I looked through, and was conscious of two glowing eyes,
and of a vague figure huddled up in the darkness.

“\,‘Don’t be frightened,’ said my employer, laughing at the
start which I had given. ‘It’s only Carlo, my mastiff. I call
%%333
him mine, but really old Toller, my groom, is the only man
who can do anything with him. We feed him once a day, and
not too much then, so that he is always as keen as mustard.
Toller lets him loose every night, and God help the trespasser
whom he lays his fangs upon. For goodness’ sake don’t you
ever on any pretext set your foot over the threshold at night,
for it is as much as your life is worth.’

“The warning was no idle one, for two nights later I happened
to look out of my bedroom window about two o’clock
in the morning. It was a beautiful moonlight night, and the
lawn in front of the house was silvered over and almost as
bright as day. I was standing, wrapt in the peaceful beauty
of the scene, when I was aware that something was moving
under the shadow of the copper beeches. As it emerged into
the moonshine I saw what it was. It was a giant dog, as
large as a calf, tawny tinted, with hanging jowl, black muzzle,
and huge projecting bones. It walked slowly across the lawn
and vanished into the shadow upon the other side. That
dreadful silent sentinel sent a chill to my heart which I do
not think that any burglar could have done.

“And now I have a very strange experience to tell you. I
had, as you know, cut off my hair in London, and I had
placed it in a great coil at the bottom of my trunk. One
evening, after the child was in bed, I began to amuse myself
by examining the furniture of my room and by rearranging
my own little things. There was an old chest of drawers in
the room, the two upper ones empty and open, the lower one
locked. I had filled the first two with my linen, and, as I had
still much to pack away, I was naturally annoyed at not having
the use of the third drawer. It struck me that it might
have been fastened by a mere oversight, so I took out my
bunch of keys and tried to open it. The very first key fitted
to perfection, and I drew the drawer open. There was only
one thing in it, but I am sure that you would never guess what
it was. It was my coil of hair.

“I took it up and examined it. It was of the same peculiar
%%334
tint, and the same thickness. But then the impossibility of
the thing obtruded itself upon me. How \textit{could} my hair have
been locked in the drawer? With trembling hands I undid
my trunk, turned out the contents, and drew from the bottom
my own hair. I laid the two tresses together, and I assure
you that they were identical. Was it not extraordinary? Puzzle
as I would, I could make nothing at all of what it meant.
I returned the strange hair to the drawer, and I said nothing
of the matter to the Rucastles, as I felt that I had put
myself in the wrong by opening a drawer which they had
locked.

“I am naturally observant, as you may have remarked, Mr.
Holmes, and I soon had a pretty good plan of the whole
house in my head. There was one wing, however, which appeared
not to be inhabited at all. A door which faced that
which led into the quarters of the Tollers opened into this
suite, but it was invariably locked. One day, however, as I
ascended the stair, I met Mr.~Rucastle coming out through
this door, his keys in his hand, and a look on his face which
made him a very different person to the round, jovial man to
whom I was accustomed. His cheeks were red, his brow was
all crinkled with anger, and the veins stood out at his temples
with passion. He locked the door and hurried past me without
a word or a look.

“This aroused my curiosity; so when I went out for a walk
in the grounds with my charge, I strolled round to the side
from which I could see the windows of this part of the house.
There were four of them in a row, three of which were simply
dirty, while the fourth was shuttered up. They were evidently
all deserted. As I strolled up and down, glancing at them
occasionally, Mr.~Rucastle came out to me, looking as merry
and jovial as ever.

“\,‘Ah!’ said he, ‘you must not think me rude if I passed
you without a word, my dear young lady. I was preoccupied
with business matters.’

“I assured him that I was not offended. ‘By-the-way,’ said
%%335
I, ‘you seem to have quite a suite of spare rooms up there,
and one of them has the shutters up.’

“He looked surprised, and, as it seemed to me, a little
startled at my remark.

“\,‘Photography is one of my hobbies,’ said he. ‘I have
made my dark room up there. But, dear me! what an observant
young lady we have come upon. Who would have believed
it? Who would have ever believed it?’ He spoke in
a jesting tone, but there was no jest in his eyes as he looked
at me. I read suspicion there and annoyance, but no jest.

“Well, Mr.~Holmes, from the moment that I understood
that there was something about that suite of rooms which I
was not to know, I was all on fire to go over them. It was not
mere curiosity, though I have my share of that. It was more
a feeling of duty -- a feeling that some good might come from
my penetrating to this place. They talk of woman’s instinct;
perhaps it was woman’s instinct which gave me that feeling.
At any rate, it was there, and I was keenly on the lookout
for any chance to pass the forbidden door.

“It was only yesterday that the chance came. I may tell
you that, besides Mr.~Rucastle, both Toller and his wife find
something to do in these deserted rooms, and I once saw him
carrying a large black linen bag with him through the door.
Recently he has been drinking hard, and yesterday evening
he was very drunk; and, when I came up-stairs, there was the
key in the door. I have no doubt at all that he had left it
there. Mr.~and Mrs.~Rucastle were both down-stairs, and the
child was with them, so that I had an admirable opportunity.
I turned the key gently in the lock, opened the door, and
slipped through.

“There was a little passage in front of me, unpapered and
uncarpeted, which turned at a right angle at the farther end.
Round this corner were three doors in a line, the first and
third of which were open. They each led into an empty room,
dusty and cheerless, with two windows in the one and one in
the other, so thick with dirt that the evening light glimmered
%%336
dimly through them. The centre door was closed, and across
the outside of it had been fastened one of the broad bars of
an iron bed, padlocked at one end to a ring in the wall, and
fastened at the other with stout cord. The door itself was
locked as well, and the key was not there. This barricaded
door corresponded clearly with the shuttered window outside,
and yet I could see by the glimmer from beneath it that the
room was not in darkness. Evidently there was a skylight
which let in light from above. As I stood in the passage
gazing at the sinister door, and wondering what secret it
might veil, I suddenly heard the sound of steps within the
room, and saw a shadow pass backward and forward against
the little slit of dim light which shone out from under the
door. A mad, unreasoning terror rose up in me at the sight,
Mr.~Holmes. My overstrung nerves failed me suddenly, and
I turned and ran -- ran as though some dreadful hand were behind
me clutching at the skirt of my dress. I rushed down
the passage, through the door, and straight into the arms of
Mr.~Rucastle, who was waiting outside.

“\,‘So,’ said he, smiling, ‘it was you, then. I thought that it
must be when I saw the door open.’

“\,‘Oh, I am so frightened!’ I panted.

“\,‘My dear young lady! my dear young lady!’ -- you cannot
think how caressing and soothing his manner was -- ‘and what
has frightened you, my dear young lady?’

“But his voice was just a little too coaxing. He overdid it.
I was keenly on my guard against him.

“\,‘I was foolish enough to go into the empty wing,’ I answered.
‘But it is so lonely and eerie in this dim light that I
was frightened and ran out again. Oh, it is so dreadfully
still in there!’

“\,‘Only that?’ said he, looking at me keenly.

“\,‘Why, what did you think?’ I asked.

“\,‘Why do you think that I lock this door?’

“\,‘I am sure that I do not know.’

“\,‘It is to keep people out who have no business there.
%%337
Do you see?’ He was still smiling in the most amiable
manner.

“\,‘I am sure if I had known -- ’

“\,‘Well, then, you know now. And if you ever put your
foot over that threshold again -- ’ here in an instant the
smile hardened into a grin of rage, and he glared down
at me with the face of a demon -- ‘I’ll throw you to the
mastiff.’

“I was so terrified that I do not know what I did. I suppose
that I must have rushed past him into my room. I remember
nothing until I found myself lying on my bed trembling
all over. Then I thought of you, Mr.~Holmes. I could not
live there longer without some advice. I was frightened of
the house, of the man, of the woman, of the servants, even of
the child. They were all horrible to me. If I could only
bring you down all would be well. Of course I might have
fled from the house, but my curiosity was almost as strong as
my fears. My mind was soon made up. I would send you a
wire. I put on my hat and cloak, went down to the office,
which is about half a mile from the house, and then returned,
feeling very much easier. A horrible doubt came into my
mind as I approached the door lest the dog might be loose,
but I remembered that Toller had drunk himself into a state
of insensibility that evening, and I knew that he was the only
one in the household who had any influence with the savage
creature, or who would venture to set him free. I slipped in
in safety, and lay awake half the night in my joy at the
thought of seeing you. I had no difficulty in getting leave to
come into Winchester this morning, but I must be back before
three o’clock, for Mr.~and Mrs.~Rucastle are going on a visit,
and will be away all the evening, so that I must look after the
child. Now I have told you all my adventures, Mr.~Holmes,
and I should be very glad if you could tell me what it all
means, and, above all, what I should do.”

Holmes and I had listened spellbound to this extraordinary
story. My friend rose now and paced up and down the room,
%%338
his hands in his pockets, and an expression of the most profound
gravity upon his face.

“Is Toller still drunk?” he asked.

“Yes. I heard his wife tell Mrs.~Rucastle that she could
do nothing with him.”

“That is well. And the Rucastles go out to-night?”

“Yes.”

“Is there a cellar with a good strong lock?”

“Yes, the wine-cellar.”

“You seem to me to have acted all through this matter like
a very brave and sensible girl, Miss Hunter. Do you think
that you could perform one more feat? I should not ask it of
you if I did not think you a quite exceptional woman.”

“I will try. What is it?”

“We shall be at the Copper Beeches by seven o’clock, my
friend and I. The Rucastles will be gone by that time, and
Toller will, we hope, be incapable. There only remains Mrs.
Toller, who might give the alarm. If you could send her into
the cellar on some errand, and then turn the key upon her,
you would facilitate matters immensely.”

“I will do it.”

“Excellent! We shall then look thoroughly into the affair.
Of course there is only one feasible explanation. You have
been brought there to personate some one, and the real person
is imprisoned in this chamber. That is obvious. As to who
this prisoner is, I have no doubt that it is the daughter, Miss
Alice Rucastle, if I remember right, who was said to have
gone to America. You were chosen, doubtless, as resembling
her in height, figure, and the color of your hair. Hers had
been cut off, very possibly in some illness through which she
has passed, and so, of course, yours had to be sacrificed also.
By a curious chance you came upon her tresses. The man in
the road was, undoubtedly, some friend of hers -- possibly her
fiancé -- and no doubt, as you wore the girl’s dress and were so
like her, he was convinced from your laughter, whenever he
saw you, and afterwards from your gesture, that Miss Rucastle
%%339
was perfectly happy, and that she no longer desired his attentions.
The dog is let loose at night to prevent him from
endeavoring to communicate with her. So much is fairly
clear. The most serious point in the case is the disposition
of the child.”

“What on earth has that to do with it?” I ejaculated.

“My dear Watson, you as a medical man are continually
gaining light as to the tendencies of a child by the study of the
parents. Don’t you see that the converse is equally valid. I
have frequently gained my first real insight into the character
of parents by studying their children. This child’s disposition
is abnormally cruel, merely for cruelty’s sake, and whether he
derives this from his smiling father, as I should suspect, or from
his mother, it bodes evil for the poor girl who is in their power.”

“I am sure that you are right, Mr.~Holmes,” cried our client.
“A thousand things come back to me which make me certain
that you have hit it. Oh, let us lose not an instant in bringing
help to this poor creature.”

“We must be circumspect, for we are dealing with a very
cunning man. We can do nothing until seven o’clock. At
that hour we shall be with you, and it will not be long before
we solve the mystery.”

We were as good as our word, for it was just seven when we
reached the Copper Beeches, having put up our trap at a wayside
public-house. The group of trees, with their dark leaves
shining like burnished metal in the light of the setting sun,
were sufficient to mark the house even had Miss Hunter not
been standing smiling on the door-step.

“Have you managed it?” asked Holmes.

A loud thudding noise came from somewhere down-stairs.
“That is Mrs.~Toller in the cellar,” said she. “Her husband
lies snoring on the kitchen rug. Here are his keys, which are
the duplicates of Mr.~Rucastle’s.”

“You have done well indeed!” cried Holmes, with enthusiasm.
“Now lead the way, and we shall soon see the end of
this black business.”
%%340

We passed up the stair, unlocked the door, followed on
down a passage, and found ourselves in front of the barricade
which Miss Hunter had described. Holmes cut the cord and
removed the transverse bar. Then he tried the various keys
in the lock, but without success. No sound came from within,
and at the silence Holmes’s face clouded over.

“I trust that we are not too late,” said he. “I think, Miss
Hunter, that we had better go in without you. Now, Watson,
put your shoulder to it, and we shall see whether we cannot
make our way in.”

It was an old rickety door, and gave at once before our
united strength. Together we rushed into the room. It was
empty. There was no furniture save a little pallet bed, a
small table, and a basketful of linen. The skylight above was
open, and the prisoner gone.

“There has been some villainy here,” said Holmes; “this
beauty has guessed Miss Hunter’s intentions, and has carried
his victim off.”

“But how?”

“Through the skylight. We shall soon see how he managed
it.” He swung himself up onto the roof. “Ah, yes,”
he cried; “here’s the end of a long light ladder against the
eaves. That is how he did it.”

“But it is impossible,” said Miss Hunter; “the ladder was
not there when the Rucastles went away.”

“He has come back and done it. I tell you that he is a
clever and dangerous man. I should not be very much surprised
if this were he whose step I hear now upon the stair.
I think, Watson, that it would be as well for you to have your
pistol ready.”

The words were hardly out of his mouth before a man
appeared at the door of the room, a very fat and burly man,
with a heavy stick in his hand. Miss Hunter screamed and
shrunk against the wall at the sight of him, but Sherlock
Holmes sprang forward and confronted him.

“You villain!” said he, “where’s your daughter?”
%%341

The fat man cast his eyes round, and then up at the open
skylight.

“It is for me to ask you that,” he shrieked, “you thieves!
Spies and thieves! I have caught you, have I? You are in
my power. I’ll serve you!” He turned and clattered down
the stairs as hard as he could go.

“He’s gone for the dog!” cried Miss Hunter.

“I have my revolver,” said I.

“Better close the front door,” cried Holmes, and we all
rushed down the stairs together. We had hardly reached the
hall when we heard the baying of a hound, and then a scream
of agony, with a horrible worrying sound which it was dreadful
to listen to. An elderly man with a red face and shaking
limbs came staggering out at a side door.

“My God!” he cried. “Some one has loosed the dog. It’s
not been fed for two days. Quick, quick, or it’ll be too late!”

Holmes and I rushed out and round the angle of the house,
with Toller hurrying behind us. There was the huge famished
brute, its black muzzle buried in Rucastle’s throat, while
he writhed and screamed upon the ground. Running up, I
blew its brains out, and it fell over with its keen white teeth
still meeting in the great creases of his neck. With much
labor we separated them, and carried him, living but horribly
mangled, into the house. We laid him upon the drawing-room
sofa, and, having despatched the sobered Toller to bear the
news to his wife, I did what I could to relieve his pain. We
were all assembled round him when the door opened, and a
tall, gaunt woman entered the room.

“Mrs.~Toller!” cried Miss Hunter.

“Yes, miss. Mr.~Rucastle let me out when he came back
before he went up to you. Ah, miss, it is a pity you didn’t let
me know what you were planning, for I would have told you
that your pains were wasted.”

“Ha!” said Holmes, looking keenly at her. “It is clear
that Mrs.~Toller knows more about this matter than any one
else.”
%%342

“Yes, sir, I do, and I am ready enough to tell what I
know.”

“Then, pray, sit down, and let us hear it, for there are
several points on which I must confess that I am still in the
dark.”

“I will soon make it clear to you,” said she; “and I’d have
done so before now if I could ha’ got out from the cellar. If
there’s police-court business over this, you’ll remember that I
was the one that stood your friend, and that I was Miss Alice’s
friend too.

“She was never happy at home, Miss Alice wasn’t, from the
time that her father married again. She was slighted like, and
had no say in anything; but it never really became bad for
her until after she met Mr.~Fowler at a friend’s house. As
well as I could learn, Miss Alice had rights of her own by will,
but she was so quiet and patient, she was, that she never said
a word about them, but just left everything in Mr.~Rucastle’s
hands. He knew he was safe with her; but when there was a
chance of a husband coming forward, who would ask for all that
the law would give him, then her father thought it time to put a
stop on it. He wanted her to sign a paper, so that whether she
married or not, he could use her money. When she wouldn’t
do it, he kept on worrying her until she got brain-fever, and
for six weeks was at death’s door. Then she got better at
last, all worn to a shadow, and with her beautiful hair cut off;
but that didn’t make no change in her young man, and he
stuck to her as true as man could be.”

“Ah,” said Holmes, “I think that what you have been
good enough to tell us makes the matter fairly clear, and that
I can deduce all that remains. Mr.~Rucastle then, I presume,
took to this system of imprisonment?”

“Yes, sir.”

“And brought Miss Hunter down from London in order to
get rid of the disagreeable persistence of Mr.~Fowler.”

“That was it, sir.”

“But Mr.~Fowler being a persevering man, as a good
%%343
seaman should be, blockaded the house, and, having met you,
succeeded by certain arguments, metallic or otherwise, in
convincing you that your interests were the same as his.”

“Mr.~Fowler was a very kind-spoken, free-handed gentleman,”
said Mrs.~Toller, serenely.

“And in this way he managed that your good man should
have no want of drink, and that a ladder should be ready at
the moment when your master had gone out.”

“You have it, sir, just as it happened.”

“I am sure we owe you an apology, Mrs.~Toller,” said
Holmes, “for you have certainly cleared up everything which
puzzled us. And here comes the country surgeon and Mrs.
Rucastle, so I think, Watson, that we had best escort Miss
Hunter back to Winchester, as it seems to me that our \textit{locus
standi} now is rather a questionable one.”

And thus was solved the mystery of the sinister house with
the copper beeches in front of the door. Mr.~Rucastle survived,
but was always a broken man, kept alive solely through
the care of his devoted wife. They still live with their old
servants, who probably know so much of Rucastle’s past life
that he finds it difficult to part from them. Mr.~Fowler and
Miss Rucastle were married, by special license, in Southampton
the day after their flight, and he is now the holder of a
Government appointment in the Island of Mauritius. As to
Miss Violet Hunter, my friend Holmes, rather to my
disappointment, manifested no further interest in her when once
she had ceased to be the centre of one of his problems, and
she is now the head of a private school at Walsall, where I
believe that she has met with considerable success.

THE END
%%344
