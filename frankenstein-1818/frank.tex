
\newcommand{\frDate}[1]{
\vspace{1em minus 1em}
{\small\raggedleft #1

}
\vspace{1em minus 0.5em}}

\newcommand{\frLetterSig}[1]{
\vspace{0em plus 4em}
\begin{flushright}
\vbox{#1}
\end{flushright}}

\frontmatter
\title{Frankenstein}
\author{Mary Shelley}
\date{1818}


\maketitle
\tableofcontents


%%003%%

\namedchapter{PREFACE}

\textsc{The} event on which this fiction is
founded has been supposed, by Dr.~%
Darwin, and some of the physiological
writers of Germany, as not
of impossible occurrence. I shall
not be supposed as according the
remotest degree of serious faith to
such an imagination; yet, in assuming
it as the basis of a work of
fancy, I have not considered myself
as merely weaving a series of supernatural
terrors. The event on
which the interest of the story depends
is exempt from the disadvantages
of a mere tale of spectres
or enchantment. It was recommended
by the novelty of the situations
which it developes; and,
however impossible as a physical
fact, affords a point of view to the
imagination for the delineating of
human passions more comprehensive
and commanding than any
which the ordinary relations of existing
events can yield.

I have thus endeavoured to preserve
the truth of the elementary
%%004%%
principles of human nature, while I
have not scrupled to innovate upon
their combinations. The \emph{Iliad}, the
tragic poetry of Greece, --- Shakespeare,
in the \emph{Tempest} and \emph{Midsummer
Night's Dream}, --- and most
especially Milton, in \emph{Paradise Lost},
conform to this rule; and the most
humble novelist, who seeks to confer
or receive amusement from his labours,
may, without presumption,
apply to prose fiction a licence, or
rather a rule, from the adoption of
which so many exquisite combinations
of human feeling have resulted
in the highest specimens of
poetry.

The circumstance on which my
story rests was suggested in casual
conversation. It was commenced,
partly as a source of amusement,
and partly as an expedient for exercising
any untried resources of
mind. Other motives were mingled
with these, as the work proceeded.
I am by no means indifferent
to the manner in which
whatever moral tendencies exist in
the sentiments or characters it contains
shall affect the reader; yet
my chief concern in this respect
has been limited to the avoiding
of the enervating effects of the novels
of the present day, and to the exhibition
of the amiableness of domestic
affection, and the excellence
of universal virtue. The opinions
which naturally spring from the
%%005%%
character and situation of the hero
are by no means to be conceived
as existing always in my own conviction;
nor is any inference justly
to be drawn from the following
pages as prejudicing any philosophical
doctrine of whatever kind.

It is a subject also of additional
interest to the author, that this
story was begun in the majestic
region where the scene is principally
laid, and in society which
cannot cease to be regretted. I
passed the summer of 1816 in the
environs of Geneva. The season
was cold and rainy, and in the
evenings we crowded around a
blazing wood fire, and occasionally
amused ourselves with some German
stories of ghosts, which happened
to fall into our hands.
These tales excited in us a playful
desire of imitation. Two other
friends (a tale from the pen of one
of whom would be far more acceptable
to the public than any thing
I can ever hope to produce) and
myself agreed to write each a story,
founded on some supernatural occurrence.

The weather, however, suddenly
became ser\-ene; and my two friends
left me on a journey among the
Alps, and lost, in the magnificent
scenes which they present, all memory
of their ghostly visions. The
following tale is the only one which
has been completed.
%%006%%
\mainmatter
\namedpart{Volume I}
\namedchapter{Letter I}

\emph{To Mrs.}~\textsc{Saville}, \emph{England}.

\frDate{St.~Petersburgh, Dec. 11th, 17---.}

\noindent\textsc{You} will rejoice to hear that no disaster
has accompanied the commencement
of an enterprise which you have
regarded with such evil forebodings.
I arrived here yesterday; and my first
task is to assure my dear sister of my
welfare, and increasing confidence in
the success of my undertaking.

I am already far north of London;
and as I walk in the streets of Petersburgh,
I feel a cold northern breeze
play upon my cheeks, which braces my
nerves, and fills me with delight. Do
you understand this feeling? This
breeze, which has travelled from the
regions towards which I am advancing,
gives me a foretaste of those icy climes.
Inspirited by this wind of promise, my
day dreams become more fervent and
vivid. I try in vain to be persuaded
that the pole is the seat of frost and desolation;
it ever presents itself to my
imagination as the region of beauty
and delight. There, Margaret, the
%%007%%
sun is for ever visible; its broad disk
just skirting the horizon, and diffusing
a perpetual splendour. There --- for
with your leave, my sister, I will put
some trust in preceding navigators --- there
snow and frost are banished; and,
sailing over a calm sea, we may be
wafted to a land surpassing in wonders
and in beauty every region hitherto discovered
on the habitable globe. Its
productions and features may be without
example, as the phænomena of the
heavenly bodies undoubtedly are in
those undiscovered solitudes. What
may not be expected in a country of
eternal light? I may there discover
%%008%%
the wondrous power which attracts the
needle; and may regulate a thousand
celestial observations, that require only
this voyage to render their seeming
eccentricities consistent for ever. I
shall satiate my ardent curiosity with
the sight of a part of the world never
before visited, and may tread a land
never before imprinted by the foot of
man. These are my enticements, and
they are sufficient to conquer all fear of
danger or death, and to induce me to
commence this laborious voyage with
the joy a child feels when he embarks
in a little boat, with his holiday mates,
on an expedition of discovery up his
native river. But, supposing all these
conjectures to be false, you cannot contest
the inestimable benefit which I shall
confer on all mankind to the last generation,
by discovering a passage near
the pole to those countries, to reach
which at present so many months are
requisite; or by ascertaining the secret
of the magnet, which, if at all possible,
can only be effected by an undertaking
such as mine.

These reflections have dispelled the
agitation with which I began my letter,
and I feel my heart glow with an enthusiasm
which elevates me to heaven;
for nothing contributes so much to
tranquillize the mind as a steady purpose, --- a
point on which the soul may fix
its intellectual eye. This expedition
has been the favourite dream of my
early years. I have read with ardour
the accounts of the various voyages
which have been made in the prospect
of arriving at the North Pacific Ocean
through the seas which surround the
%%009%%
pole. You may remember, that a
history of all the voyages made for
purposes of discovery composed the
whole of our good uncle Thomas's library.
My education was neglected,
yet I was passionately fond of reading.
These volumes were my study
day and night, and my familiarity
with them increased that regret which
I had felt, as a child, on learning that
my father's dying injunction had forbidden
my uncle to allow me to embark
in a sea-faring life.

These visions faded when I perused,
for the first time, those poets whose
effusions entranced my soul, and lifted
it to heaven. I also became a poet,
and for one year lived in a Paradise of
my own creation; I imagined that I
also might obtain a niche in the temple
where the names of Homer and Shakespeare
are consecrated. You are well
acquainted with my failure, and how
%%010%%
heavily I bore the disappointment.
But just at that time I inherited the
fortune of my cousin, and my thoughts
were turned into the channel of their
earlier bent.

Six years have passed since I resolved
on my present undertaking. I can, even
now, remember the hour from which I
dedicated myself to this great enterprise.
I commenced by inuring my body to
hardship. I accompanied the whale-fishers
on several expeditions to the
North Sea; I voluntarily endured cold,
famine, thirst, and want of sleep; I
often worked harder than the common
sailors during the day, and devoted my
nights to the study of mathematics,
the theory of medicine, and those
branches of physical science from
which a naval adventurer might derive
the greatest practical advantage. Twice
%%011%%
I actually hired myself as an under-mate
in a Greenland whaler, and acquitted
myself to admiration. I must
own I felt a little proud, when my captain
offered me the second dignity in
the vessel, and entreated me to remain
with the greatest earnestness; so valuable
did he consider my services.

And now, dear Margaret, do I not
deserve to accomplish some great purpose.
My life might have been passed
in ease and luxury; but I preferred
glory to every enticement that wealth
placed in my path. Oh, that some encouraging
voice would answer in the
affirmative! My courage and my resolution
is firm; but my hopes fluctuate,
and my spirits are often depressed. I
am about to proceed on a long and difficult
voyage; the emergencies of which
will demand all my fortitude: I am required
not only to raise the spirits of
others, but sometimes to sustain my
own, when their's are failing.

This is the most favourable period
for travelling in Russia. They fly
quickly over the snow in their sledges;
the motion is pleasant, and, in my opinion,
far more agreeable than that of an
English stage-coach. The cold is not
excessive, if you are wrapt in furs, a
dress which I have already adopted;
%%012%%
for there is a great difference between
walking the deck and remaining seated
motionless for hours, when no exercise
prevents the blood from actually freezing
in your veins. I have no ambition
to lose my life on the post-road between
St.~Petersburgh and Archangel.

I shall depart for the latter town in
a fortnight or three weeks; and my intention
is to hire a ship there, which
can easily be done by paying the insurance
for the owner, and to engage as
many sailors as I think necessary among
those who are accustomed to the whale-fishing.
I do not intend to sail until
the month of June: and when shall I
return? Ah, dear sister, how can I
answer this question? If I succeed,
many, many months, perhaps years,
will pass before you and I may meet.
If I fail, you will see me again soon, or
never.

Farewell, my dear, excellent, Margaret.
Heaven shower down blessings
on you, and save me, that I may again
and again testify my gratitude for all
your love and kindness.

\frLetterSig{
Your affectionate brother,\\
R.~\textsc{Walton}.}

%%013%%

\namedchapter{Letter II}

\emph{To Mrs.}~\textsc{Saville}, \emph{England}.

\frDate{Archangel, 28th March, 17---.}

\noindent\textsc{How} slowly the time passes here, encompassed
as I am by frost and snow;
yet a second step is taken towards my
enterprise. I have hired a vessel, and
am occupied in collecting my sailors;
those whom I have already engaged appear
to be men on whom I can depend,
and are certainly possessed of dauntless
courage.

But I have one want which I have
never yet been able to satisfy; and
the absence of the object of which I
now feel as a most severe evil. I have
no friend, Margaret: when I am glowing
with the enthusiasm of success,
there will be none to participate my
joy; if I am assailed by disappointment,
no one will endeavour to sustain
me in dejection. I shall commit my
thoughts to paper, it is true; but that
is a poor medium for the communication
of feeling. I desire the company
of a man who could sympathize with
me; whose eyes would reply to mine.
You may deem me romantic, my dear
sister, but I bitterly feel the want of a
friend. I have no one near me, gentle
yet courageous, possessed of a cultivated
as well as of a capacious mind,
whose tastes are like my own, to
%%014%%
approve or amend my plans. How would
such a friend repair the faults of your
poor brother! I am too ardent in execution,
and too impatient of difficulties.
But it is a still greater evil to me
that I am self-educated: for the first
fourteen years of my life I ran wild on
a common, and read nothing but our
uncle Thomas's books of voyages. At
that age I became acquainted with the
celebrated poets of our own country;
but it was only when it had ceased to be
in my power to derive its most important
benefits from such a conviction, that I
perceived the necessity of becoming
acquainted with more languages than
that of my native country. Now I am
twenty-eight, and am in reality more
illiterate than many school-boys of fifteen.
It is true that I have thought
more, and that my day dreams are
more extended and magnificent; but
they want (as the painters call it)
\emph{keeping}; and I greatly need a friend
who would have sense enough not to
despise me as romantic, and affection
enough for me to endeavour to regulate
my mind.

Well, these are useless complaints; I
shall certainly find no friend on the
wide ocean, nor even here in Archangel,
among merchants and seamen. Yet
some feelings, unallied to the dross of
human nature, beat even in these rugged
bosoms. My lieutenant, for instance,
is a man of wonderful courage
and enterprise; he is madly desirous of
glory. He is an Englishman, and in
the midst of national and professional
prejudices, unsoftened by cultivation,
retains some of the noblest endowments
%%015%%
of humanity. I first became acquainted
with him on board a whale vessel: finding
that he was unemployed in this
city, I easily engaged him to assist in
my enterprise.

The master is a person of an excellent
disposition, and is remarkable in
the ship for his gentleness, and the
mildness of his discipline. He is, indeed,
of so amiable a nature, that he
will not hunt (a favourite, and almost
the only amusement here), because he
cannot endure to spill blood. He is,
moreover, heroically generous. Some
years ago he loved a young Russian
lady, of moderate fortune; and having
amassed a considerable sum in prize-money,
the father of the girl consented
to the match. He saw his mistress
once before the destined ceremony; but
she was bathed in tears, and, throwing
herself at his feet, entreated him to
spare her, confessing at the same time
that she loved another, but that he was
poor, and that her father would never
consent to the union. My generous
friend reassured the suppliant, and on
being informed of the name of her lover
instantly abandoned his pursuit. He had
already bought a farm with his money,
on which he had designed to pass the remainder
of his life; but he bestowed
the whole on his rival, together with
the remains of his prize-money to purchase
stock, and then himself solicited
the young woman's father to consent
to her marriage with her lover. But
the old man decidedly refused, thinking
himself bound in honour to
my friend; who, when he found the
father inexorable, quitted his country,
%%016%%
nor returned until he heard that his
former mistress was married according
to her inclinations. ``What a noble
fellow!'' you will exclaim. He is so;
but then he has passed all his life on
board a vessel, and has scarcely an
idea beyond the rope and the shroud.

But do not suppose that, because I
complain a little, or because I can conceive
a consolation for my toils which I
may never know, that I am wavering
in my resolutions. Those are as fixed
as fate; and my voyage is only now
delayed until the weather shall permit
my embarkation. The winter has been
dreadfully severe; but the spring promises
well, and it is considered as a remarkably
early season; so that, perhaps,
I may sail sooner than I expected.
I shall do nothing rashly;
you know me sufficiently to confide in
my prudence and considerateness whenever
the safety of others is committed
to my care.

I cannot describe to you my sensations
on the near prospect of my undertaking.
It is impossible to communicate
to you a conception of the trembling
sensation, half pleasurable and
half fearful, with which I am preparing
to depart. I am going to unexplored
regions, to ``the land of mist and
snow;'' but I shall kill no albatross,
therefore do not be alarmed for my
safety.

Shall I meet you again, after having
traversed immense seas, and returned
by the most southern cape of Africa or
America? I dare not expect such success,
yet I cannot bear to look on the
%%017%%
reverse of the picture. Continue to
write to me by every opportunity: I
may receive your letters (though the
chance is very doubtful) on some occasions
when I need them most to support
my spirits. I love you very tenderly.
Remember me with affection, should
you never hear from me again.

\frLetterSig{
Your affectionate brother,\\
\textsc{Robert Walton}.}

%%018%%

\namedchapter{Letter III}

\emph{To Mrs.}~\textsc{Saville}, \emph{England}.

\frDate{July 7th, 17---.}

\noindent\textsc{my dear sister},
\medskip

\noindent\textsc{I write} a few lines in haste, to say
that I am safe, and well advanced on
my voyage. This letter will reach
England by a merchant-man now on
its homeward voyage from Archangel;
more fortunate than I, who may not see
my native land, perhaps, for many
years. I am, however, in good spirits:
my men are bold, and apparently firm
of purpose; nor do the floating sheets
of ice that continually pass us, indicating
the dangers of the region towards
which we are advancing, appear
to dismay them. We have already
reached a very high latitude; but it is
the height of summer, and although
not so warm as in England, the southern
gales, which blow us speedily towards
those shores which I so ardently
desire to attain, breathe a degree of
renovating warmth which I had not
expected.

No incidents have hitherto befallen
us, that would make a figure in a letter.
One or two stiff gales, and the breaking
of a mast, are accidents which experienced
navigators scarcely remember
to record; and I shall be well
%%019%%
content, if nothing worse happen to us
during our voyage.

Adieu, my dear Margaret. Be assured,
that for my own sake, as well as
your's, I will not rashly encounter danger.
I will be cool, persevering, and
prudent.

Remember me to all my English
friends.

\frLetterSig{
Most affectionately yours,\\
R. W.}
%%020%%

\namedchapter{Letter IV}

\emph{To Mrs.}~\textsc{Saville}, \emph{England}.

\frDate{August 5th, 17---.}

\noindent\textsc{So} strange an accident has happened
to us, that I cannot forbear recording
it, although it is very probable that
you will see me before these papers
can come into your possession.

Last Monday (July 31st), we were
nearly surrounded by ice, which closed
in the ship on all sides, scarcely leaving
her the sea room in which she
floated. Our situation was somewhat
dangerous, especially as we were compassed
round by a very thick fog. We
accordingly lay to, hoping that some
change would take place in the atmosphere
and weather.

About two o'clock the mist cleared
away, and we beheld, stretched out in
every direction, vast and irregular
plains of ice, which seemed to have no
end. Some of my comrades groaned,
and my own mind began to grow
watchful with anxious thoughts, when
a strange sight suddenly attracted our
attention, and diverted our solicitude
from our own situation. We perceived
a low carriage, fixed on a sledge and
drawn by dogs, pass on towards the
north, at the distance of half a mile:
a being which had the shape of a
man, but apparently of gigantic stature,
sat in the sledge, and guided
%%021%%
the dogs. We watched the rapid progress
of the traveller with our telescopes,
until he was lost among the
distant inequalities of the ice.

This appearance excited our unqualified
wonder. We were, as we believed,
many hundred miles from any
land; but this apparition seemed to
denote that it was not, in reality, so
distant as we had supposed. Shut
in, however, by ice, it was impossible
to follow his track, which we
had observed with the greatest attention.

About two hours after this occurrence,
we heard the ground sea; and
before night the ice broke, and freed
our ship. We, however, lay to until
the morning, fearing to encounter in
the dark those large loose masses
which float about after the breaking up
of the ice. I profited of this time to
rest for a few hours.

In the morning, however, as soon
as it was light, I went upon deck, and
found all the sailors busy on one side
of the vessel, apparently talking to
some one in the sea. It was, in fact, a
sledge, like that we had seen before,
which had drifted towards us in the
night, on a large fragment of ice.
Only one dog remained alive; but
there was a human being within it,
whom the sailors were persuading to
enter the vessel. He was not, as the
other traveller seemed to be, a savage
inhabitant of some undiscovered island,
but an European. When I appeared
on deck, the master said, ``Here is our
captain, and he will not allow you to
perish on the open sea.''
%%022%%

On perceiving me, the stranger addressed
me in English, although with
a foreign accent. ``Before I come on
board your vessel,'' said he, ``will you
have the kindness to inform me whither
you are bound?''

You may conceive my astonishment
on hearing such a question addressed
to me from a man on the brink of destruction,
and to whom I should have
supposed that my vessel would have
been a resource which he would not
have exchanged for the most precious
wealth the earth can afford. I replied,
however, that we were on a voyage
of discovery towards the northern
pole.

Upon hearing this he appeared satisfied,
and consented to come on board.
Good God! Margaret, if you had seen
the man who thus capitulated for his
safety, your surprise would have been
boundless. His limbs were nearly frozen,
and his body dreadfully emaciated
by fatigue and suffering. I never saw a
man in so wretched a condition. We
attempted to carry him into the cabin;
but as soon as he had quitted the fresh
air, he fainted. We accordingly
brought him back to the deck, and
restored him to animation by rubbing
him with brandy, and forcing him to
swallow a small quantity. As soon as
he shewed signs of life, we wrapped
him up in blankets, and placed him
near the chimney of the kitchen-stove.
By slow degrees he recovered, and ate
a little soup, which restored him wonderfully.

Two days passed in this manner before
he was able to speak; and I often
%%023%%
feared that his sufferings had deprived
him of understanding. When he had
in some measure recovered, I removed
him to my own cabin, and attended on
him as much as my duty would permit.
I never saw a more interesting creature:
his eyes have generally an expression
of wildness, and even madness;
but there are moments when, if
any one performs an act of kindness
towards him, or does him any the most
trifling service, his whole countenance
is lighted up, as it were, with a beam
of benevolence and sweetness that I
never saw equalled. But he is generally
melancholy and despairing; and
sometimes he gnashes his teeth, as if
impatient of the weight of woes that
oppresses him.

When my guest was a little recovered,
I had great trouble to keep off
the men, who wished to ask him a
thousand questions; but I would not
allow him to be tormented by their
idle curiosity, in a state of body and
mind whose restoration evidently depended
upon entire repose. Once,
however, the lieutenant asked, Why he
had come so far upon the ice in so
strange a vehicle?

His countenance instantly assumed
an aspect of the deepest gloom; and he
replied, ``To seek one who fled from
me.''

``And did the man whom you pursued
travel in the same fashion?''

``Yes.''

``Then I fancy we have seen him;
for, the day before we picked you up,
we saw some dogs drawing a sledge,
with a man in it, across the ice.''
%%024%%

This aroused the stranger's attention;
and he asked a multitude of questions
concerning the route which the
dæmon, as he called him, had pursued.
Soon after, when he was alone with me,
he said, ``I have, doubtless, excited your
curiosity, as well as that of these good
people; but you are too considerate to
make inquiries.''

``Certainly; it would indeed be very
impertinent and inhuman in me to
trouble you with any inquisitiveness of
mine.''

``And yet you rescued me from a
strange and perilous situation; you
have benevolently restored me to life.''

Soon after this he inquired, if I
thought that the breaking up of the ice
had destroyed the other sledge? I replied,
that I could not answer with any
degree of certainty; for the ice had
not broken until near midnight, and
the traveller might have arrived at a
place of safety before that time; but of
this I could not judge.

From this time the stranger seemed
very eager to be upon deck, to watch
for the sledge which had before appeared;
but I have persuaded him to remain
in the cabin, for he is far too weak to
sustain the rawness of the atmosphere.
But I have promised that some one
should watch for him, and give him
instant notice if any new object should
appear in sight.

Such is my journal of what relates
to this strange occurrence up to the
present day. The stranger has gradually
improved in health, but is very
silent, and appears uneasy when any
one except myself enters his cabin.
%%025%%
Yet his manners are so conciliating
and gentle, that the sailors are all interested
in him, although they have
had very little communication with
him. For my own part, I begin to
love him as a brother; and his constant
and deep grief fills me with sympathy
and compassion. He must have
been a noble creature in his better
days, being even now in wreck so attractive
and amiable.

I said in one of my letters, my dear
Margaret, that I should find no friend
on the wide ocean; yet I have found a
man who, before his spirit had been
broken by misery, I should have been
happy to have possessed as the brother
of my heart.

I shall continue my journal concerning
the stranger at intervals, should I
have any fresh incidents to record.

\frDate{August 13th, 17---.}

My affection for my guest increases
every day. He excites at once my admiration
and my pity to an astonishing
degree. How can I see so noble a
creature destroyed by misery without
feeling the most poignant grief? He is
so gentle, yet so wise; his mind is so
cultivated; and when he speaks, although
his words are culled with the choicest
art, yet they flow with rapidity and unparalleled
eloquence.

He is now much recovered from his
illness, and is continually on the deck,
apparently watching for the sledge that
preceded his own. Yet, although unhappy,
he is not so utterly occupied by
his own misery, but that he interests
himself deeply in the employments of
%%026%%
others. He has asked me many questions
concerning my design; and I
have related my little history frankly
to him. He appeared pleased with the
confidence, and suggested several alterations
in my plan, which I shall
find exceedingly useful. There is no
pedantry in his manner; but all he
does appears to spring solely from the
interest he instinctively takes in the
welfare of those who surround him. He
is often overcome by gloom, and then
he sits by himself, and tries to overcome
all that is sullen or unsocial in his humour.
These paroxysms pass from him
like a cloud from before the sun,
though his dejection never leaves him.
I have endeavoured to win his confidence;
and I trust that I have succeeded.
One day I mentioned to him
the desire I had always felt of finding
a friend who might sympathize with
me, and direct me by his counsel. I
said, I did not belong to that class of
men who are offended by advice. ``I am
self-educated, and perhaps I hardly rely
sufficiently upon my own powers. I wish
therefore that my companion should
be wiser and more experienced than
myself, to confirm and support me;
nor have I believed it impossible to
find a true friend.''

``I agree with you,'' replied the stranger,
``in believing that friendship is
not only a desirable, but a possible acquisition.
I once had a friend, the
most noble of human creatures, and am
entitled, therefore, to judge respecting
friendship. You have hope, and the
world before you, and have no cause
for despair. But I ------ I have lost
%%027%%
every thing, and cannot begin life
anew.''

As he said this, his countenance became
expressive of a calm settled grief,
that touched me to the heart. But
he was silent, and presently retired to
his cabin.

Even broken in spirit as he is, no
one can feel more deeply than he does
the beauties of nature. The starry
sky, the sea, and every sight afforded
by these wonderful regions, seems still
to have the power of elevating his soul
from earth. Such a man has a double
existence: he may suffer misery, and
be overwhelmed by disappointments;
yet when he has retired into himself, he
will be like a celestial spirit, that has
a halo around him, within whose circle
no grief or folly ventures.

Will you laugh at the enthusiasm I
express concerning this divine wanderer?
If you do, you must have certainly
lost that simplicity which was
once your characteristic charm. Yet,
if you will, smile at the warmth of my
expressions, while I find every day new
causes for repeating them.

\frDate{August 19th, 17---.}

Yesterday the stranger said to me,
``You may easily perceive, Captain
Walton, that I have suffered great and
unparalleled misfortunes. I had determined,
once, that the memory of these
evils should die with me; but you
have won me to alter my determination.
You seek for knowledge and
wisdom, as I once did; and I ardently
hope that the gratification of your
wishes may not be a serpent to sting
%%028%%
you, as mine has been. I do not know
that the relation of my misfortunes will
be useful to you, yet, if you are inclined,
listen to my tale. I believe that the
strange incidents connected with it will
afford a view of nature, which may
enlarge your faculties and understanding.
You will hear of powers and
occurrences, such as you have been
accustomed to believe impossible: but
I do not doubt that my tale conveys in
its series internal evidence of the truth
of the events of which it is composed.''

You may easily conceive that I was
much gratified by the offered communication;
yet I could not endure that
he should renew his grief by a recital
of his misfortunes. I felt the greatest
eagerness to hear the promised narrative,
partly from curiosity, and partly
from a strong desire to ameliorate his
fate, if it were in my power. I
expressed these feelings in my answer.

``I thank you,'' he replied, ``for
your sympathy, but it is useless; my
fate is nearly fulfilled. I wait but for
one event, and then I shall repose in
peace. I understand your feeling,''
continued he, perceiving that I wished
to interrupt him; ``but you are mistaken,
my friend, if thus you will allow
me to name you; nothing can alter my
destiny: listen to my history, and you
will perceive how irrevocably it is determined.''

He then told me, that he would commence
his narrative the next day when
I should be at leisure. This promise
drew from me the warmest thanks. I
have resolved every night, when I am
not engaged, to record, as nearly as
%%029%%
possible in his own words, what he has
related during the day. If I should
be engaged, I will at least make notes.
This manuscript will doubtless afford
you the greatest pleasure: but to me,
who know him, and who hear it from
his own lips, with what interest and
sympathy shall I read it in some future
day!
%%030%%

\namedchapter{Chapter I}

\textsc{I am} by birth a Genevese; and my
family is one of the most distinguished
of that republic. My ancestors had
been for many years counsellors and
syndics; and my father had filled several
public situations with honour and
reputation. He was respected by all
who knew him for his integrity and
indefatigable attention to public business.
He passed his younger days
perpetually occupied by the affairs of
his country; and it was not until the
decline of life that he thought of marrying,
and bestowing on the state sons
who might carry his virtues and his
name down to posterity.

As the circumstances of his marriage
illustrate his character, I cannot refrain
from relating them. One of his most
intimate friends was a merchant, who,
from a flourishing state, fell, through
numerous mischances, into poverty.
This man, whose name was Beaufort,
was of a proud and unbending disposition,
and could not bear to live in poverty
and oblivion in the same country
where he had formerly been distinguished
for his rank and magnificence.
%%031%%
Having paid his debts, therefore, in the
most honourable manner, he retreated
with his daughter to the town of Lucerne,
where he lived unknown and in
wretchedness. My father loved Beaufort
with the truest friendship, and was
deeply grieved by his retreat in these
unfortunate circumstances. He grieved
also for the loss of his society, and
resolved to seek him out and endeavour
to persuade him to begin the world
again through his credit and assistance.

Beaufort had taken effectual measures
to conceal himself; and it was
ten months before my father discovered
his abode. Overjoyed at this discovery,
he hastened to the house, which was
situated in a mean street, near the
Reuss. But when he entered, misery
and despair alone welcomed him. Beaufort
had saved but a very small sum of
money from the wreck of his fortunes;
but it was sufficient to provide him
with sustenance for some months, and
in the mean time he hoped to procure
some respectable employment in a merchant's
house. The interval was consequently
spent in inaction; his grief
only became more deep and rankling,
when he had leisure for reflection; and
at length it took so fast hold of his
mind, that at the end of three months
he lay on a bed of sickness, incapable
of any exertion.

His daughter attended him with the
greatest tenderness; but she saw with
despair that their little fund was rapidly
decreasing, and that there was no other
prospect of support. But Caroline
Beaufort possessed a mind of an
%%032%%
uncommon mould; and her courage rose
to support her in her adversity. She
procured plain work; she plaited straw;
and by various means contrived to earn
a pittance scarcely sufficient to support
life.

Several months passed in this manner.
Her father grew worse; her time
was more entirely occupied in attending
him; her means of subsistence decreased;
and in the tenth month her
father died in her arms, leaving her an
orphan and a beggar. This last blow
overcame her; and she knelt by Beaufort's
coffin, weeping bitterly, when my
father entered the chamber. He came
like a protecting spirit to the poor girl,
who committed herself to his care, and
after the interment of his friend he
conducted her to Geneva, and placed
her under the protection of a relation.
Two years after this event Caroline
became his wife.

When my father became a husband
and a parent, he found his time so occupied
by the duties of his new situation,
that he relinquished many of his public
employments, and devoted himself to
the education of his children. Of these
I was the eldest, and the destined successor
to all his labours and utility.
No creature could have more tender
parents than mine. My improvement
and health were their constant care,
especially as I remained for several years
their only child. But before I continue
my narrative, I must record an incident
which took place when I was four years
of age.

My father had a sister, whom he
tenderly loved, and who had married
%%033%%
early in life an Italian gentleman. Soon
after her marriage, she had accompanied
her husband into her native
country, and for some years my father
had very little communication with her.
About the time I mentioned she died;
and a few months afterwards he received
a letter from her husband, acquainting
him with his intention of
marrying an Italian lady, and requesting
my father to take charge of the infant
Elizabeth, the only child of his
deceased sister. ``It is my wish,'' he
said, ``that you should consider her as
your own daughter, and educate her
thus. Her mother's fortune is secured
to her, the documents of which I will
commit to your keeping. Reflect upon
this proposition; and decide whether
you would prefer educating your niece
yourself to her being brought up by a
stepmother.''

My father did not hestitate, and immediately
went to Italy, that he might
accompany the little Elizabeth to her
future home. I have often heard my
mother say, that she was at that time the
most beautiful child she had ever seen,
and shewed signs even then of a gentle
and affectionate disposition. These
indications, and a desire to bind as
closely as possible the ties of domestic
love, determined my mother to consider
Elizabeth as my future wife; a design
which she never found reason to repent.

From this time Elizabeth Lavenza
became my playfellow, and, as we grew
older, my friend. She was docile and
good tempered, yet gay and playful as
a summer insect. Although she was
lively and animated, her feelings were
%%034%%
strong and deep, and her disposition
uncommonly affectionate. No one
could better enjoy liberty, yet no one
could submit with more grace than she
did to constraint and caprice. Her
imagination was luxuriant, yet her capability
of application was great. Her
person was the image of her mind; her
hazel eyes, although as lively as a
bird's, possessed an attractive softness.
Her figure was light and airy; and,
though capable of enduring great fatigue,
she appeared the most fragile
creature in the world. While I admired
her understanding and fancy, I
loved to tend on her, as I should on a
favourite animal; and I never saw so
much grace both of person and mind
united to so little pretension.

Every one adored Elizabeth. If the
servants had any request to make, it was
always through her intercession. We
were strangers to any species of disunion
and dispute; for although there
was a great dissimilitude in our characters,
there was an harmony in that
very dissimilitude. I was more calm
and philosophical than my companion;
yet my temper was not so yielding.
My application was of longer endurance;
but it was not so severe whilst it
endured. I delighted in investigating
the facts relative to the actual world;
she busied herself in following the
aërial creations of the poets. The world
was to me a secret, which I desired to
discover; to her it was a vacancy,
which she sought to people with imaginations
of her own.

My brothers were considerably younger
than myself; but I had a friend in
%%035%%
one of my schoolfellows, who compensated
for this deficiency. Henry Clerval
was the son of a merchant of Geneva,
an intimate friend of my father.
He was a boy of singular talent and
fancy. I remember, when he was nine
years old, he wrote a fairy tale, which
was the delight and amazement of all
his companions. His favourite study
consisted in books of chivalry and romance;
and when very young, I can
remember, that we used to act plays
composed by him out of these favourite
books, the principal characters of
which were Orlando, Robin Hood,
Amadis, and St.~George.

No youth could have passed more
happily than mine. My parents were
indulgent, and my companions amiable.
Our studies were never forced; and by
some means we always had an end
placed in view, which excited us to ardour
in the prosecution of them. It
was by this method, and not by emulation,
that we were urged to application.
Elizabeth was not incited to apply herself
to drawing, that her companions
might not outstrip her; but through the
desire of pleasing her aunt, by the
representation of some favourite scene
done by her own hand. We learned
Latin and English, that we might read
the writings in those languages; and
so far from study being made odious to
us through punishment, we loved application,
and our amusements would
have been the labours of other children.
Perhaps we did not read so
many books, or learn languages so
quickly, as those who are disciplined
according to the ordinary methods; but
%%036%%
what we learned was impressed the
more deeply on our memories.

In this description of our domestic
circle I include Henry Clerval; for he
was constantly with us. He went to
school with me, and generally passed
the afternoon at our house; for being
an only child, and destitute of companions
at home, his father was well
pleased that he should find associates at
our house; and we were never completely
happy when Clerval was absent.

I feel pleasure in dwelling on the recollections
of childhood, before misfortune
had tainted my mind, and changed
its bright visions of extensive usefulness
into gloomy and narrow reflections
upon self. But, in drawing the picture
of my early days, I must not omit to
record those events which led, by insensible
steps to my after tale of misery:
for when I would account to myself for
the birth of that passion, which afterwards
ruled my destiny, I find it arise,
like a mountain river, from ignoble
and almost forgotten sources; but,
swelling as it proceeded, it became the
torrent which, in its course, has swept
away all my hopes and joys.

Natural philosophy is the genius that
has regulated my fate; I desire therefore,
in this narration, to state those facts
which led to my predilection for that
science. When I was thirteen years of
age, we all went on a party of pleasure
to the baths near Thonon: the inclemency
of the weather obliged us to remain
a day confined to the inn. In
this house I chanced to find a volume
of the works of Cornelius Agrippa. I
opened it with apathy; the theory which
%%037%%
he attempts to demonstrate, and the
wonderful facts which he relates, soon
changed this feeling into enthusiasm.
A new light seemed to dawn upon my
mind; and, bounding with joy, I communicated
my discovery to my father.
I cannot help remarking here the many
opportunities instructors possess of directing
the attention of their pupils to
useful knowledge, which they utterly
neglect. My father looked carelessly
at the title-page of my book, and said,
``Ah! Cornelius Agrippa! My dear
Victor, do not waste your time upon
this; it is sad trash.''

If, instead of this remark, my father
had taken the pains, to explain to me,
that the principles of Agrippa had been
entirely exploded, and that a modern
system of science had been introduced,
which possessed much greater powers
than the ancient, because the powers
of the latter were chimerical, while
those of the former were real and practical;
under such circumstances, I
should certainly have thrown Agrippa
aside, and, with my imagination warmed
as it was, should probably have applied
myself to the more rational theory of
chemistry which has resulted from
modern discoveries. It is even possible,
that the train of my ideas would
never have received the fatal impulse
that led to my ruin. But the cursory
glance my father had taken of my volume
by no means assured me that he
was acquainted with its contents; and
I continued to read with the greatest
avidity.

When I returned home, my first care
was to procure the whole works of this
%%038%%
author, and afterwards of Paracelsus
and Albertus Magnus. I read and
studied the wild fancies of these writers
with delight; they appeared to me treasures
known to few beside myself;
and although I often wished to communicate
these secret stores of knowledge
to my father, yet his indefinite
censure of my favourite Agrippa always
withheld me. I disclosed my discoveries
to Elizabeth, therefore, under a
promise of strict secrecy; but she did
not interest herself in the subject,
and I was left by her to pursue my
studies alone.

It may appear very strange, that a
disciple of Albertus Magnus should
arise in the eighteenth century; but
our family was not scientifical, and I
had not attended any of the lectures
given at the schools of Geneva. My
dreams were therefore undisturbed by
reality; and I entered with the greatest
diligence into the search of the philosopher's
stone and the elixir of life.
%%039%%
But the latter obtained my most undivided
attention: wealth was an inferior
object; but what glory would attend
the discovery, if I could banish disease
from the human frame, and render
man invulnerable to any but a violent
death!

Nor were these my only visions. The
raising of ghosts or devils was a promise
liberally accorded by my favourite
authors, the fulfilment of which I most
eagerly sought; and if my incantations
were always unsuccessful, I attributed
the failure rather to my own inexperience
and mistake, than to a want of
skill or fidelity in my instructors.

The natural phænomena that take
place every day before our eyes did not
escape my examinations. Distillation,
and the wonderful effects of steam, processes
of which my favourite authors
were utterly ignorant, excited my astonishment;
but my utmost wonder was
engaged by some experiments on an air-pump,
which I saw employed by a gentleman
whom we were in the habit of
visiting.

The ignorance of the early philosophers
on these and several other points
served to decrease their credit with me:
but I could not entirely throw them
aside, before some other system should
occupy their place in my mind.

When I was about fifteen years old,
we had retired to our house near Belrive,
when we witnessed a most violent
and terrible thunder-storm. It advanced
from behind the mountains of Jura; and
the thunder burst at once with frightful
loudness from various quarters of the
heavens. I remained, while the storm
%%040%%
lasted, watching its progress with curiosity
and delight. As I stood at the
door, on a sudden I beheld a stream of
fire issue from an old and beautiful oak,
which stood about twenty yards from
our house; and so soon as the dazzling
light vanished, the oak had disappeared,
and nothing remained but a blasted
stump. When we visited it the next
morning, we found the tree shattered
in a singular manner. It was not splintered
by the shock, but entirely reduced
to thin ribbands of wood. I never beheld
any thing so utterly destroyed.

The catastrophe of this tree excited
my extreme astonishment; and I eagerly
inquired of my father the nature
and origin of thunder and lightning.
He replied, ``Electricity;'' describing
at the same time the various effects of
that power. He constructed a small
electrical machine, and exhibited a few
experiments; he made also a kite, with
%%041%%
a wire and string, which drew down
that fluid from the clouds.

This last stroke completed the overthrow
of Cornelius Agrippa, Albertus
Magnus, and Paracelsus, who had so
long reigned the lords of my imagination.
But by some fatality I did not
feel inclined to commence the study of
any modern system; and this disinclination
was influenced by the following
circumstance.

My father expressed a wish that I
should attend a course of lectures upon
natural philosophy, to which I cheerfully
consented. Some accident prevented
my attending these lectures
until the course was nearly finished.
The lecture, being therefore one of the
last, was entirely incomprehensible to
me. The professor discoursed with
the greatest fluency of potassium and
boron, of sulphates and oxyds, terms
to which I could affix no idea; and I
%%042%%
became disgusted with the science of
natural philosophy, although I still read
Pliny and Buffon with delight, authors,
in my estimation, of nearly equal interest
and utility.

My occupations at this age were
principally the mathematics, and most
of the branches of study appertaining
to that science. I was busily employed
in learning languages; Latin was already
familiar to me, and I began to
read some of the easiest Greek authors
without the help of a lexicon. I also
perfectly understood English and German.
This is the list of my accomplishments
at the age of seventeen;
and you may conceive that my hours
were fully employed in acquiring and
maintaining a knowledge of this various
literature.

Another task also devolved upon me,
when I became the instructor of my
brothers. Ernest was six years younger
than myself, and was my principal pupil.
He had been afflicted with ill
health from his infancy, through which
Elizabeth and I had been his constant
nurses: his disposition was gentle, but
he was incapable of any severe application.
William, the youngest of our
family, was yet an infant, and the most
beautiful little fellow in the world; his
lively blue eyes, dimpled cheeks, and
endearing manners, inspired the tenderest
affection.

Such was our domestic circle, from
which care and pain seemed for ever
banished. My father directed our studies,
and my mother partook of our enjoyments.
Neither of us possessed the
slightest pre-eminence over the other;
%%043%%
the voice of command was never heard
amongst us; but mutual affection engaged
us all to comply with and obey
the slightest desire of each other.
%%044%%

\namedchapter{Chapter II}

\textsc{When} I had attained the age of seventeen,
my parents resolved that I should
become a student at the university of
Ingolstadt. I had hitherto attended
the schools of Geneva; but my father
thought it necessary, for the completion
of my education, that I should be
made acquainted with other customs
than those of my native country. My
departure was therefore fixed at an
early date; but, before the day resolved
upon could arrive, the first misfortune
of my life occurred --- an omen, as it
were, of my future misery.

Elizabeth had caught the scarlet
fever; but her illness was not severe,
and she quickly recovered. During
her confinement, many arguments had
been urged to persuade my mother to
refrain from attending upon her. She
had, at first, yielded to our entreaties;
but when she heard that her favourite
was recovering, she could no longer
debar herself from her society, and entered
her chamber long before the danger
of infection was past. The consequences
of this imprudence were fatal.
On the third day my mother sickened;
her fever was very malignant,
and the looks of her attendants prognosticated
the worst event. On her
death-bed the fortitude and benignity
of this admirable woman did not
%%045%%
desert her. She joined the hands of
Elizabeth and myself: ``My children,''
she said, ``my firmest hopes of future
happiness were placed on the prospect of
your union. This expectation will now
be the consolation of your father. Elizabeth,
my love, you must supply my
place to your younger cousins. Alas!
I regret that I am taken from you; and,
happy and beloved as I have been, is
it not hard to quit you all? But these
are not thoughts befitting me; I will
endeavour to resign myself cheerfully
to death, and will indulge a hope of
meeting you in another world.''

She died calmly; and her countenance
expressed affection even in death.
I need not describe the feelings of
those whose dearest ties are rent by
that most irreparable evil, the void
that presents itself to the soul, and the
despair that is exhibited on the countenance.
It is so long before the mind
can persuade itself that she, whom we
saw every day, and whose very existence
appeared a part of our own, can have
departed for ever --- that the brightness
of a beloved eye can have been extinguished,
and the sound of a voice so
familiar, and dear to the ear, can be
hushed, never more to be heard. These
are the reflections of the first days;
but when the lapse of time proves the
reality of the evil, then the actual bitterness
of grief commences. Yet from
whom has not that rude hand rent
away some dear connexion; and why
should I describe a sorrow which
all have felt, and must feel? The time
at length arrives, when grief is rather
an indulgence than a necessity; and
%%046%%
the smile that plays upon the lips, although
it may be deemed a sacrilege,
is not banished. My mother was dead,
but we had still duties which we ought
to perform; we must continue our
course with the rest, and learn to think
ourselves fortunate, whilst one remains
whom the spoiler has not seized.

My journey to Ingolstadt, which had
been deferred by these events, was now
again determined upon. I obtained
from my father a respite of some weeks.
This period was spent sadly; my mother's
death, and my speedy departure,
depressed our spirits; but Elizabeth
endeavoured to renew the spirit of
cheerfulness in our little society. Since
the death of her aunt, her mind had
acquired new firmness and vigour. She
determined to fulfil her duties with the
greatest exactness; and she felt that
that most imperious duty, of rendering
her uncle and cousins happy, had devolved
upon her. She consoled me,
amused her uncle, instructed my brothers;
and I never beheld her so enchanting
as at this time, when she
was continually endeavouring to contribute
to the happiness of others, entirely
forgetful of herself.

The day of my departure at length
arrived. I had taken leave of all my
friends, excepting Clerval, who spent
the last evening with us. He bitterly
lamented that he was unable to accompany
me: but his father could not be
persuaded to part with him, intending
that he should become a partner with
him in business, in compliance with
his favourite theory, that learning
was superfluous in the commerce of
%%047%%
ordinary life. Henry had a refined
mind; he had no desire to be idle, and
was well pleased to become his father's
partner, but he believed that a man
might be a very good trader, and yet
possess a cultivated understanding.

We sat late, listening to his complaints,
and making many little arrangements
for the future. The next
morning early I departed. Tears
gushed from the eyes of Elizabeth;
they proceeded partly from sorrow at
my departure, and partly because she
reflected that the same journey was to
have taken place three months before,
when a mother's blessing would have
accompanied me.

I threw myself into the chaise that
was to convey me away, and indulged
in the most melancholy reflections. I,
who had ever been surrounded by amiable
companions, continually engaged
in endeavouring to bestow mutual
%%048%%
pleasure, I was now alone. In the university,
whither I was going, I must form
my own friends, and be my own protector.
My life had hitherto been remarkably
secluded and domestic; and
this had given me invincible repugnance
to new countenances. I loved
my brothers, Elizabeth, and Clerval;
these were ``old familiar faces;'' but I
believed myself totally unfitted for the
company of strangers. Such were my
reflections as I commenced my journey;
but as I proceeded, my spirits and hopes
rose. I ardently desired the acquisition
of knowledge. I had often, when
at home, thought it hard to remain
during my youth cooped up in one
place, and had longed to enter the
world, and take my station among other
human beings. Now my desires were
complied with, and it would, indeed,
have been folly to repent.
%%049%%

I had sufficient leisure for these and
many other reflections during my journey
to Ingolstadt, which was long and
fatiguing. At length the high white
steeple of the town met my eyes. I
alighted, and was conducted to my solitary
apartment, to spend the evening
as I pleased.

The next morning I delivered my
letters of introduction, and paid a visit
to some of the principal professors, and
among others to M.~Krempe, professor
of natural philosophy. He received
me with politeness, and asked me several
questions concerning my progress
in the different branches of science
appertaining to natural philosophy. I
mentioned, it is true, with fear and
%%050%%
trembling, the only authors I had ever
read upon those subjects. The professor
stared: ``Have you,'' he said,
``really spent your time in studying
such nonsense?''

I replied in the affirmative. ``Every
minute,'' continued M.~Krempe with
warmth, ``every instant that you have
wasted on those books is utterly and
entirely lost. You have burdened your
memory with exploded systems, and
useless names. Good God! in what
desert land have you lived, where no
one was kind enough to inform you
that these fancies, which you have so
greedily imbibed, are a thousand years
old, and as musty as they are ancient?
I little expected in this enlightened and
scientific age to find a disciple of Albertus
Magnus and Paracelsus. My
dear Sir, you must begin your studies
entirely anew.''

So saying, he stept aside, and wrote
down a list of several books treating of
natural philosophy, which he desired
me to procure, and dismissed me, after
%%051%%
mentioning that in the beginning of
the following week he intended to commence
a course of lectures upon natural
philosophy in its general relations, and
that M.~Waldman, a fellow-professor,
would lecture upon chemistry the alternate
days that he missed.

I returned home, not disappointed,
for I had long considered those authors
useless whom the professor had so
strongly reprobated; but I did not feel
much inclined to study the books which
I procured at his recommendation. M.~%
Krempe was a little squat man, with a
gruff voice and repulsive countenance;
the teacher, therefore, did not prepossess
me in favour of his doctrine. Besides,
I had a contempt for the uses of modern
natural philosophy. It was very different,
when the masters of the science
sought immortality and power; such
views, although futile, were grand: but
now the scene was changed. The ambition
of the inquirer seemed to limit
itself to the annihilation of those visions
on which my interest in science was
chiefly founded. I was required to
exchange chimeras of boundless grandeur
for realities of little worth.

Such were my reflections during the
first two or three days spent almost in
solitude. But as the ensuing week commenced,
I thought of the information
which M.~Krempe had given me concerning
the lectures. And although I
could not consent to go and hear that
little conceited fellow deliver sentences
out of a pulpit, I recollected what he
had said of M.~Waldman, whom I had
never seen, as he had hitherto been out
of town.

Partly from curiosity, and partly
%%052%%
from idleness, I went into the lecturing
room, which M.~Waldman entered
shortly after. This professor was very
unlike his colleague. He appeared
about fifty years of age, but with an
aspect expressive of the greatest benevolence;
a few gray hairs covered his
temples, but those at the back of his
head were nearly black. His person
was short, but remarkably erect; and his
voice the sweetest I had ever heard.
He began his lecture by a recapitulation
of the history of chemistry and
the various improvements made by different
men of learning, pronouncing
with fervour the names of the most distinguished
discoverers. He then took a
cursory view of the present state of the
science, and explained many of its elementary
terms. After having made a few
preparatory experiments, he concluded
with a panegyric upon modern chemistry,
the terms of which I shall never
forget:---

``The ancient teachers of this science,''
said he, ``prom\-ised impossibilities,
and performed nothing. The
modern masters promise very little;
they know that metals cannot be transmuted,
and that the elixir of life is a
chimera. But these philosophers,
whose hands seem only made to dabble
in dirt, and their eyes to pour over
the microscope or crucible, have indeed
performed miracles. They penetrate
into the recesses of nature, and
shew how she works in her hiding
places. They ascend into the heavens;
they have discovered how the blood
circulates, and the nature of the air we
breathe. They have acquired new and
%%053%%
almost unlimited powers; they can
command the thunders of heaven, mimic
the earthquake, and even mock
the invisible world with its own shadows.''

I departed highly pleased with the
professor and his lecture, and paid him
a visit the same evening. His manners
in private were even more mild
and attractive than in public; for there
was a certain dignity in his mien
during his lecture, which in his own
house was replaced by the greatest
affability and kindness. He heard with
attention my little narration concerning
my studies, and smiled at the names of
Cornelius Agrippa, and Paracelsus,
but without the contempt that M.~%
Krempe had exhibited. He said, that
``these were men to whose indefatigable
zeal modern philosophers were indebted
for most of the foundations of their
knowledge. They had left to us, as
an easier task, to give new names, and
arrange in connected classifications, the
facts which they in a great degree
had been the instruments of bringing
to light. The labours of men of genius,
however erroneously directed, scarcely
ever fail in ultimately turning to the
solid advantage of mankind.'' I listened
to his statement, which was delivered
without any presumption or affectation;
and then added, that his lecture had
removed my prejudices against modern
chemists; and I, at the same time, requested
his advice concerning the books
I ought to procure.

``I am happy,'' said M.~Waldman,
``to have gained a disciple; and if
your application equals your ability, I
%%054%%
have no doubt of your success. Chemistry
is that branch of natural philosophy
in which the greatest improvements
have been and may be made;
it is on that account that I have made
it my peculiar study; but at the same
time I have not neglected the other
branches of science. A man would
make but a very sorry chemist, if he
attended to that department of human
knowledge alone. If your wish is to
become really a man of science, and
not merely a petty experimentalist, I
should advise you to apply to every
branch of natural philosophy, including
mathematics.''

He then took me into his laboratory,
%%055%%
and explained to me the uses of his various
machines; instructing me as to
what I ought to procure, and promising
me the use of his own, when I should
have advanced far enough in the science
not to derange their mechanism. He
also gave me the list of books which I
had requested; and I took my leave.

Thus ended a day memorable to me;
it decided my future destiny.
%%056%%

\namedchapter{Chapter III}

\textsc{From} this day natural philosophy, and
particularly chemistry, in the most comprehensive
sense of the term, became
nearly my sole occupation. I read with
ardour those works, so full of genius and
discrimination, which modern inquirers
have written on these subjects. I attended
the lectures, and cultivated the
acquaintance, of the men of science of
the university; and I found even in
M.~Krempe a great deal of sound sense
and real information, combined, it is
true, with a repulsive physiognomy and
manners, but not on that account the
less valuable. In M.~Waldman I found
a true friend. His gentleness was never
tinged by dogmatism; and his instructions
were given with an air of
frankness and good nature, that banished
every idea of pedantry. It was,
perhaps, the amiable character of this
man that inclined me more to that
branch of natural philosophy which he
professed, than an intrinsic love for the
science itself. But this state of mind
had place only in the first steps towards
knowledge: the more fully I entered
into the science, the more exclusively
I pursued it for its own sake. That
application, which at first had been a
matter of duty and resolution, now became
so ardent and eager, that the stars
often disappeared in the light of
%%057%%
morning whilst I was yet engaged in my
laboratory.

As I applied so closely, it may be
easily conceived that I improved rapidly.
My ardour was indeed the astonishment
of the students; and my proficiency,
that of the masters. Professor
Krempe often asked me, with a sly
smile, how Cornelius Agrippa went on?
whilst M.~Waldman expressed the most
heart-felt exultation in my progress.
Two years passed in this manner, during
which I paid no visit to Geneva,
but was engaged, heart and soul, in the
pursuit of some discoveries, which I
hoped to make. None but those who
have experienced them can conceive of
the enticements of science. In other studies
you go as far as others have gone
before you, and there is nothing more
to know; but in a scientific pursuit
there is continual food for discovery
and wonder. A mind of moderate capacity,
which closely pursues one study,
must infallibly arrive at great proficiency
in that study; and I, who continually
sought the attainment of one
object of pursuit, and was solely wrapt
up in this, improved so rapidly, that, at
the end of two years, I made some discoveries
in the improvement of some
chemical instruments, which procured
me great esteem and admiration at the
university. When I had arrived at this
point, and had become as well acquainted
with the theory and practice
of natural philosophy as depended on
the lessons of any of the professors at
Ingolstadt, my residence there being no
longer conducive to my improvements,
I thought of returning to my friends
%%058%%
and my native town, when an incident
happened that protracted my stay.

One of the phænonema which had
peculiarly attrac\-ted my attention was
the structure of the human frame, and,
indeed, any animal endued with life.
Whence, I often asked myself, did the
principle of life proceed? It was a
bold question, and one which has ever
been considered as a mystery; yet with
how many things are we upon the brink
of becoming acquainted, if cowardice
or carelessness did not restrain our
inquiries. I revolved these circumstances
in my mind, and determined
thenceforth to apply myself more particularly
to those branches of natural
philosophy which relate to physiology.
Unless I had been animated
by an almost supernatural enthusiasm,
my application to this study would have
been irksome, and almost intolerable.
To examine the causes of life, we must
first have recourse to death. I became
acquainted with the science of anatomy:
but this was not sufficient; I
must also observe the natural decay and
corruption of the human body. In my
education my father had taken the
greatest precautions that my mind
should be impressed with no supernatural
horrors. I do not ever remember
to have trembled at a tale of superstition,
or to have feared the apparition
of a spirit. Darkness had no effect
upon my fancy; and a church-yard was
to me merely the receptacle of bodies
deprived of life, which, from being the
seat of beauty and strength, had become
food for the worm. Now I was led to
examine the cause and progress of this
decay, and forced to spend days and
%%059%%
nights in vaults and charnel houses.
My attention was fixed upon every object
the most insupportable to the delicacy
of the human feelings. I saw how
the fine form of man was degraded
and wasted; I beheld the corruption of
death succeed to the blooming cheek
of life; I saw how the worm inherited
the wonders of the eye and brain. I
paused, examining and analysing all
the minutiæ of causation, as exemplified
in the change from life to death,
and death to life, until from the midst
of this darkness a sudden light broke
in upon me --- a light so brilliant and
wondrous, yet so simple, that while I
became dizzy with the immensity of the
prospect which it illustrated, I was surprised
that among so many men of genius,
who had directed their inquiries
towards the same science, that I alone
should be reserved to discover so astonishing
a secret.

Remember, I am not recording the
vision of a madman. The sun does not
more certainly shine in the heavens,
than that which I now affirm is true.
Some miracle might have produced it,
yet the stages of the discovery were distinct
and probable. After days and
nights of incredible labour and fatigue,
I succeeded in discovering the cause of
generation and life; nay, more, I became
myself capable of bestowing animation
upon lifeless matter.

The astonishment which I had at first
experienced on this discovery soon gave
place to delight and rapture. After so
much time spent in painful labour, to arrive
at once at the summit of my desires,
was the most gratifying consummation of
%%060%%
my toils. But this discovery was so great
and overwhelming, that all the steps by
which I had been progressively led to
it were obliterated, and I beheld only
the result. What had been the study
and desire of the wisest men since the
creation of the world, was now within
my grasp. Not that, like a magic
scene, it all opened upon me at once:
the information I had obtained was of
a nature rather to direct my endeavours
so soon as I should point them towards
the object of my search, than to exhibit
that object already accomplished. I
was like the Arabian who had been
buried with the dead, and found a
%%061%%
passage to life aided only by one
glimmering, and seemingly ineffectual
light.

I see by your eagerness, and the
wonder and hope which your eyes express,
my friend, that you expect to
be informed of the secret with which
I am acquainted; that cannot be: listen
patiently until the end of my story,
and you will easily perceive why I am
reserved upon that subject. I will not
lead you on, unguarded and ardent as
I then was, to your destruction and
infallible misery. Learn from me, if
not by my precepts, at least by my example,
how dangerous is the acquirement
of knowledge, and how much
happier that man is who believes his
native town to be the world, than he
who aspires to become greater than his
nature will allow.

When I found so astonishing a power
placed within my hands, I hesitated a
long time concerning the manner in
which I should employ it. Although
I possessed the capacity of bestowing
animation, yet to prepare a frame for
the reception of it, with all its intricacies
of fibres, muscles, and veins, still
remained a work of inconceivable difficulty
and labour. I doubted at first
whether I should attempt the creation
of a being like myself or one of simpler
organization; but my imagination
was too much exalted by my first success
to permit me to doubt of my ability
to give life to an animal as complex
and wonderful as man. The materials
at present within my command hardly
appeared adequate to so arduous an
undertaking; but I doubted not that
%%062%%
I should ultimately succeed. I prepared
myself for a multitude of reverses;
my operations might be incessantly
baffled, and at last my work be
imperfect: yet, when I considered the
improvement which every day takes
place in science and mechanics, I was
encouraged to hope my present attempts
would at least lay the foundations
of future success. Nor could I
consider the magnitude and complexity
of my plan as any argument of its impracticability.
It was with these feelings
that I began the creation of a human
being. As the minuteness of the
parts formed a great hindrance to my
speed, I resolved, contrary to my first
intention, to make the being of a gigantic
stature; that is to say, about
eight feet in height, and proportionably
large. After having formed this determination,
and having spent some
months in successfully collecting and
arranging my materials, I began.

No one can conceive the variety of
feelings which bore me onwards, like
a hurricane, in the first enthusiasm of
success. Life and death appeared to
me ideal bounds, which I should first
break through, and pour a torrent of
light into our dark world. A new species
would bless me as its creator and
source; many happy and excellent natures
would owe their being to me. No
father could claim the gratitude of his
child so completely as I should deserve
their's. Pursuing these reflections, I
thought, that if I could bestow animation
upon lifeless matter, I might in
process of time (although I now found
it impossible) renew life where death
%%063%%
had apparently devoted the body to
corruption.

These thoughts supported my spirits,
while I pursued my undertaking with
unremitting ardour. My cheek had
grown pale with study, and my person
had become emaciated with confinement.
Sometimes, on the very brink of certainty,
I failed; yet still I clung to the
hope which the next day or the next
hour might realize. One secret which I
alone possessed was the hope to which
I had dedicated myself; and the moon
gazed on my midnight labours, while,
with unrelaxed and breathless eagerness,
I pursued nature to her hiding places.
Who shall conceive the horrors of my
secret toil, as I dabbled among the unhallowed
damps of the grave, or tortured
the living animal to animate the lifeless
clay? My limbs now tremble, and my
eyes swim with the remembrance; but
then a resistless, and almost frantic impulse,
urged me forward; I seemed to
have lost all soul or sensation but for this
one pursuit. It was indeed but a passing trance,
that only made me feel with
renewed acuteness so soon as, the unnatural
stimulus ceasing to operate, I
had returned to my old habits. I
collected bones from charnel houses;
and disturbed, with profane fingers, the
tremendous secrets of the human frame.
In a solitary chamber, or rather cell,
at the top of the house, and separated
from all the other apartments by a
gallery and staircase, I kept my workshop
of filthy creation; my eyeballs
were starting from their sockets in attending
to the details of my employment.
The dissecting room and the
%%064%%
slaughter-house furnished many of my materials;
and often did my human nature
turn with loathing from my occupation,
whilst, still urged on by an eagerness
which perpetually increased,
I brought my work near to a conclusion.

The summer months passed while I
was thus engaged, heart and soul, in
one pursuit. It was a most beautiful
season; never did the fields bestow a
more plentiful harvest, or the vines
yield a more luxuriant vintage: but
my eyes were insensible to the charms
of nature. And the same feelings which
made me neglect the scenes around
me caused me also to forget those
friends who were so many miles absent,
%%065%%
and whom I had not seen for so long a
time. I knew my silence disquieted
them; and I well remembered the words
of my father: ``I know that while you
are pleased with yourself, you will
think of us with affection, and we shall
hear regularly from you. You must
pardon me, if I regard any interruption
in your correspondence as a proof
that your other duties are equally neglected.''

I knew well therefore what would be
my father's feelings; but I could not
tear my thoughts from my employment,
loathsome in itself, but which had
taken an irresistible hold of my imagination.
I wished, as it were, to procrastinate
all that related to my feelings
of affection until the great object,
which swallowed up every habit of my
nature, should be completed.

I then thought that my father would
be unjust if he ascribed my neglect to
vice, or faultiness on my part; but I
am now convinced that he was justified
in conceiving that I should not
be altogether free from blame. A human
being in perfection ought always
to preserve a calm and peaceful mind,
and never to allow passion or a transitory
desire to disturb his tranquillity.
I do not think that the pursuit of knowledge
is an exception to this rule. If
the study to which you apply yourself
has a tendency to weaken your affections,
and to destroy your taste for those
simple pleasures in which no alloy can
possibly mix, then that study is certainly
unlawful, that is to say, not befitting
the human mind. If this rule
were always observed; if no man
%%066%%
allowed any pursuit whatsoever to interfere
with the tranquillity of his domestic
affections, Greece had not been enslaved;
Cæsar would have spared his
country; America would have been
discovered more gradually; and the
empires of Mexico and Peru had not
been destroyed.

But I forget that I am moralizing in
the most interesting part of my tale;
and your looks remind me to proceed.

My father made no reproach in his
letters; and only took notice of my
silence by inquiring into my occupations
more particularly than before.
Winter, spring, and summer, passed
away during my labours; but I did not
watch the blossom or the expanding
leaves --- sights which before always
yielded me supreme delight, so deeply
was I engrossed in my occupation. The
leaves of that year had withered before
my work drew near to a close; and
now every day shewed me more plainly
how well I had succeeded. But my
enthusiasm was checked by my anxiety,
and I appeared rather like one doomed
by slavery to toil in the mines, or any
other unwholesome trade, than an artist
occupied by his favourite employment.
Every night I was oppressed by a slow
fever, and I became nervous to a most
painful degree; a disease that I regretted
the more because I had hitherto enjoyed
most excellent health, and had
always boasted of the firmness of my
nerves. But I believed that exercise
and amusement would soon drive away
such symptoms; and I promised myself
both of these, when my creation should
be complete.
%%067%%

\namedchapter{Chapter IV}

\textsc{It} was on a dreary night of November,
that I beheld the accomplishment
of my toils. With an anxiety that almost
amounted to agony, I collected
the instruments of life around me, that
I might infuse a spark of being into
the lifeless thing that lay at my feet.
It was already one in the morning; the
rain pattered dismally against the panes,
and my candle was nearly burnt out,
when, by the glimmer of the half-extinguished
light, I saw the dull yellow
eye of the creature open; it breathed
hard, and a convulsive motion agitated
its limbs.

How can I describe my emotions at
this catastrophe, or how delineate the
wretch whom with such infinite pains
and care I had endeavoured to form?
His limbs were in proportion, and I
had selected his features as beautiful.
Beautiful! --- Great God! His yellow skin
scarcely covered the work of muscles
and arteries beneath; his hair was of a
lustrous black, and flowing; his teeth
of a pearly whiteness; but these luxuriances
only formed a more horrid contrast
with his watery eyes, that seemed
almost of the same colour as the dun
%%068%%
white sockets in which they were set,
his shrivelled complexion, and straight
black lips.

The different accidents of life are
not so changeable as the feelings of
human nature. I had worked hard for
nearly two years, for the sole purpose of
infusing life into an inanimate body.
For this I had deprived myself of rest
and health. I had desired it with an
ardour that far exceeded moderation;
but now that I had finished, the beauty
of the dream vanished, and breathless
horror and disgust filled my heart.
Unable to endure the aspect of the
being I had created, I rushed out of
the room, and continued a long time
traversing my bed-chamber, unable to
%%069%%
compose my mind to sleep. At length
lassitude succeeded to the tumult I had
before endured; and I threw myself on
the bed in my clothes, endeavouring to
seek a few moments of forgetfulness.
But it was in vain: I slept indeed, but
I was disturbed by the wildest dreams.
I thought I saw Elizabeth, in the bloom
of health, walking in the streets of Ingolstadt.
Delighted and surprised, I embraced
her; but as I imprinted the first
kiss on her lips, they became livid with
the hue of death; her features appeared
to change, and I thought that I held
the corpse of my dead mother in my
arms; a shroud enveloped her form,
and I saw the grave-worms crawling in
the folds of the flannel. I started from
my sleep with horror; a cold dew covered
my forehead, my teeth chattered,
and every limb became convulsed;
when, by the dim and yellow light of
the moon, as it forced its way through
the window-shutters, I beheld the wretch --- the
miserable monster whom I had
created. He held up the curtain of the
bed; and his eyes, if eyes they may be
called, were fixed on me. His jaws
opened, and he muttered some inarticulate
sounds, while a grin wrinkled
his cheeks. He might have spoken,
but I did not hear; one hand was
stretched out, seemingly to detain me,
but I escaped, and rushed down stairs.
I took refuge in the court-yard belonging
to the house which I inhabited;
where I remained during the rest of the
night, walking up and down in the
greatest agitation, listening attentively,
catching and fearing each sound as if
%%070%%
it were to announce the approach of
the demoniacal corpse to which I had
so miserably given life.

Oh! no mortal could support the
horror of that countenance. A mummy
again endued with animation could
not be so hideous as that wretch. I
had gazed on him while unfinished;
he was ugly then; but when those muscles
and joints were rendered capable
of motion, it became a thing such as
even Dante could not have conceived.

I passed the night wretchedly. Sometimes
my pulse beat so quickly and
hardly, that I felt the palpitation of every
artery; at others, I nearly sank to the
ground through languor and extreme
weakness. Mingled with this horror, I felt
the bitterness of disappointment: dreams
that had been my food and pleasant rest
for so long a space, were now become a
hell to me; and the change was so rapid,
the overthrow so complete!

Morning, dismal and wet, at length
dawned, and discovered to my sleepless
and aching eyes the church of Ingolstadt,
its white steeple and clock, which
indicated the sixth hour. The porter
opened the gates of the court, which
had that night been my asylum, and
I issued into the streets, pacing them
with quick steps, as if I sought to avoid
the wretch whom I feared every turning
of the street would present to my
view. I did not dare return to the
apartment which I inhabited, but felt
impelled to hurry on, although wetted
by the rain, which poured from a black
and comfortless sky.

I continued walking in this manner
for some time, endeavouring, by bodily
%%071%%
exercise, to ease the load that weighed
upon my mind. I traversed the streets,
without any clear conception of where
I was, or what I was doing. My heart
palpitated in the sickness of fear; and I
hurried on with irregular steps, not
daring to look about me:

\vspace{0.5em plus 1.5em}
\begin{verse}
\small
Like one who, on a lonely road,\\
\hspace{1.5em} Doth walk in fear and dread,\\
And, having once turn'd round, walks on,\\
\hspace{1.5em} And turns no more his head;\\
Because he knows a frightful fiend\\
\hspace{1.5em} Doth close behind him tread.
\end{verse}
\vspace{0.5em plus 1.5em}

Continuing thus, I came at length
opposite to the inn at which the various
diligences and carriages usually stopped.
Here I paused, I knew not why; but I
remained some minutes with my eyes
fixed on a coach that was coming towards
me from the other end of the
street. As it drew nearer, I observed
that it was the Swiss diligence: it stopped
just where I was standing; and, on
the door being opened, I perceived
Henry Clerval, who, on seeing me, instantly
sprung out. ``My dear Frankenstein,''
exclaimed he, ``how glad I
am to see you! how fortunate that you
should be here at the very moment of
my alighting!''

Nothing could equal my delight on
seeing Clerval; his presence brought
back to my thoughts my father, Elizabeth,
and all those scenes of home so
dear to my recollection. I grasped his
hand, and in a moment forgot my horror
and misfortune; I felt suddenly, and
for the first time during many months,
calm and serene joy. I welcomed my
%%072%%
friend, therefore, in the most cordial
manner, and we walked towards my
college. Clerval continued talking for
some time about our mutual friends,
and his own good fortune in being
permitted to come to Ingolstadt. ``You
may easily believe,'' said he, ``how
great was the difficulty to persuade
my father that it was not absolutely
necessary for a merchant not to understand
any thing except book-keeping;
and, indeed, I believe I left him incredulous
to the last, for his constant answer
to my unwearied entreaties was
the same as that of the Dutch schoolmaster
in the Vicar of Wakefield: `I
have ten thousand florins a year without
Greek, I eat heartily without
Greek.' But his affection for me at
length overcame his dislike of learning,
and he has permitted me to undertake
a voyage of discovery to the land of
knowledge.''

``It gives me the greatest delight
to see you; but tell me how you left
my father, brothers, and Elizabeth.''

``Very well, and very happy, only a
little uneasy that they hear from you
so seldom. By the bye, I mean to lecture
you a little upon their account myself. --- But,
my dear Frankenstein,'' continued
he, stopping short, and gazing
full in my face, ``I did not before remark
how very ill you appear; so thin
and pale; you look as if you had been
watching for several nights.''

``You have guessed right; I have
lately been so deeply engaged in one
occupation, that I have not allowed myself
sufficient rest, as you see: but I
%%073%%
hope, I sincerely hope, that all these
employments are now at an end, and
that I am at length free.''

I trembled excessively; I could not
endure to think of, and far less to allude
to the occurrences of the preceding
night. I walked with a quick pace,
and we soon arrived at my college. I
then reflected, and the thought made
me shiver, that the creature whom I
had left in my apartment might still
be there, alive, and walking about.
I dreaded to behold this monster;
but I feared still more that Henry
should see him. Entreating him therefore
to remain a few minutes at the
bottom of the stairs, I darted up towards
my own room. My hand was
already on the lock of the door before
I recollected myself. I then paused;
and a cold shivering came over me. I
threw the door forcibly open, as children
are accustomed to do when they
expect a spectre to stand in waiting
for them on the other side; but nothing
appeared. I stepped fearfully in: the
apartment was empty; and my bedroom
was also freed from its hideous
guest. I could hardly believe that so
great a good-fortune could have befallen
me; but when I became assured
that my enemy had indeed fled, I clapped
my hands for joy, and ran down to
Clerval.

We ascended into my room, and the
servant pres\-ently brought breakfast;
but I was unable to contain myself.
It was not joy only that possessed me;
I felt my flesh tingle with excess of sensitiveness,
and my pulse beat rapidly.
%%074%%
I was unable to remain for a single instant
in the same place; I jumped over
the chairs, clapped my hands, and
laughed aloud. Clerval at first attributed
my unusual spirits to joy on his
arrival; but when he observed me
more attentively, he saw a wildness in
my eyes for which he could not account;
and my loud, unrestrained, heartless
laughter, frightened and astonished
him.

``My dear Victor,'' cried he, ``what,
for God's sake, is the matter? Do not
laugh in that manner. How ill you are!
What is the cause of all this?''

``Do not ask me,'' cried I, putting
my hands before my eyes, for I thought
I saw the dreaded spectre glide into
the room; ``\emph{he} can tell. --- Oh, save me!
save me!'' I imagined that the monster
seized me; I struggled furiously,
and fell down in a fit.

Poor Clerval! what must have been
his feelings? A meeting, which he anticipated
with such joy, so strangely
turned to bitterness. But I was not
the witness of his grief; for I was lifeless,
and did not recover my senses for
a long, long time.

This was the commencement of a
nervous fever, which confined me for
several months. During all that time
Henry was my only nurse. I afterwards
learned that, knowing my father's
advanced age, and unfitness for so long
a journey, and how wretched my sickness
would make Elizabeth, he spared
them this grief by concealing the extent
of my disorder. He knew that
I could not have a more kind and
%%075%%
attentive nurse than himself; and, firm
in the hope he felt of my recovery, he
did not doubt that, instead of doing
harm, he performed the kindest action
that he could towards them.

But I was in reality very ill; and
surely nothing but the unbounded and
unremitting attentions of my friend
could have restored me to life. The
form of the monster on whom I had
bestowed existence was for ever before
my eyes, and I raved incessantly concerning
him. Doubtless my words surprised
Henry: he at first believed them
to be the wanderings of my disturbed
imagination; but the pertinacity with
which I continually recurred to the
same subject persuaded him that my
disorder indeed owed its origin to some
uncommon and terrible event.

By very slow degrees, and with frequent
relapses, that alarmed and grieved
my friend, I recovered. I remember
the first time I became capable of observing
outward objects with any kind
of pleasure, I perceived that the fallen
leaves had disappeared, and that the
young buds were shooting forth from
the trees that shaded my window. It
was a divine spring; and the season
contributed greatly to my convalescence.
I felt also sentiments of joy
and affection revive in my bosom; my
gloom disappeared, and in a short time
I became as cheerful as before I was
attacked by the fatal passion.

``Dearest Clerval,'' exclaimed I,
``how kind, how very good you are to
me. This whole winter, instead of
being spent in study, as you promised
yourself, has been consumed in my
%%076%%
sick room. How shall I ever repay
you? I feel the greatest remorse for
the disappointment of which I have
been the occasion; but you will forgive
me.''

``You will repay me entirely, if you
do not discompose yourself, but get
well as fast as you can; and since you
appear in such good spirits, I may
speak to you on one subject, may I
not?''

I trembled. One subject! what could
it be? Could he allude to an object on
whom I dared not even think?

``Compose yourself,'' said Clerval,
who observed my ch\-ange of colour, ``I
will not mention it, if it agitates you;
but your father and cousin would be
very happy if they received a letter from
you in your own hand-writing. They
hardly know how ill you have been,
and are uneasy at your long silence.''

``Is that all? my dear Henry. How
could you suppose that my first thought
would not fly towards those dear, dear
friends whom I love, and who are so
deserving of my love.''

``If this is your present temper, my
friend, you will perhaps be glad to see
a letter that has been lying here some
days for you: it is from your cousin, I
believe.''
%%077%%

\namedchapter{Chapter V}

\textsc{Clerval} then put the following letter
into my hands.

\bigskip
\noindent ``\emph{To} V.~\textsc{Frankenstein}.
\medskip

\noindent ``\textsc{my dear cousin},
\medskip

``I cannot describe to you the uneasiness
we have all felt concerning
your health. We cannot help imagining
that your friend Clerval conceals the extent
of your disorder: for it is now several
months since we have seen your
hand-writing; and all this time you have
been obliged to dictate your letters to
Henry. Surely, Victor, you must have
been exceedingly ill; and this makes us
all very wretched, as much so nearly as
after the death of your dear mother. My
uncle was almost persuaded that you
were indeed dangerously ill, and could
hardly be restrained from undertaking
a journey to Ingolstadt. Clerval always
writes that you are getting better; I
eagerly hope that you will confirm this
intelligence soon in your own hand-writing;
for indeed, indeed, Victor, we are
all very miserable on this account. Relieve
us from this fear, and we shall be
the happiest creatures in the world.
Your father's health is now so vigorous,
that he appears ten years younger since
last winter. Ernest also is so much
improved, that you would hardly know
him: he is now nearly sixteen, and has
%%078%%
lost that sickly appearance which he
had some years ago; he is grown quite
robust and active.

``My uncle and I conversed a long
time last night about what profession
Ernest should follow. His constant illness
when young has deprived him of
the habits of application; and now that
he enjoys good health, he is continually
in the open air, climbing the hills, or
rowing on the lake. I therefore proposed
that he should be a farmer;
which you know, Cousin, is a favourite
scheme of mine. A farmer's is a very
healthy happy life; and the least hurtful,
or rather the most beneficial profession
of any. My uncle had an idea
of his being educated as an advocate,
that through his interest he might become
a judge. But, besides that he is
not at all fitted for such an occupation,
it is certainly more creditable to cultivate
the earth for the sustenance of man,
than to be the confidant, and sometimes
the accomplice, of his vices; which is
the profession of a lawyer. I said, that
the employments of a prosperous farmer,
if they were not a more honourable,
they were at least a happier species
of occupation than that of a judge,
whose misfortune it was always to meddle
with the dark side of human nature.
My uncle smiled, and said, that
I ought to be an advocate myself, which
put an end to the conversation on that
subject.

``And now I must tell you a little
story that will please, and perhaps
amuse you. Do you not remember
Justine Moritz? Probably you do not;
I will relate her history, therefore, in a
%%079%%
few words. Madame Moritz, her mother,
was a widow with four children,
of whom Justine was the third. This
girl had always been the favourite of
her father; but, through a strange perversity,
her mother could not endure
her, and, after the death of M.~Moritz,
treated her very ill. My aunt observed
this; and, when Justine was twelve
years of age, prevailed on her mother
to allow her to live at her house. The
republican institutions of our country
have produced simpler and happier
manners than those which prevail in
the great monarchies that surround it.
Hence there is less distinction between
the several classes of its inhabitants;
and the lower orders being neither so
poor nor so despised, their manners
are more refined and moral. A servant
in Geneva does not mean the same
thing as a servant in France and England.
Justine, thus received in our family,
learned the duties of a servant; a
condition which, in our fortunate country,
does not include the idea of ignorance,
and a sacrifice of the dignity of
a human being.

``After what I have said, I dare say
you well remember the heroine of my
little tale: for Justine was a great favourite
of your's; and I recollect you
once remarked, that if you were in an
ill humour, one glance from Justine
could dissipate it, for the same reason
that Ariosto gives concerning the beauty
of Angelica --- she looked so frank-hearted
and happy. My aunt conceived a great
attachment for her, by which she was
induced to give her an education superior
to that which she had at first
%%080%%
intended. This benefit was fully repaid;
Justine was the most grateful
little creature in the world: I do not
mean that she made any professions, I
never heard one pass her lips; but you
could see by her eyes that she almost
adored her protectress. Although her
disposition was gay, and in many respects
inconsiderate, yet she paid the
greatest attention to every gesture of
my aunt. She thought her the model
of all excellence, and endeavoured to
imitate her phraseology and manners,
so that even now she often reminds me
of her.

``When my dearest aunt died, every
one was too much occupied in their
own grief to notice poor Justine, who
had attended her during her illness
with the most anxious affection. Poor
Justine was very ill; but other trials
were reserved for her.

``One by one, her brothers and sister
died; and her mother, with the exception
of her neglected daughter, was left
childless. The conscience of the woman
was troubled; she began to think
that the deaths of her favourites was a
judgment from heaven to chastise her
partiality. She was a Roman Catholic;
and I believe her confessor confirmed
the idea which she had conceived. Accordingly,
a few months after your
departure for Ingolstadt, Justine was
called home by her repentant mother.
Poor girl! she wept when she quitted
our house: she was much altered since
the death of my aunt; grief had given
softness and a winning mildness to her
manners, which had before been remarkable
for vivacity. Nor was her
%%081%%
residence at her mother's house of a
nature to restore her gaiety. The poor
woman was very vacillating in her repentance.
She sometimes begged Justine
to forgive her unkindness, but
much oftener accused her of having
caused the deaths of her brothers and
sister. Perpetual fretting at length
threw Madame Moritz into a decline,
which at first increased her irritability,
but she is now at peace for ever. She
died on the first approach of cold weather,
at the beginning of this last winter.
Justine has returned to us; and I
assure you I love her tenderly. She is
very clever and gentle, and extremely
pretty; as I mentioned before, her mien
and her expressions continually remind
me of my dear aunt.

``I must say also a few words to you,
my dear cousin, of little darling William.
I wish you could see him; he is
very tall of his age, with sweet laughing
blue eyes, dark eye-lashes, and curling
hair. When he smiles, two little dimples
appear on each cheek, which are
rosy with health. He has already had
one or two little \emph{wives}, but Louisa Biron
is his favourite, a pretty little girl of
five years of age.

``Now, dear Victor, I dare say you
wish to be indulged in a little gossip
concerning the good people of Geneva.
The pretty Miss Mansfield has already
received the congratulatory visits on
her approaching marriage with a young
Englishman, John Melbourne, Esq.
Her ugly sister, Manon, married M.~%
Duvillard, the rich banker, last autumn.
Your favourite schoolfellow,
Louis Manoir, has suffered several
%%082%%
misfortunes since the departure of Clerval
from Geneva. But he has already recovered
his spirits, and is reported to
be on the point of marrying a very
lively pretty Frenchwoman, Madame
Tavernier. She is a widow, and much
older than Manoir; but she is very
much admired, and a favourite with
every body.

``I have written myself into good
spirits, dear cousin; yet I cannot conclude
without again anxiously inquiring
concerning your health. Dear
Victor, if you are not very ill, write
yourself, and make your father and all
of us happy; or ------ I cannot bear to
think of the other side of the question;
my tears already flow. Adieu, my
dearest cousin.''

\frLetterSig{``\textsc{Elizabeth Lavenza}.\\
\small ``Geneva, March 18th, 17--- .''}

``Dear, dear Elizabeth!'' I exclaimed
when I had read her letter, ``I will
write instantly, and relieve them from
the anxiety they must feel.'' I wrote,
and this exertion greatly fatigued me;
but my convalescence had commenced,
and proceeded regularly. In another
fortnight I was able to leave my chamber.

One of my first duties on my recovery
was to introduce Clerval to the
several professors of the university. In
doing this, I underwent a kind of rough
usage, ill befitting the wounds that my
mind had sustained. Ever since the
fatal night, the end of my labours, and
the beginning of my misfortunes, I had
conceived a violent antipathy even to
the name of natural philosophy. When
%%083%%
I was otherwise quite restored to health,
the sight of a chemical instrument
would renew all the agony of my nervous
symptoms. Henry saw this, and
had removed all my apparatus from
my view. He had also changed my
apartment; for he perceived that I had
acquired a dislike for the room which
had previously been my laboratory.
But these cares of Clerval were made
of no avail when I visited the professors.
M.~Waldman inflicted torture
when he praised, with kindness and
warmth, the astonishing progress I had
made in the sciences. He soon perceived
that I disliked the subject; but,
not guessing the real cause, he attributed
my feelings to modesty, and
changed the subject from my improvement
to the science itself, with a desire,
as I evidently saw, of drawing me
out. What could I do? He meant to
please, and he tormented me. I felt as
if he had placed carefully, one by one, in
my view those instruments which were
to be afterwards used in putting me to
a slow and cruel death. I writhed
under his words, yet dared not exhibit
the pain I felt. Clerval, whose eyes
and feelings were always quick in discerning
the sensations of others, declined
the subject, alleging, in excuse,
his total ignorance; and the conversation
took a more general turn. I
thanked my friend from my heart, but
I did not speak. I saw plainly that he
was surprised, but he never attempted
to draw my secret from me; and although
I loved him with a mixture of
affection and reverence that knew no
bounds, yet I could never persuade
%%084%%
myself to confide to him that event
which was so often present to my recollection,
but which I feared the detail
to another would only impress more
deeply.

M.~Krempe was not equally docile;
and in my condition at that time, of
almost insupportable sensitiveness, his
harsh blunt encomiums gave me even
more pain than the benevolent approbation
of M.~Waldman. ``D---n the
fellow!'' cried he; ``why, M.~Clerval,
I assure you he has outstript us all.
Aye, stare if you please; but it is nevertheless
true. A youngster who, but
a few years ago, believed Cornelius
Agrippa as firmly as the gospel, has
now set himself at the head of the university;
and if he is not soon pulled
down, we shall all be out of countenance. --- Aye,
aye,'' continued he, observing
my face expressive of suffering,
``M.~Frankenstein is modest; an excellent
quality in a young man. Young
men should be diffident of themselves,
you know, M.~Clerval; I was myself
when young: but that wears out in a
very short time.''

M.~Krempe had now commenced an
eulogy on himself, which happily turned
the conversation from a subject that
was so annoying to me.

Clerval was no natural philosopher.
His imagination was too vivid for the
minutiæ of science. Languages were his
principal study; and he sought, by acquiring
their elements, to open a field for
self-instruction on his return to Geneva.
Persian, Arabic, and Hebrew, gained
his attention, after he had made himself
perfectly master of Greek and Latin. For
%%085%%
my own part, idleness had ever been
irksome to me; and now that I wished
to fly from reflection, and hated my former
studies, I felt great relief in being
the fellow-pupil with my friend, and
found not only instruction but consolation
in the works of the orientalists.
Their melancholy is soothing, and their
joy elevating to a degree I never experienced
in studying the authors of any
other country. When you read their
writings, life appears to consist in a
warm sun and garden of roses, --- in the
smiles and frowns of a fair enemy, and
the fire that consumes your own heart.
How different from the manly and heroical
poetry of Greece and Rome.

Summer passed away in these occupations,
and my return to Geneva
was fixed for the latter end of autumn;
but being delayed by several accidents,
winter and snow arrived, the roads were
deemed impassable, and my journey
was retarded until the ensuing spring.
I felt this delay very bitterly; for I
longed to see my native town, and my
beloved friends. My return had only
been delayed so long from an unwillingness
to leave Clerval in a strange
place, before he had become acquainted
with any of its inhabitants. The winter,
however, was spent cheerfully; and
although the spring was uncommonly
late, when it came, its beauty compensated
for its dilatoriness.

The month of May had already commenced,
and I expected the letter daily
which was to fix the date of my departure,
when Henry proposed a pedestrian
tour in the environs of Ingolstadt that
I might bid a personal farewell to the
%%086%%
country I had so long inhabited. I acceded
with pleasure to this proposition:
I was fond of exercise, and Clerval had
always been my favourite companion in
the rambles of this nature that I had
taken among the scenes of my native
country.

We passed a fortnight in these perambulations:
my health and spirits
had long been restored, and they gained
additional strength from the salubrious
air I br\-eath\-ed, the natural incidents of
our progress, and the conversation of
my friend. Study had before secluded
me from the intercourse of my fellow-creatures,
and rendered me unsocial;
but Clerval called forth the better feelings
of my heart; he again taught me
to love the aspect of nature, and the
cheerful faces of children. Excellent
friend! how sincerely did you love me,
and endeavour to elevate my mind,
until it was on a level with your own.
A selfish pursuit had cramped and narrowed
me, until your gentleness and
affection warmed and opened my senses;
I became the same happy creature who,
a few years ago, loving and beloved by
all, had no sorrow or care. When
happy, inanimate nature had the power
of bestowing on me the most delightful
sensations. A serene sky and verdant
fields filled me with ecstacy. The
present season was indeed divine; the
flowers of spring bloomed in the hedges,
while those of summer were already in
bud: I was undisturbed by thoughts
which during the preceding year had
pressed upon me, notwithstanding my
endeavours to throw them off, with an
invincible burden.
%%087%%

Henry rejoiced in my gaiety, and
sincerely sympathized in my feelings:
he exerted himself to amuse me, while
he expressed the sensations that filled
his soul. The resources of his mind
on this occasion were truly astonishing:
his conversation was full of imagination;
and very often, in imitation
of the Persian and Arabic writers, he
invented tales of wonderful fancy and
passion. At other times he repeated
my favourite poems, or drew me out
into arguments, which he supported
with great ingenuity.

We returned to our college on a
Sunday afternoon: the peasants were
dancing, and every one we met appeared
gay and happy. My own spirits
were high, and I bounded along
with feelings of unbridled joy and
hilarity.
%%088%%

\namedchapter{Chapter VI}

\textsc{On} my return, I found the following
letter from my father:

\bigskip
\noindent ``\emph{To} V.~\textsc{Frankenstein}.

\medskip
\noindent ``\textsc{my dear victor},

\medskip
``You have probably waited impatiently
for a letter to fix the date of
your return to us; and I was at first
tempted to write only a few lines, merely
mentioning the day on which I should
expect you. But that would be a cruel
kindness, and I dare not do it. What
would be your surprise, my son, when
you expected a happy and gay welcome,
to behold, on the contrary, tears and
wretchedness? And how, Victor, can I
relate our misfortune? Absence cannot
have rendered you callous to our
joys and griefs; and how shall I inflict
pain on an absent child? I wish to prepare
you for the woeful news, but I
know it is impossible; even now your
eye skims over the page, to seek the
words which are to convey to you the
horrible tidings.

``William is dead! --- that sweet child,
whose smiles delighted and warmed
my heart, who was so gentle, yet so
gay! Victor, he is murdered!

``I will not attempt to console you;
but will simply relate the circumstances
of the transaction.

``Last Thursday (May 7th) I, my
%%089%%
niece, and your two brothers, went to
walk in Plainpalais. The evening was
warm and serene, and we prolonged
our walk farther than usual. It was
already dusk before we thought of returning;
and then we discovered that
William and Ernest, who had gone on
before, were not to be found. We accordingly
rested on a seat until they
should return. Presently Ernest came,
and inquired if we had seen his brother:
he said, that they had been playing
together, that William had run away
to hide himself, and that he vainly
sought for him, and afterwards waited
for him a long time, but that he did
not return.

``This account rather alarmed us,
and we continued to search for him
until night fell, when Elizabeth conjectured
that he might have returned to
the house. He was not there. We returned
again, with torches; for I could
not rest, when I thought that my sweet
boy had lost himself, and was exposed
to all the damps and dews of night:
Elizabeth also suffered extreme anguish.
About five in the morning I discovered
my lovely boy, whom the night before
I had seen blooming and active in
health, stretched on the grass livid and
motionless: the print of the murderer's
finger was on his neck.

``He was conveyed home, and the
anguish that was visible in my countenance
betrayed the secret to Elizabeth.
She was very earnest to see the corpse.
At first I attempted to prevent her; but
she persisted, and entering the room
where it lay, hastily examined the neck
%%090%%
of the victim, and clasping her hands
exclaimed, `O God! I have murdered
my darling infant!'

``She fainted, and was restored with
extreme difficulty. When she again
lived, it was only to weep and sigh.
She told me, that that same evening
William had teazed her to let him wear
a very valuable miniature that she possessed
of your mother. This picture is
gone, and was doubtless the temptation
which urged the murderer to the deed.
We have no trace of him at present,
although our exertions to discover him
are unremitted; but they will not restore
my beloved William.

``Come, dearest Victor; you alone
can console Elizabeth. She weeps continually,
and accuses herself unjustly as
the cause of his death; her words pierce
my heart. We are all unhappy; but
will not that be an additional motive
for you, my son, to return and be our
comforter? Your dear mother! Alas,
Victor! I now say, Thank God she did
not live to witness the cruel, miserable
death of her youngest darling!

``Come, Victor; not brooding thoughts
of veng\-eance ag\-ainst the assassin, but
with feelings of peace and gentleness,
that will heal, instead of festering the
wounds of our minds. Enter the house
of mourning, my friend, but with kindness
and affection for those who love
you, and not with hatred for your enemies.

``Your affectionate and afflicted father,
\frLetterSig{``\textsc{Alphonse Frankenstein}.\\
\small ``Geneva, May 12th, 17---.''}
%%091%%

Clerval, who had watched my countenance
as I read this letter, was surprised
to observe the despair that succeeded
to the joy I at first expressed
on receiving news from my friends. I
threw the letter on the table, and covered
my face with my hands.

``My dear Frankenstein,'' exclaimed
Henry, when he perceived me weep
with bitterness, ``are you always to be
unhappy? My dear friend, what has
happened?''

I motioned to him to take up the
letter, while I walked up and down the
room in the extremest agitation. Tears
also gushed from the eyes of Clerval,
as he read the account of my misfortune.

``I can offer you no consolation, my
friend,'' said he; ``your disaster is irreparable.
What do you intend to
do?''

``To go instantly to Geneva: come
with me, Henry, to order the horses.''

During our walk, Clerval endeavoured
to raise my spirits. He did not
do this by common topics of consolation,
but by exhibiting the truest sympathy.
``Poor William!'' said he, ``that
dear child; he now sleeps with his angel
mother. His friends mourn and
weep, but he is at rest: he does not
now feel the murderer's grasp; a sod
covers his gentle form, and he knows
no pain. He can no longer be a fit
subject for pity; the survivors are the
greatest sufferers, and for them time is
the only consolation. Those maxims
of the Stoics, that death was no evil,
and that the mind of man ought to be
superior to despair on the eternal
%%092%%
absence of a beloved object, ought not to
be urged. Even Cato wept over the
dead body of his brother.''

Clerval spoke thus as we hurried
through the streets; the words impressed
themselves on my mind, and I
remembered them afterwards in solitude.
But now, as soon as the horses
arrived, I hurried into a cabriole, and
bade farewell to my friend.

My journey was very melancholy.
At first I wished to hurry on, for I
longed to console and sympathize with
my loved and sorrowing friends; but
when I drew near my native town, I
slackened my progress. I could hardly
sustain the multitude of feelings that
crowded into my mind. I passed
through scenes familiar to my youth,
but which I had not seen for nearly six
years. How altered every thing might
be during that time? One sudden and
desolating change had taken place; but
a thousand little circumstances might
have by degrees worked other alterations
which, although they were done
more tranquilly, might not be the less
decisive. Fear overcame me; I dared
not advance, dreading a thousand
%%093%%
nameless evils that made me tremble, although
I was unable to define them.

I remained two days at Lausanne, in
this painful state of mind. I contemplated
the lake: the waters were placid;
all around was calm, and the snowy
mountains, ``the palaces of nature,''
were not changed. By degrees the
calm and heavenly scene restored me,
and I continued my journey towards
Geneva.

The road ran by the side of the lake,
which became narrower as I approached
my native town. I discovered more
distinctly the black sides of Jura, and
the bright summit of Mont Blânc; I
wept like a child: ``Dear mountains!
my own beautiful lake! how do you
welcome your wanderer? Your summits
are clear; the sky and lake are
blue and placid. Is this to prognosticate
peace, or to mock at my unhappiness?''

I fear, my friend, that I shall render
myself tedious by dwelling on these
preliminary circumstances; but they
were days of comparative happiness,
and I think of them with pleasure. My
%%094%%
country, my beloved country! who but
a native can tell the delight I took in
again beholding thy streams, thy mountains,
and, more than all, thy lovely lake.

Yet, as I drew nearer home, grief
and fear again overcame me. Night
also closed around; and when I could
hardly see the dark mountains, I felt
still more gloomily. The picture appeared
a vast and dim scene of evil,
and I foresaw obscurely that I was destined
to become the most wretched of
human beings. Alas! I prophesied
truly, and failed only in one single
circumstance, that in all the misery I
imagined and dreaded, I did not conceive
the hundredth part of the anguish
I was destined to endure.

It was completely dark when I arrived
in the environs of Geneva; the
gates of the town were already shut;
and I was obliged to pass the night at
Secheron, a village half a league to the
east of the city. The sky was serene;
and, as I was unable to rest, I resolved
to visit the spot where my poor William
had been murdered. As I could not
pass through the town, I was obliged
to cross the lake in a boat to arrive at
Plainpalais. During this short voyage
I saw the lightnings playing on the
summit of Mont Blânc in the most
beautiful figures. The storm appeared
to approach rapidly; and, on landing,
I ascended a low hill, that I might observe
its progress. It advanced; the
heavens were clouded, and I soon felt
the rain coming slowly in large drops,
but its violence quickly increased.

I quitted my seat, and walked on,
although the darkness and storm
%%095%%
increased every minute, and the thunder
burst with a terrific crash over my
head. It was echoed from Salêve, the
Juras, and the Alps of Savoy; vivid
flashes of lightning dazzled my eyes,
illuminating the lake, making it appear
like a vast sheet of fire; then for
an instant every thing seemed of a
pitchy darkness, until the eye recovered
itself from the preceding flash. The
storm, as is often the case in Switzerland,
appeared at once in various parts
of the heavens. The most violent
storm hung exactly north of the town,
over that part of the lake which lies
between the promontory of Belrive
and the village of Copêt. Another
storm enlightened Jura with faint
flashes; and another darkened and
sometimes disclosed the Môle, a peaked
mountain to the east of the lake.

While I watched the storm, so beautiful
yet terrific, I wandered on with a
hasty step. This noble war in the sky
elevated my spirits; I clasped my hands,
and exclaimed aloud, ``William, dear
angel! this is thy funeral, this thy
dirge!'' As I said these words, I perceived
in the gloom a figure which
stole from behind a clump of trees near
me; I stood fixed, gazing intently: I
could not be mistaken. A flash of
lightning illuminated the object, and
discovered its shape plainly to me;
its gigantic stature, and the deformity
of its aspect, more hideous than belongs
to humanity, instantly informed
me that it was the wretch, the filthy
dæmon to whom I had given life.
What did he there? Could he be (I
shuddered at the conception) the
%%096%%
murderer of my brother? No sooner did
that idea cross my imagination, than I
became convinced of its truth; my
teeth chattered, and I was forced to
lean against a tree for support. The
figure passed me quickly, and I lost
it in the gloom. Nothing in human
shape could have destroyed that fair
child. \emph{He} was the murderer! I could
not doubt it. The mere presence of
the idea was an irresistible proof of the
fact. I thought of pursuing the devil;
but it would have been in vain, for
another flash discovered him to me
hanging among the rocks of the nearly
perpendicular ascent of Mont Salêve,
a hill that bounds Plainpalais on the
south. He soon reached the summit,
and disappeared.

I remained motionless. The thunder
ceased; but the rain still continued,
and the scene was enveloped in an impenetrable
darkness. I revolved in my
mind the events which I had until now
sought to forget: the whole train of
my progress towards the creation; the
appearance of the work of my own
hands alive at my bed side; its departure.
Two years had now nearly
elapsed since the night on which he
first received life; and was this his
first crime? Alas! I had turned loose
into the world a depraved wretch, whose
delight was in carnage and misery;
had he not murdered my brother?

No one can conceive the anguish I
suffered during the remainder of the
night, which I spent, cold and wet, in
the open air. But I did not feel the
inconvenience of the weather; my imagination
was busy in scenes of evil and
%%097%%
despair. I considered the being whom
I had cast among mankind, and endowed
with the will and power to effect
purposes of horror, such as the deed
which he had now done, nearly in the
light of my own vampire, my own spirit
let loose from the grave, and forced
to destroy all that was dear to me.

Day dawned; and I directed my steps
towards the town. The gates were open;
and I hastened to my father's house.
My first thought was to discover what
I knew of the murderer, and cause instant
pursuit to be made. But I paused
when I reflected on the story that I had
to tell. A being whom I myself had
formed, and endued with life, had met
me at midnight among the precipices
of an inaccessible mountain. I remembered
also the nervous fever with which
I had been seized just at the time that
I dated my creation, and which would
give an air of delirium to a tale otherwise
so utterly improbable. I well
knew that if any other had communicated
such a relation to me, I should
have looked upon it as the ravings of
insanity. Besides, the strange nature
of the animal would elude all pursuit,
even if I were so far credited as to persuade
my relatives to commence it.
Besides, of what use would be pursuit?
Who could arrest a creature
capable of scaling the overhanging
sides of Mont Salêve? These reflections
determined me, and I resolved to
remain silent.

It was about five in the morning
when I entered my father's house. I
told the servants not to disturb the family,
and went into the library to attend
their usual hour of rising.
%%098%%

Six years had elapsed, passed as a
dream but for one indelible trace, and
I stood in the same place where I had
last embraced my father before my
departure for Ingolstadt. Beloved and
respectable parent! He still remained
to me. I gazed on the picture of my
mother, which stood over the mantle-piece.
It was an historical subject,
painted at my father's desire, and represented
Caroline Beaufort in an agony
of despair, kneeling by the coffin of
her dead father. Her garb was rustic,
and her cheek pale; but there was an
air of dignity and beauty, that hardly
permitted the sentiment of pity. Below
this picture was a miniature of
William; and my tears flowed when I
looked upon it. While I was thus engaged,
Ernest entered: he had heard
me arrive, and hastened to welcome
me. He expressed a sorrowful delight to
see me: ``Welcome, my dearest Victor,''
said he. ``Ah! I wish you had come
three months ago, and then you would
have found us all joyous and delighted.
But we are now unhappy; and, I am
afraid, tears instead of smiles will be
your welcome. Our father looks so
sorrowful: this dreadful event seems to
have revived in his mind his grief on
the death of Mamma. Poor Elizabeth
also is quite inconsolable.'' Ernest
began to weep as he said these
words.

``Do not,'' said I, ``welcome me
thus; try to be more calm, that I may
not be absolutely miserable the moment
I enter my father's house after so long
an absence. But, tell me, how does
my father support his misfortunes? and
how is my poor Elizabeth?''
%%099%%

``She indeed requires consolation;
she accused herself of having caused
the death of my brother, and that made
her very wretched. But since the
murderer has been discovered ------ ''

``The murderer discovered! Good
God! how can that be? who could attempt
to pursue him? It is impossible;
one might as well try to overtake the
winds, or confine a mountain-stream
with a straw.''

``I do not know what you mean;
but we were all very unhappy when
she was discovered. No one would
believe it at first; and even now Elizabeth
will not be convinced, notwithstanding
all the evidence. Indeed, who
would credit that Justine Moritz, who
was so amiable, and fond of all the
family, could all at once become so
extremely wicked?''

``Justine Moritz! Poor, poor girl,
is she the accused? But it is wrongfully;
every one knows that; no one
believes it, surely, Ernest?''

``No one did at first; but several circumstances
came out, that have almost
forced conviction upon us: and her
own behaviour has been so confused,
as to add to the evidence of facts a
weight that, I fear, leaves no hope for
doubt. But she will be tried to-day,
and you will then hear all.''

He related that, the morning on
which the murder of poor William had
been discovered, Justine had been taken
ill, and confined to her bed; and, after
several days, one of the servants, happening
to examine the apparel she had
worn on the night of the murder, had
discovered in her pocket the picture of
%%100%%
my mother, which had been judged to
be the temptation of the murderer.
The servant instantly shewed it to one
of the others, who, without saying a
word to any of the family, went to a
magistrate; and, upon their deposition,
Justine was apprehended. On being
charged with the fact, the poor girl confirmed
the suspicion in a great measure
by her extreme confusion of
manner.

This was a strange tale, but it did
not shake my faith; and I replied earnestly,
``You are all mistaken; I know
the murderer. Justine, poor, good Justine,
is innocent.''

At that instant my father entered.
I saw unhappiness deeply impressed
on his countenance, but he endeavoured
to welcome me cheerfully; and,
after we had exchanged our mournful
greeting, would have introduced some
other topic than that of our disaster,
had not Ernest exclaimed, ``Good
God, Papa! Victor says that he knows
who was the murderer of poor William.''

``We do also, unfortunately,'' replied
my father; ``for indeed I had rather
have been for ever ignorant than
have discovered so much depravity and
ingratitude in one I valued so highly.''

``My dear father, you are mistaken;
Justine is innocent.''

``If she is, God forbid that she
should suffer as guilty. She is to be
tried to-day, and I hope, I sincerely
hope, that she will be acquitted.''

This speech calmed me. I was firmly
convinced in my own mind that Justine,
and indeed every human being, was
%%101%%
guiltless of this murder. I had no
fear, therefore, that any circumstantial
evidence could be brought forward
strong enough to convict her; and, in
this assurance, I calmed myself, expecting
the trial with eagerness, but without
prognosticating an evil result.

We were soon joined by Elizabeth.
Time had made great alterations in her
form since I had last beheld her. Six
years before she had been a pretty,
good-humoured girl, whom every one
loved and caressed. She was now a
woman in stature and expression of
countenance, which was uncommonly
lovely. An open and capacious forehead
gave indications of a good understanding,
joined to great frankness of
disposition. Her eyes were hazel, and
expressive of mildness, now through
recent affliction allied to sadness. Her
hair was of a rich, dark auburn, her
complexion fair, and her figure slight
and graceful. She welcomed me with
the greatest affection. ``Your arrival,
my dear cousin,'' said she, ``fills me
with hope. You perhaps will find
some means to justify my poor guiltless
Justine. Alas! who is safe, if she
be convicted of crime? I rely on her
innocence as certainly as I do upon my
own. Our misfortune is doubly hard
to us; we have not only lost that lovely
darling boy, but this poor girl, whom I
sincerely love, is to be torn away by
even a worse fate. If she is condemned,
I never shall know joy more. But she
will not, I am sure she will not; and
then I shall be happy again, even after
the sad death of my little William.''

``She is innocent, my Elizabeth,''
%%102%%
said I, ``and that shall be proved; fear
nothing, but let your spirits be cheered
by the assurance of her acquittal.''

``How kind you are! every one else
believes in her guilt, and that made me
wretched; for I knew that it was impossible:
and to see every one else prejudiced
in so deadly a manner, rendered
me hopeless and despairing.'' She
wept.

``Sweet niece,'' said my father,
``dry your tears. If she is, as you believe,
innocent, rely on the justice of
our judges, and the activity with which
I shall prevent the slightest shadow of
partiality.''
%%103%%

\namedchapter{Chapter VII}

\textsc{We} passed a few sad hours, until
eleven o'clock, when the trial was to
commence. My father and the rest of
the family being obliged to attend as
witnesses, I accompanied them to the
court. During the whole of this
wretched mockery of justice, I suffered
living torture. It was to be decided,
whether the result of my curiosity and
lawless devices would cause the death
of two of my fellow-beings: one a
smiling babe, full of innocence and
joy; the other far more dreadfully murdered,
with every aggravation of infamy
that could make the murder memorable
in horror. Justine also was a
girl of merit, and possessed qualities
which promised to render her life
happy: now all was to be obliterated
in an ignominious grave; and I the
cause! A thousand times rather would
I have confessed myself guilty of the
crime ascribed to Justine; but I was
absent when it was committed, and such
a declaration would have been considered
as the ravings of a madman, and
would not have exculpated her who
suffered through me.

The appearance of Justine was calm.
She was dressed in mourning; and her
countenance, always engaging, was
rendered, by the solemnity of her feelings,
exquisitely beautiful. Yet she
%%104%%
appeared confident in innocence, and
did not tremble, although gazed on and
execrated by thousands; for all the
kindness which her beauty might otherwise
have excited, was obliterated in
the minds of the spectators by the imagination
of the enormity she was supposed
to have committed. She was
tranquil, yet her tranquillity was evidently
constrained; and as her confusion
had before been adduced as a
proof of her guilt, she worked up her
mind to an appearance of courage.
When she entered the court, she threw
her eyes round it, and quickly discovered
where we were seated. A tear
seemed to dim her eye when she saw
us; but she quickly recovered herself,
and a look of sorrowful affection seemed
to attest her utter guiltlessness.

The trial began; and after the advocate
against her had stated the charge,
several witnesses were called. Several
strange facts combined against her,
which might have staggered any one
who had not such proof of her innocence
as I had. She had been out the
whole of the night on which the murder
had been committed, and towards
morning had been perceived by a market-woman
not far from the spot where
the body of the murdered child had
been afterwards found. The woman
asked her what she did there; but she
looked very strangely, and only returned
a confused and unintelligible answer.
She returned to the house about
eight o'clock; and when one inquired
where she had passed the night, she replied,
that she had been looking for
the child, and demanded earnestly, if
%%105%%
any thing had been heard concerning
him. When shewn the body, she fell
into violent hysterics, and kept her bed
for several days. The picture was then
produced, which the servant had found
in her pocket; and when Elizabeth, in
a faltering voice, proved that it was the
same which, an hour before the child
had been missed, she had placed round
his neck, a murmur of horror and indignation
filled the court.

Justine was called on for her defence.
As the trial had proceeded, her countenance
had altered. Surprise, horror,
and misery, were strongly expressed.
Sometimes she struggled with her tears;
but when she was desired to plead, she
collected her powers, and spoke in an
audible although variable voice:---

``God knows,'' she said, ``how entirely
I am innocent. But I do not
pretend that my protestations should
acquit me: I rest my innocence on a
plain and simple explanation of the
facts which have been adduced against
me; and I hope the character I have
always borne will incline my judges
to a favourable interpretation, where
any circumstance appears doubtful or
suspicious.''

She then related that, by the permission
of Elizabeth, she had passed the
evening of the night on which the murder
had been committed, at the house
of an aunt at Chêne, a village situated
at about a league from Geneva. On
her return, at about nine o'clock, she
met a man, who asked her if she had
seen any thing of the child who was
lost. She was alarmed by this account,
and passed several hours in looking for
%%106%%
him, when the gates of Geneva were
shut, and she was forced to remain several
hours of the night in a barn belonging
to a cottage, being unwilling
to call up the inhabitants, to whom she
was well known. Unable to rest or
sleep, she quitted her asylum early, that
she might again endeavour to find my
brother. If she had gone near the spot
where his body lay, it was without her
knowledge. That she had been bewildered
when questioned by the market-woman,
was not surprising, since she
had passed a sleepless night, and the
fate of poor William was yet uncertain.
Concerning the picture she could give
no account.

``I know,'' continued the unhappy
victim, ``how heavily and fatally this
one circumstance weighs against me,
but I have no power of explaining it;
and when I have expressed my utter
ignorance, I am only left to conjecture
concerning the probabilities by which
it might have been placed in my
pocket. But here also I am checked.
I believe that I have no enemy on earth,
and none surely would have been so
wicked as to destroy me wantonly.
Did the murderer place it there? I
know of no opportunity afforded him
for so doing; or if I had, why should
he have stolen the jewel, to part with it
again so soon?

``I commit my cause to the justice
of my judges, yet I see no room for
hope. I beg permission to have a few
witnesses examined concerning my
character; and if their testimony shall
not overweigh my supposed guilt, I
must be condemned, although I would
%%107%%
pledge my salvation on my innocence.''

Several witnesses were called, who
had known her for many years, and
they spoke well of her; but fear, and
hatred of the crime of which they supposed
her guilty, rendered them timorous,
and unwilling to come forward.
Elizabeth saw even this last resource,
her excellent dispositions and irreproachable
conduct, about to fail the
accused, when, although violently agitated,
she desired permission to address
the court.

``I am,'' said she, ``the cousin of the
unhappy child who was murdered, or
rather his sister, for I was educated by
and have lived with his parents ever
since and even long before his birth. It
may therefore be judged indecent in me
to come forward on this occasion; but
when I see a fellow-creature about to
perish through the cowardice of her
pretended friends, I wish to be allowed
to speak, that I may say what I know of
her character. I am well acquainted
with the accused. I have lived in the
same house with her, at one time for
five, and at another for nearly two years.
During all that period she appeared to
me the most amiable and benevolent of
human creatures. She nursed Madame
Frankenstein, my aunt, in her last illness
with the greatest affection and care; and
afterwards attended her own mother
during a tedious illness, in a manner
that excited the admiration of all who
knew her. After which she again lived
in my uncle's house, where she was
beloved by all the family. She was
warmly attached to the child who is
%%108%%
now dead, and acted towards him like
a most affectionate mother. For my own
part, I do not hesitate to say, that, notwithstanding
all the evidence produced
against her, I believe and rely on her
perfect innocence. She had no temptation
for such an action: as to the bauble
on which the chief proof rests, if she
had earnestly desired it, I should have
willingly given it to her; so much do
I esteem and value her.''

Excellent Elizabeth! A murmur of
approbation was heard; but it was excited
by her generous interference, and
not in favour of poor Justine, on whom
the public indignation was turned with
renewed violence, charging her with
the blackest ingratitude. She herself
wept as Elizabeth spoke, but she did
not answer. My own agitation and
anguish was extreme during the whole
trial. I believed in her innocence; I
knew it. Could the dæmon, who had
(I did not for a minute doubt) murdered
my brother, also in his hellish
sport have betrayed the innocent to
death and ignominy. I could not sustain
the horror of my situation; and
when I perceived that the popular voice,
and the countenances of the judges, had
already condemned my unhappy victim,
I rushed out of the court in agony.
The tortures of the accused did not
equal mine; she was sustained by innocence,
but the fangs of remorse tore my
bosom, and would not forego their hold.

I passed a night of unmingled wretchedness.
In the morning I went to the
court; my lips and throat were parched.
I dared not ask the fatal question; but
I was known, and the officer guessed
%%109%%
the cause of my visit. The ballots had
been thrown; they were all black, and
Justine was condemned.

I cannot pretend to describe what I
then felt. I had before experienced
sensations of horror; and I have endeavoured
to bestow upon them adequate
expressions, but words cannot convey
an idea of the heart-sickening despair
that I then endured. The person to
whom I addressed myself added, that
Justine had already confessed her guilt.
``That evidence,'' he observed, ``was
hardly required in so glaring a case,
but I am glad of it; and, indeed, none
of our judges like to condemn a criminal
upon circumstantial evidence, be
it ever so decisive.''

When I returned home, Elizabeth
eagerly demanded the result.

``My cousin,'' replied I, ``it is decided
as you may have expected; all
judges had rather that ten innocent
should suffer, than that one guilty should
escape. But she has confessed.''

This was a dire blow to poor Elizabeth,
who had relied with firmness upon
Justine's innocence. ``Alas!'' said
she, ``how shall I ever again believe in
human benevolence? Justine, whom I
loved and esteemed as my sister, how
could she put on those smiles of innocence
only to betray; her mild eyes
seemed incapable of any severity or ill-humour,
and yet she has committed a
murder.''

Soon after we heard that the poor
victim had expressed a wish to see my
cousin. My father wished her not to
go; but said, that he left it to her own
judgment and feelings to decide.
%%110%%
``Yes,'' said Elizabeth, ``I will go,
although she is guilty; and you, Victor,
shall accompany me: I cannot go alone.''
The idea of this visit was torture to
me, yet I could not refuse.

We entered the gloomy prison-chamber,
and beheld Justine sitting on some
straw at the further end; her hands
were manacled, and her head rested on
her knees. She rose on seeing us enter;
and when we were left alone with
her, she threw herself at the feet of
Elizabeth, weeping bitterly. My cousin
wept also.

``Oh, Justine!'' said she, ``why did
you rob me of my last consolation. I
relied on your innocence; and although
I was then very wretched, I was
not so miserable as I am now.''

``And do you also believe that I am
so very, very wicked? Do you also
join with my enemies to crush me?''
Her voice was suffocated with sobs.

``Rise, my poor girl,'' said Elizabeth,
``why do you kneel, if you are
innocent? I am not one of your enemies;
I believed you guiltless, notwithstanding
every evidence, until I heard
that you had yourself declared your
guilt. That report, you say, is false;
and be assured, dear Justine, that nothing
can shake my confidence in you
for a moment, but your own confession.''

``I did confess; but I confessed a
lie. I confessed, that I might obtain
absolution; but now that falsehood lies
heavier at my heart than all my other
sins. The God of heaven forgive me!
Ever since I was condemned, my confessor
has besieged me; he threatened
%%111%%
and menaced, until I almost began to
think that I was the monster that he
said I was. He threatened excommunication
and hell fire in my last moments,
if I continued obdurate. Dear
lady, I had none to support me; all
looked on me as a wretch doomed to
ignominy and perdition. What could
I do? In an evil hour I subscribed to
a lie; and now only am I truly miserable.''

She paused, weeping, and then continued --- ``I
thought with horror, my
sweet lady, that you should believe
your Justine, whom your blessed aunt
had so highly honoured, and whom
you loved, was a creature capable of a
crime which none but the devil himself
could have perpetrated. Dear William!
dearest blessed child! I soon
shall see you again in heaven, where
we shall all be happy; and that consoles
me, going as I am to suffer ignominy
and death.''

``Oh, Justine! forgive me for having
for one moment distrusted you. Why
did you confess? But do not mourn,
my dear girl; I will every where proclaim
your innocence, and force belief.
Yet you must die; you, my playfellow,
my companion, my more than sister.
I never can survive so horrible a misfortune.''

``Dear, sweet Elizabeth, do not
weep. You ought to raise me with
thoughts of a better life, and elevate
me from the petty cares of this world
of injustice and strife. Do not you,
excellent friend, drive me to despair.''

``I will try to comfort you; but this,
I fear, is an evil too deep and poignant
%%112%%
to admit of consolation, for there is no
hope. Yet heaven bless thee, my dearest
Justine, with resignation, and a confidence
elevated beyond this world.
Oh! how I hate its shews and mockeries!
when one creature is murdered,
another is immediately deprived of life
in a slow torturing manner; then the
executioners, their hands yet reeking
with the blood of innocence, believe
that they have done a great deed. They
call this \emph{retribution}. Hateful name!
When that word is pronounced, I know
greater and more horrid punishments
are going to be inflicted than the gloomiest
tyrant has ever invented to satiate
his utmost revenge. Yet this is not
consolation for you, my Justine, unless
indeed that you may glory in escaping
from so miserable a den. Alas! I
would I were in peace with my aunt
and my lovely William, escaped from a
world which is hateful to me, and the
visages of men which I abhor.''

Justine smiled languidly. ``This,
dear lady, is despair, and not resignation.
I must not learn the lesson that
you would teach me. Talk of something
else, something that will bring
peace, and not increase of misery.''

During this conversation I had retired
to a corner of the prison-room, where
I could conceal the horrid anguish that
possessed me. Despair! Who dared
talk of that? The poor victim, who
on the morrow was to pass the dreary
boundary between life and death, felt
not as I did, such deep and bitter
agony. I gnashed my teeth, and
ground them together, uttering a groan
that came from my inmost soul.
%%113%%
Justine started. When she saw who it
was, she approached me, and said,
``Dear Sir, you are very kind to visit
me; you, I hope, do not believe that I
am guilty.''

I could not answer. ``No, Justine,''
said Elizabeth; ``he is more convinced
of your innocence than I was; for
even when he heard that you had confessed,
he did not credit it.''

``I truly thank him. In these last
moments I feel the sincerest gratitude
towards those who think of me with
kindness. How sweet is the affection
of others to such a wretch as I am!
It removes more than half my misfortune;
and I feel as if I could die in
peace, now that my innocence is acknowledged
by you, dear lady, and
your cousin.''

Thus the poor sufferer tried to comfort
others and herself. She indeed
gained the resignation she desired.
But I, the true murderer, felt the never-dying
worm alive in my bosom, which
allowed of no hope or consolation.
Elizabeth also wept, and was unhappy;
but her's also was the misery of innocence,
which, like a cloud that passes
over the fair moon, for a while hides,
but cannot tarnish its brightness. Anguish
and despair had penetrated into
the core of my heart; I bore a hell
within me, which nothing could extinguish.
We staid several hours with
Justine; and it was with great difficulty
that Elizabeth could tear herself
away. ``I wish,'' cried she, ``that I
were to die with you; I cannot live in
this world of misery.''

Justine assumed an air of
%%114%%
cheerfulness, while she with difficulty repressed
her bitter tears. She embraced Elizabeth,
and said, in a voice of half-suppressed
emotion, ``Farewell, sweet lady,
dearest Elizabeth, my belov\-ed and only
friend; may heaven in its bounty bless
and preserve you; may this be the last
misfortune that you will ever suffer.
Live, and be happy, and make others
so.''

As we returned, Elizabeth said, ``You
know not, my dear Victor, how much I
am relieved, now that I trust in the innocence
of this unfortunate girl. I
never could again have known peace,
if I had been deceived in my reliance
on her. For the moment that I did
believe her guilty, I felt an anguish
that I could not have long sustained.
Now my heart is lightened. The innocent
suffers; but she whom I thought
amiable and good has not betrayed the
trust I reposed in her, and I am consoled.''

Amiable cousin! such were your
thoughts, mild and gentle as your own
dear eyes and voice. But I --- I was a
wretch, and none ever conceived of the
misery that I then endured.

%%115%%
%%116%%
\namedpart{Volume II}

\namedchapter{Chapter I}

\textsc{Nothing} is more painful to the
human mind, than, after the feelings
have been worked up by a quick succession
of events, the dead calmness of
inaction and certainty which follows,
and deprives the soul both of hope and
fear. Justine died; she rested; and I
was alive. The blood flowed freely in
my veins, but a weight of despair and
remorse pressed on my heart, which
nothing could remove. Sleep fled from
my eyes; I wandered like an evil spirit,
for I had committed deeds of mischief
beyond description horrible, and more,
much more, (I persuaded myself) was
yet behind. Yet my heart overflowed
with kindness, and the love of virtue.
I had begun life with benevolent intentions,
and thirsted for the moment
when I should put them in practice,
and make myself useful to my fellow-beings.
Now all was blasted: instead
of that serenity of conscience, which
allowed me to look back upon the past
with self-satisfaction, and from thence
to gather promise of new hopes, I was
%%117%%
seized by remorse and the sense of
guilt, which hurried me away to a hell
of intense tortures, such as no language
can describe.

This state of mind preyed upon my
health, which had entirely recovered
from the first shock it had sustained.
I shunned the face of man; all sound
of joy or complacency was torture to
me; solitude was my only consolation --- deep,
dark, death-like solitude.

My father observed with pain the alteration
perceptible in my disposition and
habits, and endeavoured to reason with
me on the folly of giving way to immoderate
grief. ``Do you think, Victor,''
said he, ``that I do not suffer also? No
one could love a child more than I
loved your brother;'' (tears came into
his eyes as he spoke); ``but is it not a
duty to the survivors, that we should
refrain from augmenting their unhappiness
by an appearance of immoderate
grief? It is also a duty owed to yourself;
for excessive sorrow prevents improvement
or enjoyment, or even the
discharge of daily usefulness, without
which no man is fit for society.''

This advice, although good, was totally
inapplicable to my case; I should
have been the first to hide my grief, and
console my friends, if remorse had not
mingled its bitterness with my other
sensations. Now I could only answer
my father with a look of despair, and endeavour
to hide myself from his view.

About this time we retired to our house
at Belrive. This change was particularly
agreeable to me. The shutting of the
gates regularly at ten o'clock, and the
impossibility of remaining on the lake
%%118%%
after that hour, had rendered our
residence within the walls of Geneva
very irksome to me. I was now free.
Often, after the rest of the family had
retired for the night, I took the boat,
and passed many hours upon the water.
Sometimes, with my sails set, I was carried
by the wind; and sometimes, after
rowing into the middle of the lake, I
left the boat to pursue its own course,
and gave way to my own miserable reflections.
I was often tempted, when all
was at peace around me, and I the only
unquiet thing that wandered restless in
a scene so beautiful and heavenly, if
I except some bat, or the frogs, whose
harsh and interrupted croaking was
heard only when I approached the
shore --- often, I say, I was tempted to
plunge into the silent lake, that the
waters might close over me and my calamities
for ever. But I was restrained,
when I thought of the heroic and suffering
Elizabeth, whom I tenderly loved,
and whose existence was bound up in
mine. I thought also of my father,
and surviving brother: should I by my
base desertion leave them exposed and
unprotected to the malice of the fiend
whom I had let loose among them?

At these moments I wept bitterly,
and wished that peace would revisit my
mind only that I might afford them
consolation and happiness. But that
could not be. Remorse extinguished
every hope. I had been the author of unalterable
evils; and I lived in daily fear,
lest the monster whom I had created
should perpetrate some new wickedness.
I had an obscure feeling that all
was not over, and that he would still
%%119%%
commit some signal crime, which by
its enormity should almost efface the
recollection of the past. There was
always scope for fear, so long as any
thing I loved remained behind. My
abhorrence of this fiend cannot be conceived.
When I thought of him, I
gnashed my teeth, my eyes became inflamed,
and I ardently wished to extinguish
that life which I had so thoughtlessly
bestowed. When I reflected on
his crimes and malice, my hatred and
revenge burst all bounds of moderation.
I would have made a pilgrimage to the
highest peak of the Andes, could I, when
there, have precipitated him to their
base. I wished to see him again, that
I might wreak the utmost extent of
anger on his head, and avenge the
deaths of William and Justine.

Our house was the house of mourning.
My father's health was deeply
shaken by the horror of the recent
events. Elizabeth was sad and desponding;
she no longer took delight in her ordinary
occupations; all pleasure seemed
to her sacrilege toward the dead; eternal
woe and tears she then thought was
the just tribute she should pay to innocence
so blasted and destroyed. She
was no longer that happy creature, who
in earlier youth wandered with me on
the banks of the lake, and talked with
ecstacy of our future prospects. She
had become grave, and often conversed
of the inconstancy of fortune, and the
instability of human life.

``When I reflect, my dear cousin,''
said she, ``on the miserable death of Justine
Moritz, I no longer see the world
and its works as they before appeared
%%120%%
to me. Before, I looked upon the accounts
of vice and injustice, that I read
in books or heard from others, as tales
of ancient days, or imaginary evils; at
least they were remote, and more familiar
to reason than to the imagination;
but now misery has come home, and
men appear to me as monsters thirsting
for each other's blood. Yet I am
certainly unjust. Every body believed
that poor girl to be guilty; and if she
could have committed the crime for
which she suffered, assuredly she would
have been the most depraved of human
creatures. For the sake of a few jewels,
to have murdered the son of her benefactor
and friend, a child whom she had
nursed from its birth, and appeared to
love as if it had been her own! I
could not consent to the death of any
human being; but certainly I should
have thought such a creature unfit to
remain in the society of men. Yet she
was innocent. I know, I feel she was
innocent; you are of the same opinion,
and that confirms me. Alas! Victor,
when falsehood can look so like the
truth, who can assure themselves of
certain happiness? I feel as if I were
walking on the edge of a precipice, towards
which thousands are crowding,
and endeavouring to plunge me into
the abyss. William and Justine were
assassinated, and the murderer escapes;
he walks about the world free, and perhaps
respected. But even if I were
condemned to suffer on the scaffold
for the same crimes, I would not change
places with such a wretch.''

I listened to this discourse with the
extremest agony. I, not in deed, but
%%121%%
in effect, was the true murderer. Elizabeth
read my anguish in my countenance,
and kindly taking my hand
said, ``My dearest cousin, you must
calm yourself. These events have
affected me, God knows how deeply;
but I am not so wretched as you are.
There is an expression of despair, and
sometimes of revenge, in your countenance,
that makes me tremble. Be calm,
my dear Victor; I would sacrifice my
life to your peace. We surely shall be
happy: quiet in our native country, and
not mingling in the world, what can disturb
our tranquillity?''

She shed tears as she said this, distrusting
the very solace that she gave;
but at the same time she smiled, that
she might chase away the fiend that
lurked in my heart. My father, who
saw in the unhappiness that was painted
in my face only an exaggeration of that
sorrow which I might naturally feel,
thought that an amusement suited to
my taste would be the best means of
restoring to me my wonted serenity.
It was from this cause that he had removed
to the country; and, induced by
the same motive, he now proposed that
we should all make an excursion to
the valley of Chamounix. I had been
there before, but Elizabeth and Ernest
never had; and both had often expressed
an earnest desire to see the scenery of
this place, which had been described
to them as so wonderful and sublime.
Accordingly we departed from Geneva
on this tour about the middle of the
month of August, nearly two months
after the death of Justine.

The weather was uncommonly fine;
%%122%%
and if mine had been a sorrow to be
chased away by any fleeting circumstance,
this excursion would certainly
have had the effect intended by my father.
As it was, I was somewhat interested
in the scene; it sometimes lulled,
although it could not extinguish my
grief. During the first day we travelled
in a carriage. In the morning
we had seen the mountains at a distance,
towards which we gradually advanced.
We perceived that the valley
through which we wound, and which
was formed by the river Arve, whose
course we followed, closed in upon us
by degrees; and when the sun had set,
we beheld immense mountains and precipices
overhanging us on every side,
and heard the sound of the river raging
among rocks, and the dashing of water-falls
around.
%%123%%

The next day we pursued our journey
upon mules; and as we ascended
still higher, the valley assumed a more
magnificent and astonishing character.
Ruined castles hanging on the precipices
of piny mountains; the impetuous
Arve, and cottages every here and
there peeping forth from among the
trees, formed a scene of singular beauty.
But it was augmented and rendered
sublime by the mighty Alps, whose
white and shining pyramids and domes
towered above all, as belonging to another
earth, the habitations of another
race of beings.

We passed the bridge of Pelissier,
where the ravine, which the river forms,
opened before us, and we began to ascend
the mountain that overhangs it.
Soon after we entered the valley of
Chamounix. This valley is more wonderful
and sublime, but not so beautiful
and picturesque as that of Servox,
through which we had just passed.
The high and snowy mountains were
its immediate boundaries; but we saw
no more ruined castles and fertile fields.
Immense glaciers approached the road;
we heard the rumbling thunder of the
falling avalanche, and marked the
%%124%%
smoke of its passage. Mont Blânc,
the supreme and magnificent Mont
Blânc, raised itself from the surrounding
\emph{aiguilles}, and its tremendous \emph{dome}
overlooked the valley.

During this journey, I sometimes
joined Elizabeth, and exerted myself to
point out to her the various beauties of
the scene. I often suffered my mule to
lag behind, and indulged in the misery
of reflection. At other times I spurred
on the animal before my companions,
that I might forget them, the world,
and, more than all, myself. When at
a distance, I alighted, and threw myself
on the grass, weighed down by
horror and despair. At eight in the
evening I arrived at Chamounix. My
father and Elizabeth were very much
fatigued; Ernest, who accompanied
us, was delighted, and in high spirits:
the only circumstance that detracted
from his pleasure was the south wind,
and the rain it seemed to promise for
the next day.

We retired early to our apartments,
but not to sleep; at least I did not.
I remained many hours at the window,
watching the pallid lightning that played
above Mont Blânc, and listening to
the rushing of the Arve, which ran
below my window.
%%125%%

\namedchapter{Chapter II}

\textsc{The} next day, contrary to the prognostications
of our guides, was fine, although
clouded. We visited the source
of the Arveiron, and rode about the
valley until evening. These sublime
and magnificent scenes afforded me the
greatest consolation that I was capable
of receiving. They elevated me from
all littleness of feeling; and although
they did not remove my grief, they
subdued and tranquillized it. In some
degree, also, they diverted my mind
from the thoughts over which it had
brooded for the last month. I returned
in the evening, fatigued, but less unhappy,
and conversed with my family
with more cheerfulness than had been
my custom for some time. My father
was pleased, and Elizabeth overjoyed.
``My dear cousin,'' said she, ``you see
what happiness you diffuse when you
are happy; do not relapse again!''

The following morning the rain
poured down in torrents, and thick
mists hid the summits of the mountains.
I rose early, but felt unusually
melancholy. The rain depressed me;
my old feelings recurred, and I was
miserable. I knew how disappointed
my father would be at this sudden
change, and I wished to avoid him until
I had recovered myself so far as to
%%126%%
be enabled to conceal those feelings
that overpowered me. I knew that
they would remain that day at the inn;
and as I had ever inured myself to
rain, moisture, and cold, I resolved to
go alone to the summit of Montanvert.
I remembered the effect that the view
of the tremendous and ever-moving
glacier had produced upon my mind
when I first saw it. It had then filled
me with a sublime ecstacy that gave
wings to the soul, and allowed it to
soar from the obscure world to light
and joy. The sight of the awful and
majestic in nature had indeed always
the effect of solemnizing my mind, and
causing me to forget the passing cares
of life. I determined to go alone, for I
was well acquainted with the path, and
the presence of another would destroy
the solitary grandeur of the scene.

The ascent is precipitous, but the
path is cut into continual and short
windings, which enable you to surmount
the perpendicularity of the
mountain. It is a scene terrifically
desolate. In a thousand spots the traces
of the winter avalanche may be perceived,
where trees lie broken and
strewed on the ground; some entirely
destroyed, others bent, leaning upon the
jutting rocks of the mountain, or transversely
upon other trees. The path, as
you ascend higher, is intersected by ravines
of snow, down which stones continually
roll from above; one of them
is particularly dangerous, as the slightest
sound, such as even speaking in a
loud voice, produces a concussion of
air sufficient to draw destruction upon
%%127%%
the head of the speaker. The pines
are not tall or luxuriant, but they are
sombre, and add an air of severity to
the scene. I looked on the valley beneath;
vast mists were rising from the
rivers which ran through it, and curling
in thick wreaths around the opposite
mountains, whose summits were
hid in the uniform clouds, while rain
poured from the dark sky, and added
to the melancholy impression I received
from the objects around me. Alas!
why does man boast of sensibilities superior
to those apparent in the brute;
it only renders them more necessary
beings. If our impulses were confined
to hunger, thirst, and desire, we might
be nearly free; but now we are moved
by every wind that blows, and a chance
word or scene that that word may convey
to us.

\vspace*{0.5em plus 3em}
{\noindent\small
\hspace*{1.5em}We rest; a dream has power to poison sleep.\\
\hspace*{2.5em} We rise; one wand'ring thought pollutes the day.\\
\hspace*{1.5em}We feel, conceive, or reason; laugh, or weep,\\
\hspace*{2.5em} Embrace fond woe, or cast our cares away;\\
\hspace*{1.5em}It is the same: for, be it joy or sorrow,\\
\hspace*{2.5em} The path of its departure still is free.\\
\hspace*{1.5em}Man's yesterday may ne'er be like his morrow;\\
\hspace*{2.5em} Nought may endure but mutability!}
\vspace{0.5em plus 1em}

It was nearly noon when I arrived at
the top of the ascent. For some time
I sat upon the rock that overlooks the
sea of ice. A mist covered both that
and the surrounding mountains. Presently
a breeze dissipated the cloud,
and I descended upon the glacier. The
surface is very uneven, rising like the
waves of a troubled sea, descending
low, and interspersed by rifts that sink
deep. The field of ice is almost a
%%128%%
league in width, but I spent nearly two
hours in crossing it. The opposite
mountain is a bare perpendicular rock.
From the side where I now stood Montanvert
was exactly opposite, at the distance
of a league; and above it rose
Mont Blânc, in awful majesty. I remained
in a recess of the rock, gazing
on this wonderful and stupendous scene.
The sea, or rather the vast river of ice,
wound among its dependent mountains,
whose aërial summits hung over its recesses.
Their icy and glittering peaks
shone in the sunlight over the clouds.
My heart, which was before sorrowful,
now swelled with something like joy;
I exclaimed --- ``Wandering spirits, if
indeed ye wander, and do not rest in
your narrow beds, allow me this faint
happiness, or take me, as your companion,
away from the joys of life.''

As I said this, I suddenly beheld the
figure of a man, at some distance, advancing
towards me with superhuman
speed. He bounded over the crevices
in the ice, among which I had walked
%%129%%
with caution; his stature also, as he
approached, seemed to exceed that of
man. I was troubled: a mist came
over my eyes, and I felt a faintness
seize me; but I was quickly restored by
the cold gale of the mountains. I perceived,
as the shape came nearer, (sight
tremendous and abhorred!) that it was
the wretch whom I had created. I
trembled with rage and horror, resolving
to wait his approach, and then
close with him in mortal combat. He
approached; his countenance bespoke
bitter anguish, combined with disdain
and malignity, while its unearthly ugliness
rendered it almost too horrible
for human eyes. But I scarcely observed
this; anger and hatred had at
first deprived me of utterance, and I
recovered only to overwhelm him with
words expressive of furious detestation
and contempt.

``Devil!'' I exclaimed, ``do you
dare approach me? and do not you fear
the fierce vengeance of my arm wreaked
on your miserable head? Begone, vile
insect! or rather stay, that I may trample
you to dust! and, oh, that I could,
with the extinction of your miserable
existence, restore those victims whom
you have so diabolically murdered!''

``I expected this reception,'' said the
dæmon. ``All men hate the wretched;
how then must I be hated, who am miserable
beyond all living things! Yet
you, my creator, detest and spurn me,
thy creature, to whom thou art bound by
ties only dissoluble by the annihilation
of one of us. You purpose to kill me.
How dare you sport thus with life?
Do your duty towards me, and I will
%%130%%
do mine towards you and the rest of
mankind. If you will comply with my
conditions, I will leave them and you
at peace; but if you refuse, I will glut
the maw of death, until it be satiated
with the blood of your remaining
friends.''

``Abhorred monster! fiend that
thou art! the tortures of hell are too
mild a vengeance for thy crimes.
Wretched devil! you reproach me with
your creation; come on then, that I
may extinguish the spark which I so
negligently bestowed.''

My rage was without bounds; I
sprang on him, impelled by all the
feelings which can arm one being
against the existence of another.

He easily eluded me, and said,

``Be calm! I entreat you to hear me,
before you give vent to your hatred on
my devoted head. Have I not suffered
enough, that you seek to increase my
misery? Life, although it may only be
an accumulation of anguish, is dear to
me, and I will defend it. Remember,
thou hast made me more powerful than
thyself; my height is superior to thine;
my joints more supple. But I will not
be tempted to set myself in opposition
to thee. I am thy creature, and I will
be even mild and docile to my natural
lord and king, if thou wilt also perform
thy part, the which thou owest me.
Oh, Frankenstein, be not equitable to
every other, and trample upon me
alone, to whom thy justice, and even
thy clemency and affection, is most
due. Remember, that I am thy creature:
I ought to be thy Adam; but I
am rather the fallen angel, whom thou
%%131%%
drivest from joy for no misdeed. Every
where I see bliss, from which I alone
am irrevocably excluded. I was benevolent
and good; misery made me a
fiend. Make me happy, and I shall
again be virtuous.''

``Begone! I will not hear you.
There can be no community between
you and me; we are enemies. Begone,
or let us try our strength in a fight, in
which one must fall.''

``How can I move thee? Will no
entreaties cause thee to turn a favourable
eye upon thy creature, who implores
thy goodness and compassion?
Believe me, Frankenstein: I was benevolent;
my soul glowed with love and
humanity: but am I not alone, miserably
alone? You, my creator, abhor
me; what hope can I gather from
your fellow-creatures, who owe me nothing?
they spurn and hate me. The
desert mountains and dreary glaciers
are my refuge. I have wandered
here many days; the caves of ice,
which I only do not fear, are a dwelling
to me, and the only one which
man does not grudge. These bleak
skies I hail, for they are kinder to me
than your fellow-beings. If the multitude
of mankind knew of my existence,
they would do as you do, and
arm themselves for my destruction.
Shall I not then hate them who abhor
me? I will keep no terms with my
enemies. I am miserable, and they
shall share my wretchedness. Yet it
is in your power to recompense me,
and deliver them from an evil which it
only remains for you to make so
great, that not only you and your
%%132%%
family, but thousands of others, shall be
swallowed up in the whirlwinds of its
rage. Let your compassion be moved,
and do not disdain me. Listen to my
tale: when you have heard that, abandon
or commiserate me, as you shall
judge that I deserve. But hear me.
The guilty are allowed, by human
laws, bloody as they may be, to speak in
their own defence before they are condemned.
Listen to me, Frankenstein.
You accuse me of murder; and yet
you would, with a satisfied conscience,
destroy your own creature. Oh, praise
the eternal justice of man! Yet I ask
you not to spare me: listen to me; and
then, if you can, and if you will, destroy
the work of your hands.''

``Why do you call to my remembrance
circumstances of which I shudder
to reflect, that I have been the miserable
origin and author? Cursed be
the day, abhorred devil, in which you
first saw light! Cursed (although I
curse myself) be the hands that formed
you! You have made me wretched
beyond expression. You have left me
no power to consider whether I am
just to you, or not. Begone! relieve
me from the sight of your detested
form.''

``Thus I relieve thee, my creator,''
he said, and placed his hated hands
before my eyes, which I flung from me
with violence; ``thus I take from thee
a sight which you abhor. Still thou
canst listen to me, and grant me thy
compassion. By the virtues that I
once possessed, I demand this from you.
Hear my tale; it is long and strange,
and the temperature of this place is
%%133%%
not fitting to your fine sensations; come
to the hut upon the mountain. The sun
is yet high in the heavens; before it descends
to hide itself behind yon snowy
precipices, and illuminate another
world, you will have heard my story,
and can decide. On you it rests, whether
I quit for ever the neighbourhood
of man, and lead a harmless life, or
become the scourge of your fellow-creatures,
and the author of your own
speedy ruin.''

As he said this, he led the way across
the ice: I followed. My heart was
full, and I did not answer him; but,
as I proceeded, I weighed the various
arguments that he had used, and determined
at least to listen to his tale. I
was partly urged by curiosity, and
compassion confirmed my resolution.
I had hitherto supposed him to be the
murderer of my brother, and I eagerly
sought a confirmation or denial of
this opinion. For the first time, also,
I felt what the duties of a creator towards
his creature were, and that I
ought to render him happy before I
complained of his wickedness. These
motives urged me to comply with his
demand. We crossed the ice, therefore,
and ascended the opposite rock.
The air was cold, and the rain again
began to descend: we entered the hut,
the fiend with an air of exultation, I
with a heavy heart, and depressed spirits.
But I consented to listen; and,
seating myself by the fire which my
odious companion had lighted, he thus
began his tale.
%%134%%

\namedchapter{Chapter III}

``\textsc{It} is with considerable difficulty that
I remember the original æra of my being:
all the events of that period appear
confused and indistinct. A strange multiplicity
of sensations seized me, and I
saw, felt, heard, and smelt, at the same
time; and it was, indeed, a long time before
I learned to distinguish between the
operations of my various senses. By
degrees, I remember, a stronger light
pressed upon my nerves, so that I was
obliged to shut my eyes. Darkness then
came over me, and troubled me; but
hardly had I felt this, when, by opening
my eyes, as I now suppose, the light
poured in upon me again. I walked,
and, I believe, descended; but I presently
found a great alteration in my
sensations. Before, dark and opaque
bodies had surrounded me, impervious
to my touch or sight; but I now found
that I could wander on at liberty, with
no obstacles which I could not either
surmount or avoid. The light became
more and more oppressive to me; and,
the heat wearying me as I walked, I
sought a place where I could receive
shade. This was the forest near Ingolstadt;
and here I lay by the side
of a brook resting from my fatigue,
until I felt tormented by hunger and
thirst. This roused me from my nearly
%%135%%
dormant state, and I ate some berries
which I found hanging on the trees, or
lying on the ground. I slaked my
thirst at the brook; and then lying
down, was overcome by sleep.

``It was dark when I awoke; I felt
cold also, and half-frightened as it were
instinctively, finding myself so desolate.
Before I had quitted your apartment,
on a sensation of cold, I had covered
myself with some clothes; but
these were insufficient to secure me
from the dews of night. I was a poor,
helpless, miserable wretch; I knew,
and could distinguish, nothing; but,
%%136%%
feeling pain invade me on all sides, I
sat down and wept.

``Soon a gentle light stole over the
heavens, and gave me a sensation of
pleasure. I started up, and beheld a
radiant form rise from among the trees.
I gazed with a kind of wonder. It
moved slowly, but it enlightened my
path; and I again went out in search of
berries. I was still cold, when under
one of the trees I found a huge cloak,
with which I covered myself, and sat
down upon the ground. No distinct
ideas occupied my mind; all was confused.
I felt light, and hunger, and
thirst, and darkness; innumerable
sounds rung in my ears, and on all
sides various scents saluted me: the
only object that I could distinguish was
the bright moon, and I fixed my eyes
on that with pleasure.

``Several changes of day and night
passed, and the orb of night had greatly
lessened when I began to distinguish
my sensations from each other. I gradually
saw plainly the clear stream
that supplied me with drink, and the
trees that shaded me with their foliage.
I was delighted when I first discovered
that a pleasant sound, which often
saluted my ears, proceeded from the
throats of the little winged animals who
had often intercepted the light from
my eyes. I began also to observe, with
greater accuracy, the forms that surrounded
me, and to perceive the boundaries
of the radiant roof of light which
canopied me. Sometimes I tried to imitate
the pleasant songs of the birds, but
was unable. Sometimes I wished to
%%137%%
express my sensations in my own mode,
but the uncouth and inarticulate sounds
which broke from me frightened me
into silence again.

``The moon had disappeared from
the night, and again, with a lessened
form, shewed itself, while I still remained
in the forest. My sensations
had, by this time, become distinct, and
my mind received every day additional
ideas. My eyes became accustomed
to the light, and to perceive objects
in their right forms; I distinguished
the insect from the herb, and, by degrees,
one herb from another. I found
that the sparrow uttered none but
harsh notes, whilst those of the blackbird
and thrush were sweet and enticing.

``One day, when I was oppressed by
cold, I found a fire which had been left
by some wandering beggars, and was
overcome with delight at the warmth I
experienced from it. In my joy I thrust
my hand into the live embers, but
quickly drew it out again with a cry of
pain. How strange, I thought, that the
same cause should produce such opposite
effects! I examined the materials
of the fire, and to my joy found it to be
composed of wood. I quickly collected
some branches; but they were wet, and
would not burn. I was pained at this,
and sat still watching the operation of
the fire. The wet wood which I had
placed near the heat dried, and itself
became inflamed. I reflected on this;
and, by touching the various branches,
I discovered the cause, and busied myself
in collecting a great quantity of
wood, that I might dry it, and have a
%%138%%
plentiful supply of fire. When night
came on, and brought sleep with it, I
was in the greatest fear lest my fire
should be extinguished. I covered it
carefully with dry wood and leaves,
and placed wet branches upon it; and
then, spreading my cloak, I lay on the
ground, and sunk into sleep.

``It was morning when I awoke,
and my first care was to visit the fire.
I uncovered it, and a gentle breeze
quickly fanned it into a flame. I observed
this also, and contrived a fan of
branches, which roused the embers
when they were nearly extinguished.
When night came again, I found, with
pleasure, that the fire gave light as
well as heat; and that the discovery
of this element was useful to me in my
food; for I found some of the offals
that the travellers had left had been
roasted, and tasted much more savoury
than the berries I gathered from the
trees. I tried, therefore, to dress my
food in the same manner, placing it on
the live embers. I found that the berries
were spoiled by this operation, and
the nuts and roots much improved.

``Food, however, became scarce; and
I often spent the whole day searching
in vain for a few acorns to assuage the
pangs of hunger. When I found this,
I resolved to quit the place that I had
hitherto inhabited, to seek for one
where the few wants I experienced
would be more easily satisfied. In this
emigration, I exceedingly lamented the
loss of the fire which I had obtained
through accident, and knew not how to
re-produce it. I gave several hours to
the serious consideration of this
%%139%%
difficulty; but I was obliged to relinquish
all attempt to supply it; and, wrapping
myself up in my cloak, I struck across
the wood towards the setting sun. I
passed three days in these rambles, and
at length discovered the open country.
A great fall of snow had taken place
the night before, and the fields were
of one uniform white; the appearance
was disconsolate, and I found my feet
chilled by the cold damp substance that
covered the ground.

``It was about seven in the morning,
and I longed to obtain food
and shelter; at length I perceived
a small hut, on a rising ground,
which had doubtless been built for
the convenience of some shepherd.
This was a new sight to me; and I
examined the structure with great curiosity.
Finding the door open, I entered.
An old man sat in it, near a
fire, over which he was preparing his
breakfast. He turned on hearing a
noise; and, perceiving me, shrieked
loudly, and, quitting the hut, ran across
the fields with a speed of which
his debilitated form hardly appeared
capable. His appearance, different
from any I had ever before seen, and
his flight, somewhat surprised me. But
I was enchanted by the appearance of
the hut: here the snow and rain could
not penetrate; the ground was dry; and
it presented to me then as exquisite
and divine a retreat as Pandæmonium
appeared to the dæmons of hell after
their sufferings in the lake of fire. I
greedily devoured the remnants of the
shepherd's breakfast, which consisted
of bread, cheese, milk, and wine; the
%%140%%
latter, however, I did not like. Then
overcome by fatigue, I lay down among
some straw, and fell asleep.

``It was noon when I awoke; and,
allured by the warmth of the sun, which
shone brightly on the white ground, I
determined to recommence my travels;
and, depositing the remains of the peasant's
breakfast in a wallet I found, I
proceeded across the fields for several
hours, until at sunset I arrived at a
village. How miraculous did this appear!
the huts, the neater cottages, and
stately houses, engaged my admiration
by turns. The vegetables in the gardens,
the milk and cheese that I saw
placed at the windows of some of the
cottages, allured my appetite. One of
the best of these I entered; but I had
hardly placed my foot within the door,
before the children shrieked, and one
of the women fainted. The whole village
was roused; some fled, some attacked
me, until, grievously bruised by
stones and many other kinds of missile
weapons, I escaped to the open country,
and fearfully took refuge in a low
hovel, quite bare, and making a wretched
appearance after the palaces I had
beheld in the village. This hovel,
however, joined a cottage of a neat and
pleasant appearance; but, after my late
dearly-bought experience, I dared not
enter it. My place of refuge was constructed
of wood, but so low, that I
could with difficulty sit upright in it.
No wood, however, was placed on the
earth, which formed the floor, but it
was dry; and although the wind entered
it by innumerable chinks, I found
it an agreeable asylum from the snow
and rain.
%%141%%

``Here then I retreated, and lay
down, happy to have found a shelter,
however miserable, from the inclemency
of the season, and still more from the
barbarity of man.

``As soon as morning dawned, I
crept from my kennel, that I might view
the adjacent cottage, and discover if I
could remain in the habitation I had
found. It was situated against the back
of the cottage, and surrounded on the
sides which were exposed by a pig-stye
and a clear pool of water. One part was
open, and by that I had crept in; but
now I covered every crevice by which I
might be perceived with stones and
wood, yet in such a manner that I
might move them on occasion to pass
out: all the light I enjoyed came
through the stye, and that was sufficient
for me.

``Having thus arranged my dwelling,
and carpeted it with clean straw, I
retired; for I saw the figure of a man
at a distance, and I remembered too
well my treatment the night before, to
trust myself in his power. I had first,
however, provided for my sustenance
for that day, by a loaf of coarse bread,
which I purloined, and a cup with
which I could drink, more conveniently
than from my hand, of the pure water
which flowed by my retreat. The floor
was a little raised, so that it was kept
perfectly dry, and by its vicinity to the
chimney of the cottage it was tolerably
warm.

``Being thus provided, I resolved to
reside in this hovel, until something
should occur which might alter my determination.
It was indeed a paradise,
compared to the bleak forest, my former
%%142%%
residence, the rain-dropping branches,
and dank earth. I ate my breakfast
with pleasure, and was about to remove
a plank to procure myself a little water,
when I heard a step, and, looking
through a small chink, I beheld a young
creature, with a pail on her head, passing
before my hovel. The girl was
young and of gentle demeanour, unlike
what I have since found cottagers and
farm-house servants to be. Yet she
was meanly dressed, a coarse blue petticoat
and a linen jacket being her
only garb; her fair hair was plaited,
but not adorned; she looked patient,
yet sad. I lost sight of her; and in
about a quarter of an hour she returned,
bearing the pail, which was now
partly filled with milk. As she walked
along, seemingly incommoded by the
burden, a young man met her, whose
countenance expressed a deeper despondence.
Uttering a few sounds with
an air of melancholy, he took the pail
from her head, and bore it to the cottage
himself. She followed, and they
disappeared. Presently I saw the young
man again, with some tools in his hand,
cross the field behind the cottage; and
the girl was also busied, sometimes in
the house, and sometimes in the yard.

``On examining my dwelling, I found
that one of the windows of the cottage
had formerly occupied a part of it,
but the panes had been filled up with
wood. In one of these was a small and
almost imperceptible chink, through
which the eye could just penetrate.
Through this crevice, a small room
was visible, white-washed and clean,
but very bare of furniture. In one
%%143%%
corner, near a small fire, sat an old
man, leaning his head on his hands in a
disconsolate attitude. The young girl
was occupied in arranging the cottage;
but presently she took something out
of a drawer, which employed her hands,
and she sat down beside the old man,
who, taking up an instrument, began
to play, and to produce sounds, sweeter
than the voice of the thrush or the nightingale.
It was a lovely sight, even to
me, poor wretch! who had never beheld
aught beautiful before. The silver
hair and benevolent countenance of
the aged cottager, won my reverence;
while the gentle manners of the girl
enticed my love. He played a sweet
mournful air, which I perceived drew
tears from the eyes of his amiable companion,
of which the old man took no
notice, until she sobbed audibly; he
then pronounced a few sounds, and the
fair creature, leaving her work, knelt
at his feet. He raised her, and smiled
with such kindness and affection, that
I felt sensations of a peculiar and over-powering
nature: they were a mixture
of pain and pleasure, such as I had
never before experienced, either from
hunger or cold, warmth or food; and I
withdrew from the window, unable to
bear these emotions.

``Soon after this the young man returned,
bearing on his shoulders a load
of wood. The girl met him at the door,
helped to relieve him of his burden, and,
taking some of the fuel into the cottage,
placed it on the fire; then she and the
youth went apart into a nook of the
cottage, and he shewed her a large loaf
and a piece of cheese. She seemed
%%144%%
pleased; and went into the garden for
some roots and plants, which she placed
in water, and then upon the fire. She
afterwards continued her work, whilst
the young man went into the garden,
and appeared busily employed in digging
and pulling up roots. After he
had been employed thus about an hour,
the young woman joined him, and they
entered the cottage together.

``The old man had, in the mean time,
been pensive; but, on the appearance
of his companions, he assumed a more
cheerful air, and they sat down to eat.
The meal was quickly dispatched. The
young woman was again occupied
in arranging the cottage; the old man
walked before the cottage in the sun
for a few minutes, leaning on the arm
of the youth. Nothing could exceed in
beauty the contrast between these two
excellent creatures. One was old, with
silver hairs and a countenance beaming
with benevolence and love: the younger
was slight and graceful in his figure,
and his features were moulded with the
finest symmetry; yet his eyes and attitude
expressed the utmost sadness and
despondency. The old man returned
to the cottage; and the youth, with
tools different from those he had used
in the morning, directed his steps
across the fields.

``Night quickly shut in; but, to my
extreme wonder, I found that the
cottagers had a means of prolonging
light, by the use of tapers, and
was delighted to find, that the setting
of the sun did not put an end to the
pleasure I experienced in watching
my human neighbours. In the
%%145%%
evening, the young girl and her companion
were employed in various occupations
which I did not understand; and the
old man again took up the instrument,
which produced the divine sounds that
had enchanted me in the morning. So
soon as he had finished, the youth
began, not to play, but to utter sounds
that were monotonous, and neither resembling
the harmony of the old man's
instrument or the songs of the birds; I
since found that he read aloud, but at
that time I knew nothing of the science
of words or letters.

``The family, after having been thus
occupied for a short time, extinguished
their lights, and retired, as I conjectured,
to rest.''
%%146%%

\namedchapter{Chapter IV}

``\textsc{I lay} on my straw, but I could not
sleep. I thought of the occurrences of
the day. What chiefly struck me was
the gentle manners of these people; and
I longed to join them, but dared not.
I remembered too well the treatment I
had suffered the night before from the
barbarous villagers, and resolved, whatever
course of conduct I might hereafter
think it right to pursue, that for the
present I would remain quietly in my
hovel, watching, and endeavouring to
discover the motives which influenced
their actions.

``The cottagers arose the next morning
before the sun. The young woman
arranged the cottage, and prepared the
food; and the youth departed after the
first meal.

``This day was passed in the same
routine as that which preceded it. The
young man was constantly employed out
of doors, and the girl in various laborious
occupations within. The old man,
whom I soon perceived to be blind,
employed his leisure hours on his instrument,
or in contemplation. Nothing
could exceed the love and respect which
the younger cottagers exhibited towards
their venerable companion. They performed
towards him every little office
of affection and duty with gentleness;
and he rewarded them by his benevolent
smiles.
%%147%%

``They were not entirely happy. The
young man and his companion often
went apart, and appeared to weep. I
saw no cause for their unhappiness; but
I was deeply affected by it. If such
lovely creatures were miserable, it was
less strange that I, an imperfect and solitary
being, should be wretched. Yet
why were these gentle beings unhappy?
They possessed a delightful house (for
such it was in my eyes), and every
luxury; they had a fire to warm them
when chill, and delicious viands when
hungry; they were dressed in excellent
clothes; and, still more, they enjoyed
one another's company and speech,
interchanging each day looks of affection
and kindness. What did their
tears imply? Did they really express
pain? I was at first unable to solve
these questions; but perpetual attention,
and time, explained to me many
appearances which were at first enigmatic.

``A considerable period elapsed before
I discovered one of the causes of the uneasiness
of this amiable family; it was
poverty: and they suffered that evil in a
very distressing degree. Their nourishment
consisted entirely of the vegetables
of their garden, and the milk of one
cow, who gave very little during the
winter, when its masters could scarcely
procure food to support it. They
often, I believe, suffered the pangs of
hunger very poignantly, especially the
two younger cottagers; for several times
they placed food before the old man,
when they reserved none for themselves.

``This trait of kindness moved me
sensibly. I had been accustomed,
%%148%%
during the night, to steal a part of their
store for my own consumption; but
when I found that in doing this I inflicted
pain on the cottagers, I abstained,
and satisfied myself with berries, nuts,
and roots, which I gathered from a
neighbouring wood.

``I discovered also another means
through which I was enabled to assist
their labours. I found that the youth
spent a great part of each day in collecting
wood for the family fire; and, during
the night, I often took his tools, the use
of which I quickly discovered, and
brought home firing sufficient for the
consumption of several days.

``I remember, the first time that I
did this, the young woman, when she
opened the door in the morning, appeared
greatly astonished on seeing a
great pile of wood on the outside. She
uttered some words in a loud voice,
and the youth joined her, who also
expressed surprise. I observed, with
pleasure, that he did not go to the
forest that day, but spent it in repairing
the cottage, and cultivating the
garden.

``By degrees I made a discovery of
still greater moment. I found that
these people possessed a method of
communicating their experience and
feelings to one another by articulate
sounds. I perceived that the words
they spoke sometimes produced pleasure
or pain, smiles or sadness, in the
minds and countenances of the hearers.
This was indeed a godlike science, and
I ardently desired to become acquainted
with it. But I was baffled in every attempt
I made for this purpose. Their
%%149%%
pronunciation was quick; and the
words they uttered, not having any
apparent connexion with visible objects,
I was unable to discover any clue
by which I could unravel the mystery
of their reference. By great application,
however, and after having remained
during the space of several
revolutions of the moon in my hovel, I
discovered the names that were given
to some of the most familiar objects of
discourse: I learned and applied the
words \emph{fire}, \emph{milk}, \emph{bread}, and \emph{wood}. I
learned also the names of the cottagers
themselves. The youth and his companion
had each of them several names,
but the old man had only one, which
was \emph{father}. The girl was called \emph{sister},
or \emph{Agatha}; and the youth \emph{Felix}, \emph{brother},
or \emph{son}. I cannot describe the delight
I felt when I learned the ideas appropriated
to each of these sounds, and
was able to pronounce them. I distinguished
several other words, without
being able as yet to understand or apply
them; such as \emph{good}, \emph{dearest}, \emph{unhappy}.

``I spent the winter in this manner.
The gentle manners and beauty of the
cottagers greatly endeared them to me:
when they were unhappy, I felt depressed;
when they rejoiced, I sympathized
in their joys. I saw few human beings
beside them; and if any other happened
to enter the cottage, their harsh manners
and rude gait only enhanced to me the
superior accomplishments of my friends.
The old man, I could perceive, often
endeavoured to encourage his children,
as sometimes I found that he called
them, to cast off their melancholy. He
%%150%%
would talk in a cheerful accent, with
an expression of goodness that bestowed
pleasure even upon me. Agatha listened
with respect, her eyes sometimes
filled with tears, which she endeavoured
to wipe away unperceived; but I generally
found that her countenance and
tone were more cheerful after having
listened to the exhortations of her father.
It was not thus with Felix. He
was always the saddest of the groupe;
and, even to my unpractised senses, he
appeared to have suffered more deeply
than his friends. But if his countenance
was more sorrowful, his voice
was more cheerful than that of his sister,
especially when he addressed the
old man.

``I could mention innumerable instances,
which, although slight, marked
the dispositions of these amiable cottagers.
In the midst of poverty and
want, Felix carried with pleasure to his
sister the first little white flower that
peeped out from beneath the snowy
ground. Early in the morning before
she had risen, he cleared away the
snow that obstructed her path to the
milk-house, drew water from the well,
and brought the wood from the out-house,
where, to his perpetual astonishment,
he found his store always replenished
by an invisible hand. In the
day, I believe, he worked sometimes
for a neighbouring farmer, because he
often went forth, and did not return
until dinner, yet brought no wood with
him. At other times he worked in the
garden; but, as there was little to do
in the frosty season, he read to the old
man and Agatha.
%%151%%

``This reading had puzzled me extremely
at first; but, by degrees, I discovered
that he uttered many of the
same sounds when he read as when he
talked. I conjectured, therefore, that
he found on the paper signs for speech
which he understood, and I ardently
longed to comprehend these also; but
how was that possible, when I did not
even understand the sounds for which
they stood as signs? I improved, however,
sensibly in this science, but not
sufficiently to follow up any kind of
conversation, although I applied my
whole mind to the endeavour: for I
easily perceived that, although I eagerly
longed to discover myself to the
cottagers, I ought not to make the attempt
until I had first become master
of their language; which knowledge
might enable me to make them overlook
the deformity of my figure; for
with this also the contrast perpetually
presented to my eyes had made me
acquainted.

``I had admired the perfect forms of
my cottagers --- their grace, beauty, and
delicate complexions: but how was I
terrified, when I viewed myself in a
transparent pool! At first I started
back, unable to believe that it was indeed
I who was reflected in the mirror;
and when I became fully convinced
that I was in reality the monster that I
am, I was filled with the bitterest sensations
of despondence and mortification.
Alas! I did not yet entirely
know the fatal effects of this miserable
deformity.

``As the sun became warmer, and
the light of day longer, the snow
%%152%%
vanished, and I beheld the bare trees and
the black earth. From this time Felix
was more employed; and the heart-moving
indications of impending famine
disappeared. Their food, as I
afterwards found, was coarse, but it
was wholesome; and they procured a
sufficiency of it. Several new kinds
of plants sprung up in the garden,
which they dressed; and these signs of
comfort increased daily as the season
advanced.

``The old man, leaning on his son,
walked each day at noon, when it did
not rain, as I found it was called when
the heavens poured forth its waters.
This frequently took place; but a high
wind quickly dried the earth, and the
season became far more pleasant than
it had been.

``My mode of life in my hovel was
uniform. During the morning I attended
the motions of the cottagers;
and when they were dispersed in various
occupations, I slept: the remainder
of the day was spent in observing
my friends. When they had retired to
rest, if there was any moon, or the night
was star-light, I went into the woods,
and collected my own food and fuel
for the cottage. When I returned, as
often as it was necessary, I cleared their
path from the snow, and performed
those offices that I had seen done by
Felix. I afterwards found that these
labours, performed by an invisible
hand, greatly astonished them; and
once or twice I heard them, on these
occasions, utter the words \emph{good spirit},
\emph{wonderful}; but I did not then understand
the signification of these terms.
%%153%%

``My thoughts now became more
active, and I longed to discover the
motives and feelings of these lovely
creatures; I was inquisitive to know
why Felix appeared so miserable, and
Agatha so sad. I thought (foolish
wretch!) that it might be in my power
to restore happiness to these deserving
people. When I slept, or was absent,
the forms of the venerable blind father,
the gentle Agatha, and the excellent
Felix, flitted before me. I looked upon
them as superior beings, who would be
the arbiters of my future destiny. I
formed in my imagination a thousand
pictures of presenting myself to them,
and their reception of me. I imagined
that they would be disgusted, until, by
my gentle demeanour and conciliating
words, I should first win their favour,
and afterwards their love.

``These thoughts exhilarated me,
and led me to apply with fresh ardour
to the acquiring the art of language.
My organs were indeed harsh, but supple;
and although my voice was very
unlike the soft music of their tones,
yet I pronounced such words as I understood
with tolerable ease. It was
as the ass and the lap-dog; yet surely
the gentle ass, whose intentions were
affectionate, although his manners were
rude, deserved better treatment than
blows and execration.

``The pleasant showers and genial
warmth of spring gr\-eatly altered the
aspect of the earth. Men, who before
this change seemed to have been hid
in caves, dispersed themselves, and
were employed in various arts of cultivation.
The birds sang in more
%%154%%
cheerful notes, and the leaves began to bud
forth on the trees. Happy, happy
earth! fit habitation for gods, which,
so short a time before, was bleak, damp,
and unwholesome. My spirits were
elevated by the enchanting appearance
of nature; the past was blotted from
my memory, the present was tranquil,
and the future gilded by bright rays
of hope, and anticipations of joy.''
%%155%%

\namedchapter{Chapter V}

``\textsc{I now} hasten to the more moving
part of my story. I shall relate events
that impressed me with feelings which,
from what I was, have made me what
I am.

``Spring advanced rapidly; the weather
became fine, and the skies cloudless.
It surprised me, that what before
was desert and gloomy should now
bloom with the most beautiful flowers
and verdure. My senses were gratified
and refreshed by a thousand scents of
delight, and a thousand sights of
beauty.

``It was on one of these days, when
my cottagers periodically rested from
labour --- the old man played on his
guitar, and the children listened to
him --- I observed that the countenance
of Felix was melancholy beyond expression:
he sighed frequently; and once
his father paused in his music, and I
conjectured by his manner that he inquired
the cause of his son's sorrow.
Felix replied in a cheerful accent, and
the old man was recommencing his
music, when some one tapped at the
door.

``It was a lady on horseback, accompanied
by a countryman as a guide.
The lady was dressed in a dark suit, and
covered with a thick black veil. Agatha
asked a question; to which the
%%156%%
stranger only replied by pronouncing, in a
sweet accent, the name of Felix. Her
voice was musical, but unlike that of
either of my friends. On hearing this
word, Felix came up hastily to the lady;
who, when she saw him, threw up her
veil, and I beheld a countenance of angelic
beauty and expression. Her hair
of a shining raven black, and curiously
braided; her eyes were dark, but gentle,
although animated; her features of a
regular proportion, and her complexion
wondrously fair, each cheek tinged
with a lovely pink.

``Felix seemed ravished with delight
when he saw her, every trait
of sorrow vanished from his face,
and it instantly expressed a degree
of ecstatic joy, of which I could hardly
have believed it capable; his eyes
sparkled, as his cheek flushed with
pleasure; and at that moment I thought
him as beautiful as the stranger. She
appeared affected by different feelings;
wiping a few tears from her lovely eyes,
she held out her hand to Felix, who
kissed it rapturously, and called her, as
well as I could distinguish, his sweet
Arabian. She did not appear to understand
him, but smiled. He assisted her
to dismount, and, dismissing her guide,
conducted her into the cottage. Some
conversation took place between him
and his father; and the young stranger
knelt at the old man's feet, and would
have kissed his hand, but he raised her,
and embraced her affectionately.

``I soon perceived, that although the
stranger uttered articulate sounds, and
appeared to have a language of her
own, she was neither understood by, or
%%157%%
herself understood, the cottagers. They
made many signs which I did not comprehend;
but I saw that her presence
diffused gladness through the cottage,
dispelling their sorrow as the sun dissipates
the morning mists. Felix seemed
peculiarly happy, and with smiles of
delight welcomed his Arabian. Agatha,
the ever-gentle Agatha, kissed the hands
of the lovely stranger; and, pointing to
her brother, made signs which appeared
to me to mean that he had been sorrowful
until she came. Some hours passed
thus, while they, by their countenances,
expressed joy, the cause of which I did
not comprehend. Presently I found, by
the frequent recurrence of one sound
which the stranger repeated after them,
that she was endeavouring to learn
their language; and the idea instantly
occurred to me, that I should make use
of the same instructions to the same
end. The stranger learned about
twenty words at the first lesson, most
of them indeed were those which I had
before understood, but I profited by the
others.

``As night came on, Agatha and the
Arabian retired early. When they separated,
Felix kissed the hand of the
stranger, and said, `Good night, sweet
Safie.' He sat up much longer, conversing
with his father; and, by the frequent
repetition of her name, I conjectured
that their lovely guest was the
subject of their conversation. I ardently
desired to understand them, and
bent every faculty towards that purpose,
but found it utterly impossible.

``The next morning Felix went out
to his work; and, after the usual
%%158%%
occupations of Agatha were finished, the
Arabian sat at the feet of the old man,
and, taking his guitar, played some airs
so entrancingly beautiful, that they at
once drew tears of sorrow and delight
from my eyes. She sang, and her voice
flowed in a rich cadence, swelling or
dying away, like a nightingale of the
woods.

``When she had finished, she gave the
guitar to Agatha, who at first declined
it. She played a simple air, and her
voice accompanied it in sweet accents,
but unlike the wondrous strain of the
stranger. The old man appeared enraptured,
and said some words, which
Agatha endeavoured to explain to Safie,
and by which he appeared to wish to
express that she bestowed on him the
greatest delight by her music.

``The days now passed as peaceably as
before, with the sole alteration, that joy
had taken place of sadness in the countenances
of my friends. Safie was always
gay and happy; she and I improved
rapidly in the knowledge of
language, so that in two months I began
to comprehend most of the words
uttered by my protectors.

``In the meanwhile also the black
ground was covered with herbage, and
the green banks interspersed with innumerable
flowers, sweet to the scent
and the eyes, stars of pale radiance
among the moonlight woods; the sun
became warmer, the nights clear and
balmy; and my nocturnal rambles were
an extreme pleasure to me, although
they were considerably shortened by the
late setting and early rising of the sun;
for I never ventured abroad during daylight,
fearful of meeting with the same
%%159%%
treatment as I had formerly endured in
the first village which I entered.

``My days were spent in close attention,
that I might more speedily master
the language; and I may boast that I
improved more rapidly than the Arabian,
who understood very little, and
conversed in broken accents, whilst I
comprehended and could imitate almost
every word that was spoken.

``While I improved in speech, I also
learned the science of letters, as it was
taught to the stranger; and this opened
before me a wide field for wonder and
delight.

``The book from which Felix instructed
Safie was Volney's \emph{Ruins of Empires}.
I should not have understood the purport
of this book, had not Felix, in reading
it, given very minute explanations.
He had chosen this work, he said, because
the declamatory style was framed
in imitation of the eastern authors.
Through this work I obtained a cursory
knowledge of history, and a view of
the several empires at present existing
in the world; it gave me an insight into
the manners, governments, and religions
of the different nations of the
earth. I heard of the slothful Asiatics;
of the stupendous genius and mental activity
of the Grecians; of the wars and
wonderful virtue of the early Romans --- of
their subsequent degeneration --- of the
decline of that mighty empire; of chivalry,
Christianity, and kings. I heard
of the discovery of the American hemisphere,
and wept with Safie over the
hapless fate of its original inhabitants.

``These wonderful narrations inspired
me with strange feelings. Was man, indeed,
at once so powerful, so virtuous,
%%160%%
and magnificent, yet so vicious and
base? He appeared at one time a mere
scion of the evil principle, and at another
as all that can be conceived of noble
and godlike. To be a great and virtuous
man appeared the highest honour
that can befall a sensitive being; to be
base and vicious, as many on record
have been, appeared the lowest degradation,
a condition more abject than
that of the blind mole or harmless
worm. For a long time I could
not conceive how one man could go
forth to murder his fellow, or even
why there were laws and governments;
but when I heard details of vice and
bloodshed, my wonder ceased, and I
turned away with disgust and loathing.

``Every conversation of the cottagers
now op\-ened new wonders to me. While
I listened to the instructions which Felix
bestowed upon the Arabian, the
strange system of human society was
explained to me. I heard of the division
of property, of immense wealth
and squalid poverty; of rank, descent,
and noble blood.

``The words induced me to turn towards
myself. I learned that the possessions
most esteemed by your fellow-creatures
were, high and unsullied descent
united with riches. A man might
be respected with only one of these acquisitions;
but without either he was considered,
except in very rare instances,
as a vagabond and a slave, doomed to
waste his powers for the profit of the
chosen few. And what was I? Of my
creation and creator I was absolutely
ignorant; but I knew that I possessed
no money, no friends, no kind of
%%161%%
property. I was, besides, endowed with a
figure hideously deformed and loathsome;
I was not even of the same nature
as man. I was more agile than
they, and could subsist upon coarser
diet; I bore the extremes of heat and cold
with less injury to my frame; my stature
far exceeded their's. When I
looked around, I saw and heard of none
like me. Was I then a monster, a blot
upon the earth, from which all men fled,
and whom all men disowned?

``I cannot describe to you the agony
that these reflections inflicted upon me;
I tried to dispel them, but sorrow only
increased with knowledge. Oh, that
I had for ever remained in my native
wood, nor known or felt beyond
the sensations of hunger, thirst, and
heat!

``Of what a strange nature is knowledge!
It clings to the mind, when it
has once seized on it, like a lichen on
the rock. I wished sometimes to shake
off all thought and feeling; but I learned
that there was but one means to overcome
the sensation of pain, and that
was death --- a state which I feared yet
did not understand. I admired virtue
and good feelings, and loved the gentle
manners and amiable qualities of my
cottagers; but I was shut out from intercourse
with them, except through
means which I obtained by stealth, when
I was unseen and unknown, and which
rather increased than satisfied the desire
I had of becoming one among my fellows.
The gentle words of Agatha,
and the animated smiles of the charming
Arabian, were not for me. The
mild exhortations of the old man, and
%%162%%
the lively conversation of the loved
Felix, were not for me. Miserable, unhappy
wretch!

``Other lessons were impressed upon
me even more deeply. I heard of the
difference of sexes; of the birth and
growth of children; how the father
doated on the smiles of the infant, and
the lively sallies of the older child;
how all the life and cares of the mother
were wrapt up in the precious charge;
how the mind of youth expanded and
gained knowledge; of brother, sister,
and all the various relationships which
bind one human being to another in
mutual bonds.

``But where were my friends and relations?
No father had watched my
infant days, no mother had blessed me
with smiles and caresses; or if they had,
all my past life was now a blot, a blind
vacancy in which I distinguished nothing.
From my earliest remembrance
I had been as I then was in height and
proportion. I had never yet seen a
being resembling me, or who claimed
any intercourse with me. What was I?
The question again recurred, to be answered
only with groans.

``I will soon explain to what these
feelings tended; but allow me now to
return to the cottagers, whose story excited
in me such various feelings of indignation,
delight, and wonder, but
which all terminated in additional love
and reverence for my protectors (for so
I loved, in an innocent, half painful
self-deceit, to call them).''
%%163%%

\namedchapter{Chapter VI}

``\textsc{Some} time elapsed before I learned
the history of my friends. It was one
which could not fail to impress itself
deeply on my mind, unfolding as it did
a number of circumstances each interesting
and wonderful to one so utterly
inexperienced as I was.

``The name of the old man was De
Lacey. He was descended from a good
family in France, where he had lived
for many years in affluence, respected
by his superiors, and beloved by his
equals. His son was bred in the service
of his country; and Agatha had
ranked with ladies of the highest distinction.
A few months before my
arrival, they had lived in a large and
luxurious city, called Paris, surrounded
by friends, and possessed of every enjoyment
which virtue, refinement of
intellect, or taste, accompanied by a
moderate fortune, could afford.

``The father of Safie had been the
cause of their ruin. He was a Turkish
merchant, and had inhabited Paris for
many years, when, for some reason
which I could not learn, he became
obnoxious to the government. He was
seized and cast into prison the very day
that Safie arrived from Constantinople
to join him. He was tried, and condemned
to death. The injustice of his
sentence was very flagrant; all Paris
%%164%%
was indignant; and it was judged that
his religion and wealth, rather than the
crime alleged against him, had been
the cause of his condemnation.

``Felix had been present at the trial;
his horror and indignation were uncontrollable,
when he heard the decision of
the court. He made, at that moment,
a solemn vow to deliver him, and then
looked around for the means. After
many fruitless attempts to gain admittance
to the prison, he found a strongly
grated window in an unguarded part
of the building, which lighted the dungeon
of the unfortunate Mahometan;
who, loaded with chains, waited in despair
the execution of the barbarous
sentence. Felix visited the grate at
night, and made known to the prisoner
his intentions in his favour. The Turk,
amazed and delighted, endeavoured to
kindle the zeal of his deliverer by promises
of reward and wealth. Felix
rejected his offers with contempt; yet
when he saw the lovely Safie, who was
allowed to visit her father, and who, by
her gestures, expressed her lively gratitude,
the youth could not help owning
to his own mind, that the captive
possessed a treasure which would fully
reward his toil and hazard.

``The Turk quickly perceived the
impression that his daughter had made
on the heart of Felix, and endeavoured
to secure him more entirely in his interests
by the promise of her hand in
marriage, so soon as he should be conveyed
to a place of safety. Felix was
too delicate to accept this offer; yet he
looked forward to the probability of
%%165%%
that event as to the consummation of
his happiness.

``During the ensuing days, while
the preparations were going forward
for the escape of the merchant, the zeal
of Felix was warmed by several letters
that he received from this lovely girl,
who found means to express her
thoughts in the language of her lover
by the aid of an old man, a servant of
her father's, who understood French.
She thanked him in the most ardent
terms for his intended services towards
her father; and at the same time she
gently deplored her own fate.

``I have copies of these letters; for I
found means, during my residence in
the hovel, to procure the implements
of writing; and the letters were often
in the hands of Felix or Agatha. Before
I depart, I will give them to you,
they will prove the truth of my tale;
%%166%%
but at present, as the sun is already far
declined, I shall only have time to repeat
the substance of them to you.

``Safie related, that her mother was
a Christian Arab, seized and made a
slave by the Turks; recommended by
her beauty, she had won the heart of
the father of Safie, who married her.
The young girl spoke in high and
enthusiastic terms of her mother, who,
born in freedom spurned the bondage
to which she was now reduced. She
instructed her daughter in the tenets
of her religion, and taught her to aspire
to higher powers of intellect, and an
independence of spirit, forbidden to the
female followers of Mahomet. This
lady died; but her lessons were indelibly
impressed on the mind of Safie,
who sickened at the prospect of again
returning to Asia, and the being immured
within the walls of a haram,
allowed only to occupy herself with
puerile amusements, ill suited to the
temper of her soul, now accustomed to
grand ideas and a noble emulation for
virtue. The prospect of marrying a
Christian, and remaining in a country
%%167%%
where women were allowed to take a
rank in society, was enchanting to her.

``The day for the execution of the
Turk was fixed; but, on the night
previous to it, he had quitted prison,
and before morning was distant many
leagues from Paris. Felix had procured
passports in the name of his father,
sister, and himself. He had previously
communicated his plan to the
former, who aided the deceit by quitting
his house, under the pretence of a
journey, and concealed himself, with his
daughter, in an obscure part of Paris.

``Felix conducted the fugitives
through France to Ly\-ons, and across
Mont Cenis to Leghorn, where the
merchant had decided to wait a favourable
opportunity of passing into some
part of the Turkish dominions.

``Safie resolved to remain with her
father until the moment of his departure,
before which time the Turk renewed
his promise that she should be
united to his deliverer; and Felix remained
with them in expectation of
that event; and in the mean time he
enjoyed the society of the Arabian, who
exhibited towards him the simplest and
tenderest affection. They conversed
with one another through the means of
an interpreter, and sometimes with the
interpretation of looks; and Safie sang to
him the divine airs of her native country.

``The Turk allowed this intimacy
to take place, and encouraged the hopes
of the youthful lovers, while in his
heart he had formed far other plans.
He loathed the idea that his daughter
should be united to a Christian; but he
feared the resentment of Felix if he
%%168%%
should appear lukewarm; for he knew
that he was still in the power of his
deliverer, if he should choose to betray
him to the Italian state which they inhabited.
He revolved a thousand plans
by which he should be enabled to prolong
the deceit until it might be no longer
necessary, and secretly to take his
daughter with him when he departed.
His plans were greatly facilitated by
the news which arrived from Paris.

``The government of France were
greatly enraged at the escape of their
victim, and spared no pains to detect
and punish his deliverer. The plot of
Felix was quickly discovered, and De
Lacey and Agatha were thrown into
prison. The news reached Felix, and
roused him from his dream of pleasure.
His blind and aged father, and his gentle
sister, lay in a noisome dungeon,
while he enjoyed the free air, and the
society of her whom he loved. This
idea was torture to him. He quickly
arranged with the Turk, that if the latter
should find a favourable opportunity
for escape before Felix could return
to Italy, Safie should remain as a
boarder at a convent at Leghorn; and
then, quitting the lovely Arabian, he
hastened to Paris, and delivered himself
up to the vengeance of the law,
hoping to free De Lacey and Agatha
by this proceeding.

``He did not succeed. They remained
confined for five months before
the trial took place; the result of which
deprived them of their fortune, and
condemned them to a perpetual exile
from their native country.

``They found a miserable asylum
%%169%%
in the cottage in Germany, where I
discovered them. Felix soon learned
that the treacherous Turk, for whom he
and his family endured such unheard-of
oppression, on discovering that his deliverer
was thus reduced to poverty and
impotence, became a traitor to good
feeling and honour, and had quitted
Italy with his daughter, insultingly
sending Felix a pittance of money to
aid him, as he said, in some plan of
future maintenance.

``Such were the events that preyed
on the heart of Felix, and rendered
him, when I first saw him, the most
miserable of his family. He could
have endured poverty, and when this
distress had been the meed of his virtue,
he would have gloried in it: but
the ingratitude of the Turk, and the
loss of his beloved Safie, were misfortunes
more bitter and irreparable. The
arrival of the Arabian now infused new
life into his soul.

``When the news reached Leghorn,
that Felix was deprived of his wealth
and rank, the merchant commanded
his daughter to think no more of her
lover, but to prepare to return with
him to her native country. The generous
nature of Safie was outraged by
this command; she attempted to expostulate
with her father, but he left
her angrily, reiterating his tyrannical
mandate.

``A few days after, the Turk entered
his daughter's apartment, and told her
hastily, that he had reason to believe
that his residence at Leghorn had been
divulged, and that he should speedily
be delivered up to the French
%%170%%
government; he had, consequently, hired a
vessel to convey him to Constantinople,
for which city he should sail in a few
hours. He intended to leave his daughter
under the care of a confidential
servant, to follow at her leisure with
the greater part of his property, which
had not yet arrived at Leghorn.

``When alone, Safie resolved in her
own mind the plan of conduct that it
would become her to pursue in this
emergency. A residence in Turkey
was abhorrent to her; her religion and
feelings were alike adverse to it. By
some papers of her father's, which fell
into her hands, she heard of the exile
of her lover, and learnt the name of
the spot where he then resided. She
hesitated some time, but at length she
formed her determination. Taking
with her some jewels that belonged to
her, and a small sum of money, she
quitted Italy, with an attendant, a native
of Leghorn, but who understood
the common language of Turkey, and
departed for Germany.

``She arrived in safety at a town
about twenty leagues from the cottage
of De Lacey, when her attendant fell
dangerously ill. Safie nursed her with
the most devoted affection; but the
poor girl died, and the Arabian was
left alone, unacquainted with the language
of the country, and utterly ignorant
of the customs of the world. She
fell, however, into good hands. The
Italian had mentioned the name of the
spot for which they were bound; and,
after her death, the woman of the house
in which they had lived took care that
Safie should arrive in safety at the cottage
of her lover.''
%%171%%

\namedchapter{Chapter VII}

``\textsc{Such} was the history of my beloved
cottagers. It impressed me deeply. I
learned, from the views of social life
which it developed, to admire their virtues,
and to deprecate the vices of mankind.

``As yet I looked upon crime as a distant
evil; benevolence and generosity
were ever present before me, inciting
within me a desire to become an actor
in the busy scene where so many admirable
qualities were called forth and
displayed. But, in giving an account
of the progress of my intellect, I must
not omit a circumstance which occurred
in the beginning of the month of
August of the same year.

``One night, during my accustomed
visit to the neighbouring wood, where
I collected my own food, and brought
home firing for my protectors, I found
on the ground a leathern portmanteau,
containing several articles of dress and
some books. I eagerly seized the
prize, and returned with it to my hovel.
Fortunately the books were written in
the language the elements of which I
had acquired at the cottage; they consisted
of \emph{Paradise Lost}, a volume of
\emph{Plutarch's Lives}, and the \emph{Sorrows of
Werter}. The possession of these treasures
gave me extreme delight; I now
continually studied and exercised my
mind upon these histories, whilst my
%%172%%
friends were employed in their ordinary
occupations.

``I can hardly describe to you the
effect of these books. They produced
in me an infinity of new images and
feelings, that sometimes raised me to
ecstacy, but more frequently sunk me
into the lowest dejection. In the \emph{Sorrows
of Werter}, besides the interest of its
simple and affecting story, so many
opinions are canvassed, and so many
lights thrown upon what had hitherto
been to me obscure subjects, that I
found in it a never-ending source of
speculation and astonishment. The
gentle and domestic manners it described,
combined with lofty sentiments
and feelings, which had for their
object something out of self, accorded
well with my experience among my
protectors, and with the wants which
were for ever alive in my own bosom.
But I thought Werter himself a more
divine being than I had ever beheld or
imagined; his character contained no
pretension, but it sunk deep. The
disquisitions upon death and suicide
were calculated to fill me with wonder.
I did not pretend to enter into the merits
of the case, yet I inclined towards
the opinions of the hero, whose extinction
I wept, without precisely understanding
it.

``As I read, however, I applied
much personally to my own feelings
and condition. I found myself similar,
yet at the same time strangely unlike
the beings concerning whom I read,
and to whose conversation I was a
listener. I sympathized with, and
partly understood them, but I was
%%173%%
unformed in mind; I was dependent
on none, and related to none. `The
path of my departure was free;' and
there was none to lament my annihilation.
My person was hideous, and
my stature gigantic: what did this
mean? Who was I? What was I?
Whence did I come? What was my
destination? These questions continually
recurred, but I was unable to
solve them.

``The volume of \emph{Plutarch's Lives}
which I possessed, contained the histories
of the first founders of the ancient
republics. This book had a far
different effect upon me from the \emph{Sorrows
of Werter}. I learned from Werter's
imaginations despondency and
gloom: but Plutarch taught me high
thoughts; he elevated me above the
wretched sphere of my own reflections,
to admire and love the heroes of past
ages. Many things I read surpassed
my understanding and experience. I
had a very confused knowledge of kingdoms,
wide extents of country, mighty
rivers, and boundless seas. But I was
perfectly unacquainted with towns, and
large assemblages of men. The cottage
of my protectors had been the only
school in which I had studied human
nature; but this book developed new
and mightier scenes of action. I read
of men concerned in public affairs governing
or massacring their species. I
felt the greatest ardour for virtue rise
within me, and abhorrence for vice, as
far as I understood the signification of
those terms, relative as they were, as I
applied them, to pleasure and pain alone.
Induced by these feelings, I was of course
%%174%%
led to admire peaceable law-givers,
Numa, Solon, and Lycurgus, in preference
to Romulus and Theseus. The
patriarchal lives of my protectors caused
these impressions to take a firm hold on
my mind; perhaps, if my first introduction
to humanity had been made by a
young soldier, burning for glory and
slaughter, I should have been imbued
with different sensations.

``But \emph{Paradise Lost} excited different
and far deeper emotions. I read it, as I
had read the other volumes which had
fallen into my hands, as a true history.
It moved every feeling of wonder and
awe, that the picture of an omnipotent
God warring with his creatures was
capable of exciting. I often referred
the several situations, as their similarity
struck me, to my own. Like Adam, I
was created apparently united by no
link to any other being in existence;
but his state was far different from mine
in every other respect. He had come
forth from the hands of God a perfect
creature, happy and prosperous, guarded
by the especial care of his Creator;
he was allowed to converse with, and
acquire knowledge from beings of a
superior nature: but I was wretched,
helpless, and alone. Many times I
considered Satan as the fitter emblem of
my condition; for often, like him, when
I viewed the bliss of my protectors, the
bitter gall of envy rose within me.

``Another circumstance strengthened
and confirmed these feelings. Soon
after my arrival in the hovel, I discovered
some papers in the pocket of
the dress which I had taken from your
laboratory. At first I had neglected
%%175%%
them; but now that I was able to decypher
the characters in which they
were written, I began to study them
with diligence. It was your journal of
the four months that preceded my creation.
You minutely described in these
papers every step you took in the progress
of your work; this history was
mingled with accounts of domestic occurrences.
You, doubtless, recollect
these papers. Here they are. Every
thing is related in them which bears
reference to my accursed origin; the
whole detail of that series of disgusting
circumstances which produced it is set
in view; the minutest description of
my odious and loathsome person is
given, in language which painted your
own horrors, and rendered mine ineffaceable.
I sickened as I read. `Hateful
day when I received life!' I exclaimed
in agony. `Cursed creator!
Why did you form a monster so hideous
that even you turned from me in disgust?
God in pity made man beautiful
and alluring, after his own image;
but my form is a filthy type of your's,
more horrid from its very resemblance.
Satan had his companions, fellow-devils,
to admire and encourage him;
but I am solitary and detested.'

``These were the reflections of my
hours of despondency and solitude;
but when I contemplated the virtues of
the cottagers, their amiable and benevolent
dispositions, I persuaded myself
that when they should become acquainted
with my admiration of their virtues,
they would compassionate me, and
overlook my personal deformity. Could
they turn from their door one, however
%%176%%
monstrous, who solicited their compassion
and friendship? I resolved, at
least, not to despair, but in every way
to fit myself for an interview with them
which would decide my fate. I postponed
this attempt for some months
longer; for the importance attached to
its success inspired me with a dread
lest I should fail. Besides, I found
that my understanding improved so
much with every day's experience, that
I was unwilling to commence this undertaking
until a few more months
should have added to my wisdom.

``Several changes, in the mean time,
took place in the cottage. The presence
of Safie diffused happiness among
its inhabitants; and I also found that a
greater degree of plenty reigned there.
Felix and Agatha spent more time in
amusement and conversation, and were
assisted in their labours by servants.
They did not appear rich, but they
were contented and happy; their feelings
were serene and peaceful, while
mine became every day more tumultuous.
Increase of knowledge only
discovered to me more clearly what a
wretched outcast I was. I cherished
hope, it is true; but it vanished, when I
beheld my person reflected in water, or
my shadow in the moon-shine, even
as that frail image and that inconstant
shade.

``I endeavoured to crush these fears,
and to fortify myself for the trial which
in a few months I resolved to undergo;
and sometimes I allowed my thoughts,
unchecked by reason, to ramble in the
fields of Paradise, and dared to fancy
amiable and lovely creatures
%%177%%
sympathizing with my feelings and cheering
my gloom; their angelic countenances
breathed smiles of consolation. But it
was all a dream: no Eve soothed my
sorrows, or shared my thoughts; I was
alone. I remembered Adam's supplication
to his Creator; but where was mine?
he had abandoned me, and, in the bitterness
of my heart, I cursed him.

``Autumn passed thus. I saw, with
surprise and grief, the leaves decay and
fall, and nature again assume the barren
and bleak appearance it had worn
when I first beheld the woods and the
lovely moon. Yet I did not heed the
bleakness of the weather; I was better
fitted by my conformation for the endurance
of cold than heat. But my chief
delights were the sight of the flowers,
the birds, and all the gay apparel of
summer; when those deserted me, I
turned with more attention towards the
cottagers. Their happiness was not
decreased by the absence of summer.
They loved, and sympathized with one
another; and their joys, depending on
each other, were not interrupted by
the casualties that took place around
them. The more I saw of them, the
greater became my desire to claim their
protection and kindness; my heart
yearned to be known and loved by
these amiable creatures: to see their
sweet looks turned towards me with
affection, was the utmost limit of my
ambition. I dared not think that they
would turn them from me with disdain
and horror. The poor that stopped at
their door were never driven away. I
asked, it is true, for greater treasures
than a little food or rest; I required
%%178%%
kindness and sympathy; but I did not
believe myself utterly unworthy of it.

``The winter advanced, and an entire
revolution of the seasons had taken
place since I awoke into life. My attention,
at this time, was solely directed
towards my plan of introducing myself
into the cottage of my protectors. I revolved
many projects; but that on
which I finally fixed was, to enter the
dwelling when the blind old man should
be alone. I had sagacity enough to
discover, that the unnatural hideousness
of my person was the chief object
of horror with those who had formerly
beheld me. My voice, although harsh,
had nothing terrible in it; I thought,
therefore, that if, in the absence of his
children, I could gain the good-will and
mediation of the old De Lacy, I might,
by his means, be tolerated by my
younger protectors.

``One day, when the sun shone on
the red leaves that strewed the ground,
and diffused cheerfulness, although it
denied warmth, Safie, Agatha, and Felix,
departed on a long country walk,
and the old man, at his own desire, was
left alone in the cottage. When his
children had departed, he took up his
guitar, and played several mournful,
but sweet airs, more sweet and mournful
than I had ever heard him play before.
At first his countenance was illuminated
with pleasure, but, as he
continued, thoughtfulness and sadness
succeeded; at length, laying aside the
instrument, he sat absorbed in reflection.

``My heart beat quick; this was the
hour and moment of trial, which would
%%179%%
decide my hopes, or realize my fears.
The servants were gone to a neighbouring
fair. All was silent in and around
the cottage: it was an excellent opportunity;
yet, when I proceeded to execute
my plan, my limbs failed me, and
I sunk to the ground. Again I rose;
and, exerting all the firmness of which
I was master, removed the planks which
I had placed before my hovel to conceal
my retreat. The fresh air revived me,
and, with renewed determination, I approached
the door of their cottage.

``I knocked. `Who is there?' said
the old man --- `Come in.'

``I entered; `Pardon this intrusion,'
said I, `I am a traveller in want of a
little rest; you would greatly oblige me,
if you would allow me to remain a few
minutes before the fire.'

``\,`Enter,' said De Lacy; `and I will
try in what manner I can relieve your
wants; but, unfortunately, my children
are from home, and, as I am blind, I
am afraid I shall find it difficult to procure
food for you.'

``\,`Do not trouble yourself, my kind
host, I have food; it is warmth and
rest only that I need.'

``I sat down, and a silence ensued.
I knew that every minute was precious
to me, yet I remained irresolute in what
manner to commence the interview;
when the old man addressed me~---

``\,`By your language, stranger, I
suppose you are my countryman; --- are
you French?'

``\,`No; but I was educated by a
French family, and understand that
language only. I am now going to
%%180%%
claim the protection of some friends,
whom I sincerely love, and of whose
favour I have some hopes.'

``\,`Are these Germans?'

``\,`No, they are French. But let us
change the subject. I am an unfortunate
and deserted creature; I look
around, and I have no relation or friend
upon earth. These amiable people to
whom I go have never seen me, and
know little of me. I am full of fears;
for if I fail there, I am an outcast in
the world for ever.'

``\,`Do not despair. To be friendless
is indeed to be unfortunate; but the
hearts of men, when unprejudiced by
any obvious self-interest, are full of brotherly
love and charity. Rely, therefore,
on your hopes; and if these friends
are good and amiable, do not despair.'

``\,`They are kind --- they are the
most excellent creatures in the world;
but, unfortunately, they are prejudiced
against me. I have good dispositions;
my life has been hitherto harmless, and,
in some degree, beneficial; but a fatal
prejudice clouds their eyes, and where
they ought to see a feeling and kind
friend, they behold only a detestable
monster.'

``\,`That is indeed unfortunate; but
if you are really blameless, cannot you
undeceive them?'

``\,`I am about to undertake that
task; and it is on that account that I
feel so many overwhelming terrors. I
tenderly love these friends; I have, unknown
to them, been for many months
in the habits of daily kindness towards
them; but they believe that I wish to
%%181%%
injure them, and it is that prejudice
which I wish to overcome.'

``\,`Where do these friends reside?'

``\,`Near this spot.'

``The old man paused, and then
continued, `If you will unreservedly
confide to me the particulars of your
tale, I perhaps may be of use in undeceiving
them. I am blind, and cannot
judge of your countenance, but
there is something in your words which
persuades me that you are sincere. I
am poor, and an exile; but it will afford
me true pleasure to be in any way serviceable
to a human creature.'

``\,`Excellent man! I thank you, and
accept your generous offer. You raise
me from the dust by this kindness; and
I trust that, by your aid, I shall not be
driven from the society and sympathy
of your fellow-creatures.'

``\,`Heaven forbid! even if you were
really criminal; for that can only drive
you to desperation, and not instigate
you to virtue. I also am unfortunate;
I and my family have been condemned,
although innocent: judge, therefore, if
I do not feel for your misfortunes.'

``\,`How can I thank you, my best
and only benefactor? from your lips
first have I heard the voice of kindness
directed towards me; I shall
be for ever grateful; and your present
humanity assures me of success
with those friends whom I am on the
point of meeting.'

``\,`May I know the names and residence
of those friends?'

``I paused. This, I thought, was
the moment of decision, which was to
rob me of, or bestow happiness on me
%%182%%
for ever. I struggled vainly for firmness
sufficient to answer him, but
the effort destroyed all my remaining
strength; I sank on the chair, and
sobbed aloud. At that moment I heard
the steps of my younger protectors. I
had not a moment to lose; but, seizing
the hand of the old man, I cried, `Now
is the time! --- save and protect me!
You and your family are the friends
whom I seek. Do not you desert me
in the hour of trial!'

``\,`Great God!' exclaimed the old
man, `who are you?'

``At that instant the cottage door
was opened, and Felix, Safie, and Agatha
entered. Who can describe their
horror and consternation on beholding
me? Agatha fainted; and Safie, unable
to attend to her friend, rushed out of
the cottage. Felix darted forward, and
with supernatural force tore me from
his father, to whose knees I clung: in
a transport of fury, he dashed me to the
ground, and struck me violently with a
stick. I could have torn him limb from
limb, as the lion rends the antelope.
But my heart sunk within me as with
bitter sickness, and I refrained. I saw
him on the point of repeating his blow,
when, overcome by pain and anguish,
I quitted the cottage, and in the general
tumult escaped unperceived to my
hovel.''
%%183%%

\namedchapter{Chapter VIII}

``\textsc{Cursed}, cursed creator! Why did
I live? Why, in that instant, did I
not extinguish the spark of existence
which you had so wantonly bestowed?
I know not; despair had not yet taken
possession of me; my feelings were
those of rage and revenge. I could
with pleasure have destroyed the cottage
and its inhabitants, and have glutted
myself with their shrieks and misery.

``When night came, I quitted my
retreat, and wandered in the wood;
and now, no longer restrained by the
fear of discovery, I gave vent to my
anguish in fearful howlings. I was like
a wild beast that had broken the toils;
destroying the objects that obstructed
me, and ranging through the wood with
a stag-like swiftness. Oh! what a miserable
night I passed! the cold stars
shone in mockery, and the bare trees
waved their branches above me: now
and then the sweet voice of a bird burst
forth amidst the universal stillness.
All, save I, were at rest or in enjoyment:
I, like the arch fiend, bore a hell within
me; and, finding myself unsympathized
with, wished to tear up the trees,
spread havoc and destruction around
me, and then to have sat down and
enjoyed the ruin.

``But this was a luxury of sensation
%%184%%
that could not endure; I became fatigued
with excess of bodily exertion,
and sank on the damp grass in the
sick impotence of despair. There was
none among the myriads of men that
existed who would pity or assist me;
and should I feel kindness towards my
enemies? No: from that moment I
declared everlasting war against the
species, and, more than all, against
him who had formed me, and sent me
forth to this insupportable misery.

``The sun rose; I heard the voices
of men, and knew that it was impossible
to return to my retreat during
that day. Accordingly I hid myself in
some thick underwood, determining to
devote the ensuing hours to reflection
on my situation.

``The pleasant sunshine, and the
pure air of day, restored me to some
degree of tranquillity; and when I
considered what had passed at the cottage,
I could not help believing that I
had been too hasty in my conclusions.
I had certainly acted imprudently. It
was apparent that my conversation had
interested the father in my behalf, and
I was a fool in having exposed my
person to the horror of his children.
I ought to have familiarized the old
De Lacy to me, and by degrees have
discovered myself to the rest of his
family, when they should have been
prepared for my approach. But I did
not believe my errors to be irretrievable;
and, after much consideration,
I resolved to return to the cottage, seek
the old man, and by my representations
win him to my party.

``These thoughts calmed me, and in
%%185%%
the afternoon I sank into a profound
sleep; but the fever of my blood did
not allow me to be visited by peaceful
dreams. The horrible scene of the
preceding day was for ever acting before
my eyes; the females were flying,
and the enraged Felix tearing me from
his father's feet. I awoke exhausted;
and, finding that it was already night,
I crept forth from my hiding-place,
and went in search of food.

``When my hunger was appeased, I
directed my steps towards the well-known
path that conducted to the cottage.
All there was at peace. I crept
into my hovel, and remained in silent
expectation of the accustomed hour
when the family arose. That hour past,
the sun mounted high in the heavens,
but the cottagers did not appear. I
trembled violently, apprehending some
dreadful misfortune. The inside of
the cottage was dark, and I heard no
motion; I cannot describe the agony
of this suspence.

``Presently two countrymen passed
by; but, pausing near the cottage, they
entered into conversation, using violent
gesticulations; but I did not understand
what they said, as they spoke the
language of the country, which differed
from that of my protectors. Soon after,
however, Felix approached with another
man: I was surprised, as I knew
that he had not quitted the cottage
that morning, and waited anxiously to
discover, from his discourse, the meaning
of these unusual appearances.

``\,`Do you consider,' said his companion
to him, `that you will be
obliged to pay three months' rent, and
%%186%%
to lose the produce of your garden?
I do not wish to take any unfair advantage,
and I beg therefore that you will
take some days to consider of your
determination.'

``\,`It is utterly useless,' replied
Felix, `we can never again inhabit
your cottage. The life of my father is
in the greatest danger, owing to the
dreadful circumstance that I have related.
My wife and my sister will
never recover their horror. I entreat
you not to reason with me any more.
Take possession of your tenement, and
let me fly from this place.'

``Felix trembled violently as he said
this. He and his companion entered
the cottage, in which they remained for
a few minutes, and then departed. I
never saw any of the family of De
Lacy more.

``I continued for the remainder of
the day in my hovel in a state of utter
and stupid despair. My protectors
had departed, and had broken the only
link that held me to the world. For
the first time the feelings of revenge and
hatred filled my bosom, and I did not
strive to controul them; but, allowing
myself to be borne away by the stream,
I bent my mind towards injury and
death. When I thought of my friends,
of the mild voice of De Lacy, the
gentle eyes of Agatha, and the exquisite
beauty of the Arabian, these
thoughts vanished, and a gush of tears
somewhat soothed me. But again,
when I reflected that they had spurned
and deserted me, anger returned, a rage
of anger; and, unable to injure any
thing human, I turned my fury towards
%%187%%
inanimate objects. As night advanced,
I placed a variety of combustibles
around the cottage; and, after having
destroyed every vestige of cultivation
in the garden, I waited with forced
impatience until the moon had sunk to
commence my operations.

``As the night advanced, a fierce
wind arose from the woods, and quickly
dispersed the clouds that had loitered
in the heavens: the blast tore along
like a mighty avalanche, and produced
a kind of insanity in my spirits, that
burst all bounds of reason and reflection.
I lighted the dry branch of a
tree, and danced with fury around the
devoted cottage, my eyes still fixed on
the western horizon, the edge of which
the moon nearly touched. A part of
its orb was at length hid, and I waved
my brand; it sunk, and, with a loud
scream, I fired the straw, and heath,
and bushes, which I had collected. The
wind fanned the fire, and the cottage was
quickly enveloped by the flames, which
clung to it, and licked it with their
forked and destroying tongues.

``As soon as I was convinced that
no assistance could save any part of the
habitation, I quitted the scene, and
sought for refuge in the woods.

``And now, with the world before
me, whither should I bend my steps? I
resolved to fly far from the scene of my
misfortunes; but to me, hated and despised,
every country must be equally
horrible. At length the thought of
you crossed my mind. I learned from
your papers that you were my father, my
creator; and to whom could I apply with
more fitness than to him who had given
%%188%%
me life? Among the lessons that Felix
had bestowed upon Safie geography
had not been omitted: I had learned
from these the relative situations of the
different countries of the earth. You
had mentioned Geneva as the name of
your native town; and towards this
place I resolved to proceed.

``But how was I to direct myself?
I knew that I must travel in a south-westerly
direction to reach my destination;
but the sun was my only guide.
I did not know the names of the towns
that I was to pass through, nor could
I ask information from a single human
being; but I did not despair. From
you only could I hope for succour, although
towards you I felt no sentiment
but that of hatred. Unfeeling, heartless
creator! you had endowed me
with perceptions and passions, and then
cast me abroad an object for the scorn
and horror of mankind. But on you
only had I any claim for pity and redress,
and from you I determined to
seek that justice which I vainly attempted
to gain from any other being
that wore the human form.

``My travels were long, and the sufferings
I endured intense. It was late
in autumn when I quitted the district
where I had so long resided. I travelled
only at night, fearful of encountering
the visage of a human being.
Nature decayed around me, and the
sun became heatless; rain and snow
poured around me; mighty rivers were
frozen; the surface of the earth was
hard, and chill, and bare, and I found no
shelter. Oh, earth! how often did I imprecate
curses on the cause of my being!
The mildness of my nature had fled,
%%189%%
and all within me was turned to gall and
bitterness. The nearer I approached to
your habitation, the more deeply did I
feel the spirit of revenge enkindled in
my heart. Snow fell, and the waters
were hardened, but I rested not. A
few incidents now and then directed
me, and I possessed a map of the country;
but I often wandered wide from
my path. The agony of my feelings
allowed me no respite: no incident
occurred from which my rage and misery
could not extract its food; but a
circumstance that happened when I
arrived on the confines of Switzerland,
when the sun had recovered its warmth,
and the earth again began to look
green, confirmed in an especial manner
the bitterness and horror of my
feelings.

``I generally rested during the day,
and travelled only when I was secured
by night from the view of man. One
morning, however, finding that my
path lay through a deep wood, I ventured
to continue my journey after the
sun had risen; the day, which was one
of the first of spring, cheered even me
by the loveliness of its sunshine and
the balminess of the air. I felt emotions
of gentleness and pleasure, that
had long appeared dead, revive within
me. Half surprised by the novelty of
these sensations, I allowed myself to be
borne away by them; and, forgetting
my solitude and deformity, dared to be
happy. Soft tears again bedewed my
cheeks, and I even raised my humid
eyes with thankfulness towards the
blessed sun which bestowed such joy
upon me.

``I continued to wind among the
%%190%%
paths of the wood, until I came to its
boundary, which was skirted by a deep
and rapid river, into which many of the
trees bent their branches, now budding
with the fresh spring. Here I paused,
not exactly knowing what path to pursue,
when I heard the sound of voices,
that induced me to conceal myself under
the shade of a cypress. I was
scarcely hid, when a young girl came
running towards the spot where I was
concealed, laughing as if she ran from
some one in sport. She continued her
course along the precipitous sides of
the river, when suddenly her foot slipt,
and she fell into the rapid stream. I
rushed from my hiding place, and, with
extreme labour from the force of the
current, saved her, and dragged her
to shore. She was senseless; and I endeavoured,
by every means in my
power, to restore animation, when I
was suddenly interrupted by the approach
of a rustic, who was probably
the person from whom she had playfully
fled. On seeing me, he darted
towards me, and, tearing the girl from
my arms, hastened towards the deeper
parts of the wood. I followed speedily,
I hardly knew why; but when the
man saw me draw near, he aimed a
gun, which he carried, at my body, and
fired. I sunk to the ground, and
my injurer, with increased swiftness,
escaped into the wood.

``This was then the reward of my
benevolence! I had saved a human
being from destruction, and, as a recompence,
I now writhed under the
miserable pain of a wound, which
shattered the flesh and bone. The
%%191%%
feelings of kindness and gentleness,
which I had entertained but a few moments
before, gave place to hellish
rage and gnashing of teeth. Inflamed
by pain, I vowed eternal hatred and
vengeance to all mankind. But the
agony of my wound overcame me; my
pulses paused, and I fainted.

``For some weeks I led a miserable
life in the woods, endeavouring to cure
the wound which I had received. The
ball had entered my shoulder, and I
knew not wheth\-er it had remained
there or passed through; at any rate
I had no means of extracting it. My
sufferings were augmented also by the
oppressive sense of the injustice and
ingratitude of their infliction. My
daily vows rose for revenge --- a deep
and deadly revenge, such as would
alone compensate for the outrages and
anguish I had endured.

``After some weeks my wound
healed, and I continued my journey.
The labours I endured were no longer
to be alleviated by the bright sun or
gentle breezes of spring; all joy was
but a mockery, which insulted my desolate
state, and made me feel more
painfully that I was not made for the
enjoyment of pleasure.

``But my toils now drew near a
close; and, two months from this time,
I reached the environs of Geneva.

``It was evening when I arrived,
and I retired to a hiding-place among
the fields that surround it, to meditate
in what manner I should apply to you.
I was oppressed by fatigue and hunger,
and far too unhappy to enjoy the gentle
breezes of evening, or the prospect of
%%192%%
the sun setting behind the stupendous
mountains of Jura.

``At this time a slight sleep relieved
me from the pain of reflection, which
was disturbed by the approach of a
beautiful child, who came running into
the recess I had chosen with all the
sportiveness of infancy. Suddenly, as
I gazed on him, an idea seized me, that
this little creature was unprejudiced,
and had lived too short a time to have
imbibed a horror of deformity. If,
therefore, I could seize him, and educate
him as my companion and friend,
I should not be so desolate in this peopled
earth.

``Urged by this impulse, I seized on
the boy as he passed, and drew him towards
me. As soon as he beheld my
form, he placed his hands before his
eyes, and uttered a shrill scream: I
drew his hand forcibly from his face,
and said, `Child, what is the meaning
of this? I do not intend to hurt you;
listen to me.'

``He struggled violently; `Let me
go,' he cried; `monster! ugly wretch!
you wish to eat me, and tear me to
pieces --- You are an ogre --- Let me go,
or I will tell my papa.'

``\,`Boy, you will never see your father
again; you must come with me.'

``\,`Hideous monster! let me go;
My papa is a Syndic --- he is M.~Frankenstein --- he
would punish you. You
dare not keep me.'

``\,`Frankenstein! you belong then
to my enemy --- to him towards whom
I have sworn eternal revenge; you
shall be my first victim.'

``The child still struggled, and
%%193%%
loaded me with epithets which carried
despair to my heart: I grasped his
throat to silence him, and in a moment
he lay dead at my feet.

``I gazed on my victim, and my
heart swelled with exultation and hellish
triumph: clapping my hands, I exclaimed,
`I, too, can create desolation;
my enemy is not impregnable; this
death will carry despair to him, and a
thousand other miseries shall torment
and destroy him.'

``As I fixed my eyes on the child, I
saw something glittering on his breast.
I took it; it was a portrait of a most
lovely woman. In spite of my malignity,
it softened and attracted me.
For a few moments I gazed with delight
on her dark eyes, fringed by deep lashes,
and her lovely lips; but presently my
rage returned: I remembered that I
was for ever deprived of the delights
that such beautiful creatures could bestow;
and that she whose resemblance
I contemplated would, in regarding me,
have changed that air of divine benignity
to one expressive of disgust and
affright.

``Can you wonder that such thoughts
transported me with rage? I only
wonder that at that moment, instead of
venting my sensations in exclamations
and agony, I did not rush among mankind,
and perish in the attempt to destroy
them.

``While I was overcome by these
feelings, I left the spot where I had
committed the murder, and was seeking
a more secluded hiding-place, when
I perceived a woman passing near me.
She was young, not indeed so beautiful
%%194%%
as her whose portrait I held, but of an
agreeable aspect, and blooming in the
loveliness of youth and health. Here,
I thought, is one of those whose smiles
are bestowed on all but me; she shall
not escape: thanks to the lessons of
Felix, and the sanguinary laws of man,
I have learned how to work mischief. I
approached her unperceived, and placed
the portrait securely in one of the folds
of her dress.

``For some days I haunted the spot
where these scenes had taken place;
sometimes wishing to see you, sometimes
resolved to quit the world and its
miseries for ever. At length I wandered
towards these mountains, and
have ranged through their immense recesses,
consumed by a burning passion
which you alone can gratify. We may
not part until you have promised to
comply with my requisition. I am
alone, and miserable; man will not associate
with me; but one as deformed
and horrible as myself would not deny
herself to me. My companion must be
of the same species, and have the same
defects. This being you must create.''
%%195%%

\namedchapter{Chapter IX}

\textsc{The} being finished speaking, and
fixed his looks upon me in expectation
of a reply. But I was bewildered,
perplexed, and unable to arrange my
ideas sufficiently to understand the full
extent of his proposition. He continued~---

``You must create a female for me,
with whom I can live in the interchange
of those sympathies necessary
for my being. This you alone can do;
and I demand it of you as a right which
you must not refuse.''

The latter part of his tale had kindled
anew in me the anger that had
died away while he narrated his peaceful
life among the cottagers, and, as he
said this, I could no longer suppress
the rage that burned within me.

``I do refuse it,'' I replied; ``and no
torture shall ever extort a consent from
me. You may render me the most
miserable of men, but you shall never
make me base in my own eyes. Shall
I create another like yourself, whose
joint wickedness might desolate the
world. Begone! I have answered you;
you may torture me, but I will never
consent.''

``You are in the wrong,'' replied the
fiend; ``and, instead of threatening, I
am content to reason with you. I am
malicious because I am miserable; am
%%196%%
I not shunned and hated by all mankind?
You, my creator, would tear me
to pieces, and triumph; remember that,
and tell me why I should pity man
more than he pities me? You would
not call it murder, if you could precipitate
me into one of those ice-rifts, and
destroy my frame, the work of your
own hands. Shall I respect man, when
he contemns me? Let him live with
me in the interchange of kindness, and,
instead of injury, I would bestow every
benefit upon him with tears of gratitude
at his acceptance. But that cannot
be; the human senses are insurmountable
barriers to our union. Yet
mine shall not be the submission of
abject slavery. I will revenge my injuries:
if I cannot inspire love, I will
cause fear; and chiefly towards you
my arch-enemy, because my creator, do
I swear inextinguishable hatred. Have
a care: I will work at your destruction,
nor finish until I desolate your
heart, so that you curse the hour of
your birth.''

A fiendish rage animated him as he
said this; his face was wrinkled into
contortions too horrible for human eyes
to behold; but presently he calmed
himself, and proceeded~---

``I intended to reason. This passion
is detrimental to me; for you do not
reflect that you are the cause of its
excess. If any being felt emotions of
benevolence towards me, I should return
them an hundred and an hundred
fold; for that one creature's sake, I
would make peace with the whole kind!
But I now indulge in dreams of bliss
that cannot be realized. What I ask
%%197%%
of you is reasonable and moderate; I
demand a creature of another sex, but
as hideous as myself: the gratification
is small, but it is all that I can receive,
and it shall content me. It is true, we
shall be monsters, cut off from all the
world; but on that account we shall be
more attached to one another. Our lives
will not be happy, but they will be
harmless, and free from the misery I
now feel. Oh! my creator, make me
happy; let me feel gratitude towards
you for one benefit! Let me see
that I excite the sympathy of some
existing thing; do not deny me my request!''

I was moved. I shuddered when I
thought of the possible consequences
of my consent; but I felt that there was
some justice in his argument. His
tale, and the feelings he now expressed,
proved him to be a creature of fine sensations;
and did I not, as his maker,
owe him all the portion of happiness
that it was in my power to bestow? He
%%198%%
saw my change of feeling, and continued~---

``If you consent, neither you nor
any other human being shall ever see us
again: I will go to the vast wilds of
South America. My food is not that of
man; I do not destroy the lamb and
the kid, to glut my appetite; acorns
and berries afford me sufficient nourishment.
My companion will be of the
same nature as myself, and will be content
with the same fare. We shall
make our bed of dried leaves; the sun
will shine on us as on man, and will
ripen our food. The picture I present
to you is peaceful and human, and you
must feel that you could deny it only in
the wantonness of power and cruelty.
Pitiless as you have been towards me,
I now see compassion in your eyes:
let me seize the favourable moment,
and persuade you to promise what I
so ardently desire.''

``You propose,'' replied I, ``to fly
from the habitations of man, to dwell in
those wilds where the beasts of the
field will be your only companions.
How can you, who long for the love
and sympathy of man, persevere in this
exile? You will return, and again
seek their kindness, and you will meet
with their detestation; your evil passions
will be renewed, and you will then
have a companion to aid you in the task
of destruction. This may not be;
cease to argue the point, for I cannot
consent.''

``How inconstant are your feelings!
but a moment ago you were moved by
my representations, and why do you
again harden yourself to my
%%199%%
complaints? I swear to you, by the earth
which I inhabit, and by you that made
me, that, with the companion you
bestow, I will quit the neighbourhood
of man, and dwell, as it may
chance, in the most savage of places.
My evil passions will have fled, for
I shall meet with sympathy; my life
will flow quietly away, and, in my
dying moments, I shall not curse my
maker.''

His words had a strange effect upon
me. I compassionated him, and sometimes
felt a wish to console him; but
when I looked upon him, when I saw
the filthy mass that moved and talked,
my heart sickened, and my feelings
were altered to those of horror and
hatred. I tried to stifle these sensations;
I thought, that as I could not
sympathize with him, I had no right to
withhold from him the small portion of
happiness which was yet in my power
to bestow.

``You swear,'' I said, ``to be harmless;
but have you not already shewn a
degree of malice that should reasonably
make me distrust you? May not
even this be a feint that will increase
your triumph by affording a wider
scope for your revenge?''

``How is this? I thought I had
moved your compassion, and yet you
still refuse to bestow on me the only
benefit that can soften my heart, and
render me harmless. If I have no
ties and no affections, hatred and vice
must be my portion; the love of another
will destroy the cause of my crimes,
and I shall become a thing, of whose
existence every one will be ignorant.
%%200%%
My vices are the children of a forced
solitude that I abhor; and my virtues
will necessarily arise when I live in
communion with an equal. I shall
feel the affections of a sensitive being,
and become linked to the chain of existence
and events, from which I am
now excluded.''

I paused some time to reflect on all
he had related, and the various arguments
which he had employed. I
thought of the promise of virtues which
he had displayed on the opening of his
existence, and the subsequent blight of
all kindly feeling by the loathing and
scorn which his protectors had manifested
towards him. His power and
threats were not omitted in my calculations:
a creature who could exist in
the ice caves of the glaciers, and hide
himself from pursuit among the ridges
of inaccessible precipices, was a being
possessing faculties it would be vain to
cope with. After a long pause of reflection,
I concluded, that the justice
due both to him and my fellow-creatures
demanded of me that I should
comply with his request. Turning to
him, therefore, I said~---

``I consent to your demand, on your
solemn oath to quit Europe for ever,
and every other place in the neighbourhood
of man, as soon as I shall deliver
into your hands a female who will accompany
you in your exile.''

``I swear,'' he cried, ``by the sun,
and by the blue sky of heaven, that if
you grant my prayer, while they exist
you shall never behold me again. Depart
to your home, and commence
your labours: I shall watch their progress
with unutterable anxiety; and fear
%%201%%
not but that when you are ready I shall
appear.''

Saying this, he suddenly quitted me,
fearful, perhaps, of any change in my
sentiments. I saw him descend the
mountain with greater speed than the
flight of an eagle, and quickly lost him
among the undulations of the sea of
ice.

His tale had occupied the whole day;
and the sun was upon the verge of the
horizon when he departed. I knew
that I ought to hasten my descent towards
the valley, as I should soon be
encompassed in darkness; but my heart
was heavy, and my steps slow. The
labour of winding among the little
paths of the mountains, and fixing my
feet firmly as I advanced, perplexed
me, occupied as I was by the emotions
which the occurrences of the day had
produced. Night was far advanced,
when I came to the half-way resting-place,
and seated myself beside the
fountain. The stars shone at intervals,
as the clouds passed from over them;
the dark pines rose before me, and
every here and there a broken tree lay
on the ground: it was a scene of wonderful
solemnity, and stirred strange
thoughts within me. I wept bitterly;
and, clasping my hands in agony, I
exclaimed, ``Oh! stars, and clouds,
and winds, ye are all about to mock
me: if ye really pity me, crush sensation
and memory; let me become as
nought; but if not, depart, depart and
leave me in darkness.''

These were wild and miserable
thoughts; but I cannot describe to you
how the eternal twinkling of the stars
weighed upon me, and how I listened
%%202%%
to every blast of wind, as if it were a dull
ugly siroc on its way to consume me.

Morning dawned before I arrived at
the village of Ch\-amou\-nix; but my presence,
so haggard and strange, hardly
calmed the fears of my family, who had
waited the whole night in anxious expectation
of my return.

The following day we returned to
Geneva. The intention of my father
in coming had been to divert my mind,
and to restore me to my lost tranquillity;
but the medicine had been fatal. And,
unable to account for the excess of
misery I appeared to suffer, he hastened
to return home, hoping the quiet and
monotony of a domestic life would by
degrees alleviate my sufferings from
whatsoever cause they might spring.

For myself, I was passive in all their
arrangements; and the gentle affection
of my beloved Elizabeth was inadequate
to draw me from the depth of my
despair. The promise I had made to
the dæmon weighed upon my mind,
like Dante's iron cowl on the heads of
the hellish hypocrites. All pleasures
of earth and sky passed before me like
a dream, and that thought only had to
me the reality of life. Can you wonder,
that sometimes a kind of insanity possessed
me, or that I saw continually
about me a multitude of filthy animals
inflicting on me incessant torture,
that often extorted screams and bitter
groans?

By degrees, however, these feelings
became calmed. I entered again into
the every-day scene of life, if not with
interest, at least with some degree of
tranquillity.
%%203%%
%%204%%
\namedpart{Volume III}
\namedchapter{Chapter I}

\textsc{Day} after day, week after week, passed
away on my return to Geneva; and I
could not collect the courage to recommence
my work. I feared the vengeance
of the disappointed fiend, yet I was
unable to overcome my repugnance
to the task which was enjoined me. I
found that I could not compose a female
without again devoting several
months to profound study and laborious
disquisition. I had heard of
some discoveries having been made by
an English philosopher, the knowledge
of which was material to my success,
and I sometimes thought of obtaining
my father's consent to visit England
for this purpose; but I clung to every
pretence of delay, and could not resolve
to interrupt my returning tranquillity.
My health, which had hitherto declined,
was now much restored; and my spirits,
when unchecked by the memory of
my unhappy promise, rose proportionably.
My father saw this change with
pleasure, and he turned his thoughts
towards the best method of eradicating
%%205%%
the remains of my melancholy, which
every now and then would return by
fits, and with a devouring blackness
overcast the approaching sunshine. At
these moments I took refuge in the
most perfect solitude. I passed whole
days on the lake alone in a little boat,
watching the clouds, and listening to
the rippling of the waves, silent and
listless. But the fresh air and bright
sun seldom failed to restore me to some
degree of composure; and, on my return,
I met the salutations of my friends
with a readier smile and a more cheerful
heart.

It was after my return from one of
these rambles that my father, calling
me aside, thus addressed me:---

``I am happy to remark, my dear
son, that you have resumed your former
pleasures, and seem to be returning to
yourself. And yet you are still unhappy,
and still avoid our society. For
some time I was lost in conjecture as
to the cause of this; but yesterday an
idea struck me, and if it is well founded,
I conjure you to avow it. Reserve
on such a point would be not only
useless, but draw down treble misery
on us all.''

I trembled violently at this exordium,
and my father continued~---

``I confess, my son, that I have always
looked forward to your marriage
with your cousin as the tie of our domestic
comfort, and the stay of my declining
years. You were attached to
each other from your earliest infancy;
you studied together, and appeared, in
dispositions and tastes, entirely suited
to one another. But so blind is the
%%206%%
experience of man, that what I conceived
to be the best assistants to my
plan may have entirely destroyed it.
You, perhaps, regard her as your sister,
without any wish that she might become
your wife. Nay, you may have
met with another whom you may love;
and, considering yourself as bound in
honour to your cousin, this struggle
may occasion the poignant misery which
you appear to feel.''

``My dear father, re-assure yourself.
I love my cousin tenderly and sincerely.
I never saw any woman who excited,
as Elizabeth does, my warmest admiration
and affection. My future hopes
and prospects are entirely bound up in
the expectation of our union.''

``The expression of your sentiments
on this subject, my dear Victor, gives
me more pleasure than I have for some
time experienced. If you feel thus,
we shall assuredly be happy, however
present events may cast a gloom over
us. But it is this gloom, which appears
to have taken so strong a hold of
your mind, that I wish to dissipate.
Tell me, therefore, whether you object
to an immediate solemnization of the
marriage. We have been unfortunate,
and recent events have drawn us from
that every-day tranquillity befitting my
years and infirmities. You are younger;
yet I do not suppose, possessed as you
are of a competent fortune, that an
early marriage would at all interfere
with any future plans of honour and
utility that you may have formed. Do
not suppose, however, that I wish to
dictate happiness to you, or that a delay
on your part would cause me any
%%207%%
serious uneasiness. Interpret my words
with candour, and answer me, I conjure
you, with confidence and sincerity.''

I listened to my father in silence,
and remained for some time incapable
of offering any reply. I revolved rapidly
in my mind a multitude of thoughts,
and endeavoured to arrive at some conclusion.
Alas! to me the idea of an
immediate union with my cousin was
one of horror and dismay. I was bound
by a solemn promise, which I had not
yet fulfilled, and dared not break; or,
if I did, what manifold miseries might
not impend over me and my devoted
family! Could I enter into a festival
with this deadly weight yet hanging
round my neck, and bowing me to the
ground. I must perform my engagement,
and let the monster depart with
his mate, before I allowed myself to
enjoy the delight of an union from
which I expected peace.

I remembered also the necessity imposed
upon me of either journeying to
England, or entering into a long correspondence
with those philosophers of
that country, whose knowledge and discoveries
were of indispensable use to
me in my present undertaking. The
latter method of obtaining the desired
intelligence was dilatory and unsatisfactory:
besides, any variation was
agreeable to me, and I was delighted
with the idea of spending a year or two
in change of scene and variety of occupation,
in absence from my family;
during which period some event might
happen which would restore me to them
in peace and happiness: my promise
%%208%%
might be fulfilled, and the monster
have departed; or some accident might
occur to destroy him, and put an end
to my slavery for ever.

These feelings dictated my answer to
my father. I expressed a wish to visit
England; but, concealing the true reasons
of this request, I clothed my desires
under the guise of wishing to
travel and see the world before I sat
down for life within the walls of my
native town.

I urged my entreaty with earnestness,
and my father was easily induced
to comply; for a more indulgent and
less dictatorial parent did not exist
upon earth. Our plan was soon arranged.
I should travel to Strasburgh,
where Clerval would join me. Some
short time would be spent in the towns
of Holland, and our principal stay
would be in England. We should return
by France; and it was agreed
that the tour should occupy the space
of two years.

My father pleased himself with the
reflection, that my union with Elizabeth
should take place immediately on my
return to Geneva. ``These two years,''
said be, ``will pass swiftly, and it will
be the last delay that will oppose itself
to your happiness. And, indeed, I
earnestly desire that period to arrive,
when we shall all be united, and neither
hopes or fears arise to disturb our
domestic calm.''

``I am content,'' I replied, ``with
your arrangement. By that time we
shall both have become wiser, and I
hope happier, than we at present are.''
I sighed; but my father kindly forbore
%%209%%
to question me further concerning the
cause of my dejection. He hoped that
new scenes, and the amusement of travelling,
would restore my tranquillity.

I now made arrangements for my
journey; but one feeling haunted me,
which filled me with fear and agitation.
During my absence I should leave my
friends unconscious of the existence of
their enemy, and unprotected from his
attacks, exasperated as he might be
by my departure. But he had promised
to follow me wherever I might go; and
would he not accompany me to England?
This imagination was dreadful
in itself, but soothing, inasmuch as it
supposed the safety of my friends. I
was agonized with the idea of the possibility
that the reverse of this might
happen. But through the whole period
during which I was the slave of my
creature, I allowed myself to be governed
by the impulses of the moment;
and my present sensations strongly intimated
that the fiend would follow me,
and exempt my family from the danger
of his machinations.

It was in the latter end of August
that I departed, to pass two years of
exile. Elizabeth approved of the reasons
of my departure, and only regretted
that she had not the same opportunities
of enlarging her experience,
and cultivating her understanding. She
wept, however, as she bade me farewell,
and entreated me to return happy
and tranquil. ``We all,'' said she,
``depend upon you; and if you are
miserable, what must be our feelings?''
%%210%%

I threw myself into the carriage that
was to convey me away, hardly knowing
whither I was going, and careless
of what was passing around. I remembered
only, and it was with a
bitter anguish that I reflected on it,
to order that my chemical instruments
should be packed to go with me: for
I resolved to fulfil my promise while
abroad, and return, if possible, a free
man. Filled with dreary imaginations,
I passed through many beautiful and
majestic scenes; but my eyes were
fixed and unobserving. I could only
think of the bourne of my travels, and
the work which was to occupy me
whilst they endured.

After some days spent in listless indolence,
during which I traversed many
leagues, I arrived at Strasburgh, where I
waited two days for Clerval. He came.
Alas, how great was the contrast between
us! He was alive to every new scene; joyful
when he saw the beauties of the setting
sun, and more happy when he beheld
it rise, and recommence a new day. He
pointed out to me the shifting colours
of the landscape, and the appearances
of the sky. ``This is what it is to
live;'' he cried, ``now I enjoy existence!
But you, my dear Frankenstein, wherefore
are you desponding and sorrowful?''
In truth, I was occupied by
gloomy thoughts, and neither saw the
descent of the evening star, nor the
golden sun-rise reflected in the Rhine. --- And
you, my friend, would be far more
amused with the journal of Clerval,
who observed the scenery with an eye
of feeling and delight, than to listen to
%%211%%
my reflections. I, a miserable wretch,
haunted by a curse that shut up every
avenue to enjoyment.

We had agreed to descend the Rhine
in a boat from Strasburgh to Rotterdam,
whence we might take shipping
for London. During this voyage, we
passed by many willowy islands, and
saw several beautiful towns. We staid
a day at Manheim, and, on the fifth
from our departure from Strasburgh,
arrived at Mayence. The course of the
Rhine below Mayence becomes much
more picturesque. The river descends
rapidly, and winds between hills, not
high, but steep, and of beautiful forms.
We saw many ruined castles standing
on the edges of precipices, surrounded
by black woods, high and inaccessible.
This part of the Rhine, indeed, presents
a singularly variegated landscape.
In one spot you view rugged hills,
ruined castles overlooking tremendous
precipices, with the dark Rhine rushing
beneath; and, on the sudden turn of a
promontory, flourishing vineyards, with
green sloping banks, and a meandering
river, and populous towns, occupy the
scene.

We travelled at the time of the vintage,
and heard the song of the labourers,
as we glided down the stream.
Even I, depressed in mind, and my
spirits continually agitated by gloomy
feelings, even I was pleased. I lay at
the bottom of the boat, and, as I gazed
on the cloudless blue sky, I seemed
to drink in a tranquillity to which I
had long been a stranger. And if these
were my sensations, who can describe
those of Henry? He felt as if he had
%%212%%
been transported to Fairy-land, and enjoyed
a happiness seldom tasted by
man. ``I have seen,'' he said, ``the
most beautiful scenes of my own
country; I have visited the lakes of
Lucerne and Uri, where the snowy
mountains descend almost perpendicularly
to the water, casting black and impenetrable
shades, which would cause
a gloomy and mournful appearance,
were it not for the most verdant islands
that relieve the eye by their gay appearance;
I have seen this lake agitated
by a tempest, when the wind tore up
whirlwinds of water, and gave you an
idea of what the water-spout must be
on the great ocean, and the waves dash
with fury the base of the mountain,
where the priest and his mistress were
overwhelmed by an avalanche, and
where their dying voices are still said
to be heard amid the pauses of the
nightly wind; I have seen the mountains
of La Valais, and the Pays de
Vaud: but this country, Victor, pleases
me more than all those wonders. The
mountains of Switzerland are more
majestic and strange; but there is a
charm in the banks of this divine river,
that I never before saw equalled. Look
at that castle which overhangs yon precipice;
and that also on the island, almost
concealed amongst the foliage of
those lovely trees; and now that group
of labourers coming from among their
vines; and that village half-hid in the
recess of the mountain. Oh, surely,
the spirit that inhabits and guards
this place has a soul more in harmony
with man, than those who pile
the glacier, or retire to the inaccessible
%%213%%
peaks of the mountains of our own
country.''

Clerval! beloved friend! even now
it delights me to record your words, and
to dwell on the praise of which you are
so eminently deserving. He was a
being formed in the ``very poetry of
nature.'' His wild and enthusiastic
imagination was chastened by the sensibility
of his heart. His soul overflowed
with ardent affections, and his
friendship was of that devoted and wondrous
nature that the worldly-minded
teach us to look for only in the imagination.
But even human sympathies
were not sufficient to satisfy his eager
mind. The scenery of external nature,
which others regard only with admiration,
he loved with ardour:

\vspace*{1em}

{\noindent\small
\hspace*{2em}------ ------ ``The sounding cataract\\
\hspace*{2em}Haunted \emph{him} like a passion: the tall rock,\\
\hspace*{2em}The mountain, and the deep and gloomy wood,\\
\hspace*{2em}Their colours and their forms, were then to him\\
\hspace*{2em}An appetite; a feeling, and a love,\\
\hspace*{2em}That had no need of a remoter charm,\\
\hspace*{2em}By thought supplied, or any interest\\
\hspace*{2em}Unborrowed from the eye.''}

\vspace*{1em plus 1em}

And where does he now exist? Is
this gentle and lovely being lost for
ever? Has this mind so replete with
ideas, imaginations fanciful and magnificent,
which formed a world, whose
existence depended on the life of its
creator; has this mind perished? Does
it now only exist in my memory? No,
it is not thus; your form so divinely
wrought, and beaming with beauty,
has decayed, but your spirit still visits
and consoles your unhappy friend.

Pardon this gush of sorrow; these
ineffectual words are but a slight
%%214%%
tribute to the unexampled worth of Henry,
but they soothe my heart, overflowing
with the anguish which his remembrance
creates. I will proceed with
my tale.

Beyond Cologne we descended to the
plains of Holland; and we resolved to
post the remainder of our way; for the
wind was contrary, and the stream of
the river was too gentle to aid us.

Our journey here lost the interest
arising from beautiful scenery; but we
arrived in a few days at Rotterdam,
whence we proceeded by sea to England.
It was on a clear morning,
in the latter days of December, that I
first saw the white cliffs of Britain.
The banks of the Thames presented a
new scene; they were flat, but fertile,
and almost every town was marked by
%%215%%
the remembrance of some story. We
saw Tilbury Fort, and remembered the
Spanish armada; Gravesend, Woolwich,
and Greenwich, places which I had
heard of even in my country.

At length we saw the numerous
steeples of London, St.~Paul's towering
above all, and the Tower famed in
English history.
%%216%%

\namedchapter{Chapter II}

\textsc{London} was our present point of rest;
we determined to remain several months
in this wonderful and celebrated city.
Clerval desired the intercourse of the
men of genius and talent who flourished
at this time; but this was with
me a secondary object; I was principally
occupied with the means of obtaining
the information necessary for
the completion of my promise, and
quickly availed myself of the letters of
introduction that I had brought with
me, addressed to the most distinguished
natural philosophers.

If this journey had taken place during
my days of study and happiness, it would
have afforded me inexpressible pleasure.
But a blight had come over my existence,
and I only visited these people for
the sake of the information they might
give me on the subject in which my
interest was so terribly profound. Company
was irksome to me; when alone,
I could fill my mind with the sights of
heaven and earth; the voice of Henry
soothed me, and I could thus cheat myself
into a transitory peace. But busy
uninteresting joyous faces brought back
despair to my heart. I saw an insurmountable
barrier placed between me
and my fellow-men; this barrier was
sealed with the blood of William and
Justine; and to reflect on the events
%%217%%
connected with those names filled my
soul with anguish.

But in Clerval I saw the image of my
former self; he was inquisitive, and anxious
to gain experience and instruction.
The difference of manners which
he observed was to him an inexhaustible
source of instruction and
amusement. He was for ever busy; and
the only check to his enjoyments was
my sorrowful and dejected mien. I
tried to conceal this as much as possible,
that I might not debar him from
the pleasures natural to one who was
entering on a new scene of life, undisturbed
by any care or bitter recollection.
I often refused to accompany
him, alleging another engagement, that
I might remain alone. I now also began
to collect the materials necessary
for my new creation, and this was to
me like the torture of single drops of
water continually falling on the head.
Every thought that was devoted to it
was an extreme anguish, and every
word that I spoke in allusion to it
caused my lips to quiver, and my heart
to palpitate.

After passing some months in London,
we received a letter from a person
in Scotland, who had formerly been
our visitor at Geneva. He mentioned
the beauties of his native country, and
asked us if those were not sufficient
allurements to induce us to prolong
our journey as far north as Perth, where
he resided. Clerval eagerly desired to
accept this invitation; and I, although
I abhorred society, wished to view
again mountains and streams, and all
the wondrous works with which Nature
adorns her chosen dwelling-places.
%%218%%

We had arrived in England at the
beginning of October, and it was now
February. We accordingly determined
to commence our journey towards the
north at the expiration of another
month. In this expedition we did not
%%219%%
intend to follow the great road to Edinburgh,
but to visit Windsor, Oxford,
Matlock, and the Cumberland lakes,
resolving to arrive at the completion of
this tour about the end of July. I
packed my chemical instruments, and
the materials I had collected, resolving
to finish my labours in some obscure
nook in the northern highlands of Scotland.

We quitted London on the 27th of
March, and remained a few days at
Windsor, rambling in its beautiful forest.
This was a new scene to us mountaineers;
the majestic oaks, the quantity
of game, and the herds of stately
deer, were all novelties to us.

From thence we proceeded to Oxford.
As we entered this city, our minds
were filled with the remembrance of
the events that had been transacted
there more than a century and a half
before. It was here that Charles I.~%
had collected his forces. This city had
remained faithful to him, after the whole
nation had forsaken his cause to join
the standard of parliament and liberty.
The memory of that unfortunate king,
%%220%%
and his companions, the amiable Falkland,
the insolent Gower, his queen,
and son, gave a peculiar interest to
every part of the city, which they might
be supposed to have inhabited. The
spirit of elder days found a dwelling
here, and we delighted to trace its
footsteps. If these feelings had not
found an imaginary gratification, the
appearance of the city had yet in itself
sufficient beauty to obtain our admiration.
The colleges are ancient and
picturesque; the streets are almost magnificent;
and the lovely Isis, which
flows beside it through meadows of exquisite
verdure, is spread forth into a
placid expanse of waters, which reflects
its majestic assemblage of towers, and
spires, and domes, embosomed among
aged trees.

I enjoyed this scene; and yet my
%%221%%
enjoyment was embittered both by the
memory of the past, and the anticipation
of the future. I was formed for
peaceful happiness. During my youthful
days discontent never visited my
mind; and if I was ever overcome by
\emph{ennui}, the sight of what is beautiful in
nature, or the study of what is excellent
and sublime in the productions of man,
could always interest my heart, and
communicate elasticity to my spirits.
But I am a blasted tree; the bolt has
entered my soul; and I felt then that I
should survive to exhibit, what I shall
soon cease to be --- a miserable spectacle
of wrecked humanity, pitiable to others,
and abhorrent to myself.

We passed a considerable period at
Oxford, rambling among its environs,
and endeavouring to identify every spot
which might relate to the most animating
epoch of English history. Our
little voyages of discovery were often
prolonged by the successive objects that
presented themselves. We visited the
tomb of the illustrious Hampden, and
the field on which that patriot fell. For
a moment my soul was elevated from
its debasing and miserable fears to contemplate
the divine ideas of liberty and
self-sacrifice, of which these sights were
the monuments and the remembrancers.
For an instant I dared to shake off my
chains, and look around me with a free
and lofty spirit; but the iron had eaten
into my flesh, and I sank again, trembling
and hopeless, into my miserable
self.

We left Oxford with regret, and proceeded
to Matlock, which was our next
place of rest. The country in the
%%222%%
neighbourhood of this village resembled,
to a greater degree, the scenery
of Switzerland; but every thing is
on a lower scale, and the green hills
want the crown of distant white Alps,
which always attend on the piny
mountains of my native country. We
visited the wondrous cave, and the little
cabinets of natural history, where the
curiosities are disposed in the same
manner as in the collections at Servox
and Chamounix. The latter name made
me tremble, when pronounced by Henry;
and I hastened to quit Matlock, with
which that terrible scene was thus associated.

From Derby still journeying northward,
we passed two months in Cumberland
and Westmoreland. I could
now almost fancy myself among the
Swiss mountains. The little patches
of snow which yet lingered on the
northern sides of the mountains, the
lakes, and the dashing of the rocky
streams, were all familiar and dear
sights to me. Here also we made some
acquaintances, who almost contrived to
cheat me into happiness. The delight
of Clerval was proportionably greater
than mine; his mind expanded in the
company of men of talent, and he found
in his own nature greater capacities and
resources than he could have imagined
himself to have possessed while he associated
with his inferiors. ``I could pass
my life here,'' said he to me; ``and among
these mountains I should scarcely regret
Switzerland and the Rhine.''

But he found that a traveller's life is
one that includes much pain amidst
its enjoyments. His feelings are for
%%223%%
ever on the stretch; and when he begins
to sink into repose, he finds himself
obliged to quit that on which he
rests in pleasure for something new,
which again engages his attention, and
which also he forsakes for other novelties.

We had scarcely visited the various
lakes of Cumberland and Westmoreland,
and conceived an affection for
some of the inhabitants, when the period
of our appointment with our Scotch
friend approached, and we left them to
travel on. For my own part I was not
sorry. I had now neglected my promise
for some time, and I feared the effects
of the dæmon's disappointment. He
might remain in Switzerland, and wreak
his vengeance on my relatives. This
idea pursued me, and tormented me at
every moment from which I might otherwise
have snatched repose and peace. I
waited for my letters with feverish impatience:
if they were delayed, I was
miserable, and overcome by a thousand
fears; and when they arrived, and I saw
the superscription of Elizabeth or my
father, I hardly dared to read and ascertain
my fate. Sometimes I thought
that the fiend followed me, and might
expedite my remissness by murdering
my companion. When these thoughts
possessed me, I would not quit Henry
for a moment, but followed him as his
shadow, to protect him from the fancied
rage of his destroyer. I felt as if I
had committed some great crime, the
consciousness of which haunted me. I
was guiltless, but I had indeed drawn
down a horrible curse upon my head,
as mortal as that of crime.

I visited Edinburgh with languid
%%224%%
eyes and mind; and yet that city might
have interested the most unfortunate
being. Clerval did not like it so well
as Oxford; for the antiquity of the latter
city was more pleasing to him. But
the beauty and regularity of the new
town of Edinburgh, its romantic castle,
and its environs, the most delightful
in the world, Arthur's Seat, St.~Bernard's
Well, and the Pentland Hills,
compensated him for the change, and
filled him with cheerfulness and admiration.
But I was impatient to
arrive at the termination of my journey.

We left Edinburgh in a week, passing
through Coupar, St.~Andrews, and
%%225%%
along the banks of the Tay, to Perth,
where our friend expected us. But
I was in no mood to laugh and talk
with strangers, or enter into their feelings
or plans with the good humour
expected from a guest; and accordingly
I told Clerval that I wished to
make the tour of Scotland alone. ``Do
you,'' said I, ``enjoy yourself, and let
this be our rendezvous. I may be absent
a month or two; but do not
interfere with my motions, I entreat
you: leave me to peace and solitude
for a short time; and when I return, I
hope it will be with a lighter heart,
more congenial to your own temper.''

Henry wished to dissuade me; but,
seeing me bent on this plan, ceased to
remonstrate. He entreated me to write
often. ``I had rather be with you,''
he said, ``in your solitary rambles,
than with these Scotch people, whom
I do not know: hasten then, my dear
friend, to return, that I may again feel
myself somewhat at home, which I
cannot do in your absence.''

Having parted from my friend, I
determined to visit some remote spot
of Scotland, and finish my work in solitude.
I did not doubt but that the
%%226%%
monster followed me, and would discover
himself to me when I should
have finished, that he might receive his
companion.

With this resolution I traversed the
northern highlands, and fixed on one of
the remotest of the Orkneys as the
scene labours. It was a place fitted for
such a work, being hardly more than a
rock, whose high sides were continually
beaten upon by the waves. The soil
was barren, scarcely affording pasture
for a few miserable cows, and oatmeal
for its inhabitants, which consisted of
five persons, whose gaunt and scraggy
limbs gave tokens of their miserable
fare. Vegetables and bread, when they
indulged in such luxuries, and even
fresh water, was to be procured from
%%227%%
the main land, which was about five
miles distant.

On the whole island there were but
three miserable huts, and one of these
was vacant when I arrived. This I
hired. It contained but two rooms, and
these exhibited all the squalidness of
the most miserable penury. The thatch
had fallen in, the walls were unplastered,
and the door was off its hinges.
I ordered it to be repaired, bought
some furniture, and took possession;
an incident which would, doubtless,
have occasioned some surprise, had not
all the senses of the cottagers been benumbed
by want and squalid poverty.
As it was, I lived ungazed at and unmolested,
hardly thanked for the pittance
of food and clothes which I gave; so
much does suffering blunt even the
coarsest sensations of men.

In this retreat I devoted the morning
to labour; but in the evening, when the
weather permitted, I walked on the
stony beach of the sea, to listen to the
waves as they roared, and dashed at
my feet. It was a monotonous, yet
ever-changing scene. I thought of
Switzerland; it was far different from
this desolate and appalling landscape.
Its hills are covered with vines, and its
cottages are scattered thickly in the
plains. Its fair lakes reflect a blue and
gentle sky; and, when troubled by the
winds, their tumult is but as the play
of a lively infant, when compared to the
roarings of the giant ocean.

In this manner I distributed my occupations
when I first arrived; but, as
I proceeded in my labour, it became
%%228%%
every day more horrible and irksome
to me. Sometimes I could not prevail
on myself to enter my laboratory
for several days; and at other times
I toiled day and night in order to complete
my work. It was indeed a filthy
process in which I was engaged. During
my first experiment, a kind of enthusiastic
frenzy had blinded me to the horror
of my employment; my mind was
intently fixed on the sequel of my labour,
and my eyes were shut to the
horror of my proceedings. But now
I went to it in cold blood, and my
heart often sickened at the work of my
hands.

Thus situated, employed in the most
detestable occupation, immersed in a
solitude where nothing could for an instant
call my attention from the actual
scene in which I was engaged, my spirits
became unequal; I grew restless and
nervous. Every moment I feared to
meet my persecutor. Sometimes I sat
with my eyes fixed on the ground,
fearing to raise them lest they should
encounter the object which I so much
dreaded to behold. I feared to wander
from the sight of my fellow-creatures,
lest when alone he should come to
claim his companion.

In the mean time I worked on, and
my labour was already considerably advanced.
I looked towards its completion
with a tremulous and eager
hope, which I dared not trust myself
to question, but which was intermixed
with obscure forebodings of evil, that
made my heart sicken in my bosom.
%%229%%

\namedchapter{Chapter III}

\textsc{I sat} one evening in my laboratory;
the sun had set, and the moon was just
rising from the sea; I had not sufficient
light for my employment, and I remained
idle, in a pause of consideration of whether
I should leave my labour for the
night, or hasten its conclusion by an
unremitting attention to it. As I sat, a
train of reflection occurred to me,
which led me to consider the effects of
what I was now doing. Three years
before I was engaged in the same manner,
and had created a fiend whose unparalleled
barbarity had desolated my
heart, and filled it for ever with the bitterest
remorse. I was now about to
form another being, of whose dispositions
I was alike ignorant; she might become
ten thousand times more malignant
than her mate, and delight, for its
own sake, in murder and wretchedness.
He had sworn to quit the neighbourhood
of man, and hide himself in deserts; but
she had not; and she, who in all probability
was to become a thinking and
reasoning animal, might refuse to comply
with a compact made before her
creation. They might even hate each
other; the creature who already lived
loathed his own deformity, and might
he not conceive a greater abhorence for
it when it came before his eyes in the
%%230%%
female form? She also might turn with
disgust from him to the superior beauty
of man; she might quit him, and he be
again alone, exasperated by the fresh
provocation of being deserted by one
of his own species.

Even if they were to leave Europe,
and inhabit the deserts of the new
world, yet one of the first results of
those sympathies for which the dæmon
thirsted would be children, and a race
of devils would be propagated upon the
earth, who might make the very existence
of the species of man a condition
precarious and full of terror. Had
I a right, for my own benefit, to inflict
this curse upon everlasting generations?
I had before been moved
by the sophisms of the being I had
created; I had been struck senseless
by his fiendish threats: but now, for
the first time, the wickedness of my
promise burst upon me; I shuddered
to think that future ages might curse
me as their pest, whose selfishness had
not hesitated to buy its own peace at
the price perhaps of the existence of
the whole human race.

I trembled, and my heart failed within
me; when, on looking up, I saw, by
the light of the moon, the dæmon at
the casement. A ghastly grin wrinkled
his lips as he gazed on me, where I sat
fulfilling the task which he had allotted
to me. Yes, he had followed me in my
travels; he had loitered in forests, hid
himself in caves, or taken refuge in
wide and desert heaths; and he now
came to mark my progress, and claim
the fulfilment of my promise.

As I looked on him, his countenance
%%231%%
expressed the utmost extent of malice
and treachery. I thought with a sensation
of madness on my promise of
creating another like to him, and, trembling
with passion, tore to pieces the
thing on which I was engaged. The
wretch saw me destroy the creature on
whose future existence he depended for
happiness, and, with a howl of devilish
despair and revenge, withdrew.

I left the room, and, locking the
door, made a solemn vow in my own
heart never to resume my labours; and
then, with trembling steps, I sought my
own apartment. I was alone; none
were near me to dissipate the gloom,
and relieve me from the sickening oppression
of the most terrible reveries.

Several hours past, and I remained
near my window gazing on the sea; it
was almost motionless, for the winds
were hushed, and all nature reposed
under the eye of the quiet moon. A
few fishing vessels alone specked the
water, and now and then the gentle
breeze wafted the sound of voices, as
the fishermen called to one another.
I felt the silence, although I was hardly
conscious of its extreme profundity
until my ear was suddenly arrested by
the paddling of oars near the shore,
and a person landed close to my house.

In a few minutes after, I heard the
creaking of my door, as if some one
endeavoured to open it softly. I trembled
from head to foot; I felt a presentiment
of who it was, and wished to rouse
one of the peasants who dwelt in a
cottage not far from mine; but I was
overcome by the sensation of helplessness,
so often felt in frightful dreams,
%%232%%
when you in vain endeavour to fly from
an impending danger, and was rooted
to the spot.

Presently I heard the sound of footsteps
along the passage; the door
opened, and the wretch whom I dreaded
appeared. Shutting the door, he approached
me, and said, in a smothered
voice~---

``You have destroyed the work
which you began; what is it that you intend?
Do you dare to break your promise?
I have endured toil and misery:
I left Switzerland with you; I crept
along the shores of the Rhine, among
its willow islands, and over the summits
of its hills. I have dwelt many
months in the heaths of England, and
among the deserts of Scotland. I have
endured incalculable fatigue, and cold,
and hunger; do you dare destroy my
hopes?''

``Begone! I do break my promise;
never will I create another like yourself,
equal in deformity and wickedness.''

``Slave, I before reasoned with you,
but you have proved yourself unworthy
of my condescension. Remember that
I have power; you believe yourself
miserable, but I can make you so
wretched that the light of day will be
hateful to you. You are my creator,
but I am your master; --- obey!''

``The hour of my weakness is past,
and the period of your power is arrived.
Your threats cannot move me
to do an act of wickedness; but they
confirm me in a resolution of not
creating you a companion in vice.
Shall I, in cool blood, set loose upon
%%233%%
the earth a dæmon, whose delight is
in death and wretchedness. Begone!
I am firm, and your words will only
exasperate my rage.''

The monster saw my determination
in my face, and gnashed his teeth in the
impotence of anger. ``Shall each man,''
cried he, ``find a wife for his bos\-om,
and each beast have his mate, and I be
alone? I had feelings of affection, and
they were requited by detestation and
scorn. Man, you may hate; but beware!
Your hours will pass in dread and
misery, and soon the bolt will fall
which must ravish from you your happiness
for ever. Are you to be happy,
while I grovel in the intensity of my
wretchedness? You can blast my
other passions; but revenge remains --- revenge,
henceforth dearer than light or
food! I may die; but first you, my tyrant
and tormentor, shall curse the sun that
gazes on your misery. Beware; for I
am fearless, and therefore powerful. I
will watch with the wiliness of a snake,
that I may sting with its venom. Man,
you shall repent of the injuries you
inflict.''

``Devil, cease; and do not poison
the air with these sounds of malice. I
have declared my resolution to you,
and I am no coward to bend beneath
words. Leave me; I am inexorable.''

``It is well. I go; but remember, I shall
be with you on your wedding-night.''

I started forward, and exclaimed,
``Villain! before you sign my death-warrant,
be sure that you are yourself
safe.''

I would have seized him; but he
eluded me, and quitted the house with
%%234%%
precipitation: in a few moments I saw
him in his boat, which shot across the
waters with an arrowy swiftness, and
was soon lost amidst the waves.

All was again silent; but his words
rung in my ears. I burned with rage
to pursue the murderer of my peace,
and precipitate him into the ocean. I
walked up and down my room hastily
and perturbed, while my imagination
conjured up a thousand images to torment
and sting me. Why had I not
followed him, and closed with him in
mortal strife? But I had suffered him
to depart, and he had directed his
course towards the main land. I shuddered
to think who might be the next
victim sacrificed to his insatiate revenge.
And then I thought again of
his words --- ``\emph{I will be with you on your
wedding-night}.'' That then was the
period fixed for the fulfilment of my
destiny. In that hour I should die,
and at once satisfy and extinguish his
malice. The prospect did not move
me to fear; yet when I thought of my
beloved Elizabeth, --- of her tears and
endless sorrow, when she should find
her lover so barbarously snatched from
her, --- tears, the first I had shed for
many months, streamed from my eyes,
and I resolved not to fall before my
enemy without a bitter struggle.

The night passed away, and the sun
rose from the ocean; my feelings became
calmer, if it may be called calmness,
when the violence of rage sinks
into the depths of despair. I left the
house, the horrid scene of the last
night's contention, and walked on the
beach of the sea, which I almost
%%235%%
regarded as an insuperable barrier between
me and my fellow-creatures;
nay, a wish that such should prove the
fact stole across me. I desired that I
might pass my life on that barren rock,
wearily it is true, but uninterrupted by
any sudden shock of misery. If I returned,
it was to be sacrificed, or to see
those whom I most loved die under the
grasp of a dæmon whom I had myself
created.

I walked about the isle like a restless
spectre, separated from all it loved,
and miserable in the separation. When
it became noon, and the sun rose
higher, I lay down on the grass, and
was overpowered by a deep sleep. I
had been awake the whole of the preceding
night, my nerves were agitated,
and my eyes inflamed by watching and
misery. The sleep into which I now
sunk refreshed me; and when I awoke,
I again felt as if I belonged to a race
of human beings like myself, and I
began to reflect upon what had passed
with greater composure; yet still the
words of the fiend rung in my ears like
a death-knell, they appeared like a
dream, yet distinct and oppressive as a
reality.

The sun had far descended, and I
still sat on the shore, satisfying my appetite,
which had become rav\-en\-ous,
with an oaten cake, when I saw a fishing-boat
land close to me, and one of the
men brought me a packet; it contained
letters from Geneva, and one from
Clerval, entreating me to join him.
He said that nearly a year had elapsed
since we had quitted Switzerland, and
France was yet unvisited. He
%%236%%
entreated me, therefore, to leave my
solitary isle, and meet him at Perth, in
a week from that time, when we might
arrange the plan of our future proceedings.
This letter in a degree recalled
me to life, and I determined to
quit my island at the expiration of two
days.

Yet, before I departed, there was a
task to perform, on which I shuddered
to reflect: I must pack my chemical
instruments; and for that purpose I
must enter the room which had been
the scene of my odious work, and I
must handle those utensils, the sight of
which was sickening to me. The next
morning, at day-break, I summoned
sufficient courage, and unlocked the
door of my laboratory. The remains
of the half-finished creature, whom I
had destroyed, lay scattered on the
floor, and I almost felt as if I had mangled
the living flesh of a human being.
I paused to collect myself, and then
entered the chamber. With trembling
hand I conveyed the instruments out of
the room; but I reflected that I ought
not to leave the relics of my work to
excite the horror and suspicion of the
peasants, and I accordingly put them
into a basket, with a great quantity of
stones, and laying them up, determined
to throw them into the sea that very
night; and in the mean time I sat upon
the beach, employed in cleaning and
arranging my chemical apparatus.

Nothing could be more complete
than the alteration that had taken place
in my feelings since the night of the
appearance of the dæmon. I had
before regarded my promise with a
%%237%%
gloomy despair, as a thing that, with
whatever consequences, must be fulfilled;
but I now felt as if a film had
been taken from before my eyes, and
that I, for the first time, saw clearly.
The idea of renewing my labours did
not for one instant occur to me; the
threat I had heard weighed on my
thoughts, but I did not reflect that a
voluntary act of mine could avert it.
I had resolved in my own mind, that to
create another like the fiend I had first
made would be an act of the basest and
most atrocious selfishness; and I banished
from my mind every thought
that could lead to a different conclusion.

Between two and three in the morning
the moon rose; and I then, putting
my basket aboard a little skiff, sailed
out about four miles from the shore.
The scene was perfectly solitary: a few
boats were returning towards land, but
I sailed away from them. I felt as if
I was about the commission of a dreadful
crime, and avoided with shuddering
anxiety any encounter with my fellow-creatures.
At one time the moon,
which had before been clear, was suddenly
overspread by a thick cloud, and
I took advantage of the moment of
darkness, and cast my basket into the
sea; I listened to the gurgling sound
as it sunk, and then sailed away from
the spot. The sky became clouded;
but the air was pure, although chilled
by the north-east breeze that was then
rising. But it refreshed me, and filled
me with such agreeable sensations, that
I resolved to prolong my stay on the
water, and fixing the rudder in a direct
position, stretched myself at the bottom
%%238%%
of the boat. Clouds hid the moon,
every thing was obscure, and I heard
only the sound of the boat, as its keel
cut through the waves; the murmur
lulled me, and in a short time I slept
soundly.

I do not know how long I remained
in this situation, but when I awoke I
found that the sun had already mounted
considerably. The wind was high, and
the waves continually threatened the
safety of my little skiff. I found that
the wind was north-east, and must have
driven me far from the coast from
which I had embarked. I endeavoured
to change my course, but quickly found
that if I again made the attempt the
boat would be instantly filled with
water. Thus situated, my only resource
was to drive before the wind. I confess
that I felt a few sensations of terror. I
had no compass with me, and was so
little acquainted with the geography
of this part of the world that the sun
was of little benefit to me. I might
be driven into the wide Atlantic, and
feel all the tortures of starvation, or be
swallowed up in the immeasurable
waters that roared and buffeted around
me. I had already been out many
hours, and felt the torment of a burning
thirst, a prelude to my other sufferings.
I looked on the heavens, which were
covered by clouds that flew before the
wind only to be replaced by others: I
looked upon the sea, it was to be my
grave. ``Fiend,'' I exclaimed, ``your
task is already fulfilled!'' I thought of
Elizabeth, of my father, and of Clerval;
and sunk into a reverie, so despairing
and frightful, that even now, when the
%%239%%
scene is on the point of closing before
me for ever, I shudder to reflect on it.

Some hours passed thus; but by degrees,
as the sun declined towards the
horizon, the wind died away into a
gentle breeze, and the sea became free
from breakers. But these gave place
to a heavy swell; I felt sick, and hardly
able to hold the rudder, when suddenly
I saw a line of high land towards the
south.

Almost spent, as I was, by fatigue,
and the dreadful suspense I endured
for several hours, this sudden certainty
of life rushed like a flood of warm joy
to my heart, and tears gushed from my
eyes.

How mutable are our feelings, and
how strange is that clinging love we
have of life even in the excess of misery!
I constructed another sail with
a part of my dress, and eagerly steered
my course towards the land. It had a
wild and rocky appearance; but as I
approached nearer, I easily perceived
the traces of cultivation. I saw vessels
near the shore, and found myself
suddenly transported back to the neighbourhood
of civilized man. I eagerly
traced the windings of the land, and
hailed a steeple which I at length saw
issuing from behind a small promontory.
As I was in a state of extreme
debility, I resolved to sail directly towards
the town as a place where I
could most easily procure nourishment.
Fortunately I had money with me. As
I turned the promontory, I perceived a
small neat town and a good harbour,
which I entered, my heart bounding
with joy at my unexpected escape.
%%240%%

As I was occupied in fixing the boat
and arranging the sails, several people
crowded towards the spot. They seem\-ed
very much surprised at my appearance;
but, instead of offering me any assistance,
whispered together with gestures
that at any other time might have produced
in me a slight sensation of alarm.
As it was, I merely remarked that they
spoke English; and I therefore addressed
them in that language: ``My
good friends,'' said I, ``will you be so
kind as to tell me the name of this
town, and inform me where I am?''

``You will know that soon enough,''
replied a man with a gruff voice.
``May be you are come to a place that
will not prove much to your taste; but
you will not be consulted as to your
quarters, I promise you.''

I was exceedingly surprised on receiving
so rude an answer from a
stranger; and I was also disconcerted
on perceiving the frowning and angry
countenances of his companions. ``Why
do you answer me so roughly?'' I replied:
``surely it is not the custom of
Englishmen to receive strangers so inhospitably.''

``I do not know,'' said the man,
``what the custom of the English may
be; but it is the custom of the Irish to
hate villains.''

While this strange dialogue continued,
I perceived the crowd rapidly
increase. Their faces expressed a mixture
of curiosity and anger, which annoyed,
and in some degree alarmed
me. I inquired the way to the inn;
but no one replied. I then moved forward,
and a murmuring sound arose
%%241%%
from the crowd as they followed and
surrounded me; when an ill-looking
man approaching, tapped me on the
shoulder, and said, ``Come, Sir, you
must follow me to Mr.~Kirwin's, to give
an account of yourself.''

``Who is Mr.~Kirwin? Why am I
to give an account of myself? Is not
this a free country?''

``Aye, Sir, free enough for honest
folks. Mr.~Kirwin is a magistrate;
and you are to give an account of the
death of a gentleman who was found
murdered here last night.''

This answer startled me; but I presently
recovered myself. I was innocent;
that could easily be proved: accordingly
I followed my conductor in
silence, and was led to one of the best
houses in the town. I was ready to
sink from fatigue and hunger; but,
being surrounded by a crowd, I thought
it politic to rouse all my strength, that
no physical debility might be construed
into apprehension or conscious guilt.
Little did I then expect the calamity
that was in a few moments to overwhelm
me, and extinguish in horror
and despair all fear of ignominy or
death.

I must pause here; for it requires all
my fortitude to recall the memory of
the frightful events which I am about
to relate, in proper detail, to my recollection.
%%242%%

\namedchapter{Chapter IV}

\textsc{I was} soon introduced into the presence
of the magistrate, an old benevolent
man, with calm and mild manners.
He looked upon me, however, with
some degree of severity; and then,
turning towards my conductors, he
asked who appeared as witnesses on
this occasion.

About half a dozen men came forward;
and one being selected by the
magistrate, he deposed, that he had
been out fishing the night before with
his son and brother-in-law, Daniel Nugent,
when, about ten o'clock, they observed
a strong northerly blast rising,
and they accordingly put in for port.
It was a very dark night, as the moon
had not yet risen; they did not land at
the harbour, but, as they had been accustomed,
at a creek about two miles
below. He walked on first, carrying
a part of the fishing tackle, and his
companions followed him at some distance.
As he was proceeding along the
sands, he struck his foot against something,
and fell all his length on the
ground. His companions came up to
assist him; and, by the light of their
lantern, they found that he had fallen
on the body of a man, who was to all
appearance dead. Their first supposition
was, that it was the corpse of
some person who had been drowned,
and was thrown on shore by the waves;
%%243%%
but, upon examination, they found that
the clothes were not wet, and even that
the body was not then cold. They instantly
carried it to the cottage of an
old woman near the spot, and endeavoured,
but in vain, to restore it to life.
He appeared to be a handsome young
man, about five and twenty years of age.
He had apparently been strangled; for
there was no sign of any violence, except
the black mark of fingers on his
neck.

The first part of this deposition did
not in the least interest me; but when
the mark of the fingers was mentioned,
I remembered the murder of my brother,
and felt myself extremely agitated;
my limbs trembled, and a mist came
over my eyes, which obliged me to lean
on a chair for support. The magistrate
observed me with a keen eye,
and of course drew an unfavourable
augury from my manner.

The son confirmed his father's account:
but when Daniel Nugent was
called, he swore positively that, just
before the fall of his companion, he
saw a boat, with a single man in it, at
a short distance from the shore; and,
as far as he could judge by the light of
a few stars, it was the same boat in
which I had just landed.

A woman deposed, that she lived
near the beach, and was standing at
the door of her cottage, waiting for the
return of the fishermen, about an hour
before she heard of the discovery of the
body, when she saw a boat, with only
one man in it, push off from that
part of the shore where the corpse was
afterwards found.

Another woman confirmed the
%%244%%
account of the fishermen having brought
the body into her house; it was not
cold. They put it into a bed, and
rubbed it; and Daniel went to the
town for an apothecary, but life was
quite gone.

Several other men were examined
concerning my landing; and they
agreed, that, with the strong north
wind that had arisen during the night,
it was very probable that I had beaten
about for many hours, and had been
obliged to return nearly to the same
spot from which I had departed. Besides,
they observed that it appeared
that I had brought the body from another
place, and it was likely, that as
I did not appear to know the shore, I
might have put into the harbour ignorant
of the distance of the town of  ------  from
the place where I had deposited
the corpse.

Mr.~Kirwin, on hearing this evidence,
desired that I should be taken
into the room where the body lay for
interment that it might be observed
what effect the sight of it would produce
upon me. This idea was probably
suggested by the extreme agitation
I had exhibited when the mode of the
murder had been described. I was accordingly
conducted, by the magistrate
and several other persons, to the inn.
I could not help being struck by the
strange coincidences that had taken
place during this eventful night; but,
knowing that I had been conversing
with several persons in the island I
had inhabited about the time that the
body had been found, I was perfectly
tranquil as to the consequences of the
affair.
%%245%%

I entered the room where the corpse
lay, and was led up to the coffin. How
can I describe my sensations on beholding
it? I feel yet parched with
horror, nor can I reflect on that terrible
moment without shuddering and agony,
that faintly reminds me of the anguish
of the recognition. The trial, the
presence of the magistrate and witnesses,
passed like a dream from my
memory, when I saw the lifeless form
of Henry Clerval stretched before me.
I gasped for breath; and, throwing
myself on the body, I exclaimed,
``Have my murderous machinations
deprived you also, my dearest Henry,
of life? Two I have already destroyed;
other victims await their destiny:
but you, Clerval, my friend, my benefactor'' ------

The human frame could no longer
support the agonizing suffering that I
endured, and I was carried out of the
room in strong convulsions.

A fever succeeded to this. I lay for
two months on the point of death:
my ravings, as I afterwards heard, were
frightful; I called myself the murderer
of William, of Justine, and of Clerval.
Sometimes I entreated my attendants
to assist me in the destruction of the
fiend by whom I was tormented; and,
%%246%%
at others, I felt the fingers of the monster
already grasping my neck, and
screamed aloud with agony and terror.
Fortunately, as I spoke my native language,
Mr.~Kirwin alone understood
me; but my gestures and bitter cries
were sufficient to affright the other
witnesses.

Why did I not die? More miserable
than man ever was before, why did
I not sink into forgetfulness and rest?
Death snatches away many blooming
children, the only hopes of their doating
parents: how many brides and
youthful lovers have been one day in
the bloom of health and hope, and the
next a prey for worms and the decay
of the tomb! Of what materials was
I made, that I could thus resist so many
shocks, which, like the turning of the
wheel, continually renewed the torture.

But I was doomed to live; and, in
two months, found myself as awaking
from a dream, in a prison, stretched on
a wretched bed, surrounded by gaolers,
turnkeys, bolts, and all the miserable apparatus
of a dungeon. It was morning,
I remember, when I thus awoke to understanding:
I had forgotten the particulars
of what had happened, and
only felt as if some great misfortune
had suddenly overwhelmed me; but
when I looked around, and saw the
barred windows, and the squalidness
of the room in which I was, all flashed
across my memory, and I groaned bitterly.

This sound disturbed an old woman
who was sleeping in a chair beside me.
She was a hired nurse, the wife of one
%%247%%
of the turnkeys, and her countenance
expressed all those bad qualities which
often characterize that class. The lines
of her face were hard and rude, like
that of persons accustomed to see without
sympathizing in sights of misery.
Her tone expressed her entire indifference;
she addressed me in English,
and the voice struck me as one that I
had heard during my sufferings:

``Are you better now, Sir?'' said
she.

I replied in the same language, with
a feeble voice, ``I believe I am; but
if it be all true, if indeed I did not
dream, I am sorry that I am still alive
to feel this misery and horror.''

``For that matter,'' replied the old
woman, ``if you mean about the gentleman
you murdered, I believe that
it were better for you if you were dead,
for I fancy it will go hard with you;
but you will be hung when the next
sessions come on. However, that's
none of my business, I am sent to nurse
you, and get you well; I do my duty
with a safe conscience, it were well if
every body did the same.''

I turned with loathing from the woman
who could utter so unfeeling a
speech to a person just saved, on the
very edge of death; but I felt languid,
and unable to reflect on all that had
passed. The whole series of my life
appeared to me as a dream; I sometimes
doubted if indeed it were all true,
for it never presented itself to my mind
with the force of reality.

As the images that floated before me
became more distinct, I grew feverish;
a darkness pressed around me; no one
%%248%%
was near me who soothed me with the
gentle voice of love; no dear hand supported
me. The physician came and
prescribed medicines, and the old woman
prepared them for me; but utter
carelessness was visible in the first, and
the expression of brutality was strongly
marked in the visage of the second.
Who could be interested in the fate of
a murderer, but the hangman who
would gain his fee?

These were my first reflections; but
I soon learned that Mr.~Kirwin had
shewn me extreme kindness. He had
caused the best room in the prison to
be prepared for me (wretched indeed
was the best); and it was he who had
provided a physician and a nurse. It
is true, he seldom came to see me; for,
although he ardently desired to relieve
the sufferings of every human creature,
he did not wish to be present at the
agonies and miserable ravings of a
murderer. He came, therefore, sometimes
to see that I was not neglected;
but his visits were short, and at long
intervals.

One day, when I was gradually recovering,
I was seated in a chair, my
eyes half open, and my cheeks livid
like those in death, I was overcome by
gloom and misery, and often reflected
I had better seek death than remain
miserably pent up only to be let loose
in a world replete with wretchedness.
At one time I considered whether I
should not declare myself guilty, and
suffer the penalty of the law, less innocent
than poor Justine had been. Such
were my thoughts, when the door of
my apartment was opened, and Mr.~%
%%249%%
Kirwin entered. His countenance expressed
sympathy and compassion; he
drew a chair close to mine, and addressed
me in French~---

``I fear that this place is very shocking
to you; can I do any thing to make
you more comfortable?''

``I thank you; but all that you
mention is nothing to me: on the whole
earth there is no comfort which I am
capable of receiving.''

``I know that the sympathy of a
stranger can be but of little relief to
one borne down as you are by so
strange a misfortune. But you will, I
hope, soon quit this melancholy abode;
for, doubtless, evidence can easily be
brought to free you from the criminal
charge.''

``That is my least concern: I am,
by a course of strange events, become
the most miserable of mortals. Persecuted
and tortured as I am and have
been, can death be any evil to me?''

``Nothing indeed could be more
unfortunate and agonizing than the
strange chances that have lately occurred.
You were thrown, by some surprising
accident, on this shore, renowned
for its hospitality: seized immediately,
and charged with murder.
The first sight that was presented to
your eyes was the body of your friend,
murdered in so unaccountable a manner,
and placed, as it were, by some
fiend across your path.''

As Mr.~Kirwin said this, notwithstanding
the agitation I endured on
this retrospect of my sufferings, I also
felt considerable surprise at the knowledge
he seemed to possess concerning
%%250%%
me. I suppose some astonishment was
exhibited in my countenance; for Mr.~%
Kirwin hastened to say~---

``It was not until a day or two after
your illness that I thought of examining
your dress, that I might discover some
trace by which I could send to your relations
an account of your misfortune
and illness. I found several letters, and,
among others, one which I discovered
from its commencement to be from your
father. I instantly wrote to Geneva:
nearly two months have elapsed since
the departure of my letter. --- But you
are ill; even now you tremble: you
are unfit for agitation of any kind.''

``This suspense is a thousand times
worse than the most horrible event:
tell me what new scene of death has
been acted, and whose murder I am
now to lament.''

``Your family is perfectly well,''
said Mr.~Kirwin, with gentleness; ``and
some one, a friend, is come to visit
you.''

I know not by what chain of thought
the idea presented itself, but it instantly
darted into my mind that the murderer
had come to mock at my misery, and
taunt me with the death of Clerval, as
a new incitement for me to comply with
his hellish desires. I put my hand before
my eyes, and cried out in agony~---

``Oh! take him away! I cannot see
him; for God's sake, do not let him
enter!''

Mr.~Kirwin regarded me with a
troubled countenance. He could not
help regarding my exclamation as a
presumption of my guilt, and said, in
rather a severe tone~---
%%251%%

``I should have thought, young man,
that the presence of your father would
have been welcome, instead of inspiring
such violent repugnance.''

``My father!'' cried I, while every
feature and every muscle was relaxed
from anguish to pleasure. ``Is my
father, indeed, come? How kind,
how very kind. But where is he, why
does he not hasten to me?''

My change of manner surprised and
pleased the magistrate; perhaps he
thought that my former exclamation
was a momentary return of delirium,
and now he instantly resumed his
former benevolence. He rose, and
quitted the room with my nurse, and
in a moment my father entered it.

Nothing, at this moment, could have
given me greater pleasure than the arrival
of my father. I stretched out my
hand to him, and cried~---

``Are you then safe --- and Elizabeth --- and
Ernest?''

My father calmed me with assurances
of their welfare, and endeavoured, by
dwelling on these subjects so interesting
to my heart, to raise my desponding
spirits; but he soon felt that a prison
cannot be the abode of cheerfulness.
``What a place is this that you inhabit,
my son!'' said he, looking mournfully
at the barred windows, and wretched
appearance of the room. ``You travelled
to seek happiness, but a fatality
seems to pursue you. And poor Clerval --- ''

The name of my unfortunate and
murdered friend was an agitation too
great to be endured in my weak state;
I shed tears.
%%252%%

``Alas! yes, my father,'' replied I;
``some destiny of the most horrible
kind hangs over me, and I must live to
fulfil it, or surely I should have died on
the coffin of Henry.''

We were not allowed to converse for
any length of time, for the precarious
state of my health rendered every precaution
necessary that could insure
tranquillity. Mr.~Kirwin came in, and
insisted that my strength should not be
exhausted by too much exertion. But
the appearance of my father was to me
like that of my good angel, and I gradually
recovered my health.

As my sickness quitted me, I was absorbed
by a gloomy and black melancholy,
that nothing could dissipate.
The image of Clerval was for ever
before me, ghastly and murdered.
More than once the agitation into
which these reflections threw me made
my friends dread a dangerous relapse.
Alas! why did they preserve so miserable
and detested a life? It was surely
that I might fulfil my destiny, which is
now drawing to a close. Soon, oh, very
soon, will death extinguish these throbbings,
and relieve me from the mighty
weight of anguish that bears me to the
dust; and, in executing the award of
justice, I shall also sink to rest. Then
the appearance of death was distant,
although the wish was ever present to
my thoughts; and I often sat for hours
motionless and speechless, wishing for
some mighty revolution that might
bury me and my destroyer in its
ruins.

The season of the assizes approached.
I had already been three months in
%%253%%
prison; and although I was still weak,
and in continual danger of a relapse, I
was obliged to travel nearly a hundred
miles to the county-town, where the
court was held. Mr.~Kirwin charged
himself with every care of collecting
witnesses, and arranging my defence.
I was spared the disgrace of appearing
publicly as a criminal, as the case was
not brought before the court that decides
on life and death. The grand
jury rejected the bill, on its being
proved that I was on the Orkney Islands
at the hour the body of my friend was
found, and a fortnight after my removal
I was liberated from prison.

My father was enraptured on finding
me freed from the vexations of a criminal
charge, that I was again allowed
to breathe the fresh atmosphere, and
allowed to return to my native country.
I did not participate in these feelings;
for to me the walls of a dungeon or a
palace were alike hateful. The cup of
life was poisoned for ever; and although
the sun shone upon me, as upon
the happy and gay of heart, I saw
around me nothing but a dense and
frightful darkness, penetrated by no
light but the glimmer of two eyes that
glared upon me. Sometimes they were
the expressive eyes of Henry, languishing
in death, the dark orbs nearly
covered by the lids, and the long black
lashes that fringed them; sometimes it
was the watery clouded eyes of the monster,
as I first saw them in my chamber
at Ingolstadt.

My father tried to awaken in me the
feelings of affection. He talked of
Geneva, which I should soon visit --- of
%%254%%
Elizabeth, and Ernest; but these words
only drew deep groans from me. Sometimes,
indeed, I felt a wish for happiness;
and thought, with melancholy delight,
of my beloved cousin; or longed,
with a devouring \emph{maladie du pays}, to
see once more the blue lake and rapid
Rhone, that had been so dear to me in
early childhood: but my general state
of feeling was a torpor, in which a
prison was as welcome a residence as
the divinest scene in nature; and these
fits were seldom interrupted, but by
paroxysms of anguish and despair. At
these moments I often endeavoured to
put an end to the existence I loathed;
and it required unceasing attendance
and vigilance to restrain me from
committing some dreadful act of violence.

I remember, as I quitted the prison,
I heard one of the men say, ``He may
be innocent of the murder, but he has
certainly a bad conscience.'' These
words struck me. A bad conscience!
yes, surely I had one. William, Justine,
and Clerval, had died through
my infernal machinations; ``And whose
death,'' cried I, ``is to finish the tragedy?
Ah! my father, do not remain
in this wretched country; take me
where I may forget myself, my existence,
and all the world.''

My father easily acceded to my desire;
and, after having taken leave of
Mr.~Kirwin, we hastened to Dublin. I
felt as if I was relieved from a heavy
weight, when the packet sailed with a
fair wind from Ireland, and I had
quitted for ever the country which had
been to me the scene of so much
misery.
%%255%%

It was midnight. My father slept in
the cabin; and I lay on the deck, looking
at the stars, and listening to the
dashing of the waves. I hailed the
darkness that shut Ireland from my
sight, and my pulse beat with a feverish
joy, when I reflected that I should soon
see Geneva. The past appeared to
me in the light of a frightful dream;
yet the vessel in which I was, the wind
that blew me from the detested shore of
Ireland, and the sea which surrounded
me, told me too forcibly that I was deceived
by no vision, and that Clerval,
my friend and dearest companion, had
fallen a victim to me and the monster
of my creation. I repassed, in my
memory, my whole life; my quiet happiness
while residing with my family in
Geneva, the death of my mother, and
my departure for Ingolstadt. I remembered
shuddering at the mad enthusiasm
that hurried me on to the creation
of my hideous enemy, and I called
to mind the night during which he first
lived. I was unable to pursue the train
of thought; a thousand feelings pressed
upon me, and I wept bitterly.

Ever since my recovery from the fever
I had been in the custom of taking every
night a small quantity of laudanum;
for it was by means of this drug only
that I was enabled to gain the rest
necessary for the preservation of life.
Oppressed by the recollection of my
various misfortunes, I now took a
double dose, and soon slept profoundly.
But sleep did not afford me respite
from thought and misery; my dreams
presented a thousand objects that scared
me. Towards morning I was possessed
by a kind of night-mare; I felt the
%%256%%
fiend's grasp in my neck, and could not
free myself from it; groans and cries
rung in my ears. My father, who was
watching over me, perceiving my restlessness,
awoke me, and pointed to the
port of Holyhead, which we were now
entering.
%%257%%

\namedchapter{Chapter V}

\textsc{We} had resolved not to go to London,
but to cross the country to Portsmouth,
and thence to embark for Havre. I
preferred this plan principally because
I dreaded to see again those places in
which I had enjoyed a few moments of
tranquillity with my beloved Clerval.
I thought with horror of seeing again
those persons whom we had been accustomed
to visit together, and who
might make inquiries concerning an
event, the very remembrance of which
made me again feel the pang I endured
when I gazed on his lifeless form in
the inn at ------.

As for my father, his desires and exertions
were bounded to the again seeing
me restored to health and peace of
mind. His tenderness and attentions
were unremitting; my grief and gloom
was obstinate, but he would not despair.
Sometimes he thought that I felt deeply
the degradation of being obliged to
answer a charge of murder, and he endeavoured
to prove to me the futility of
pride.

``Alas! my father,'' said I, ``how
little do you know me. Human beings,
their feelings and passions, would indeed
be degraded, if such a wretch as
I felt pride. Justine, poor unhappy
Justine, was as innocent as I, and she
suffered the same charge; she died
%%258%%
for it; and I am the cause of this --- I
murdered her. William, Justine,
and Henry --- they all died by my
hands.''

My father had often, during my imprisonment,
heard me make the same
assertion; when I thus accused myself,
he sometimes seemed to desire an explanation,
and at others he appeared to
consider it as caused by delirium, and
that, during my illness, some idea of
this kind had presented itself to my
imagination, the remembrance of which
I preserved in my convalescence. I
avoided explanation, and maintained
a continual silence concerning the
wretch I had created. I had a feeling
that I should be supposed mad, and this
for ever chained my tongue, when I
would have given the whole world to
have confided the fatal secret.

Upon this occasion my father said,
with an expression of unbounded
wonder, ``What do you mean, Victor?
are you mad? My dear son, I entreat
you never to make such an assertion
again.''

``I am not mad,'' I cried energetically;
``the sun and the heavens, who
have viewed my operations, can bear
witness of my truth. I am the assassin
of those most innocent victims; they
died by my machinations. A thousand
times would I have shed my own blood,
drop by drop, to have saved their lives;
but I could not, my father, indeed I
could not sacrifice the whole human
race.''

The conclusion of this speech convinced
my father that my ideas were
deranged, and he instantly changed the
%%259%%
subject of our conversation, and endeavoured
to alter the course of my
thoughts. He wished as much as possible
to obliterate the memory of the
scenes that had taken place in Ireland,
and never alluded to them, or suffered
me to speak of my misfortunes.

As time passed away I became more
calm: misery had her dwelling in my
heart, but I no longer talked in the
same incoherent manner of my own
crimes; sufficient for me was the consciousness
of them. By the utmost
self-violence, I curbed the imperious
voice of wretchedness, which sometimes
desired to declare itself to the whole
world; and my manners were calmer
and more composed than they had ever
been since my journey to the sea of
ice.

We arrived at Havre on the 8th of
May, and instantly proceeded to Paris,
where my father had some business
which detained us a few weeks. In
this city, I received the following letter
from Elizabeth:---

\bigskip
\noindent ``\emph{To} \textsc{Victor Frankenstein}.
\medskip

\noindent ``\textsc{my dearest friend},
\medskip

``It gave me the greatest pleasure to
receive a letter from my uncle dated at
Paris; you are no longer at a formidable
distance, and I may hope to see
you in less than a fortnight. My poor
cousin, how much you must have suffered!
I expect to see you looking even
more ill than when you quitted Geneva.
This winter has been passed most
miserably, tortured as I have been
by anxious suspense; yet I hope to
see peace in your countenance, and
%%260%%
to find that your heart is not
totally devoid of comfort and tranquillity.

``Yet I fear that the same feelings
now exist that made you so miserable a
year ago, even perhaps augmented by
time. I would not disturb you at this
period, when so many misfortunes weigh
upon you; but a conversation that I had
with my uncle previous to his departure
renders some explanation necessary
before we meet.

``Explanation! you may possibly
say; what can Elizabeth have to explain?
If you really say this, my questions
are answered, and I have no more
to do than to sign myself your affectionate
cousin. But you are distant
from me, and it is possible that you
may dread, and yet be pleased with
this explanation; and, in a probability
of this being the case, I dare not any
longer postpone writing what, during
your absence, I have often wished to
express to you, but have never had the
courage to begin.

``You well know, Victor, that our
union had been the favourite plan of
your parents ever since our infancy.
We were told this when young, and
taught to look forward to it as an event
that would certainly take place. We
were affectionate playfellows during
childhood, and, I believe, dear and
valued friends to one another as we
grew older. But as brother and sister
often entertain a lively affection towards
each other, without desiring a
more intimate union, may not such also
be our case? Tell me, dearest Victor.
Answer me, I conjure you, by our
%%261%%
mutual happiness, with simple truth --- Do
you not love another?

``You have travelled; you have
spent several years of your life at Ingolstadt;
and I confess to you, my
friend, that when I saw you last autumn
so unhappy, flying to solitude,
from the society of every creature, I
could not help supposing that you
might regret our connexion, and believe
yourself bound in honour to fulfil
the wishes of your parents, although
they opposed themselves to your inclinations.
But this is false reasoning.
I confess to you, my cousin, that I love
you, and that in my airy dreams of
futurity you have been my constant
friend and companion. But it is your
happiness I desire as well as my own,
when I declare to you, that our marriage
would render me eternally miserable,
unless it were the dictate of your own
free choice. Even now I weep to think,
that, borne down as you are by the
cruelest misfortunes, you may stifle, by
the word \emph{honour}, all hope of that love
and happiness which would alone restore
you to yourself. I, who have so
interested an affection for you, may increase
your miseries ten-fold, by being
an obstacle to your wishes. Ah, Victor,
be assured that your cousin and
playmate has too sincere a love for you
not to be made miserable by this supposition.
Be happy, my friend; and if
you obey me in this one request, remain
satisfied that nothing on earth will
have the power to interrupt my tranquillity.

``Do not let this letter disturb you;
do not answer it to-morrow, or the
%%262%%
next day, or even until you come, if it
will give you pain. My uncle will send
me news of your health; and if I see
but one smile on your lips when we
meet, occasioned by this or any other
exertion of mine, I shall need no other
happiness.

\frLetterSig{``\textsc{Elizabeth Lavenza}.}

\frDate{``Geneva, May 18th, 17--- .''}

This letter revived in my memory
what I had before forgotten, the threat
of the fiend --- ``\emph{I will be with you on
your wedding-night!}'' Such was my
sentence, and on that night would the
dæmon employ every art to destroy me,
and tear me from the glimpse of happiness
which promised partly to console
my sufferings. On that night he had
determined to consummate his crimes
by my death. Well, be it so; a deadly
struggle would then assuredly take
place, in which if he was victorious, I
should be at peace, and his power over
me be at an end. If he were vanquished,
I should be a free man. Alas!
what freedom? such as the peasant
enjoys when his family have been massacred
before his eyes, his cottage
burnt, his lands laid waste, and he is
turned adrift, homeless, pennyless, and
alone, but free. Such would be my
liberty, except that in my Elizabeth I
possessed a treasure; alas! balanced
by those horrors of remorse and guilt,
which would pursue me until death.

Sweet and beloved Elizabeth! I
read and re-read her letter, and some
softened feelings stole into my heart,
and dared to whisper paradisaical
dreams of love and joy; but the apple
%%263%%
was already eaten, and the angel's arm
bared to drive me from all hope. Yet
I would die to make her happy. If the
monster executed his threat, death was
inevitable; yet, again, I considered
whether my marriage would hasten my
fate. My destruction might indeed
arrive a few months sooner; but if my
torturer should suspect that I postponed
it, influenced by his menaces,
he would surely find other, and perhaps
more dreadful means of revenge.
He had vowed \emph{to be with me on my
wedding-night}, yet he did not consider
that threat as binding him to peace in
the mean time; for, as if to shew me
that he was not yet satiated with blood,
he had murdered Clerval immediately
after the enunciation of his threats. I
resolved, therefore, that if my immediate
union with my cousin would conduce
either to her's or my father's happiness,
my adversary's designs against
my life should not retard it a single
hour.

In this state of mind I wrote to Elizabeth.
My letter was calm and affectionate.
``I fear, my beloved girl,'' I
said, ``little happiness remains for us
on earth; yet all that I may one day
enjoy is concentered in you. Chase
away your idle fears; to you alone do I
consecrate my life, and my endeavours
for contentment. I have one secret,
Elizabeth, a dreadful one; when revealed
to you, it will chill your frame
with horror, and then, far from being
surprised at my misery, you will only
wonder that I survive what I have endured.
I will confide this tale of
misery and terror to you the day after
%%264%%
our marriage shall take place; for, my
sweet cousin, there must be perfect confidence
between us. But until then, I
conjure you, do not mention or allude
to it. This I most earnestly entreat,
and I know you will comply.''

In about a week after the arrival of
Elizabeth's letter, we returned to Geneva.
My cousin welcomed me with
warm affection; yet tears were in her
eyes, as she beheld my emaciated frame
and feverish cheeks. I saw a change
in her also. She was thinner, and had
lost much of that heavenly vivacity that
had before charmed me; but her gentleness,
and soft looks of compassion,
made her a more fit companion for one
blasted and miserable as I was.

The tranquillity which I now enjoyed
did not endure. Memory brought
madness with it; and when I thought
on what had passed, a real insanity possessed
me; sometimes I was furious,
and burnt with rage, sometimes low
and despondent. I neither spoke or
looked, but sat motionless, bewildered
by the multitude of miseries that overcame
me.

Elizabeth alone had the power to
draw me from these fits; her gentle
voice would soothe me when transported
by passion, and inspire me with
human feelings when sunk in torpor.
She wept with me, and for me. When
reason returned, she would remonstrate,
and endeavour to inspire me with resignation.
Ah! it is well for the unfortunate
to be resigned, but for the
guilty there is no peace. The agonies
of remorse poison the luxury there is
otherwise sometimes found in indulging
the excess of grief.
%%265%%

Soon after my arrival my father
spoke of my immediate marriage with
my cousin. I remained silent.

``Have you, then, some other attachment?''

``None on earth. I love Elizabeth,
and look forward to our union with delight.
Let the day therefore be fixed;
and on it I will consecrate myself, in
life or death, to the happiness of my
cousin.''

``My dear Victor, do not speak thus.
Heavy misfortunes have befallen us;
but let us only cling closer to what remains,
and transfer our love for those
whom we have lost to those who yet
live. Our circle will be small, but
bound close by the ties of affection and
mutual misfortune. And when time
shall have softened your despair, new
and dear objects of care will be born
to replace those of whom we have been
so cruelly deprived.''

Such were the lessons of my father.
But to me the remembrance of the threat
returned: nor can you wonder, that,
omnipotent as the fiend had yet been
in his deeds of blood, I should almost
regard him as invincible; and that
when he had pronounced the words,
``\emph{I shall be with you on your wedding-night},''
I should regard the threatened
fate as unavoidable. But death was no
evil to me, if the loss of Elizabeth were
balanced with it; and I therefore, with
a contented and even cheerful countenance,
agreed with my father, that if
my cousin would consent, the ceremony
should take place in ten days, and
thus put, as I imagined, the seal to my
fate.

Great God! if for one instant I
%%266%%
had thought what might be the hellish
intention of my fiendish adversary, I
would rather have banished myself for
ever from my native country, and wandered
a friendless outcast over the earth,
than have consented to this miserable
marriage. But, as if possessed of
magic powers, the monster had blinded
me to his real intentions; and when I
thought that I prepared only my own
death, I hastened that of a far dearer
victim.

As the period fixed for our marriage
drew nearer, whether from cowardice
or a prophetic feeling, I felt my heart
sink within me. But I concealed my
feelings by an appearance of hilarity,
that brought smiles and joy to the countenance
of my father, but hardly deceived
the ever-watchful and nicer eye
of Elizabeth. She looked forward to
our union with placid contentment, not
unmingled with a little fear, which past
misfortunes had impressed, that what
now appeared certain and tangible
happiness, might soon dissipate into
an airy dream, and leave no trace but
deep and everlasting regret.

Preparations were made for the event;
congratulatory visits were received;
and all wore a smiling appearance. I
shut up, as well as I could, in my own
heart the anxiety that preyed there, and
entered with seeming earnestness into
the plans of my father, although they
might only serve as the decorations of
my tragedy. A house was purchased
for us near Cologny, by which we
should enjoy the pleasures of the
country, and yet be so near Geneva as
to see my father every day; who would
%%267%%
still reside within the walls, for the
benefit of Ernest, that he might follow
his studies at the schools.

In the mean time I took every
precaution to defend my person, in case
the fiend should openly attack me. I
carried pistols and a dagger constantly
about me, and was ever on the watch
to prevent artifice; and by these means
gained a greater degree of tranquillity.
Indeed, as the period approached, the
threat appeared more as a delusion, not
to be regarded as worthy to disturb my
peace, while the happiness I hoped for
in my marriage wore a greater appearance
of certainty, as the day fixed for
its solemnization drew nearer, and I
heard it continually spoken of as an
occurrence which no accident could
possibly prevent.

Elizabeth seemed happy; my tranquil
dem\-eanour contributed greatly
to calm her mind. But on the day
that was to fulfil my wishes and
my destiny, she was melancholy, and
a presentiment of evil pervaded her;
and perhaps also she thought of the
dreadful secret, which I had promised
to reveal to her the following day. My
father was in the mean time overjoyed,
and, in the bustle of preparation, only
observed in the melancholy of his niece
the diffidence of a bride.

After the ceremony was performed,
a large party assembled at my father's;
but it was agreed that Elizabeth and I
should pass the afternoon and night at
Evian, and return to Cologny the next
morning. As the day was fair, and the
wind favourable, we resolved to go by
water.
%%268%%

Those were the last moments of my
life during which I enjoyed the feeling
of happiness. We passed rapidly along:
the sun was hot, but we were sheltered
from its rays by a kind of canopy, while
we enjoyed the beauty of the scene,
sometimes on one side of the lake,
where we saw Mont Salêve, the pleasant
banks of Montalêgre, and at a distance,
surmounting all, the beautiful
Mont Blânc, and the assemblage of
snowy mountains that in vain endeavour
to emulate her; sometimes coasting
the opposite banks, we saw the
mighty Jura opposing its dark side to
the ambition that would quit its native
country, and an almost insurmountable
barrier to the invader who should wish
to enslave it.

I took the hand of Elizabeth: ``You
are sorrowful, my love. Ah! if you knew
what I have suffered, and what I may
yet endure, you would endeavour to let
me taste the quiet, and freedom from
despair, that this one day at least permits
me to enjoy.''

``Be happy, my dear Victor,'' replied
Elizabeth; ``there is, I hope,
nothing to distress you; and be assured
that if a lively joy is not painted in my
face, my heart is contented. Something
whispers to me not to depend too much
on the prospect that is opened before
us; but I will not listen to such a sinister
voice. Observe how fast we
move along, and how the clouds which
sometimes obscure, and sometimes rise
above the dome of Mont Blânc, render
this scene of beauty still more interesting.
Look also at the innumerable
fish that are swimming in the clear
%%269%%
waters, where we can distinguish every
pebble that lies at the bottom. What a
divine day! how happy and serene all
nature appears!''

Thus Elizabeth endeavoured to divert
her th\-oughts and mine from all reflection
upon melancholy subjects. But
her temper was fluctuating; joy for a
few instants shone in her eyes, but it
continually gave place to distraction
and reverie.

The sun sunk lower in the heavens;
we passed the river Drance, and observed
its path through the chasms of
the higher, and the glens of the lower
hills. The Alps here come closer to
the lake, and we approached the amphitheatre
of mountains which forms
its eastern boundary. The spire of
Evian shone under the woods that
surrounded it, and the range of
mountain above mountain by which it
was overhung.

The wind, which had hitherto carried
us along with amazing rapidity, sunk at
sunset to a light breeze; the soft air
just ruffled the water, and caused a pleasant
motion among the trees as we approached
the shore, from which it
wafted the most delightful scent of
flowers and hay. The sun sunk beneath
the horizon as we landed; and
as I touched the shore, I felt those
cares and fears revive, which soon
were to clasp me, and cling to me
for ever.
%%270%%

\namedchapter{Chapter VI}

\textsc{It} was eight o'clock when we landed;
we walked for a short time on the shore,
enjoying the transitory light, and then
retired to the inn, and contemplated
the lovely scene of waters, woods, and
mountains, obscured in darkness, yet
still displaying their black outlines.

The wind, which had fallen in the
south, now rose with great violence in
the west. The moon had reached her
summit in the heavens, and was beginning
to descend; the clouds swept
across it swifter than the flight of the
vulture, and dimmed her rays, while
the lake reflected the scene of the busy
heavens, rendered still busier by the
restless waves that were beginning to
rise. Suddenly a heavy storm of rain
descended.

I had been calm during the day; but
so soon as night obscured the shapes
of objects, a thousand fears arose in my
mind. I was anxious and watchful,
while my right hand grasped a pistol
which was hidden in my bosom; every
sound terrified me; but I resolved that
I would sell my life dearly, and not relax
the impending conflict until my
own life, or that of my adversary, were
extinguished.

Elizabeth observed my agitation for
some time in timid and fearful silence;
at length she said, ``What is it that
%%271%%
agitates you, my dear Victor? What is
it you fear?''

``Oh! peace, peace, my love,'' replied
I, ``this night, and all will be
safe: but this night is dreadful, very
dreadful.''

I passed an hour in this state of
mind, when suddenly I reflected how
dreadful the combat which I momentarily
expected would be to my wife,
and I earnestly entreated her to retire,
resolving not to join her until I had
obtained some knowledge as to the
situation of my enemy.

She left me, and I continued some
time walking up and down the passages
of the house, and inspecting every corner
that might afford a retreat to my
adversary. But I discovered no trace
of him, and was beginning to conjecture
that some fortunate chance had
intervened to prevent the execution of
his menaces; when suddenly I heard
a shrill and dreadful scream. It came
from the room into which Elizabeth
had retired. As I heard it, the whole
truth rushed into my mind, my arms
dropped, the motion of every muscle
and fibre was suspended; I could feel
the blood trickling in my veins, and
tingling in the extremities of my limbs.
This state lasted but for an instant; the
scream was repeated, and I rushed into
the room.

Great God! why did I not then expire!
Why am I here to relate the
destruction of the best hope, and the
purest creature of earth. She was
there, lifeless and inanimate, thrown
across the bed, her head hanging down,
and her pale and distorted features half
%%272%%
covered by her hair. Every where I
turn I see the same figure --- her bloodless
arms and relaxed form flung by
the murderer on its bridal bier. Could
I behold this, and live? Alas! life is
obstinate, and clings closest where it
is most hated. For a moment only did
I lose recollection; I fainted.

When I recovered, I found myself
surrounded by the people of the inn;
their countenances expressed a breathless
terror: but the horror of others
appeared only as a mockery, a shadow
of the feelings that oppressed me. I
escaped from them to the room where
lay the body of Elizabeth, my love,
my wife, so lately living, so dear, so
worthy. She had been moved from
the posture in which I had first beheld
her; and now, as she lay, her head upon
her arm, and a handkerchief thrown
across her face and neck, I might have
supposed her asleep. I rushed towards
her, and embraced her with ardour; but
the deathly languor and coldness of the
limbs told me, that what I now held in
my arms had ceased to be the Elizabeth
whom I had loved and cherished. The
murderous mark of the fiend's grasp
was on her neck, and the breath had
ceased to issue from her lips.

While I still hung over her in the
agony of despair, I happened to look
up. The windows of the room had
before been darkened; and I felt a
kind of panic on seeing the pale yellow
light of the moon illuminate the chamber.
The shutters had been thrown
back; and, with a sensation of horror
not to be described, I saw at the open
window a figure the most hideous and
%%273%%
abhorred. A grin was on the face of the
monster; he seemed to jeer, as with his
fiendish finger he pointed towards the
corpse of my wife. I rushed towards
the window, and drawing a pistol from
my bosom, shot; but he eluded me,
leaped from his station, and, running
with the swiftness of lightning, plunged
into the lake.

The report of the pistol brought a
crowd into the room. I pointed to the
spot where he had disappeared, and we
followed the track with boats; nets were
cast, but in vain. After passing several
hours, we returned hopeless, most
of my companions believing it to have
been a form conjured by my fancy.
After having landed, they proceeded to
search the country, parties going in
different directions among the woods
and vines.

I did not accompany them; I was
exhausted: a film covered my eyes,
and my skin was parched with the heat
of fever. In this state I lay on a bed,
hardly conscious of what had happened;
my eyes wandered round the
room, as if to seek something that I
had lost.

At length I remembered that my father
would anxiously expect the return
of Elizabeth and myself, and that I
must return alone. This reflection
brought tears into my eyes, and I wept
for a long time; but my thoughts rambled
to various subjects, reflecting on
my misfortunes, and their cause. I
was bewildered in a cloud of wonder
and horror. The death of William,
the execution of Justine, the murder
of Clerval, and lastly of my wife; even
%%274%%
at that moment I knew not that my
only remaining friends were safe from
the malignity of the fiend; my father
even now might be writhing under his
grasp, and Ernest might be dead at his
feet. This idea made me shudder, and
recalled me to action. I started up,
and resolved to return to Geneva with
all possible speed.

There were no horses to be procured,
and I must return by the lake;
but the wind was unfavourable, and
the rain fell in torrents. However, it
was hardly morning, and I might reasonably
hope to arrive by night. I
hired men to row, and took an oar
myself, for I had always experienced
relief from mental torment in bodily
exercise. But the overflowing misery
I now felt, and the excess of agitation
that I endured, rendered me incapable
of any exertion. I threw down the oar;
and, leaning my head upon my hands,
gave way to every gloomy idea that
arose. If I looked up, I saw the scenes
which were familiar to me in my happier
time, and which I had contemplated
but the day before in the company
of her who was now but a shadow
and a recollection. Tears streamed
from my eyes. The rain had ceased for
a moment, and I saw the fish play in
the waters as they had done a few hours
before; they had then been observed
by Elizabeth. Nothing is so painful
to the human mind as a great and sudden
change. The sun might shine, or
the clouds might lour; but nothing
could appear to me as it had done the
day before. A fiend had snatched from
me every hope of future happiness: no
%%275%%
creature had ever been so miserable as
I was; so frightful an event is single
in the history of man.

But why should I dwell upon the incidents
that followed this last overwhelming
event. Mine has been a
tale of horrors; I have reached their
\emph{acme}, and what I must now relate can
but be tedious to you. Know that,
one by one, my friends were snatched
away; I was left desolate. My own
strength is exhausted; and I must tell,
in a few words, what remains of my
hideous narration.

I arrived at Geneva. My father and
Ernest yet lived; but the former sunk
under the tidings that I bore. I see
him now, excellent and venerable old
man! his eyes wandered in vacancy, for
they had lost their charm and their delight --- his
niece, his more than daughter,
whom he doated on with all that affection
which a man feels, who, in the decline
of life, having few affections, clings
more earnestly to those that remain.
Cursed, cursed be the fiend that brought
misery on his grey hairs, and doomed
him to waste in wretchedness! He
could not live under the horrors that
were accumulated around him; an
apoplectic fit was brought on, and in
a few days he died in my arms.

What then became of me? I know
not; I lost sensation, and chains and
darkness were the only objects that
pressed upon me. Sometimes, indeed,
I dreamt that I wandered in flowery
meadows and pleasant vales with the
friends of my youth; but awoke, and
found myself in a dungeon. Melancholy
followed, but by degrees I gained
%%276%%
a clear conception of my miseries and
situation, and was then released from
my prison. For they had called me
mad; and during many months, as I
understood, a solitary cell had been my
habitation.

But liberty had been a useless gift
to me had I not, as I awakened to reason,
at the same time awakened to revenge.
As the memory of past misfortunes
pressed upon me, I began to
reflect on their cause --- the monster
whom I had created, the miserable dæmon
whom I had sent abroad into the
world for my destruction. I was possessed
by a maddening rage when I
thought of him, and desired and ardently
prayed that I might have him
within my grasp to wreak a great and
signal revenge on his cursed head.

Nor did my hate long confine itself
to useless wishes; I began to reflect on
the best means of securing him; and
for this purpose, about a month after
my release, I repaired to a criminal
judge in the town, and told him that I
had an accusation to make; that I knew
the destroyer of my family; and that
I required him to exert his whole authority
for the apprehension of the
murderer.

The magistrate listened to me with
attention and kindness: ``Be assured,
sir,'' said he, ``no pains or exertions
on my part shall be spared to discover
the villain.''

``I thank you,'' replied I; ``listen,
therefore, to the deposition that I have
to make. It is indeed a tale so strange,
that I should fear you would not credit
it, were there not something in truth
%%277%%
which, however wonderful, forces conviction.
The story is too connected to
be mistaken for a dream, and I have no
motive for falsehood.'' My manner, as
I thus addressed him, was impressive,
but calm; I had formed in my own
heart a resolution to pursue my destroyer
to death; and this purpose
quieted my agony, and provisionally
reconciled me to life. I now related
my history briefly, but with firmness
and precision, marking the dates with
accuracy, and never deviating into invective
or exclamation.

The magistrate appeared at first perfectly
incredulous, but as I continued
he became more attentive and interested;
I saw him sometimes shudder
with horror, at others a lively surprise,
unmingled with disbelief, was painted
on his countenance.

When I had concluded my narration,
I said. ``This is the being whom I accuse,
and for whose detection and punishment
I call upon you to exert your
whole power. It is your duty as a magistrate,
and I believe and hope that
your feelings as a man will not revolt
from the execution of those functions
on this occasion.''

This address caused a considerable
change in the physiognomy of my
auditor. He had heard my story with
that half kind of belief that is given to
a tale of spirits and supernatural events;
but when he was called upon to act
officially in consequence, the whole
tide of his incredulity returned. He,
however, answered mildly, ``I would
willingly afford you every aid in your
pursuit; but the creature of whom you
%%278%%
speak appears to have powers which
would put all my exertions to defiance.
Who can follow an animal which can
traverse the sea of ice, and inhabit
caves and dens, where no man would
venture to intrude? Besides, some
months have elapsed since the commission
of his crimes, and no one can
conjecture to what place he has wandered,
or what region he may now
inhabit.''

``I do not doubt that he hovers
near the spot which I inhabit; and if
he has indeed taken refuge in the
Alps, he may be hunted like the
chamois, and destroyed as a beast of
prey. But I perceive your thoughts:
you do not credit my narrative, and
do not intend to pursue my enemy
with the punishment which is his
desert.''

As I spoke, rage sparkled in my
eyes; the magistrate was intimidated;
``You are mistaken,'' said he, ``I will
exert myself; and if it is in my power
to seize the monster, be assured that
he shall suffer punishment proportionate
to his crimes. But I fear, from
what you have yourself described to
be his properties, that this will prove
impracticable, and that, while every
proper measure is pursued, you should
endeavour to make up your mind to
disappointment.''

``That cannot be; but all that I
can say will be of little avail. My revenge
is of no moment to you; yet,
while I allow it to be a vice, I confess
that it is the devouring and only passion
of my soul. My rage is unspeakable,
when I reflect that the murderer,
%%279%%
whom I have turned loose upon society,
still exists. You refuse my just
demand: I have but one resource; and
I devote myself, either in my life or
death, to his destruction.''

I trembled with excess of agitation
as I said this; there was a phrenzy in
my manner, and something, I doubt
not, of that haughty fierceness, which
the martyrs of old are said to have possessed.
But to a Genevan magistrate,
whose mind was occupied by far other
ideas than those of devotion and
heroism, this elevation of mind had
much the appearance of madness. He
endeavoured to soothe me as a nurse
does a child, and reverted to my tale
as the effects of delirium.

``Man,'' I cried, ``how ignorant
art thou in thy pride of wisdom!
Cease; you know not what it is you
say.''

I broke from the house angry and
disturbed, and retired to meditate on
some other mode of action.
%%280%%

\namedchapter{Chapter VII}

\textsc{My} present situation was one in
which all voluntary thought was swallowed
up and lost. I was hurried
away by fury; revenge alone endowed
me with strength and composure; it
modelled my feelings, and allowed me
to be calculating and calm, at periods
when otherwise delirium or death
would have been my portion.

My first resolution was to quit
Geneva for ever; my country, which,
when I was happy and beloved, was
dear to me, now, in my adversity, became
hateful. I provided myself with
a sum of money, together with a few
jewels which had belonged to my
mother, and departed.

And now my wanderings began,
which are to cease but with life. I
have traversed a vast portion of the
earth, and have endured all the hardships
which travellers, in deserts and
barbarous countries, are wont to meet.
How I have lived I hardly know; many
times have I stretched my failing limbs
upon the sandy plain, and prayed for
death. But revenge kept me alive; I
dared not die, and leave my adversary
in being.

When I quitted Geneva, my first
labour was to gain some clue by which
I might trace the steps of my fiendish
%%281%%
enemy. But my plan was unsettled;
and I wandered many hours around
the confines of the town, uncertain
what path I should pursue. As night
approached, I found myself at the entrance
of the cemetery where William,
Elizabeth, and my father, reposed.
I entered it, and approached
the tomb which marked their graves.
Every thing was silent, except the
leaves of the trees, which were gently
agitated by the wind; the night was
nearly dark; and the scene would have
been solemn and affecting even to an
uninterested observer. The spirits of
the departed seemed to flit around, and
to cast a shadow, which was felt but
seen not, around the head of the
mourner.

The deep grief which this scene had
at first excited quickly gave way to
rage and despair. They were dead,
and I lived; their murderer also lived,
and to destroy him I must drag out my
weary existence. I knelt on the grass,
and kissed the earth, and with quivering
lips exclaimed, ``By the sacred
earth on which I kneel, by the shades
that wander near me, by the deep and
eternal grief that I feel, I swear; and
by thee, O Night, and by the spirits
that preside over thee, I swear to pursue
the dæmon, who caused this misery,
until he or I shall perish in mortal conflict.
For this purpose I will preserve
my life: to execute this dear revenge,
will I again behold the sun, and tread
the green herbage of earth, which
otherwise should vanish from my eyes
for ever. And I call on you, spirits of
the dead; and on you, wandering ministers
of vengeance, to aid and
%%282%%
conduct me in my work. Let the cursed
and hellish monster drink deep of
agony; let him feel the despair that
now torments me.''

I had begun my adjuration with
solemnity, and an awe which almost
assured me that the shades of my
murdered friends heard and approved
my devotion; but the furies possessed
me as I concluded, and rage choaked
my utterance.

I was answered through the stillness
of night by a loud and fiendish laugh.
It rung on my ears long and heavily;
the mountains re-echoed it, and I felt
as if all hell surrounded me with
mockery and laughter. Surely in that
moment I should have been possessed
by phrenzy, and have destroyed my
miserable existence, but that my vow
was heard, and that I was reserved for
vengeance. The laughter died away:
when a well-known and abhorred voice,
apparently close to my ear, addressed
me in an audible whisper --- ``I am
satisfied: miserable wretch! you have
determined to live, and I am satisfied.''

I darted towards the spot from which
the sound proceeded; but the devil
eluded my grasp. Suddenly the broad
disk of the moon arose, and shone full
upon his ghastly and distorted shape,
as he fled with more than mortal
speed.

I pursued him; and for many months
this has been my task. Guided by a
slight clue, I followed the windings of
the Rhone, but vainly. The blue
Mediterranean appeared; and, by a
strange chance, I saw the fiend enter
%%283%%
by night, and hide himself in a vessel
bound for the Black Sea. I took my
passage in the same ship; but he escaped,
I know not how.

Amidst the wilds of Tartary and
Russia, although he still evaded me,
I have ever followed in his track.
Sometimes the peasants, scared by this
horrid apparition, informed me of his
path; sometimes he himself, who feared
that if I lost all trace I should despair
and die, often left some mark to guide
me. The snows descended on my
head, and I saw the print of his huge
step on the white plain. To you first
entering on life, to whom care is new,
and agony unknown, how can you understand
what I have felt, and still feel?
Cold, want, and fatigue, were the least
pains which I was destined to endure;
I was cursed by some devil, and carried
about with me my eternal hell; yet
still a spirit of good followed and
directed my steps, and, when I most
murmured, would suddenly extricate
me from seemingly insurmountable
difficulties. Sometimes, when nature,
overcome by hunger, sunk under the
exhaustion, a repast was prepared for
me in the desert, that restored and
inspirited me. The fare was indeed
coarse, such as the peasants of the
country ate; but I may not doubt that
it was set there by the spirits that I had
invoked to aid me. Often, when all
was dry, the heavens cloudless, and I
was parched by thirst, a slight cloud
would bedim the sky, shed the few
drops that revived me, and vanish.

I followed, when I could, the courses
of the rivers; but the dæmon generally
%%284%%
avoided these, as it was here that the
population of the country chiefly collected.
In other places human beings
were seldom seen; and I generally subsisted
on the wild animals that crossed
my path. I had money with me, and
gained the friendship of the villagers
by distributing it, or bringing with me
some food that I had killed, which, after
taking a small part, I always presented
to those who had provided me with fire
and utensils for cooking.

My life, as it passed thus, was indeed
hateful to me, and it was during sleep
alone that I could taste joy. O
blessed sleep! often, when most miserable,
I sank to repose, and my dreams
lulled me even to rapture. The spirits
that guarded me had provided these
moments, or rather hours, of happiness,
that I might retain strength to fulfil
my pilgrimage. Deprived of this respite,
I should have sunk under my
hardships. During the day I was sustained
and inspirited by the hope of
night: for in sleep I saw my friends,
my wife, and my beloved country;
again I saw the benevolent countenance
of my father, heard the silver
tones of my Elizabeth's voice, and beheld
Clerval enjoying health and youth.
Often, when wearied by a toilsome
march, I persuaded myself that I was
dreaming until night should come, and
that I should then enjoy reality in the
arms of my dearest friends. What
agonizing fondness did I feel for them!
how did I cling to their dear forms, as
sometimes they haunted even my waking
hours, and persuade myself that they
still lived! At such moments
%%285%%
vengeance, that burned within me, died
in my heart, and I pursued my path
towards the destruction of the dæmon,
more as a task enjoined by heaven, as
the mechanical impulse of some power
of which I was unconscious, than as the
ardent desire of my soul.

What his feelings were whom I pursued,
I cannot know. Sometimes, indeed,
he left marks in writing on the
barks of the trees, or cut in stone, that
guided me, and instigated my fury.
``My reign is not yet over,'' (these words
were legible in one of these inscriptions);
``you live, and my power is
complete. Follow me; I seek the
everlasting ices of the north, where you
will feel the misery of cold and frost,
to which I am impassive. You will
find near this place, if you follow not
too tardily, a dead hare; eat, and be refreshed.
Come on, my enemy; we have
yet to wrestle for our lives; but many
hard and miserable hours must you
endure, until that period shall arrive.''

Scoffing devil! Again do I vow vengeance;
again do I devote thee, miserable
fiend, to torture and death. Never
will I omit my search, until he or I
perish; and then with what ecstacy shall
I join my Elizabeth, and those who even
now prepare for me the reward of my
tedious toil and horrible pilgrimage.

As I still pursued my journey to the
northward, the snows thickened, and
the cold increased in a degree almost
too severe to support. The peasants
were shut up in their hovels, and only
a few of the most hardy ventured forth
to seize the animals whom starvation
had forced from their hiding-places to
%%286%%
seek for prey. The rivers were covered
with ice, and no fish could be procured;
and thus I was cut off from my chief
article of maintenance.

The triumph of my enemy increased
with the difficulty of my labours. One
inscription that he left was in these
words: ``Prepare! your toils only begin:
wrap yourself in furs, and provide
food, for we shall soon enter upon
a journey where your sufferings will
satisfy my everlasting hatred.''

My courage and perseverance were
invigorated by these scoffing words; I
resolved not to fail in my purpose;
and, calling on heaven to support me,
I continued with unabated fervour to
traverse immense deserts, until the
ocean appeared at a distance, and
formed the utmost boundary of the horizon.
Oh! how unlike it was to the
blue seas of the south! Covered with
ice, it was only to be distinguished from
land by its superior wildness and ruggedness.
The Greeks wept for joy
when they beheld the Mediterranean
from the hills of Asia, and hailed with
rapture the boundary of their toils. I
did not weep; but I knelt down, and,
with a full heart, thanked my guiding
spirit for conducting me in safety to
the place where I hoped, notwithstanding
my adversary's gibe, to meet and
grapple with him.

Some weeks before this period I
had procured a sledge and dogs, and
thus traversed the snows with inconceivable
speed. I know not whether
the fiend possessed the same advantages;
but I found that, as before I had
daily lost ground in the pursuit, I now
%%287%%
gained on him; so much so, that when
I first saw the ocean, he was but one
day's journey in advance, and I hoped
to intercept him before he should reach
the beach. With new courage, therefore,
I pressed on, and in two days arrived
at a wretched hamlet on the seashore.
I inquired of the inhabitants concerning
the fiend, and gained accurate
information. A gigantic monster, they
said, had arrived the night before, armed
with a gun and many pistols; putting
to flight the inhabitants of a solitary
cottage, through fear of his terrific appearance.
He had carried off their store
of winter food, and, placing it in a
sledge, to draw which he had seized
on a numerous drove of trained dogs,
he had harnessed them, and the same
night, to the joy of the horror-struck
villagers, had pursued his journey
across the sea in a direction that
led to no land; and they conjectured
that he must speedily be destroyed by
the breaking of the ice, or frozen by
the eternal frosts.

On hearing this information, I suffered
a temporary access of despair. He
had escaped me; and I must commence
a destructive and almost endless journey
across the mountainous ices of the
ocean, --- amidst cold that few of the inhabitants
could long endure, and which
I, the native of a genial and sunny climate,
could not hope to survive. Yet
at the idea that the fiend should live
and be triumphant, my rage and vengeance
returned, and, like a mighty
tide, overwhelmed every other feeling.
After a slight repose, during which the
spirits of the dead hovered round, and
%%288%%
instigated me to toil and revenge, I
prepared for my journey.

I exchanged my land sledge for one
fashioned for the inequalities of the
frozen ocean; and, purchasing a plentiful
stock of provisions, I departed
from land.

I cannot guess how many days have
passed since then; but I have endured
misery, which nothing but the eternal
sentiment of a just retribution burning
within my heart could have enabled
me to support. Immense and rugged
mountains of ice often barred up my
passage, and I often heard the thunder
of the ground sea, which threatened
my destruction. But again the
frost came, and made the paths of the
sea secure.

By the quantity of provision which
I had consumed I should guess that I
had passed three weeks in this journey;
and the continual protraction of
hope, returning back upon the heart,
often wrung bitter drops of despondency
and grief from my eyes. Despair
had indeed almost secured her
prey, and I should soon have sunk beneath
this misery; when once, after
the poor animals that carried me had
with incredible toil gained the summit
of a sloping ice mountain, and one sinking
under his fatigue died, I viewed the
expanse before me with anguish, when
suddenly my eye caught a dark speck
upon the dusky plain. I strained my
sight to discover what it could be, and
uttered a wild cry of ecstacy when I
distinguished a sledge, and the distorted
proportions of a well-known form
within. Oh! with what a burning
%%289%%
gush did hope revisit my heart! warm
tears filled my eyes, which I hastily
wiped away, that they might not intercept
the view I had of the dæmon; but
still my sight was dimmed by the burning
drops, until, giving way to the emotions
that oppressed me, I wept aloud.

But this was not the time for delay;
I disencumbered the dogs of their dead
companion, gave them a plentiful portion
of food; and, after an hour's rest,
which was absolutely necessary, and
yet which was bitterly irksome to me,
I continued my route. The sledge was
still visible; nor did I again lose sight
of it, except at the moments when for a
short time some ice rock concealed it
with its intervening crags. I indeed
perceptibly gained on it; and when,
after nearly two days' journey, I beheld
my enemy at no more than a mile distant,
my heart bounded within me.

But now, when I appeared almost
within grasp of my enemy, my hopes
were suddenly extinguished, and I lost
all trace of him more utterly than I
had ever done before. A ground sea
was heard; the thunder of its progress,
as the waters rolled and swelled beneath
me, became every moment more ominous
and terrific. I pressed on, but in
vain. The wind arose; the sea roared;
and, as with the mighty shock of an
earthquake, it split, and cracked with a
tremendous and overwhelming sound.
The work was soon finished: in a few
minutes a tumultuous sea rolled between
me and my enemy, and I was left
drifting on a scattered piece of ice, that
was continually lessening, and thus
preparing for me a hideous death.
%%290%%

In this manner many appalling hours
passed; several of my dogs died; and
I myself was about to sink under the
accumulation of distress, when I saw
your vessel riding at anchor, and holding
forth to me hopes of succour and
life. I had no conception that vessels
ever came so far north, and was
astounded at the sight. I quickly destroyed
part of my sledge to construct
oars; and by these means was enabled,
with infinite fatigue, to move my ice-raft
in the direction of your ship. I
had determined, if you were going
southward, still to trust myself to the
mercy of the seas, rather than abandon
my purpose. I hoped to induce you
to grant me a boat with which I could
still pursue my enemy. But your direction
was northward. You took me on
board when my vigour was exhausted,
and I should soon have sunk under my
multiplied hardships into a death, which
I still dread, --- for my task is unfulfilled.

Oh! when will my guiding spirit,
in conducting me to the dæmon, allow
me the rest I so much desire; or must
I die, and he yet live? If I do, swear
to me, Walton, that he shall not escape;
that you will seek him, and satisfy my
vengeance in his death. Yet, do I
dare ask you to undertake my pilgrimage,
to endure the hardships that I
have undergone? No; I am not so
selfish. Yet, when I am dead, if he
should appear; if the ministers of vengeance
should conduct him to you,
swear that he shall not live --- swear
that he shall not triumph over my
accumulated woes, and live to make
another such a wretch as I am. He is
%%291%%
eloquent and persuasive; and once
his words had even power over my
heart: but trust him not. His soul is
as hellish as his form, full of treachery
and fiend-like malice. Hear him not;
call on the manes of William, Justine,
Clerval, Elizabeth, my father, and of
the wretched Victor, and thrust your
sword into his heart. I will hover near,
and direct the steel aright.

\bigskip
\textsc{Walton}, \emph{in continuation}.

\frDate{August 26th, 17--- .}

\noindent\textsc{You} have read this strange and terrific
story, Margaret; and do you not
feel your blood congealed with horror,
like that which even now curdles mine?
Sometimes, seized with sudden agony,
he could not continue his tale; at
others, his voice broken, yet piercing,
uttered with difficulty the words so
replete with agony. His fine and lovely
eyes were now lighted up with indignation,
now subdued to downcast sorrow,
and quenched in infinite wretchedness.
Sometimes he commanded his
countenance and tones, and related the
most horrible incidents with a tranquil
voice, suppressing every mark of agitation;
then, like a volcano bursting
forth, his face would suddenly change
to an expression of the wildest rage,
as he shrieked out imprecations on his
persecutor.

His tale is connected, and told with
an appearance of the simplest truth;
yet I own to you that the letters of
Felix and Safie, which he shewed me,
and the apparition of the monster,
%%292%%
seen from our ship, brought to me a
greater conviction of the truth of his
narrative than his asseverations, however
earnest and connected. Such a
monster has then really existence; I
cannot doubt it; yet I am lost in surprise
and admiration. Sometimes I
endeavoured to gain from Frankenstein
the particulars of his creature's formation;
but on this point he was impenetrable.

``Are you mad, my friend?'' said he,
``or whither does your senseless curiosity
lead you? Would you also create
for yourself and the world a demoniacal
enemy? Or to what do your questions
tend? Peace, peace! learn my
miseries, and do not seek to increase
your own.''

Frankenstein discovered that I made
notes concerning his history: he asked
to see them, and then himself corrected
and augmented them in many places;
but principally in giving the life and
spirit to the conversations he held with
his enemy. ``Since you have preserved
my narration,'' said he, ``I would
not that a mutilated one should go
down to posterity.''

Thus has a week passed away, while
I have listened to the strangest tale
that ever imagination formed. My
thoughts, and every feeling of my soul,
have been drunk up by the interest for
my guest, which this tale, and his own
elevated and gentle manners have created.
I wish to soothe him; yet can I
counsel one so infinitely miserable, so
destitute of every hope of consolation,
to live? Oh, no! the only joy that
he can now know will be when he
%%293%%
composes his shattered feelings to peace
and death. Yet he enjoys one comfort,
the offspring of solitude and delirium:
he believes, that, when in dreams he
holds converse with his friends, and
derives from that communion consolation
for his miseries, or excitements to
his vengeance, that they are not the
creations of his fancy, but the real
beings who visit him from the regions
of a remote world. This faith gives a
solemnity to his reveries that render
them to me almost as imposing and
interesting as truth.

Our conversations are not always
confined to his own history and misfortunes.
On every point of general
literature he displays unbounded knowledge,
and a quick and piercing apprehension.
His eloquence is forcible and
touching; nor can I hear him, when
he relates a pathetic incident, or endeavours
to move the passions of pity
or love, without tears. What a glorious
creature must he have been in
the days of his prosperity, when he is
thus noble and godlike in ruin. He
seems to feel his own worth, and the
greatness of his fall.

``When younger,'' said he, ``I felt
as if I were destined for some great enterprise.
My feelings are profound;
but I possessed a coolness of judgment
that fitted me for illustrious achievements.
This sentiment of the worth
of my nature supported me, when others
would have been oppressed; for I deemed
it criminal to throw away in useless
grief those talents that might be useful
to my fellow-creatures. When I reflected
on the work I had completed,
%%294%%
no less a one than the creation of a sensitive
and rational animal, I could not
rank myself with the herd of common
projectors. But this feeling, which supported
me in the commencement of my
career, now serves only to plunge me
lower in the dust. All my speculations
and hopes are as nothing; and, like the
archangel who aspired to omnipotence,
I am chained in an eternal hell. My
imagination was vivid, yet my powers of
analysis and application were intense; by
the union of these qualities I conceived
the idea, and executed the creation of
a man. Even now I cannot recollect,
without passion, my reveries while the
work was incomplete. I trod heaven
in my thoughts, now exulting in my
powers, now burning with the idea of
their effects. From my infancy I was
imbued with high hopes and a lofty
ambition; but how am I sunk! Oh!
my friend, if you had known me as I
once was, you would not recognize me
in this state of degradation. Despondency
rarely visited my heart; a high
destiny seemed to bear me on, until I
fell, never, never again to rise.''

Must I then lose this admirable
being? I have longed for a friend; I
have sought one who would sympathize
with and love me. Behold, on these
desert seas I have found such a one;
but, I fear, I have gained him only to
know his value, and lose him. I would
reconcile him to life, but he repulses
the idea.

``I thank you, Walton,'' he said,
``for your kind intentions towards so
miserable a wretch; but when you
speak of new ties, and fresh affections,
%%295%%
think you that any can replace those
who are gone? Can any man be to
me as Clerval was; or any woman
another Elizabeth? Even where the
affections are not strongly moved by any
superior excellence, the companions of
our childhood always possess a certain
power over our minds, which hardly
any later friend can obtain. They
know our infantine dispositions, which,
however they may be afterwards modified,
are never eradicated; and they
can judge of our actions with more
certain conclusions as to the integrity
of our motives. A sister or a brother
can never, unless indeed such symptoms
have been shewn early, suspect
the other of fraud or false dealing,
when another friend, however strongly
he may be attached, may, in spite of
himself, be invaded with suspicion.
But I enjoyed friends, dear not only
through habit and association, but from
their own merits; and, wherever I am,
the soothing voice of my Elizabeth,
and the conversation of Clerval, will
be ever whispered in my ear. They
are dead; and but one feeling in such
a solitude can persuade me to preserve
my life. If I were engaged in any
high undertaking or design, fraught
with extensive utility to my fellow-creatures,
then could I live to fulfil it.
But such is not my destiny; I must pursue
and destroy the being to whom I
gave existence; then my lot on earth
will be fulfilled, and I may die.''

\bigskip
\frDate{September 2d.}

\noindent\textsc{my beloved sister},

I write to you, encompassed by
peril, and ignorant wh\-ether I am ever
%%296%%
doomed to see again dear England, and
the dearer friends that inhabit it. I
am surrounded by mountains of ice,
which admit of no escape, and threaten
every moment to crush my vessel. The
brave fellows, whom I have persuaded
to be my companions, look towards me
for aid; but I have none to bestow.
There is something terribly appalling
in our situation, yet my courage and
hopes do not desert me. We may
survive; and if we do not, I will repeat
the lessons of my Seneca, and die with
a good heart.

Yet what, Margaret, will be the
state of your mind? You will not hear
of my destruction, and you will anxiously
await my return. Years will
pass, and you will have visitings of
despair, and yet be tortured by hope.
Oh! my beloved sister, the sickening
failings of your heart-felt expectations
are, in prospect, more terrible to me
than my own death. But you have a
husband, and lovely children; you
may be happy: heaven bless you, and
make you so!

My unfortunate guest regards me
with the tenderest compassion. He
endeavours to fill me with hope; and
talks as if life were a possession which
he valued. He reminds me how often
the same accidents have happened to
other navigators, who have attempted
this sea, and, in spite of myself, he fills
me with cheerful auguries. Even the
sailors feel the power of his eloquence:
when he speaks, they no longer despair;
he rouses their energies, and, while
they hear his voice, they believe these
vast mountains of ice are mole-hills,
which will vanish before the resolutions
%%297%%
of man. These feelings are transitory;
each day's expectation delayed fills
them with fear, and I almost dread a
mutiny caused by this despair.

\medskip
\frDate{September 5th.}

A scene has just passed of such
uncommon interest, that although it is
highly probable that these papers may
never reach you, yet I cannot forbear
recording it.

We are still surrounded by mountains
of ice, still in imminent danger
of being crushed in their conflict.
The cold is excessive, and many of
my unfortunate comrades have already
found a grave amidst this scene of
desolation. Frankenstein has daily
declined in health: a feverish fire still
glimmers in his eyes; but he is exhausted,
and, when suddenly roused to
any exertion, he speedily sinks again
into apparent lifelessness.

I mentioned in my last letter the
fears I entertained of a mutiny. This
morning, as I sat watching the wan
countenance of my friend --- his eyes
half closed, and his limbs hanging
listlessly, --- I was roused by half a dozen
of the sailors, who desired admission
into the cabin. They entered; and their
leader addressed me. He told me that
he and his companions had been chosen
by the other sailors to come in deputation
to me, to make me a demand,
which, in justice, I could not refuse.
We were immured in ice, and should
probably never escape; but they feared
that if, as was possible, the ice should
dissipate, and a free passage be opened,
I should be rash enough to continue
%%298%%
my voyage, and lead them into fresh
dangers, after they might happily have
surmounted this. They desired, therefore,
that I should engage with a solemn
promise, that if the vessel should
be freed, I would instantly direct my
coarse southward.

This speech troubled me. I had
not despaired; nor had I yet conceived
the idea of returning, if set free. Yet
could I, in justice, or even in possibility,
refuse this demand? I hesitated
before I answered; when Frankenstein,
who had at first been silent, and, indeed,
appeared hardly to have force
enough to attend, now roused himself;
his eyes sparkled, and his cheeks
flushed with momentary vigour. Turning
towards the men, he said~---

``What do you mean? What do
you demand of your captain? Are
you then so easily turned from your
design? Did you not call this a glorious
expedition? and wherefore was it
glorious? Not because the way was
smooth and placid as a southern sea,
but because it was full of dangers and
terror; because, at every new incident,
your fortitude was to be called forth,
and your courage exhibited; because
danger and death surrounded, and these
dangers you were to brave and overcome.
For this was it a glorious, for
this was it an honourable undertaking.
You were hereafter to be hailed as
the benefactors of your species; your
name adored, as belonging to brave
men who encountered death for honour
and the benefit of mankind. And
now, behold, with the first imagination
of danger, or, if you will, the first
%%299%%
mighty and terrific trial of your courage,
you shrink away, and are content
to be handed down as men who had
not strength enough to endure cold
and peril; and so, poor souls, they
were chilly, and returned to their warm
fire-sides. Why, that requires not this
preparation; ye need not have come
thus far, and dragged your captain to
the shame of a defeat, merely to prove
yourselves cowards. Oh! be men, or
be more than men. Be steady to your
purposes, and firm as a rock. This ice
is not made of such stuff as your hearts
might be; it is mutable, cannot withstand
you, if you say that it shall not.
Do not return to your families with the
stigma of disgrace marked on your
brows. Return as heroes who have
fought and conquered, and who know
not what it is to turn their backs on
the foe.''

He spoke this with a voice so modulated
to the different feelings expressed
in his speech, with an eye so full of
lofty design and heroism, that can you
wonder that these men were moved.
They looked at one another, and were
unable to reply. I spoke; I told them
to retire, and consider of what had been
said: that I would not lead them
further north, if they strenuously desired
the contrary; but that I hoped
that, with reflection, their courage
would return.

They retired, and I turned towards
my friend; but he was sunk in languor,
and almost deprived of life.

How all this will terminate, I know
not; but I had rather die, than return
shamefully, --- my purpose unfulfilled.
%%300%%
Yet I fear such will be my fate; the
men, unsupported by ideas of glory
and honour, can never willingly continue
to endure their present hardships.

\medskip
\frDate{September 7th.}

The die is cast; I have consented to
return, if we are not destroyed. Thus
are my hopes blasted by cowardice and
indecision; I come back ignorant and
disappointed. It requires more philosophy
than I possess, to bear this injustice
with patience.

\medskip
\frDate{September 12th.}

It is past; I am returning to England.
I have lost my hopes of utility
and glory; --- I have lost my friend.
But I will endeavour to detail these
bitter circumstances to you, my dear
sister; and, while I am wafted towards
England, and towards you, I will not
despond.

September 19th, the ice began to
move, and roarings like thunder were
heard at a distance, as the islands
split and cracked in every direction.
We were in the most imminent peril;
but, as we could only remain passive,
my chief attention was occupied by my
unfortunate guest, whose illness increased
in such a degree, that he was
entirely confined to his bed. The ice
cracked behind us, and was driven with
force towards the north; a breeze
sprung from the west, and on the 11th
the passage towards the south became
perfectly free. When the sailors saw
this, and that their return to their
native country was apparently assured,
%%301%%
a shout of tumultuous joy broke from
them, loud and long-continued. Frankenstein,
who was dozing, awoke, and
asked the cause of the tumult. ``They
shout,'' I said, ``because they will soon
return to England.''

``Do you then really return?''

``Alas! yes; I cannot withstand their
demands. I cannot lead them unwillingly
to danger, and I must return.''

``Do so, if you will; but I will not.
You may give up your purpose; but
mine is assigned to me by heaven, and
I dare not. I am weak; but surely the
spirits who assist my vengeance will
endow me with sufficient strength.''
Saying this, he endeavoured to spring
from the bed, but the exertion was
too great for him; he fell back, and
fainted.

It was long before he was restored;
and I often thought that life was entirely
extinct. At length he opened his eyes,
but he breathed with difficulty, and was
unable to speak. The surgeon gave
him a composing draught, and ordered
us to leave him undisturbed. In
the mean time he told me, that my
friend had certainly not many hours to
live.

His sentence was pronounced; and I
could only grieve, and be patient. I
sat by his bed watching him; his eyes
were closed, and I thought he slept;
but presently he called to me in a
feeble voice, and, bidding me come
near, said --- ``Alas! the strength I relied
on is gone; I feel that I shall
soon die, and he, my enemy and persecutor,
may still be in being. Think
not, Walton, that in the last moments
%%302%%
of my existence I feel that burning
hatred, and ardent desire of revenge, I
once expressed, but I feel myself justified
in desiring the death of my adversary.
During these last days I have
been occupied in examining my past
conduct; nor do I find it blameable. In
a fit of enthusiastic madness I created
a rational creature, and was bound towards
him, to assure, as far as was in
my power, his happiness and well-being.
This was my duty; but there
was another still paramount to that.
My duties towards my fellow-creatures
had greater claims to my attention, because
they included a greater proportion
of happiness or misery. Urged
by this view, I refused, and I did right
in refusing, to create a companion for
the first creature. He shewed unparalleled
malignity and selfishness, in
evil: he destroyed my friends; he devoted
to destruction beings who possessed
exquisite sensations, happiness,
and wisdom; nor do I know where
this thirst for vengeance may end.
Miserable himself, that he may render
no other wretched, he ought to die.
The task of his destruction was mine,
but I have failed. When actuated by
selfish and vicious motives, I asked
you to undertake my unfinished work;
and I renew this request now, when
I am only induced by reason and
virtue.

``Yet I cannot ask you to renounce
your country and friends, to fulfil this
task; and now, that you are returning
to England, you will have little chance
of meeting with him. But the consideration
of these points, and the
%%303%%
well-balancing of what you may esteem
your duties, I leave to you; my judgment
and ideas are already disturbed
by the near approach of death. I
dare not ask you to do what I think
right, for I may still be misled by
passion.

``That he should live to be an instrument
of mischief disturbs me; in other
respects this hour, when I momentarily
expect my release, is the only happy
one which I have enjoyed for several
years. The forms of the beloved dead
flit before me, and I hasten to their
arms. Farewell, Walton! Seek happiness
in tranquillity, and avoid ambition,
even if it be only the apparently
innocent one of distinguishing yourself
in science and discoveries. Yet
why do I say this? I have myself been
blasted in these hopes, yet another may
succeed.''

His voice became fainter as he
spoke; and at length, exhausted by
his effort, he sunk into silence. About
half an hour afterwards he attempted
again to speak, but was unable; he
pressed my hand feebly, and his eyes
closed for ever, while the irradiation
of a gentle smile passed away from his
lips.

Margaret, what comment can I make
on the untimely extinction of this glorious
spirit? What can I say, that will
enable you to understand the depth of
my sorrow? All that I should express
would be inadequate and feeble. My
tears flow; my mind is overshadowed
by a cloud of disappointment. But I
journey towards England, and I may
there find consolation.
%%304%%

I am interrupted. What do these
sounds portend? It is midnight; the
breeze blows fairly, and the watch on
deck scarcely stir. Again; there is a
sound as of a human voice, but
hoarser; it comes from the cabin where
the remains of Frankenstein still lie.
I must arise, and examine. Good night,
my sister.

Great God! what a scene has just
taken place! I am yet dizzy with the
remembrance of it. I hardly know
whether I shall have the power to detail
it; yet the tale which I have recorded
would be incomplete without
this final and wonderful catastrophe.

I entered the cabin, where lay the
remains of my ill-fated and admirable
friend. Over him hung a form which
I cannot find words to describe; gigantic
in stature, yet uncouth and distorted
in its proportions. As he hung
over the coffin, his face was concealed by
long locks of ragged hair; but one vast
hand was extended, in colour and apparent
texture like that of a mummy.
When he heard the sound of my approach,
he ceased to utter exclamations
of grief and horror, and sprung towards
the window. Never did I behold
a vision so horrible as his face, of such
loathsome, yet appalling hideousness.
I shut my eyes involuntarily, and endeavoured
to recollect what were my
duties with regard to this destroyer.
I called on him to stay.

He paused, looking on me with wonder;
and, again turning towards the
lifeless form of his creator, he seem\-ed
to forget my presence, and every feature
and gesture seemed instigated by
%%305%%
the wildest rage of some uncontrollable
passion.

``That is also my victim!'' he exclaimed;
``in his murder my crimes are
consummated; the miserable series of
my being is wound to its close! Oh,
Frankenstein! generous and self-devoted
being! what does it avail that
I now ask thee to pardon me? I, who
irretrievably destroyed thee by destroying
all thou lovedst. Alas! he is cold;
he may not answer me.''

His voice seemed suffocated; and
my first impulses, which had suggested
to me the duty of obeying the dying
request of my friend, in destroying his
enemy, were now suspended by a mixture
of curiosity and compassion. I
approached this tremendous being; I
dared not again raise my looks upon
his face, there was something so scaring
and unearthly in his ugliness. I
attempted to speak, but the words died
away on my lips. The monster continued
to utter wild and incoherent self-reproaches.
At length I gathered resolution
to address him, in a pause of the
tempest of his passion: ``Your repentance,''
I said, ``is now superfluous.
If you had listened to the voice of conscience,
and heeded the stings of remorse,
before you had urged your diabolical
vengeance to this extremity,
Frankenstein would yet have lived.''

``And do you dream?'' said the dæmon;
``do you think that I was then
dead to agony and remorse? --- He,'' he
continued, pointing to the corpse, ``he
suffered not more in the consummation
of the deed; --- oh! not the ten-thousandth
portion of the anguish that was
mine during the lingering detail of its
%%306%%
execution. A frightful selfishness hurried
me on, while my heart was poisoned
with remorse. Think ye that
the groans of Clerval were music to
my ears? My heart was fashioned to
be susceptible of love and sympathy;
and, when wrenched by misery to vice
and hatred, it did not endure the violence
of the change without torture
such as you cannot even imagine.

``After the murder of Clerval, I returned
to Switzerland, heart-broken
and overcome. I pitied Frankenstein;
my pity amounted to horror: I abhorred
myself. But when I discovered
that he, the author at once of my existence
and of its unspeakable torments,
dared to hope for happiness; that while
he accumulated wretchedness and despair
upon me, he sought his own enjoyment
in feelings and passions from
the indulgence of which I was for ever
barred, then impotent envy and bitter
indignation filled me with an insatiable
thirst for vengeance. I recollected my
threat, and resolved that it should be
accomplished. I knew that I was preparing
for myself a deadly torture; but
I was the slave, not the master of an
impulse, which I detested, yet could
not disobey. Yet when she died! --- nay,
then I was not miserable. I had cast
off all feeling, subdued all anguish
to riot in the excess of my despair.
Evil thenceforth became my good.
Urged thus far, I had no choice but to
adapt my nature to an element which
I had willingly chosen. The completion
of my demoniacal design became
an insatiable passion. And now it
is ended; there is my last victim!''

I was at first touched by the
%%307%%
expressions of his misery; yet when I called to
mind what Frankenstein had said of
his powers of eloquence and persuasion,
and when I again cast my eyes on
the lifeless form of my friend, indignation
was re-kindled within me.
``Wretch!'' I said, ``it is well that
you come here to whine over the desolation
that you have made. You throw
a torch into a pile of buildings, and
when they are consumed you sit among
the ruins, and lament the fall. Hypocritical
fiend! if he whom you mourn
still lived, still would he be the object,
again would he become the prey of
your accursed vengeance. It is not
pity that you feel; you lament only because
the victim of your malignity is
withdrawn from your power.''

``Oh, it is not thus --- not thus,'' interrupted
the being; ``yet such must
be the impression conveyed to you by
what appears to be the purport of my
actions. Yet I seek not a fellow-feeling
in my misery. No sympathy
may I ever find. When I first sought
it, it was the love of virtue, the feelings
of happiness and affection with
which my whole being overflowed, that
I wished to be participated. But now,
that virtue has become to me a shadow,
and that happiness and affection
are turned into bitter and loathing despair,
in what should I seek for sympathy?
I am content to suffer alone,
while my sufferings shall endure: when
I die, I am well satisfied that abhorrence
and opprobrium should load my
memory. Once my fancy was soothed
with dreams of virtue, of fame, and of
enjoyment. Once I falsely hoped to
%%308%%
meet with beings, who, pardoning my
outward form, would love me for the
excellent qualities which I was capable
of bringing forth. I was nourished
with high thoughts of honour and devotion.
But now vice has degraded
me beneath the meanest animal. No
crime, no mischief, no malignity, no
misery, can be found comparable to
mine. When I call over the frightful
catalogue of my deeds, I cannot believe
that I am he whose thoughts were
once filled with sublime and transcendant
visions of the beauty and the majesty
of goodness. But it is even so;
the fallen angel becomes a malignant
devil. Yet even that enemy of God
and man had friends and associates in
his desolation; I am quite alone.

``You, who call Frankenstein your
friend, seem to have a knowledge of
my crimes and his misfortunes. But,
in the detail which he gave you of
them, he could not sum up the hours
and months of misery which I endured,
wasting in impotent passions. For
whilst I destroyed his hopes, I did not
satisfy my own desires. They were for
ever ardent and craving; still I desired
love and fellowship, and I was still
spurned. Was there no injustice in
this? Am I to be thought the only
criminal, when all human kind sinned
against me? Why do you not hate
Felix, who drove his friend from his
door with contumely? Why do you
not execrate the rustic who sought to
destroy the saviour of his child? Nay,
these are virtuous and immaculate beings!
I, the miserable and the abandoned,
am an abortion, to be spurned
%%309%%
at, and kicked, and trampled on. Even
now my blood boils at the recollection
of this injustice.

``But it is true that I am a wretch.
I have murdered the lovely and the
helpless; I have strangled the innocent
as they slept, and grasped to death
his throat who never injured me or
any other living thing. I have devoted
my creator, the select specimen of all
that is worthy of love and admiration
among men, to misery; I have pursued
him even to that irremediable ruin.
There he lies, white and cold in death.
You hate me; but your abhorrence
cannot equal that with which I regard
myself. I look on the hands which
executed the deed; I think on the
heart in which the imagination of it
was conceived, and long for the moment
when they will meet my eyes,
when it will haunt my thoughts, no
more.

``Fear not that I shall be the instrument
of future mischief. My work is
nearly complete. Neither your's nor
any man's death is needed to consummate
the series of my being, and accomplish
that which must be done;
but it requires my own. Do not think
that I shall be slow to perform this sacrifice.
I shall quit your vessel on the
ice-raft which brought me hither, and
shall seek the most northern extremity
of the globe; I shall collect my funeral
pile, and consume to ashes this miserable
frame, that its remains may
afford no light to any curious and unhallowed
wretch, who would create
such another as I have been. I shall
die. I shall no longer feel the agonies
%%310%%
which now consume me, or be the
prey of feelings unsatisfied, yet unquenched.
He is dead who called me
into being; and when I shall be no
more, the very remembrance of us
both will speedily vanish. I shall no
longer see the sun or stars, or feel the
winds play on my cheeks. Light, feeling,
and sense, will pass away; and in
this condition must I find my happiness.
Some years ago, when the
images which this world affords first
opened upon me, when I felt the cheering
warmth of summer, and heard the
rustling of the leaves and the chirping
of the birds, and these were all to me,
I should have wept to die; now it is
my only consolation. Polluted by
crimes, and torn by the bitterest remorse,
where can I find rest but in
death?

``Farewell! I leave you, and in you
the last of human kind whom these
eyes will ever behold. Farewell, Frankenstein!
If thou wert yet alive, and
yet cherished a desire of revenge against
me, it would be better satiated in my
life than in my destruction. But it was
not so; thou didst seek my extinction,
that I might not cause greater wretchedness;
and if yet, in some mode unknown
to me, thou hast not yet ceased
to think and feel, thou desirest not my
life for my own misery. Blasted as
thou wert, my agony was still superior
to thine; for the bitter sting of remorse
may not cease to rankle in my wounds
until death shall close them for ever.

``But soon,'' he cried, with sad and
solemn enthusiasm, ``I shall die, and
what I now feel be no longer felt.
%%311%%
Soon these burning miseries will be
extinct. I shall ascend my funeral
pile triumphantly, and exult in the
agony of the torturing flames. The
light of that conflagration will fade
away; my ashes will be swept into the
sea by the winds. My spirit will sleep
in peace; or if it thinks, it will not
surely think thus. Farewell.''

He sprung from the cabin-window,
as he said this, upon the ice-raft which
lay close to the vessel. He was soon
borne away by the waves, and lost in
darkness and distance.

\bigskip
\begin{center}
\textsc{the end.}
\end{center}
